%%%%%%%%%%%%%%%%%%%%%%%%%%%%%%%%%%%%%%%%%%%%%%%%%%%%%%%%%%%%%%%%%%%%%%%%%%%%%
%%
%A  chvutil.tex       CHEVIE documentation                        Jean Michel
%%
%Y  Copyright (C) 1992 - 2010                       University  Paris VII.
%%
%%  This file contains formatting functions.
%%%%%%%%%%%%%%%%%%%%%%%%%%%%%%%%%%%%%%%%%%%%%%%%%%%%%%%%%%%%%%%%%%%%%%%%%

\Chapter{CHEVIE String and Formatting functions}

\CHEVIE\ enhances the facilities of \GAP3 for formatting and displaying
objects.

First,  it provides  some useful  string functions,  such as 'Replace', and
'IntListToString'.

Second,  it enforces a general  policy on how to  format and print objects.
The  most  basic  method  which  should  be  provided  for an object is the
'Format'  method. This  is a  method whose  second argument  is a record of
options  to  control  printing/formatting  the  object.  When  the  second
argument  is absent,  or equivalently  the empty  record, one  has the most
basic formatting, which is used to make the 'Display' method of the object.
When  the option  'GAP' is  set (that  is the  record second argument has a
field  'GAP'), the output should  be a form which  can, as far as possible,
read  back in \GAP. This  output is what is  used by default in the methods
'String' and 'Print'.

In  addition to the  above options, most  \CHEVIE\ objects also provide the
formatting  options 'TeX' (resp.  'LaTeX'), to output  strings suitable for
TeX  or LaTeX  typesetting. The  objects for  which this  makes sense (like
polynomials)  provide  a  'Maple'  option  for  formatting to create output
readable by Maple.
%%%%%%%%%%%%%%%%%%%%%%%%%%%%%%%%%%%%%%%%%%%%%%%%%%%%%%%%%%%%%%%%%%%%%%%%%
\Section{Replace}%
\index{Replace}%

'Replace( <s> [, <s1>, <r1> [, <s2>, <r2> [...]]])'

Replaces in list <s> all (non-overlapping) occurrences of sublist <s1> by
list <r1>, then all occurrences of <s2> by <r2>, etc...

|    gap> Replace("aabaabaabbb","aabaa","c","cba","def","bbb","ult");
    "default"|

%%%%%%%%%%%%%%%%%%%%%%%%%%%%%%%%%%%%%%%%%%%%%%%%%%%%%%%%%%%%%%%%%%%%%%%%%
\Section{IntListToString}%
\index{IntListToString}%

'IntListToString( <part> [, <brackets>] )'

<part>  must be a list  of positive integers. If all  of them are smaller
than 10 then a string of digits corresponding to the entries of <part> is
returned.  If  an entry is  $\ge 10$  then  the elements of  <part>  are
converted to strings, concatenated  with separating commas and the result
surrounded by brackets.  By default '()' brackets  are used.  This may be
changed  by giving  as  second argument a  length   two string specifying
another kind of brackets.

|    gap> IntListToString( [ 4, 2, 2, 1, 1 ] );
    "42211"
    gap> IntListToString( [ 14, 2, 2, 1, 1 ] );
    "(14,2,2,1,1)"
    gap> IntListToString( [ 14, 2, 2, 1, 1 ], "{}" );
    "{14,2,2,1,1}" |

%%%%%%%%%%%%%%%%%%%%%%%%%%%%%%%%%%%%%%%%%%%%%%%%%%%%%%%%%%%%%%%%%%%%%%%%%
\Section{FormatTable}%
\index{FormatTable}%

'FormatTable( <table>, <options> )'

This is a general routine to format a table (a rectangular array, that is a
list of lists of the same length).

The option is a record whose fields can be:

'rowLabels':  at least this  field must be  present. It will contain labels
for the rows of the table.

'columnLabels': labels for the columns of the table.

'rowsLabel': label for the first column (containing the 'rowLabels').

'separators': by default, a separating line is put after the line of column
labels.  This  option  contains  the  indices  of  lines after which to put
separating lines, so the default is equivalent to '.separators:=[0]'.

'rows':  a list  of indices.  If given,  only the  rows specified  by these
indices are formatted.

'columns': a list of indices. If given, only the columns specified by these
indices are formatted.

'TeX': if set to 'true', TeX output is generated to format the table.

'LaTeX':  'TeX'  also  should  be  set  if  this  is  used. LaTeX output is
generated  using the package 'longtable', so the output can be split across
several pages.

'columnRepartition':  This is  used to  specify how  to split  the table in
several parts typeset one after the other. The variable 'columnRepartition'
should  be a list of integers to specify  how many columns to print in each
part.  When using plain  text output, this  is unnecessary as 'FormatTable'
can  automatically split the table into parts not exceeding 'screenColumns'
columns, if this option is specified.

'screenColumns':  As explained above, is used to split the table when doing
plain  text output. A  good value to  set it is  'SizeScreen()[1]', so each
part of the table does not exceed the screen width.

|    gap> t:=IdentityMat(3);;o:=rec(rowLabels:=[1..3]);;
    gap> Print(FormatTable(t,o));
    1 |'\|'|1 0 0
    2 |'\|'|0 1 0
    3 |'\|'|0 0 1
    gap> o.columnLabels:=[6..8];;Print(FormatTable(t,o));
      |'\|'|6 7 8
    _________
    1 |'\|'|1 0 0
    2 |'\|'|0 1 0
    3 |'\|'|0 0 1
    gap> o.rowsLabel:="N";;Print(FormatTable(t,o));
    N |'\|'|6 7 8
    _________
    1 |'\|'|1 0 0
    2 |'\|'|0 1 0
    3 |'\|'|0 0 1
    gap> o.separators:=[0,2];;Print(FormatTable(t,o));
    N |'\|'|6 7 8
    _________
    1 |'\|'|1 0 0
    2 |'\|'|0 1 0
    _________
    3 |'\|'|0 0 1|

%%%%%%%%%%%%%%%%%%%%%%%%%%%%%%%%%%%%%%%%%%%%%%%%%%%%%%%%%%%%%%%%%%%%%%%%%
\Section{Format}%
\index{Format}%

'Format( <object>[, <options>] )'

\index{FormatGAP}%
'FormatGAP( <object>[, <options>] )'

\index{FormatMaple}%
'FormatMaple( <object>[, <options>] )'

\index{FormatTeX}%
'FormatTeX( <object>[, <options>] )'

\index{FormatLaTeX}%
'FormatLaTeX( <object>[, <options>] )'

'Format'  is a  general routine  for formatting  an object.  <options> is a
record  of options; if not given,  it is equivalent to '<options>\:=rec()'.
The  routines 'FormatGAP', 'FormatMaple', 'FormatTeX' and 'FormatLaTeX' add
some  options (or setup a record with some options if no second argument is
given);    respectively   they   set   up   'GAP\:=true',   'Maple\:=true',
'TeX\:=true', and for 'FormatLaTeX' both 'TeX\:=true' and 'LaTeX\:=true'.

If  <object> is a  record, 'Format' looks  if it has a '.operations.Format'
method  and  then  calls  it.  Otherwise,  'Format'  knows how to format in
various ways\: polynomials, cyclotomics, lists, matrices, booleans.

Here are some examples.

|    gap> q:=X(Rationals);;q.name:="q";;
    gap> Format(q^-3-13*q);
    "-13q+q^-3"
    gap> FormatGAP(q^-3-13*q);
    "-13*q+q^-3"
    gap> FormatMaple(q^-3-13*q);
    "-13*q+q^(-3)"
    gap> FormatTeX(q^-3-13*q);
    "-13q+q^{-3}"|

By default, 'Format' tries to recognize cyclotomics which are in quadratic
number fields. If the option 'noQuadrat\:=true' is given it does not.

|    gap> Format(E(3)-E(3)^2);
    "ER(-3)"
    gap> Format(E(3)-E(3)^2,rec(noQuadrat:=true));
    "-E3^2+E3"
    gap> FormatTeX(E(3)-E(3)^2,rec(noQuadrat:=true));
    "-\\zeta_3^2+\\zeta_3"
    gap> FormatTeX(E(3)-E(3)^2);
    "\\sqrt {-3}"
    gap> FormatMaple(E(3)-E(3)^2);
    "sqrt(-3)"|

Formatting  of arrays gives  output usable for  typesetting if the 'TeX' or
'LaTeX' options are given.

|    gap> m:=IdentityMat(3);;
    gap> Print(Format(m),"\n");
    1 0 0
    0 1 0
    0 0 1
    gap> FormatTeX(m);
    "1&0&0\\cr\n0&1&0\\cr\n0&0&1\\cr\n"
    gap> FormatGAP(m);
    "[[1,0,0],[0,1,0],[0,0,1]]"
    gap> FormatLaTeX(m);
    "1&0&0\\\\\n0&1&0\\\\\n0&0&1\\\\\n"|

%%%%%%%%%%%%%%%%%%%%%%%%%%%%%%%%%%%%%%%%%%%%%%%%%%%%%%%%%%%%%%%%%%%%%%%%%
