%%%%%%%%%%%%%%%%%%%%%%%%%%%%%%%%%%%%%%%%%%%%%%%%%%%%%%%%%%%%%%%%%%%%%%%%%%%%%
%%
%A  chvutil.tex       CHEVIE documentation       Meinolf Geck, Frank Luebeck,
%A                                                Jean Michel, G"otz Pfeiffer
%%
%Y  Copyright (C) 1992 - 1996  Lehrstuhl D f\"ur Mathematik, RWTH Aachen, IWR
%Y  der Universit\"at Heidelberg, University of St. Andrews, and   University
%Y  Paris VII.
%%
%%  This  file  contains  utility functions.
%%%%%%%%%%%%%%%%%%%%%%%%%%%%%%%%%%%%%%%%%%%%%%%%%%%%%%%%%%%%%%%%%%%%%%%%%

\Chapter{Cyclotomic polynomials}

Cyclotomic  numbers, and cyclotomic polynomials  over the rationals or some
cyclotomic  field, play an important role in the study of reductive groups,
so  they do in \CHEVIE. Special facilities  are provided to deal with them.
The  most prominent is the type 'CycPol'  which represents the product of a
polynomial  with  a  rational  fraction  in  one variable with all poles or
zeroes equal to 0 or roots of unity.

The  advantages  of  representing  as  'CycPol'  objects  which  can  be so
represented   are\:\  nice  display   (factorized),  less  storage,  faster
multiplication,  division and evaluation. The big drawback is that addition
and subtraction are not implemented!

|    gap> q:=X(Cyclotomics);;q.name:="q";;
    gap> p:=CycPol(q^18 + q^16 + 2*q^12 + q^8 + q^6);
    (1+q^2-q^4+q^6+q^8)q^6P8
    gap> p/CycPol(q^2+q+1);
    (1+q^2-q^4+q^6+q^8)q^6P3^-1P8|

The variable in a 'CycPol' will be denoted by 'q'. It is usually printed as
'q'  but it is possible to change  its name, see 'Format' in 
"Functions for CycPols".

'CycPol's are represented internally by a record with fields\:

'.coeff':  a coefficient, usually a cyclotomic number, but it can also be a
polynomial  and actually can be any \GAP\ object which can be multiplied by
cyclotomic polynomials.

'.valuation': the valuation, positive  or negative.

'.vcyc':  a list of  pairs $[e_i,m_i]$ representing  a root of  unity and a
multiplicity  $m_i$. Actually $e_i$ should be a fraction $p/d$ with $p\< d$
representing $E(d)^p$. The pair represents $(q-E(d)^p)^{m_i}$.

So  if  we  let  |mu(e):=e->E(Denominator(e))^Numerator(e)|,  a  record 'r'
represents 

|r.coeff*q^r.valuation*Product(r.vcyc,p->(q-mu(p[1]))^p[2])|.

%%%%%%%%%%%%%%%%%%%%%%%%%%%%%%%%%%%%%%%%%%%%%%%%%%%%%%%%%%%%%%%%%%%%%%%%%
\Section{AsRootOfUnity}%
\index{AsRootOfUnity}%

'AsRootOfUnity( <c> )'

<c>  should be  a cyclotomic  number. 'AsRootOfUnity'  returns the rational
'e/n'  with $0\le e\<n$ (that is,  $e/n\in\Q/\Z$) if 'c=E(n)\^e', and false
if <c> is not a root of unity. The code for this function has been provided
by Thomas Breuer; we thank him for his help.

|    gap> AsRootOfUnity(-E(9)^2-E(9)^5);
    8/9
    gap> AsRootOfUnity(-E(9)^4-E(9)^5);
    false
    gap> AsRootOfUnity(1);
    0|

%%%%%%%%%%%%%%%%%%%%%%%%%%%%%%%%%%%%%%%%%%%%%%%%%%%%%%%%%%%%%%%%%%%%%%%%%
\Section{CycPol}%
\index{CycPol}%

'CycPol( <p> )'

In the first form 'CycPol( <p> )' the argument is a polynomial\:

|    gap> CycPol(3*q^3-3);
    3P1P3|

Special code  makes the conversion fast  if '<p>' has not  more than two
nonzero coefficients.

The  second form is  a fast and  efficient way of  specifying a CycPol with
only positive multiplicities\: <p> should be a vector. The first element is
taken  as a  the '.coeff'  of the  CycPol, the  second as the '.valuation'.
Subsequent  elements  are  rationals  'i/d'  (with  $i\<  d$)  representing
'(q-E(d)\^i)' or are integers $d$ representing $\Phi_d(q)$.

|    gap> CycPol([3,-5,6,3/7]);
    3q^-5P6(q-E7^3) |

%%%%%%%%%%%%%%%%%%%%%%%%%%%%%%%%%%%%%%%%%%%%%%%%%%%%%%%%%%%%%%%%%%%%%%%%%
\Section{IsCycPol}%
\index{IsCycPol}%

'IsCycPol( <p> )'

This function returns 'true' if <p> is a 'CycPol' and 'false' otherwise.

|    gap> IsCycPol(CycPol(1));
    true
    gap> IsCycPol(1);        
    false|

%%%%%%%%%%%%%%%%%%%%%%%%%%%%%%%%%%%%%%%%%%%%%%%%%%%%%%%%%%%%%%%%%%%%%%%%%
\Section{Functions for CycPols}

Multiplication '\*' division '/' and exponentiation '\^' work as usual, and
the functions 'Degree', 'Valuation' and 'Value' work as for polynomials\:

|    gap> p:=CycPol(q^18 + q^16 + 2*q^12 + q^8 + q^6);
    (1+q^2-q^4+q^6+q^8)q^6P8
    gap> Value(p,q);
    q^18 + q^16 + 2*q^12 + q^8 + q^6
    gap> p:=p/CycPol(q^2+q+1);
    (1+q^2-q^4+q^6+q^8)q^6P3^-1P8
    gap> Value(p,q);
    Error, Cannot evaluate the non-Laurent polynomial CycPol (1+q^2-q^4+q^\
    6+q^8)q^6P3^-1P8 in
    f.operations.Value( f, x ) called from
    Value( p, q ) called from
    main loop
    brk>
    gap> Degree(p);
    16
    gap> Value(p,3);
    431537382/13|

\index{ComplexConjugate}
The  function  'ComplexConjugate'  conjugates  '.coeff'  as well as all the
roots of unity making up the 'CycPol'.

Functions  'String' and  'Print' display  the $d$-th  cyclotomic polynomial
$\Phi_d$  over the rationals as 'Pd'.  They also display as 'P\'d', 'P\"d',
'P\"\'d', 'P\"\"d' factors of cyclotomic polynomials over extensions of the
rationals\:

|    gap> List(SchurElements(Hecke(ComplexReflectionGroup(4),q)),CycPol);
    [ P2^2P3P4P6, 2ER(-3)q^-4P2^2P'3P'6, -2ER(-3)q^-4P2^2P"3P"6,
      2q^-4P3P4, (3-ER(-3))/2q^-1P2^2P'3P"6, (3+ER(-3))/2q^-1P2^2P"3P'6, 
      q^-2P2^2P4 ]|

If  $\Phi_d$ factors in only  two pieces, the one  which has root 'E(d)' is
denoted  'P\'d' and the  other one 'P\"d'  . The list  of commonly occuring
factors  is as  follows (note  that the  conventions in \cite{Car85}, pages
489--490 are different)\:

% The following program makes the table below:
% for n in Concatenation([3..16],[18,20,21,22,24,25,26,27,30,42]) do
%   d:=CycPolOps.PhiDecompositions(n);d:=d{[2..Length(d)]};
%   for l in d do
%     p:=CycPol(Product(l,i->Mvp("q")-E(n)^i));
%     Print(p,"=",Format(Value(p,Mvp("q")),rec(GAP:=true,reverse:=true)),"\n");
%   od;
% od;
|    P'3=q-E(3)
    P"3=q-E(3)^2
    P'4=q-E(4)
    P"4=q+E(4)
    P'5=q^2+(1-ER(5))/2*q+1
    P"5=q^2+(1+ER(5))/2*q+1
    P'6=q+E(3)^2
    P"6=q+E(3)
    P'7=q^3+(1-ER(-7))/2*q^2+(-1-ER(-7))/2*q-1
    P"7=q^3+(1+ER(-7))/2*q^2+(-1+ER(-7))/2*q-1
    P'8=q^2-E(4)
    P"8=q^2+E(4)
    P"'8=q^2-ER(2)*q+1
    P""8=q^2+ER(2)*q+1
    P""'8=q^2-ER(-2)*q-1
    P"""8=q^2+ER(-2)*q-1
    P'9=q^3-E(3)
    P"9=q^3-E(3)^2
    P'10=q^2+(-1-ER(5))/2*q+1
    P"10=q^2+(-1+ER(5))/2*q+1
    P'11=q^5+(1-ER(-11))/2*q^4-q^3+q^2+(-1-ER(-11))/2*q-1
    P"11=q^5+(1+ER(-11))/2*q^4-q^3+q^2+(-1+ER(-11))/2*q-1
    P'12=q^2-E(4)*q-1
    P"12=q^2+E(4)*q-1
    P"'12=q^2+E(3)^2
    P""12=q^2+E(3)
    P""'12=q^2-ER(3)*q+1
    P"""12=q^2+ER(3)*q+1
    P(7)12=q+E(12)^7
    P(8)12=q+E(12)^11
    P(9)12=q+E(12)
    P(10)12=q+E(12)^5
    P'13=q^6+(1-ER(13))/2*q^5+2*q^4+(-1-ER(13))/2*q^3+2*q^2+(1-ER(13))/2*q+1
    P"13=q^6+(1+ER(13))/2*q^5+2*q^4+(-1+ER(13))/2*q^3+2*q^2+(1+ER(13))/2*q+1
    P'14=q^3+(-1+ER(-7))/2*q^2+(-1-ER(-7))/2*q+1
    P"14=q^3+(-1-ER(-7))/2*q^2+(-1+ER(-7))/2*q+1
    P'15=q^4+(-1-ER(5))/2*q^3+(1+ER(5))/2*q^2+(-1-ER(5))/2*q+1
    P"15=q^4+(-1+ER(5))/2*q^3+(1-ER(5))/2*q^2+(-1+ER(5))/2*q+1
    P"'15=q^4+E(3)^2*q^3+E(3)*q^2+q+E(3)^2
    P""15=q^4+E(3)*q^3+E(3)^2*q^2+q+E(3)
    P""'15=q^2+((1+ER(5))*E(3)^2)/2*q+E(3)
    P"""15=q^2+((1-ER(5))*E(3)^2)/2*q+E(3)
    P(7)15=q^2+((1+ER(5))*E(3))/2*q+E(3)^2
    P(8)15=q^2+((1-ER(5))*E(3))/2*q+E(3)^2
    P'16=q^4-ER(2)*q^2+1
    P"16=q^4+ER(2)*q^2+1
    P'18=q^3+E(3)^2
    P"18=q^3+E(3)
    P'20=q^4+(-1-ER(5))/2*q^2+1
    P"20=q^4+(-1+ER(5))/2*q^2+1
    P"'20=q^4+E(4)*q^3-q^2-E(4)*q+1
    P""20=q^4-E(4)*q^3-q^2+E(4)*q+1
    P'21=q^6+E(3)*q^5+E(3)^2*q^4+q^3+E(3)*q^2+E(3)^2*q+1
    P"21=q^6+E(3)^2*q^5+E(3)*q^4+q^3+E(3)^2*q^2+E(3)*q+1
    P'22=q^5+(-1-ER(-11))/2*q^4-q^3-q^2+(-1+ER(-11))/2*q+1
    P"22=q^5+(-1+ER(-11))/2*q^4-q^3-q^2+(-1-ER(-11))/2*q+1
    P'24=q^4+E(3)^2
    P"24=q^4+E(3)
    P"'24=q^4-ER(2)*q^3+q^2-ER(2)*q+1
    P""24=q^4+ER(2)*q^3+q^2+ER(2)*q+1
    P""'24=q^4-ER(6)*q^3+3*q^2-ER(6)*q+1
    P"""24=q^4+ER(6)*q^3+3*q^2+ER(6)*q+1
    P(7)24=q^4+ER(-2)*q^3-q^2-ER(-2)*q+1
    P(8)24=q^4-ER(-2)*q^3-q^2+ER(-2)*q+1
    P(9)24=q^2+ER(-2)*E(3)^2*q-E(3)
    P(10)24=q^2-ER(-2)*E(3)^2*q-E(3)
    P(11)24=q^2+ER(-2)*E(3)*q-E(3)^2
    P(12)24=q^2-ER(-2)*E(3)*q-E(3)^2
    P'25=q^10+(1-ER(5))/2*q^5+1
    P"25=q^10+(1+ER(5))/2*q^5+1
    P'26=q^6+(-1-ER(13))/2*q^5+2*q^4+(1-ER(13))/2*q^3+2*q^2+(-1-ER(13))/2*q+1
    P"26=q^6+(-1+ER(13))/2*q^5+2*q^4+(1+ER(13))/2*q^3+2*q^2+(-1+ER(13))/2*q+1
    P'27=q^9-E(3)
    P"27=q^9-E(3)^2
    P'30=q^4+(1-ER(5))/2*q^3+(1-ER(5))/2*q^2+(1-ER(5))/2*q+1
    P"30=q^4+(1+ER(5))/2*q^3+(1+ER(5))/2*q^2+(1+ER(5))/2*q+1
    P"'30=q^4-E(3)*q^3+E(3)^2*q^2-q+E(3)
    P""30=q^4-E(3)^2*q^3+E(3)*q^2-q+E(3)^2
    P""'30=q^2+((-1+ER(5))*E(3)^2)/2*q+E(3)
    P"""30=q^2+((-1-ER(5))*E(3)^2)/2*q+E(3)
    P(7)30=q^2+((-1+ER(5))*E(3))/2*q+E(3)^2
    P(8)30=q^2+((-1-ER(5))*E(3))/2*q+E(3)^2
    P'42=q^6-E(3)^2*q^5+E(3)*q^4-q^3+E(3)^2*q^2-E(3)*q+1
    P"42=q^6-E(3)*q^5+E(3)^2*q^4-q^3+E(3)*q^2-E(3)^2*q+1|

Finally the function 'Format(c,options)' takes the options\:

'.vname':  a  string,  the  name  to  use  for printing the variable of the
'CycPol' instead of 'q'.

'.expand':  if set to 'true', each cyclotomic polynomial is replaced by its
value before being printed.

|    gap> p:=CycPol(q^6-1);
    P1P2P3P6
    gap> Format(p,rec(expand:=true));
    "(q-1)(q+1)(q^2+q+1)(q^2-q+1)"
    gap> Format(p,rec(expand:=true,vname:="x"));
    "(x-1)(x+1)(x^2+x+1)(x^2-x+1)"|

%%%%%%%%%%%%%%%%%%%%%%%%%%%%%%%%%%%%%%%%%%%%%%%%%%%%%%%%%%%%%%%%%%%%%%%%
