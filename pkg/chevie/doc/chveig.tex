%%%%%%%%%%%%%%%%%%%%%%%%%%%%%%%%%%%%%%%%%%%%%%%%%%%%%%%%%%%%%%%%%%%%%%%%%%%%%
%%
%A  chveig.tex       CHEVIE documentation        Jean Michel
%%
%Y  Copyright (C) 2010 - 2014   University  Paris VII.
%%
%%  This  file  contains  the  description  of  the  GAP functions of CHEVIE
%%  dealing with reflection eigenspaces a d-Harish-Chandra series.
%%%%%%%%%%%%%%%%%%%%%%%%%%%%%%%%%%%%%%%%%%%%%%%%%%%%%%%%%%%%%%%%%%%%%%%%%

\Chapter{Eigenspaces and $d$-Harish-Chandra series}

Let  $W\phi$ be  a reflection  coset on  a vector  space $V$ and $Lw\phi$ a
reflection  subcoset where  $L$ is  a parabolic  subgroup (the fixator of a
subspace  of $V$). There are several  interesting cases where the *relative
group*  $N_W(Lw\phi)/L$, or a subgroup of  it normalizing some further data
attached to $L$, is itself a reflection group.

A first example is the case where $\phi=1$ and $w=1$, $W$ is the Weyl group
of  a  finite  reductive  group  $\bG^F$  and  the  Levi  subgroup  $\bL^F$
corresponding to $L$ has a cuspidal unipotent character. Then $N_W(L)/L$ is
a  Coxeter group acting  on the space  $X(Z\bL)\otimes\R$. A combinatorial
characterization of such parabolic subgroups of Coxeter groups is that they
are  normalized by the  longest element of  larger parabolic subgroups (see
\cite[5.7.1]{Lus76}).

A  second example is when $L$ is  trivial and $w\phi$ is a *$\zeta$-regular
element*,  that is the  $\zeta$-eigenspace $V_\zeta$ of  $w\phi$ contains a
vector outside all the reflecting hyperplanes of $W$. Then
$N_W(Lw\phi)/L=C_W(w\phi)$   is  a  reflection  group   in  its  action  on
$V_\zeta$.

A similar but more general example is when $V_\zeta$ is the
$\zeta$-eigenspace  of some element of the reflection coset $W\phi$, and is
of  maximal dimension among such possible $\zeta$-eigenspaces. Then the set
of  elements of  $W\phi$ which  act by  $\zeta$ on  $V_\zeta$ is  a certain
subcoset  $Lw\phi$, and $N_W(Lw\phi)/L$ is a reflection group in its action
on $V_\zeta$ (see \cite[2.5]{LS99}).

Finally,  a  still  more  general  example,  but which only occurs for Weyl
groups or Spetsial reflection groups, is when $\bL$ is a $\zeta$-split Levi
subgroup (which means that the corresponding subcoset $Lw\phi$ is formed of
all  the elements which act by $\zeta$  on some subspace $V_\zeta$ of $V$),
and  $\lambda$ is a $d$-cuspidal unipotent  character of $\bL$ (which means
that  the multiplicity of $\zeta$  as a root of  the degree of $\lambda$ is
the  same as the multiplicity of $\zeta$ as  a root of the generic order of
the  semi-simple part of $\bG$);  then $N_W(Lw\phi,\lambda)/L$ is a complex
reflection group in its action on $V_\zeta$.

Further, in the above cases the relative group describes the decomposition
of a Lusztig induction.

When  $\bG^F$  is  a  finite  reductive  group,  and  $\lambda$  a cuspidal
unipotent character of the Levi subgroup $\bL^F$, then the
$\bG^F$-endomorphism  algebra of the  Harish-Chandra induced representation
$R_\bL^\bG(\lambda)$  is a Hecke algebra  attached to the group $N_W(L)/L$,
thus   the  dimension  of  the  characters   of  this  group  describe  the
multiplicities in the Harish-Chandra induced.

Similarly,  when $\bL$ is a $\zeta$-split Levi subgroup, and $\lambda$ is a
$d$-cuspidal   unipotent  character  of   $\bL$  then  (conjecturally)  the
$\bG^F$-endomorphism algebra of the Lusztig induced $R_\bL^\bG(\lambda)$ is
a  cyclotomic Hecke algebra  for to the  group $N_W(Lw\phi,\lambda)/L$. The
constituents  of $R_\bL^\bG(\lambda)$  are called  a $\zeta$-Harish-Chandra
series.  In the case of rational  groups or cosets, corresponding to finite
reductive groups, the conjugacy class of $Lw\phi$ depends only on the order
$d$  of  $\zeta$,  so  one  also  talks of $d$-Harish-Chandra series. These
series  correspond  to  $\ell$-blocks  where  $l$  is  a  prime  divisor of
$\Phi_d(q)$  which does not divide any other cyclotomic factor of the order
of $\bG^F$.

The  \CHEVIE\ functions  described in  this chapter  allow to explore these
situations.
%%%%%%%%%%%%%%%%%%%%%%%%%%%%%%%%%%%%%%%%%%%%%%%%%%%%%%%%%%%%%%%%%%%%%%%%%
\Section{RelativeDegrees}
\index{RelativeDegrees}

'RelativeDegrees(<WF> [,<d>])'

Let  <WF> be a reflection group or a reflection coset. Here <d> specifies a
root  of unity  $\zeta$\:\ either  <d> is  an integer and specifies $\zeta$
'=E(d)' or is a fraction smaller $a/b$ with $0\<a\<b$ and specifies $\zeta$
'=E(b)\^a'.  If omitted, <d> is taken to be $0$, specifying $\zeta=1$. Then
if  $V_\zeta$ is the $\zeta$-eigenspace of some  element of <WF>, and is of
maximal  dimension among such possible  $\zeta$-eigenspaces, and <W> is the
group of <WF> then $N_W(V_\zeta)/C_W(V_\zeta)$ is a reflection group in its
action  on $V_\zeta$. The function 'RelativeDegrees' returns the reflection
degrees  of this complex reflection  group, which are a  subset of those of
<W>.

The point is that these degrees are obtained quickly by invariant-theoretic
computations\:\ if $(d_1,\varepsilon_1),\ldots,(d_n,\varepsilon_n)$ are the
generalized degrees of <WF> they are the $d_i$ such that
$\zeta^{d_i}=\varepsilon_i$.

|    gap> W:=CoxeterGroup("E",8);
    CoxeterGroup("E",8)
    gap> RelativeDegrees(W,4);
    [ 8, 12, 20, 24 ]|

%%%%%%%%%%%%%%%%%%%%%%%%%%%%%%%%%%%%%%%%%%%%%%%%%%%%%%%%%%%%%%%%%%%%%%%%%
\Section{RegularEigenvalues}
\index{RegularEigenvalues}

'RegularEigenvalues(<W>)'

Let  <W>  be  a  reflection  group  or  a reflection coset. A root of unity
$\zeta$  is a *regular eigenvalue*  for <W> if some  element of $<W>$ has a
$\zeta$-eigenvector  which lies outside of  the reflecting hyperplanes. The
function 'RelativeDegree' returns a list describing the regular eigenvalues
for   <W>.  If  all  the  primitive  $n$-th  roots  of  unity  are  regular
eigenvalues,  then $n$ is  put on the  result list. Otherwise the fractions
$a/n$  are added to the list for each $a$ such that $E(n)^a$ is a primitive
$n$-root of unity and a regular eigenvalue for <W>.

|    gap> W:=CoxeterGroup("E",8);;
    gap> RegularEigenvalues(W);
    [ 1, 2, 3, 4, 5, 6, 8, 10, 12, 15, 20, 24, 30 ]
    gap> W:=ComplexReflectionGroup(6);;
    gap> L:=Twistings(W,[2])[4];
    Z3[I]<2>.(q-I)
    gap> RegularEigenvalues(L);
    [ 1/4, 7/12, 11/12 ]|

%%%%%%%%%%%%%%%%%%%%%%%%%%%%%%%%%%%%%%%%%%%%%%%%%%%%%%%%%%%%%%%%%%%%%%%%%
\Section{PositionRegularClass}
\index{PositionRegularClass}

'PositionRegularClass(<WF> [,<d>])'

Let  <WF> be a reflection group or a reflection coset. Here <d> specifies a
root  of unity  $\zeta$\:\ either  <d> is  an integer and specifies $\zeta$
'=E(d)' or is a fraction smaller $a/b$ with $0\<a\<b$ and specifies $\zeta$
'=E(b)\^a'.  If omitted, <d> is taken  to be $0$, specifying $\zeta=1$. The
root $\zeta$ should be a regular eigenvalue for <WF> (see
"RegularEigenvalues").  The  function  returns  the  index of the conjugacy
class of <WF> which has a $\zeta$-regular eigenvector.

|    gap>  W:=CoxeterGroup("E",8);;
    gap> PositionRegularClass(W,30);
    65
    gap> W:=ComplexReflectionGroup(6);;
    gap> L:=Twistings(W,[2])[4];
    Z3[I]<2>.(q-I)
    gap> PositionRegularClass(L,7/12);
    2|

%%%%%%%%%%%%%%%%%%%%%%%%%%%%%%%%%%%%%%%%%%%%%%%%%%%%%%%%%%%%%%%%%%%%%%%%%
\Section{EigenspaceProjector}
\index{EigenspaceProjector}

'EigenspaceProjector(<WF>, <w> ,<d>)'

Let  <WF> be a reflection group or a reflection coset. Here <d> specifies a
root  of unity  $\zeta$\:\ either  <d> is  an integer and specifies $\zeta$
'=E(d)' or is a fraction smaller $a/b$ with $0\<a\<b$ and specifies $\zeta$
'=E(b)\^a'.  The function returns the unique <w>-invariant projector on the
$\zeta$-eigenspace of <w>.

|    gap> W:=CoxeterGroup("A",3);
    CoxeterGroup("A",3)
    gap> w:=EltWord(W,[1..3]);
    ( 1,12, 3, 2)( 4,11,10, 5)( 6, 9, 8, 7)
    gap> EigenspaceProjector(W,w,1/4);
    [ [ 1/4+1/4*E(4), 1/2*E(4), -1/4+1/4*E(4) ],
      [ 1/4-1/4*E(4), 1/2, 1/4+1/4*E(4) ], 
      [ -1/4-1/4*E(4), -1/2*E(4), 1/4-1/4*E(4) ] ]
    gap> RankMat(last);
    1|

%%%%%%%%%%%%%%%%%%%%%%%%%%%%%%%%%%%%%%%%%%%%%%%%%%%%%%%%%%%%%%%%%%%%%%%%%
\Section{SplitLevis}
\index{SplitLevis}

'SplitLevis(<WF> [, <d> [,<ad>]])'

Let  <WF>  be  a  reflection  group  or  a  reflection  coset.  If <W> is a
reflection  group it is  treated as the  trivial coset 'Spets(W)'. 

Here  <d> specifies a root of unity $\zeta$\:\ either <d> is an integer and
specifies  $\zeta$'=E(d)'  or  is  a  fraction  $a/b$  with  $0\<a\<b$  and
specifies $\zeta$'=E(b)\^a'. If omitted, <d> is taken to be $0$, specifying
$\zeta=1$.

A  *Levi* is a  subcoset of the  form $W_1F_1$ where  $W_1$ is a *parabolic
subgroup* of $W$, that is the centralizer of some subspace of $V$.

The  function returns  a list  of representatives  of conjugacy  classes of
$d$-split  Levis of <W>. A  $d$-split Levi is a  subcoset of <WF> formed of
all the elements which act by $\zeta$ on a given subspace $V_\zeta$. If the
additional  argument <ad>  is given,  it returns  only those subcosets such
that the common $\zeta$-eigenspace of their elements is of dimension <ad>.

|    gap> W:=CoxeterGroup("A",3);
    CoxeterGroup("A",3)
    gap> SplitLevis(W,4);
    [ A3, (q+1)(q^2+1) ]
    gap> 3D4:=CoxeterCoset(CoxeterGroup("D",4),(1,2,4));
    3D4
    gap> SplitLevis(3D4,3);
    [ 3D4, A2<1,3>.(q^2+q+1), (q^2+q+1)^2 ]
    gap> W:=CoxeterGroup("E",8);                      
    CoxeterGroup("E",8)
    gap> SplitLevis(W,4,2);
    [ D4<3,2,4,5>.(q^2+1)^2, (A1xA1)<5,7>x(A1xA1)<2,3>.(q^2+1)^2, 
      2(A2xA2)<3,1,5,6>.(q^2+1)^2 ]|

%%%%%%%%%%%%%%%%%%%%%%%%%%%%%%%%%%%%%%%%%%%%%%%%%%%%%%%%%%%%%%%%%%%%%%%%%
