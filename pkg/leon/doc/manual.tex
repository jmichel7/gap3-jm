% Twelve point font
\font\twelverm=cmr12
\font\twelvei=cmmi12
\font\twelvesy=cmsy10 scaled \magstep1
\font\twelveex=cmex10 scaled \magstep1
\font\twelvebf=cmbx12
\font\twelvesl=cmsl12
\font\twelvett=cmtt12
\font\twelveit=cmti12
\font\ninerm=cmr9
\font\ninei=cmmi9
\font\ninesy=cmsy9
\font\ninebf=cmbx9
\font\sevenrm=cmr7
\font\seveni=cmmi7
\font\sevensy=cmsy7
\font\sevenbf=cmbx7
\def\twelvepoint{\def\rm{\fam0\twelverm}
   \textfont0=\twelverm \scriptfont0=\ninerm \scriptscriptfont0=\sevenrm
   \textfont1=\twelvei  \scriptfont1=\ninei  \scriptscriptfont1=\seveni
   \textfont2=\twelvesy \scriptfont2=\ninesy \scriptscriptfont2=\sevensy
   \textfont3=\twelveex \scriptfont3=\twelveex\scriptscriptfont3=\twelveex
   \textfont\itfam=\twelveit  \def\it{\fam\itfam\twelveit}%
   \textfont\slfam=\twelvesl  \def\sl{\fam\slfam\twelvesl}%
   \textfont\ttfam=\twelvett  \def\tt{\fam\ttfam\twelvett}%
   \textfont\bffam=\twelvebf  \scriptfont\bffam=\ninebf
   \scriptscriptfont\bffam=\sevenbf  \def\bf{\fam\bffam\twelvebf}%
   \normalbaselineskip=14pt
   \setbox\strutbox=\hbox{\vrule height9.5pt depth4.5pt width0pt}%
   \normalbaselines\rm}
%
% 11 point font
\font\elevenrm=cmr10    scaled \magstephalf
\font\eleveni=cmmi10    scaled \magstephalf
\font\elevensy=cmsy10   scaled \magstephalf
\font\elevenex=cmex10   scaled \magstephalf
\font\elevenbf=cmbx10   scaled \magstephalf
\font\elevensl=cmsl10   scaled \magstephalf
\font\eleventt=cmtt10   scaled \magstephalf
\font\elevenit=cmti10   scaled \magstephalf
\font\eightrm=cmr8
\font\eighti=cmmi8
\font\eightsy=cmsy8
\font\eightbf=cmbx8
\font\sixrm=cmr6
\font\sixi=cmmi6
\font\sixsy=cmsy6
\font\sixbf=cmbx6
\def\elevenpoint{\def\rm{\fam0\elevenrm}% switch to 11-point type
   \textfont0=\elevenrm \scriptfont0=\eightrm \scriptscriptfont0=\sixrm
   \textfont1=\eleveni  \scriptfont1=\eighti  \scriptscriptfont1=\sixi
   \textfont2=\elevensy \scriptfont2=\eightsy \scriptscriptfont2=\sixsy
   \textfont3=\elevenex \scriptfont3=\elevenex\scriptscriptfont3=\elevenex
   \textfont\itfam=\elevenit  \def\it{\fam\itfam\elevenit}%
   \textfont\slfam=\elevensl  \def\sl{\fam\slfam\elevensl}%
   \textfont\ttfam=\eleventt  \def\tt{\fam\ttfam\eleventt}%
   \textfont\bffam=\elevenbf  \scriptfont\bffam=\eightbf
   \scriptscriptfont\bffam=\sixbf  \def\bf{\fam\bffam\elevenbf}%
   \normalbaselineskip=13pt
   \setbox\strutbox=\hbox{\vrule height9pt depth4pt width0pt}%
   \normalbaselines\rm}
%
\twelvepoint
\parindent=0pt
\raggedbottom

% Make headline containing date
\def\makeheadline{\vbox to 0pt{\vskip-45pt
   \line{\vbox to 8.5pt{}\hskip-40pt\fiverm{\number\month}/{\number\day}/92\hfil}\vss}
   \nointerlineskip}
%
\def\makeactive#1{\catcode`#1 = \active \ignorespaces}%
\chardef\letter = 11 \chardef\other = 12
\catcode`@=\letter
\def\alwaysspace{\hglue\fontdimen2\the\font \relax}%
{\makeactive\^^M \makeactive\ %
\gdef\obeywhitespace{%
\makeactive\^^M\def^^M{\par\indent}%
\aftergroup\@removebox%
\makeactive\ \let =\alwaysspace}}%
\def\@removebox{\setbox0=\lastbox}
%
\font\titlefont=cmbx10 scaled\magstep4
\font\authorfont=cmr12 scaled\magstep1
\font\sectionheaderfont=cmbx12 scaled\magstep1
\font\subsectionheaderfont=cmbx12
\def\filenamefont{\tt}
%
\long\def\section#1{\bigbigbreak{\sectionheaderfont #1}\nobreak\medskip\nobreak}
\long\def\subsection#1{\bigbreak{\subsectionheaderfont #1}\hskip0.75em}
\def\twocols#1#2{\hskip0.45truein\hbox to 3.0truein{#1\hfil}\hskip0.4truein\relax
                 \hbox to 2.45truein{#2\hfil}\hfil}
\long\def\objectfile#1{{\elevenpoint\obeylines\obeyspaces\tt\ttraggedright
                        #1\vskip0pt}}
\newdimen\optionIndent\optionIndent=1.1in
\long\def\defoption#1#2{{\advance\leftskip by1em\advance\leftskip by\optionIndent
                       \noindent\llap{\hbox to\optionIndent{\tt #1\hfil}}#2\vskip0pt}}
%
\def\germR{\underline{{\rm R}}}
\def\bfgermR{\underline{{\bf R}}}
%
\def\options{{\rm [}{\it options\/}{\rm ]}}
%
\newskip\bigbigskipamount \bigbigskipamount=20pt plus 7pt minus 5pt
\def\bigbigskip{\vskip\bigbigskipamount}
\def\bigbigbreak{\par\ifdim\lastskip<\bigbigskipamount
  \removelastskip\penalty-300\bigbigskip\fi}
\multiply\smallskipamount by4\divide\smallskipamount by3
\multiply\medskipamount by4\divide\medskipamount by3
\multiply\bigskipamount by4\divide\bigskipamount by3
\multiply\bigbigskipamount by4\divide\bigbigskipamount by3
%
\pageno=1
\centerline{{\titlefont PARTITION BACKTRACK PROGRAMS:}}
\vskip10pt
\centerline{{\titlefont USER'S MANUAL}}
\vskip35pt
\centerline{{\authorfont JEFFREY S. LEON}}
\vskip6pt
\centerline{Mathmatics Dept, m/c 249}
\centerline{University of Illinois at Chicago}
\centerline{Box 4348}
\centerline{Chicago, IL 60680}
\vskip40pt
%
\section{I.\quad INTRODUCTION}
%
This document describes a collection of programs for permutation group
computations employing the partition backtrack method, as described in a
recent article by the author (Leon, 1991).  At present, these programs
perform the following computations:
\medskip
\twocols{set stabilizers,}{set images (see below),}
\vskip4pt
\twocols{ordered partition stabilizers,}{ordered partition images,}
\vskip4pt
\twocols{intersections,}{}
\vskip4pt
\twocols{centralizers (of elements),}{conjugacy (of elements),}
\vskip4pt
\twocols{centralizers (of subgroups),}{}
\vskip4pt
\twocols{automorphism groups of designs,}{isomorphism of designs,}
\vskip4pt
\twocols{automorphism groups of matrices,}{isomorphism of matrices,}
\vskip4pt
\twocols{monomial automorphism groups of}{monomial isomorphism of matrices}
\vskip-1.5pt
\twocols{matrices over small fields,}{over small fields,}
\vskip4pt
\twocols{automorphism groups of linear codes,}{isomorphism of linear codes.}
\medbreak
The term {\it set image} problem is used here to refer to the following
problem: Given a permutation group
$G$ and subsets $\Lambda$ and $\Phi$ of the domain, determine if there exists
$g \in G$ such that $\Lambda^g = \Phi$.  The ordered partition image problem is
defined analogously.  The term {\it design\/} is used here to refer to any
collection of points and blocks.  An automorphism of a matrix is any permutation
of the rows and columns that leaves the matrix invariant; two matrices are
isomorphic if one may be transformed to the other by permutation of rows and
columns.  For monomial automorphism groups and monomial isomorphism, the matrix entries
must be taken from a finite field; in addition to permutation of rows and
columns, we allow each row and each column to be multiplied by a nonzero
field element.
%
\medbreak
Note that each of the problems in the first column above involves
computation of a subgroup; these problems will be referred to
as {\it subgroup computations}.  For each problem
in the second column, the set of permutations having the desired property
either is empty or forms a right coset of an appropriate subgroup; we are
seeking one coset representative (if it exists).  These problems
will be referred to as {\it coset representative computations}.
%
\medbreak
Some of the programs described here can be used to compute in groups of
relatively high degree, considerably higher than those that can be
handled by programs based on conventional algorithms.
However, it should be kept mind that the programs are new.
All appear to work correctly, but most have not
been thoroughly tested, especially on intransitive groups.  (The set
stabilizer program has been tested the most thoroughly, and in general those
for subgroup computations have received more testing than those for coset
representative calculations.)  The author would appreciate any reports of errors; they
may be sent to {\tt leon@turing.math.uic.edu}.
\medbreak
Work on programs for the following computations is in progress:
\medskip
\twocols{unordered partition stabilizers,}{unordered partition images,}
\vskip4pt
\twocols{normalizers,}{conjugacy (of subgroups),}
\vskip4pt
\twocols{}{coset intersections,}
\medbreak
In the course of constructing test cases for the partition backtrack programs
and verifying their output,
the author has developed several other programs, not based on backtrack search.
These programs are described briefly in Section~IX.  Many of these programs
were put together quickly, with a view toward simplicity rather than
efficiency and ease of use.
\medbreak
At present, the programs run on the following machines: The Sun/3, the Sun/4,
the IBM~3090, and the IBM~PC.  Two versions are available for the IBM~PC: a standard version,
which is limited to groups of degree no more than 1000 (roughly) due
to the 640K memory limitation, and a 386/486 version using a DOS extender, that
can handle larger groups.  The source code for all programs is written
entirely in C and, with very minor exceptions, conforms to the ANSI
standard.  The programs should compile, with minimal changes, with
any C compiler fully supporting the ANSI standard.  The Sun/3 and Sun/4
versions have been compiled with the GNU C compiler, the IBM~3090 version with the
Waterloo~C compiler, and the IBM/PC versions with the Borland C++ and
Zortech C++ compilers (both configured as C compilers).
%
%
\section{II.\quad OBJECTS AND FILE FORMATS}
%
At present, the programs compute with objects of seven types:  Permutation
groups, permutations, point sets, partitions (ordered or unordered),
block designs, matrices, and linear codes.  Each object used as
input by the programs is read from a file.  Likewise, each object constructed
by the programs is written to a file.  All of these files
are ordinary text files.  The format of the files is designed for
compatibility with Cayley (Cannon, 1984); it is that of a Cayley library,
with certain restrictions added.  Essentially, the restrictions say that
the library may contain only statements defining the object, and (at present)
that only certain attributes of the object may specified in the library.
(Many of the permutation group libraries distributed with Cayley conform
to these restrictions.)  Thus objects constructed by these programs 
described here may be read into Cayley for further investigation.
Likewise, objects defined in existing Cayley libraries may, in many cases,
be used as input to the programs described here.  An alternative format,
compatible with Gap, may be added at a later date.
\medbreak
The examples which follow illustrate the correct format for these object files.
Note that the contents of the files is case--insensitive; upper and lower case
letters may be used interchangeably.  (However, the names of the files may
be case--sensitive, depending on the operating system.)  Also, the
use of white space (blanks, tabs, newline characters) is optional:  Except
within integers and identifiers, any number of whitespace characters may
occur.  Text enclosed by ampersands or quotation marks is treated as a comment.
%
\subsection{a)\enskip Permutation groups:}The format for permutation group files
is illustrated by following file, named {\filenamefont psp62}, which defines
${\rm PSp}_6(2)$ as a permutation group
on nonzero vectors (degree 63).
\medbreak
\vbox{{\elevenpoint\advance\leftskip by 0.45truein\obeywhitespace\tt\catcode`^=\other
LIBRARY psp62;
" PSp(6,2) acting on nonzero vectors, degree 63."
psp62: permutation group(63);
psp62.forder: 2^9 * 3^4 * 5 * 7;
psp62.generators:
  a = (1,2)(3,5)(4,7)(8,12)(11,16)(13,19)(17,18)(20,26)(21,28)(23,30)(24,32)
      (25,34)(29,37)(31,40)(33,43)(36,46)(38,41)(39,49)(42,44)(45,52)(48,51)
      (53,58)(57,62)(59,61),
  b = (1,3,6,10,15,22)(2,4,8,13,20,27)(5,9,14,21,29,38)(7,11,17,23,31,41)
      (12,18,24,33,44,34)(16,19,25)(26,35,45,53,32,42)(28,36,47)(30,39,50,
      56,61,58)(37,48)(40,51,46,54,59,62)(49,55,60,63,52,57);
FINISH;\vskip0pt}}
\medbreak
The line specifying the factored group order may be omitted; however, since the
random Schreier method is currently used to construct a base and strong
generating set for the group, there is a possibility (probably small) that
the group may be generated incorrectly if this line is removed.  
When generators are written in cycle format, as above, inclusion of cycles
of length one is optional.  (For compatibility with Cayley, they should be
omitted.)
It is also
possible to write the generators in image (rather than cycle) format; in
this case, the file shown above would become:
\medbreak
\vbox{{\elevenpoint\advance\leftskip by 0.45truein\obeywhitespace\tt\catcode`^=\other
LIBRARY psp62;
" PSp(6,2) acting on nonzero vectors, degree 63."
psp62: permutation group(63);
psp62.forder: 2^9 * 3^4 * 5 * 7;
psp62.generators:
   a = /2,1,5,7,3,6,4,12,9,10,16,8,19,14,15,11,18,17,13,26,28,22,30,
        32,34,20,27,21,37,23,40,24,43,25,35,46,29,41,49,31,38,44,33,42,
        52,36,47,51,39,50,48,45,58,54,55,56,62,53,61,60,59,57,63/,
   b = /3,4,6,8,9,10,11,13,14,15,17,18,20,21,22,19,23,24,25,27,29,1,
        31,33,16,35,2,36,38,39,41,42,44,12,45,47,48,5,50,51,7,26,43,34,
        53,54,28,37,55,56,46,57,32,59,60,61,49,30,62,63,58,40,52/;
FINISH;\vskip0pt}}
\medbreak
Finally, if a base and strong generating set for the group are already
known, they may be included in the file.  This eliminates the need for
the programs to first construct a base and strong generating set for
the input group.  The file format is then as follows.
\nobreak\medskip\nobreak
\vbox{{\elevenpoint\advance\leftskip by 0.45truein\obeywhitespace\tt\catcode`^=\other
LIBRARY psp62;
" PSp(6,2) acting on nonzero vectors, degree 63."
psp62: permutation group(63);
psp62.forder: 2^9 * 3^4 * 5 * 7;
psp62.base:  seq(1,3,6,2,4,5);
psp62.strong generators:  [
   x01 = (1,3)(4,27)(5,9)(7,45)(10,48)(11,17)(12,34)(13,20)(14,25)(15,
         39)(16,63)(19,60)(21,55)(22,58)(23,28)(24,37)(31,53)(32,47)(33,
         61)(36,42)(38,49)(44,50)(51,62)(54,59),
   x02 = (2,16)(3,6)(5,55)(8,21)(9,40)(13,51)(14,46)(15,47)(17,44)(19,
         59)(20,29)(22,36)(23,58)(24,61)(25,63)(26,37)(27,60)(31,50)(32,
         48)(33,41)(34,56)(38,57)(43,45)(52,62),
    .
    .

   x12 = (5,51)(7,12)(8,29)(9,62)(10,42)(13,55)(15,23)(18,35)(20,21)
         (22,32)(28,39)(34,45)(36,48)(40,52)(43,56)(47,58)];
FINISH;\vskip0pt}}
\medbreak
(The vertical dots indicated that part of the file has been omitted.)
When a base and strong generating set are given, inclusion of the factored
order of the group is purely optional.
At present it is {\it not} possible to specify both generators and a base and
strong generating set for a group.  (This obviously undesirable restriction
will be removed eventually.)
%
\subsection{b)\enskip Permutations:}The format for permutation files
is illustrated by following file, named {\filenamefont g}, which defines a permutation of
$g$ of degree 63 and order 4, which turns out to lie in the group 
${\rm PSp}_6(2)$ given above.
\medbreak
\vbox{{\elevenpoint\advance\leftskip by 0.45truein\obeywhitespace\tt\catcode`^=\other
LIBRARY g;
" An element of order 4 in the group psp62 above."
g = (1,40,50,6)(2,58,18,34)(3,8,44,30)(4,10,15,48)(5,11)(7,60,38,32)
    (12,46,22,56)(13,62,20,61)(16,42,36,63)(19,49,47,45)(21,53,31,
    55)(24,33,37,51)(25,28)(26,52)(27,59,39,54)(29,41);
FINISH;\vskip0pt}}
\medbreak
As with generators for permutation groups, permutations may be written in
image format, rather than cycle format.  Note that, when cycle format is used,
the file contains no explicit indication of the degree of the permutation.
Thus, for example, the permutation {\tt g} above could be used wherever a
permutation of degree 63 (the largest point appearing explicitly) or greater
is expected.
\medbreak
Given the files above, the centralizer in ${\rm PSp}_6(2)$ of $g$ could be
computed by the command
\smallskip
\hskip0.45truein{\tt cent\quad psp62\quad g\quad C}
\smallskip
which would save the centralizer (in the format described in part~(a) above)
in the file {\filenamefont C}.
%
\subsection{c)\enskip Point sets:}The format for point sets is illustrated
by following file, named {\filenamefont lambda}, which defines a subset $\Lambda$ of
$\{1,\ldots,63\}$.
\medbreak
\vbox{{\elevenpoint\advance\leftskip by 0.45truein\obeywhitespace\tt\catcode`^=\other
LIBRARY lambda;
" A subset of {1,...,63} of size 31."
lambda = [10,16,44,3,5,33,48,63,56,50,6,52,55,19,34,25,2,35,17,40,21,
          58,49,36,39,12,60,30,15,29,37];
FINISH;\vskip0pt}}
\medbreak
Note that there is no explicit indication of the size 
of the base set.  Thus the set {\tt lambda} above could be used wherever
a subset of $\{1,\ldots,m\}$ is expected for any $m$ with $m \geq 63$
(the largest point appearing explicitly).
\medbreak
The set stabilizer in ${\rm PSp}_6(2)$ of $\Lambda$ could be computed
by the command
\smallskip
\hskip0.45truein{\tt setstab\quad psp62\quad lambda\quad S}
\smallskip
which would save the stabilizer in the file {\filenamefont S}.
%
\subsection{d)\enskip Partitions:}The format for partitions (ordered or
unordered) is illustrated
by following file, named {\filenamefont pi}, which defines a partition
$\Pi$ of $\{1,\ldots,63\}$.
\medbreak
\vbox{{\elevenpoint\advance\leftskip by 0.45truein\obeywhitespace\tt\catcode`^=\other
LIBRARY pi;
" An (ordered or unordered) partition of 1,...,63 having four "
" cells of sizes 15, 20, 13, and 15. respectively."
pi = seq([1,34,28,48,37,41,13,54,57,51,4,38,8,46,16],[2,40,21,
          18,6,53,30,56,42,12,3,11,33,15,32,5,60,31,55,63],[7,
          36,25,29,35,9,26,49,14,47,10,24,43],[17,58,52,50,59,
          45,20,61,23,39,44,19,22,62,27]);
FINISH;\vskip0pt}}
\medbreak
Note that the individual cells are delimited by square brackets.
Note also that the file contains no indication whether the partition is ordered
or unordered; rather each program operating on partitions interprets the
partition as ordered or unordered, whichever is appropriate for the the
program.   
\medbreak
The stabilizer in ${\rm PSp}_6(2)$ of $\Pi$, interpreted as an ordered
partition, could be computed by the command
\smallskip
\hskip0.45truein{\tt parstab\quad psp62\quad pi\quad T}
\smallskip
which saves the stabilizer in the file {\filenamefont T}.
%
\subsection{e)\enskip Block designs:}Here a block design refers to any
collection of subsets of $\{1,\ldots,n\}$; it is even possible to have
repeated blocks, i.e., two different blocks containing exactly the same points.
(If repeated blocks occur, we require that an automorphism preserve
multiplicities.)  The file format for block designs
is illustrated by following file, named {\filenamefont d17}, which defines
a block design $D_{17}$ with 17 points and with 34 blocks, each of size 5.
\medbreak
\vbox{{\elevenpoint\advance\leftskip by 0.45truein\obeywhitespace\tt\catcode`^=\other
LIBRARY d17;
" The design with 17 points and 34 blocks, each containing "
" 5 points, obtained from the codewords of weight 5 in the "
" quadratic residue code of length 17 and dimension 9."
d17 = seq( 17, 34,
           [3,6,8,15,17], [1,4,7,9,16], [1,4,5,11,17],
           [2,5,8,10,17], [3,4,7,8,14], [1,2,5,6,12],
           [1,4,6,13,15], [1,3,6,9,11], [1,3,10,12,15],
           [3,5,8,11,13], [2,8,9,12,13], [1,7,13,14,17],
           [6,8,11,14,16], [4,5,8,9,15], [2,5,7,14,16],
           [2,3,6,7,13], [3,4,10,16,17], [3,5,12,14,17],
           [1,7,8,11,12], [5,7,10,13,15], [6,12,13,16,17],
           [2,4,7,10,12], [2,9,11,14,17], [2,3,9,15,16],
           [2,4,11,13,16], [6,7,10,11,17], [4,6,9,12,14],
           [5,11,12,15,16], [1,8,10,13,16], [1,2,8,14,15],
           [5,6,9,10,16], [4,10,11,14,15], [7,9,12,15,17],
           [3,9,10,13,14] );
FINISH;\vskip0pt}}
\medbreak
Note that the file contains the number of points, followed by the number
of blocks, followed by a listing of the blocks.  Each block is delimited
by square brackets.
\medbreak
The automorphism group of this block design could be computed by the command
\smallskip
\hskip0.45truein{\tt desauto\quad d17\quad A}
\smallskip
which saves the automorphism group in the file {\filenamefont A}.
%
\subsection{f)\enskip Matrices:}Here a matrix is merely a $k\times n$
array of integers, each in the set $\{0,\ldots,q-1\}$ for some $k$, $n$, and
$q$.  (At present, $q$ cannot exceed 256; this limit could be raised, at the
cost of additional space, by minor changes to the source code.)
The file format for matrices is
illustrated by following file, named {\filenamefont m17},
which represents incidence matrix $M_{17}$ for the block design $D_{17}$
in part~(e) above.  (In the incidence matrix, rows correspond to points and
columns to blocks.)
\medbreak
\vbox{{\elevenpoint\advance\leftskip by 0.45truein\obeywhitespace\tt\catcode`^=\other
LIBRARY m17;
" The incidence matrix of d17."
m17 = seq( 2, 17, 34, seq(
       0,1,1,0,0,1,1,1,1,0,0,1,0,0,0,0,0,0,1,0,0,0,0,0,0,0,0,0,1,1,0,0,0,0,
       0,0,0,1,0,1,0,0,0,0,1,0,0,0,1,1,0,0,0,0,0,1,1,1,1,0,0,0,0,1,0,0,0,0,
       1,0,0,0,1,0,0,1,1,1,0,0,0,0,0,1,1,1,0,0,0,0,0,1,0,0,0,0,0,0,0,0,0,1,
       0,1,1,0,1,0,1,0,0,0,0,0,0,1,0,0,1,0,0,0,0,1,0,0,1,0,1,0,0,0,0,1,0,0,
       0,0,1,1,0,1,0,0,0,1,0,0,0,1,1,0,0,1,0,1,0,0,0,0,0,0,0,1,0,0,1,0,0,0,
       1,0,0,0,0,1,1,1,0,0,0,0,1,0,0,1,0,0,0,0,1,0,0,0,0,1,1,0,0,0,1,0,0,0,
       0,1,0,0,1,0,0,0,0,0,0,1,0,0,1,1,0,0,1,1,0,1,0,0,0,1,0,0,0,0,0,0,1,0,
       1,0,0,1,1,0,0,0,0,1,1,0,1,1,0,0,0,0,1,0,0,0,0,0,0,0,0,0,1,1,0,0,0,0,
       0,1,0,0,0,0,0,1,0,0,1,0,0,1,0,0,0,0,0,0,0,0,1,1,0,0,1,0,0,0,1,0,1,1,
       0,0,0,1,0,0,0,0,1,0,0,0,0,0,0,0,1,0,0,1,0,1,0,0,0,1,0,0,1,0,1,1,0,1,
       0,0,1,0,0,0,0,1,0,1,0,0,1,0,0,0,0,0,1,0,0,0,1,0,1,1,0,1,0,0,0,1,0,0,
       0,0,0,0,0,1,0,0,1,0,1,0,0,0,0,0,0,1,1,0,1,1,0,0,0,0,1,1,0,0,0,0,1,0,
       0,0,0,0,0,0,1,0,0,1,1,1,0,0,0,1,0,0,0,1,1,0,0,0,1,0,0,0,1,0,0,0,0,1,
       0,0,0,0,1,0,0,0,0,0,0,1,1,0,1,0,0,1,0,0,0,0,1,0,0,0,1,0,0,1,0,1,0,1,
       1,0,0,0,0,0,1,0,1,0,0,0,0,1,0,0,0,0,0,1,0,0,0,1,0,0,0,1,0,1,0,1,1,0,
       0,1,0,0,0,0,0,0,0,0,0,0,1,0,1,0,1,0,0,0,1,0,0,1,1,0,0,1,1,0,1,0,0,0,
       1,0,1,1,0,0,0,0,0,0,0,1,0,0,0,0,1,1,0,0,1,0,1,0,0,1,0,0,0,0,0,0,1,0
      ));
FINISH;\vskip0pt}}
\medbreak
Note that the file contains the set size $q$, followed by the number $k$ of rows,
followed by the number $n$ of columns, followed by the entries of the matrix
listed in row--major order.  Note that matrix entries are not, in general,
limited to 0 and 1; if $q$ is the set size, matrix entries may be integers
in the range 0 through $q-1$.
\medbreak
The automorphism group of this matrix could be computed by the command
\smallskip
\hskip0.45truein{\tt matauto\quad m17\quad B}
\smallskip
which saves the automorphism group in the file {\filenamefont B}.
\medbreak
Matrices may also be specified using an alternate format, which is not
compatible with Cayley, but which saves considerable space for large
matrices.\footnote{${}^{\dag}$}{\elevenpoint At time of writing, the alternate format does not work
correctly on some machines.}  This alternate format is available only when the set size $q$ is
at most 9.  Using the alternate format, the matrix $M_{17}$ would be
specified by a file as follows:
\medbreak
\vbox{{\elevenpoint\advance\leftskip by 0.45truein\obeywhitespace\tt\catcode`^=\other
m17
2  17  34
0110011110010000001000000000110000
0001010000100011000001111000010000
1000100111000001110000010000000001
0110101000000100100001001010000100
0011010001000110010100000001001000
1000011100001001000010000110001000
0100100000010011001101000100000010
1001100001101100001000000000110000
0100000100100100000000110010001011
0001000010000000100101000100101101
0010000101001000001000101101000100
0000010010100000011011000011000010
0000001001110001000110001000100001
0000100000011010010000100010010101
1000001010000100000100010001010110
0100000000001010100010011001101000
1011000000010000110010100100000010
\vskip0pt}}
\medbreak
With the alternate format, blanks may occur between matrix entries, but
are not required.  The first line of the file is reserved for the matrix
name; nothing else may be placed on this line.
%
\subsection{g)\enskip Linear codes:}
The file format for lines codes is
illustrated by following file, named {\filenamefont q17}, which represents the
binary (17,9)--quadratic residue code $Q_{17}$ mentioned in part~(e) above.
This file gives a generator matrix for the code, in the format of
part~(f) above.
\medbreak
\vbox{{\elevenpoint\advance\leftskip by 0.45truein\obeywhitespace\tt\catcode`^=\other
LIBRARY q17;
" The (17,9) binary quadratic residue code."
q17 = seq( 2, 9, 17, seq(
           1,1,1,0,1,0,0,0,1,1,0,0,0,1,0,1,1,
           1,1,1,1,0,1,0,0,0,1,1,0,0,0,1,0,1,
           1,1,1,1,1,0,1,0,0,0,1,1,0,0,0,1,0,
           0,1,1,1,1,1,0,1,0,0,0,1,1,0,0,0,1,
           1,0,1,1,1,1,1,0,1,0,0,0,1,1,0,0,0,
           0,1,0,1,1,1,1,1,0,1,0,0,0,1,1,0,0,
           0,0,1,0,1,1,1,1,1,0,1,0,0,0,1,1,0,
           0,0,0,1,0,1,1,1,1,1,0,1,0,0,0,1,1,
           1,1,1,1,1,1,1,1,1,1,1,1,1,1,1,1,1,
      ));
FINISH;\vskip0pt}}
\medbreak
Note that the file contains the size of the field for the code, followed by the 
dimension of the code, followed by the
length of the code, followed by a generator matrix whose entries are
listed in row--major order.  At present, the field size is restricted to
a prime integer (less than 255) or to 4; automorphism group and isomorphism
calculations are practical only when the field size is quite small.
  In the case of a prime, the field
is taken as the integers modulo that prime.  
\medbreak
The automorphism group of this code could be computed by the command
\smallskip
\hskip0.45truein{\tt codeauto\quad q17\quad v17\quad H}
\smallskip
which saves the automorphism group as the group {\filenamefont H}.
(Here file {\filenamefont v17} defines a matrix $V_{17}$ which is the
transpose of the matrix $M_{17}$ of part~(f) above; the role of
$V_{17}$ will be explained later.)
\medbreak
As with matrices, an alternate (not Cayley--compatible) format is provided
for codes over field of size at most 9.\footnote{${}^{\dag}$}{\elevenpoint As with matrices, this
alternate format at present fails to work correctly on some machines.}
With the alternate format, the file defining the code $Q_{17}$
would be as follows:
\medbreak
\vbox{{\elevenpoint\advance\leftskip by 0.45truein\obeywhitespace\tt\catcode`^=\other
q17
2  9  17
11101000110001011
11110100011000101
11111010001100010
01111101000110001
10111110100011000
01011111010001100
00101111101000110
00010111110100011
11111111111111111
\vskip0pt}}
\bigbreak
\hrule
\bigbigbreak
In the examples above, it was assumed that the name of file matched the
name of the Cayley library that it contained.  Although this is
recommended for simplicity, it need not be the case.  When the names
do not match, an object is specified using the format
\hbox{{\it fileName\/}{\tt ::}{\it libraryName\/}}.  For example, if the file
{\filenamefont psp62} in part~(a) above were renamed {\filenamefont psp}
and the file {\filenamefont g} in part~(b) were named {\filenamefont pspx4},
but if the contents of both files remained unchanged, the command to
compute the centralizer in ${\rm PSp}_6(2)$ of $g$ might be
\smallskip
\hskip0.45truein{\tt cent\quad psp::psp62\quad pspx4::g\quad gCentr::C}
\smallskip
where now the centralizer is saved in the file {\filenamefont gCentr}, but in a
Cayley library named {\tt C}.  A path may also be specified, for example,
\smallskip
\hskip0.45in{\tt cent\quad ../groups/psp::psp62\quad ../groups/pspx4::g\quad gCentr::C}
\smallskip
in Unix.  (In MS~DOS on the IBM~PC, the forward slash must be replaced 
by a backslash.  Under CMS on the IBM~370, the file name and file type must
be separated by a period, rather than the blank normally used under CMS.)
If however, the file name and the library name for input files
differ only in that the file name contains path information, the
{\tt -p} option (discussed later) may be useful.
%
\medbreak
In the examples above, not only did the file name and the Cayley library
name match, but both matched the actual name for the object appearing in
the Cayley library.  Actually, the name for the object need not be
related to the name of the Cayley library.  For example, the following
is acceptable.
\medbreak
\vbox{{\elevenpoint\advance\leftskip by 0.45truein\obeywhitespace\tt\catcode`^=\other
LIBRARY psp62;
" PSp(6,2) acting on nonzero vectors, degree 63."
G: permutation group(63);
G.forder: 2^9 * 3^4 * 5 * 7;
G.generators:
  a = (1,2)(3,5)(4,7)(8,12)(11,16)(13,19)(17,18)(20,26)(21,28)(23,30)(24,32)
      (25,34)(29,37)(31,40)(33,43)(36,46)(38,41)(39,49)(42,44)(45,52)(48,51)
      (53,58)(57,62)(59,61),
  b = (1,3,6,10,15,22)(2,4,8,13,20,27)(5,9,14,21,29,38)(7,11,17,23,31,41)
      (12,18,24,33,44,34)(16,19,25)(26,35,45,53,32,42)(28,36,47)(30,39,50,
      56,61,58)(37,48)(40,51,46,54,59,62)(49,55,60,63,52,57);
FINISH;\vskip0pt}}
\medbreak
In this situation, the command line must still specify the Cayley library name
(and file name, if different), but informative messages printed as the
command executes use the object name ({\tt G}, in this case).  For
objects created by commands, an object name different from the Cayley
library name may be specified by means of the {\tt -n} option, discussed
later.
%
%
\section{III.\quad\kern-3pt FIELDS, MONOMIAL PERMUTATIONS, AND MATRICES\kern-2pt}
%
This section treats several topics that arise primarily in connection
with computations involving combinatorial structures -- designs, matrices,
and codes.
%
\subsection{i)\enskip Finite fields:}Finite fields arise when computing with
codes, or with matrices whose entries belong to the field.  At present, only 
fields ${\rm GF}(q)$ whose order $q$ is either 4 or a prime integer are 
supported; moreover, we must have $q\leq 255$.  (For automorphism
group and isomorphism calculations, time and space considerations generally 
dictate a practical limit on $q$ that is far lower.)  Field elements are 
numbered $0,1,\ldots,q-1$.  When $q$ is prime, the field is taken as the 
integers modulo $q$.  When $q=4$, there is an essentially unique way to 
number the field elements.  
\smallbreak
We denote the set of nonzero elements of
${\rm GF}(q)$ by ${\rm GF}(q)^{\#}$.
\bigbreak
\subsection{ii)\enskip Monomial permutations:}Given a fixed field 
${\rm GF}(q)$, a monomial permutation of monomial degree $n$ over 
${\rm GF}(q)$ is essentially a permutation $s$ on 
${\rm GF}(q)^{\#} \times \{1,\ldots,n\}$ which satisfies the following property, 
henceforth referred to as the {\it monomial property:\/}
$$\displaylines{(\alpha,i)^s = (\beta,j)\quad {\rm implies}\quad
  (\gamma\alpha,i)^s = (\gamma\beta,j)\kern40pt\cr\kern40pt 
  {\rm \hbox{for all}}\enskip
  \alpha,\beta,\gamma\in {\rm GF}(q)^{\#}\enskip {\rm and}\enskip 
  i,j\in\{1,\ldots,n\}.\cr}$$
Note that $s$ is determined completely by its action on the points $(1,i)$,
$1\leq i\leq n$; note also that the actual degree of $s$ is $(q-1)n$.
\smallbreak
For purposes of actual computation, however, we want 
a representation of $s$ as a 
permutation on $\{1,\ldots,(q-1)n\}$; to obtain this, 
we number the pair $(\alpha,i)$ by 
$(q-1)(i-1)+\overline{\alpha}$, where $\overline{\alpha}$ denotes the integer
representing $\alpha$.  Then the monomial property becomes
$$\displaylines{\bigl((q-1)(i-1)+\overline{\alpha}\bigr)^s = 
(q-1)(j-1)+\overline{\beta}\quad{\rm implies}\kern40pt\cr
  \kern40pt\bigl((q-1)(i-1)+\overline{\gamma\alpha}\bigr)^s = 
(q-1)(j-1)+\overline{\gamma\beta}.\cr}$$
\medbreak
For example, over the field ${\rm GF}(4)$, the monomial permutation $s$ on
${\rm GF}(4) \times \{1,2,3,4\}$ determined by
$$ (1,1)^s = (3,2),\quad (1,2)^s = (2,4),\quad (1,3)^s = (1,1),\quad
   (1,4)^s = (2,3)$$
is represented as a permutation on $\{1,\ldots,12\}$ as follows:
$$(1,6,10,8,2,4,11,9,3,5,12,7).$$
%
\bigbreak
\subsection{iii)\enskip Permutations acting on matrices:}Let 
$A = (a_{ij})$ be an $r \times c$ matrix with entries from an arbitrary set.
A permutation $s$ of degree $r+c$ which fixes $\{1,\ldots,r\}$ setwise
induces an action on $A$ as follows:  Row $i$ of $A$ is moved to row position $i^s$\enskip
$(1\leq i\leq r)$ and column $j$ of $A$ is moved to column position
$(r+j)^s-r$\enskip $(1\leq j\leq c)$.  Thus
$$A^s = (b_{ij}),\quad {\rm where}\quad 
  b_{ij} = a_{i^\prime j^\prime},\enskip {\rm with}\enskip
  i^\prime = i^{s^{-1}}\enskip {\rm and}\enskip
  j^\prime = (r+j)^{s^{-1}}-r.$$
If $A^s = B$, we say that $s$ is an {\it isomorphism\/} of $A$ to $B$.  When
$A^s=A$,\enskip $s$ is called an {\it automorphism\/} of $A$.  The group formed by 
the automorphisms is called the {\it automorphism group\/} of $A$, and denoted
${\rm AUT}(A)$.
\medbreak
For example, the action of a permutation $s$ of degree 7 on a $3\times 4$
matrix $A$ is illustrated by the following.
$$s = (1,3,2)(4,7,5),\qquad
  A =   \pmatrix{8&0&4&3\cr
                2&9&3&0\cr
                0&1&7&5\cr},\qquad
  A^s = \pmatrix{9&0&3&2\cr
                1&5&7&0\cr
                0&3&4&8\cr}.$$
%
\bigbreak
\subsection{iv)\enskip Monomial permutations acting on matrices:}Now let 
$A = (a_{ij})$ be an $r \times c$ matrix with entries from a field
${\rm GF}(q)$.  A monomial permutation $s$ of monomial degree $r+c\kern3pt$
(actual degree $(q-1)(r+c)\kern2pt)$ which fixes $\{1,\ldots,(q-1)r\}$ setwise
induces an action on $A$.  This action is most easily described if we think of $s$ as
a permutation on ${\rm GF}(q)^{\#} \times \{1,\ldots,n\}$, as in~(ii) above;
then $s$ fixes $\{(\alpha,i)\vert \alpha\in{\rm GF}(q)^{\#},1\leq i\leq r\}$
setwise.  If $(1,i)^s = (\alpha,k)$ and $(1,r+j)^s = (\beta,r+m)$, 
then row $i$ of $A$ is multiplied by $\alpha$ and moved to row position $k$,\enskip
and column $j$ of $A$ is multiplied by $\beta$ and moved to column position
$m$.  Thus
$$ A^s = (b_{ij}),\quad {\rm where}\quad
   b_{ij} = \lambda^{-1}\mu^{-1}a_{i^\prime j^\prime},$$
with $\lambda$, $\mu$, $i^\prime$, and $j^\prime$ determined by
$$ (\lambda,i^\prime) = (1,i)^{s^{-1}}\quad {\rm and}\quad
   (\mu,r+j^\prime) = (1,r+j)^{s^{-1}}.$$
If $A^s = B$, we say that $s$ is an {\it monomial isomorphism\/} of $A$ to $B$.  When
$A^s=A$,\enskip $s$ is called a {\it monomial automorphism\/} of $A$.  The group 
formed by the automorphisms is called the {\it monomial automorphism group\/} of $A$, and denoted
${\rm AUT}^*(A)$.
\medbreak
For example, over the field ${\rm GF}(4)$, the action of a 
monomial permutation $s$ of monomial degree 5 (actual degree 15) 
on a $2\times 3$ matrix $A$ is illustrated by the following.
$$\displaylines{(1,1)^s = (3,2),\enskip\kern2pt (1,2)^s = (1,1),\enskip\kern2pt
  (1,3)^s = (2,5),\enskip\kern2pt (1,4)^s = (3,3),\enskip\kern2pt (1,5)^s = (2,4),\cr
  s = (1,6,3,5,2,4)(7,14,12,8,15,10,9,13,11),\quad\enskip
  A =   \pmatrix{2&0&3\cr
                 0&3&1\cr},\quad\enskip
  A^s = \pmatrix{2&2&0\cr
                 0&3&2\cr}.\cr}$$
%
% 
%
%
\section{IV.\quad PARTITION BACKTRACK COMMANDS}
%
The commands employing the partition backtrack method that are currently 
available are described below.  Note material
in square brackets is optional.  (The brackets themselves are not to be
typed.)  Discussion of most of the available options will be
deferred to Section~V; only those unique to a specific command will be
mentioned here.  
\medskip
Options are never required, but they may prove
useful in controlling the format of the output or the procedures used in the
computation. For example, certain options allow for a time versus space
tradeoff.  For some ``unusual'' groups (e.g., very dense imprimitive groups),
it may be necessary to specify nonstandard options in order to obtain
acceptable performance.
\medskip
The partition backtrack programs described here represent full implementations
of the partition backtrack method, as set forth in (Leon, 1991), with
two exceptions.
{\smallskip\advance\leftskip by 0.45in\relax
 \item{i)}The criterion in Prop.~8(iii) is not checked.
 \smallskip
 \item{ii)}In coset--type computations, the refinement $\germR^+$ of Figure~8 is
           always taken as $\germR$\footnote{${}^{\dag}$}{\elevenpoint In order to allow
           this manual to be printed without special AMS~TeX fonts, underlined
           letters (e.g., $\germR$) are used here as a substitute for 
           letters appearing in the Euler Fraktur (German) font in (Leon, 1991).}
\smallskip}
%
%
\subsection{Set stabilizers:}Set stabilizers may be computed
by the {\tt setstab} command.  The format is
\smallskip
\hskip0.45truein{\tt setstab}\quad \options\quad {\it permGroup\quad pointSet\quad
                stabilizerSubgroup}
\smallskip
This command computes the set stabilizer in the permutation group {\it permGroup\/}
of the set {\it pointSet\/} and saves the result (in Cayley library format)
as the permutation group {\it stabilizerSubgroup}.
\medbreak
At present, the set stabilizer program sometimes
run slowly in doubly transitive groups, and often runs very slowly in groups
that are triply transitive or ``almost'' triply transitive (e.g.,
${\rm SL}_n(2)$\kern2pt), especially when both the point set and its complement
are large and when the set stabilizer turns out to be small.  Imprimitive
groups closely related to doubly transitive groups may also cause
difficulty.  Modifications to alleviate this difficulty, at least in part,
will be added eventually.
%
%
\subsection{Set images:}Given a permutation group $G$ on $\{1,\ldots,n\}$
and subsets $\Lambda$ and $\Phi$ of $\{1,\ldots,n\}$, the {\tt setimage}
command may be used to determine if there exists an element $g$ of $G$ such
that $\Lambda^g = \Phi$.  The format is
\smallskip
\hskip0.45truein{\tt setimage}\quad \options\quad {\it permGroup\quad pointSet1\quad
                pointSet2\quad groupElement}
\smallskip
where {\it permGroup}, {\it pointSet1}, {\it pointSet2}, and {\it groupElement}
play the role of $G$, $\Lambda$, $\Phi$, and $g$, respectively.  That is,
the command determines whether there exists an element of {\it permGroup\/}
mapping {\it pointSet1\/} to {\it pointSet2\/} and, if so, saves one such
element as the permutation {\it groupElement}.
Note that {\it groupElement\/} will not be created if $\Phi\notin\Lambda^G$.
(Unless the {\tt -q} option is specified, a message indicating whether
$\Phi\in\Lambda^G$ will be written to the standard output.)
The potential difficulties with doubly and triply transitive groups mentioned
for set stabilizer computations apply here also.
%
%
\subsection{Ordered partition stabilizers:}Stabilizers of ordered partitions
may be computed by the {\tt parstab} command.  The format is
\smallskip
\hskip0.45truein{\tt parstab}\quad \options\quad {\it permGroup\quad ordPartition\quad
                stabilizerSubgroup}
\smallskip
This command computes the stabilizer in the permutation group {\it permGroup\/}
of the ordered partition {\it ordPartition\/} and saves the result as the
permutation group {\it stabilizerSubgroup}.  The remarks about performance
on doubly and triply transitive groups for set stabilizer computations
apply here also.
%
%
\subsection{Ordered partition images:}Given a permutation group $G$ on
$\{1,\ldots,n\}$ and ordered partitions $\Pi$ and $\Sigma$ of $\{1,\ldots,n\}$,
the {\tt parimage} command may be used to determine if there exists an
element $g$ of $G$ such that $\Pi^g = \Sigma$.  The format is
\smallskip
\hskip0.45truein{\tt parimage}\quad \options\quad {\it permGroup\quad ordPartition1\quad
                ordPartition2\quad groupElement}
\smallskip
where {\it permGroup}, {\it ordPartition1}, {\it ordPartition2}, and {\it groupElement}
play the role of $G$, $\Pi$, $\Sigma$, and $g$, respectively.
That is,
the command determines whether there exists an element of {\it permGroup\/}
mapping {\it ordPartition1\/} to {\it ordPartition2\/} and, if so, saves one
such element as the permutation {\it groupElement}.
The permutation {\it groupElement\/} is created only if $\Sigma\in\Pi^G$.
The remarks about performance on doubly and triply
transitive groups given above for
set stabilizer computations apply here also.
%
%
\subsection{Group intersections:}Given permutation groups $G$ and $H$ on
$\{1,\ldots,n\}$, the {\tt inter} command may be used compute the
intersection $G\cap H$.  The format is
\smallskip
\hskip0.45truein{\tt inter}\quad \options\quad {\it permGroup1\quad permGroup2\quad
                interGroup}
\smallskip
This command computes the intersection of groups {\it permGroup1\/} and {\it permGroup2\/}
and saves the result as the group {\it interGroup\/}.  The potential
difficulty with doubly and triply transitive groups discussed above for set
stabilizer computations applies here also when both groups are doubly
or triply transitive.
%
%
\subsection{Centralizers of elements:}Given a permutation group $G$ and
a permutation $x$ (not necessarily contained in $G$), the
{\tt cent} command may be used compute $C_G(x)$, the centralizer in $G$ of $x$.
The format is
\smallskip
\hskip0.45truein{\tt cent}\quad \options\quad {\it permGroup\quad permutation\quad
                centralizerSubgroup}
\smallskip
Here {\it permGroup}, {\it permutation}, and {\it centralizerSubgroup\/}
play the role of $G$, $x$, and $C_G(x)$ above.  That is, the command computes
the centralizer in the group {\it permGroup\/} of
the permutation {\it permutation\/}
and saves the result as the group {\it centralizerSubgroup}.
\medbreak
For this command, it is permissible to specify
{\it permGroup\/} as {\tt \#}{\it n}, where {\it n\/}
is an integer at least 2, in which case {\it permGroup\/} is taken as the symmetric
group of degree {\it n}; in this situation, the normal restrictions on
base size (discussed later) do not apply to {\it permGroup}, although they do apply to
{\it centralizerSubgroup}.
\medbreak
The {\tt cent} command accepts an option {\tt -np} which can have an effect
(often small) on performance.  If this option is specified, a refinement
process based on the cycle structure of $x$ will not be used.  The effect is
to reduce memory requirements a bit.  In many cases, the running time does
not change significantly, but in some cases it does increase a great deal.
\medbreak
It should be noted that in many cases, perhaps most
cases arising in practice, centralizer computations are fairly easy
even for conventional algorithms, and the partition backtrack
program may perform no better than, and perhaps not even as well as,
programs based on conventional techniques, such as those in
Cayley.  (Note, however, that, unlike Cayley, the program here does not require
that the permutation to be centralized lie in the group.)
%
%
\subsection{Conjugacy of elements:}Given a permutation group $G$ and
permutations $x$ and $y$ (not necessarily contained in $G$), the
{\tt conj} command may be used to determine if $x$ and $y$ are conjugate
under $G$ and, if so, to find $g$ in $G$ with $x^g = y$.  The format is
\smallskip
\hskip0.45truein{\tt conj}\quad \options\quad {\it permGroup\quad permutation1\quad
                permutation2\quad conjugatingElement}
\smallskip
Here {\it permGroup}, {\it permutation1}, {\it permutation2}, and
{\it conjugatingElement\/} play the role of $G$, $x$, $y$, and $g$ above.
That is, the command determines if there exists an element of
{\it permGroup\/} conjugating
{\it permutation1\/} to {\it permutation2\/} and, if so, it saves one such
element as the permutation {\it conjugatingElement}.  If the two permutations
are not conjugate in {\it permGroup}, then {\it conjugatingElement} is not
created.  In any case, a message indicating the result is written to the
standard output (unless the {\tt -q} option is specified).
\medbreak
As with the {\tt cent} command, {\it permGroup\/} may be specified
as {\tt \#}{\it n}, in which case conjugacy in the symmetric group of
degree {\it n} is checked.  (In this case, the program merely checks that
the two permutations have the same cycle structure.)  Also, the {\tt -np}
option is accepted, and it works as described above for the {\tt cent}
command.
\medbreak
As with centralizer computations, conjugacy calculations are usually
easy with conventional algorithms, and the partition backtrack method
may not yield an improvement.
%
%
\subsection{Centralizers of groups:}Given a permutation groups $G$ and
a second permutation group $E$ (not necessarily contained in $G$), the
{\tt gcent} command may be used compute $C_G(E)$, the centralizer in
$G$ of $E$.  The format is
\smallskip
\hskip0.45truein{\tt gcent}\quad \options\quad {\it permGroup1\quad permGroup2\quad
                centralizerSubgroup}
\smallskip
Here {\it permGroup1}, {\it permGroup2}, and {\it centralizerSubgroup\/}
play the role of $G$, $E$, and $C_G(E)$ above.  That is, the command computes
the centralizer in the group {\it permGroup1\/} of
the group {\it permGroup2\/}
and saves the result as the group {\it centralizerSubgroup}.
\medbreak
As with the element centralizer command ({\tt cent}),
it is permissible to specify {\it permGroup1\/} as {\tt \#}{\it n},
indicating the symmetric group of degree {\it n}.
\medbreak
To an even greater extent than element centralizer
calculations, group centralizer calculations tend to
be easy ones for conventional algorithms; the full power of the
partition method is not needed, and perhaps not even desirable.  For this
reason, little effort has gone into development of the {\tt gcent} command;
its implementation is fairly crude, and it is included primarily
for completeness.  There are two options, {\tt -cg:}{\it m} and
{\tt -cp:}{\it p}, which affect its performance; for some groups $G$,
it may be necessary to assign them values different from the defaults
(current 3 and 10, respectively).  A full description of the significance
of {\it m\/} and {\it p\/} will not be given here; however, we note that
higher values (especially for {\it m}) increase
memory requirements, and often increase execution time as well, but may be
needed if the group $E$ fails to have a small generating set (e.g., if E is
a large elementary abelian group).
\medbreak
By specifying {\it permGroup1\/} and {\it permGroup2\/} as the same group,
the {\tt gcent} command may be used to compute the center of a group; note,
however, that it represents an exceptionally inefficient algorithm for
this purpose.
%
%
%
\subsection{Automorphism groups of designs:}The
{\tt desauto} command may be used to compute the
automorphism group of a design.  Here a {\it design\/} means any set
of points (numbered $1,\ldots,n$ for some $n$) and any collection of subsets
of the point set. The format of the design automorphism group command is:
\smallskip
\hskip0.45truein{\tt desauto}\quad \options\quad {\it design\quad autoGroup}
\smallskip
and the command sets {\it autoGroup} to the automorphism group of the
design {\it design\/}.
\medbreak
The interpretation of the group {\it autoGroup\/} that is created depends
on whether the {\tt -pb} (points and blocks) option is specified. 
Let $p$ and $b$ denote the number of points and blocks, respectively, of
the design.
\smallskip
{\advance\leftskip by0.70truein
\noindent\llap{i)\enspace}If the option {\tt -pb} is specified, then {\it autoGroup} is constructed as
a group of degree $p+b$, in which the action on $1,\ldots,p$ is the action
on points and in which the action on $p+1,\ldots,p+b$ is the
action on blocks, the $j$\kern1ptth block being
represented by $p+j$.
\smallskip\vskip2pt
\noindent\llap{ii)\enspace}If the {\tt -pb} option is omitted, then
{\it autoGroup} is constructed as a group of degree $p$, representing the
action on points only.
In this case, if there are repeated blocks, the group acting on
points only has lower
order than the group acting on points and blocks).
When this situation arises, the group saved as {\it autoGroup} represents 
the group on points
only, but the information written to the standard output during the computation
refers to the group acting on points and blocks.  (This occurs because the
computation is carried out on points and blocks; restriction to points
is performed only at the end; note also, for this reason, restriction to
points only does not save time or memory.)\vskip0pt}
\medbreak
%
%
%
\subsection{Isomorphism of designs:}The
{\tt desiso} command may be used to check
isomorphism of designs. The format is
\smallskip
\hskip0.45truein{\tt desiso}\quad \options\quad {\it design1\quad
                 design2\quad isoPerm}
\smallskip
and the command sets {\it isoPerm} to an isomorphism from
design {\it design1\/} to design {\it design2}, provided the designs
are isomorphic.  (If not, the permutation {\it isoPerm\/} is
not created.  In any case, a message indicating the result is written to
the standard output, unless the {\tt -q} option is specified.)
\medbreak
As in the case of the {\tt desauto} command, described
above, the presence or absence of the {\tt -pb} option determines whether
{\it isoPerm\/} is constructed as a permutation on points and blocks, 
or on points only (the default).  When the action on blocks is included,
the $j$\kern1ptth block is represented by $p+j$.
%
%
%
%
\subsection{Automorphism groups and monomial groups of matrices:}The
{\tt matauto} command may be used to compute the
automorphism group of a matrix.  If the matrix elements are taken from a
small finite field ${\rm GF}(q)$, then optionally the monomial automorphism group may
be computed.  (See Section~III for definitions.)
The command format is:
\smallskip
\hskip0.45truein{\tt matauto}\quad \options\quad {\it matrix\quad autoGroup}
\smallskip
and the command sets {\it autoGroup} to the automorphism group of the
matrix {\it matrix\/} or, if the {\tt -mm} option is specified, to the
monomial automorphism group of {\it matrix\/}.
\medbreak
If the {\tt -tr} option is specified, the matrix is transposed after it is
read in, and all computations apply to the transposed matrix.  
\medbreak
Let $r$ and $c$ denote the number of rows and columns, respectively, of
the matrix $A = (a_{ij})$ whose group is to be constructed.  Normally
the automorphism group has degree $r+c$ and the monomial automorphism group 
has degree $(q-1)(r+c)$; the interpretation of these groups is described
in Section~III.  However, if the {\tt -ro} (rows only) option is specified,
the degree will be $r$ or $(q-1)r$, and the group will represent the action
on rows only.  Note that restriction to rows only may reduce the order of
the group, just as in the case of designs restriction to points only may
reduce the order of the group.  When this occurs, the remarks above for
design groups apply here also. 
\medbreak
At present, the program for computing monomial groups of matrices is a very
crude one.  As a result, although it works reasonably for many matrices of
fairly large size, it can fail to run in acceptable time even for very small
matrices, e.g., matrices of all 0s.  Sometimes use of the {\tt -tr} option
can get around this difficulty (which will be fixed eventually).
%
%
%
\subsection{Isomorphism and monomial isomorphism of matrices:}The
{\tt matiso} command may be used to check
if two matrices are isomorphic or, if the matrix elements are from a 
finite field ${\rm GF}(q)$, monomially isomorphic.  (See Section~III for
definitions.)  The command format is
\smallskip
\hskip0.45truein{\tt matiso}\quad \options\quad {\it matrix1\quad
                 matrix2\quad isoPerm}
\smallskip
In the absence of the {\tt -mm} option, the command sets {\it isoPerm} to an 
isomorphism from matrix {\it matrix1\/} to matrix {\it matrix2}, provided the matrices
are isomorphic.  (If not, the permutation {\it isoPerm\/} is
not created).  If the {\tt -mm} option is specified, the command sets 
{\it isoPerm} to a monomial isomorphism from matrix {\it matrix1\/} to matrix {\it matrix2}, 
provided the matrices are monomially isomorphic.  (In this case, the matrix
entries should be field elements.)  The effect of the {\tt -ro} option is as
described above for matrix automorphism group calculations.
\medbreak
Currently the monomial isomorphism program suffers from the same limitations
as the monomial automorphism group program, as mentioned above.
\medbreak
%
%
\subsection{Automorphism groups of linear codes:}The
{\tt codeauto} command may be used to compute the
automorphism group of a linear code over a small field ${\rm GF}(q)$.  
However, before the automorphism group of a code $C$ may be computed, 
it is necessary to have a set $V$ of vectors (not necessarily codewords) 
such that the following conditions hold.  In these conditions, $V^*$
denotes the set of all nonzero scalar multiples of vectors in $V$.
\smallbreak
{\advance\leftskip by1em\parindent=1em\relax
\item{i)} No vector in $V$ is a scalar multiple of any other vector in $V$.
(In particular, $\vert V^*\vert = (q-1)\vert V\vert$.)
\smallbreak
\item{ii)} $V$ is ``reasonably small''.  (With a very large memory,
           ``reasonably small'' might mean 100,000 or more.)
\smallbreak
\item{iii)} $V^*$ is invariant under ${\rm AUT}(C)$\enskip (the automorphism
          group of $C$),
\smallbreak
\item{iv)} $\bigl\vert{\rm AUT}(V^*):{\rm AUT}(C)\bigr\vert$ is very small.
            (The running time rises very rapidly as a function of this
            index.  Note that, if $V$ spans $C$, the index is 1.)
\smallbreak}
Often the set of minimal weight vectors of the code (scalar multiples 
removed if $q > 2$) make a suitable choice for $V$; minimum weight vectors
of the dual code may also be used.  This choice for $V$ certainly 
satisfies~(i) and~(iii), 
may well satisfy~(ii), and in many cases satisfies~(iv).  The author has available
programs for computing the set of minimum weight vectors (or vectors of
any specified weight.)
\medbreak
The format of the code automorphism group command is
\smallskip
\hskip0.45truein{\tt codeauto}\quad \options\quad {\it code\quad
                 invarVectors\quad autoGroup}
\smallskip
where {\it invarVectors\/} is the set $V$ of vectors described above
(in the format of a matrix, whose rows are the vectors).  The command
sets {\it autoGroup} to the automorphism group of the
code {\it code}.
\medbreak
The {\tt -cv} (coordinates and vectors) option for codes 
has essentially the same effect as the
{\tt -pb} option for designs.  With this option, the automorphism
group is saved in {\it autoGroup\/} as a permutation group
of degree $\bigl(q-1)(n+\vert V\vert\bigr)$\enskip ($n =$ length of code),
representing the action on (monomial) coordinates 
and invariant vectors; without the {\tt -cv} option, it is saved as a permutation group 
acting of degree $(q-1)n$, representing the action on (monomial)
coordinates only.   (However, restriction to coordinates only can never lead to a reduction
in the group order, as occurred with restriction to points or rows for
designs or matrices.)  For an explanation of the format of monomial
permutations, see Section~III.
\medbreak
At present, the program for computing groups of non--binary codes is a very
crude one; sometimes it can fail to run in reasonable time even on small
codes.  Eventually this program will be improved.
\medbreak
%
%
%
\subsection{Isomorphism of linear codes:}The
{\tt codeiso} command may be used to check isomorphism
of linear codes.  However, before isomorphism of two codes
$C_1$ and $C_2$ may be checked, it is necessary to have a sets $V_1$
and $V_2$ of vectors (not necessarily codewords of the two codes)
such that $V_1$ and $V_2$ satisfy conditions~(i), (ii), (iii), and~(iv) above
relative to $C_1$ and $C_2$, respectively, and in addition such that
any isomorphism of $C_1$ to $C_2$ must map $V_1^*$ to $V_2^*$.  (As with
code automorphism groups, $V_1^*$ and $V_2^*$ denote the sets of nonzero scalar
multiples of vectors in $V_1$ and $V_2$, respectively.
Often suitable choices for $V_1$ and $V_2$ are the minimal weight vectors
of $C_1$ and $C_2$, respectively (scalar multiples removed.); minimal weight vectors of the duals
of the two codes also could be used.
\medbreak
The format of the code isomorphism command is
\smallskip
\hskip0.45truein{\tt codeiso}\quad \options\quad {\it code1\quad
                 code2\quad invarVectors1\quad invarVectors2\quad
                 isoPerm}
\smallskip
where {\it invarVectors1\/} and {\it invarVectors2\/}
are the sets $V_1$ and $V_2$, respectively, of vectors described above
(each in the format of a matrix, whose rows are the vectors).  The command
sets {\it isoPerm} to an isomorphism from {\it code1\/} to {\it code2},
if the codes are isomorphic; if not, {\it isoPerm\/} is not created.
\medbreak
As in the case of the {\tt codeauto} command, described
above, the presence or absence of the {\tt -cv} option determines whether
{\it isoPerm\/} is a permutation on (monomial) coordinates and invariant vectors,
or on (monomial) coordinates only.  The interpretation of monomial permutations
is described in Section~III..
%
%
\vskip10pt
\bigbigbreak
Note that a number of the commands above are implemented as shell 
files (under Unix), batch files (under MS~DOS), or exec files
(under CMS).  The commands that are implemented in this manner, and the
contents of the Unix shell files, are as follows.  (The list includes a few 
commands to be discussed in Section~IX.)
\bigskip
\long\def\shellfile#1#2{\noindent\hskip0.45truein\hbox to1.0truein{#1\hfil}\relax
                   \hbox to4.0truein{#2\hfil}\vskip2.0pt}\relax
\vbox{{\it
\shellfile{command}{contents of shell file}
\vskip6pt
\elevenpoint\tt
\shellfile{setimage}{setstab -image \$*}
\shellfile{parstab} {setstab -partn \$*}
\shellfile{parimage}{setstab -image -partn \$*}
\shellfile{conj}    {cent -conj \$*}
\shellfile{gcent}   {cent -group \$*}
\shellfile{desiso}  {desauto -iso \$*}
\shellfile{matauto} {desauto -matrix \$*}
\shellfile{matiso}  {desauto -iso -matrix \$*}
\shellfile{codeauto}{desauto -code \$*}
\shellfile{codeiso} {desauto -iso -code \$*}
\shellfile{cjper}   {cjrndper -perm \$*}
\shellfile{ncl}     {commut -ncl \$*}
\shellfile{compper} {compgrp -perm:\$1 \$2 \$3}
\shellfile{compset} {compgrp -set:\$1 \$2 \$3}
\shellfile{chbase}  {orblist -chbase \$*}
\shellfile{ptstab}  {orblist -ptstab \$*}
\vskip0pt}}
%
%
%
\section{V.\quad OPTIONS}
%
A partial description of the options that are currently available
follows.  Most of the options are available with all of
the commands described in Section~IV.  A few options apply only to
subgroup computations, or only to coset--representative computations;
these restrictions are noted below.  Options applicable only to a single command
are discussed with that command in Section~IV.
\medbreak
In general, options may be specified in any order.  However, if
conflicting options are specified, the one specified last is
the one that is used.  (In some cases, conflicting options are treated
as an error.  Also, the {\tt -l} and {\tt -v} options, discussed later,
are an exception to the general rule that options may be specified in
any order; these options, if present, must come first, and the remainder
of the command line is ignored.)
\medbreak
Entering any command with no options or arguments causes a brief
summary of the command format to be displayed.
\bigbreak
\subsection{Options affecting file handling:}
\nobreak\medskip\nobreak
%
\defoption{-a}{Normally, if a file name is specified for an object
                 to be constructed, and if a file by that name already
                 exists, the programs overwrite the existing file.  With
                 the {\tt -a} option, they append to the existing file,
                 rather than overwriting it.}
\medbreak
\defoption{-p:{\it path}}{Here {\it path\/} is a string.
                 The string {\it path\/} is concatenated to
                 the file name of every input file.  This option can be useful
                 if all the input files are in another directory.
                 For example,
                 \smallskip
                 \hskip0.15truein{\tt setstab\quad\kern-1pt  ../groups/psp62::psp62\quad\kern-1pt
                 ../groups/lambda::lambda\quad\kern-2pt S}
                 \smallskip
                 may be written more compactly as
                 \smallskip
                 \hskip0.15truein{\tt setstab\quad  -p:../groups/\quad
                 psp62\quad lambda\quad S}
                 \smallskip
                 (Note the final slash following {\tt groups} is required.).
                 The {\tt -p} option has no effect on output files.}
%
\bigbreak
\subsection{Options affecting output format:}
\nobreak\medskip\nobreak
%
\medbreak
\defoption{-i}{This option applies to commands that construct and write out
               either a permutation or a permutation group.  It causes
               permutations to be written in image format, rather
               than in cycle format (the default).}
%
\medbreak
\defoption{-n:{\it name}}{Here {\it name\/} is a string.  The object created
                         by the command will be named {\it name}.  By
                         default, the name assigned to the object will be
                         the name of the Cayley library containing its
                         definition.  Note this option affects only the
                         name of the object, not that of the file or the
                         Cayley library.}
%
\medbreak
\defoption{-q}{Suppresses informative messages on the state of the
               computation, normally written to the standard output during
               the computation.}
%
\medbreak
\defoption{-s}{Causes statistics on the pruning of the backtrack
               search tree to be written out to the standard output.
               These statistics relate to the backtrack search tree
               defined in the author's paper (Leon, 1991), and are
               likely to be meaningful only to users familiar with that
               paper.}
%
\medbreak
\defoption{-w:{\it n}}{Here {\it n\/} should be a nonnegative integer.
                       This option applies only to coset representative
                       computations.   If the degree is less
                       than or equal to {\it n}, and if a coset representative
                       is found, it will be included in the informative
                       messages written to the standard output.  (In any case,
                       the coset representative will be written to a file in
                       Cayley library format.)  The default value of {\it n}
                       is currently 300.}
\bigbreak
%
\subsection{Options affecting performance of the algorithm:}
\nobreak\medskip\nobreak
\defoption{-b:{\it k}}{Here {\it k\/} is a nonnegative integer which
                 determines the extent to which base changes are performed
                 in an attempt to
                 improve pruning of the backtrack search tree using
                 tests on double--coset minimality (Leon, 1991, Prop~8).
                 When {\it k}~=~0,
                 base change is never performed (except during $\bfgermR$--base
                 construction, when it is used for a different purpose).
                 As {\it k\/} increases, the number of base change operations
                 performed increases;  however,
                 increasing {\it k\/} beyond the base size produces no
                 further increase in the number of base change operations.
                 Designating {\it k}~=~0 reduces memory requirements and often
                 produces the best running times as well.  On the other
                 hand, some high--density groups seem to require a higher
                 value of {\it k\/} in order to obtain acceptable performance.
                 By default, the program chooses a value of {\it k\/} based
                 on the density and degree of transitivity of the group;
                 quite often, this default value is 0.
                 \smallskip
                 {\it Note:}\enskip For coset representative computations,
                 this option has no effect unless known subgroups of the two
                 associated groups are specified; see discussion of the
                 {\tt -kL} and {\tt -kR} options below.}
%
\medbreak
\defoption{-g:$m$}{Here $m$ should be a nonnegative integer.
                 This is one of several parameters providing a
                 time vs.\ space tradeoff.
                 Small values of $m$, say 10 or less, minimize memory
                 requirements, while large values of $m$, say 100 or
                 greater, reduce the running time moderately for most
                 difficult groups.  Use of a high value is recommended
                 for multiply transitive groups.
                 \smallskip
                 After $\bfgermR$--base construction, the program attempts to
                 reduce the height of the Schreier trees for the
                 containing group by adding new strong generators.  However,
                 it will never add generators for this purpose if doing
                 so would cause the total number of strong generators to
                 exceed $m$.  (It will also stop adding generators if the height
                 falls below certain goals currently fixed in the program.)}
%
\medbreak
\defoption{-k:$H$}{Here $H$ specifies a permutation group (in the format
                 {\it cayleyLibraryName\/} or
                 {\it fileName}\kern1pt::\kern1pt{\it cayleyLibraryName}).
                 This option applies only to subgroup
                 calculations.  The group $H$ must
                 be a known subgroup of the group being computed.  In
                 principle, this option allows one to take advantage
                 of any subgroup of the group being computed that happens
                 to be known in advance. In practice, however, it seldom
                 appears to speed up the computation by very much,
                 and it increases memory requirements.}
%
\medbreak
\newdimen\optionWidth \optionWidth=\hsize \advance\optionWidth by-1em\relax
                     \advance\optionWidth by-\optionIndent
{\advance\leftskip by1em\noindent\vtop{\hsize=\optionIndent\parindent=0pt\tt\noindent
                 -kL:$J$\vskip1pt\noindent -kR:$M$}\relax
                 \vtop{\hsize=\optionWidth\parindent=0pt
                 Here $J$ and $M$ must specify permutation groups (each in the
                 format  {\it cayleyLibraryName\/} or
                 {\it fileName}\kern1pt::\kern1pt{\it cayleyLibraryName}).  These options
                 apply only to coset representative calculations.  Either or
                 both may be specified.  Associated with every coset
                 representative computation, there are ``left'' and ``right''
                 groups, as explained in Section~2 of (Leon, 1991).
                 The groups $J$ and $M$ must be known subgroups of these
                 left and right groups, respectively.  Specifying $J$ and/or $M$,
                 if known, increases memory requirements, but in some cases
                 it may improve the running time.  For some very dense groups,
                 one or both of these options may be needed in order to allow
                 the computation to finish in an acceptable amount of time.}\vskip0pt}
%
\medbreak
\defoption{-mb:$k$}{Here $k$ should be a nonnegative integer.  This integer
                represents an upper bound on the size of the base for a 
                permutation group.  The default value of $k$ is 62, which is
                more than adequate for many groups.  For further discussion of
                the {\tt -mb} option, see Section~VII.}
%
\medbreak
\defoption{-mw:$\ell$}{Here $\ell$ should be a nonnegative integer whose value
                is at least several hundred.  This integer represents an upper 
                bound on the length of any word in the generators of any
                permutation group.  For further discussion, see Section~VII.}
%
\medbreak
\defoption{-r:$p$}{Here $p$ should be a nonnegative integer, normally smaller
                 than the integer $m$ specified for the {\tt -g} option
                 described above.  This is another option providing for a
                 time versus space
                 tradeoff.  Small values of $p$, say less than 10, minimize
                 memory requirements, while larger values, say 50 or higher,
                 {\it may\/} reducing the running time, although usually not
                 a great deal.
                 \smallskip
                 Whenever the number of strong generators for the containing
                 group exceeds $p$, redundant strong generators are eliminated,
                 using a procedure originally due to Sims (1971).}
\bigbreak
\subsection{Special options:}
\nobreak\medskip\nobreak
\defoption{-l}{This option, if present, must be the first option on the
               command line, and the remainder of the command line is
               ignored.  (It may be omitted.)  The {\tt -l} option merely
               prints out limits on the default maximum base size, 
               default maximum word length, degree, and other
               quantities with which this version of the program has been
               compiled.  (See Section~VII for discussion of these limits.)}
\medbreak
\defoption{-v}{This option, if present, must be the first option on the
               command line, and the remainder of the command line is
               ignored.  (It may be omitted.)  The {\tt -v} option is
               intended to be used once following compilation of the program.
               It attempts to check that all the source files for the
               program were compiled with the same options and size limits.
               (See Section~VII for discussion of size limits.)}
%
%
%
\section{VI.\quad OUTPUT AND RETURN CODES}
%
All programs for subgroup computations return a value of 0 if the
computation is completed successfully and a nonzero value (currently 15) if
the computation terminates due to an error (input file not found, incorrect
format in input file, memory exhausted, size limit in program exceeded, etc.)
All programs for coset representative
computations return a value of 0 if the computation is completed
successfully and a coset representative exists, 1 if it is
completed but a coset representative does {\it not\/} exist, and a value different from
0 and 1 (currently 15) if the computation terminates due to an error.
\medbreak
%
Unless the {\tt -q} option is specified, all of the programs write
information about the progress of the computation to the standard output.
Some of this information, most notably that relating to the
$\bfgermR$--base and the backtrack search tree (the latter
given only if {\tt -s} is specified)
will probably be meaningful primarily to users familiar with the
author's paper (Leon, 1991).  Information of more general interest
includes:
\smallskip
{\advance\leftskip by0.45truein\noindent
\item{i)}The order of the containing group (unless it is the symmetric 
group).  Note that this order is determined by computing a base and strong
generating set for the containing group when it is read in, unless they are 
supplied in the input file. 
\smallskip
\item{ii)}The new (changed) base and strong generating set for the 
containing group computed during $\bfgermR$--base construction, and the corresponding
basic orbit lengths.  In the notation of (Leon, 1991), this is the
base $(\alpha_1,\ldots,\alpha_k)$ associated with the $\bfgermR$--base.
\smallskip
\item{iii)}A base for the subgroup to be computed (subgroup computations)
or for the subgroup associated with the right coset whose representative
is to be computed (coset representative computations).  
This is the subgroup base associated
with the $\bfgermR$--base; in the notation of (Leon, 1991), it is
$(\hat{\alpha}_1,\ldots,\hat{\alpha}_{\ell})$.  Note that this base is a
subsequence of the base for the containing group in~(ii) above.
\smallskip
\item{iv)}The basic cell sizes corresponding to the subgroup base in~(iii)
above (for definitions, see (Leon, 1991)).  Note that each basic
cell size provides an upper bound for the corresponding basic orbit length of the
subgroup to be computed (subgroup--type computations).
(Usually the bound is not sharp).
\smallskip
\item{v)}The number of strong generators for the containing group and the
mean node depths in the Schreier trees for the basic orbits of the containing
group.  Depending on the {\tt -g} and {\tt -r} options, following $\bfgermR$--base
construction, additional strong generators may be added in an attempt to
reduce the height of the Schreier trees.  Figures are provided both before
and after additional strong generators are added.
\smallskip
\item{vi)}[subgroup computations only]\enskip A message for each strong 
generator that is found for the subgroup.  The message gives the level and the basic
orbit lengths for the subgroup constructed thus far.  (A generator will 
be said to be at level $i$ if it fixes the first $i-1$ base points but
moves the $i$\kern1ptth.)
\smallskip
\item{vii)}[subgroup computations only]\enskip The order of the subgroup that was
computed.
\smallskip
\item{viii)}[subgroup computations only]\enskip The base (same as in~(iii) above)
and basic orbit lengths for the subgroup that was computed.
\smallskip
\item{ix)}[coset representative computations only]\enskip A message indicating whether a coset
representative exists.
\smallskip
\item{x)}[coset representative computations only]\enskip If a coset representative exists and the degree is sufficiently
low (depending on the {\tt -w} option), the coset representative that was
found.  
\smallskip
\item{xi)}The time required for the computation.  Note that the time to
read in the containing group from a file, construct the initial base and
strong generating set for the containing group (if not present in the input
file), and to write out the subgroup or coset representative to a file is
not included in this time.  All computations relating to calculation of
the subgroup or coset representative (including base changes in the containing
group) are included.
\medbreak}
%
Note that, in subgroup computations, the actual strong generators for the
subgroup are not written to standard output, and in coset computations,
the actual coset representative found may not (depending on the degree and
{\tt -w} option) be written to the standard output.  However, both may be
found (in Cayley library format) in the output file that is created.
\medbreak
For design, matrix, or code isomorphism computations, the isomorphism that
is constructed is written to the standard output (assuming that the degree
is sufficiently low) in a more easily readable
(but not Cayley compatible) format than that described in Sections~II
and~III.  For designs with the {\tt -pb} option,
the action on points and blocks is given separately.  For matrices, the
action on rows and columns is given separately.  For monomial isomorphism of
matrices for non--binary codes, the monomial 
isomorphism is written in the following format:
$$\bigl(\kern1pt[\lambda_1]i_1,[\lambda_2]i_2,\ldots,[\lambda_k]i_k\kern1pt\bigr)$$
This denotes the monomial permutation mapping 1 to $\lambda_1i_1$,
2 to $\lambda_2i_2$, etc.  For example, to apply this monomial permutation
to the rows of an $r$ by $c$ matrix, row $j$ is multiplied by $\lambda_j$
and the result is moved to row position $i_j$.

%
\medbreak
%
%
%
\section{VII.\quad SIZE LIMITS}
%
There are a few fixed limits on the sizes of objects that the programs can
handle.  These limits can be changed only by recompiling the programs.
The order of any group may have at most 30 distinct prime divisors.  The
name of any file may have at most 60 characters (including path information
supplied with the {\tt -p} option).   The name of any object may have at
most 16 characters.
Most importantly, if the program is compiled using 16--bit integers,
the maximum degree of any permutation group is limited to slightly less
than $2^{16}$ (about 65000).  If it is compiled using 32--bit integers,
there is, for practical purposes, no fixed limit.
Note, however, that use of 16--bit
integers reduces memory requirements substantially, and it is recommended
unless groups of degree greater than 65000 (approx) are to be used.
Only machines having at least 20 to 25 megabytes of memory are
likely to be able to handle groups of degree high enough to require 
32--bit integers.   Currently both 16--bit and 32--bit compiled versions 
of the programs are available.
\medbreak
Although there is no fixed limit on the base size for a permutation group,
a limit must be established at the time that the program is initiated, and
this limit remains fixed during that run.
This limit may be set at $k$ by means of the {\tt -mb:$k$} option, or it
may be allowed to default to 62.  Note that large values of $k$ increase 
running times and memory requirements slightly even if the actual base size 
turns out to be much less than $k$. 
\medbreak
For the most part, the amount of memory (real and virtual) available
determines the sizes of objects that can be handled by the programs.
Memory requirements depend heavily on the degree of the group, and 
to a somewhat lesser extent on the base size.
The programs can use virtual memory to some extent; however, if 
virtual memory used exceeds real memory by a factor of more than 1.6 to 1.8,
excessive paging is likely to occur.  The following steps may be taken
to reduce memory requirements; the steps are listed in order of decreasing
benefit.
\medskip{\advance\leftskip by0.45in\noindent
\item{i)}If the degree of the group is less than 65000 (approx), use a
         16--bit version of the program rather than a 32--bit version.
         The 16--bit version is likely to run about as fast as the 32--bit
         version, and it requires a great deal less memory.
\smallskip
\item{ii)}Specify options of {\tt -g:1} and {\tt -r:1}.  These options
          are likely to increase the execution time substantially, but
          they often save a good deal of memory.   As a compromise,
          values greater than 1 but less than the defaults may be
          specified, e.g., {\tt -g:20} and {\tt -r:15} 
\smallskip
\item{iii)}Specify the option {\tt -b:0} if it is not already the default.
           In the majority of cases, this option will not increase
           execution time, and it reduces memory requirements considerably.
           However, in a great many cases, {\tt -b:0} 
           will already be the default.   (The value for this and other
           options is displayed on the standard output when the program is
           is run.)
\smallskip
\item{iv)}For (element) centralizer and conjugacy calculations, specify the
          {\tt -np} option.  This saves a modest amount of memory.  The
          effect on execution time is hard to predict; in some cases, it
          may lead to a major increase.  For group centralizer calculations,
          options of {\tt -cg:2} and {\tt -cp:$i$}, where $i$ is 3 to 5,
          may be tried, although on some groups they may raise the 
          running time to unacceptable levels.  
\smallskip          
\item{v)}Specify the option {\tt -mb:$k$} for a value of $k$ less than
          the default of 62.  The {\tt -mb:$k$} options sets a limit of $k$
          on the base size for the group; often a value considerably less
          than 62 (e.g., 15 or 20) will be adequate.  However, the amount of 
          memory saved is relatively small.
\vskip0pt}
\medbreak
In the author's experience, the programs can often handle groups of degree
as high as 2000$m$ to 3000$m$, where $m$ is the number of megabytes of
{\it real\/} memory available.  However, for groups lacking a relatively
small base, the limit on the degree is much lower.  Also, this limit
applies only to memory requirements; depending on the type of computation
and the specific groups, it may or may not be possible to perform 
computations in groups this large in an acceptable amount of time.
%
%
\section{VIII.\quad EXAMPLES}
%
The author has prepared a number of sample objects that may be used to
test the programs.  In the Unix distribution, these objects
appear in various subdirectories of the directory {\tt partn/examples}.
%
\medbreak
The subdirectories {\tt psp62}, {\tt psp82}, {\tt psu72},
{\tt omg84}, {\tt fi23}, {\tt ahs2}, {\tt rubik4}, and 
{\tt syl128} of directory {\tt partn/examples} contain
examples for computation in the groups ${\rm PSp}_6(2)$ of degree 63,
${\rm PSp}_8(2)$ of degree 255, ${\rm PSU}_7(2)$ of degree 2709,
$\Omega^+_8(4)$ of degree 5525, ${\rm Fi}_{23}$ of degree 31671, 
${\rm AUT(HS)}\times {\rm AUT(HS)}$ of degree 200, 
the group of a $4\times 4$ Rubik's cube (degree 96), and a Sylow 2--subgroup
of the symmetric group {\it Sym\/}(128)\enskip of degree 128, respectively.
Note that, for the last two groups, any base will be large, and the 
{\tt -mb} option (e.g., {\tt -mb:75}) will need to be specified.
Each of the directories contains files as follows, where {\it grp\/} is
to be replaced by the actual name of the directory.
\medbreak
{\advance\leftskip by 1.0truein\relax
   \noindent\llap{\hbox to0.75in{{\it grp}\hfil}}The permutation group mentioned above.
   \medbreak
   \noindent\llap{\hbox to0.75in{{\it grp\/}{\tt\kern1ptx}\hfil}}A group permutation isomorphic to the
          group {\it grp\/} and having a small but nontrivial intersection with
          {\it grp}.  The intersection of {\it grp\/} and {\it grp\/}{\tt x} may
           be computed by the command
   \smallskip
   \hskip0.4in{\tt inter\quad {\it grp\/}\quad {\it grp\/}x\quad int}
   \smallskip
          which saves the intersection as the group {\tt int}.  The file
          {\it grp\/}{\tt x} contains a comment giving the order of the
          intersection.
   \medbreak
   \noindent\llap{\hbox to0.75in{\tt set1\hfil}}A random point set of size half the degree
                of {\it grp}.  Except in the case of {\tt rubik4}
                and {\tt syl128}, the set stabilizer of
                 {\tt set1} in the group {\it grp\/} turns out to be trivial.  This
                 set stabilizer may be computed by the command
   \smallskip
   \hskip0.4in{\tt setstab\quad {\it grp\/}\quad set1\quad stab1}
   \smallskip
          which saves the set stabilizer as the group {\tt stab1}.
   \medbreak
   \noindent\llap{\hbox to0.75in{\tt set2\hfil}}A point set of size approximately
                 half the degree whose set stabilizer in {\it grp\/} 
                 is a dihedral group of low order, except in the case 
                 of {\tt rubik4} and
                 {\tt syl128}.  This set stabilizer may be computed by the command
   \smallskip
   \hskip0.4in{\tt setstab\quad {\it grp\/}\quad set2\quad stab2}
   \smallskip
          which saves the set stabilizer as the group {\tt stab2}.
          The file {\tt set2} contains a comment indicating the order of the
          stabilizer.
   \medbreak
   \noindent\llap{\hbox to0.75in{\tt set3\hfil}}A point set of size roughly
                 half the degree (in most cases ) whose set stabilizer in {\it grp\/} is a
                 group of high order.  This set stabilizer may be
                 computed by the command
   \smallskip
   \hskip0.4in{\tt setstab\quad {\it grp\/}\quad set3\quad stab3}
   \smallskip
          which saves the set stabilizer as the group {\tt stab3}.
          The file {\tt set3} contains a comment indicating the order of the
          stabilizer.
   \medbreak
   \noindent\llap{\hbox to0.75in{\tt set1x\hfil}}A point set obtained by applying a
                 randomly--chosen element of the group {\it grp\/} to {\tt set1}.
                 The command
   \smallskip
   \hskip0.4in{\tt setimage\quad {\it grp\/}\quad set1\quad set1x\quad g}
   \smallskip
          may be used to find an element {\tt g} of the group {\it grp\/} mapping
          {\tt set1} to {\tt set1x}.
   \medbreak
   \noindent\llap{\hbox to0.75in{\tt set1y\hfil}}A point set having the same cardinality
                 as {\tt set1} but not equal to the image of {\tt set1}
                 under any element of {\it grp\/}.  The command
   \smallskip
   \hskip0.4in{\tt setimage\quad {\it grp\/}\quad set1\quad set1y\quad h}
   \smallskip
          may be used to determine that {\tt set1y} is not in fact an
          image of {\tt set1} under the group.  (The permutation $h$ will
          {\it not\/} be created.
   \medbreak
   \noindent\llap{\hbox to0.75in{\tt par1\hfil}}A partition of the set $\{1,\ldots,n\}$, where
                 $n$ is the degree of group {\it grp}.  (The file contains
                 a comment indicating the number of cells and cell sizes.)
                 The stabilizer in {\it grp\/} of
                 {\tt par1}, treated as an ordered partition, may be computed by the
                 command
   \smallskip
   \hskip0.4in{\tt parstab\quad {\it grp\/}\quad par1\quad pstab1}
   \smallskip
          which saves the ordered partition stabilizer as the group {\tt pstab1}.
          The file {\tt par1} contains a comment giving the order of the
          stabilizer.
   \medbreak
   \noindent\llap{\hbox to0.75in{\tt par1x\hfil}}A partition obtained by applying a
                 randomly--chosen element of the group {\it grp\/} to {\tt par1}.
                 The command
   \smallskip
   \hskip0.4in{\tt parimage\quad {\it grp\/}\quad par1\quad par1x\quad i}
   \smallskip
          may be used to find an element {\tt i} of the group {\it grp\/} mapping
          {\tt par1} to {\tt par1x}.
   \medbreak
   \noindent\llap{\hbox to0.75in{\tt par1y\hfil}}A partition in which the sizes of the
                 cells match those in {\tt par1}, but which is not the
                 image of {\tt par1} under any element of the group {\it grp\/}.
                 The command
   \smallskip
   \hskip0.4in{\tt parimage\quad {\it grp\/}\quad par1\quad par1y\quad j}
   \smallskip
          may be used to demonstrate that {\tt par1y} is not an image of
          {\tt par1} under any group element.
   \medbreak
   \noindent\llap{\hbox to0.75in{\tt elt1\hfil}}An element of the group {\it grp\/}
                 having a fairly
                 large centralizer in {\it grp\/}.  This centralizer may be
                 computed by the command
   \smallskip
   \hskip0.4in{\tt cent\quad {\it grp\/}\quad elt1\quad cent1}
   \smallskip
          which saves the centralizer as the group {\tt cent1}.
          The file {\tt elt1} contains a comment stating the order of the
          centralizer.
   \medbreak
   \noindent\llap{\hbox to0.75in{\tt elt1x\hfil}}An element conjugate
                 under the group {\it grp\/}
                 to {\tt elt1}.  An element of {\it grp\/} conjugating {\tt elt1}
                 to {\tt elt1x} may be found by the command
   \smallskip
   \hskip0.4in{\tt conj\quad {\it grp\/}\quad elt1\quad elt1x\quad c1}
   \smallskip
          which sets {\tt c1} to such a conjugating element.
   \medbreak
   \noindent\llap{\hbox to0.75in{\tt elt2\hfil}}An permutation {\it not\/}
                 in the group {\it grp\/} having a nontrivial
                 centralizer in {\it grp}.  This centralizer may be
                 computed by the command
   \smallskip
   \hskip0.4in{\tt cent\quad {\it grp\/}\quad elt2\quad cent2}
   \smallskip
          which saves the centralizer as the group {\tt cent2}.
          The file {\tt cent2} contains a comment indicating the order of the
          centralizer.
   \medbreak
   \noindent\llap{\hbox to0.75in{\tt elt2x\hfil}}A permutation (not in {\it grp\/})
                 conjugate under {\it grp\/}
                 to {\tt elt2}.  An element of {\it grp\/} conjugating {\tt elt2}
                 to {\tt elt2x} may be found by the command
   \smallskip
   \hskip0.4in{\tt conj\quad {\it grp\/}\quad elt2\quad elt2x\quad c2}
   \smallskip
          which sets {\tt c2} to such an element.
   \medbreak
   \noindent\llap{\hbox to0.75in{\tt elt2y\hfil}}A permutation not in {\it grp\/}
                 having the same cycle structure as {\tt elt2} but not
                 conjugate under {\it grp\/} to {\tt elt2}.  Non--conjugacy may
                 be demonstrated by the command
   \smallskip
   \hskip0.4in{\tt conj\quad {\it grp\/}\quad elt2\quad elt2y\quad d}
   \smallskip
          which does {\it not} create a permutation {\tt d}.
\vskip0pt}
\medbreak
Note that, in the case of the group {\tt fi23}, about 16 megabytes of real
memory may be needed to perform the calculations above.
\bigbreak
The subdirectories {\tt q17} and {\tt q32} contain designs, (0,1)--matrices,
and codes based on the quadratic residue code $Q_{17}$ of length 17 and
on the extended quadratic residue code $Q_{32}$ of length 32, respectively.
The contents of these directories are as follows, where $i$ denotes either
17 or 32.
\medbreak
{\advance\leftskip by 1.0truein\relax
   \noindent\llap{\hbox to0.75in{\tt q\kern1pt$i$\hfil}}The quadratic or extended
                      quadratic residue code.
   \medbreak
   \noindent\llap{\hbox to0.75in{\tt v\kern1pt$i$\hfil}}The matrix whose rows are
                   the weight 5 ($i = 17$) or weight 8 ($i = 32$) codewords
                   of the code {\tt q}\kern1pt$i$.  The automorphism group of the code
                   {\tt q}\kern1pt$i$ may be computed by the command
   \smallskip
   \hskip0.4in{\tt codeauto\quad q\kern1pt$i$\quad v\kern1pt$i$\quad A}
   \smallskip
   or
   \smallskip
   \hskip0.4in{\tt codeauto\quad -cv\quad q\kern1pt$i$\quad v\kern1pt$i$\quad A}
   \smallskip
          which saves the automorphism group as the group {\tt A},
          either as a group of degree $i$ or as a group of degree
          $i+k$, where $k$ is the number of codewords in the set {\tt v}\kern1pt$i$.
   \medbreak
   \noindent\llap{\hbox to0.75in{\tt q\kern1pt$i$\kern1ptx\hfil}}Another quadratic residue
                     code obtained from
                     {\tt q}\kern1pt$i$ by applying a random permutation to the
                      coordinates..
   \medbreak
   \noindent\llap{\hbox to0.75in{\tt v\kern1pt$i$\kern1ptx\hfil}}The matrix whose rows are
                   the weight 5 ($i = 17$) or weight 8 ($i = 32$) codewords
                   of the code {\tt q}\kern1pt$i$\kern1pt{\tt x}.  An isomorphism from
                   {\tt q}\kern1pt$i$ to {\tt q\kern1pt$i$\kern1ptx}
                   may be found by the command
   \smallskip
   \hskip0.4in{\tt codeiso\quad q\kern1pt$i$\quad q\kern1pt$i$\kern1ptx\quad
                v\kern1pt$i$\quad v\kern1pt$i$\kern1ptx\quad s}
   \smallskip
          which saves the isomorphism found as the permutation {\tt s}.
   \medbreak
   \noindent\llap{\hbox to0.75in{\tt d\kern1pt$i$\hfil}}The design on $\{1,\ldots,i\}$
                     whose blocks correspond to the codewords of
                     weight 5 ($i = 17$) or weight 8 ($i = 32$) in {\tt q}\kern1pt$i$.
                     The automorphism group of this design (which must
                     contain the group of the corresponding code, and which
                     in fact equals it) may be computed by the command
   \smallskip
   \hskip0.4in{\tt desauto\quad d\kern1pt$i$\quad B}
   \smallskip
   or
   \smallskip
   \hskip0.4in{\tt desauto\quad -pb\quad d\kern1pt$i$\quad B}
   \smallskip
          which saves the automorphism group as the group {\tt B}, either as a
          group on points only, or as a group on points and blocks.  Note
          that the incidence matrix of the design {\tt d}\kern1pt$i$ is the
          transpose of the matrix {\tt v}\kern1pt$i$, so the same automorphism
          group could be computed by the command
   \smallskip
   \hskip0.4in{\tt matauto\quad -tr\quad v\kern1pt$i$\quad A}
   \medbreak
   \noindent\llap{\hbox to0.75in{\tt d\kern1pt$i$\kern1ptx\hfil}}A design obtained from
                     {\tt d}\kern1pt$i$ by applying a random permutation to the
                     points.  (The order of the blocks is also permuted
                     randomly.)  An isomorphism from {\tt d}\kern1pt$i$ to
                     {\tt d\kern1pt$i$\kern1ptx} may be found by the command
   \smallskip
   \hskip0.4in{\tt desiso\quad d\kern1pt$i$\quad d\kern1pt$i$x t}
   \smallskip
          which sets {\tt t} to one such isomorphism.
\vskip0pt}
\bigbreak
The subdirectory {\tt dmcl} contains a design based on
the sporadic simple group of McLaughlin (McL, degree 275).  In this group,
a point stabilizer has orbits of length 1, 112, and 162.  The design
{\tt dmcl} on $\{1,\ldots,275\}$ has 275 blocks, each of size 112; the
blocks are the orbits of length 112 in the 275 point stabilizers.
The group of this design, which must contain AUT(McL) and turns out to
equal to AUT(McL), may be computed by the command
\smallskip
\hskip0.4in{\tt desauto\quad dmcl\quad Y}
\smallskip
which saves the group as {\tt Y}.  (Note that we are computing the group
of the design, not the group of the graph associated with the orbit of
length 112; in general, the design group is larger than the graph group,
although in this case they are the equal.)  The directory also contains a
second design {\tt dmclx}, isomorphic to {\tt dmcl}.  An isomorphism may
be found by the command
\smallskip
\hskip0.4in{\tt desiso\quad dmcl\quad dmclx\quad s}
\smallskip
which sets {\tt s} to an isomorphism from {\tt dmcl} to {\tt dmclx}.
\bigbreak
Finally, the subdirectories {\tt had32} and {\tt had104} 
contains  $32\times32$ and $104\times104$ matrices
over ${\rm GF}(3)$, respectively, which are 
essentially the Paley--Hadamard matrices.  
(The entries of -1 have been changed to 2.)  The (monomial) automorphism
group of either of these Hadamard matrices may be computed by the command
\smallskip
\hskip0.4in{\tt matauto\quad -mm\quad had\kern1pt$i$\quad Z}
\smallskip
($i = 32{\rm\ or\ }104$),
which sets {\tt Z} to the automorphism group.  This subdirectories also
contain matrices {\tt had\kern1pt$i$\kern1ptx} obtained by 
applying random monomial
permutations to the rows and columns of {\tt had\kern1pt$i$}.  Equivalence of
{\tt had\kern1pt$i$} and {\tt had\kern1pt$i$\kern1ptx} may be 
established by the command
\smallskip
\hskip0.4in{\tt matiso\quad -mm\quad had\kern1pt$i$\quad had\kern1pt$i$\kern1ptx\quad w}
\smallskip
which sets {\tt w} to an monomial isomorphism 
from {\tt had\kern1pt$i$} to {\tt had\kern1pt$i$\kern1ptx}.
Finally, the subdirectories contain Hadamard designs {\tt dhad\kern1pt$i$}\kern3pt\
($i = 32{\rm\ or\ }104$) and equivalent designs {\tt dhad\kern1pt$i$\kern1ptx},
whose groups may be computed using the {\tt desauto} command, and whose
equivalence may be established with the {\tt desiso} command.
%
%
\section{IX.\quad OTHER COMMANDS}
%
In the course of testing and benchmarking the partition backtrack
algorithms described in Section~IV, the author developed a number of other 
programs.  Most of these programs were put
together quickly, with a view toward simplicity rather than efficiency;
in some cases, they are very inefficient.  Also, some of them perform
only minimal error checking.  Nonetheless, they may prove useful since they 
operate on objects specified in the format described in Section~II.
\medbreak
All of these programs accept the {\tt -a}, {\tt -i}, {\tt -mb:$k$}, {\tt -mw:$w$},
{\tt -n:}{\it name}, {\tt -p:}{\it path}, and {\tt -q} options, as described
in Section~V, whenever they would be meaningful.  (For example, the {\tt -i}
option is meaningful only if the command creates a permutation or
permutation group.)  Other options vary by command, and are discussed separately
for each command below.
%
\subsection{Base and strong generating set construction:}The {\tt generate}
command may be used to construct a probable base and strong generating set
for the permutation group generated by specified permutations.  The random
Schreier method (Leon, 1980) is used.  If the group order is known in advance,
this method always produces a correct base and strong generating set, although
there is no bound on the time required to do so.  Otherwise,
there is no guarantee that the method will produce a correct result.  However,
in the author's experience, it nearly always does give a correct result, and
it runs far more quickly than alternative methods, such as the Schreier--Sims
or Schreier--Todd-Coxeter--Sims algorithms.
\medbreak
The format for the command is
%
\smallskip
\centerline{{\tt generate}\quad {\it options\/}\quad {\it inputGroup\/}\quad 
                         {\it outputGroup}}
%
\smallskip
where {\it options\/} denotes
\smallskip
\centerline{
            [{\tt -a}]\kern-1pt\enskip
            [{\tt -i}]\kern-1pt\enskip
            [{\tt -mb:}$k$]\kern-1pt\enskip
            [{\tt -mw:}$w$]\kern-1pt\enskip
            [{\tt -n:}{\it name\/}]\kern-1pt\enskip
            [{\tt -nro}]\kern-1pt\enskip
            [{\tt -p:}{\it path\/}]\kern-1pt\enskip
            [{\tt -q}]\kern-1pt\enskip
            [{\tt -s:}{\it seed\/}]\kern-1pt\enskip
            [{\tt -ti:}$i$]\kern-1pt\enskip
            [{\tt -tr:}$m$]\kern-1pt\enskip
            [{\tt -z}]}
\smallskip
Here {\it inputGroup\/} denotes the original permutation group, for which
a base and strong generating set are not yet available.
The factored group order for {\it inputGroup\/} may or may not be present.
The random Schreier method is used to construct a probable base and strong
generating set for {\it inputGroup},, and the result is saved as the 
permutation group {\it outputGroup}.
If the factored group order is present, the computation will continue until
a base and strong generating set has been found.  Otherwise it continues until
{\it m\/} consecutive quasi--random elements of the group factor in terms of the
possible base and strong generating set, where {\it m\/} is the integer
specified in the {\tt -tr:}$m$ option (default 40).  High values of 
{\it m} may be specified to reduce the chance of an incorrect result, at the
cost of slowing down the computation.
\medbreak
Normally, before a new strong generator is added to the strong generating
set, an attempt is made to replace the new generator by a power of it, in
order to obtain generators of low order.  (This may be desirable later on if the 
Schreier--Todd--Coxeter--Sims method is used to verify the base and strong
generating set; in addition, it saves space whenever a non--involutory
generator is converted to an involution.)    This attempt may be suppressed
by the {\tt -nro} option.  Note, however, that replacement of generators
by powers is relatively inexpensive, so the {\tt -nro} option saves little time.
The {\tt -ti:}$i$ option may be specified in order to have the program
try harder to find involutory generators.  Up to $i$ consecutive generators that
cannot be converted to involutory generators will be rejected.  The default
for $i$ is 0; higher values often increase the execution time a good
deal.
\medbreak
If the {\tt -z} option is specified, the program will make some attempt
to remove certain redundant strong generators from the strong generating set for
{\it outputGroup}. 
%
\subsection{Base change:}The {\tt chbase}
command may be used to change the base in a permutation group.  The
command format is
%
\smallskip
\centerline{{\tt chbase}\quad {\it inputGroup\/}\quad 
                         $p_1,p_2,\ldots,p_k$\quad {\it outputGroup}}
\smallskip
where {\it options\/} denotes
\smallskip
\centerline{
        [{\tt -a}]\enskip
        [{\tt -i}]\enskip
        [{\tt -mb:}$k$]\enskip
        [{\tt -mw:}$w$]\enskip
        [{\tt -n:}{\it name\/}]\enskip
        [{\tt -p:}{\it path\/}]\enskip
        [{\tt -q}]\enskip
        [{\tt -z}]}
\smallskip
The base for permutation group {\it inputGroup\/} is
changed, if necessary, so that it begins  with $p_1,p_2,\ldots,p_k$,
and the group with this new base is saved as {\it outputGroup}.  Note
that, in the list $p_1,p_2,\ldots,p_k$ of points, individual points
are separated by commas but {\it not\/} by blanks.  Note also that the
points $p_1,p_2,\ldots,p_k$ are included in the new base even if
they are redundant as base points.  However, no other redundant base points
will appear in the new base.  If the {\tt -z} option is specified, certain
redundant strong generators will be removed following the base change.
%
%
\subsection{Conjugation by a specified permutation:}The {\tt cjper}
command may be used to conjugate an object (group, permutation, point set,
partition, design, matrix, or code) by a specified permutation.  The command
format is
%
\smallskip
\centerline{{\tt cjper}\quad {\it options\quad type\quad object\quad
                             conjugateObject\quad conjugatingPerm}}
\smallskip
where {\it options\/} denotes
\smallskip
\centerline{
            [{\tt -a}]\enskip
            [{\tt -b}]\enskip
            [{\tt -d:}{\it deg\/}]\enskip
            [{\tt -i}]\enskip
            [{\tt -mb:}$k$]\enskip
            [{\tt -mm}]\enskip
            [{\tt -mw:}$w$]\enskip
            [{\tt -n:}{\it name\/}]\enskip
            [{\tt -p:}{\it path\/}]\enskip
            [{\tt -q}]\enskip}
\smallskip
Here {\it type\/} must be one of the keywords {\tt group}, {\tt perm},
{\tt set}, {\tt partition}, {\tt design}, {\tt matrix}, or {\tt code},\enskip
{\it object\/} must be an object of the type designated by {\it type},\enskip
and {\it conjugatingPerm\/} must be a permutation.  In the event that {\it object\/}
is a permutation, set, or partition, the {\tt -d:}{\it deg} option {\it must\/}
be used to specify the degree {\it deg}.  The program sets
{\it conjugateObject\/} to the object obtained by conjugating {\it object\/}
by {\it conjugatingPerm\/}, in the case that {\it object\/} is a group 
or permutation, or to the object obtained by
applying {\it conjugatingPerm\/} to {\it object}, if {\it object} is a
point set, partition, design, matrix, or code.  In the event that
{\it object\/} is a group, the {\tt -b} option forces the program to
compute a base and strong generating set for {\it conjugateObject\/}; by
default, {\it conjugateObject\/} will have a base and strong generating
set only if {\it object\/} does.
\medbreak
In the event that {\it object\/} is a design or matrix, the degree
is treated as the number of points plus the number of blocks, or the
number of rows plus the number of columns, respectively.
Blocks or columns are permuted as well as points or rows.  However,
since the input format for permutations permits a permutation of degree $n$
to be treated as a permutation of any higher degree, this represents no
real restriction.
\medbreak
If {\it object\/} is a matrix over a finite field ${\rm GF}(q)$, the {\tt -mm}
option may be specified.  Then {\it conjugatingPerm\/} must have degree
$(q-1)(r+c)$, where $r$ and $c$ are the number of rows and columns, and
must satisfy the ``monomial property'', as described in Section~III.

%
\subsection{Conjugation by a random permutation:}The {\tt cjrndper}
command may be used to conjugate an object by a either a random permutation,
or by a permutation chosen at random from a specified permutation group.  The command
format is
%
\smallskip
\centerline{{\tt cjrndper}\quad {\it options\quad type\quad object\quad
                             conjugateObject}\quad [{\it conjugatingPerm\/}]}
\smallskip
where {\it options\/} denotes
\smallskip
\centerline{
        [{\tt -a}]\kern-1pt\enskip
        [{\tt -b}]\kern-1pt\enskip
        [{\tt -d:}{\it deg\/}]\kern-1pt\enskip
        [{\tt -g:}{\it grp\/}]\kern-1pt\enskip
        [{\tt -i}]\kern-1pt\enskip
        [{\tt -mb:}$k$]\kern-1pt\enskip
        [{\tt -mm}]\kern-1pt\enskip
        [{\tt -mw:}$w$]\kern-1pt\enskip
        [{\tt -n:}{\it name}]\kern-1pt\enskip
        [{\tt -p:}{\it path\/}]\kern-1pt\enskip
        [{\tt -s:}{\it seed\/}]\kern-1pt\enskip
        [{\tt -q}]}
\smallskip
Here {\it type\/} must be one of the keywords {\tt group}, {\tt perm},
{\tt set}, {\tt partition}, {\tt design}, {\tt matrix}, or {\tt code},\enskip
and {\it object\/} must be an object of the type designated by {\it type}.
In the event that {\it object\/}
is a permutation, set, or partition, the {\tt -d:}{\it deg} option {\it must}
be used to specify the degree {\it deg}.  If {\it object\/} is a permutation
or permutation group, the program sets
{\it conjugateObject\/} to the object obtained by conjugating {\it object\/}
by a certain permutation $x$; if {\it object\/} is a point set, partition,
design, matrix, or code, it sets {\it conjugateObject\/} to the
object obtained by applying a certain permutation $x$ to {\it object}.
In either case, the permutation $x$ is
chosen at random from the group {\it grp}, if the {\tt -g:}{\it grp} option
is specified, or at random from the symmetric group, if the
{\tt -g:}{\it grp} option is omitted.  If {\it conjugatingPerm\/} is specified, 
the permutation $x$ is saved as {\it conjugatingPerm}.
The {\tt -s:}{\it seed} option may be used to specify
a specific seed for the random number generator that is used internally.
In the event that {\it object\/} is a group, the {\tt -b} option forces the
program to compute a base and strong generating set for {\it conjObject}, even
when one is not available for {\it object}.
\medbreak
In the event that {\it object\/} is a design (or matrix), both points
and blocks (or both rows and columns) are permuted.
\medbreak
If {\it object\/} is a matrix over a finite field ${\rm GF}(q)$, the {\tt -mm}
option may be specified.  Then a random monomial permutation is applied
to the rows and columns of the input matrix.
%
%
%
\subsection{Commutator groups and lower central series:}The {\tt commut}
command may be used to compute commutator groups.  Repeated application of
the command may be used to compute lower central series.  Specifically, given
a group $G$ and a (not necessarily normal) subgroup $H$ of $G$, the command
computes the commutator group $C = [G,H]$.  The command format is
%
\smallskip
\centerline{{\tt commut}\quad {\it options}\quad {\it permGroup}\quad
                        [{\it subgroup\/}]\quad {\it commutatorGroup}}
\smallskip
where {\it options\/} denotes
\smallskip
\centerline{
        [{\tt -a}]\enskip
        [{\tt -i}]\enskip
        [{\tt -mb:}$k$]\enskip
        [{\tt -mw:}$w$]\enskip
        [{\tt -n:}{\it name}]\enskip
        [{\tt -p:}{\it path\/}]\enskip
        [{\tt -q}]}
\smallskip
Here, {\it permGroup\/}, {\it subgroup\/}, and {\it commutatorGroup\/}
play the role of $G$, $H$, and $C$ above, respectively.
If {\it subgroup\/} is specified, it must be a subgroup of {\it permGroup\/}
(not checked); the command sets {\it commutatorGroup\/}
to the commutator of {\it permGroup\/} and {\it subgroup}.  If subgroup is
omitted, the command sets {\it commutatorGroup\/} to the commutator of
{\it permGroup\/} with itself (the derived group). 
\medbreak
At present, the strong generating set for the commutator group is
constructed only by the random Schreier method.  Thus there is a
(probably small) possibility that the base and strong generating set 
constructed for the commutator group will be incorrect.  (If this
occurs, the generators constructed will generate the commutator group, but
not strongly.  This undesirable feature will be fixed eventually.)
\medbreak
The return code from the {\tt commut} command will 0 if the commutator group
has order 1.  
Otherwise the return code will depend on  whether the order of $H$ (i.e.,
{\it subgroup\/})
is known in advance.  If so, the return code will  1 if $\vert C\vert \neq
\vert H\vert$ and 2 otherwise.  (If $H$ is normal in $G$, these correspond
to the cases $H \subset G$ and $H = G$, respectively.)  If not, the return
code will be 3. 
%
%
%
\subsection{Comparison of groups, normality, and centralization:}The {\tt compgrp} command may be
used to check if either of two permutation groups is contained
in the other, or normalizes the other, or whether the two groups
centralize each other.  The command format is
%
\smallskip
\centerline{{\tt compgrp}\quad [{\tt -c}] \enskip[{\tt -n}]\enskip
            [{\tt -mb:}$k$]\enskip [{\tt -mw:}$w$]\enskip
            [{\tt -p:}{\it path\/}]\quad {\it permGroup1\quad permGroup2}}
\smallskip
This command checks if either of {\it permGroup1\/} or {\it permGroup2\/} is contained
in the other.  If the {\tt -n} option is specified, it also checks if
either group normalizes the other.  (Note here the {\tt -n} option is used
for a different purpose that that described in Section~V.)  If the {\tt -c} option is given, it
checks whether the two groups centralize each other.  Messages indicating the result are
written to the standard output.   The return code is 0 if the two groups
are equal, 1 if {\it permGroup1\/} is a proper subgroup of {\it permGroup2\/},
2 if {\it permGroup2\/} is a proper subgroup of {\it permGroup1\/}, and 3
otherwise.\quad  Note:  The procedure for checking normality
is, at present, extremely inefficient in many cases.
%
%
%
\subsection{Comparison of permutations:}The {\tt compper} command may be
used to check if two permutations are equal, or if a permutation is the
identity.  The command format is
%
\smallskip
\centerline{{\tt compper}\quad {\it degree\quad permutation1}\quad 
            {\tt [}{\it permutation2\/}{\tt ]}}
\smallskip
This command checks if {\it permutation1\/} and {\it permutation2}, both of
which must be permutations of degree {\it degree}, are equal.  It prints a
message indicating whether the permutations are equal, and gives a return
code of 0 if they are equal and 1 otherwise.  If {\it permutation2\/} is
omitted, it is taken as the identity; thus the command checks if
{\it permutation1\/} is the identity.
%
%
%
\subsection{Comparison of point sets:}The {\tt compset} command may be
used to check if two point sets are equal.  The command format is
%
\smallskip
\centerline{{\tt compset}\quad {\it degree\quad set1}\quad 
            {\tt [}{\it set2\/}{\tt ]}}
\smallskip
This command checks if {\it set1\/} and {\it set2}, both of
which must be points sets of degree {\it degree}, are equal, or if either
is contained in the other.  if {\it set2\/} is omitted, it is taken to be
the empty set, and the command checks if {\it set1\/} is empty.  It prints a
message indicating the result.  If {\it set2\/} is specified, the return code is 0 if
the two sets are equal, 1 if {\it set1\/} is properly contained in {\it set2\/}, 
2 if {\it set1\/} properly contains {\it set2\/}, and 3 otherwise.
If {\it set2\/} is omitted, the return code is 0 if {\it set1\/} is empty
and 1 otherwise.
%
%
\subsection{Coset weight distributions of codes:}The {\tt cwtdist}
command\footnote{${}^{\dag}$}{\elevenpoint At time of writing, this command
was not complete.  It should be available shortly.}
may be used to compute the coset weight distribution of a linear code.
At present, the program is restricted to binary codes of codimension at
most 32.  The program is highly optimized; nonetheless, time and space
requirements may restrict the codimension to values less than its maximum
of 32.  The command format is
%
\smallskip
\centerline{{\tt cwtdist}\quad {\it options}\quad {\it code}\quad
            [{\it maxCosWeight\/}\quad [{\it matrix\/}]]}
\smallskip
where {\it options\/} denotes
\smallskip
\centerline{
        [{\tt -a}]\enskip
        [{\tt -n:}{\it name\/}]\enskip
        [{\tt -p:}{\it path\/}]\enskip
        [{\tt -q}]}
\smallskip
The program computes the coset weight distribution of the code 
{\it code\/}, that is, for each $d$, it determines the number of cosets of
the code having minimum weight $d$.  If {\it maxCosWeight\/} is specified
(It should be an integer.), the computation is performed only for
$d \leq {\it maxCosWeight}$.  If {\it matrix\/} is given, a coset representative
is saved for one coset whose minimum weight is the minimum of the covering 
radius and {\it maxCosetWeight\/}.  {\it Note:\/}  It is not possible to 
specify {\it matrix\/} without specifying {\it maxCosWeight\/}; however,
an artificially large value of {\it maxCosWeight\/} may be given instead.
%
%
%
\subsection{Designs from groups:}The {\tt orbdes} command may be used to
construct a block design $D$ from a permutation group $G$, which normally
should be transitive.  The design $D$ will have $n$ points and $n$ blocks, each
having the same number of points, where $n$ is the degree.  The automorphism
group ${\rm AUT}(D)$ will contain the group $G$ (perhaps properly).
For a specified point $\gamma$, let $\Gamma$ denote the orbit of $\gamma$
under the point stabilizer $G_1$.  The blocks of $D$ are exactly
$\Gamma^{u_1},\ldots,\Gamma^{u_k}$, where $k$ is the size of the $G$--orbit
of 1, and $u_1,\ldots,u_k$ map 1 to the different points of the $G$--orbit
of 1.  The command format is
%
\smallskip
\centerline{{\tt orbdes}\quad [{\tt -a}]\enskip[{\tt -m}]\enskip
            [{\tt -mb:}$k$]\enskip [{\tt -mw:}$w$]\enskip
            [{\tt -p:}{\it path\/}]\quad
            {\it permGroup}\quad {\it point}\quad {\it design}}
\smallskip
Here {\it permGroup}, {\it point}, and {\it design\/} play the role of
$G$, $\gamma$, and $D$ above, respectively.  If the {\tt -m} option is
given, the incidence matrix of the design is written out as a matrix;
otherwise the design is written in the standard format for designs.
%
%
%
\subsection{Finding group elements of specified order:}The {\tt fndelt}
command may be used to locate group elements of specified order, using
a quasi-random search process.  (Random group elements are examined in
searching for ones whose order is a multiple of the specified order.)
The command format is
%
\smallskip
\centerline{{\tt fndelt}\quad {\it options}\quad {\it permGroup} \quad
                        $n$\quad $k_1$\quad $k_2$\quad$\ldots$}
\smallskip
where {\it options\/} denotes
\smallskip
\centerline{
        [{\tt -a}]\enskip
        [{\tt -f}]\enskip
        [{\tt -i}]\enskip
        [{\tt -m:}{\it trials}]\enskip
        [{\tt -mb:}$k$]\enskip
        [{\tt -mw:}$w$]\enskip
        [{\tt -n:}{\it name}]}
\vskip2pt
\centerline{
        [{\tt -o:}{\it ord\/}]\enskip
        [{\tt -p:}{\it path\/}]\enskip
        [{\tt -po}]\enskip
        [{\tt -s:}{\it seed\/}]\enskip
        [{\tt -wg:}{\it grp\/}]\enskip
        [{\tt -wp:}{\it perm\/}]}
\smallskip
By default, the command searches quasi--randomly
until it finds $n$ involutions in the
group {\it permGroup\/}, and
if $k_1,\,k_2,\,\ldots$ are specified, it saves the $k_1$\kern1ptst,
$k_2$\kern1ptnd, $\ldots$ elements found.  (See discussion of the 
{\tt -wp} and {\tt -wg} options below.)  The value of $n$ may be at
most 99.  The seed used in the random number generator may be specified
by the {\tt -s:}{\it seed\/} option; two invocations with the same
(default or specified) seed should produce identical results.
If the {\tt -o:}{\it ord\/}
option is given, the command searches for elements of order {\it ord}, rather
than for involutions.  If the {\tt -f} option is specified, it prints
the number of fixed points of each element found.  If the {\tt -po} option
is specified, it prints the orders of all products of the elements found.
If the {\tt -wp:}{\it perm\/} option is given and $k_1$ is specified,
the $k_1$\kern1ptst element found is saved as the permutation {\it perm}.
If the {\tt -wg:}{\it grp\/} is given and at least $k_1$ is specified,
a group {\it grp\/} is created using the $k_1$\kern1ptst,
$k_2$\kern1ptnd, $\ldots$ elements found as generators.
\medbreak
%
In some groups, elements of a specified order may be very difficult
to find using the quasi--random search technique employed by {\tt fndelt}.
For example, it is very difficult to find involutions in ${\rm PGL}_2(2^k)$
if $k$ is fairly high (say 10 to 15).  The {\tt -m:}{\it trials\/} option
may be used to terminate the command after {\it trials\/} random
group elements have been generated, regardless of how many elements of
the desired order have been found.  By default, {\it trials\/} is
essentially infinite.
\medbreak
%
%
%
\subsection{Normal closures:}The {\tt ncl}
command may be used to compute normal closures of subgroups.  Specifically, 
given a group $G$ and a subgroup $H$ of $G$, the command
computes the normal closure $H^G$ of $H$ in $G$.  The command format is
%
\smallskip
\centerline{{\tt ncl}\quad {\it options}\quad {\it permGroup}\quad
                        {\it subgroup\/}\quad {\it normal closure}}
\smallskip
where {\it options\/} denotes
\smallskip
\centerline{
        [{\tt -a}]\enskip
        [{\tt -i}]\enskip
        [{\tt -mb:}$k$]\enskip
        [{\tt -mw:}$w$]\enskip
        [{\tt -n:}{\it name}]\enskip
        [{\tt -p:}{\it path\/}]\enskip
        [{\tt -q}]}
\smallskip
The command sets {\it normalClosure\/}
to the normal closure in {\it permGroup\/} of {\it subgroup}.  (Note that
{\it subgroup\/} must be a subgroup of {\it permGroup\/}; this is not checked.)
\medbreak
At present, the strong generating set for the normal closure is
constructed only by the random Schreier method.  Thus there is a
(probably small) possibility that the strong generating set
may be incorrect.  (This will
be fixed eventually; if it occurs, the generators obtained will generate
the normal closure, but not strongly.)
%
%
%
\subsection{Orbit structure:}The {\tt orblist} command performs various
calculations relating to the orbit structure of a specified group, or to the 
orbit structure of point stabilizers within the group.  The command format
%
\smallskip
\centerline{{\tt orblist}\quad {\it options}\quad {\it permGroup}\quad
            [\kern1pt$p_1,p_2,\ldots,p_k$\quad[{\it ptStabGroup\/}]\kern1pt]}
\smallskip
where {\it options\/} denotes
\smallskip
\centerline{
        [{\tt -a}]\enskip
        [{\tt -i}]\enskip
        [{\tt -len}]\enskip
        [{\tt -lr}]\enskip
        [{\tt -mb:}$k$]\enskip
        [{\tt -mw:}$w$]\enskip
        [{\tt -n:}{\it name\/}]\enskip
        [{\tt -p:}{\it path\/}]}
\vskip2pt
\centerline{
        [{\tt -ps:}{\it set\/}]\enskip
        [{\tt -q}]\enskip
        [{\tt -r}]\enskip
        [{\tt -s:}{\it seed\/}]\enskip
        [{\tt -wno:}{\it $k$}]\enskip
        [{\tt -wo:}{\it $p_1,p_2,\ldots$}]\enskip
        [{\tt -wp:}{\it partn\/}]\enskip
        [{\tt -z}]}
\smallskip
In the absence of any options, and with only one non--option parameter,
the command writes (to standard output)
the orbits of the group {\it permGroup}.  For each orbit, the orbit
representative, the orbit length, and the list of points in the orbit
are written.  If the {\tt -r} option is given, the order in which the
orbits are listed is randomized; the seed for the random number generator
that is used may be specified by means of the {\tt -s:}{\it seed\/} option.
(Note that the {\tt -r} option is ignored if the {\tt -len} or {\tt -lr} options are
specified.)
\medbreak
If the {\tt -len} option is specified, only the orbit lengths are
written.  If the {\tt -lr} option is given, only the
orbit representatives and orbit lengths are given; the output consists of
a list of pairs {\it rep\/}{\tt :}{\it len}, where {\it rep\/} is an
orbit representative and {\it len\/} is the length of the corresponding
orbit.
\medbreak
The {\tt -wp:}{\it partn} may be used save the orbit partition of the
group as the partition {\it partn}.
The {\tt -wo:}{\it $p_1,p_2,\ldots$} option may be used to save the
union of the orbits of points $p_1,p_2,\ldots$ as a point set; the
name of the point set is the string {\it set\/} given by 
the {\tt -ps:}{\it set\/} option.
Alternatively, the {\tt -wno:}{\it $k$} option causes the union of
the first $k$ orbits to be saved as a point set, whose name again
is specified by the {\tt -ps:}{\it set\/} option.
\medbreak
If a second non--option parameter $p_1,p_2,\ldots,p_k$ is specified, 
all orbit calculations are carried out in the stabilizer in ${\it permGroup\/}$ 
of the point sequence $p_1,p_2,\ldots,p_k$, rather than in {\it permGroup\/} 
itself.  Note that this entails a base change for the group {\it permGroup}.
The {\tt -z} option causes redundant strong generators for {\it permGroup}
to be removed following this base change.
Specifying a third non--option parameter {\it ptStabGroup\/} option causes 
point stabilizer of $p_1,p_2,\ldots,p_k$ to be saved
as the permutation group {\it ptStab}.
%
%
\subsection{Point stabilizers:}The {\tt ptstab}
command may be used to compute point stabilizers in permutation groups.  The
command format is
%
\smallskip
\centerline{{\tt ptstab}\quad {\it options}\quad {\it permGroup}\quad
            $p_1,p_2,\ldots,p_k$\quad {\it ptStabGroup\/}}
\smallskip
where {\it options\/} denotes
\smallskip
\centerline{
        [{\tt -a}]\enskip
        [{\tt -i}]\enskip
        [{\tt -mb:}$k$]\enskip
        [{\tt -mw:}$w$]\enskip
        [{\tt -n:}{\it name\/}]\enskip
        [{\tt -p:}{\it path\/}]\enskip
        [{\tt -q}]\enskip
        [{\tt -z}]}
\smallskip
The point stabilizer in the group {\it permGroup\/} of the list
$p_1,p_2,\ldots,p_k$ is computed and saved as the permutation group
{\it ptStabGroup}.  Note that, in the list $p_1,p_2,\ldots,p_k$, individual
points are separated by commas but {\it not\/} by blanks.
The {\tt -z} option causes certain redundant strong generators for {\it ptStabGroup}
to be removed before the group is saved.
%
%
%
%
\subsection{Random point sets and partitions:}The {\tt randobj} 
command may be used to construct a random $k$--element subset of 
$\{1,\ldots,n\}$ for specified $n$ and $k$, or to construct a partition
of $\{1,\ldots,n\}$ that is random subject to the cells having specified
sizes.  The command format
%
\smallskip
\centerline{{\tt randobj}\quad {\it options}\quad {\it type}\quad $n$\quad 
            $k_1,k_2,\ldots,k_p$\quad {\it newObject}}
\smallskip
where {\it options\/} denotes
\smallskip
\centerline{
        [{\tt -a}]\enskip
        [{\tt -e}]\enskip
        [{\tt -n:}{\it name\/}]\enskip
        [{\tt -s:}{\it seed\/}]}
\smallskip
Here {\it type\/} must be either {\tt set} or {\tt partition}; it indicates
whether a point set or partition is to be constructed.  For a point set,
$p$ must be 1; a random $k_1$--element subset of $\{1,\ldots,n\}$ is constructed.
For a partition, in the absence of the {\tt -e} option, 
$k_1 + k_2 + \ldots + k_p$ must equal $n$; a partition of $\{1,\ldots,n\}$
random subject to having cell sizes $k_1$,~$k_2$,...,~$k_p$ is constructed.
However, if the {\tt -e} option is specified, then $p$ must equal 1, and
a random partition having $k_1$ cells of equal size (as closely as possible)
is constructed.  In any case, the point set or partition constructed is saved
as the object {\it newObject}.
\medbreak
The {\tt -s:}{\it seed} option may be used to specify an initial seed for
the random number generator that is used.
%
%
\subsection{Weight distributions of codes:}The {\tt wtdist}
command may be used to compute the weight distribution of a linear code.
The program is highly optimized, both for binary and nonbinary codes.
The command format is
%
\smallskip
\centerline{{\tt wtdist}\quad {\it options}\quad {\it code}\quad
            [{\it saveWeight\/}\quad {\it matrix\/}]}
\smallskip
where {\it options\/} denotes
\smallskip
\centerline{
        [{\tt -a}]\enskip
        [{\tt -b}]\enskip
        [{\tt -g}]\enskip
        [{\tt -n:}{\it name\/}]\enskip
        [{\tt -p:}{\it path\/}]\enskip
        [{\tt -pf:}$p$]\enskip
        [{\tt -s:}$m$]\enskip
        [{\tt -q}]\enskip
        [{\tt -1}]}
\smallskip
The weight distribution of the code {\it code\/} is computed and written to
the standard output.  If {\it saveWeight\/} (which should be a positive
integer) and {\it matrix\/} are specified, the codewords of weight
{\it saveWeight\/} are saved; specifically, a new matrix {\it matrix\/} is
created whose rows are the vectors of weight {\it saveWeight\/} in the
code {\it code}.  However, for nonbinary fields, only one codeword from
each one-dimensional subspace is saved.  (Thus, for a code over 
${\rm GF}(q)$ with $k$ vectors of weight {\it saveWeight}, {\it matrix\/} will
have $k/(q-1)$ rows.)  Also, if the {\tt -1} option is specified, only one
codeword of weight {\it saveWeight\/} will be saved, and thus {\it matrix\/}
will have only one row.  In the event that the code has no codewords of the
specified weight, the matrix is not created.
\medbreak
Four options, {\tt -b}, {\tt -g}, {\tt pf:}$p$, and {\tt s:}$m$, are never
necessary but may be used to optimize performance.  The {\tt -s:}$m$
option may be coded if codewords of weight {\it saveWeight\/} are to be
saved, and if the number of such codewords is known in advance; the value
of $m$ should be the number of such codewords (excluding scalar multiples,
for nonbinary fields).  Use of this option saves some time and space.  (If
an incorrect value of $m$ is specified, the savings in time and space may
be lost, but the results will still be correct.)
\medbreak
To understand the other optimization options,
it is necessary to understand that the {\tt wtdist} command invokes one of 
two programs: a considerably optimized program for codes over any field, or 
an even more highly optimized one for binary codes meeting certain 
constraints on length and dimension.  The binary code program requires a fixed
amount of time (several seconds or more on a micro or workstation) to 
construct a certain table of size 65536, even if the 
dimension of the code is small.  Accordingly, the {\tt wtdist} command
invokes the general code program for binary codes of low dimension; however,
the criteria for choosing the cutoff point is relatively crude, and often
a nonoptimal choice may be made.  The {\tt -g} option forces the general
code program to be used, even if the binary one would be chosen by default.
The {\tt -b} option forces the binary code program to be used, provided the
code is binary, provided its length is at most 128, and provided its 
dimension is at most 3.  (If any of these conditions fail, the binary
program cannot be used.)  
\medbreak
The {\tt -pf:}$p$ option is applicable only if
the general code program is employed.  The value of $p$, which is referred
to as the {\it packing factor}, should be a positive integer such that
$q^p\leq 65535$.  (Here $q$ is the field size.)  Internally, the program will pack
$p$ coordinates of each codeword into a single 16--bit word.  Higher values
of $p$ improve performance on codes of high dimension.  On the other hand,
the size of the internal tables that must be allocated and initialized
rise very rapidly as a function of $p$, and in particular, a value of 
$p$ maximal subject to $q^p\leq 65535$ will be practical only with a 
rather large memory, and will be optimal with respect to time 
only for codes of quite high dimension, because of the large amount of time
spent initializing the tables.  (The size in bytes of the largest single
table used internally is roughly $q^p e k \lceil n/p\rceil$, where $n$ is
the length, $k$ is the dimension, and $e$ is the exponent of the field.)
The program will choose a packing factor by default, but at present the
procedure for making this choice is crude.  In particular, if the program
runs out of memory, a smaller packing pactor should be specified.
%
%
\section{X.\quad THE UNIX DISTRIBUTION}
%
On Unix, the programs, documentation, and examples will be available
by anonymous ftp from {\tt math.uic.edu}.  The 
directory {\tt pub/leon/partn} should contain the following files,
where {\it r} denotes the release number, e.g. {\tt 1.00}:
\vskip1pt
\smallskip
\vbox{{\elevenpoint\advance\leftskip by 0.45truein\obeywhitespace\tt\catcode`^=\other
doc-\kern-1pt{\it r}.tar.Z
examples-\kern-1pt{\it r}.tar.Z
src-\kern-1pt{\it r}.tar.Z
bin16-\kern-1pt{\it r}.sun4.tar.Z
bin16-\kern-1pt{\it r}.sun3.tar.Z
\vskip-2pt
bin32-\kern-1pt{\it r}.sun4.tar.Z
bin32-\kern-1pt{\it r}.sun3.tar.Z
\vskip0pt}}
\vskip-2pt
(The last two files may be omitted to conserve disk space, but can be made
available on request; contact the author at {\tt leon@turing.math.uic.edu}.)
Users with a Sun/4 should obtain the files {\tt doc.tar.Z}, 
{\tt examples.tar.Z}, and {\tt bin16.sun4.tar.Z}, which contain, respectively,
the documentation, the examples, and 16-bit executables for the Sun/4.
Users desiring the C language source code (of limited use, due to inadequate
documentation) should obtain the file {\tt src.tar.Z}.  Note that source code
is not required since binary executables for the programs are supplied.
Users needing
to compute with groups of degree greater that 65000 (approximate) should
obtain the file {\tt bin32.sun4.tar.Z}, which contains 32-bit executables.
Users with a Sun/3 merely substitute ``{\tt sun3}'' for ``{\tt sun4}'' in
the preceding instructions.  Other users should obtain only the files
{\tt doc.tar.Z}, {\tt examples.tar.Z}, and {\tt src.tar.Z}.  It will be
necessary to compile the source; a make file is provided for this purpose
(See Section~XI).
\smallskip
A directory, say {\tt partn}, should be created, and all the files should
be downloaded or moved to this directory.  They should then be uncompressed
and extracted by commands such as
\smallskip
\vbox{{\elevenpoint\advance\leftskip by 0.45truein\obeywhitespace\tt\catcode`^=\other
uncompress -v doc.tar.Z
tar xvf doc.tar
\vskip0pt}}
\smallskip
(Repeat the above for each of the files.)
The result should be subdirectories of {\tt partn} as follows:
\medbreak
\hskip0.45truein\hbox to 0.85truein{\tt doc\hfil}{Documentation for the programs --
           primarily this manual.}
\vskip2pt
\hskip0.45truein\hbox to 0.85truein{\tt examples\hfil}{The subdirectories of this
           directory contain examples.  See Section~VIII.}
\vskip2pt
\hskip0.45truein\hbox to 0.85truein{\tt test\hfil}{Unix shell scripts to test
          the partition backtrack programs, using some of the examples provided
          in the {\tt examples} subdirectory.}
\vskip2pt
\hskip0.45truein\hbox to 0.85truein{\tt src\hfil}\vtop{\hsize=5.2truein\relax
          Source code and a make file, 
          to allow compilation of the programs on machines other than the
          Sun/3 and Sun/4.}
\vskip2pt
\hskip0.45truein\hbox to 0.85truein{\tt bin16\hfil}{executable programs for the Sun/3 
                       or Sun/4, using 16--bit integers.}
\vskip2pt
\hskip0.45truein\hbox to 0.85truein{\tt bin32\hfil}{executable programs for the Sun/3 
                       or Sun/4, using 32--bit integers.}
\smallskip
For the Sun/3 or Sun/4, the appropriate directory {\tt partn/bin16} or 
{\tt partn/bin32} should be added to the path, or the files from one of
those directories should be copied to a directory on the path.  For other
Unix machines, it will be necessary to recompile the source; see Section~XI.
%
\medbreak
Once installation is complete, the programs may be tested using the 
shell scripts in the subdirectory {\tt test}.  Specifically, the following
shell scripts, written for the Bourne shell, are provided:
\smallskip
\vbox{{\elevenpoint\advance\leftskip by 0.45truein\obeywhitespace\tt\catcode`^=\other
test\_setstab
test\_cent
test\_inter
test\_desauto
test\_setimage
test\_conj
test\_desiso
\vskip0pt}}
Each of the above shell files may be run, without options or parameters.
When {\tt test\_setstab} is run, the output is collected in a file named
{\tt setstab.output}, as well as appearing on the screen.  This file 
may be compared to the file {\tt setstab.correct}, supplied in the subdirectory
{\tt tests}, which contains the correct output.  The files should match
exactly.  The comparison may be performed, for example, with the command
\smallskip
\vskip2pt
\vbox{{\elevenpoint\advance\leftskip by 0.45truein\obeywhitespace\tt\catcode`^=\other
diff setstab.output setstab.correct
\vskip0pt}}
The other shell files work in an analogous manner.
\medbreak
The tests above may take a number of hours, depending on the speed of the
machine.  In addition, they require several megabytes of memory.  Each of
the shell files accepts an option {\tt -s}, which runs a much less time--taking
series of tests, requiring less memory.  When this option is used with
{\tt test\_setstab}, the output in file {\tt setstab.output} should be
compared to the file {\tt setstab-s.correct}, rather than to 
{\tt setstab.correct}.  A similar remark applies to the other tests.
\medbreak
The programs described in this manual are also available, in binary form,
for the IBM~PC and compatibles, under MS~DOS, and for the IBM~370
family of machines, under CMS.  For details, please contact the author.
%
%
\section{XI.\quad COMPILING THE SOURCE CODE}
%
As mentioned earlier, compiled versions of the programs described here are
available for the IBM~PC (DOS), the Sun/3 and Sun/4 (Unix), and the
IBM~370 (CMS).  For other systems, it will be necessary to compile the
C source code.  The information in this section is intended for users
intending to compile the source code.
\medbreak
To compile the C source code, a compiler that supports ANSI Standard C is
required.
With very minor exceptions, the source code conforms to the ANSI standard
for C; it does, however, make use of a number of features not present in
most pre--ANSI versions of C.  The code was written under the assumption
that it would be compiled with optimization turned on, so it is important
to enable optimization, as the programs tend to run quite slowly if
compiled with optimization disabled.\footnote{${}^{\dag}$}{\elevenpoint However, some compilers
fail to optimize the code correctly.  GNU~C~1.39 and~2.1 (Sun/3, Sun/4) and 
Waterloo~C 3.2 (IBM~370) optimize it correctly; at present, 
Borland~C++~3.0 and Microsoft~C~5.1 do not.  
Microsoft~C~6.0/7.0 and Borland~C++~3.1 have not been
tested.}  At present large sections of the source code
are not commented or are commented incorrectly.  Accordingly, the source
is provided primarily so that users may compile the programs on other
machines, rather than modify them.  Eventually the author hopes to
distribute adequately documented source code.
\medbreak
Prior to compiling the source, it is necessary to determine whether a
special timing function needs to be supplied.  By default the programs
use the C standard library function {\tt clock()} to measure execution
times.  This is adequate on many systems.  However, on some systems (e.g., 
Sun/3 and Sun/4 Unix), a problem arises because the {\tt clock()} function returns a 
result with a resolution of one microsecond and thus ``wraps around'' in about
36 minutes.  Unless the user is content with timing statistics correct modulo
$2^{32}$ microseconds (about 71.58 minutes), an alternate timing function
must be supplied in a file which should be named {\tt cputime.c}; 
this function must return a value of type {\tt long}
which represents the elapsed CPU time, and C macros 
{\tt CPU\_TIME} and {\tt TICK} must be
defined to specify the name of this special 
function (e.g., {\tt cpuTime}) and the resolution
of the function (in clock ticks per second), respectively.  
In addition, a make file macro {\tt CPUTIME} must be defined to have
the value {\tt cputime.o} (assuming object code files have suffix {\tt o}).
These macros are discussed in more detail below.
For the Sun/3 and Sun/4, source code for such a function 
{\tt cpuTime()} is supplied in the file {\tt cputime.c}; perhaps this function
will work on other Unix systems as well.  In the remainder of this section,
it is assumed that such a function, if needed, has already been written.
\medbreak
A make file (named {\tt Makefile}) is provided with the source.  This make file
is designed for the GNU~C compiler, version 2, on the Sun/3 and Sun/4, but with some
other compilers and systems, only simple editting of the first few lines of the make file
should be needed; the purpose of these lines is to define appropriate
make file macros.  For some compilers, more extensive editting of the
make file may be
required.  Note that the make file assumes that all source code and include
files (except system include files) are in the currect directory, and that
all object and executable files are to be placed in this directory.
\medbreak
The make file provided with the source begins as follows:
\medbreak
\vbox{{\elevenpoint\advance\leftskip by 0.45truein\obeywhitespace\tt\catcode`^=\other
COMPILE = gcc
DEFINES = -DSUN\_UNIX\_GCC -DINT\_SIZE=32 -DCPU\_TIME=cpuTime -DTICK=1000 
          -DLONG\_EXTERNAL\_NAMES -Dclock\_t=long 
INCLUDES =
COMPOPT = -c -O2 -Wall  
LINKNAME = -o
LINKOPT = -v
OBJ = o
CPUTIME = cputime.o
BOUNDS =
}}
\medbreak
The purpose of the make file macros defined here is as follows:
\smallbreak
{\advance\leftskip by 0.3in\noindent
   {\tt COMPILE}:\quad  This macro specifies the command to invoke the C compiler;
                   it may include path information.
\smallbreak
   {\tt DEFINES}:\quad  This make file macro specifies C macros to be defined
                        for the compilation; these are discussed below.  
\smallbreak
   {\tt INCLUDES}:\quad  This macro is used to tell the compiler where
                         to search for include files, if they are located
                         other than in the standard location.  
\smallbreak
   {\tt COMPOPT}:\quad  This macro is used to specify compile--time options,
                        other than those given by means of the 
                        {\tt DEFINES} and {\tt INCLUDES} macros.  These
                        options should include code optimization and
                        compile--only (no linking) if these are not
                        defaults.  For the IBM~PC, an option specifying the
                        large memory model should be given. 
\smallbreak
   {\tt LINKNAME}:\quad  This macro is used to specify the linker option
                        used to give the name of the executable file
                        to be created.
\smallbreak
   {\tt LINKOPT}:\quad  This macro is used to specify linker options, other
                        than that given by the {\tt LINKNAME} macro above.
\smallbreak
   {\tt OBJ}:\quad   This macro is used to specify the suffix (file type)
                     for object files.  Normally this would be {\tt o}
                     on Unix and {\tt obj} on MS~DOS.
\smallbreak
   {\tt CPUTIME}:\quad  Unless a special timing function is to be supplied
                     (see discussion above), this value of this macro should
                      be the null string.  If a special timing function is
                      supplied, the value of the {\tt CPUTIME} macro should 
                      be {\tt cputime.o}, assuming object files have suffix
                      {\tt o}.
\smallbreak
   {\tt BOUNDS}:\quad  This macro is present for use on MS~DOS with Bounds Checker,
                     a debugging product published by Nu-Mega Technologies
                     and used heavily by the author in debugging the programs.
                     Normally its value should be the null string.
\medbreak}
It remains to discuss the C language macros that must, or may, be defined
by the make macro {\tt DEFINES}.  One of these C macros, {\tt INT\_SIZE}, always
must be defined; the others are optional, or are required only in
certain circumstances.  The C language macros are as follows:
\smallbreak
{\advance\leftskip by 0.3in\noindent
    {\tt INT\_SIZE}:\quad  This macro is always required.  Its value must
             be the number of bits in the C data type {\tt int}, usually
             16 or 32.  (The programs have never been adapted to a system
             in which the integer size is other than 16 or 32, although
             it should not be difficult to do so.)
             {\it Note:\/}\enskip This macro only specifies the size of the
             C data type {\tt int}; it does {\it not\/} determine whether
             the programs are compiled to use 16 or 32 bit integers.  
\smallbreak
   {\tt EXTRA\_LARGE}:\quad If this macro is defined, and {\tt INT\_SIZE}
             above is 32, the programs will be compiled using 32--bit integers;
             that is, points will be represented using the C data type
             {\tt unsigned}.  Otherwise, if {\tt INT\_SIZE} is 32, the programs
             will be compiled using 16--bit integers (with most compilers),
             as points will be represented using the C data type
             {\tt unsigned short}.  Note {\tt EXTRA\_LARGE} should not be
             defined when {\tt INT\_SIZE} is 16.  Care must be taken, of
             course, not to link object files compiled with {\tt EXTRA\_LARGE}
             defined with those compiled without it defined, as the make file
             cannot protect against this error.  (However, running the programs
             with the {\tt -v} option will reveal this error.)
         
\smallbreak
   {\tt LONG\_EXTERNAL\_NAMES}:\quad   This macro should (but need not) be defined
             if the linker supports long external names (up to 31 characters).
             If defined, its value is irrelevent.
\smallbreak
   {\tt CPU\_TIME}:\quad  This macro must be defined if a special timing function
             is supplied, and its value must be the name of that function.  
             If the standard C function {\tt clock()} is to be used, the
             macro must not be defined (not even as the null string).  
\smallbreak
   {\tt NOFLOAT}:\quad  This macro probably should be defined on a system 
              lacking floating point hardware support.  (If defined, its
              value is irrelevant.)  Normally, the programs perform a 
              small amount of floating point arithmetic in attempting to
              produce a good $\bfgermR$--base; with software emulation of floating
              point instructions, the cost of this use of floating point
              arithmetic probably exceeds its benefit.  If the 
              {\tt NOFLOAT} macro is defined, no floating point arithmetic
              is used.  (In this case, it may be advantageous to add a
              compiler option telling the compiler not to incorporate
              floating--point support into the executable program.  For
              example, under Borland C++, the {\tt -f-} option has this
              effect.)
\smallbreak
   {\tt TICK}:\quad  This macro should be defined in two situations: (1)
                If a special CPU time function is supplied, {\tt TICK}
                should be defined to be the resolution (in clock ticks per
                second) of that function, and (2) if the
                {\tt NOFLOAT} macro is defined and if the compiler--supplied
                definition of the standard C macro {\tt CLK\_TCK} is a
                floating point constant (as occurs with Borland~C++), 
                {\tt TICK} should be defined to be the nearest integer 
                approximation to the value of {\tt CLK\_TCK}, e.g., 18 for
                Borland~C++.
\smallbreak
   {\tt HUGE}:\quad This macro must be defined for the IBM PC (unless a
               DOS extender is being used).  It simply causes a few pointers
               to be declared as {\tt huge}.  Only one program, the
               weight distribution program, uses huge pointers.
\medbreak}
In addition, the author recommends defining a symbol unique to the system
and compiler, e.g., {\tt SUN\_UNIX\_GCC} above.  Then any code modifications
unique to this system and compiler can be bracketed as follows:
\medbreak
\vbox{{\elevenpoint\advance\leftskip by 0.45truein\obeywhitespace\tt\catcode`^=\other
\#ifdef SUN\_UNIX\_GCC
   /* Code modifications for Sun Unix, GNU C compiler. */
\#endif
}}
\medbreak
Once the make file has been editted appropriately, all of the programs
may be compiled at once by the command
\smallskip
\hskip0.45truein\relax {\tt make all}
\smallskip
or, alternatively, individual programs may be compiled by the commands
\smallskip
\hskip0.45truein\relax \hbox to1.5truein{\tt make setstab\hfil}for
                    {\tt setstab}, {\tt setimage}, {\tt parstab}, and {\tt parimage},
\vskip1pt\noindent
\hskip0.45truein\relax \hbox to1.5truein{\tt make cent\hfil}for 
                    {\tt cent}, {\tt conj}, {\tt gcent},
\vskip1pt\noindent
\hskip0.45truein\relax \hbox to1.5truein{\tt make inter\hfil}for
                    {\tt inter},
\vskip1pt\noindent
\hskip0.45truein\relax \hbox to1.5truein{\tt make desauto\hfil}for
                    {\tt desauto}, {\tt desiso}, {\tt matauto}, {\tt matiso},
                    {\tt codeauto}, and {\tt codeiso},
\vskip1pt\noindent
\hskip0.45truein\relax \hbox to1.5truein{\tt make cjrndper\hfil}for
                    {\tt cjrndper} and {\tt cjper},
\vskip1pt\noindent
\hskip0.45truein\relax \hbox to1.5truein{\tt make commut\hfil}for
                    {\tt commut} and {\tt ncl},
\vskip1pt\noindent
\hskip0.45truein\relax \hbox to1.5truein{\tt make compgrp\hfil}for
                    {\tt compgrp},
\vskip1pt\noindent
\hskip0.45truein\relax \hbox to1.5truein{\tt make fndelt\hfil}for
                    {\tt fndelt},
\vskip1pt\noindent
\hskip0.45truein\relax \hbox to1.5truein{\tt make generate\hfil}for
                    {\tt generate},
\vskip1pt\noindent
\hskip0.45truein\relax \hbox to1.5truein{\tt make orblist\hfil}for
                    {\tt orblist}, {\tt chbase}, and {\tt ptstab},
\vskip1pt\noindent
\hskip0.45truein\relax \hbox to1.5truein{\tt make randobj\hfil}for
                    {\tt randobj},
\vskip1pt\noindent
\hskip0.45truein\relax \hbox to1.5truein{\tt make wtdist\hfil}for
                    {\tt wtdist} and {\tt cwtdist}.
\smallskip
Ideally, there should be no errors in compilation.  However, with some
compilers (e.g., Zortech~C++), it is necessary to define {\tt const} to
be a null string in order to avoid compile--time errors.  Typically
compilers will generate a number of warnings; a number of them involve
implicit conversion between pointers to constant and non--constant types.
\medbreak
The above commands place the executable programs in the current directory
(that containing the source).   The final step is to move the executables
to the directory in which they should reside, preferably one on the current
path.  A shell file named {\tt install} is provided for this purpose.
The command
\vskip2pt
\hskip0.45in\relax {\tt sh install\ }{\it bindir}
\vskip2pt
should be issued, where {\it bindir\/} is the name of the directory in
which the binary files should be placed, e.g., {\tt ../bin}.  Note that
this command also copies certain shell files from the source directory
to the binary directory, renaming them by dropping the {\tt sh} suffix
in the process.
%
\bigbreak
\vskip15pt
\centerline{\sectionheaderfont REFERENCES}
\nobreak\medskip\nobreak
Leon, J.\enskip (1980a),\enskip
On an algorithm for finding a base and a strong generating set
for a group given by generating permutations,\enskip
{\it Math.\ Comp.}
{\bf 35},
941--974.
\medbreak
Leon, J.\enskip(1991),\enskip
Permutation group algorithms based on partitions, I: Theory and algorithms,\enskip
{\it J. Symbolic Comp.}
{\bf 12},
533--583.
\medbreak
Cannon, J.\enskip (1984),\enskip
{\it A language for group theory,}\enskip
Dept.\ of Pure Mathematics, University of Sydney, Australia.
\medbreak
Sims, C.~C.\enskip (1971),\enskip
Computation with permutation groups,\enskip
In (Petrick, S.~R., ed.),\enskip
{\it Proc.\ of the Second Symposium on Symbolic and
Algebraic Manipulation,}\enskip
New York: Assoc.\ for Computing Mach.
\bye
