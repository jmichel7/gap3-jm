\psection{Installing \Specht}

When you unpack \Specht\ you will find the following files in a 
directory called specht-\VersionNo\:

|README       -this file
doc/         -Specht documentation (see below)
gap/         -|\GAP| source
init.g       -initialization file
lib/         -Specht library files |

Ideally, \Specht\ should be installed in the \GAP\ packages directory (in 
a subdirectory called ``specht\'\'); however, it can be installed anywhere. 
If \Specht\ is not installed in the \GAP\ packages directory then include a
line of the form

|Add(PKGNAME, "/path/to/directory/containing/specht/"); 
PKGNAME:=Reversed(PKGNAME);|

in your .gaprc in your home directory (create such a file with this line if
you do not already have one). \Specht\ is now ready to use.

|gap> RequirePackage("specht");
gap> H:=Specht(3);
Specht(e=3, S(), P(), D(), Pq())|

\psection{Installing \Specht s online documentation}


The documentation for Specht can be found in the subdirectory 'doc'. The
more important files in this directory are\:

|specht.tex  -|\LaTeX| source for the manual
specht.html -an HTML version of the manual
manual.tex  -header file for |<specht.tex>| (a |\LaTeX2e| file) 
manual.dvi  -a dvi file for |\Specht|
install.tex -these installation notes|

To install the *online* documentation for \Specht\ proceed as 
follows\:

|1. Copy |<specht.tex>| into the |\GAP\ <doc/>| directory, 
2. Change directory to |\GAP|s <doc/> directory and add the line 
   |<$\backslash$Include\{specht\}>| to |<manual.tex>|.
3. |\LaTeX\ <manual.tex>|.
4. Run |'makeindex'| (if available) and repeat 3. |

A copy of just the \Specht\ manual can be obtained by printing the dvi file
<manual.dvi> (or \LaTeX2eing the file <manual.tex>) in \Specht s <doc/> 
directory.

\endinput
