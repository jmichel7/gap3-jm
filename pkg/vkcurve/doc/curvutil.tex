%%%%%%%%%%%%%%%%%%%%%%%%%%%%%%%%%%%%%%%%%%%%%%%%%%%%%%%%%%%%%%%%%%%%%%%%%
%%
%A  curvutil.tex      VKCURVE documentation    David Bessis,  Jean Michel
%%
%Y  Copyright (C) 2001-2002  University  Paris VII.
%%
%%  This  file  introduces the VKCURVE package.
%%
%%%%%%%%%%%%%%%%%%%%%%%%%%%%%%%%%%%%%%%%%%%%%%%%%%%%%%%%%%%%%%%%%%%%%%%%%
\def\VKCURVE{{\sf VKCURVE}}
\def\CHEVIE{{\sf CHEVIE}}
\Chapter{Some VKCURVE utility functions}

We document here various utility functions defined by \VKCURVE\ package and
which may be useful also in other contexts.

%%%%%%%%%%%%%%%%%%%%%%%%%%%%%%%%%%%%%%%%%%%%%%%%%%%%%%%%%%%%%%%%%%%%%%%%%
\Section{BigNorm}
\index{BigNorm}

'BigNorm(<c>)'

Given  a 'complex' number  <c> with real  part <r> and  imaginary part <j>,
returns a \"cheap substitute\" to the norm of <c> given by $|r|+|j|$.

|    gap> BigNorm(Complex(-1,-1));
    2|

%%%%%%%%%%%%%%%%%%%%%%%%%%%%%%%%%%%%%%%%%%%%%%%%%%%%%%%%%%%%%%%%%%%%%%%%%
\Section{DecimalLog}
\index{DecimalLog}

'DecimalLog(<r>)'

Given  a rational number <r>, returns an  integer <k> such that $10^k\< |r|
\le 10^{k+1}$.

|    gap> List([1,1/10,1/2,2,10],DecimalLog);
    [ -1, -2, -1, 0, 1 ]|

%%%%%%%%%%%%%%%%%%%%%%%%%%%%%%%%%%%%%%%%%%%%%%%%%%%%%%%%%%%%%%%%%%%%%%%%%
\Section{ComplexRational}
\index{ComplexRational}

'ComplexRational(<c>)'

<c> is a cyclotomic or a 'Complex' number with 'Decimal' or real cyclotomic
real  and imaginary parts. This function returns the corresponding rational
complex number.

|    gap> evalf(E(3)/3);
    -0.1666666667+0.2886751346I
    gap> ComplexRational(last);
    -16666666667/100000000000+28867513459/100000000000I
    gap> ComplexRational(E(3)/3);
    -1/6+28867513457/100000000000I|

%%%%%%%%%%%%%%%%%%%%%%%%%%%%%%%%%%%%%%%%%%%%%%%%%%%%%%%%%%%%%%%%%%%%%%%%%
\Section{Dispersal}
\index{Dispersal}

'Dispersal(v)'

<v>  is a list of 'complex' numbers  representing points in the real plane.
The  result is a pair  whose first element is  the minimum distance between
two  elements of <v>, and the second is a pair of indices '[i,j]' such that
'v[i]', 'v[j]' achieves this minimum distance.

|    gap> Dispersal([Complex(1,1),Complex(0),Complex(1)]);
    [ 1, [ 1, 3 ] ]|

%%%%%%%%%%%%%%%%%%%%%%%%%%%%%%%%%%%%%%%%%%%%%%%%%%%%%%%%%%%%%%%%%%%%%%%%%
\Section{ConjugatePresentation}
\index{ConjugatePresentation}

'ConjugatePresentation(<p> ,<conjugation>)'

This program modifies a presentation by conjugating a generator by another.
The  conjugation to  apply is  described by  a length-3  string of the same
style  as  the  result  of  'DisplayPresentation',  that is '\"abA\"' means
replace  the second generator by its  conjugate by the first, and '\"Aba\"'
means replace it by its conjugate by the inverse of the first.

|    gap> F:=FreeGroup(4);;
    gap> p:=PresentationFpGroup(F/[F.4*F.1*F.2*F.3*F.4*F.1^-1*F.4^-1*
    > F.3^-1*F.2^-1*F.1^-1,F.4*F.1*F.2*F.3*F.4*F.2*F.1^-1*F.4^-1*F.3^-1*
    > F.2^-1*F.1^-1*F.3^-1,F.2*F.3*F.4*F.1*F.2*F.3*F.4*F.3^-1*F.2^-1*
    > F.4^-1*F.3^-1*F.2^-1*F.1^-1*F.4^-1]);
    gap> DisplayPresentation(p);
    1: dabcd=abcda
    2: dabcdb=cabcda
    3: bcdabcd=dabcdbc
    gap> DisplayPresentation(ConjugatePresentation(p,"cdC"));
    #I  there are 4 generators and 3 relators of total length 36
    1: cabdca=dcabdc
    2: dcabdc=bdcabd
    3: cabdca=abdcab|

%%%%%%%%%%%%%%%%%%%%%%%%%%%%%%%%%%%%%%%%%%%%%%%%%%%%%%%%%%%%%%%%%%%%%%%%%
\Section{TryConjugatePresentation}
\index{TryConjugatePresentation}

'TryConjugatePresentation(<p> [,<goal> [,<printlevel>]])'

This program tries to simplify group presentations by applying conjugations
to  the  generators.  The  algorithm  depends  on  random  numbers,  and on
tree-searching,  so is  not reproducible.  By default  the program stops as
soon  as a shorter presentation is found.  Sometimes this does not give the
desired  presentation.  One  can  give  a  second argument <goal>, then the
program  will only stop when  a presentation of length  less than <goal> is
found.  Finally, a third  argument can be  given and then all presentations
the  programs runs  over which  are of  length less  than or  equal to this
argument are displayed. Due to the non-deterministic nature of the program,
it  may be useful to  run it several times  on the same input. Upon failure
(to improve the presentation), the program returns <p>.

|    gap> Display(p);
    1: ba=ab
    2: dbd=bdb
    3: cac=aca
    4: bcb=cbc
    5: dAca=Acad
    6: dcdc=cdcd
    7: adad=dada
    8: dcDbdc=bdcbdB
    9: dcdadc=adcdad
    10: adcDad=dcDadc
    11: BcccbdcAb=dcbACdddc
    gap> p:=TryConjugatePresentation(p); 
    #I  there are 4 generators and 11 relators of total length 100
    #I  there are 4 generators and 11 relators of total length 120
    #I  there are 4 generators and 10 relators of total length 100
    #I  there are 4 generators and 11 relators of total length 132
    #I  there are 4 generators and 11 relators of total length 114
    #I  there are 4 generators and 11 relators of total length 110
    #I  there are 4 generators and 11 relators of total length 104
    #I  there are 4 generators and 11 relators of total length 114
    #I  there are 4 generators and 11 relators of total length 110
    #I  there are 4 generators and 11 relators of total length 104
    #I  there are 4 generators and 8 relators of total length 76
    #I  there are 4 generators and 8 relators of total length 74
    #I  there are 4 generators and 8 relators of total length 72
    #I  there are 4 generators and 8 relators of total length 70
    #I  there are 4 generators and 7 relators of total length 52
    # d-$>$adA gives length 52
    << presentation with 4 gens and 7 rels of total length 52 >>
    gap> Display(p); 
    1: ba=ab
    2: dc=cd
    3: aca=cac
    4: dbd=bdb
    5: bcb=cbc
    6: adad=dada
    7: aBcADbdac=dBCacbdaB
    gap> TryConjugatePresentation(p,48);
    #I  there are 4 generators and 7 relators of total length 54
    #I  there are 4 generators and 7 relators of total length 54
    #I  there are 4 generators and 7 relators of total length 60
    #I  there are 4 generators and 7 relators of total length 60
    #I  there are 4 generators and 7 relators of total length 48
    # d-$>$bdB gives length 48
    << presentation with 4 gens and 7 rels of total length 48 >>
    gap> Display(last);
    1: ba=ab
    2: bcb=cbc
    3: cac=aca
    4: dbd=bdb
    5: cdc=dcd
    6: adad=dada
    7: dAbcBa=bAcBad|

%%%%%%%%%%%%%%%%%%%%%%%%%%%%%%%%%%%%%%%%%%%%%%%%%%%%%%%%%%%%%%%%%%%%%%%%%
\Section{FindRoots}
\index{FindRoots}

'FindRoots(<p>, <approx>)'

<p>  should be a univariate 'Mvp'  with cyclotomic or 'Complex' rational or
decimal  coefficients or  a list  of cyclotomics  or 'Complex' rationals or
decimals  which represents  the coefficients  of a  complex polynomial. The
function  returns  'Complex'  rational  approximations  to the roots of <p>
which  are  better  than  <approx>  (a  positive rational). Contrary to the
functions  'SeparateRoots', etc... described in  the previous chapter, this
function handles quite well polynomials with multiple roots. We rely on the
algorithms explained in detail in \cite{HSS01}.

|    gap> FindRoots((x-1)^5,1/100000000000);
    [ 6249999999993/6250000000000+29/12500000000000I, 
      12499999999993/12500000000000-39/12500000000000I, 
      12500000000023/12500000000000+11/6250000000000I, 
      12500000000023/12500000000000+11/6250000000000I, 
      312499999999/312500000000-3/6250000000000I ]
    gap> evalf(last);
    [ 1, 1, 1, 1, 1 ]
    gap> FindRoots(x^3-1,1/10);            
    [ -1/2-108253175473/125000000000I, 1, -1/2+108253175473/125000000000I ]
    gap> evalf(last);
    [ -0.5-0.8660254038I, 1, -0.5+0.8660254038I ]
    gap> List(last,x->x^3);
    [ 1, 1, 1 ]|

%%%%%%%%%%%%%%%%%%%%%%%%%%%%%%%%%%%%%%%%%%%%%%%%%%%%%%%%%%%%%%%%%%%%%%%%%
\Section{Cut}
\index{Cut}

'Cut(<string s> [, opt])'

The  first argument is a string, and the second one a record of options, if
not  given taken equal to 'rec()'. This function prints its string argument
<s> on several lines not exceeding <opt.width>; if not given <opt.width> is
taken  to be equal  'SizeScreen[1]-2'. This is  similar to how \GAP\ prints
strings, excepted no continuation line characters are printed. The user can
specify  after  which  characters,  or  before  which characters to cut the
string  by  giving  fields  'opt.before'  and  'opt.after';  the  defaut is
'opt.after=\",\"', but some other characters can be used --- for instance a
good  choice for printing big polynomials could be 'opt.before:=\"+-\"'. If
a field 'opt.file' is given, the result is appended to that file instead of
written  to standard output;  this may be  quite useful in conjunction with
'FormatGAP'for dumping some \GAP\ values to a file for later re-reading.

|    gap> Cut("an, example, with, plenty, of, commas\n",rec(width:=10));
    an,
    example,
    with,
    plenty,
    of,
    commas
    gap>|

%%%%%%%%%%%%%%%%%%%%%%%%%%%%%%%%%%%%%%%%%%%%%%%%%%%%%%%%%%%%%%%%%%%%%%%%%
