%%%%%%%%%%%%%%%%%%%%%%%%%%%%%%%%%%%%%%%%%%%%%%%%%%%%%%%%%%%%%%%%%%%%%%%%%%%%%
%%
%A  xmmanual.tex  for the XMOD package                       version 13/ 1/97
%A
%A  GAP documentation                               Chris Wensley & Murat Alp
%%
%%  This file is a separate header LaTeX file for the XMOD package.
%%  It should be used when producing the XMOD chapter on its own
%%  using:    ...> latex xmmanual
%%  It contains the GAP macros from the main  manual.tex  file.
%%  To print it, you must format it with \LaTeX\ document preparation system.
%%  This file must be accompanied by the files that contain the chapters.
%%
%%  Instructions taken from the full GAP manual:
%%  It should also be acommpanied  by  'manual.toc'  (the table of contents),
%%  'manual.bbl' (bibliography),  'manual.bib' (source for the bibliography),
%%  'manual.ind' (the index), and 'manual.idx' (the source for 'manual.ind').
%%  If they are missing you can  reproduce  them  again  with {\LaTeX}.  Read
%%  'install.tex' for the necessary steps.
%%
%%  Note that the file 'manual.idx' is also used by the on-screen help.
%%

\documentstyle{book}

%%%%%%%%%%%%%%%%%%%%%%%%%%%%%%%%%%%%%%%%%%%%%%%%%%%%%%%%%%%%%%%%%%%%%%%%%%%%%
%%
%%  With '\includeonly{<chapters>}' you can specify that you want  only  some
%%  chapters to be printed.  If all  the  '.aux'  files  are  there, chapter-
%%  section- and page-numbers will all be correct.
%%
%\includeonly{}


%%%%%%%%%%%%%%%%%%%%%%%%%%%%%%%%%%%%%%%%%%%%%%%%%%%%%%%%%%%%%%%%%%%%%%%%%%%%%
%%
%%  The following commands intructs {\LaTeX} to stuff more on each  page  and
%%  to move each page towards to outer border.
%%
\topmargin 0 pt
\textheight 47\baselineskip
\advance\textheight by \topskip
\oddsidemargin  0.5 in
\evensidemargin  .25in
\textwidth 5.5in


%%%%%%%%%%%%%%%%%%%%%%%%%%%%%%%%%%%%%%%%%%%%%%%%%%%%%%%%%%%%%%%%%%%%%%%%%%%%%
%%
%%  The following commands instruct  {\LaTeX}  to  separate the paragraphs in
%%  this manual with a small space and to leave them unindented.
%%
\parskip 1.0ex plus 0.5ex minus 0.5ex
\parindent 0pt


%%%%%%%%%%%%%%%%%%%%%%%%%%%%%%%%%%%%%%%%%%%%%%%%%%%%%%%%%%%%%%%%%%%%%%%%%%%%%
%%
%%  'text'
%%
%%  'text' prints the text in  monospaced  typewriter  font  in  the  printed
%%  manual  and  is  displayed  unchanged  in  the  on-screen  documentation.
%%  It should be used for names of GAP variables and functions and other text
%%  that the user may actually enter into his computer and see on his screen.
%%  The text may contain all the usual characters  and |<name>| placeholders.
%%  |\'| can be used to enter a single  quote  character  into  the  text.
%%
\catcode`\'=13 \gdef'#1'{{\tt #1}}
\gdef\'{\char`'}


%%%%%%%%%%%%%%%%%%%%%%%%%%%%%%%%%%%%%%%%%%%%%%%%%%%%%%%%%%%%%%%%%%%%%%%%%%%%%
%%
%%  <text>
%%
%%  <text> prints  the text in  an italics font  in the printed manual and is
%%  displayed  unchanged in the  on-screen  documentation.  It should be used
%%  for arguments  in description of  GAP functions and   other placeholders.
%%  The  text should not contain any special characters.  |\<| can be used to
%%  enter a less than character into the text.
%%
\catcode`\<=13 \gdef<#1>{{\it #1\/}}
\gdef\<{\char`<}


%%%%%%%%%%%%%%%%%%%%%%%%%%%%%%%%%%%%%%%%%%%%%%%%%%%%%%%%%%%%%%%%%%%%%%%%%%%%%
%%
%%  *text*
%%
%%  *text*  prints the text  in boldface font   in the printed manual and  is
%%  displayed unchanged in the on-screen  documentation.  It  should  be used
%%  for the definition  of mathematical  keywords.   The text may contain all
%%  the usual characters.  |\*| can be used to enter a star into the text.
%%
\catcode`\*=13 \gdef*#1*{{\bf #1}}
\gdef\*{\char`*}
\gdef\^{\char`^}


%%%%%%%%%%%%%%%%%%%%%%%%%%%%%%%%%%%%%%%%%%%%%%%%%%%%%%%%%%%%%%%%%%%%%%%%%%%%%
%%
%%  |text|
%%
%%  |text| prints the text between the two  pipe  symbols in typewriter style
%%  obeying the   linebreaks and spaces  in  the   manual.   In the on-screen
%%  documentation it  remains unchanged, except  that the pipes are stripped.
%%  It should be used to  enter lengthy examples  into the text.  If the hash
%%  character '\#' appears in the example the text between it and  the end of
%%  the line  is set in  ordinary mode,  i.e., in  roman   font with  all the
%%  possibilities ordinary available.  |\|'\|' can be used to  enter  a  pipe
%%  symbol into the text.
%%
\catcode`\@=11

{\catcode`\ =\active\gdef\xvobeyspaces{\catcode`\ \active\let \xobeysp}}
\def\xobeysp{\leavevmode{} }

\catcode`\|=13
\gdef|{\leavevmode{}\hbox{}\begingroup
\def|{\endgroup}%
\catcode`\\=12\catcode`\{=12\catcode`\}=12\catcode`\$=12\catcode`\&=13
\catcode`\#=13\catcode`\^=12\catcode`\_=12\catcode`\ =12\catcode`\%=12
\catcode`\~=12\catcode`\'=12\catcode`\<=12\catcode`\"=12\catcode`\|=13
\catcode`\*=12\catcode`\:=12
\leftskip\@totalleftmargin\rightskip\z@
\parindent\z@\parfillskip\@flushglue\parskip\z@
\@tempswafalse\def\par{\if@tempswa\hbox{}\fi\@tempswatrue\@@par}%
\tt\obeylines\frenchspacing\xvobeyspaces\samepage}

\catcode`\@=12

{\catcode`\#=13
\gdef#{\begingroup
\catcode`\\=0 \catcode`\{=1 \catcode`\}=2 \catcode`\$=3 \catcode`\&=4
\catcode`\#=6 \catcode`\^=7 \catcode`\_=8 \catcode`\ =10\catcode`\%=14
\catcode`\~=13\catcode`\'=13\catcode`\<=13\catcode`\"=13\catcode`\|=13
\catcode`\*=13\catcode`\:=13
\catcode`\^^M=12 \Comment}}
{\catcode`\^^M=12
\gdef\Comment#1^^M{\rm \# #1 \endgroup \Newline}}
{\obeylines
\gdef\Newline{
}}

{\catcode`\&=13
\gdef&{\#}}

\gdef\|{\char`|}


%%%%%%%%%%%%%%%%%%%%%%%%%%%%%%%%%%%%%%%%%%%%%%%%%%%%%%%%%%%%%%%%%%%%%%%%%%%%%
%%
%%  <item>: <text>
%%
%%  This formats the  paragraph  <text>, i.e.,  everything between  the colon
%%  '\:' and the next  empty  line, indented 1 cm to the right in the printed
%%  manual and is displayed  unchanged in the  on-screen documentation.  This
%%  convention should be  used to format  a list or an enumeration.    <item>
%%  should be  a single  word  or a short phrase.  It  may contain  all usual
%%  characters and the usual formatting stuff.  <text> is a  normal paragraph
%%  and may contain everything.   \:  can be used  to enter a colon character
%%  into the text.  As  an example consider the  following description.  This
%%  will print quite similar in the printed manual.
%%
%%      A group is represented by a record that must have the components
%%
%%      'generators': \\
%%              a list of group elements that  generate  the  group  that  is
%%              given by the group record.
%%
%%      'identity': \\
%%              the identity element of the group that is given by the  group
%%              record.
%%
\catcode`\:=13
\gdef:{\hangafter=1\hangindent=1cm\hspace{1cm}{}}
\gdef\:{\char`:}


%%%%%%%%%%%%%%%%%%%%%%%%%%%%%%%%%%%%%%%%%%%%%%%%%%%%%%%%%%%%%%%%%%%%%%%%%%%%%
%%
%%  "reference"
%%
%%  "reference" prints the  number of the  chapter or section  in the printed
%%  manual and is  displayed unchanged  in the  on-screen  documentation.  It
%%  should be used when referring to other  chapters or  sections.   The text
%%  should  not contain any special characters.  \"  can be  used  to enter a
%%  double quote into the text.
%%
\catcode`\"=13 \gdef"#1"{\ref{#1}}

\gdef\"{\char`"}


%%%%%%%%%%%%%%%%%%%%%%%%%%%%%%%%%%%%%%%%%%%%%%%%%%%%%%%%%%%%%%%%%%%%%%%%%%%%%
%%
%%  \GAP
%%
%%  \GAP can be used to enter the *sans serif* GAP logo  into  the  text.  If
%%  this is followed by spaces it should be enclosed in curly  braces  as  in
%%  |{\GAP}| is wonderful.
%%
\newcommand{\GAP}{{\sf GAP}}
\newcommand{\CAS}{{\sf CAS}}
\newcommand{\ATLAS}{{\sf ATLAS}}
\newcommand{\Z}{\mbox{$Z\!\!\! Z$}}
\newcommand{\Q}{\mbox{$Q\mskip-11mu\prime\,\,$}}

%%%%%%%%%%%%%%%%%%%%%%%%%%%%%%%%%%%%%%%%%%%%%%%%%%%%%%%%%%%%%%%%%%%%%%%%%%%%%
%%
%%  \Include{<filename>}
%%
%%  |\Include| instructs \LaTeX\ to include the file with the name <filename>
%%  into a document.  It works basically like |\include| except that it  also
%%  writes a comment line to the index file where the on-screen help function
%%  may use it.
%%
\catcode`\@=11 \catcode`\%=12 \catcode`\~=14
\newcommand{\Include}[1]{\write\@indexfile{% #1.tex}\include{#1}}
\catcode`\@=12 \catcode`\%=14 \catcode`\~=13


%%%%%%%%%%%%%%%%%%%%%%%%%%%%%%%%%%%%%%%%%%%%%%%%%%%%%%%%%%%%%%%%%%%%%%%%%%%%%
%%
%%  \Chapter{<name>}
%%  \Section{<name>}
%%
%%  |\Chapter| and |\Section| begin a  new  chapter  or  section.  They  work
%%  basically like the ordinary |\chapter| and |\section| macros except  that
%%  they also create a label for <name>   and write a  comment  line  to  the
%%  index file where the on-screen help function may use it.
%%
\catcode`\@=11 \catcode`\%=12 \catcode`\~=14

\newcommand{\Chapter}[1]{{\chapter{#1}~
\label{#1}~
\write\@indexfile{% 1}\index{#1}}}

\newcommand{\Section}[1]{{\pagebreak[3]\section{#1}~
\label{#1}~
\write\@indexfile{% 2}\index{#1}}}

\catcode`\@=12 \catcode`\%=14 \catcode`\~=13


%%%%%%%%%%%%%%%%%%%%%%%%%%%%%%%%%%%%%%%%%%%%%%%%%%%%%%%%%%%%%%%%%%%%%%%%%%%%%
%%
%%  tell LaTeX to prepare an index
%%
\makeindex


%%%%%%%%%%%%%%%%%%%%%%%%%%%%%%%%%%%%%%%%%%%%%%%%%%%%%%%%%%%%%%%%%%%%%%%%%%%%%
%%
%%  make the title page
%%
\begin{document}

\title{
{\Huge XMOD}  \\
  \mbox{}     \\
Crossed Modules and Cat1-Groups in GAP\\
  \mbox{}     \\
Version 1.3
}
\author{Chris Wensley \& Murat Alp\\
School of Mathematics,\\
University of Wales, Bangor,\\
Gwynedd, LL57 1UT, U.K.}
\date{13th January 1997}

\maketitle


%%%%%%%%%%%%%%%%%%%%%%%%%%%%%%%%%%%%%%%%%%%%%%%%%%%%%%%%%%%%%%%%%%%%%%%%%%%%%
%%
%%  include the copyright notice
%%
%% %%%%%%%%%%%%%%%%%%%%%%%%%%%%%%%%%%%%%%%%%%%%%%%%%%%%%%%%%%%%%%%%%%%%%%%%%%%%%
%%
%A  copyrigh.tex                GAP documentation            Martin Schoenert
%%
%A  @(#)$Id: copyrigh.tex,v 1.1.1.1 1996/12/11 12:36:44 werner Exp $
%%
%Y  Copyright 1990-1992,  Lehrstuhl D fuer Mathematik,  RWTH Aachen,  Germany
%%
%%  This file contains the GAP copyright
%%
%H  $Log: copyrigh.tex,v $
%H  Revision 1.1.1.1  1996/12/11 12:36:44  werner
%H  Preparing 3.4.4 for release
%H
%H  Revision 3.3  1992/04/05  23:41:40  martin
%H  made a minor modification
%H
%H  Revision 3.2  1992/03/31  12:50:36  martin
%H  fixed several typos
%H
%H  Revision 3.1  1992/03/31  08:22:32  martin
%H  initial revision under RCS
%H
%%
\thispagestyle{empty}

{\large Copyright {\copyright} 1992 by Lehrstuhl D f{\accent127u}r Mathematik}

RWTH, Templergraben 64, D 5100 Aachen, Germany

{\GAP}  can  be  copied  and  distributed  freely for any  non-commercial
purpose.

If you copy {\GAP} for somebody else, you may ask this person for  refund
of your expenses.  This should cover cost of media, copying and shipping.
You are not allowed to ask for more than this.  In any case you must give
a copy of this copyright notice along with the program.

If you obtain {\GAP} please send us  a short notice to that effect, e.g.,
an  e-mail  message   to  the  address  'gap@samson.math.rwth-aachen.de',
containing your full  name and address.  This  allows us to keep track of
the number of {\GAP} users.

If you  publish  a mathematical result  that  was  partly obtained  using
{\GAP}, please cite {\GAP}, just as you would cite another paper that you
used.   Also we  would appreciate it if you could inform us about  such a
paper.

You  are permitted  to modify and  redistribute  {\GAP},  but you are not
allowed  to restrict further redistribution.  That is to say  proprietary
modifications will  not  be allowed.  We want all  versions  of {\GAP} to
remain free.

If you  modify any part of {\GAP} and redistribute it,  you must supply a
'README'  document.   This should specify what modifications you made  in
which  files.  We do  not  want  to take  credit  or  be blamed  for your
modifications.

Of course we are interested in all of your modifications.  In  particular
we would like to see bug-fixes, improvements and new functions.  So again
we would appreciate it if you would inform us about all modifications you
make.

{\GAP} is distributed by us without any warranty, to the extent permitted
by applicable state law.  We  distribute {\GAP} *as is* without  warranty
of any kind, either expressed or implied, including,  but not limited to,
the implied  warranties  of merchantability and  fitness for a particular
purpose.

The entire risk as to the quality and performance of the program is  with
you.  Should {\GAP} prove defective, you assume the cost of all necessary
servicing, repair or correction.

In no  case  unless  required by applicable law will we, and/or any other
party who  may  modify  and  redistribute  {\GAP}  as permitted above, be
liable  to you for damages, including lost profits, lost monies or  other
special, incidental or consequential damages  arising out  of the  use or
inability to use {\GAP}.


%%%%%%%%%%%%%%%%%%%%%%%%%%%%%%%%%%%%%%%%%%%%%%%%%%%%%%%%%%%%%%%%%%%%%%%%%%%%%
%%
%%  include the preface
%%
%% %%%%%%%%%%%%%%%%%%%%%%%%%%%%%%%%%%%%%%%%%%%%%%%%%%%%%%%%%%%%%%%%%%%%%%%%%%%%%
%%
%A  preface.tex                 GAP documentation           Joachim Neubueser
%%
%A  @(#)$Id: preface.tex,v 1.4 1997/04/16 12:17:34 gap Exp $
%%
%Y  Copyright 1990-1992,  Lehrstuhl D fuer Mathematik,  RWTH Aachen,  Germany
%%
%%  This file contains the preface of the GAP manual.
%%
%H  $Log: preface.tex,v $
%H  Revision 1.4  1997/04/16 12:17:34  gap
%H  Final(?) changes and index re-creation.
%H
%H  Revision 1.3  1997/04/16 09:46:43  gap
%H  The transitive groups are only in up to degree 23 (remainder still is to be checked).
%H
%H  Revision 1.2  1997/04/16 09:17:50  gap
%H  Added "preface for release 3.4.4"  SL
%H
%H  Revision 1.1.1.1  1996/12/11 12:36:49  werner
%H  Preparing 3.4.4 for release
%H
%H  Revision 3.12  1996/08/12 08:26:25  fceller
%H  fixed typo
%H
%H  Revision 3.11  1996/08/12 08:23:28  fceller
%H  added acknowledgment
%H
%H  Revision 3.10  1994/06/03 08:55:33  mschoene
%H  changed preface for GAP 3.4
%H
%H  Revision 3.9  1993/07/05  10:07:51  fceller
%H  added GRAPE and 3.3
%H
%H  Revision 3.8  1993/07/02  12:49:50  sam
%H  fixed '47'
%H
%H  Revision 3.7  1993/06/28  15:28:33  sam
%H  fixed very little bug
%H
%H  Revision 3.6  1993/02/19  10:48:42  gap
%H  adjustments in line length and spelling
%H
%H  Revision 3.5  1993/02/11  11:20:32  martin
%H  changed for 3.2
%H
%H  Revision 3.4  1992/04/30  12:16:36  martin
%H  changed a few sentences to avoid bold non-roman fonts
%H
%H  Revision 3.3  1992/04/06  09:32:52  martin
%H  fixed a very minor typo
%H
%H  Revision 3.2  1992/03/31  13:04:12  martin
%H  fixed several typos
%H
%H  Revision 3.1  1992/03/31  08:22:32  martin
%H  initial revision under RCS
%H
%%
\catcode`\*=12
\chapter*{Preface}
\catcode`\*=13
\markboth{PREFACE}{PREFACE to Release 3.4.4}

Welcome to the first release of {\GAP} from St Andrews. In the two
years since the release of {\GAP} 3.4.3, most of the efforts of the
{\GAP} team in Aachen have been devoted to the forth-coming major
release, {\tt GAP4.1}, which will feature a re-engineered kernel with
many extra facilities, a completely new scheme for structuring the
library, many new and enhanced algorithms and algorithms for new
structures such as algebras and semigroups. 

While this was going on, however, our users were not idle, and a
number of bugs and blemishes in the system were found, while a
substantial number of new or improved share packages have been
submitted and accepted. Once it was decided that the computational
algebra group at St Andrews would take over {\GAP} development, we
agreed, as a learning exercise, to release a new upgrade of {\GAP}
3.4, incorporating the bug fixes and new packages. 

Assembling the release has indeed been a learning experience, and has,
of course, taken much longer than we hoped. The release incorporates
fixes to all known bugs in the library and kernel. In addition, there
are two large new data libraries\: of transitivie permutation groups up
to degree 23; and of all groups of order up to 1000, except those of
order 512 or 768 and some others have been extended. This release
includes a number of share packages that are new since 3.4.3\: 

'autag':\\ for computing the automorphism groups of soluble groups;

'CHEVIE':\\ for computing with finite Coxeter groups, Hecke algebras,
Chevalley groups and related structures;

'CrystGap':\\ for computing with crystallographic groups;

'glissando':\\ for comnputing with near-rings and semigroups;

'grim':\\ for computing with rational and integer matrix groups;

'kbmag':\\ linking to Knuth-Bendix package for monoids and groups;

'matrix':\\ for analysing matrix groups over finite fields, replacing
'smash' and 'classic';

'pcqa':\\ linking to a polycyclic quotient program;

'specht':\\ for computing the representation theory of the symmetric
group and related structures; and

'xmod':\\ for computing with crossed modules. 

A number of other share packages have also been updated. Full details
of all of these can be found in the updated manual, which is now also
supplied in an HTML version.

Despite the tribulations of this release, we are looking forward to
taking over a central role in {\GAP} development in the future, and to
working with the users and contributors who are so essential a part of
making {\GAP} what it is.

St Andrews, April 18.,1997, \hfill Steve Linton.
\bigskip

In the distribution {\tt gap3-jm}, there are the following additional packages\:

'anupq':\\ The $p$-quotient algorithm, to work with $p$-groups.

'anusq':\\ The soluble quotient algorithm.

'arep':\\ Constructive representation theory.

'cohomolo':\\ Cohomology and extensions of finite groups.

'dce':\\ Double coset enumeration.

'grape':\\ Computing with graphs and group.

'guava':\\ Coding theory algorithms.

'meataxe':\\ Splitting modular representations.

'monoid':\\ Computing with monoids and semigroups.

'nq':\\ The nilpotent quotient algorithm.

'sisyphos':\\ Modular group algebras of $p$-groups.

've':\\ Vector enumeration, for representations of finitely presented
algebras.

'algebra':\\ Finite-dimensional algebras.

'vkcurve':\\ Fundamental group of the complement of a complex hypersurface.
Also provides multivariate polynomials and rational fractions.
\newpage

\markboth{PREFACE}{PREFACE}

{\GAP} stands for *Groups,  Algorithms  and  Programming*.  The name  was
chosen to reflect the  aim of the  system,  which is  introduced in  this
manual.

Until  well into the  eighties  the  interest  of pure mathematicians  in
computational  group theory  was  stirred  by,  but in  most  cases  also
confined to  the  information  that  was  produced  by  group theoretical
software  for  their special research  problems  --  and  hampered by the
uneasy  feeling  that  one  was   using  black  boxes  of  uncontrollable
reliability.  However the last years have seen a rapid spread of interest
in the understanding, design and even implementation of group theoretical
algorithms.  These are gradually becoming accepted both as standard tools
for a working group theoretician,  like certain  methods of proof, and as
worthwhile  objects of study, like  connections between notions expressed
in theorems.

{\GAP} was  started as  an attempt to meet  this  interest.   Therefore a
primary design goal has  been to give its user full access  to algorithms
and the data  structures used  by them, thus  allowing  critical study as
well as  modification of existing methods.  We also intend to relieve the
user from unwanted technical chores and to assist him in the programming,
thus supporting invention and implementation of new algorithms as well as
experimentation with them.

We have tried  to achieve these goals by a design which in addition makes
{\GAP} easily portable, even to computers such as Atari ST and Amiga, and
at the same  time facilitates the maintenance of {\GAP} with  the limited
resources of an academic environment.

While I had felt for some time rather strongly  the wish for such a truly
*open* system for computational group theory, the concrete idea of {\GAP}
was born when, together with a larger group of  students, among whom were
Johannes   Meier,    Werner   Nickel,   Alice     Niemeyer,   and  Martin
Sch{\accent127o}nert who eventually wrote the first  version of {\GAP}, I
had my first contact   with the Maple system   at the EUROCAL  meeting in
Linz/Austria  in  1985.  Maple demonstrated   to us the feasibility  of a
strong  and efficient computer algebra system  built from a small kernel,
with an  interpreted library of   routines written in  a  problem-adapted
language.  The discussion of the plan of a system for computational group
theory organized  in    a similar  way  started  in  the  fall  of  1985,
programming only in the second half  of 1986.  A  first version of {\GAP}
was operational by  the end of 1986.  The  system was first  presented at
the Oberwolfach meeting    on computational group   theory  in May  1988.
Version  2.4  was  the first  officially  to  be  given  away from Aachen
starting in December 1988.  The strong interest in this version, in spite
of its  still rather small  collection of group theoretical  routines, as
well  as constructive criticism  by many colleagues, confirmed our belief
in the general design principles of the  system.  Nevertheless over three
years had passed until in April 1992  version 3.1 was released, which was
followed in February 1993 by version 3.2, in November 1993 by version 3.3
and is now in June 1994 followed by version 3.4.

A main reason for the long time between versions 2.4 and 3.1 and the fact
that there had not been  intermediate releases was that  we had found  it
advisable to make a number of changes to basic data structures until with
version 3.1 we  hoped  to have reached a   state where we could  maintain
upward compatibility over further  releases, which were planned to follow
much more frequently.  Both  goals have been  achieved over the last  two
years. Of course the time has  also been used to extend  the scope of the
methods implemented in {\GAP}.   A rough estimate   puts the size  of the
program library of version 3.4 at about sixteen times the size of that of
version 2.4, while for version 3.1 the factor  was about eight.  Compared
to {\GAP} 3.2,  which  was the  last version  with  major  additions, new
features of {\GAP} 3.4 include the following{\:}
\begin{description}
\item[-] New data types (and extensions of methods) for algebras, 
         modules and characters
\item[-] Further methods for working with finite presentations 
         (IMD, a fast size function)
\item[-] Some ``Almost linear\'\'\, methods and (rational) conjugacy
         classes for permutation groups
\item[-] Methods based on ``special AG systems\'\'\, for finite soluble
         groups
\item[-] A package for the calculation of Galois groups and field
         extensions
\item[-] Extensions of the library of data (transitive permutation
         groups, crystallographic groups)
\item[-] An X-window based X-{\GAP} for display of subgroup lattices
\item[-] Five further share libraries (ANU SQ, MEATAXE, SISYPHOS, 
         VECTORENUMERATOR,  SMASH)
\end{description}

Work on the  extension of {\GAP}  is going on in Aachen  as well as in an
increasing number of  other places. We  hope to be  able to have the next
release of {\GAP} after about  9 months again,  that is in the first half
of 1995.

The system that you are getting now consists of four parts{\:}
\begin{enumerate}
\item

A comparatively  small *kernel*, written  in C,  which provides  the user
with{\:}

\begin{description}
\item[-] automatic dynamic storage management, which the user needn\'t
         bother about in his programming;
\item[-] a set of time-critical basic functions, e.g.  ``arithmetic\'\'\,
         operations for integers, finite fields, permutations and words,
         as well as natural operations for lists and records;
\item[-] an interpreter for the {\GAP} language, which belongs  to  the 
         Pascal family, but, while allowing additional types  for  group
         theoretical objects, does not require type declarations;
\item[-] a set  of  programming  tools  for  testing,  debugging,  and 
         timing algorithms.
\end{description}

\item 

A  much  larger  *library  of  {\GAP}  functions*  that  implement  group
theoretical and other algorithms.  Since this is  written entirely in the
{\GAP}  language, in contrast to the situation in older group theoretical
software,  the  {\GAP} language is both  the main implementation language
and the user language of the system.  Therefore the user can as easily as
the original programmers  investigate and vary algorithms of  the library
and add new ones to it, first for  own use and eventually for the benefit
of all {\GAP}  users.   We hope that  moreover  the  structuring  of  the
library using the concept of *domains* and  the techniques used for their
handling  that  have  been   introduced   into  {\GAP}   3.1  by   Martin
Sch{\accent127o}nert will be further helpful in this respect.

\item

A  *library of  group theoretical data*   which already  contains various
libraries  of groups (cf. chapter  "Group Libraries"), large libraries of
ordinary character  tables,  including all  of  the  Cambridge *Atlas  of
Finite  Groups*    and  modular tables   (cf.   chapter "Character  Table
Libraries"), and a  *library of tables of  marks*. We hope to extend this
collection further with the help of colleagues who have undertaken larger
classifications of groups.

\item

The *documentation*.  This is available as a file that can either be used
for on-line help or be printed out  to form this manual.  Some advice for
using  this manual  may be  helpful.  The  first chapter  *About GAP*  is
really an  introduction  to the use of the  system, starting from scratch
and, for  the  beginning,  assuming neither much  knowledge  about  group
theory  nor much versatility in  using  a computer.   Some  of the  later
sections of chapter 1 assume more,  however.  For instance section 'About
Character  Tables'  definitely  assumes  familiarity  with representation
theory  of  finite  groups,  while  in  particular  sections  'About  the
Implementation of Domains' to 'About Defining New Group Elements' address
more advanced users  who want to  extend the system to meet their special
needs.  The further chapters of  the manual give  then a full description
of the functions presently available in {\GAP}.

\end{enumerate}

Together with the  system we distribute *GAP  share libraries*, which are
separate packages which have been written by various groups of people and
remain   under their responsibility.  Some  of these packages are written
completely in the  {\GAP} language, others totally or  in parts in C  (or
even  other languages). However  the  functions in these  packages can be
called  directly from   {\GAP}  and results are   returned  to {\GAP}. At
present   there   are  10  such  share    libraries  (cf. chapter  "Share
Libraries").

The policy for the further development of {\GAP} is to keep the kernel as
small  as possible,  extending  the set  of basic functions  only by very
selected   ones that  have  proved  to   be  time-critical and,  wherever
feasible,  of  general use.    In  the interest  of the    possibility of
exchanging functions written in the {\GAP} language the  kernel has to be
maintained in  a  single place  which in  the foreseeable  future will be
Aachen.  On the other hand we hoped from the beginning that the design of
{\GAP} would  allow the library  of {\GAP} functions   and the library of
data to grow not only by continued work in Aachen  but, as does any other
part of mathematics, by contributions  from  many sides, and these  hopes
have been fulfilled very well.

There are some other points to make on further policy{\:}

\begin{description}

\item[-]

When we began work on {\GAP} the typical user that we had in mind was the
one  wanting to implement his  own algorithmic ideas.  While we certainly
hope that  we still serve  such users well  it has become clear  from the
experience  of the last  years  that there are   even more  users of  two
different species, on  the one hand  the established theorist,  sometimes
with little   experience in the  use  of computers, who   wants an easily
understandable tool, on the other  hand the student, often quite familiar
with computers, who  wants to get  assistance  in learning the theory  by
being able to do  nontrivial examples.  We  think that in fact {\GAP} can
well be used by both, but we realize that for each a special introduction
would be desirable.   We apologize that  we have not  had the time yet to
write such, however  have learned  (through the  {\GAP} forum) that  in a
couple of places  work on the  development of Laboratory Manuals  for the
use of {\GAP} alongside with standard Algebra texts is undertaken.

\item[-]

When we began work on {\GAP}, we designed it as a system for doing *group
theory*.  It has already turned out that in fact the design of the system
is general enough, and  some of its functions are  also useful, for doing
work in other neighbouring areas.   For instance Leonard Soicher has used
{\GAP}  to develop a  system {\sf GRAPE}   for working with graphs, which
meanwhile is  available  as a  share library.  We  certainly enjoy seeing
this happen, but we want to emphasize that in Aachen our primary interest
is the development of a  group theory system  and that we  do not plan to
try  to extend it beyond  our  abilities into  a general computer algebra
system.

\item[-]

Rather we hope to provide tools for linking  {\GAP} to other systems that
represent years  of work   and experience  in areas such   as commutative
algebra, or to  very efficient special purpose  stand-alone  programs.  A
link of this kind exists e.g. to the MOC system for the work with modular
characters.

\item[-]

We invite you to further extend {\GAP}.  We are willing either to include
such extensions  into {\GAP} or to  make them  available through the same
channels as {\GAP} in the form of the  above mentioned *share libraries*.
Of course, we will do this only if the  extension can be distributed free
of  charge  like {\GAP}.   The copyright  for  such share libraries shall
remain with you.

\item[-]
 
Finally to answer  an often asked  question{\:} The {\GAP} language is in
principle designed to  be compilable.  Work on a  compiler is on the way,
but this is not yet ready for inclusion with this release.

\end{description}

{\GAP} is given  away under the  conditions that have always  been in use
between  mathematicians, i.e.  in particular *completely  in source*  and
*free  of  charge*.  We  hope  that  the  possibility  offered  by modern
technology of  depositing {\GAP} on a number of  computers  to be fetched
from them by 'ftp', will assist us in this policy.  We want to emphasize,
however, two points.  {\GAP} is  *not* public domain software; we want to
maintain  a *copyright* that  in particular forbids  commercialization of
{\GAP}.  Further we ask that use of {\GAP} be quoted in publications like
the use of any  other mathematical work, and  we would be grateful if  we
could keep track of where {\GAP} is implemented.  Therefore we ask you to
notify us if you have got {\GAP}, e.g., by sending a short e-mail message
to  'gap@samson.math.rwth-aachen.de'.   The simple reason,  on top of our
curiosity, is that  as anybody  else in  an academic  environment we have
from time to time to prove that we are doing meaningful work.

We  have established a {\GAP} forum, where interested  users  can discuss
{\GAP}  related  topics  by  e-mail.  In particular  this  forum  is  for
questions about  {\GAP}, general  comments, bug  reports,  and  maybe bug
fixes.   We will read this forum and answer questions  and  comments, and
distribute  bug  fixes.  Of course  others  are  also  invited to  answer
questions, etc.  We will  also announce future releases of {\GAP} in this
forum.

To subscribe send an e-mail message to 'miles@samson.math.rwth-aachen.de'
containing the line 'subscribe  gap-forum <your-name>', where <your-name>
should be your  full name, not your  e-mail address.  You will receive an
acknowledgement, and from      then  on all  e-mail   messages    sent to
'gap-forum@samson.math.rwth-aachen.de'.

'miles@samson.math.rwth-aachen.de'     also    accepts   the    following
requests. 'help'  for a short  help on how  to use  'miles', 'unsubscribe
gap-forum'  to  unsubscribe, 'recipients  gap-forum'  to  get  a list  of
subscribers, and  'statistics gap-forum' to  see how many e-mail messages
each subscriber has sent so far.

The reliability of  large systems  of  computer programs is  a well known
general problem and, although over the past year the  record of {\GAP} in
this respect has not been too  bad, of  course  {\GAP} is not exempt from
this problem.  We therefore feel that it is mandatory  that  we, but also
other users, are warned of bugs that  have been encountered in  {\GAP} or
when doubts have arisen.  We  ask all users of {\GAP} to  use  the {\GAP}
forum for issuing such warnings.

We   have  also established  an e-mail    address  'gap-trouble' to which
technical  problems   of a  more local   character  such as  installation
problems can be sent. Together  with some experienced {\GAP} users abroad
we try to give advice on such problems.

{\GAP} was started as a joint Diplom project of four students whose names
have  already  been  mentioned.   Since then many   more finished  Diplom
projects have contributed to {\GAP} as well as other members of Lehrstuhl
D  and colleagues from other  institutes.  Their individual contributions
to the programs and to the manual are documented in the respective files.
To all of   them as well    as to all who  have   helped proofreading and
improving this manual  I want to express  my thanks for their  engagement
and enthusiasm as well as to many users  of {\GAP} who  have helped us by
pointing out   deficiencies and  suggesting improvements.   Very  special
thanks however go to  Martin Sch{\accent127o}nert.  Not only  does {\GAP}
owe many  of  its  basic design  features  to his  profound  knowledge of
computer languages  and the techniques for  their  implementation, but in
many long discussions he has in the name of  future users always been the
strongest defender of clarity of the design against my impatience and the
temptation for ``quick and dirty\'\'\, solutions.

Since  1992 the development of  {\GAP}  has been financially supported by
the Deutsche     Forschungsgemeinschaft    in  the   context      of  the
Forschungsschwerpunkt  ``Algorithmische Zahlentheorie   und  Algebra\'\'.
This very important help is gratefully acknowledged.

As with the previous versions we send this version out hoping for further
feedback of constructive   criticism.  Of course  we ask  to be  notified
about bugs,  but moreover  we shall  appreciate   any suggestion  for the
improvement of the  basic  system as  well  as of  the algorithms  in the
library.  Most of all,  however, we hope that in  spite of such criticism
you will enjoy working with {\GAP}.

Aachen, June 1.,1994, \hfill Joachim Neub{\accent127u}ser.





%%%%%%%%%%%%%%%%%%%%%%%%%%%%%%%%%%%%%%%%%%%%%%%%%%%%%%%%%%%%%%%%%%%%%%%%%%%%%
%%
%%  make the table of chapters and the table of contents
%%
%%  To make the table of chapters we read the  table  of  contents  file  and
%%  make sure that section lines are ignored.  We also must  make  sure  that
%%  table  of contents  file is  not cleared, because  we want  to read  it a
%%  second time for the full table of contents.
%%
%%  In the full table of contents we have to make a little more room for  the
%%  section numbers, because we have so many sections in some chapters.
%%
\newcommand{\ignoretwoarguments}[2]{}
\catcode`\@=11
\def\l@section{\ignoretwoarguments}
\catcode`\@12

%\catcode`\@=11
%\@fileswfalse
%\catcode`\@=12
%\tableofcontents
%\catcode`\@=11
%\@fileswtrue
%\catcode`\@=12

\catcode`\@=11
\def\@pnumwidth{2em}
\def\l@section{\@dottedtocline{1}{1.5em}{4em}}
\catcode`\@=12

\tableofcontents


%%%%%%%%%%%%%%%%%%%%%%%%%%%%%%%%%%%%%%%%%%%%%%%%%%%%%%%%%%%%%%%%%%%%%%%%%%%%%
%%
%%  and now the XMOD chapter
%%
\Include{xmod}

%%%%%%%%%%%%%%%%%%%%%%%%%%%%%%%%%%%%%%%%%%%%%%%%%%%%%%%%%%%%%%%%%%%%%%%%%%%%%
%%
%%  and the bibliography
%%
\begin{sloppypar}
\bibliographystyle{alpha}
\newcommand{\ignore}[1]{}
\catcode`\'=12 \catcode`\<=12 \catcode`\*=12
\catcode`\|=12 \catcode`\:=12 \catcode`\"=12
\bibliography{xmmanual}
\end{sloppypar}


%%%%%%%%%%%%%%%%%%%%%%%%%%%%%%%%%%%%%%%%%%%%%%%%%%%%%%%%%%%%%%%%%%%%%%%%%%%%%
%%
%%  finally include the index
%%
 \begin{sloppypar}
 \raggedright
 \begin{theindex}
 \catcode`\@=11
 \@input{xmmanual.ind}
 \catcode`\@=12
 \end{theindex}
 \end{sloppypar}

\end{document}
