%%%%%%%%%%%%%%%%%%%%%%%%%%%%%%%%%%%%%%%%%%%%%%%%%%%%%%%%%%%%%%%%%%%%%%%%
% This file is part of the C Meat-Axe.                                 %
% Written by Michael Ringe <mringe@tiffy.math.rwth-aachen.de>          %
% (C) Copyright 1992:	    Lehrstuhl D fuer Mathematik                %
%                           RWTH Aachen                                %
%                           Aachen, Germany                            %
%%%%%%%%%%%%%%%%%%%%%%%%%%%%%%%%%%%%%%%%%%%%%%%%%%%%%%%%%%%%%%%%%%%%%%%%

\chapter{Introduction}
%---------------------

\section{Overview}
%-----------------
The {\MeatAxe} is a set of programs for working with matrices over
finite fields. Its primary purpose is the calculation of modular
character tables, although it can be used for other purposes, such
as investigating subgroup structure, module structure etc.\ Indeed,
there is a set of programs (see section \ref{sec:lattice}) to
compute automatically the submodule lattice of a given module.

There is one common module (ZZZ) which does all the finite field
calculations. ZZZ works with single elements and rows of elements,
does the input and output, handles the memory allocation and a few
miscellaneous simple operations besides. This makes it possible to
code the most time-consuming parts in assembler and still have the
algorithms presented in C. Another advantage is, that all the
system dependent stuff (file i/o, for example) is concentrated in
one place.

Each of the programs is self-contained, except that it calls the ZZZ
subroutines, reading its input from files, and writing its output to
files. To make the {\MeatAxe} usable, therefore, it is necessary to
write operating system commands to run the various programs. This
documentation is primarily for the programs, and further documentation
is necessary for the various implementations in differing operating
environments.

The primitive objects are of two types: matrices and permutations.
Permutation objects can be handled, but not as smoothly as you might
expect. However, the matrix part is more or less complete. For example,
it is hoped that programs such as split (ZSP) and multiply (ZMU) will
be able to work with mixed types, but at
present ZSP is restricted to matrices only, and ZMU can multiply
matrix * permutation, but not vice versa.


\subsection*{Remarks on the C implementation}
%--------------------------------------------
Most of the programs (including their documentation) in this package
are based on the FORTRAN code written by Richard~A.~Parker. The
translation from FORTRAN to C was
done in 1989 and since then both versions have developed independently.
The user interface of the C programs has been changed significantly
(e.g., command line arguments were introduced) and some new programs
have been added which are not available in the FORTRAN {\MeatAxe}.
However, the data formats are still compatible, so it is possible to
share even binary data files between FORTRAN and C programs.

The objective of this suite of programs is research, so that where
simplicity and power are in contention, power is selected. Nevertheless,
I have tried to make the programs as straightforward and user-friendly
as possible. Moreover, the source code is available, so the user can
change any parts he is annoyed with. There are dangers in changing the
source, though, in that future versions may then need to have the same
changes made to them. I hope that the file formats and ZZZ interface
will remain the same, so that the user is welcome to write his own
programs to do the special parts he needs.
Of course, I would appreciate any comments on bugs and suggestions for
improvements of the existing programs, which may be included in future
releases.



\subsection*{Non-UNIX Systems}
%----------------------------
All the programs have been developed and tested in an UNIX
environment. Some other systems --- currently MS-DOS and VM/CMS
--- are supported, but this is not thoroughly tested.


\subsubsection*{MS-DOS/DJGPP}
%---------------------------
The {\MeatAxe} must be compiled with DJGCC, a DOS port of
GNU gcc. DJGCC has a built-in DOS extender which allows
programs to run in protected mode and providing full 32-bit
memory access.

Some restrictions are imposed by the MS-DOS operating system.
\begin{itemize}
\item Process time information and time limits are not
    available. Programs that report the CPU time at end will
    print the real run time of the program. The \verb"-T"
    option is silently ignored.
\item File names are restricted to 8 characters plus a 3 character
    extension. The {\MeatAxe} i/o system will try to convert any
    name that does not fit into this pattern into a valid file name.
    For example,
    \msg{zad matrix.1 matrix.2 matrix.sum.1.2}
    will write its ouput to \verb"matrixsu.m12".
\end{itemize}


\subsubsection*{VM/CMS}
%---------------------
File names are translated as follows:
\begin{list}{}{}
\item	{\tt file1} becomes \mbox{\tt FILE1 MTX A}
\item	{\tt file2.test} becomes \mbox{\tt FILE2 TEST A}
\item	{\tt file3.test.b} becomes \mbox{\tt FILE3 TEST B}
\end{list}




\section{Using the {\MeatAxe} Programs}
%--------------------------------------
This section contains some general information about the {\MeatAxe}
programs and how to use them.


\subsection*{Setting up the Environment}
%--------------------------------------
Before you can use the {\MeatAxe} you must include the {\tt bin}
and {\tt bin/proggy} directories in your shell's path. You may also
want to define the following environment variables:
\begin{center}
\begin{tabular}{|l|p{0.75\linewidth}|}
\hline
Name		& Meaning \\
\hline
\tt MTXLIB	& Name of the {\tt lib} directory. You need to
		  define this variable only if you want to use
		  a library directory different from the default
		  library (see section \ref{sec:libdir}).\\
\hline
\tt MTXBIN	& Name of the {\tt bin} directory. This is used
		  during the automatic generation of arithmetic
		  tables.\\
\hline
\end{tabular}
\end{center}



\subsection*{Arithmetic Tables}
%-----------------------------
All programs use lookup tables for row operations and finite field
arithmetic. These tables are read from a file named
\msg{pXXX.zzz}
where \verb"XXX" is the field order. For example, tables for GF(25) are
read from the file {\tt p025.zzz}. If the table file does not exist,
the program will try to execute the \verb"maketab" program to generate
the file. If that fails, too, the program stops. The \verb"maketab"
program must be in one of the following directories:
\begin{enumerate}
\item
    In the \verb"MTXBIN" directory (see above).
\item
    The directory where the program ({e.g.} CHOP) resides. Note that
    this works only if the program was called with its full path.
\item
    Anywhere in the path.
\end{enumerate}

Table files may also be placed in the library directory (see below)
because the programs look there before they try to generate a new file.



\subsection*{File Types}
%----------------------
All {\MeatAxe} programs operate on files --- normally, no interactive
i/o is required during the run. There are two kinds of files:
\begin{itemize}
\item	Binary (internal format) data files. Programs usually read
	and write data in binary format. For a description of the
	file format, see section \ref{sec:filefmt}. Default names
	are G1, G2, \ldots for input files and P1, P2, \ldots for
	output files.
\item	Text files, used by some programs as ZCV and ZPR. The format
	of text files is described with the ZCV program (see section
	\ref{sec:zcv}).
\end{itemize}
Typically, a program reads its input from G1,\ldots,G$n$ and writes
its output to P$n+1$,\ldots,P$m$.
For example, the ZTE (tensor matrices) program
reads two matrices from G1 and G2 and writes their tensor product to
P3. To find out which files are used by a specific program, look up the
program description in chapter \ref{chap:progs}.

You may override the default names by specifying new file names on the
command line. For example, to multiply two matrices in {\tt mat1} and
{\tt mat2} and write the result to {\tt result} call ZMU with
3 arguments as follows:
	\msg{zmu mat1 mat2 result}



\subsection*{The Library}\label{sec:libdir}
%--------------------
When given a file name, all programs look for the file in the current
directory first. If the file is not found there, some programs try to
find it in the library directory, provided a library has been defined.
This extended search applies, for example, to ZEV and ZCV input files
and also to the arithmetic table files.

The library is usually defined at compile-time by assigning a
value to the symbol {\tt MTXLIB}. This definition may be overridden
at run-time by defining the environment variable {\tt MTXLIB}.
In either case {\tt MTXLIB} must be a complete directory name including
a trailing separator character (e.g., '/' for UNIX). When looking for
files in the library, the programs simply append the file name to the
value of {\tt MTXLIB} in order to build the complete file name.



