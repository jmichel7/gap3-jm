\documentstyle[a4]{article}
\def\Stab{\mathord{\mbox{Stab}}}
\def\gen<#1>{\left\langle\,#1\,\right\rangle}
\def\genr<#1|#2>{\left\langle\,#1\mid#2\,\right\rangle}
\let\iso=\cong
\def\split{\mathop{:}}
\title{GAP Double Coset Enumerator --- Internals}
\author{Steve Linton}
\begin{document}
\maketitle
\section{About This Document}
This document is not for the faint of heart. It describes the data
structures and algorithms  of the double coset enumerator, as
implemented in GAP. It should be read only when you are familiar with
the user documentation, and especially the mathematical introduction.

\section{The Structure of the Double Coset Table}
A successful double coset enumeration returns a DCE universe {\tt u}
containing the results of the enumeration and various pieces of
ancillary data. The main piece of data stored is the Double Coset
Table, which is stored in {\tt u.table}. 
To understand its structure we first need to look at the pre-computed
information stored for each gain group and generator.

Throughout this section we consider only those parts of the data which
have meaning in a closed table. Other fields may be present in the
records.
\subsection{The Gain Group Records}
The gain group information supplied in the presentation is processed
into records with the following fields (for a gain group $L$):
\begin{description}
\item[\tt dom] A GAP set on which {\tt u.K} acts like the action
of $K$ on the left cosets of $L$ (note that the image under this action of
$k$ on $k'L$ is $k^{-1}k'L$).
\item[\tt op] The GAP operation by which {\tt u.K} acts on {\tt dom}.
\item[\tt pt0] The member of {\tt dom} corresponding to $L$.
\item[\tt stab] A function for computing the stabilizer of points in
{\tt dom} under the action of subgroups of {\tt u.K}. This should have
the same syntax as {\tt Stabilizer}, to which it defaults.
\item[\tt L] The gain group itself. Should be equal to {\tt
stab(u.K,pt0,op)}.
\end{description}
A list of these records is stored in {\tt u.GGrecs}.
\subsection{The Generator Records}
For each generator $x$ input or created as an inverse, a record is
constructed. These records are listed in {\tt u.Xrecs}. The generators
are usually referred to internally by indices into this list. The fields of
each record are:
\begin{description}
\item[\tt CohortSize] The integer $|K:L_x|$.
\item[\tt action] This field described how $x$ maps $L_x$ to $L^{(x)}$. It
can be:
\begin{description}
\item[\tt false] indicating that $x$ commutes with $L_x$.
\item[an element of $K$] indicating that $x$ acts as that element does
by conjugation.
\item[an explicit homomorphism] from $L_x$ to $L^{(x)}$, giving the
action.
\end{description}
\item[\tt wgg] The index in {\tt u.GGrecs} of the gain group record
corresponding to (a conjugate of) $L_x$. Let {\tt ggr} be that record.
\item[\tt pt] The member of {\tt ggr.dom} corresponding to $L_x$.
\item[\tt Reps] A set of left coset representatives of $L_x$ in $K$. These
are ordered so that {\tt ggr.op(pt,Reps[i]\^\ -1) = ggr.dom[i]}.
\item[\tt inverse] The index in {\tt u.Xrecs} of the record corresponding
to $x^-1$.
\item[\tt L] The gain group $L_x$.
\item[\tt name] The name of $x$ as a word in abstract generators of length
1.
\item[\tt invol] Boolean, true if $x$ is was specified to be an involution.
\item[\tt ggconj] An element of $K$ which conjugates {\tt ggr.L} to {\tt
L}.
\end{description} 

\subsection{The Table itself}.
This is an array of records, stored in {\tt u.table},
each record {\tt d} corresponding to a double coset $D$ with
representative $d$. The main fields in {\tt d} are:
\begin{description}
\item[\tt d.name] An integer used to print and store {\tt d}, and
occasionally to refer to it (though mostly we use the record {\tt d}
itself). Names are allocated in increasing order as double cosets are
defined, and are {\em not} changed when the table is packed. Thus the
final set of names will not usually be sequential. 
\item[\tt d.muddle] The muddle group $M_d$ as a subgroup of {\tt
u.K}. If $M_d = K$, then {\tt IsIdentical(u.K,d.muddle)} will be
true. If $M_d$ is the trivial group then {\tt
IsIdentical(u.triv,d.muddle)} will be true.
\item[\tt d.cohorts] This contains the actual coset table entries. It
is a list of lists (cohorts), one for each generator (and inverse, if
necessary) in the same order as {\tt u.Xrecs}. The cohort for a
generator $x$ has (in a closed table) $|K:L_x|$ entries. These can be
of two kinds
\begin{description}
\item[alpha entries] These are GAP lists. If {\tt d.cohorts[x][i]} is
a list, then it will have length 2. The first entry will be a row
({\tt d1} corresponding to a double coset $D_1$ with representative
$d_1$) of the table (the actual row, not an index), while the second
will be an element $k_1$ of {\tt u.K}. This entry signifies that
$$d.\mathord{\hbox{\tt u.Xrecs[x].Reps[i]}}x = d_1.k_1.$$
\item[beta entries] These can only arise when $M_d$ is non-trivial. We
only actually need one image per $(M_d,L_x)$ double coset to describe
the action of $x$ on $D$. When such a double coset contains more than
one left $L_x$ coset we use a beta-entry to redirect enquiries to the
double-coset representative.
Beta entries are GAP records, and have two components:
\begin{description}
\item[\tt col] The position in the cohort of the double-coset
representative. This must always point to an alpha entry.
\item[\tt elt] An element of $M_d$ with the following property:
if a beta entry {\tt b} appears in column {\tt i} of the {\tt x} cohort then:
$$\begin{array}{c}\hbox{\tt b.elt*u.Xreps[x].Reps[i] in} \\
\qquad \hbox{\tt u.Xreps[x].Reps[b.col]*u.Xrecs[x].L}\end{array}$$
\end{description}
\end{description}
\end{description}

The use of these records is best understood by looking at the DCE
internal function {\tt DCEFollow}, somewhat simplified by removing
optimisations and provision for open tables.

This function takes three arguments, the DCE Universe {\tt u}, a list
{\tt dk} of length 2, containing a name $d.k$ for a single coset, and {\tt
r}, which can be either an element of $K$ or a generator number. It
computes a name for the product $d.k.r$ and overwrites {\tt dk} with it.

\begin{verbatim}
DCEFollow := function(u,dk,r)
    local col,d,xrec,ggrec,cohort,op;
    d := dk[1];
    if IsInt(r) then
        xrec := u.Xrecs[r];
        ggrec := u.GGrecs[xrec.wgg];
        cohort := d.cohorts[r];
        op := ggrec.op;
        col := PositionSorted(ggrec.dom,ggrec.op(xrec.pt,dk[2]^-1));
        if  IsRec(cohort[col]) then
            dk[2] := cohort[col].elt*dk[2];
            col := cohort[col].col;
        fi;
        dk[1] := cohort[col][1];
        dk[2] := cohort[col][2]*DCEXonEl(u,xrec.Reps[col] mod dk[2],r);
    else
        dk[2] := dk[2]*r;
    fi;
    return dk;
end;
\end{verbatim}

As can be seen the problem splits immediately into two cases, as $r$
is in $K$ or $X$. The first case (dealt with second in the code) is
easy. In the other case, after loading up pointers to various
records, we compute {\tt col}, the position in the cohort where the
image of $dk$ would be found, were $M_d$ trivial. This makes use of
the action which we supplied as part of our gain group information, to
avoid computing with cosets of $L_r$.

If this entry is a beta-entry (a record) then we must change {\tt dk}
to another name for the same single coset, by multiplying {\tt dk[2]}
(which is $k$) on the left by the {\tt elt} part of the beta
entry. The product, the new
value of $k$, lies in a different left $L_r$ coset, and so is looked up
in a different column. We could compute which column, but we can access
it more quickly by looking in the beta-entry.

Now {\tt col} contains the index of an alpha-entry (in {\tt
cohort}). Let $k_0$ be {\tt xrep.Reps[col]}. The alpha-entry tells us
a name for $dk_0 r$. Furthermore $k\in k_0L_r$, so $k_0^{-1} k \in
L_r$. The expression passed to {\tt DCEXonEl} represents this element,
and {\tt DCEXonEl} computes its conjugate by $r$, which is defined by
the definition of $L_r$. We are thus able to implement the fact that
$$dkr = dk_0 (k_0^{-1}k) r = d k_0 r (k_0^{-1}k)^r$$
and compute a name for $dkr$ from the alpha-entry.

This routine is the heart of the algorithm. Everything else is
essentially driven by the aim of making this work.

\section{Functions for Accessing a Closed Double Coset Table}

In this section we look at some of the routines used internally by DCE
which might be useful to someone wanting to write further programs
such as the Collapsed Adjacency Matrix programs that access the
completed double coset table.

We assume always that the enumeration is complete, and omit from our
description the data structures, pre-conditions and return values not
applicable to this case.

Note that these functions are intended mainly for internal use and so
{\em do not check their arguments}.

As usual the first argument is normally a DCE Universe.

\subsection{\tt DCEColumnRegular(u,k,x)}
\begin{description}
\item[\tt k] An element of {\tt u.K}
\item[\tt x] A generator (ie the index of an entry in {\tt u.Xrecs})
\end{description}
Returns the position within the {\tt x} cohort of any row $D$ of the
double coset table in {\tt u} with $M_d$ trivial, in which to look up
the image $dkx$. This basically amounts to dealing with the action of
$K$ on the left cosets of $L_x$, as stored in the appropriate gain
group records.
\subsection{\tt DCEColumn(u,dk,x)}
\begin{description}
\item[\tt dk] A name for a single coset $dk$, as a length 2 list.
\item[\tt x] A generator (number)
\end{description}
This function returns the column within the {\tt x} cohort of 
{\tt dk[1]} from which the image $dk.x$ will actually be computed,
allowing for the action of $M_d$. As a side-effect, it may alter {\tt
dk}, replacing it with another name for the same single coset. The
return value will  always be the index of an alpha entry in the {\tt
x} cohort of $d$.
\subsection{\tt DCEXonEl(u,l,x)}
\begin{description}
\item[\tt l] An elements of $L_x$.
\item[\tt x] A generator number.
\end{description}

Returns $l^x$, which must lie in $K$, by the construction of $L_x$.

\subsection{\tt DCEXonSG(u,h,x)}

\begin{description}
\item[\tt h] A subgroup of $L_x$.
\item[\tt x] A generator number.
\end{description}

Returns $h^x$, which must be a subgroup of $K$, by the construction of
$L_x$.

\subsection{\tt DCELin(u,M,x,k)}
\begin{description}
\item[\tt M] A subgroup of $K$.
\item[\tt x] A generator number.
\item[\tt k] An element of $K$.
\end{description}

Returns the subgroup $M\cap L_x^k$ of $K$, taking advantage of the
extra information about $L_x$ which is stored, to compute the
intersection more efficiently.

\subsection{\tt DCEFollow(u,dk,x)}

\begin{description}
\item[\tt dk] A name for a single coset, stored as a length 2 list.
\item[\tt x] A generator number.
\end{description}

This function replaces the entries in the list {\tt dk}, so that it
represents a name for the single coset ${\tt dk}.{\tt x}$. The same
list is returned.


\section{Words and Relators}
\subsection{Data Structures}

The Double Coset Enumerator uses two representations for elements of
the free product $K*F_X$. The {\bf external} representation is as DCE
words (which are records), as described in the DCE user
documentation. There is also an {\bf internal representation} as
lists. Each element of such a list is either an element of $K$ or an
integer, used as an index into {\tt u.Xrecs} and representing an
element of $X\cup X^{-1}$.

Conversion from external to internal form is achieved by the the
function {\tt DCEProcessRelator(u,r)}. This also combines adjacent
elements of $K$ and  expands out words in abstract generators.

\subsection{A Little More Mathematics}

The first section of the user documentation described the mathematics
behind the encoding of a permutation action in a double coset
table. An additional result gives considerable speed-ups in using that
information.

Let $w = r_1\cdots r_k$ be an element of $K*X$ and let $M$ be any
subgroup of $K$. Define $M_1=M$ and $M_2,\ldots,M_{k+1}$ by 
$$ M_{i+1} = \cases{M_i^{r_i}&$r_i\in K$\cr 
\left(M_i\cap L_{r_i}\right)^r_i & $r_i\in X$}$$

Now define $M^{(k+1)} = M_{k+1}$ and $M^{(k)},\ldots,M^{(1)}$ by
$$ M^{(i)} = M^{(i+1)\;r_i^{-1}}$$
The construction ensures that this expression will be defined even
when $r_i\in X$.

Finally let $$M_w = M^{(1)}.$$

The group $M_w$ represents that subgroup of $M$ which remains ``in
one piece'' under the action of $w$. That is: $$m\in M_w \Longrightarrow
dkmw \in dkwM$$ for all single cosets $dk$. 

This is a very useful result, most often in the case where $M=K$, but
occasionally otherwise. The enumerator itself makes use of this to
avoid unnecessary relator tracing, while the adjacency matrix
calculations in {\tt DCEColAdjSingle} use the case where $M=K\cap H$.


\subsection{More functions}

\subsubsection{\tt DCEWordGroup(u,w[,M])}

The theory of the preceding is implemented by the function {\tt
DCEWordGroup}. This takes two compulsary arguments {\tt u} the DCE
Universe and {\tt w} a word in internal form. The third (optional)
argument is the subgroup {\tt M}, if it is absent the default is {\tt
u.M} if that exists and otherwise {\tt u.K}.

\subsubsection{\tt DCERin(u,r)}

This function returns the inverse of {\tt r}, which should be an
element of a relator in internal form. That is {\tt r} is either an
element of {\tt u.K} or an integer (denoting an element of $X$).

\subsubsection{\tt DCEMultiFollow(u, dk, rel, f, b, backwards)}

This function computes the image of the single coset denoted by {\tt
dk} (as usual a list of length 2), under the subword of {\tt rel} (a
relator in internal form) beginning with position {\tt f} and ending
before position {\tt b}. If the Boolean value {\tt backwards} is {\tt false}
then {\tt f} should be less than {\tt b} and the subword is read
forwards. If {\tt backwards} is {\tt true} then {\tt f} should be
greater than {\tt b} and the subword is read backwards.

It returns the index of the first letter of {\tt rel} not used, which
in a closed table should always be {\tt b}.


\end{document}








