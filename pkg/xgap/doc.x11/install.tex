\Chapter{Installing XGAP}


%%%%%%%%%%%%%%%%%%%%%%%%%%%%%%%%%%%%%%%%%%%%%%%%%%%%%%%%%%%%%%%%%%%%%%%%%
\def\XGAP{{\sf XGAP}}
\Section{The XGAP Package}

{\XGAP} is a graphical  user interface for  {\GAP}, it extends the {\GAP}
library with functions  dealing with graphic sheets   and objects.  Using
these functions it also  supplies a graphical interface for investigating
the subgroup lattice of a group, giving you  easy access to the low index
subgroups, prime quotient  and Reidemeister-Schreier algorithms  and many
other {\GAP} functions for groups and  subgroups.  At the moment the only
supported  window system is X-Windows X11R5,  however, programs using the
{\XGAP} library functions will run on  other platforms as soon as {\XGAP}
is  available on these.  We plan  to release a  Windows  3.11 in the near
future.

The section "Installing the XGAP Package under X-Windows" is intented for
the system administrator who wants  to install {\XGAP}.  Chapter "XGAP  -
Graphic Lattices" explains how  to use {\XGAP} for investigating subgroup
lattices, and finally chapter "XGAP - Graphic Extensions" explains how to
use the  {\XGAP}   functions dealing  with  graphic  sheets  in  your own
programs.

The file  selector used under   X11R5 is copyrighted  by Erik  M. van der
Poel, see the file 'src.x11/selfile.c' for the copyright notice.

Please send comments, bug reports, etc. to

\centerline{|F.Celler@Math.RWTH-Aachen.DE| or 
            |Gap-Trouble@Math.RWTH-Aachen.DE|.}

There  are also two  TODO  lists in  the  archive which contain  features
which are missing or do not work correctly.

If {\XGAP}  fails to compile   please include the  type  of machine,  the
version number of compiler and the version of  window system used in your
bug  report.   If  possible also include  a   log  of the  compilation or
configuration attempt.

We  would  like to thank   Sarah Rees for   her  help in implementing the
interactive lattice functions and beta testing.

\centerline{Frank Celler \& Susanne Keitemeier}


%%%%%%%%%%%%%%%%%%%%%%%%%%%%%%%%%%%%%%%%%%%%%%%%%%%%%%%%%%%%%%%%%%%%%%%%%
\Section{Installing the XGAP Package under X-Windows}

{\XGAP} is written in C and the package can only  be installed under UNIX
running X-Windows X11R5 (or higher).  It  has been tested on a DECstation
running Ultrix 4.2/X11R5, a SUN running SunOS/X11R5, a PC running FreeBSD
1.1/X11R5, FreeBSD 2.0/X11R6, and Linux SLS 1.05/X11R5, and a NeXTstation
running NeXTSTEP 3.0/MouseX11R5.

{\XGAP} *might  not* compile properly  on Suns running OpenWindows and it
*will not* run on PCs running OS/2 or Windows 3.1, 3.11 or 95.

If you got a complete binary and source distribution for your machine, no
compilation   is  necessary,  you  must  however  edit   the shell script
'gap3r4p?/pkg/xgap/bin.x11/xgap' where '?'    is to  be  replaced by  the
current patch level of {\GAP}.  See *Customizing {\XGAP}* below.

If  you got a  complete source distribution, skip  the extraction part of
this section  and proceed with  the  compilation part *Compiling {\XGAP}*
below.

{\bigskip\large *Unpacking {\XGAP}*}

In the example we will assume that you, as user 'gap', are installing the
{\XGAP} package for use by several users on a network of two DECstations,
called 'bert' and 'tiffy', and a X-Terminal called 'ncd0'. We also assume
that {\GAP}   is  already installed  on    the DECstations following  the
instructions given in "Installation of GAP for UNIX".

Note that certain parts  of  the  output  in the examples should  only be
taken as rough outline, especially file sizes and file dates are *not* to
be taken literally.

First of  all you have to  get the file  'xgap.zoo' (see  "Getting GAP").
Then you must locate  the {\GAP} directory  containing 'lib/' and 'doc/',
this is usually 'gap3r4p?/'.

|    gap@tiffy:~> ls -l
    drwxr-xr-x  11 gap     1024 May 18 20:31 gap3r4p2
    -rw-r--r--   1 gap   nnnnnn Jul 29 15:16 xgap.zoo
    gap@tiffy:~> ls -l gap3r4p2
    dr-xr-xr-x   2 gap      512 Jul  6 12:08 bin
    dr-xr-xr-x   3 gap     3584 Jun 14 12:05 doc
    dr-xr-xr-x   3 gap     1024 Jul 14 23:40 etc
    dr-xr-xr-x   3 gap      512 May 18 11:34 grp
    drwxrwxr-x   3 gap     2048 Jul 19 12:14 lib
    dr-xr-xr-x  14 gap      512 Jun 14 14:47 pkg
    dr-xr-xr-x   8 gap     2048 May 18 15:29 src
    drwxr-xr-x   3 gap     2048 Jul  3 13:22 tbl
    dr-xr-xr-x   3 gap      512 Jun 14 14:43 thr
    dr-xr-xr-x   3 gap      512 Jun 14 14:43 tom
    dr-xr-xr-x   3 gap      512 Jun 14 14:43 tst
    dr-xr-xr-x   3 gap     1024 Jun 14 14:43 two
    dr-xr-xr-x   3 gap      512 Jun 14 14:43 utl |

Unpack  the share package  using 'unzoo'  (see "Installation  of GAP  for
UNIX").  Note that you must be in the directory containing 'gap3r4p?/' to
unpack the files.  After you have unpacked the source  you may remove the
archive-file 'xgap.zoo'.

If you are  installing  a newer version  of  {\XGAP}  move the  directory
'gap3r4p?/pkg/xgap/' to 'gap3r4p?/pkg/xgap.old/' before unpacking the new
version,     if   you    want    to    keep     the   resource   database
'gap3r4p?/pkg/xgap/rdb.x11/' for your servers.   In  this case  move  the
files              'gap3r4p?/pkg/xgap.old/rdb.x11/serv.\*'             to
'gap3r4?/pkg/xgap/rdb.x11/' after unpacking the new version.

|    gap@tiffy:~> mv gap3r4p2/pkg/xgap gap3r4p2/pkg/xgap.old
    gap@tiffy:~> unzoo x xgap
    gap@tiffy:~> ls -l gap3r4p2/pkg/xgap
    -rw-r--r--   1 gap     5796 Jul 24 15:25 INSTALL.x11
    -rw-r--r--   1 gap     2255 Jul 24 14:40 Makefile
    drwxr-xr-x   3 gap      512 Jul 24 14:58 bin.x11
    drwxr-xr-x   3 gap     1024 Jul 24 15:06 doc
    drwxr-xr-x   3 gap      512 Jul 24 15:56 doc.x11
    drwxr-xr-x   3 gap      512 Jul 24 12:02 lib
    drwxr-xr-x   3 gap      512 Jul 24 15:34 rdb.x11
    drwxr-xr-x   3 gap     1024 Jul 24 14:41 src.x11
    drwxr-xr-x   3 gap      512 Jul 24 11:52 tst |

{\bigskip\large *Compiling {\XGAP}*}

Switch into the directory 'gap3r4p?/pkg/xgap/' and  type 'make x11'. Note
that  {\XGAP} does  not supply an  'Imakefile'.  If  you are using  share
libraries  and not every machines  has all X11  library files, you should
link   {\XGAP}  statically.  For   the GNU    C  Compiler  use 'make  x11
LOPTS=-static'.

|    gap@tiffy:~ > cd gap3r4p2/pkg/xgap
    gap@tiffy:../xgap > make x11
    # you will see a lot of message ending in 'cc -o xgap ...' |

If the configuration  script fails and you  have GNU bash installed,  try
'make x11 SHELL=bash'.  If 'make  x11' does not  work at all, switch into
the  directory 'src.x11/'  and type  'make  machines' to  get a  list  of
supported targets.  Select  the target you  need.  For example, we  could
use  the target 'dec-mips-ultrix-gcc' in order  to compile the DECstation
version using the GNU C Compiler.

If everything worked  then there should be  an  executable 'xgap'  in the
directory 'src.x11/'.

|    gap@tiffy:../xgap > ls -l src.x11/xgap
    -rwxr-xr-x   1 gap  1629416 Jul 24 14:42 src.x11/xgap |


If {\GAP} lives in the directory 'gap3r4p?/src/' try the following:

|    gap@tiffy:../xgap > ./xgap -G ../../src/gap |

This command should create a new window in which  {\GAP} is awaiting your
input, type 'quit;' inside this window.  Strip the executable and move it
into the directory 'bin.x11/', but *do not* call it 'xgap'.

|    gap@tiffy:../xgap > strip xgap
    gap@tiffy:../xgap > mv xgap ../bin/xgap-dec-mips-ultrix |

Type 'make clean' in order to clean up, this will remove all '\*.o' files
created during the installation.

{\bigskip\large *Customizing {\XGAP}*}

Switch into the  directory 'bin.x11/' and  edit the  shell script 'xgap',
you   must  at least set    the  variable 'GAPPATH'  and possibly  'TYPE'
(depending on what name  scheme you use).  'GAPPATH'  should point to the
directory 'gap3r4p?/'.

For each  X11 server  you  use, you can  customize  the fonts and colors.
Switch into the directory 'rdb.x11/'  and copy  the file 'expl.color'  to
'serv.<servername>' for a color server and 'expl.mono' for a mono server.
In our  example, we   copy  'expl.mono' to  'serv.bert' and   'serv.ncd0'
because these machines  have  a monochrome  display  and 'expl.color'  to
'serv.tiffy'  because  this is  a color machine.   Edit   these files and
choose the colors and fonts you want.

Switch back into the 'xgap/' directory and type

|    gap@tiffy:../xgap > bin.x11/xgap --verbose |

in order to start {\XGAP}.  After {\XGAP} has started, use the pointer to
select the menu entry 'Read  file ...' from the  menu 'Gap'.  In the file
selector which will appear select the file 'tst/color.g'.  This will open
a  new window and  show  the selected  colors.   Repeat this procedure at
least with 'tst/fonts.g' to see what fonts are used.

If 'xgap' executes correctly, copy the script 'xgap'  to a location where
other users can find it, e.g., '/usr/local/bin'.

{\bigskip\large *Printing the {\XGAP} Manual*}

It  is possible to print   this manual as  standalone  manual (instead of
printing  it   as  part    of     the  {\GAP}  manual).     Switch     to
'gap3r4p?/pkg/xgap/doc/' and type  'latex latexme'.  This  will latex the
manual.  You must latex it twice in order to get the references right.

However, if you don\'t  install the manual  as part of the {\GAP} manual,
the online help in {\GAP} will not  be able to  find the new chapters and
sections.  In order to  install the manual as  part of the {\GAP} manual,
copy  the file  'xgap.tex'     and 'xgaplatt.tex'  into the     directory
'gap3r4p?/doc'  and modify  'manual.tex'  to include  these two chapters.
Latex 'manual.tex' as described in "Installation of GAP for UNIX".

