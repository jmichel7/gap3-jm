\documentclass[12pt]{amsart}
%\documentclass{amsart}

\usepackage{amssymb}

\newcommand{\comment}[1]{{\sf #1}}
%\newcommand{\comment}[1]{}

\let\nwline=\newline 
\renewcommand{\newline}{\nwline\mbox{}}

%\renewcommand{\baselinestretch}{1.47}
\renewcommand{\textwidth}{15truecm}
\renewcommand{\textheight}{23truecm}
\renewcommand{\oddsidemargin}{0.4truecm}
\renewcommand{\evensidemargin}{0.4truecm}
\renewcommand{\topmargin}{0.5truecm}

\newcommand{\C}{{\mathcal C}}
\newcommand{\E}{{\mathbb E}}
\newcommand{\F}{{\mathbb F}}
\newcommand{\Q}{{\mathbb Q}}
\newcommand{\R}{{\mathbb R}}
\newcommand{\U}{{\mathcal U}}
\newcommand{\Z}{{\mathbb Z}}

\newcommand{\GL}{{\rm GL}}
%\newcommand{\Eucl}{{\rm Eucl}}

\newcommand{\GAP}{{\sf GAP}}
\newcommand{\ledot}{< \!\!\! \cdot \;}
\newcommand{\LL}{\mathcal L}
\newcommand{\CC}{\mathcal C}
\newcommand{\RR}{\mathcal R}

\newtheorem{theorem}{Theorem}[section]
\newtheorem{definition}[theorem]{Definition}
\newtheorem{lemma}[theorem]{Lemma}
\newtheorem{remark}[theorem]{Remark}

\begin{document}

\title[Maximal Subgroups and Wyckoff Positions]
{Computing Maximal Subgroups and Wyckoff Positions of Space Groups}
\author{Bettina Eick}
\address{Lehrstuhl D f\"ur Mathematik,
         RWTH Aachen, D-52056 Aachen, Germany} 
\email{Bettina.Eick@math.RWTH-Aachen.de}
\author{Franz G\"ahler}
\address{Centre de Physique Th\'eorique, Ecole Polytechnique,
         F-91128 Palaiseau, France}
\email{gaehler@cpth.polytechnique.fr}
\author{Werner Nickel}
\address{School of Mathematical and Computational Sciences, 
         University of St~Andrews, St~Andrews, Fife KY16 9SS, Scotland}
\email{werner@dcs.st-and.ac.uk}

\subjclass{}
\keywords{space group, Wyckoff position, maximal subgroups}

\begin{abstract}
This   paper describes algorithms  for the  computation of conjugacy
classes of  maximal subgroups  and the Wyckoff positions of a space
group.  The algorithms are implemented in  the computational group  
theory system  {\GAP} and use  existing standard functions of {\GAP}  
as well as some  simple but useful group theoretical ideas.
\end{abstract}

\maketitle

\bigskip

\noindent{\bf Synopsis:} Algorithms for the computation of maximal 
subgroups and Wyckoff positions of space groups are given. 
Implementations of these algorithms are made publicly available.

\bigskip

\section{Introduction}
\label{sintro}

Traditionally, information about space groups has been made available
in the form of lists, for example in the International Tables for
Crystallography (1995) or the in list of all 4-dimensional space
groups (Brown et al., 1978). Space groups in dimensions higher than
three have important applications as symmetry groups of quasiperiodic
structures, such as incommensurately modulated crystals (Janssen \&
Janner, 1987; Janssen et al., 1992) and quasicrystals (Janot, 1994;
Gratias \& Hippert, 1994).  However, compiling large lists of
information poses many practical problems and is often impossible.  A
solution to this is to compute the desired information when the need
arises.  For this, algorithms are needed that can answer specific
questions about a given space group.  Because the computations
involved are often too complex for hand calculations, it is convenient
to have those algorithms in form of computer programs.

Many group theoretical algorithms are nowadays available as part of
group theory systems. In order to have these algorithms available, we
have built our programs on top of {\GAP} ({\GAP} Version 3.4.4, 1997),
a system which is freely availabe from a number of host machines
around the world. {\GAP} already contains a variety of algorithms
ready to use, so that we can avoid reinventing the wheel.  Moreover,
the current distribution of {\GAP} also contains the complete list of
all $3$- and $4$-dimensional space groups, as well as the lists of all
irreducible and maximal finite (i.m.f.) integral matrix groups of all
dimensions up to 24, as determined by W.~Plesken and collaborators
(see, e.g., Nebe and Plesken (1995), and references therein).
In prime dimensions and in all dimensions up to 11 these integral
matrix groups are available as $\Z$-class representatives, whereas in
the remaining dimensions only $\Q$-class representatives are
available. These lists of crystallographic groups provide plenty of
material to which our algorithms can be applied.

The algorithms described in this paper compute the Wyckoff positions
and the maximal subgroups of any given space group, independently of
its dimension. In practice, their efficiency is good enough to compute
conjugacy classes of maximal subgroups or the Wyckoff positions of all
3- and 4-dimensional space groups on an average personal computer in
little more than an hour (see Section \ref{sexam}). They can be seen
as a complement to the information contained in the space group tables
of {\GAP}. In addition, an implementation of the Zassenhaus algorithm
is also made available, which allows to determine space groups
also in dimensions larger than four. The point groups needed as input
can be determined from the list of maximal integral matrix groups
contained in {\GAP}.

An implementation of these algorithms can be obtained as a {\GAP} 
share package via anonymous ftp from

%\smallskip
\centerline{\tt ftp.math.rwth-aachen.de}

\noindent under

\centerline{\tt /pub/incoming/cryst.tar.gz.}
%\smallskip

\noindent
This package contains the algorithms as \GAP-code, manual pages and
examples. It also provides a variety of other functions for space
groups not mentioned here, so that almost the same functionality that
{\GAP} offers for other types of groups, e.g., permutation groups, is
now available also for space groups.

A different algorithm for computing the Wyckoff positions of a space
group has been given by Fuksa \& Engel (1994). Their implementation 
performs well on examples with small point groups. For examples with 
larger point groups, however, our method seems to be more efficient 
as the timings given in Section \ref{sexam} indicate.

A package for computations with space groups is also being developed
at the University of Nijmegen (Thiers et al., 1993). Results
obtained with this package, notably the Wyckoff positions for space
groups up to dimension 4, can be obtained in electronic form on
the World Wide Web (Thiers et al., 1996).

The remainder of the paper is organized as follows. Section
\ref{smath} describes the mathematical setup and recalls some basic
facts about space groups used throughout the paper. The computational
methods used together with short explanations are given in Section
\ref{scomp}.  Section \ref{swyck} explains the algorithm for the
computation of Wyckoff positions and Section \ref{smax} the algorithm
for computing conjugacy classes of maximal subgroups.  Finally, to
give an impression of their use and their efficiency, examples and
timings of their implementation in {\GAP} are given in Section
\ref{sexam}.

\section{Mathematical setup} 
\label{smath}

In this section we give a short overview on space groups, to fix the 
concepts and the notation. We also refer to the introductory chapter 
of (Brown et al., 1978), and to the article of Wondratschek (1995) 
in the International Tables.

A $d$-dimensional space group $S$ can be regarded as an affine group 
acting on $d$-dimensional Euclidean space $\E$. With respect to a 
fixed basis of the translation subgroup $T$ of $S$, the conjugation 
action of $S$ on $T$ induces a homomorphism from $S$ into $GL(d, \Z)$. 
The kernel of this homomorphism is $T$ itself; its image $P$ is called 
the point group\footnote{note that in crystallography the term 
``point group'' often denotes a whole $\Q$-class of matrix groups; 
here, however, a point group is a fixed representative of such a 
$\Q$-class} of $S$, which is a finite subgroup of $GL(d, \Z)$.

Each element of $s\in S$ can be represented by a matrix of the form
$$\left[\begin{array}{c|c}
  M_{s} & 0 \\
 \hline
  t_{s} & 1
\end{array}\right]$$ 
where $M_{s}$ is an element of $\GL(d,\Z)$ and $t_{s}\in \Q^d.$  In this 
representation, elements of the translation subgroup are matrices of the 
form
$$\left[\begin{array}{c|c}
   I_d & 0 \\
  \hline
   t & 1 
\end{array}\right]$$
where $I_d$ is the $d$-dimensional identity matrix and $t \in \Z^d$.
We call this representation of $S$ a {\em standard representation} of $S$.
In crystallographer's language, this corresponds to a setting with a
primitive cell.

Such a representation of $S$ defines an action (from the right)
of $S$ on the row space $\Q^d$ as affine maps via
$$\begin{array}{rcl}
        \Q^d\times S &   \to   & \Q^d\cr
            vs       & \mapsto & v M_s + t_s.\cr
\end{array}$$
The elements of the translation subgroup $T$ of $S$ act as translations on 
$\Q^d$. It is convenient to write an element $s$ of $S$ as a 
pair $(M_{s},t_{s}).$ 

Note that our convention of acting from the right differs from the
convention in the International Tables (1995), where the action is from 
the left on column vectors. We use this convention in order to maintain
compatibility with the rest of \GAP, where group action is always from the
right. Moreover, our notation does not distinguish between points
in Euclidean space, row vectors and column vectors.

The structure of a space group is best summarized by the
sequence of homomorphisms
$$
0 \to T \to S \to P \to I_d.
$$
This sequence is exact: the image of each of these homomorphisms 
is identical to the kernel of the following homomorphism. 
This structure of a space group is frequently used in many of 
our algorithms. In particular, the homomorphim $h:S\to P$ 
with kernel $T$ is of primary importance in the problems
dealt with in this paper.

\section{Computational methods}
\label{scomp}

This section explains the computational background for Sections
\ref{swyck} and \ref{smax}.  For definitions and explanations of basic
group theoretical terms the reader is referred to standard text books
on group theory, for example (Suzuki, 1982).
The two algorithms described in this paper take as input a set of 
generating matrices of a space group in standard representation. 
 From this the point group in the form of an integral matrix group 
can easily be extracted. It will prove convenient, however, 
to use also other representations of the point group. 
{\GAP} has built-in facilities that allow to switch easily
between different representations. In particular, we can convert the
point group $P$ into
\begin{itemize}
\item a permutation group
\item a finite presentation of $P$ 
\item a power-commutator presentation if $P$ is solvable.
\end{itemize}
These facilities allow to make use of algorithms that are much
more efficient than those available for matrix groups. The subgroup
lattice of the point group, for instance, is better computed for an 
isomorphic permutation group,
and then the result is translated back into the matrix group.
This is, in fact, the primary advantage of building our programs
on top of GAP, instead of writing a stand-alone package: there is a 
large variety of efficient, state-of-the-art algorithms available in 
\GAP, which are ready to use.  

In particular, our programs make use of routines built into {\GAP} to
compute
\begin{itemize}
\item the conjugacy classes of maximal subgroups of a finite group,
\item the conjugacy classes of subgroups of a finite group,
\item maximal submodules of a finite module for a finite group,
\item orbits of points in a set under the action of a group.
\end{itemize}

In addition to that we extended {\GAP}'s capabilities by implementing,
in the {\GAP}-programming language, functions for the computation of
\begin{itemize}
\item complements to elementary abelian normal subgroups in a finite 
group,
\item all solutions of a system of linear equations in $\Q/\Z.$
\end{itemize}
The complement routine essentially uses the method described in 
(Celler et al., 1990).

To compute the solutions (mod $\Z$) of a system of linear equations
$Mx=b$, where $M\in \Z^{m\times n}$, $b\in \Q^m$ and $x$ is a column 
vector with $n$ unknowns, we first determine unimodular 
matrices $P$ and $Q$ such that $PMQ = D$ is a matrix in diagonal form.  
This amounts to computing the Smith normal form including 
transformation matrices (see, e.g., Cohen (1993), Chapter 2, or
Sims (1994), Chaper 8). The set of all solutions (mod $\Z$) of 
the system $Dx= Pb$ is then obtained as follows.
Let $d_1,\ldots,d_n$ be the diagonal entries of $D$ and
$v_1,\ldots,v_m$ the components of $Pb.$ The system $Dx = Pb$ has
solutions if and only if $v_i=0$ whenever $d_i=0$.
If solutions exist, the set of solutions for $Dx=Pb$ 
is described by
$$
x_{i}\in\{0,\frac{1}{d_{i}},\ldots,\frac{d_{i}-1}{d_{i}}\} + 
\frac{v_{i}}{d_{i}}\quad \hbox{if}\quad d_{i}\not=0
$$
and
$$x_{i}\in\Q\quad\hbox{otherwise.}$$
The set of solutions of the system $Mx=b$ is now obtained from the
solutions of $Dx'\equiv PMQx'=Pb$, via $x=Qx'$.

\section{Computing Wyckoff positions}
\label{swyck}

In this section we describe an algorithm for the computation of Wyckoff 
positions of a space group. In the following, let $S$ be a space group, 
$T$ its translation subgroup and $\E$ the corresponding $d$-dimensional 
Euclidean space.  

\begin{definition}\label{defwyckoff} 
The {\em stabilizer} (or {\em site-symmetry group}) of a point $x\in\E$ 
is the subgroup of those elements of $S$ which leave $x$ invariant.
A {\em Wyckoff position} for $S$ is an equivalence class of all 
points $x\in\E$, whose stabilizers are conjugate subgroups of $S$.
\end{definition}

Let $A_0$ be the subset of those points having as stabilizer a fixed
representative of a conjugacy class of subgroups. Then $A_0$ is
contained in some affine subspace $A$ of $E$, such that it forms an
open, dense subset in $A$. The points in $A \setminus A_0$ have
stabilizers which are larger than the stabilizer of the points in
$A_0$. The full Wyckoff position is given by the space group orbit of
the set $A_0$. Clearly, all points in such an orbit have conjugate
stabilizers. Conversely, if $y$ is a point whose stabilizer is
conjugate to the common stabilizer of the points in $A_0$, there
exists a space group element which maps $y$ into $A_0$.  A Wyckoff
position therefore is completely specified by the representative set
$A_0$ of a space group orbit.

In the following, in line with the International Tables (1995)
(see, in particular, the article of Wondratschek (1995))
we shall not distinguish between the subset $A_0$ and the full affine 
subspace $A$, tacitly admitting that the points in some subset of 
lower dimension may have a larger stabilizer and non-characteristic
orbits. 
 
\begin{definition}\label{specpos}\label{wyckoff}
An affine subspace of $\E$ is  in {\em special position}  if its 
point-wise stabilizer in $S$ is non-trivial.  
\end{definition}

With the exception of the Wyckoff position with trivial stabilizer,
a Wyckoff position therefore consists of a space group orbit of
an affine subspace in special position.

\begin{definition}
Let $U$ be a subgroup of $S$.
An affine subspace $A\subseteq \E$ is {\em fixed by $U$ modulo $T$} 
if for all $u\in U$ there exists an element $t_{u}\in T$ such that 
$xu = xt_{u}$ for all $x\in A.$
\end{definition}

An affine subspace $A\subseteq \E$ is fixed by $U$ modulo $T$ if the orbit 
of $A$ under $T$ is fixed as a set by $U$.

\begin{lemma}\label{lspecial}
An affine subspace $A\subset\E$ is in special position if and only if $A$ 
is fixed modulo $T$ by some subgroup $T< U\leq S.$
\end{lemma}
\begin{proof} Let $A\subseteq\E$ be an affine subspace in special position 
and let $V\leq S$ be its stabilizer in $S.$  Then $U = VT$ fixes $A$ modulo 
$T.$

Let $U$ be a subgroup of $S$ properly containing $T,$ let 
$\{u_{1},\ldots,u_{k}\}$ be a generating set for $U$ and let 
$A\subseteq\E$ be fixed by $U$ modulo $T.$ Furthermore, choose 
$t_{1},\ldots,t_{k}\in T$ such that $vu_{i}=vt_{i}$ for all $v\in A.$ 
Then the subgroup $V$ generated by 
$u_{1}t_{1}^{-1},\ldots,u_{k}t_{k}^{-1}$ stabilizes $A$ pointwise and 
is not trivial.  Hence $A$ is in special position.
\end{proof}

The translation subgroup has no fixed points.  Therefore, a subgroup 
of $S$ that has a fixed point intersects $T$ trivially. Note that $V$ in 
the second part of the proof is a complement to $T$ in the subgroup 
$VT.$ The situation can be summarized by the diagram in Figure~1.

%%%%%%%%%%%%%%%%%%%%%%%%%%%%%%%%%%%%%%%%%%%%%%%%%%%%%%%%%%%%%%%%%%%%%%%%%%%%%
\medskip
\begin{figure}[h]
\begin{center}
\setlength{\unitlength}{0.01in}%
\begin{picture}(150,144)(255,575)
\thinlines
\put(280,660){\circle*{5}}
\put(280,580){\circle*{5}}
\put(340,600){\circle*{5}}
\put(340,680){\circle*{5}}
\put(400,700){\circle*{5}}
\put(280,660){\line( 0,-1){ 80}}
\put(280,580){\line( 3, 1){ 60}}
\put(340,600){\line( 0, 1){ 80}}
\put(280,660){\line( 3, 1){120}}
\put(255,575){\makebox(0,0)[lb]{$\{1\}$}}
\put(355,595){\makebox(0,0)[lb]{$V$}}
\put(300,690){\makebox(0,0)[lb]{$U=VT$}}
\put(405,710){\makebox(0,0)[lb]{$S$}}
\put(265,665){\makebox(0,0)[lb]{$T$}}
\end{picture}
\end{center}
\caption{The stabilizer $V$ of any point $x$ intersects the 
         translation subgroup $T$ trivially.}
\end{figure}
\medskip
%%%%%%%%%%%%%%%%%%%%%%%%%%%%%%%%%%%%%%%%%%%%%%%%%%%%%%%%%%%%%%%%%%%%%%%%%%%%%

Let $U$ in the previous lemma be given by a generating set 
$(M_1,t_1),\ldots,(M_k,t_k)$ in standard representation.
A point $v_{1}\in \R^{d}$ can be mapped to another point $v_{2}\in 
\R^{d}$ by a translation in $S$ if and only if their components differ by
an integer or, in other words, if $v_{1} \equiv v_{2}$ modulo $\Z.$

Computing the fixed points modulo $T$ amounts to solving the following system 
of linear equations for $v$ modulo $\Z:$
$$ v(M_{i}-{I_d}) = - t_{i}\qquad \mbox{for }1\leq i\leq k.$$
Transposing each equation we obtain a system
$$ M x = b $$
with an integer matrix $M$ and a rational vector $b$ for which we seek 
solutions modulo $\Z.$ The set of solutions of this system is a finite
union of affine subspaces of $\Q^{d}$, containing one representative
of each $T$-orbit of affine subspaces left invariant by $U$ modulo $T$.
Each affine subspace is in special position.  
For each subspace $A$ let $u_i\in \Z^d$ be such that 
$vM_i + t_i = v + u_i$ for all $v\in A.$ Then the elements 
$(M_1,t_1-u_1),\ldots,(M_k,t_k-u_k)$ generate a subgroup $V\leq S$ 
fixing $A$ pointwise.  The subgroup $V$ is the pointwise stabilizer of $A$
in $U$ and a complement to $T$ in $U.$ Note, however, that the full pointwise
stabilizer of $A$ in $S$ can be larger than $V$. This has to be checked
afterwards, by looking at the length of the space group orbit modulo $T$.

The main idea in computing the Wyckoff positions of $S$ now is to run 
through all subgroups (up to conjugacy) of $S$ containing $T$ and for 
each such subgroup $U$ to compute the set of affine subspaces of $\E$ 
fixed modulo $T$ by $U$.  By Lemma  \ref{lspecial} this gives all affine 
subspaces in special position, together with their point-wise stabilizers.

\vspace{0.5cm}
\noindent{\bf Wyckoff Positions of $S$}
\begin{enumerate}
\item \label{WCconj}
        Compute a set of representatives for the conjugacy
        classes of subgroups of $S/T$, and determine the set $\U$ 
        of their complete preimages in $S$.
\item\label{WCfixed}
   For each $U\in\U$:
   \begin{itemize}
      \item
        Determine the set of affine subspaces fixed by $U$ (modulo $T$), 
        one representative for each $T$-orbit of such spaces. 
      \item
        Eliminate multiple subspaces in the same $S$-orbit (modulo $T$).
      \item
        Retain only those spaces for which the $S$-orbit (modulo $T$) 
        has size $|S/U|$.
        Other subspaces must have a larger stabilizer,
        and will show up a second time, for a different $U$.
      \item
        For each of the remaining subspaces, determine its point-wise 
        stabilizer in $U$.
   \end{itemize}
   When $\U$ is exhausted, we have obtained representative affine 
   subspaces of all Wyckoff positions of $S$, together with their 
   point-wise stabilizers.
\end{enumerate}
\vspace{0.5cm}

The algorithm described above is dimension independent and works, 
given sufficient resources, for arbitrarily large groups and 
dimensions.  In practice, it is mainly limited by the size of the 
point group and the complexity of its subgroup lattice.
If the point group is too big to compute its subgroup lattice
completely, but some subgroups of the point group are known, one
can still compute a set $\U$ containing the preimages in $S$ of 
these subgroups of the point group, and go with this information 
directly into step 2. One then obtains all affine subspaces 
stabilized by some $U\in\U$ modulo $T$, and from these the 
corresponding Wyckoff positions. In this way, one can then obtain 
at least some partial information on the Wyckoff positions of $S$.

\section{Computing conjugacy classes of maximal subgroups}
\label{smax}

In this section an algorithm to compute conjugacy classes of maximal
subgroups of a given space group will be introduced. First we have to 
examine maximal subgroups of space groups from a more theoretical
point of view. The following definition will be helpful.

\begin{definition}
Let $U$ be a subgroup of the space group $S$ and let $T$ be the
translation subgroup of $S$.
\begin{itemize}
\item If $T \leq U$ holds, then the subgroup $U$ of $S$ is called 
      {\em lattice-equal} (or {\em translationengleich}).
\item If $T U = S$ holds, then the subgroup $U$ of $S$ is called 
      {\em class-equal} (or {\em klassengleich}).
\end{itemize}
\end{definition}

Recall that for any subgroup $U$ of $S$, the translation subgroup
$T_U$ of $U$ is just the intersection of $U$ and $T$. Therefore
$T_U = T$ holds, if $U$ is a lattice-equal subgroup of $S$. With respect
to a fixed basis of $T = T_U$, the point group of a lattice-equal subgroup
$U$ of $S$ is a subgroup
of the point group of $S$. If $U$ is a class-equal subgroup of $S$,
then $T_U \leq T$ and the point group of $U$ is $\Q$-equivalent to the
point group of $S$. 

Note that every subgroup of a space group may be obtained as a class-equal
subgroup of a lattice-equal subgroup of the space group. The space group
itself is both a class-equal and a lattice-equal subgroup. There
exist subgroups of a space group which are neither lattice-equal nor
class-equal. 
Maximal subgroups, however, are always either class-equal or lattic-equal. 
This is known as Herrmann's Theorem (Herrmann, 1929).

In the case that $M$ is a lattice-equal maximal subgroup of the space
group $S$, the point group of $M$ is a subgroup of the point group of
$S$ which has to be maximal. Thus we may obtain $M / T$ and therefore
$M$ by the methods for finite groups as described in Subsection
\ref{sslatt}.

In the case that $M$ is a class-equal maximal subgroup of the space
group $S$, its translation subgroup $T_M$ must be a proper
subgroup of $T$. In the next theorem, we investigate the connection
between $T$ and $T_M$ further. 

\begin{theorem}
\label{tpind}
Let $M$ be a class-equal maximal subgroup of $S$. Let $T$ be the
translation subgroup of $S$ and $T_M = T \cap M$ the translation 
subgroup of $M$. 
\begin{itemize}
\item[1.]
$T_M$ is a maximal $S$-invariant subgroup of $T$.
\item[2.]
The factor $T / T_M$ is an elementary abelian $p$-group for a prime $p$. 
In particular, the index of $T_M$ in $T$ is a $p$-power and thus finite. 
\item[3.]
$[S : M] = [T : T_M]$.

\end{itemize}
\end{theorem}

\begin{proof}
\begin{itemize}
\item[1.]
First we have to prove, that $T_M$ is invariant under conjugation
action of $S$. Let $s \in S$. We define $T_M^s := s^{-1} T_M s$. 
Since $S = T M$, we may write $s = t m$.
Since $T$ is abelian, we know that $T_M^t = T_M$ and thus we obtain 
$T_M^s = T_M^{tm} = (T_M^t)^m = T_M^m$. But $T_M$ is the translation
subgroup of $M$ and therefore we have that $T_M$ is normal in $M$.
So this yields $T_M^s = T_M^m = T_M$ and we obtain that $T_M$ is
$S$-invariant.

Now suppose there exists an $S$-invariant subgroup $L$ with
$T_M < L < T$. Since $L$ is normal in $S$, we have that $L M$
is a subgroup of $S$. However, this yields $S = T M > L M > T_M M = M$.
But this is a contradiction, since $M$ is a maximal subgroup of $S$.

\item[2.]
By 1 we know, that $T / T_M$ does not have any subgroup, which
is invariant under conjugation action of $S / T_M$. Since any
characteristic subgroup of $T / T_M$, i.e., any subgroup of $T / T_M$ which
is invariant under each automorphism of $T / T_M$, would be invariant under
action of $S / T_M$, the factor group $T / T_M$ cannot have any
characteristic subgroup. However, each finitely generated abelian
group without characteristic subgroups is of the form $(\Z / p \Z)^n$
for a prime $p$ and an integer $n$. I.e. $T / T_M$ is an
elementary abelian $p$-group of order $p^n$.

\item[3.]
Since $S / T_M$ is a finite group by 2, this follows directly 
from the homomorphism theorem.
\end{itemize}
\end{proof}

By Theorem \ref{tpind} a class-equal maximal subgroup of $S$ has
$p$-power index for a prime $p$. We call such a maximal subgroup
{\em $p$-maximal}. However, there may exist class-equal $p$-maximal
subgroups for infinitely many primes $p$. (In fact it is known from
group theory, that there exists at least one class-equal $p$-maximal
subgroup for each prime $p$ with $p \nmid |S / T|$, see Suzuki (1982),
Chapter 2, Theorem 8.10). Thus we 
cannot compute all class-equal maximal subgroups at once. Here we will 
restrict the algorithm to compute class-equal $p$-maximal subgroups
of a given $p$ only.

Let $M$ be a class-equal $p$-maximal subgroup of $S$. Then Theorem
\ref{tpind} shows that $T_M$ is a normal subgroup of $S$. Thus from
the group theoretical point of view, $M / T_M$ is a complement to
$T / T_M$ in $S / T_M$. The following lemma shows that this is in
fact a characterisation of class-equal $p$-maximal subgroups.

\begin{lemma}
Let $L$ be a maximal $S$-invariant subgroup of $T$ and let $M$ be
a subgroup of $S$ such that $M / L$ is a complement of $T/L$ in $S/L$.
Then $M$ is a maximal class-equal subgroup of $S$.
\end{lemma}

\begin{proof}
Suppose we have a group $K$ with $M < K < S$. Since $M$ is class-equal,
the subgroup $K$ is also class-equal. Thus we obtain $L = T_M < T_K < T$
and $T_K$ is $S$-invariant. But $L$ is maximal $S$-invariant in $T$ and
we have a contradiction.
\end{proof}

This characterisation of class-equal $p$-maximal subgroups of $S$ by 
complements will be used to compute these subgroups. The following lemma 
yields that there are only finitely many class-equal $p$-maximal subgroups 
of a given space group $S$ and fixed prime $p$. Futhermore it will lead to a
method to compute the possible translation subgroups $T_M$ for
class-equal $p$-maximal subgroups $M$.

\begin{lemma}
Let $S$ be a space group with translation subgroup $T \cong \Z^d$.
For a fixed prime $p$ we consider the subgroup $T^p$ consisting
of all $p$-th powers of elements of $T$, i.e. $T^p \cong (p \Z)^d$.
\begin{itemize}
\item[1.] $T / T^p$ is an elementary abelian group of order $p^d$.
\item[2.] $T^p$ is normal in $S$.
\item[3.] Let $M$ be a class-equal $p$-maximal subgroup of $S$.
          Then $T^p \leq T_M < T$ holds.
\end{itemize}
\end{lemma}

\begin{proof}
\begin{itemize}
\item[1.]
Since $T / T^p \cong \Z^d / (p \Z)^d \cong (\Z / p \Z)^d$, we obtain
that $T / T^p$ is an elementary abelian group of order $p^d$.
\item[2.]
Since $T^p$ is a characteristic subgroup of $T$ and $T$ is normal
in $S$, we obtain that $T^p$ has to be normal in $S$.
\item[3.]
As explained above $T_M < T$ holds. Thus it remains to prove $T^p \leq
T_M$. By Theorem \ref{tpind}, the factor $T / T_M$ is an elementary
abelian $p$-group. But $T^p$ is the smallest subgroup of $T$ which
has an elemenary abelian factor group of $p$-power order and thus
$T^p \leq T_M$ holds.
\end{itemize}
\end{proof}

So let $M$ be a class-equal $p$-maximal subgroup of a space group
$S$ for a prime $p$. Then the location of $M$ in $S$ is as illustrated 
in Figure~2.

%%%%%%%%%%%%%%%%%%%%%%%%%%%%%%%%%%%%%%%%%%%%%%%%%%%%%%%%%%%%%%%%%%%%%%%%%%%%%
\medskip
\begin{figure}[h]
\begin{center}
\setlength{\unitlength}{0.01in}%
\begin{picture}(150,179)(185,495)
\thicklines
\put(220,620){\circle*{5}}
\put(220,580){\circle*{5}}
\put(220,540){\circle*{5}}
\put(220,500){\circle*{5}}
\put(320,620){\circle*{5}}
\put(321,661){\circle*{5}}
\put(220,620){\line( 0,-1){120}}
\put(220,620){\line( 5, 2){100}}
\put(320,660){\line( 0,-1){ 40}}
\put(320,620){\line(-5,-2){100}}
\put(332,615){\makebox(0,0)[lb]{$M$}}
\put(332,665){\makebox(0,0)[lb]{$S = T M$}}
\put(200,615){\makebox(0,0)[lb]{$T$}}
\put(172,575){\makebox(0,0)[lb]{$M \cap T$}}
\put(193,535){\makebox(0,0)[lb]{$T^p$}}
\put(190,495){\makebox(0,0)[lb]{$\{1\}$}}
\end{picture}
\end{center}
\caption{The location of a class-equal $p$-maximal subgroup $M$.}
\end{figure}
\medskip
%%%%%%%%%%%%%%%%%%%%%%%%%%%%%%%%%%%%%%%%%%%%%%%%%%%%%%%%%%%%%%%%%%%%%%%%%%%%%

Now we are ready to introduce the algorithms to compute maximal subgroups
of a given space group.

\subsection{Lattice-equal maximal subgroups}
\label{sslatt}

To compute the lattice-equal maximal subgroups of a space group $S$
with translation subgroup $T$ we first obtain a permutation
representation of $S/T$. If $S/T$ is solvable, we then compute a 
power-commutator presentation of this permutation group, which 
allows to use a very efficient method to compute the conjugacy 
classes of maximal subgroups of $S/T$ (Cannon \& Leedham-Green; 
Eick, 1993). It then remains to compute preimages of the generators  
of the maximal subgroups in $S$. 

If $S/T$ is not solvable, we use the generic {\GAP} method to 
compute the conjugacy classes of  maximal subgroups of a permutation
group. This algorithm will compute all conjugacy classes of subgroups
of the given permutation group (which of course would yield all 
lattice-equal subgroups of $S$) and then reduce them to the maximal 
ones. It is evident that this involves a lot of overhead and will 
not be as efficient as the algorithm for the solvable space groups.

\subsection{Class-equal $p$-maximal subgroups}
\label{ssclas}

Let $S$ be a space group and $p$ a fixed prime. The computation of the 
conjugacy classes of class-equal $p$-maximal subgroups in the general
case is done in two parts. We will describe an alternative method for
solvable space groups afterwards. 

First compute all possible candidates for the translation subgroup 
of these maximal subgroups. By Theorem \ref{tpind} this amounts to 
the computation of all maximal $S$-invariant subgroups of $T$ with 
$p$-power index. These maximal subgroups will contain $T^p$, and
therefore we may as well compute all maximal $S$-invariant subgroups
of $T / T^p$. So we compute the linear matrix group in $GL(d, p)$
induced by the conjugation action of $S$ on $T / T^p$, and then 
determine the maximal $S$-invariant subgroups of the elementary 
abelian group $T / T^p$. It then remains to determine preimages of 
the computed subgroups in $T$.

Now suppose we have a maximal  $S$-invariant subgroup $L$ of $T$
which has $p$-power index. We need to compute the conjugacy
classes of complements to $T / L$ in $S / L$. We first determine 
a finite presentation of $S / T$, from which we obtain the conjugacy 
classes of complements to $T / L$ in $S / L$ by the method mentioned 
in Section 3. It remains to compute their preimages in $S$ to obtain 
the conjugacy classes of class-equal $p$-maximal subgroups of $S$ with
translation subgroup $L$.

If $S$ is a solvable space group, we can use the following alternative
method which is sometimes more efficient. As outlined above, in
this case we can compute a power-commutator presentation of $S / T$.
This may be extended to a power-commutator presentation of $S / T^p$.
By a slight modification of the routine to compute all conjugacy
classes of maximal subgroups in a group given by a power-commutator
presentation, we may then just compute the conjugacy classes of maximal
subgroups which do not contain $T / T^p$. It remains to compute
preimages of the computed subgroups in $S$.


%%%%%%%%%%%%%%%%%%%%%%%%%%%%%%%%%%%%%%%%%%%%%%%%%%%%%%%%%%%%%%%%%%%%%%%%%%
\section{Examples and timings}
\label{sexam}

In this section we give some examples and the timings for these
examples. All timings have been measured on a PC with a 133 Mhz Pentium
processor running FreeBSD Unix, Version 2.1.0. The timings are given in
seconds.

The first four examples are extracted from the {\GAP} library.
$S_{(3,230)}$ is the threedimensional non-symmorphic cubic space 
group $Ia\bar{3}d$, Nr.\ 230 in the International Tables (1995). 
Its point group has size 48. The next three space groups are 
fourdimensional. Their labels are those from (Brown et al., 1978), 
which are used also in the {\GAP} library. $S_{(4,22,10,1,2)}$ is 
non-symmorphic as well, and has a solvable point group of size $36.$ 
$S_{(4,31,7,1,1)}$ is symmorphic, with a non-solvable point group of 
size $240$, and $S_{(4,33,16,1,1)}$ is symmorphic as well, with a 
solvable point group of size 1152. The next two examples are 
sixdimensional space groups relevant for the description of the 
symmetry of quasicrystals. We take the two non-symmorphic space 
groups for the primitive icosahedral lattice (Levitov \& Rhyner,
1987). $S_{(6,1)}$ has a point group of size 60, isomorphic to $A_5$, 
whereas $S_{(6,2)}$ has a point group of size 120, isomorphic to 
$A_5\times C_2$. Finally, we consider an eightdimensional
space group $S_{(8)}$ with (non-solvable) point group isomorphic to 
the Coxeter group $H_4$, whose order is 14400. $H_4$ is the symmetry 
group of the famous 660-cell polytope in four dimensions. A 
fourdimensional matrix representation, with matrix entries in 
$\Z[\tau]$, where $\tau=(1+\sqrt{5})/2$ is the golden mean, is 
contained in {\GAP}'s share package Weyl. This fourdimensional
representation can easily be lifted to an eightdimensional
integral representation of $H_4$, of which we take the semi-direct
product with $\Z^8$ as our example space group $S_{(8)}$.

The computations have all been done in a workspace of 15Mb, with two
notable exceptions. For the computation of the Wyckoff positions and
the lattice-equal subgroups of $S_{(8)}$ a larger workspace was
necessary. In those two cases we have used a workspace of 30Mb.
The workspace used is in all cases considerably larger than absolutely
necessary. The computations could have been done in a smaller
workspace, at the expense of a somewhat longer runtime, due to the
more frequent garbage collections.

Table~1 contains the timings for the computation of the Wyckoff 
positions for all our examples. For each group, we give the timing 
for our program ($t_{EGN}$), and compare it to the runtime for the 
program of Fuksa \& Engel (1994) ($t_{FE}$).
In two cases, such a comparison was not possible, as the latter program
could not finish without exceeding the available memory (100Mb).

%\medskip
\begin{table}[h]
\begin{center}
\begin{tabular}{|l||r|c|c|r|r|}
\hline
space group         & size & solvable & symm. & $t_{EGN}$ & $t_{FE}$ \\
\hline
$S_{(3,230)}$       &      48 &  yes  &   no &   1.1   &    0.6     \\
$S_{(4,22,10,1,2)}$ &      36 &  yes  &   no &   0.9   &    2.8     \\
$S_{(4,31, 7,1,1)}$ &     240 &  no   &  yes &   7.1   &  430.0     \\
$S_{(4,33,16,1,1)}$ &    1152 &  yes  &  yes &  56.9   &            \\
$S_{(6,1)}$         &      60 &  no   &   no &   0.9   &   35.1     \\
$S_{(6,2)}$         &     120 &  no   &   no &   2.8   &  162.8     \\
$S_{(8)}$           &   14400 &  no   &  yes & 397.2   &            \\
\hline
\end{tabular}
\end{center}
\bigskip
\caption{Timings for the computation of Wyckoff positions.}
\end{table}

The computation of the Wyckoff positions and lattice-equal 
maximal subgroups of space group $S_{(8)}$ was not fully
automatic. For these computations the subgroup lattice of the 
point group is needed, for which {\GAP} needs a list of all perfect 
subgroups of the point group. Perfect subgroups are available in a 
format accessible by {\GAP}'s {\tt Lattice} command only up to size 5000.
The solvable residuum of the point group of $S_{(8)}$ has size 7200, 
however. The solution was to determine the perfect subgroups in a 
first step, and store them in the knowledge of the point group.
The rest of the computation is then fully automatic.
Since all perfect groups with size a divisor of 7200 are generated 
by two elements, and for each of these groups the order of these 
generators is known, we could obtain the list of perfect subgroups
by scanning through all subgroups having such a generating system,
and checking whether they are perfect. We actually need only one 
such subgroup per conjugacy class. The time for the determination
of the perfect subgroups is included in the timings. This example 
shows that even space groups with very large point groups can be 
handled by our programs.

If Wyckoff positions for several space groups in the same $\Z$-class 
or $\Q$-class are needed, it is possible to compute the subgroup
lattice of the point group only for one of these space groups, and 
use it for the other ones as well. With this trick, we can compute 
the Wyckoff positions of all 230 threedimensional space groups in 
62 seconds, and those of all 4783 fourdimensional space groups in 
4284 seconds.

In Table 2 we show for the same seven space groups the timings for 
the computation of maximal subgroups. In each case, the time required
to compute the lattice-equal maximal subgroups is given, as well
as the time to compute the class-equal $p$-maximal subgroups for
several prime numbers $p$. Also the number of the computed
conjugacy classes of maximal subgroups is included in the table.
All timings include the computation of the point group (if necessary) 
as well as the computation of the presentation that is used.

\begin{table}[h]
\begin{center}
\begin{tabular}{|l|| r| r| r| r| r| r| r| r| r| r|}
\hline
      & \multicolumn{2}{l|}{lattice} 
      & \multicolumn{2}{l|}{class, 2} 
      & \multicolumn{2}{l|}{class, 3} 
      & \multicolumn{2}{l|}{class, 5} 
      & \multicolumn{2}{l|}{class, 7}  \\
\hline
      space group
      & time & $\#$ & time & $\#$ & time & $\#$ & time & $\#$ 
      & time & $\#$ \\
\hline
$S_{(3,230)}$       & 0.3 & 5 &  0.2 & 0 &  0.2 & 1 &  0.3 & 1 &  0.3 & 1 \\
$S_{(4,22,10,1,2)}$ & 0.3 & 5 &  0.4 & 2 &  0.3 & 0 &  0.5 & 2 &  0.7 & 2 \\
$S_{(4,31, 7,1,1)}$ & 6.7 & 6 &  0.5 & 1 &  0.5 & 1 &  0.4 & 1 &  0.5 & 1 \\
$S_{(4,33,16,1,1)}$ & 2.2 & 6 &  2.1 & 1 &  2.3 & 1 &  2.4 & 1 &  2.4 & 1 \\
$S_{(6,1)}$         & 0.7 & 3 &  0.4 & 1 &  1.0 & 1 &  0.5 & 0 &  0.3 & 1 \\
$S_{(6,2)}$       &   1.9 & 4 &  0.4 & 2 &  0.2 & 1 &  0.4 & 1 &  0.3 & 1 \\
$S_{(8)}$         & 408.5 & 6 &  4.7 & 1 &  4.9 & 1 &  4.8 & 1 &  5.0 & 1 \\
\hline
\end{tabular}
\end{center}
\bigskip
\caption{Timings for the computation of maximal subgroups.}
\end{table}


\section{Acknowledgements}
We would like to thank H.~Wondratschek for suggesting the problem of
computing the maximal subgroups of a space group.  We are indebted to
J.~Neub\"user for many fruitful discussions and careful reading of this
paper.  The first and third authors acknowledge financial support from
the Graduiertenkolleg ``Analyse und Konstruktion in der Mathematik''.
The second author was supported by the Swiss Bundesamt f\"ur Bildung
und Wissenschaft in the framework of the HCM programme of the 
European Community.
This collaboration was in part made possible by financial support from
the HCM project ``Computational Group Theory''.


\begin{thebibliography}{}

\bibitem[]{BrownBulowetal78}
   Brown, H., B{\"u}low, R., Neub{\"u}ser J., Wondratschek, H.\ 
   \& Zassenhaus, H.\ (1978).
   \newline\qquad
   Crystallographic groups of four-dimensional space.
   Wiley-Interscience.

\bibitem[]{CannonLeedham}
   Cannon, J.\ \& Leedham-Green, C.\ R.
   Presentations of finite soluble groups.
   In preparation.

\bibitem[]{CellerNeubuseretal90}
   Celler, F., Neub{\"u}ser, J.\ \& Wright, C.\ (1990).
   Acta Appl.\ Math.\ 21, 57--76.

\bibitem[]{Cohen93}
   Cohen, H.\ (1993).
   A Course in Computational Algebraic Number Theory.
   Springer-Verlag.

\bibitem[]{eick93}
   Eick, B.\ (1993).
   Spezielle PAG-Systeme im Computeralgebrasystem GAP.
   \newline\qquad
   Diploma Thesis at Lehrstuhl D f\"ur Mathematik, RWTH Aachen.

\bibitem[]{EngelFuksa94}
   Fuksa, J.\ \& Engel, P.\ (1994).
   Acta Cryst.\ A50, 778--792.

\bibitem[]{GAP}
   {\sf GAP} -- Groups, Algorithms and Programming, Version 3.4.4 (1997). 
   \newline\qquad
   M.~Sch\"onert et al., Lehrstuhl D f\"ur Mathematik, RWTH Aachen.
   \newline\qquad
   {\sf GAP} can be obtained by anonymous ftp from 
   {\tt ftp.math.rwth-aachen.de}.
   \newline\qquad
   See also on the World Wide Web at 
   {\tt http://www-gap.dcs.st-and.ac.uk/\~{}gap/}.

\bibitem[]{GraHip94}
   Gratias, D.\ \& Hippert, F.\ (1994), editors.
   Lectures on Quasicrystals.
   \newline\qquad
   Les Editions de Physique and Springer-Verlag.

\bibitem[]{Hahn95}
   International Tables for Crystallography, Volume A (1995),
   4th edition, edited by T.\ Hahn. 
   \newline\qquad
   Kluwer Academic Publishers.

\bibitem[]{Hermann29}
   Hermann, C.\ (1929).
   Z.\ Krist.\ 69, 533--555.

\bibitem[]{Janot94}
   Janot, C.\ (1994).
   Quasicrystals: A Primer, 2nd edition
   (Monographs on the physics and
   \newline\qquad
   chemistry of materials, vol.~50). Oxford University Press.

\bibitem[]{JansJan87}
   Janssen, T.\ \& Janner, A.\ (1987).
   Adv.\ Phys.\ 36, 519--624.

\bibitem[]{JansJan92}
   Janssen, T., Janner, A., Looijenga-Vos, A.\ \& de~Wolff P.~M.\ (1992).
   \newline\qquad
   International Tables for Crystallography, Volume C, edited by
   A.~J.~C.~Wilson, pp. 797--844. 
   \newline\qquad
   Kluwer Academic Publishers.

\bibitem[]{LevRhy87}
   Levitov, L.~S.\ \& Rhyner, J.\ (1988).
   J.\ Phys.\ (Paris) 49, 1835--1849.

\bibitem[]{Nebe95}
   Nebe, G.\ \& Plesken, W.\ (1995).
   Finite Rational Matrix Groups.
   Memoirs of the American Mathematical Society, Vol.\ 116, Nr.\ 556.

\bibitem[]{Sims94}
   Sims, C.\ (1994).
   Computation with finitely presented groups.
   Cambridge University Press.

\bibitem[]{Suzuki82}
   Suzuki, M.\ (1982).
   Group Theory I. 
   Grundlehren Math.\ Wiss., Vol.\ 247.
   Springer-Verlag.

\bibitem[]{Thiers93}
   Thiers, A.~H.~M., Ephraim, M.~J., Janssen, T.\ and Janner, A.\ (1993).
   \newline\qquad
   Comp.\ Physics Comm.\ 77, 167--189

\bibitem[]{Thiers96}
   Thiers, A.~H.~M., Ephraim, M.~J., \& de Hilster, H.\ (1996).
   \newline\qquad{\tt http://www.caos.kun.nl/cgi-bin/csecm/csecm}

\bibitem[]{Wondra95}
   Wondratschek, H. (1995).
   Introduction to space-group symmetry, in International Tables for 
   \newline\qquad
   Crystallography, Volume A (1995),
   4th edition, edited by T.\ Hahn, pp. 711--735. 
   \newline\qquad
   Kluwer Academic Publishers. pp. 711--735.

\end{thebibliography}


\ifx\undefined\bysame
\newcommand{\bysame}{\leavevmode\hbox to3em{\hrulefill}\,}
\fi

\end{document}

