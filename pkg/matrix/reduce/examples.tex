%%%%%%%%%%%%%%%%%%%%%%%%%%%%%%%%%%%%%%%%%%%%%%%%%%%%%%%%%%%%%%%%%%%%%%%%%
\Section{Examples II}

We illustrate some of these functions in the following examples.

*Example 1* 

|gap> # Construct SL(2, 5) wr S2
gap> G := ImprimitiveWreathProduct( SL(2, 5), SymmetricGroup(2) );;
gap> # Apply RecogniseMatrixGroup to G
gap> x := RecogniseMatrixGroup( G );;
&I  Input group has dimension 4 over GF(5)
&I  Layer number 1: Type = "Unknown"
&I  Size = 1, # of matrices = 3
&I  Computing the next quotient
&I  <new> acts non-trivially on the block of dim 4

&I  Found a quotient of dim 4
&I  Restarting after finding a decomposition
&I  Layer number 1: Type = "Perm"
&I  Size = 1, # of matrices = 3
&I  Submodule is invariant under <new>
&I  Enlarging quotient, old size = 1

&I  Is quotient classical?
&I  No classical form has been found
&I  The group probably does not contain a classical group

&I  Is quotient primitive?
&I  Matrix group is imprimitive
&I  Number of blocks is 2
&I  Group primitive? false
&I  The quotient is imprimitive
&I  Permutation group has degree 2
&W  Applying Size to (permutation group) quotient
&I  New size = 2
&I  Adding generator relations to the kernel
&I  Layer number 2: Type = "Unknown"
&I  Size = 1, # of matrices = 2
&I  Computing the next quotient
&I  <new> acts non-trivially on the block of dim 4

&I  Found a quotient of dim 1
&I  Restarting after finding a decomposition
&I  Layer number 2: Type = "Perm"
&I  Size = 1, # of matrices = 2
&I  Submodule is invariant under <new>
&I  Using a permutation representation
&I  Adding generator relations to the kernel
&I  Layer number 3: Type = "Unknown"
&I  Size = 1, # of matrices = 2
&I  Computing the next quotient
&I  <new> acts non-trivially on the block of dim 3

&I  Found a quotient of dim 1
&I  Restarting after finding a decomposition
&I  Layer number 3: Type = "Perm"
&I  Size = 1, # of matrices = 2
&I  Submodule is invariant under <new>
&I  Using a permutation representation
&I  Adding generator relations to the kernel
&I  Layer number 4: Type = "Unknown"
&I  Size = 1, # of matrices = 2
&I  Computing the next quotient
&I  <new> acts non-trivially on the block of dim 2

&I  Found a quotient of dim 2
&I  Restarting after finding a decomposition
&I  Layer number 4: Type = "Perm"
&I  Size = 1, # of matrices = 2
&I  Submodule is invariant under <new>
&I  Enlarging quotient, old size = 1

&I  Is quotient classical?
&I  Dimension of group is <= 2, you must supply form
&I  The quotient contains SL
&I  New size = 120
&I  Adding generator relations to the kernel
&I  Restarting after enlarging the quotient
&I  Layer number 4: Type = "SL"
&I  Size = 120, # of matrices = 0
&I  Using the SL recognition
&I  Adding random relations at layer number 4
&I  Adding a random relation at layer number 4
&I  Adding a random relation at layer number 4
&I  Adding a random relation at layer number 4
&I  Adding a random relation at layer number 4
&I  Adding a random relation at layer number 4
&I  Adding a random relation at layer number 4
&I  Adding a random relation at layer number 4
&I  Adding a random relation at layer number 4
&I  Adding a random relation at layer number 4
&I  Adding a random relation at layer number 4
&I  Kernel is finished, size = 120
&I  Restarting after enlarging the quotient
&I  Layer number 1: Type = "Imprimitive"
&I  Size = 240, # of matrices = 0
&I  Using a permutation representation
&I  Adding random relations at layer number 1
&I  Adding a random relation at layer number 1
&I  Layer number 2: Type = "Perm"
&I  Size = 120, # of matrices = 4
&I  Submodule is not invariant under <new>
&I  Restarting at layer number 2
&I  Layer number 2: Type = "Unknown"
&I  Size = 1, # of matrices = 4
&I  Computing the next quotient
&I  <new> acts non-trivially on the block of dim 4

&I  Found a quotient of dim 2
&I  Restarting after finding a decomposition
&I  Layer number 2: Type = "Perm"
&I  Size = 1, # of matrices = 4
&I  Submodule is invariant under <new>
&I  Enlarging quotient, old size = 1

&I  Is quotient classical?
&I  Dimension of group is <= 2, you must supply form
&I  The quotient contains SL
&I  New size = 120
&I  Adding generator relations to the kernel
&I  Layer number 3: Type = "Unknown"
&I  Size = 1, # of matrices = 2
&I  Computing the next quotient
&I  <new> acts non-trivially on the block of dim 2

&I  Found a quotient of dim 2
&I  Restarting after finding a decomposition
&I  Layer number 3: Type = "Perm"
&I  Size = 1, # of matrices = 2
&I  Submodule is invariant under <new>
&I  Enlarging quotient, old size = 1

&I  Is quotient classical?
&I  Dimension of group is <= 2, you must supply form
&I  The quotient contains SL
&I  New size = 120
&I  Adding generator relations to the kernel
&I  Restarting after enlarging the quotient
&I  Layer number 3: Type = "SL"
&I  Size = 120, # of matrices = 0
&I  Using the SL recognition
&I  Adding random relations at layer number 3
&I  Adding a random relation at layer number 3
&I  Adding a random relation at layer number 3
&I  Adding a random relation at layer number 3
&I  Adding a random relation at layer number 3
&I  Adding a random relation at layer number 3
&I  Adding a random relation at layer number 3
&I  Adding a random relation at layer number 3
&I  Adding a random relation at layer number 3
&I  Adding a random relation at layer number 3
&I  Adding a random relation at layer number 3
&I  Kernel is finished, size = 120
&I  Restarting after enlarging the quotient
&I  Layer number 2: Type = "SL"
&I  Size = 14400, # of matrices = 0
&I  Using the SL recognition
&I  Adding random relations at layer number 2
&I  Adding a random relation at layer number 2
&I  Layer number 3: Type = "SL"
&I  Size = 120, # of matrices = 3
&I  Submodule is invariant under <new>
&I  Using the SL recognition
&I  Adding generator relations to the kernel
&I  Adding a random relation at layer number 2
&I  Layer number 3: Type = "SL"
&I  Size = 120, # of matrices = 3
&I  Submodule is invariant under <new>
&I  Using the SL recognition
&I  Adding generator relations to the kernel
&I  Adding a random relation at layer number 2
&I  Layer number 3: Type = "SL"
&I  Size = 120, # of matrices = 3
&I  Submodule is invariant under <new>
&I  Using the SL recognition
&I  Adding generator relations to the kernel
&I  Adding a random relation at layer number 2
&I  Layer number 3: Type = "SL"
&I  Size = 120, # of matrices = 3
&I  Submodule is invariant under <new>
&I  Using the SL recognition
&I  Adding generator relations to the kernel
&I  Adding a random relation at layer number 2
&I  Layer number 3: Type = "SL"
&I  Size = 120, # of matrices = 3
&I  Submodule is invariant under <new>
&I  Using the SL recognition
&I  Adding generator relations to the kernel
&I  Adding a random relation at layer number 2
&I  Layer number 3: Type = "SL"
&I  Size = 120, # of matrices = 3
&I  Submodule is invariant under <new>
&I  Using the SL recognition
&I  Adding generator relations to the kernel
&I  Adding a random relation at layer number 2
&I  Layer number 3: Type = "SL"
&I  Size = 120, # of matrices = 3
&I  Submodule is invariant under <new>
&I  Using the SL recognition
&I  Adding generator relations to the kernel
&I  Adding a random relation at layer number 2
&I  Layer number 3: Type = "SL"
&I  Size = 120, # of matrices = 3
&I  Submodule is invariant under <new>
&I  Using the SL recognition
&I  Adding generator relations to the kernel
&I  Adding a random relation at layer number 2
&I  Layer number 3: Type = "SL"
&I  Size = 120, # of matrices = 3
&I  Submodule is invariant under <new>
&I  Using the SL recognition
&I  Adding generator relations to the kernel
&I  Adding a random relation at layer number 2
&I  Layer number 3: Type = "SL"
&I  Size = 120, # of matrices = 3
&I  Submodule is invariant under <new>
&I  Using the SL recognition
&I  Adding generator relations to the kernel
&I  Kernel is finished, size = 14400
&I  Adding a random relation at layer number 1
&I  Layer number 2: Type = "SL"
&I  Size = 14400, # of matrices = 4
&I  Submodule is invariant under <new>
&I  Using the SL recognition
&I  Adding generator relations to the kernel
&I  Layer number 3: Type = "SL"
&I  Size = 120, # of matrices = 4
&I  Submodule is invariant under <new>
&I  Using the SL recognition
&I  Adding generator relations to the kernel
&I  Adding a random relation at layer number 1
&I  Layer number 2: Type = "SL"
&I  Size = 14400, # of matrices = 4
&I  Submodule is invariant under <new>
&I  Using the SL recognition
&I  Adding generator relations to the kernel
&I  Layer number 3: Type = "SL"
&I  Size = 120, # of matrices = 4
&I  Submodule is invariant under <new>
&I  Using the SL recognition
&I  Adding generator relations to the kernel
&I  Adding a random relation at layer number 1
&I  Layer number 2: Type = "SL"
&I  Size = 14400, # of matrices = 4
&I  Submodule is invariant under <new>
&I  Using the SL recognition
&I  Adding generator relations to the kernel
&I  Layer number 3: Type = "SL"
&I  Size = 120, # of matrices = 4
&I  Submodule is invariant under <new>
&I  Using the SL recognition
&I  Adding generator relations to the kernel
&I  Adding a random relation at layer number 1
&I  Layer number 2: Type = "SL"
&I  Size = 14400, # of matrices = 4
&I  Submodule is invariant under <new>
&I  Using the SL recognition
&I  Adding generator relations to the kernel
&I  Layer number 3: Type = "SL"
&I  Size = 120, # of matrices = 4
&I  Submodule is invariant under <new>
&I  Using the SL recognition
&I  Adding generator relations to the kernel
&I  Adding a random relation at layer number 1
&I  Layer number 2: Type = "SL"
&I  Size = 14400, # of matrices = 4
&I  Submodule is invariant under <new>
&I  Using the SL recognition
&I  Adding generator relations to the kernel
&I  Layer number 3: Type = "SL"
&I  Size = 120, # of matrices = 4
&I  Submodule is invariant under <new>
&I  Using the SL recognition
&I  Adding generator relations to the kernel
&I  Adding a random relation at layer number 1
&I  Layer number 2: Type = "SL"
&I  Size = 14400, # of matrices = 4
&I  Submodule is invariant under <new>
&I  Using the SL recognition
&I  Adding generator relations to the kernel
&I  Layer number 3: Type = "SL"
&I  Size = 120, # of matrices = 3
&I  Submodule is invariant under <new>
&I  Using the SL recognition
&I  Adding generator relations to the kernel
&I  Adding a random relation at layer number 1
&I  Layer number 2: Type = "SL"
&I  Size = 14400, # of matrices = 4
&I  Submodule is invariant under <new>
&I  Using the SL recognition
&I  Adding generator relations to the kernel
&I  Layer number 3: Type = "SL"
&I  Size = 120, # of matrices = 4
&I  Submodule is invariant under <new>
&I  Using the SL recognition
&I  Adding generator relations to the kernel
&I  Adding a random relation at layer number 1
&I  Layer number 2: Type = "SL"
&I  Size = 14400, # of matrices = 4
&I  Submodule is invariant under <new>
&I  Using the SL recognition
&I  Adding generator relations to the kernel
&I  Layer number 3: Type = "SL"
&I  Size = 120, # of matrices = 4
&I  Submodule is invariant under <new>
&I  Using the SL recognition
&I  Adding generator relations to the kernel
&I  Kernel is finished, size = 28800
gap>  DisplayMatRecord( x );
&I  Matrix group over field GF(5) of dimension 4 has size 28800
&I  Number of layers is 4
gap> DisplayMatRecord( x, 1 );
&I  Layer Number = 1
&I  Type = Imprimitive
&I  Dimension = 4
&I  Size = 2
gap> # G is imprimitive. The size is the size of the permutation group
gap> # given by the action of G on the blocks.
gap> DisplayMatRecord( x, 2 );
&I  Layer Number = 2
&I  Type = SL
&I  Dimension = 2
&I  Size = 120
gap> # Now we are looking at the decomposition of the kernel of the action.
gap> # Call this kernel K. The module for K splits into an irreducible
gap> # submodule of dimension 2 and a quotient module of dimension 2. The
gap> # restriction of K to the submodule contains SL(2, 5). Call this group K1.
gap> DisplayMatRecord( x, 3 );
&I  Layer Number = 3
&I  Type = SL
&I  Dimension = 2
&I  Size = 120
gap> # We have now taken relations on K1 and evaluated them in K to get
gap> # a group H. H is the kernel of the homomorphism from K to K1. The group
gap> # generated by the last 2x2 block on the diagonal of the matrices of H
gap> # has an irreducible module and contains SL(2, 5). Call this group H1.
gap> DisplayMatRecord( x, 4 );
&I  Layer Number = 4
&I  Type = PGroup
&I  Dimension = 4
&I  Size = 1
gap> # We have now taken relations on H1 and evaluated them in H to get the
gap> # kernel of the homomorphism from H to H1. It is trivial. | 

*Example 2*

|gap> # Construct the group SL(2, 3) x SP(4, 3)
gap> G1 := SL(2, 3);;
gap> G2 := SP(4, 3);;
gap> m1 := DiagonalMat_mtx( GF(3), G1.1, G2.1 );;
gap> m2 := DiagonalMat_mtx( GF(3), G1.2, G2.2 );;
gap> # Put something in the bottom left hand corner to give us a p-group
gap> m1[3][1] := Z(3)^0;;
gap> m2[5][2] := Z(3);;
gap> G := Group( [m1, m2], m1^0 );;
gap> # Apply RecogniseMatrixGroup to G
gap> x := RecogniseMatrixGroup( G );;
&I  Input group has dimension 6 over GF(3)
&I  Layer number 1: Type = "Unknown"
&I  Size = 1, \# of matrices = 2
&I  Computing the next quotient
&I  <new> acts non-trivially on the block of dim 6

&I  Found a quotient of dim 2
&I  Restarting after finding a decomposition
&I  Layer number 1: Type = "Perm"
&I  Size = 1, # of matrices = 2
&I  Submodule is invariant under <new>
&I  Enlarging quotient, old size = 1

&I  Is quotient classical?
&I  Dimension of group is <= 2, you must supply form
&I  The quotient contains SL
&I  New size = 24
&I  Adding generator relations to the kernel
&I  Layer number 2: Type = "Unknown"
&I  Size = 1, # of matrices = 2
&I  Computing the next quotient
&I  <new> acts non-trivially on the block of dim 4

&I  Found a quotient of dim 4
&I  Restarting after finding a decomposition
&I  Layer number 2: Type = "Perm"
&I  Size = 1, # of matrices = 2
&I  Submodule is invariant under <new>
&I  Enlarging quotient, old size = 1

&I  Is quotient classical?
&I  The case is symplectic
&I  This algorithm does not apply in this case.
&I  The quotient contains SP
&W  Applying Size to (matrix group) quotient
&I  New size = 51840
&I  Adding generator relations to the kernel
&I  Restarting after enlarging the quotient
&I  Layer number 2: Type = "Perm"
&I  Size = 51840, # of matrices = 0
&I  Using a permutation representation
&I  Adding random relations at layer number 2
&I  Adding a random relation at layer number 2
&I  Layer number 3: Type = "PGroup"
&I  Size = 1, # of matrices = 3
&I  Reached the p-group case
&I  New size = 27
&I  Adding a random relation at layer number 2
&I  Adding a random relation at layer number 2
&I  Kernel p-group, old size = 27
&I  Kernel p-group, new size = 6561
&I  Adding a random relation at layer number 2
&I  Kernel p-group, old size = 6561
&I  Kernel p-group, new size = 6561
&I  Adding a random relation at layer number 2
&I  Kernel p-group, old size = 6561
&I  Kernel p-group, new size = 6561
&I  Adding a random relation at layer number 2
&I  Kernel p-group, old size = 6561
&I  Kernel p-group, new size = 6561
&I  Adding a random relation at layer number 2
&I  Kernel p-group, old size = 6561
&I  Kernel p-group, new size = 6561
&I  Adding a random relation at layer number 2
&I  Kernel p-group, old size = 6561
&I  Kernel p-group, new size = 6561
&I  Adding a random relation at layer number 2
&I  Kernel p-group, old size = 6561
&I  Kernel p-group, new size = 6561
&I  Adding a random relation at layer number 2
&I  Kernel p-group, old size = 6561
&I  Kernel p-group, new size = 6561
&I  Adding a random relation at layer number 2
&I  Kernel p-group, old size = 6561
&I  Kernel p-group, new size = 6561
&I  Adding a random relation at layer number 2
&I  Kernel p-group, old size = 6561
&I  Kernel p-group, new size = 6561
&I  Adding a random relation at layer number 2
&I  Kernel p-group, old size = 6561
&I  Kernel p-group, new size = 6561
&I  Kernel is finished, size = 340122240
&I  Restarting after enlarging the quotient
&I  Layer number 1: Type = "SL"
&I  Size = 8162933760, # of matrices = 0
&I  Using the SL recognition
&I  Adding random relations at layer number 1
&I  Adding a random relation at layer number 1
&I  Layer number 2: Type = "Perm"
&I  Size = 340122240, # of matrices = 3
&I  Submodule is invariant under <new>
&I  Using a permutation representation
&I  Adding generator relations to the kernel
&I  Kernel p-group, old size = 6561
&I  Kernel p-group, new size = 6561
&I  Adding a random relation at layer number 1
&I  Layer number 2: Type = "Perm"
&I  Size = 340122240, # of matrices = 3
&I  Submodule is invariant under <new>
&I  Using a permutation representation
&I  Adding generator relations to the kernel
&I  Kernel p-group, old size = 6561
&I  Kernel p-group, new size = 6561
&I  Adding a random relation at layer number 1
&I  Layer number 2: Type = "Perm"
&I  Size = 340122240, # of matrices = 3
&I  Submodule is invariant under <new>
&I  Using a permutation representation
&I  Adding generator relations to the kernel
&I  Kernel p-group, old size = 6561
&I  Kernel p-group, new size = 6561
&I  Adding a random relation at layer number 1
&I  Layer number 2: Type = "Perm"
&I  Size = 340122240, # of matrices = 3
&I  Submodule is invariant under <new>
&I  Using a permutation representation
&I  Adding generator relations to the kernel
&I  Kernel p-group, old size = 6561
&I  Kernel p-group, new size = 6561
&I  Adding a random relation at layer number 1
&I  Layer number 2: Type = "Perm"
&I  Size = 340122240, # of matrices = 3
&I  Submodule is invariant under <new>
&I  Using a permutation representation
&I  Adding generator relations to the kernel
&I  Kernel p-group, old size = 6561
&I  Kernel p-group, new size = 6561
&I  Adding a random relation at layer number 1
&I  Layer number 2: Type = "Perm"
&I  Size = 340122240, # of matrices = 3
&I  Submodule is invariant under <new>
&I  Using a permutation representation
&I  Adding generator relations to the kernel
&I  Kernel p-group, old size = 6561
&I  Kernel p-group, new size = 6561
&I  Adding a random relation at layer number 1
&I  Layer number 2: Type = "Perm"
&I  Size = 340122240, # of matrices = 3
&I  Submodule is invariant under <new>
&I  Using a permutation representation
&I  Adding generator relations to the kernel
&I  Kernel p-group, old size = 6561
&I  Kernel p-group, new size = 6561
&I  Adding a random relation at layer number 1
&I  Layer number 2: Type = "Perm"
&I  Size = 340122240, # of matrices = 3
&I  Submodule is invariant under <new>
&I  Using a permutation representation
&I  Adding generator relations to the kernel
&I  Kernel p-group, old size = 6561
&I  Kernel p-group, new size = 6561
&I  Adding a random relation at layer number 1
&I  Layer number 2: Type = "Perm"
&I  Size = 340122240, # of matrices = 3
&I  Submodule is invariant under <new>
&I  Using a permutation representation
&I  Adding generator relations to the kernel
&I  Kernel p-group, old size = 6561
&I  Kernel p-group, new size = 6561
&I  Adding a random relation at layer number 1
&I  Layer number 2: Type = "Perm"
&I  Size = 340122240, # of matrices = 3
&I  Submodule is invariant under <new>
&I  Using a permutation representation
&I  Adding generator relations to the kernel
&I  Kernel p-group, old size = 6561
&I  Kernel p-group, new size = 6561
&I  Kernel is finished, size = 8162933760
gap>  
gap>  # Let us look at what we have found
gap> DisplayMatRecord( x );
&I  Matrix group over field GF(3) of dimension 6 has size 8162933760
&I  Number of layers is 3
gap> DisplayMatRecord( x, 1 );
&I  Layer Number = 1
&I  Type = SL
&I  Dimension = 2
&I  Size = 24
gap> # The module for G splits into an irreducible submodule of dimension
gap> # 2 and a quotient module of dimension 4. The restriction of G to
gap> # the submodule contains SL(2, 3). Call this group G1.
gap> DisplayMatRecord( x, 2 );
&I  Layer Number = 2
&I  Type = Perm
&I  Dimension = 4
&I  Size = 51840
gap> # We have now taken relations on G1 and evaluated them in G to get
gap> # a group H, which is the kernel of the homomorphism from G to G1.
gap> # The group generated by the last 4x4 block on the diagonal of the
gap> # matrices of H  has an irreducible module and we have computed
gap> # a permutation representation on it. Call this group H1.
gap> DisplayMatRecord( x, 3 );
&I  Layer Number = 3
&I  Type = PGroup
&I  Dimension = 6
&I  Size = 6561
gap> # We have now taken relations on H1 and evaluated them in H to get the
gap> # kernel of the homomorphism from H to H1. This kernel consists of lower
gap> # uni-triangular matrices. It is a p-group of size 6561. |

*Example 3*

|gap> Read("../data/a5xa5d25.gap");
gap> # Test whether G is primitive
gap> prim := IsPrimitive( G );;
&I  Number of blocks is 5
&I  Group primitive? false
gap> # G is imprimitive. Get the permutation representation given by action
gap> # of G on the blocks. Get the correspondence between the generators of
gap> # G and the generators of the permutation group.
gap> prim := prim[2];;
gap> bs := BlockSystemFlag( prim );;
gap> maps := MapsFlag (bs);
[ 1, 2, 3, 4 ]
gap> P := PermGroupFlag( bs );
Group( (1,2)(3,4), (1,2,4,3,5), (1,3)(2,4), (1,3,2,5,4) )
gap> # Apply ApproximateKernel
gap> x := ApproximateKernel( G, P, 100, 30, maps );;
&W Applying Size to permutation group
&I  Adding generator relations to the kernel
&I  Layer number 2: Type = "Unknown"
&I  Size = 1, # of matrices = 1
&I  Computing the next quotient
&I  <new> acts non-trivially on the block of dim 25

&I  Found a quotient of dim 1
&I  Restarting after finding a decomposition
&I  Layer number 2: Type = "Perm"
&I  Size = 1, # of matrices = 1
&I  Submodule is invariant under <new>
&I  Using a permutation representation
&I  Adding generator relations to the kernel
&I  Layer number 3: Type = "Unknown"
&I  Size = 1, # of matrices = 1
&I  Computing the next quotient
&I  <new> acts non-trivially on the block of dim 24

&I  Found a quotient of dim 1
&I  Restarting after finding a decomposition
&I  Layer number 3: Type = "Perm"
&I  Size = 1, # of matrices = 1
&I  Submodule is invariant under <new>
&I  Using a permutation representation
&I  Adding generator relations to the kernel
&I  Layer number 4: Type = "Unknown"
&I  Size = 1, # of matrices = 1
&I  Computing the next quotient
&I  <new> acts non-trivially on the block of dim 23

&I  Found a quotient of dim 1
&I  Restarting after finding a decomposition
&I  Layer number 4: Type = "Perm"
&I  Size = 1, # of matrices = 1
&I  Submodule is invariant under <new>
&I  Using a permutation representation
&I  Adding generator relations to the kernel
&I  Layer number 5: Type = "Unknown"
&I  Size = 1, # of matrices = 1
&I  Computing the next quotient
&I  <new> acts non-trivially on the block of dim 22

&I  Found a quotient of dim 1
&I  Restarting after finding a decomposition
&I  Layer number 5: Type = "Perm"
&I  Size = 1, # of matrices = 1
&I  Submodule is invariant under <new>
&I  Using a permutation representation
&I  Adding generator relations to the kernel
&I  Layer number 6: Type = "Unknown"
&I  Size = 1, # of matrices = 1
&I  Computing the next quotient
&I  <new> acts non-trivially on the block of dim 21

&I  Found a quotient of dim 1
&I  Restarting after finding a decomposition
&I  Layer number 6: Type = "Perm"
&I  Size = 1, # of matrices = 1
&I  Submodule is invariant under <new>
&I  Using a permutation representation
&I  Adding generator relations to the kernel
&I  Layer number 7: Type = "Unknown"
&I  Size = 1, # of matrices = 1
&I  Computing the next quotient
&I  <new> acts non-trivially on the block of dim 20

&I  Found a quotient of dim 4
&I  Restarting after finding a decomposition
&I  Layer number 7: Type = "Perm"
&I  Size = 1, # of matrices = 1
&I  Submodule is invariant under <new>
&I  Enlarging quotient, old size = 1

&I  Is quotient classical?
&I  The group probably does not contain a classical group
&I  It may contain an alternating group of degree in [ 5, 6 ]
&W  Applying Size to (matrix group) quotient
&I  New size = 5
&I  Adding generator relations to the kernel
&I  Restarting after enlarging the quotient
&I  Layer number 7: Type = "Perm"
&I  Size = 5, # of matrices = 0
&I  Using a permutation representation
&I  Adding random relations at layer number 7
&I  Adding a random relation at layer number 7
&I  Adding a random relation at layer number 7
&I  Adding a random relation at layer number 7
&I  Adding a random relation at layer number 7
&I  Adding a random relation at layer number 7
&I  Adding a random relation at layer number 7
&I  Adding a random relation at layer number 7
&I  Adding a random relation at layer number 7
&I  Adding a random relation at layer number 7
&I  Adding a random relation at layer number 7
&I  Adding a random relation at layer number 7
&I  Adding a random relation at layer number 7
&I  Adding a random relation at layer number 7
&I  Adding a random relation at layer number 7
&I  Adding a random relation at layer number 7
&I  Kernel is finished, size = 5
&I  Layer number 1: Type = "Perm"
&I  Size = 300, # of matrices = 0
&I  Using a permutation representation
&I  Adding random relations at layer number 1
&I  Adding a random relation at layer number 1
&I  Layer number 2: Type = "Perm"
&I  Size = 5, # of matrices = 5
&I  Submodule is not invariant under <new>
&I  Restarting at layer number 2
&I  Layer number 2: Type = "Unknown"
&I  Size = 1, # of matrices = 5
&I  Computing the next quotient
&I  <new> acts non-trivially on the block of dim 25

&I  Found a quotient of dim 5
&I  Restarting after finding a decomposition
&I  Layer number 2: Type = "Perm"
&I  Size = 1, # of matrices = 5
&I  Submodule is invariant under <new>
&I  Enlarging quotient, old size = 1

&I  Is quotient classical?
&I  No classical form has been found
&I  The group probably does not contain a classical group

&I  Is quotient primitive?
&I  Number of blocks is 5
&I  Group primitive? false
&I  The quotient is imprimitive
&I  Permutation group has degree 5
&W  Applying Size to (permutation group) quotient
&I  New size = 60
&I  Adding generator relations to the kernel
&I  Restarting after enlarging the quotient
&I  Layer number 2: Type = "Imprimitive"
&I  Size = 60, # of matrices = 0
&I  Using a permutation representation
&I  Adding random relations at layer number 2
&I  Adding a random relation at layer number 2
&I  Adding a random relation at layer number 2
&I  Adding a random relation at layer number 2
&I  Adding a random relation at layer number 2
&I  Adding a random relation at layer number 2
&I  Adding a random relation at layer number 2
&I  Kernel is finished, size = 60
&I  Adding a random relation at layer number 1
&I  Layer number 2: Type = "Imprimitive"
&I  Size = 60, # of matrices = 5
&I  Submodule is invariant under <new>
&I  Using a permutation representation
&I  Adding generator relations to the kernel
&I  Adding a random relation at layer number 1
&I  Layer number 2: Type = "Imprimitive"
&I  Size = 60, # of matrices = 5
&I  Submodule is invariant under <new>
&I  Using a permutation representation
&I  Adding generator relations to the kernel
&I  Adding a random relation at layer number 1
&I  Layer number 2: Type = "Imprimitive"
&I  Size = 60, # of matrices = 5
&I  Submodule is invariant under <new>
&I  Using a permutation representation
&I  Adding generator relations to the kernel
&I  Adding a random relation at layer number 1
&I  Adding a random relation at layer number 1
&I  Layer number 2: Type = "Imprimitive"
&I  Size = 60, # of matrices = 5
&I  Submodule is invariant under <new>
&I  Using a permutation representation
&I  Adding generator relations to the kernel
&I  Adding a random relation at layer number 1
&I  Layer number 2: Type = "Imprimitive"
&I  Size = 60, # of matrices = 5
&I  Submodule is invariant under <new>
&I  Using a permutation representation
&I  Adding generator relations to the kernel
&I  Kernel is finished, size = 3600
gap>  
gap> # Let us have a look at what we have found
gap> DisplayMatRecord( x );
&I  Matrix group over field GF(7) of dimension 25 has size 60
&I  Number of layers is 2
gap> DisplayMatRecord( x, 1 );
&I  Layer Number = 1
&I  Type = Imprimitive
&I  Dimension = 5
&I  Size = 60
gap> # The kernel K splits into an irreducible submodule of dimension
gap> # 5 and a quotient module of dimension 20. The restriction of K to the
gap> # submodule is imprimitive. Call it K1. The size is the size of the
gap> # permutation group given by the action of K1 on the blocks. This
gap> # action is faithful.
gap> DisplayMatRecord( x, 2 );
&I  Layer Number = 2
&I  Type = Unknown
&I  Dimension = 25
&I  Size = 1
gap> # We have taken relations on K1 and evaluated them in K to get the 
gap> # kernel of the homomorphism from K to K1. This kernel is trivial.
gap> # K has 5 copies of K1 down the diagonal. |

%%%%%%%%%%%%%%%%%%%%%%%%%%%%%%%%%%%%%%%%%%%%%%%%%%%%%%%%%%%%%%%%%%%%%%%%%

