%%%%%%%%%%%%%%%%%%%%%%%%%%%%%%%%%%%%%%%%%%%%%%%%%%%%%%%%%%%%%%%%%%%%%%%%%%%%%
%%
%A  gettable.tex                GAP documentation               Thomas Breuer
%%
%A  @(#)$Id: gettable.tex,v 1.2 1997/02/28 14:13:12 gap Exp $
%%
%Y  Copyright 1990-1992,  Lehrstuhl D fuer Mathematik,  RWTH Aachen,  Germany
%%
%H  $Log: gettable.tex,v $
%H  Revision 1.2  1997/02/28 14:13:12  gap
%H  sam and vfelsch updated the file to version 3.4.4
%H
%H  Revision 1.1.1.1  1996/12/11 12:36:45  werner
%H  Preparing 3.4.4 for release
%H
%H  Revision 3.19.1.1  1995/05/22  12:56:27  vfelsch
%H  updated a reference to the bibliography (and adapted the format)
%H
%H  Revision 3.19  1994/06/30  15:33:20  vfelsch
%H  included final changes for version 3.4
%H
%H  Revision 3.18  1994/06/18  12:24:10  sam
%H  fixed 'identifier' components in the text
%H
%H  Revision 3.17  1994/06/10  03:16:52  vfelsch
%H  updated examples
%H
%H  Revision 3.16  1994/06/03  08:57:20  mschoene
%H  changed a few things to avoid LaTeX warnings
%H
%H  Revision 3.15  1994/05/19  13:48:01  sam
%H  added section 'Selecting Library Tables', updates
%H
%H  Revision 3.14  1993/02/19  10:48:42  gap
%H  adjustments in line length and spelling
%H
%H  Revision 3.13  1993/02/15  10:28:24  felsch
%H  examples fixed
%H
%H  Revision 3.11  1992/10/15  08:47:25  sam
%H  mentioned new file 'ctomaxi4.tbl'
%H
%H  Revision 3.10  1992/08/07  13:39:13  sam
%H  added pictures for online help
%H
%H  Revision 3.9  1992/04/07  23:05:55  martin
%H  changed the author line
%H
%H  Revision 3.8  1992/04/01  12:05:02  sam
%H  little changes concerning contents of 'TBLNAME'
%H
%H  Revision 3.7  1992/03/27  16:45:21  sam
%H  removed reference 'Character Tables of Weyl Groups'
%H
%H  Revision 3.6  1992/03/27  14:08:09  sam
%H  removed "'" in index entries
%H
%H  Revision 3.5  1992/02/13  15:03:18  sam
%H  renamed 'MatrixAutomorphisms' to 'MatAutomorphisms'
%H
%H  Revision 3.4  1992/01/14  14:13:15  martin
%H  changed two more citations
%H
%H  Revision 3.3  1992/01/14  14:03:20  sam
%H  adjusted citations
%H
%H  Revision 3.2  1992/01/09  11:18:04  sam
%H  removed use of 'SetRecField'
%H
%H  Revision 3.1  1991/12/30  08:08:05  sam
%H  initial revision under RCS
%H
%%
\Chapter{Character Table Libraries}\index{character tables}%
\index{tables}\index{library tables}\index{generic character tables}

The utility   of {\GAP} for character theoretical    tasks depends on the
availability of many known character tables, so there  is a lot of tables
in the {\GAP} group collection.

There are three different libraries of character tables, namely *ordinary
character tables*, *Brauer tables* and *generic character tables*.

Of course,  these libraries are ``open\'\'\  in the sense that they shall
be extended. So we would  be grateful for  any further tables of interest
sent to us for inclusion into our libraries.

This chapter mainly explains  properties not of  single tables but of the
libraries and their  structure; for the  format of character tables,  see
"Character Table  Records", "Brauer  Table Records" and  chapter "Generic
Character Tables".

The chapter informs about
\begin{itemize}
\item the   actually  available  tables   (see  "Contents  of  the  Table
      Libraries"),
\item the sublibraries of {\ATLAS} tables (see "ATLAS Tables") and {\CAS}
      tables (see "CAS Tables"),
\item the  organization of the libraries  (see "Organization of the Table
      Libraries"),
\item and how to extend a library  (see "How to Extend a Table Library").
\end{itemize}

%%%%%%%%%%%%%%%%%%%%%%%%%%%%%%%%%%%%%%%%%%%%%%%%%%%%%%%%%%%%%%%%%%%%%%%%%
\Section{Contents of the Table Libraries}%
\index{character tables!libraries of}%
\index{tables!libraries of}\index{libraries of character tables}

As stated at  the beginning of the chapter,  there are three libraries of
character tables\:\ ordinary character tables, Brauer tables, and generic
character tables.

*Ordinary Character Tables*

Two different aspects are useful to list up the ordinary character tables
available  to {\GAP}:\ the aspect  of *source* of  the tables and that of
*connections* between the tables.

As  for the source,  there are two big sources,  the {\ATLAS} (see "ATLAS
Tables") and the  {\CAS} library   of  character tables.   Many  {\ATLAS}
tables are contained in the   {\CAS} library, and difficulties may  arise
because  the  succession of characters  or  classes in  {\CAS} tables and
{\ATLAS} tables are  different, so see "CAS  Tables" and "Character Table
Records" for the relations between  the (at least) two  forms of the same
table.  A large subset of the {\CAS} tables is the set of tables of Sylow
normalizers of  sporadic simple groups  as  published in~\cite{Ost86}, so
this may be viewed as another source.

To avoid confusions about the actual format of a table, authorship and so
on, the 'text' component of the table contains the information

'origin\:\ ATLAS of finite groups': \\
          for {\ATLAS} tables (see "ATLAS Tables")

'origin\:\ Ostermann': \\
          for tables of \cite{Ost86} and

'origin\:\ CAS library': \\
          for any table  of the  {\CAS}  table library  that is contained
          neither in the {\ATLAS} nor in \cite{Ost86}.

If  one is  interested in the  aspect  of connections between the tables,
i.e.,  the internal  structure of the  library of  ordinary tables (which
corresponds  to  the  access   to character   tables, as  described    in
"CharTable"), the contents can be listed up the following way\:

We have
\begin{itemize}
\item all {\ATLAS} tables  (see "ATLAS Tables"),  i.e.\ the tables of the
      simple groups  which are contained in the {\ATLAS},  and the tables
      of cyclic and bicyclic extensions of these groups;
\item most tables  of maximal subgroups  of sporadic simple groups  (*not
      all* for HN, F3+, B, M);
\item some   tables   of  maximal  subgroups  of  other  {\ATLAS}  tables
      (*which?*)
\item most  nontrivial  Sylow normalizers  of  sporadic  simple groups as
      printed in~\cite{Ost86},  where nontrivial  means that the group is
      not contained in $p$\:$(p-1)$   (*not* $J_4N2$, $Co_1N2$, $Co_1N5$,
      all  of  $Fi_{23}$,   $Fi_{24}^{\prime}$,   $B$,   $M$,  $HN$,  and
      $Fi_{22}N2$)
\item some tables of element centralizers
\item some tables of Sylow subgroups
\item a few other tables, e.g.\ 'W(F4)'\\
      {*namely which?*}
\end{itemize}

*Brauer Tables*

This library  contains the tables  of the modular  {\ATLAS} which are yet
known.  Some of them still contain unknowns (see "Unknown").  Since there
is ongoing  work in computing new tables,  this library is changed nearly
every day.

These Brauer tables contain the information

|    origin: modular ATLAS of finite groups|

in their text component.

*Generic Character Tables*

At the moment, generic  tables of the  following groups are  available in
{\GAP} (see "CharTable")\:

\begin{itemize}
\item alternating groups
\item cyclic groups,
\item dihedral groups,
\item some linear groups,
\item quaternionic (dicyclic) groups
\item Suzuki groups,
\item symmetric groups,
\item wreath   products   of  a   group   with  a  symmetric  group  (see
      "CharTableWreathSymmetric"),
\item Weyl groups of types $B_n$ and $D_n$
\end{itemize}

% *Not all these are really implemented as generic tables!!!*

%%%%%%%%%%%%%%%%%%%%%%%%%%%%%%%%%%%%%%%%%%%%%%%%%%%%%%%%%%%%%%%%%%%%%%%%%
\Section{Selecting Library Tables}

Single library tables can be selected by their  name (see "CharTable" for
admissible names of library tables, and "Contents of the Table Libraries"
for the organization of the library).

In general  it  does not  make sense  to  select tables   with respect to
certain   properties,  as is  useful   for  group  libraries (see  "Group
Libraries").  But it may  be useful to  get an overview of *all*  library
tables, or all  library tables of *simple* groups,  or all library tables
of *sporadic simple* groups.  It is sufficient to know an admissible name
of these tables, so they  need not be  loaded.  A table  can then be read
using "CharTable" 'CharTable'.

The mechanism is similar to that for group libraries.

'AllCharTableNames()': \\
         returns a list  with an admissible name for every library table,

'AllCharTableNames( IsSimple )': \\
         returns a list  with an admissible name  for every library table
         of a simple group,

'AllCharTableNames( IsSporadicSimple )': \\
         returns a list  with an admissible name  for every library table
         of a sporadic simple group.

\vspace{5mm}

Admissible  names  of *maximal subgroups*   of sporadic simple groups are
stored  in  the component 'maxes' of   the tables of  the sporadic simple
groups.  Thus

|    gap> maxes:= CharTable( "M11" ).maxes;
    [ "A6.2_3", "L2(11)", "3^2:Q8.2", "A5.2", "2.S4" ] |

returns the  list containing these  names for the Mathieu group $M_{11}$,
and

|    gap> List( maxes, CharTable );
    [ CharTable( "A6.2_3" ), CharTable( "L2(11)" ),
      CharTable( "3^2:Q8.2" ), CharTable( "A5.2" ), CharTable( "2.S4" ) ] |

will read them from the library files.

%%%%%%%%%%%%%%%%%%%%%%%%%%%%%%%%%%%%%%%%%%%%%%%%%%%%%%%%%%%%%%%%%%%%%%%%%
\Section{ATLAS Tables}\index{character tables!ATLAS}%
\index{tables!library}\index{library of character tables}

\def\ttquote{\char13}
\setlength{\unitlength}{0.1cm}

The   {\GAP} group collection contains   all  character  tables that  are
included in the Atlas of finite groups  (\cite{CCN85}, from now on called
{\ATLAS})  and the  Brauer  tables  contained  in  the modular   {\ATLAS}
(\cite{JLPW95}).  Although the Brauer tables form a library of their own,
they are described here since all  conventions for {\ATLAS} tables stated
here hold for Brauer tables, too.

*Additionally some conventions are necessary  about follower characters!*

These tables have the information

|    origin: ATLAS of finite groups|

resp.

|    origin: modular ATLAS of finite groups|

in their   'text' component, further on  they  are simply called {\ATLAS}
tables.

In addition to the  information  given in Chapters  6--8 of  the {\ATLAS}
which tell how to read the printed tables,  there are some rules relating
these to the corresponding {\GAP} tables.

*Improvements*

Note that for the {\GAP} library not the printed {\ATLAS} is relevant but
the revised version given by the list of\ \  *Improvements to the ATLAS*\
\ which can be got from Cambridge.

Also some tables are regarded as {\ATLAS} tables which are not printed in
the {\ATLAS} but available  in  {\ATLAS} format  from Cambridge; at   the
moment, these  are the tables  related to $L_2(49)$, $L_2(81)$, $L_6(2)$,
$O_8^-(3)$, $O_8^+(3)$ and $S_{10}(2)$.

*Powermaps*

In a few   cases (namely  the  tables of  $3.McL$,  $3_2.U_4(3)$  and its
covers,  $3_2.U_4(3).2_3$ and its  covers) the powermaps are not uniquely
determined by   the   given  information but  determined   up  to  matrix
automorphisms (see "MatAutomorphisms") of  the characters; then the first
possible map according  to  lexicographical ordering was chosen,  and the
automorphisms are listed in the 'text' component of the concerned table.

*Projective Characters*

For  any nontrivial  multiplier of a  simple group  or  of an automorphic
extension of a  simple group, there is a  component  'projectives' in the
table of  $G$ that is a  list of records with  the  names of the covering
group (e.g.  '\"12\_1.U4(3)\"') and the list of those faithful characters
which are printed in the \ATLAS (so--called {\it proxy characters}).

*Projections*

{\ATLAS} tables contain  the component 'projections'\:\ For any  covering
group  of $G$  for which the  character  table is  available  in {\ATLAS}
format a record is stored there containing components 'name' (the name of
the cover table) and 'map' (the  projection map); the projection maps any
class of $G$ to that preimage in the cover for that the column is printed
in the \ATLAS; it is called $g_0$ in Chapter 7, Section 14 there.

(In a sense, a projection map is an inverse of the factor fusion from the
cover table to the actual table (see "ProjectionMap").)

*Tables of Isoclinic Groups*

As described in  Chapter 6, Section  7 and  Chapter 7, Section  18 of the
\ATLAS, there  exist two different   groups of  structure  $2.G.2$ for  a
simple group $G$  which are isoclinic.  The {\ATLAS} table in the library
is that which is printed in the \ATLAS,  the isoclinic variant can be got
using "CharTableIsoclinic" 'CharTableIsoclinic'.

*Succession of characters and classes*

(Throughout this paragraph,  $G$ always means the involved simple group.)
\begin{enumerate}
\item For $G$ itself,  the succession  of classes  and  characters in the
      {\GAP} table is as printed in the \ATLAS.
\item For  an  automorphic  extension  $G.a$,  there are  three  types of
      characters\:
\begin{itemize}
\item If  a character  $\chi$  of  $G$  extends to  $G.a$,  the different
      extensions  $\chi^0,\chi^1,\ldots,\chi^{a-1}$  are consecutive (see
      {\ATLAS}, Chapter 7, Section 16).
\item If some characters of $G$ fuse to give a single character of $G.a$,
      the  position  of  that  character  is the  position  of the  first
      involved character of $G$.
\item If both,  extension and fusion,  occur,  the result  characters are
      consecutive, and each replaces the first involved character.
\end{itemize}
\item Similarly,  there are different types of classes for an automorphic
      extension $G.a$\:
\begin{itemize}
\item If some  classes  collapse,  the  result class  replaces  the first
      involved class.
\item For $a > 2$,  any proxy class and its followers are consecutive; if
      there are more than one followers for a proxy class  (the only case
      that occurs is for  $a = 5$),  the  succession of followers  is the
      natural one  of corresponding  galois automorphisms  (see {\ATLAS},
      Chapter 7, Section 19).
\end{itemize}
      The classes of $G.a_1$  always precede the outer classes of $G.a_2$
      for  $a_1, a_2$ dividing  $a$ and $a_1 \< a_2$.  This succession is
      like in the \ATLAS, with the only exception $U_3(8).6$.
\item For  a  central  extension  $M.G$,  there  are  different  types of
      characters\:
\begin{itemize}
\item Every character  can be regarded  as a  faithful  character  of the
      factor group $m.G$, where $m$ divides $M$.  Characters faithful for
      the same  factor group  are  consecutive  like in the  \ATLAS,  the
      succession  of these sets  of characters  is given  by the order of
      precedence  $1, 2, 4, 3, 6, 12$  for the  different values of  $m$.
\item If $m > 2$,  a faithful character of  $m.G$  that is printed in the
      {\ATLAS}  (a so-called  \mbox{\em proxy})  represents  one or  more
      \mbox{\em followers},  this means  galois conjugates  of the proxy;
      in any {\GAP} table,  the proxy  precedes  its followers;  the case
      $m = 12$  is the only one  that occurs  with more than one follower
      for a proxy,  then the three followers are ordered according to the
      corresponding galois automorphisms 5, 7, 11 (in that succession).
\end{itemize}
\item For the classes of a central extension we have\:
\begin{itemize}
\item The  preimages   of  a  $G$-class  in  $M.G$  are  subsequent,  the
      succession  is the same  as that  of the lifting order rows  in the
      \ATLAS.
\item The primitive roots  of unity  chosen  to represent  the generating
      central element (class 2) are  'E(3)',  'E(4)',  'E(6)\^5' ('= E(2)
      \* E(3)') and  'E(12)\^7'  ('= E(3) \* E(4)') for $m = 3, 4, 6$ and
      $12$, respectively.
\end{itemize}
\item For  tables  of bicyclic extensions  $m.G.a$,  both  the rules  for
      automorphic and central extensions hold; additionally we have\:
\begin{itemize}
\item Whenever classes of the subgroup $m.G$ collapse or characters fuse,
      the result class resp. character  replaces the first involved class
      resp.  character. 
\item Extensions  of a character are subsequent,  and the extensions of a
      proxy character precede the extensions of its followers.
\item Preimages  of a class are subsequent,  and the preimages of a proxy
      class precede the preimages of its followers.
\end{itemize}
\end{enumerate}

%%%%%%%%%%%%%%%%%%%%%%%%%%%%%%%%%%%%%%%%%%%%%%%%%%%%%%%%%%%%%%%%%%%%%%%%%
\newpage
\Section{Examples of the ATLAS format for GAP tables}
\index{character tables!CAS}\index{tables!library}
\index{library of character tables}

We give  three  little  examples  for the conventions  stated  in  "ATLAS
Tables", listing up the {\ATLAS} format and the table displayed by \GAP.

First, let $G$ be the trivial  group.  The cyclic group  $C_6$ of order 6
can be viewed in several ways\:

\begin{enumerate}
\item  As a  downward extension of the factor group $C_2$  which contains
$G$ as a subgroup; equivalently, as an upward  extension of the  subgroup
$C_3$ which has a factor group $G$\:

%ignore
\begin{picture}(110,55)
\put(-2,23){
\begin{picture}(29,29)
\put(0,29){\line(1,0){14}}
\put(0,15){\line(1,0){14}}
\put(0,14){\line(1,0){14}}
\put(0,0){\line(1,0){14}}
\put(15,29){\line(1,0){14}}
\put(15,15){\line(1,0){14}}
\put(15,14){\line(1,0){14}}
\put(15,0){\line(1,0){14}}
\put(0,15){\line(0,1){14}}
\put(0,0){\line(0,1){14}}
\put(14,15){\line(0,1){14}}
\put(15,15){\line(0,1){14}}
\put(29,15){\line(0,1){14}}
\put(14,0){\line(0,1){14}}
\put(15,0){\line(0,1){14}}
\put(29,0){\line(0,1){14}}
\put(7,7){\makebox(0,0){3.G}}
\put(22,7){\makebox(0,0){3.G.2}}
\put(7,22){\makebox(0,0){G}}
\put(22,22){\makebox(0,0){G.2}}
\end{picture}}
\put(37,52){\makebox(0,0)[tl]{
\small\tt
\begin{minipage}{2in}
\baselineskip0.9ex
\parskip0.2ex

\ \ \ \ ;\ \ \ @\ \ \ ;\ \ \ ;\ \ \ @\ \par
\ \par
\ \ \ \ \ \ \ \ 1\ \ \ \ \ \ \ \ \ \ \ 1\ \par
\ \ p\ power\ \ \ \ \ \ \ \ \ \ \ A\ \par
\ \ p\ttquote\ part\ \ \ \ \ \ \ \ \ \ \ A\ \par
\ \ ind\ \ 1A\ fus\ ind\ \ 2A\ \par
\ \par
$\chi_1$\ \ +\ \ \ 1\ \ \ \:\ \ ++\ \ \ 1\ \par
\ \par
\ \ ind\ \ \ 1\ fus\ ind\ \ \ 2\ \par
\ \ \ \ \ \ \ \ 3\ \ \ \ \ \ \ \ \ \ \ 6\ \par
\ \ \ \ \ \ \ \ 3\ \ \ \ \ \ \ \ \ \ \ 6\ \par
\ \par
$\chi_2$\ o2\ \ \ 1\ \ \ \:\ oo2\ \ \ 1\ 
\end{minipage}}}

\put(83,52){\makebox(0,0)[tl]{
\small\tt
\begin{minipage}{2in}
\baselineskip2.7ex
\parskip0ex

\ \ \ 2\ \ \ 1\ \ \ 1\ \ \ 1\ \ \ 1\ \ \ 1\ \ \ 1 \par
\ \ \ 3\ \ \ 1\ \ \ 1\ \ \ 1\ \ \ 1\ \ \ 1\ \ \ 1 \par
 \par
\ \ \ \ \ \ 1a\ \ 3a\ \ 3b\ \ 2a\ \ 6a\ \ 6b \par
\ \ 2P\ \ 1a\ \ 3b\ \ 3a\ \ 1a\ \ 3b\ \ 3a \par
\ \ 3P\ \ 1a\ \ 1a\ \ 1a\ \ 2a\ \ 2a\ \ 2a \par
 \par
X.1\ \ \ \ 1\ \ \ 1\ \ \ 1\ \ \ 1\ \ \ 1\ \ \ 1 \par
X.2\ \ \ \ 1\ \ \ 1\ \ \ 1\ \ -1\ \ -1\ \ -1 \par
X.3\ \ \ \ 1\ \ \ A\ \ /A\ \ \ 1\ \ \ A\ \ /A \par
X.4\ \ \ \ 1\ \ \ A\ \ /A\ \ -1\ \ -A\ -/A \par
X.5\ \ \ \ 1\ \ /A\ \ \ A\ \ \ 1\ \ /A\ \ \ A \par
X.6\ \ \ \ 1\ \ /A\ \ \ A\ \ -1\ -/A\ \ -A \par
 \par
A\ =\ E(3) \par
\ \ =\ (-1+ER(-3))/2\ =\ b3 \par

\end{minipage}}}
\end{picture}
%end
%display
% -------   -------       ;   @   ;   ;   @      2   1   1   1   1   1   1
%|       | |       |          1           1      3   1   1   1   1   1   1
%|   G   | |  G.2  |    p power           A
%|       | |       |    p' part           A         1a  3a  3b  2a  6a  6b
% -------   -------     ind  1A fus ind  2A     2P  1a  3b  3a  1a  3b  3a
% -------   -------                             3P  1a  1a  1a  2a  2a  2a
%|       | |       |  X1  +   1   :  ++   1
%|  3.G  | | 3.G.2 |                          X.1    1   1   1   1   1   1
%|       | |       |    ind   1 fus ind   2   X.2    1   1   1  -1  -1  -1
% -------   -------           3           6   X.3    1   A  /A   1   A  /A
%                             3           6   X.4    1   A  /A  -1  -A -/A
%                                             X.5    1  /A   A   1  /A   A
%                     X2 o2   1   : oo2   1   X.6    1  /A   A  -1 -/A  -A
%
%                                             A = E(3)
%                                               = (-1+ER(-3))/2 = b3
%end

'X.1', 'X.2'  extend  $\chi_1$.  'X.3', 'X.4'  extend the proxy character
$\chi_2$.  'X.5',  'X.6'  extend  its  follower.  '1a',  '3a',  '3b'  are
preimages of '1A', and '2a', '6a', '6b' are preimages of '2A'.

\item  As a downward extension  of the factor group  $C_3$ which contains
$G$ as a subgroup;  equivalently,  as an upward extension of the subgroup
$C_2$ which has a factor group $G$\:

%ignore
\begin{picture}(110,55)
\put(-2,23){
\begin{picture}(29,29)
\put(0,29){\line(1,0){14}}
\put(0,15){\line(1,0){14}}
\put(0,14){\line(1,0){14}}
\put(0,0){\line(1,0){14}}
\put(15,29){\line(1,0){14}}
\put(15,15){\line(1,0){14}}
\put(15,14){\line(1,0){14}}
\put(15,0){\line(1,0){14}}
\put(0,15){\line(0,1){14}}
\put(0,0){\line(0,1){14}}
\put(14,15){\line(0,1){14}}
\put(15,15){\line(0,1){14}}
\put(29,15){\line(0,1){14}}
\put(14,0){\line(0,1){14}}
\put(15,0){\line(0,1){14}}
\put(29,0){\line(0,1){14}}
\put(7,7){\makebox(0,0){2.G}}
\put(22,7){\makebox(0,0){2.G.3}}
\put(7,22){\makebox(0,0){G}}
\put(22,22){\makebox(0,0){G.3}}
\end{picture}}
\put(37,52){\makebox(0,0)[tl]{
\small\tt
\begin{minipage}{2in}
\baselineskip0.9ex
\parskip0.2ex

\ \ \ \ ;\ \ \ @\ \ \ ;\ \ \ ;\ \ \ @ \par
\ \par
\ \ \ \ \ \ \ \ 1\ \ \ \ \ \ \ \ \ \ \ 1 \par
\ \ p\ power\ \ \ \ \ \ \ \ \ \ \ A \par
\ \ p\ttquote\ part\ \ \ \ \ \ \ \ \ \ \ A \par
\ \ ind\ \ 1A\ fus\ ind\ \ 3A \par
\ \par
$\chi_1$\ \ +\ \ \ 1\ \ \ \:\ +oo\ \ \ 1 \par
\ \par
\ \ ind\ \ \ 1\ fus\ ind\ \ \ 3 \par
\ \ \ \ \ \ \ \ 2\ \ \ \ \ \ \ \ \ \ \ 6 \par
\ \par
$\chi_2$\ \ +\ \ \ 1\ \ \ \:\ +oo\ \ \ 1 \par
\end{minipage}}}

\put(83,52){\makebox(0,0)[tl]{
\small\tt
\begin{minipage}{2in}
\baselineskip2.7ex
\parskip0ex

\ \ \ 2\ \ \ 1\ \ \ 1\ \ \ 1\ \ \ 1\ \ \ 1\ \ \ 1 \par
\ \ \ 3\ \ \ 1\ \ \ 1\ \ \ 1\ \ \ 1\ \ \ 1\ \ \ 1 \par
 \par
\ \ \ \ \ \ 1a\ \ 2a\ \ 3a\ \ 6a\ \ 3b\ \ 6b \par
\ \ 2P\ \ 1a\ \ 1a\ \ 3b\ \ 3b\ \ 3a\ \ 3a \par
\ \ 3P\ \ 1a\ \ 2a\ \ 1a\ \ 2a\ \ 1a\ \ 2a \par
 \par
X.1\ \ \ \ 1\ \ \ 1\ \ \ 1\ \ \ 1\ \ \ 1\ \ \ 1 \par
X.2\ \ \ \ 1\ \ \ 1\ \ \ A\ \ \ A\ \ /A\ \ /A \par
X.3\ \ \ \ 1\ \ \ 1\ \ /A\ \ /A\ \ \ A\ \ \ A \par
X.4\ \ \ \ 1\ \ -1\ \ \ 1\ \ -1\ \ \ 1\ \ -1 \par
X.5\ \ \ \ 1\ \ -1\ \ \ A\ \ -A\ \ /A\ -/A \par
X.6\ \ \ \ 1\ \ -1\ \ /A\ -/A\ \ \ A\ \ -A \par
 \par
A\ =\ E(3) \par
\ \ =\ (-1+ER(-3))/2\ =\ b3 \par
\end{minipage}}}
\end{picture}
%end
%display
% -------   -------       ;   @   ;   ;   @      2   1   1   1   1   1   1
%|       | |       |          1           1      3   1   1   1   1   1   1
%|   G   | |  G.3  |    p power           A  
%|       | |       |    p' part           A         1a  2a  3a  6a  3b  6b
% -------   -------     ind  1A fus ind  3A     2P  1a  1a  3b  3b  3a  3a
% -------   -------                             3P  1a  2a  1a  2a  1a  2a
%|       | |       |  X1  +   1   : +oo   1  
%|  2.G  | | 2.G.3 |                          X.1    1   1   1   1   1   1
%|       | |       |    ind   1 fus ind   3   X.2    1   1   A   A  /A  /A
% -------   -------           2           6   X.3    1   1  /A  /A   A   A
%                                             X.4    1  -1   1  -1   1  -1
%                     X2  +   1   : +oo   1   X.5    1  -1   A  -A  /A -/A
%                                             X.6    1  -1  /A -/A   A  -A
%
%                                             A = E(3)
%                                               = (-1+ER(-3))/2 = b3
%end

'X.1'-'X.3' extend $\chi_1$, 'X.4'-'X.6' extend $\chi_2$.  '1a', '2a' are
preimages of '1A'. '3a', '6a' are preimages of the proxy class '3A',  and
'3b', '6b' are preimages of its follower class.

\newpage
\item  As a downward extension of the factor groups $C_3$ and $C_2$ which
have $G$ as a factor group\:

%ignore
\begin{picture}(110,70)
\put(-2,8){
\begin{picture}(14,59)
\put(0,59){\line(1,0){14}}
\put(0,45){\line(1,0){14}}
\put(0,44){\line(1,0){14}}
\put(0,30){\line(1,0){14}}
\put(0,29){\line(1,0){14}}
\put(0,15){\line(1,0){14}}
\put(0,14){\line(1,0){14}}
\put(0,0){\line(1,0){14}}
\put(0,45){\line(0,1){14}}
\put(0,30){\line(0,1){14}}
\put(0,15){\line(0,1){14}}
\put(0,0){\line(0,1){14}}
\put(14,45){\line(0,1){14}}
\put(14,30){\line(0,1){14}}
\put(14,15){\line(0,1){14}}
\put(14,0){\line(0,1){14}}
\put(7,7){\makebox(0,0){6.G}}
\put(7,22){\makebox(0,0){3.G}}
\put(7,37){\makebox(0,0){2.G}}
\put(7,52){\makebox(0,0){G}}
\end{picture}}
\put(37,67){\makebox(0,0)[tl]{
\small\tt
\begin{minipage}{2in}
\baselineskip0.9ex
\parskip0.2ex

\ \ \ \ ;\ \ \ @ \par
\ \  \par
\ \ \ \ \ \ \ \ 1 \par
\ \ p\ power \par
\ \ p\ttquote\ part \par
\ \ ind\ \ 1A \par
\ \  \par
$\chi_1$\ \ +\ \ \ 1 \par
\ \  \par
\ \ ind\ \ \ 1 \par
\ \ \ \ \ \ \ \ 2 \par
\ \  \par
$\chi_2$\ \ +\ \ \ 1 \par
\ \  \par
\ \ ind\ \ \ 1 \par
\ \ \ \ \ \ \ \ 3 \par
\ \ \ \ \ \ \ \ 3 \par
\ \  \par
$\chi_3$\ o2\ \ \ 1 \par
\ \  \par
\ \ ind\ \ \ 1 \par
\ \ \ \ \ \ \ \ 6 \par
\ \ \ \ \ \ \ \ 3 \par
\ \ \ \ \ \ \ \ 2 \par
\ \ \ \ \ \ \ \ 3 \par
\ \ \ \ \ \ \ \ 6 \par
\ \  \par
$\chi_4$\ o2\ \ \ 1 \par
\end{minipage}}}

\put(83,67){\makebox(0,0)[tl]{
\small\tt
\begin{minipage}{2in}
\baselineskip2.7ex
\parskip0ex

\ \ \ 2\ \ \ 1\ \ \ 1\ \ \ 1\ \ \ 1\ \ \ 1\ \ \ 1 \par
\ \ \ 3\ \ \ 1\ \ \ 1\ \ \ 1\ \ \ 1\ \ \ 1\ \ \ 1 \par
 \par
\ \ \ \ \ \ 1a\ \ 6a\ \ 3a\ \ 2a\ \ 3b\ \ 6b \par
\ \ 2P\ \ 1a\ \ 3a\ \ 3b\ \ 1a\ \ 3a\ \ 3b \par
\ \ 3P\ \ 1a\ \ 2a\ \ 1a\ \ 2a\ \ 1a\ \ 2a \par
 \par
X.1\ \ \ \ 1\ \ \ 1\ \ \ 1\ \ \ 1\ \ \ 1\ \ \ 1 \par
X.2\ \ \ \ 1\ \ -1\ \ \ 1\ \ -1\ \ \ 1\ \ -1 \par
X.3\ \ \ \ 1\ \ \ A\ \ /A\ \ \ 1\ \ \ A\ \ /A \par
X.4\ \ \ \ 1\ \ /A\ \ \ A\ \ \ 1\ \ /A\ \ \ A \par
X.5\ \ \ \ 1\ \ -A\ \ /A\ \ -1\ \ \ A\ -/A \par
X.6\ \ \ \ 1\ -/A\ \ \ A\ \ -1\ \ /A\ \ -A \par
 \par
A\ =\ E(3) \par
\ \ =\ (-1+ER(-3))/2\ =\ b3 \par
\end{minipage}}}
\end{picture}
%end
%display
% -------           ;   @        2   1   1   1   1   1   1 
%|       |              1        3   1   1   1   1   1   1 
%|   G   |        p power     
%|       |        p' part           1a  6a  3a  2a  3b  6b 
% -------         ind  1A       2P  1a  3a  3b  1a  3a  3b 
% -------                       3P  1a  2a  1a  2a  1a  2a 
%|       |      X1  +   1     
%|  2.G  |                    X.1    1   1   1   1   1   1 
%|       |        ind   1     X.2    1  -1   1  -1   1  -1 
% -------               2     X.3    1   A  /A   1   A  /A 
% -------                     X.4    1  /A   A   1  /A   A 
%|       |      X2  +   1     X.5    1  -A  /A  -1   A -/A 
%|  3.G  |                    X.6    1 -/A   A  -1  /A  -A 
%|       |        ind   1     
% -------               3     A = E(3) 
% -------               3       = (-1+ER(-3))/2 = b3 
%|       |                 
%|  6.G  |      X3 o2   1     
%|       |                    
% -------         ind   1     
%                       6     
%                       3     
%                       2     
%                       3     
%                       6     
%                          
%               X4 o2   1     
%end

'X.1', 'X.2' correspond to $\chi_1, \chi_2$,  respectively;  'X.3', 'X.5'
correspond  to the proxies  $\chi_3, \chi_4$,  and 'X.4',  'X.6' to their
followers.  The factor fusion onto  $3.G$ is '[ 1, 2, 3, 1, 2, 3 ]', that
onto $G.2$ is '[ 1, 2, 1, 2, 1, 2 ]'.

\item  As an  upward extension of the subgroups $C_3$ or $C_2$ which both
contain a subgroup $G$\:

%ignore
\begin{picture}(110,55)
\put(-2,38){
\begin{picture}(59,14)
\put(0,0){\line(1,0){14}}
\put(0,0){\line(0,1){14}}
\put(0,14){\line(1,0){14}}
\put(14,0){\line(0,1){14}}
\put(7,7){\makebox(0,0){G}}
\put(15,0){\line(1,0){14}}
\put(15,0){\line(0,1){14}}
\put(15,14){\line(1,0){14}}
\put(29,0){\line(0,1){14}}
\put(22,7){\makebox(0,0){G.2}}
\put(30,0){\line(1,0){14}}
\put(30,0){\line(0,1){14}}
\put(30,14){\line(1,0){14}}
\put(44,0){\line(0,1){14}}
\put(37,7){\makebox(0,0){G.3}}
\put(45,0){\line(1,0){14}}
\put(45,0){\line(0,1){14}}
\put(45,14){\line(1,0){14}}
\put(59,0){\line(0,1){14}}
\put(52,7){\makebox(0,0){G.6}}
\end{picture}}
\put(-2,30){\makebox(0,0)[tl]{
\small\tt
\begin{minipage}{4in}
\baselineskip0.9ex
\parskip0.2ex

\ \ \ \ ;\ \ \ @\ \ \ ;\ \ \ ;\ \ \ @\ \ \ ;\ \ \ ;\ \ \ @\ \ \ ;\ \ \ \ \ ;\ \ \ @\ \par
\ \par
\ \ \ \ \ \ \ \ 1\ \ \ \ \ \ \ \ \ \ \ 1\ \ \ \ \ \ \ \ \ \ \ 1\ \ \ \ \ \ \ \ \ \ \ \ \ 1\ \par
\ \ p\ power\ \ \ \ \ \ \ \ \ \ \ A\ \ \ \ \ \ \ \ \ \ \ A\ \ \ \ \ \ \ \ \ \ \ \ AA\ \par
\ \ p\ttquote\ part\ \ \ \ \ \ \ \ \ \ \ A\ \ \ \ \ \ \ \ \ \ \ A\ \ \ \ \ \ \ \ \ \ \ \ AA\ \par
\ \ ind\ \ 1A\ fus\ ind\ \ 2A\ fus\ ind\ \ 3A\ fus\ \ \ ind\ \ 6A\ \par
\ \par
$\chi_1$\ \ +\ \ \ 1\ \ \ \:\ \ ++\ \ \ 1\ \ \ \:\ +oo\ \ \ 1\ \ \ \:+oo+oo\ \ \ 1\ \par
\end{minipage}}}

\put(83,52){\makebox(0,0)[tl]{
\small\tt
\begin{minipage}{2in}
\baselineskip2.7ex
\parskip0ex

\ \ \ 2\ \ \ 1\ \ \ 1\ \ \ 1\ \ \ 1\ \ \ 1\ \ \ 1 \par
\ \ \ 3\ \ \ 1\ \ \ 1\ \ \ 1\ \ \ 1\ \ \ 1\ \ \ 1 \par
 \par
\ \ \ \ \ \ 1a\ \ 2a\ \ 3a\ \ 3b\ \ 6a\ \ 6b \par
\ \ 2P\ \ 1a\ \ 1a\ \ 3b\ \ 3a\ \ 3b\ \ 3a \par
\ \ 3P\ \ 1a\ \ 2a\ \ 1a\ \ 1a\ \ 2a\ \ 2a \par
 \par
X.1\ \ \ \ 1\ \ \ 1\ \ \ 1\ \ \ 1\ \ \ 1\ \ \ 1 \par
X.2\ \ \ \ 1\ \ -1\ \ \ A\ \ /A\ \ -A\ -/A \par
X.3\ \ \ \ 1\ \ \ 1\ \ /A\ \ \ A\ \ /A\ \ \ A \par
X.4\ \ \ \ 1\ \ -1\ \ \ 1\ \ \ 1\ \ -1\ \ -1 \par
X.5\ \ \ \ 1\ \ \ 1\ \ \ A\ \ /A\ \ \ A\ \ /A \par
X.6\ \ \ \ 1\ \ -1\ \ /A\ \ \ A\ -/A\ \ -A \par
 \par
A\ =\ E(3) \par
\ \ =\ (-1+ER(-3))/2\ =\ b3 \par
\end{minipage}}}
\end{picture}
%end
%display
% -------   -------  -------   -------          
%|       | |       ||       | |       |         
%|   G   | |  G.2  ||  G.3  | |  G.6  |         
%|       | |       ||       | |       |         
% -------   -------  -------   -------          
%                                               
%    ;   @   ;   ;   @   ;   ;   @   ;     ;   @
%                                               
%        1           1           1             1
%  p power           A           A            AA
%  p' part           A           A            AA
%  ind  1A fus ind  2A fus ind  3A fus   ind  6A
%                                               
%X1  +   1   :  ++   1   : +oo   1   :+oo+oo   1
%
%
%    2   1   1   1   1   1   1
%    3   1   1   1   1   1   1
%
%       1a  2a  3a  3b  6a  6b
%   2P  1a  1a  3b  3a  3b  3a
%   3P  1a  2a  1a  1a  2a  2a
% X.1    1   1   1   1   1   1
% X.2    1  -1   A  /A  -A -/A
% X.3    1   1  /A   A  /A   A
% X.4    1  -1   1   1  -1  -1
% X.5    1   1   A  /A   A  /A
% X.6    1  -1  /A   A -/A  -A
% 
% A = E(3) 
%   = (-1+ER(-3))/2 = b3
%end

'1a', '2a' correspond to $1A, 2A$, respectively; '3a', '6a' correspond to
the proxies $3A, 6A$, and '3b', '6b' to their followers.

\end{enumerate}

\newpage
The second example  explains the fusion case; again,  $G$ is the  trivial
group.

%ignore
\begin{picture}(110,95)
\put(0,33){
\begin{picture}(29,59)
\put(0,59){\line(1,0){14}}
\put(0,45){\line(1,0){14}}
\put(0,44){\line(1,0){14}}
\put(0,30){\line(1,0){14}}
\put(0,29){\line(1,0){14}}
\put(0,15){\line(1,0){14}}
\put(0,14){\line(1,0){14}}
\put(0,0){\line(1,0){14}}
\put(0,45){\line(0,1){14}}
\put(0,30){\line(0,1){14}}
\put(0,15){\line(0,1){14}}
\put(0,0){\line(0,1){14}}
\put(14,45){\line(0,1){14}}
\put(14,30){\line(0,1){14}}
\put(14,15){\line(0,1){14}}
\put(14,0){\line(0,1){14}}
\put(15,59){\line(1,0){14}}
\put(15,45){\line(1,0){14}}
\put(15,44){\line(1,0){14}}
\put(15,30){\line(1,0){14}}
\put(15,29){\line(1,0){14}}
\put(15,14){\line(1,0){14}}
\put(15,45){\line(0,1){14}}
\put(15,30){\line(0,1){14}}
\put(15,15){\line(0,1){14}}
\put(15,0){\line(0,1){14}}
\put(29,45){\line(0,1){14}}
\put(29,30){\line(0,1){14}}
\put(7,7){\makebox(0,0){6.G}}
\put(7,22){\makebox(0,0){3.G}}
\put(7,37){\makebox(0,0){2.G}}
\put(7,52){\makebox(0,0){G}}
\put(22,7){\makebox(0,0){6.G.2}}
\put(22,22){\makebox(0,0){3.G.2}}
\put(22,37){\makebox(0,0){2.G.2}}
\put(22,52){\makebox(0,0){G.2}}
\end{picture}}
\put(39,92){\makebox(0,0)[tl]{
\small\tt
\begin{minipage}{2in}
\baselineskip0.9ex
\parskip0.2ex

\ \ \ \ ;\ \ \ @\ \ \ ;\ \ \ ;\ \ @\ \par
\ \ \ \par
\ \ \ \ \ \ \ \ 1\ \ \ \ \ \ \ \ \ \ 1\ \par
\ \ p\ power\ \ \ \ \ \ \ \ \ \ A\ \par
\ \ p\ttquote\ part\ \ \ \ \ \ \ \ \ \ A\ \par
\ \ ind\ \ 1A\ fus\ ind\ 2A\ \par
\ \ \ \par
$\chi_1$\ \ +\ \ \ 1\ \ \ \:\ \ ++\ \ 1\ \par
\ \ \ \par
\ \ ind\ \ \ 1\ fus\ ind\ \ 2\ \par
\ \ \ \ \ \ \ \ 2\ \ \ \ \ \ \ \ \ \ 2\ \par
\ \ \ \par
$\chi_2$\ \ +\ \ \ 1\ \ \ \:\ \ ++\ \ 1\ \par
\ \ \ \par
\ \ ind\ \ \ 1\ fus\ ind\ \ 2\ \par
\ \ \ \ \ \ \ \ 3\ \par
\ \ \ \ \ \ \ \ 3\ \par
\ \ \ \par
$\chi_3$\ o2\ \ \ 1\ \ \ \*\ \ \ +\ \par
\ \ \ \par
\ \ ind\ \ \ 1\ fus\ ind\ \ 2\ \par
\ \ \ \ \ \ \ \ 6\ \ \ \ \ \ \ \ \ \ 2\par
\ \ \ \ \ \ \ \ 3\ \par
\ \ \ \ \ \ \ \ 2\ \par
\ \ \ \ \ \ \ \ 3\ \par
\ \ \ \ \ \ \ \ 6\ \par
\ \ \ \par
$\chi_4$\ o2\ \ \ 1\ \ \ \*\ \ \ +\ \par
\end{minipage}}}

\put(85,92){\makebox(0,0)[tl]{
\small\tt
\begin{minipage}{2in}
\baselineskip2.7ex
\parskip0ex

$3.G.2$ \par
 \par
\ \ \ 2\ \ \ 1\ \ \ .\ \ \ 1 \par
\ \ \ 3\ \ \ 1\ \ \ 1\ \ \ . \par
 \par
\ \ \ \ \ \ 1a\ \ 3a\ \ 2a \par
\ \ 2P\ \ 1a\ \ 3a\ \ 1a \par
\ \ 3P\ \ 1a\ \ 1a\ \ 2a \par
 \par
X.1\ \ \ \ 1\ \ \ 1\ \ \ 1 \par
X.2\ \ \ \ 1\ \ \ 1\ \ -1 \par
X.3\ \ \ \ 2\ \ -1\ \ \ . \par
 \par
\ 
 \par
$6.G.2$ \par
 \par
\ \ \ 2\ \ \ 2\ \ \ 1\ \ \ 1\ \ \ 2\ \ \ 2\ \ \ 2 \par
\ \ \ 3\ \ \ 1\ \ \ 1\ \ \ 1\ \ \ 1\ \ \ .\ \ \ . \par
 \par
\ \ \ \ \ \ 1a\ \ 6a\ \ 3a\ \ 2a\ \ 2b\ \ 2c \par
\ \ 2P\ \ 1a\ \ 3a\ \ 3a\ \ 1a\ \ 1a\ \ 1a \par
\ \ 3P\ \ 1a\ \ 2a\ \ 1a\ \ 2a\ \ 2b\ \ 2c \par
 \par
Y.1\ \ \ \ 1\ \ \ 1\ \ \ 1\ \ \ 1\ \ \ 1\ \ \ 1 \par
Y.2\ \ \ \ 1\ \ \ 1\ \ \ 1\ \ \ 1\ \ -1\ \ -1 \par
Y.3\ \ \ \ 1\ \ -1\ \ \ 1\ \ -1\ \ \ 1\ \ -1 \par
Y.4\ \ \ \ 1\ \ -1\ \ \ 1\ \ -1\ \ -1\ \ \ 1 \par
Y.5\ \ \ \ 2\ \ -1\ \ -1\ \ \ 2\ \ \ .\ \ \ . \par
Y.6\ \ \ \ 2\ \ \ 1\ \ -1\ \ -2\ \ \ .\ \ \ . \par

\end{minipage}}}
\end{picture}
%end
%display
% -------   -------        ;   @   ;   ;  @      3.G.2
%|       | |       |           1          1      
%|   G   | |  G.2  |     p power          A         2   1   .   1 
%|       | |       |     p' part          A         3   1   1   .  
% -------   -------      ind  1A fus ind 2A        
% -------   -------                                    1a 3a 2a 
%|       | |       |   X1  +   1   :  ++  1        2P  1a 3a 1a 
%|  2.G  | | 2.G.2 |                               3P  1a 1a 2a 
%|       | |       |     ind   1 fus ind  2      
% -------   -------            2          2      X.1    1  1  1 
% -------   -------                              X.2    1  1 -1 
%|       | |           X2  +   1   :  ++  1      X.3    2 -1  . 
%|  3.G  | | 3.G.2                          
%|       | |             ind   1 fus ind  2 
% -------                      3                 6.G.2 
% -------   -------            3                 
%|       | |                                        2   2  1  1  2  2  2
%|  6.G  | | 6.G.2     X3 o2   1   *   +            3   1  1  1  1  .  .
%|       | |                                    
% -------                ind   1 fus ind  2            1a 6a 3a 2a 2b 2c
%                              6          2        2P  1a 3a 3a 1a 1a 1a
%                              3                   3P  1a 2a 1a 2a 2b 2c
%                              2                 
%                              3                 Y.1    1  1  1  1  1  1
%                              6                 Y.2    1  1  1  1 -1 -1
%                                                Y.3    1 -1  1 -1  1 -1
%                      X4 o2   1   *   +         Y.4    1 -1  1 -1 -1  1
%                                                Y.5    2 -1 -1  2  .  .
%                                                Y.6    2  1 -1 -2  .  .
%end

The  tables of $G,   2.G, 3.G, 6.G$  and $G.2$  are known  from the first
example, that of $2.G.2  \cong V_4$ will  be given  in the next  one.  So
here we only  print  the {\GAP} tables  of  $3.G.2 \cong D_6$  and $6.G.2
\cong D_{12}$\:

In $3.G.2$, 'X.1', 'X.2' extend $\chi_1$;  $\chi_3$ and its follower fuse
to give 'X.3', and two of the preimages of '1A' collapse.

In  $6.G.2$,  'Y.1'-'Y.4' are extensions  of  $\chi_1, \chi_2$,  so these
characters are the inflated characters  from $2.G.2$ (with respect to the
factor fusion '[ 1,  2, 1, 2, 3, 4  ]').  'Y.5' is inflated from  $3.G.2$
(with respect to the factor fusion '[ 1, 2, 2, 1,  3, 3 ]'), and 'Y.6' is
the result of the fusion of $\chi_4$ and its follower.

\newpage
For the last example, let $G$ be the group $2^2$.
Consider the following tables\:

%ignore
\begin{picture}(110,125)
\put(0,93){
\begin{picture}(29,29)
\put(0,29){\line(1,0){14}}
\put(0,15){\line(1,0){14}}
\put(0,14){\line(1,0){14}}
\put(0,0){\line(1,0){14}}
\put(15,29){\line(1,0){14}}
\put(15,15){\line(1,0){14}}
\put(15,14){\line(1,0){14}}
\put(15,0){\line(1,0){14}}
\put(0,15){\line(0,1){14}}
\put(0,0){\line(0,1){14}}
\put(14,15){\line(0,1){14}}
\put(15,15){\line(0,1){14}}
\put(29,15){\line(0,1){14}}
\put(14,0){\line(0,1){14}}
\put(15,0){\line(0,1){14}}
\put(29,0){\line(0,1){14}}
\put(7,7){\makebox(0,0){2.G}}
\put(22,7){\makebox(0,0){2.G.3}}
\put(7,22){\makebox(0,0){G}}
\put(22,22){\makebox(0,0){G.3}}
\end{picture}}

\put(81,91){\line(0,1){8}}  % fusion sign in picture
\put(39,122){\makebox(0,0)[tl]{
\small\tt
\begin{minipage}{3in}
\baselineskip0.9ex
\parskip0.2ex

\ \ \ \ ;\ \ \ @\ \ \ @\ \ \ @\ \ \ @\ \ \ ;\ \ \ ;\ \ \ @\ \par
\ \par
\ \ \ \ \ \ \ \ 4\ \ \ 4\ \ \ 4\ \ \ 4\ \ \ \ \ \ \ \ \ \ \ 1\ \par
\ \ p\ power\ \ \ A\ \ \ A\ \ \ A\ \ \ \ \ \ \ \ \ \ \ A\ \par
\ \ p\ttquote\ part\ \ \ A\ \ \ A\ \ \ A\ \ \ \ \ \ \ \ \ \ \ A\ \par
\ \ ind\ \ 1A\ \ 2A\ \ 2B\ \ 2C\ fus\ ind\ \ 3A\ \par
\ \par
$\chi_1$\ \ +\ \ \ 1\ \ \ 1\ \ \ 1\ \ \ 1\ \ \ \:\ +oo\ \ \ 1\ \par
\ \par
$\chi_2$\ \ +\ \ \ 1\ \ \ 1\ \ -1\ \ -1\ \ \ .\ \ \ +\ \ \ 0\ \par
\ \par
$\chi_3$\ \ +\ \ \ 1\ \ -1\ \ \ 1\ \ -1\ \ \ .\ \par
\ \par
$\chi_4$\ \ +\ \ \ 1\ \ -1\ \ -1\ \ \ 1\ \ \ .\ \par
\ \par
\ \ ind\ \ \ 1\ \ \ 4\ \ \ 4\ \ \ 4\ fus\ ind\ \ \ 3\ \par
\ \ \ \ \ \ \ \ 2\ \ \ \ \ \ \ \ \ \ \ \ \ \ \ \ \ \ \ \ \ \ \ 6\ \par
\ \par
$\chi_5$\ \ -\ \ \ 2\ \ \ 0\ \ \ 0\ \ \ 0\ \ \ \:\ -oo\ \ \ 1\ \par
\end{minipage}}}

\put(102,122){\makebox(0,0)[tl]{
\small\tt
\begin{minipage}{3in}
\baselineskip2.7ex
\parskip0ex
$G.3$\par
 \par
\ \ \ 2\ \ \ 2\ \ \ 2\ \ \ .\ \ \ . \par
\ \ \ 3\ \ \ 1\ \ \ .\ \ \ 1\ \ \ 1 \par
 \par
\ \ \ \ \ \ 1a\ \ 2a\ \ 3a\ \ 3b \par
\ \ 2P\ \ 1a\ \ 1a\ \ 3b\ \ 3a \par
\ \ 3P\ \ 1a\ \ 2a\ \ 1a\ \ 1a \par
 \par
X.1\ \ \ \ 1\ \ \ 1\ \ \ 1\ \ \ 1 \par
X.2\ \ \ \ 1\ \ \ 1\ \ \ A\ \ /A \par
X.3\ \ \ \ 1\ \ \ 1\ \ /A\ \ \ A \par
X.4\ \ \ \ 3\ \ -1\ \ \ .\ \ \ . \par
 \par
A\ =\ E(3) \par
\ \ =\ (-1+ER(-3))/2\ =\ b3 \par
\end{minipage}}}

\put(0,71){\makebox(0,0)[tl]{
\small\tt
\begin{minipage}{3in}
\baselineskip2.7ex
\parskip0ex
$2.G$\par
 \par
\ \ \ 2\ \ \ 3\ \ \ 3\ \ \ 2\ \ \ 2\ \ \ 2\par
 \par
\ \ \ \ \ \ 1a\ \ 2a\ \ 4a\ \ 4b\ \ 4c\par
\ \ 2P\ \ 1a\ \ 1a\ \ 2a\ \ 1a\ \ 1a\par
\ \ 3P\ \ 1a\ \ 2a\ \ 4a\ \ 4b\ \ 4c\par
 \par
X.1\ \ \ \ 1\ \ \ 1\ \ \ 1\ \ \ 1\ \ \ 1\par
X.2\ \ \ \ 1\ \ \ 1\ \ \ 1\ \ -1\ \ -1\par
X.3\ \ \ \ 1\ \ \ 1\ \ -1\ \ \ 1\ \ -1\par
X.4\ \ \ \ 1\ \ \ 1\ \ -1\ \ -1\ \ \ 1\par
X.5\ \ \ \ 2\ \ -2\ \ \ .\ \ \ .\ \ \ .\par
\end{minipage}}}

\put(50,71){\makebox(0,0)[tl]{
\small\tt
\begin{minipage}{3in}
\baselineskip2.7ex
\parskip0ex
$2.G.3$\par
 \par
\ \ \ 2\ \ \ 3\ \ \ 3\ \ \ 2\ \ \ 1\ \ \ 1\ \ \ 1\ \ \ 1\par
\ \ \ 3\ \ \ 1\ \ \ 1\ \ \ .\ \ \ 1\ \ \ 1\ \ \ 1\ \ \ 1\par
 \par
\ \ \ \ \ \ 1a\ \ 2a\ \ 4a\ \ 3a\ \ 6a\ \ 3b\ \ 6b\par
\ \ 2P\ \ 1a\ \ 1a\ \ 2a\ \ 3b\ \ 3b\ \ 3a\ \ 3a\par
\ \ 3P\ \ 1a\ \ 2a\ \ 4a\ \ 1a\ \ 2a\ \ 1a\ \ 2a\par
 \par
X.1\ \ \ \ 1\ \ \ 1\ \ \ 1\ \ \ 1\ \ \ 1\ \ \ 1\ \ \ 1\par
X.2\ \ \ \ 1\ \ \ 1\ \ \ 1\ \ \ A\ \ \ A\ \ /A\ \ /A\par
X.3\ \ \ \ 1\ \ \ 1\ \ \ 1\ \ /A\ \ /A\ \ \ A\ \ \ A\par
X.4\ \ \ \ 3\ \ \ 3\ \ -1\ \ \ .\ \ \ .\ \ \ .\ \ \ .\par
X.5\ \ \ \ 2\ \ -2\ \ \ .\ \ \ 1\ \ \ 1\ \ \ 1\ \ \ 1\par
X.6\ \ \ \ 2\ \ -2\ \ \ .\ \ \ A\ \ -A\ \ /A\ -/A\par
X.7\ \ \ \ 2\ \ -2\ \ \ .\ \ /A\ -/A\ \ \ A\ \ -A\par
 \par
A\ =\ E(3) \par
\ \ =\ (-1+ER(-3))/2\ =\ b3 \par
\end{minipage}}}
\end{picture}
%end
%display
% -------   -------          ;   @   @   @   @   ;   ;   @
%|       | |       |             4   4   4   4           1
%|   G   | |  G.3  |       p power   A   A   A           A
%|       | |       |       p' part   A   A   A           A
% -------   -------        ind  1A  2A  2B  2C fus ind  3A
% -------   -------                                    
%|       | |       |     X1  +   1   1   1   1   : +oo   1
%|  2.G  | | 2.G.3 |     X2  +   1   1  -1  -1   .   +   0
%|       | |       |     X3  +   1  -1   1  -1   .    
% -------   -------      X4  +   1  -1  -1   1   .        
%                                                      
%                          ind   1   4   4   4 fus ind   3
%                                2                       6
%
%                        X5  -   2   0   0   0   : -oo   1
%
%  G.3
%   
%     2   2   2   .   .
%     3   1   .   1   1
%   
%        1a  2a  3a  3b
%    2P  1a  1a  3b  3a
%    3P  1a  2a  1a  1a
%
%  X.1    1   1   1   1
%  X.2    1   1   A  /A
%  X.3    1   1  /A   A
%  X.4    3  -1   .   .
%
%  A = E(3) 
%    = (-1+ER(-3))/2 = b3
%
%  2.G                          2.G.3
%                                
%     2   3   3   2   2   2        2   3   3   2   1   1   1   1
%                                  3   1   1   .   1   1   1   1
%        1a  2a  4a  4b  4c     
%    2P  1a  1a  2a  1a  1a           1a  2a  4a  3a  6a  3b  6b
%    3P  1a  2a  4a  4b  4c       2P  1a  1a  2a  3b  3b  3a  3a
%                                 3P  1a  2a  4a  1a  2a  1a  2a
%  X.1    1   1   1   1   1     
%  X.2    1   1   1  -1  -1     X.1    1   1   1   1   1   1   1
%  X.3    1   1  -1   1  -1     X.2    1   1   1   A   A  /A  /A
%  X.4    1   1  -1  -1   1     X.3    1   1   1  /A  /A   A   A
%  X.5    2  -2   .   .   .     X.4    3   3  -1   .   .   .   .
%                               X.5    2  -2   .   1   1   1   1
%                               X.6    2  -2   .   A  -A  /A -/A
%                               X.7    2  -2   .  /A -/A   A  -A
%   
%                               A = E(3) 
%                                 = (-1+ER(-3))/2 = b3 
%end

In the  table  of $G.3  \cong A_4$, the  characters  $\chi_2, \chi_3$ and
$\chi_4$ fuse, and the classes '2A', '2B'  and '2C' collapse.  To get the
table of $2.G \cong Q_8$ one just has to split  the class '2A' and adjust
the representative  orders.  Finally, the  table of $2.G.3 \cong SL_2(3)$
is    given; the subgroup   fusion  corresponding  to  the injection $2.G
\hookrightarrow  2.G.3$ is '[  1, 2,  3, 3,  3  ]', and the factor fusion
corresponding to  the epimorphism $2.G.3 \rightarrow G.3$  is '[ 1, 1, 2,
3, 3, 4, 4 ]'.

%%%%%%%%%%%%%%%%%%%%%%%%%%%%%%%%%%%%%%%%%%%%%%%%%%%%%%%%%%%%%%%%%%%%%%%%%
\newpage
\Section{CAS Tables}\index{character tables!CAS}%
\index{tables!library}\index{library of character tables}

All tables of the {\CAS} table library are  available in \GAP, too.  This
sublibrary has been completely revised, i.e.,  errors have been corrected
and powermaps have been completed.

Any {\CAS} table is accessible by each of  its {\CAS} names, that is, the
table name or the filename (see "CharTable")\:

|    gap> t:= CharTable( "m10" );; t.name;
    "A6.2_3"|

One does,  however, not always get the  original {\CAS}  table\:\ In many
cases (mostly {\ATLAS} tables, see "ATLAS Tables")  not only the name but
also the succession of classes and characters has changed; the records in
the component 'CAS' of the  table (see "Character Table Records") contain
the permutations which must  be applied to classes  and characters to get
the original {\CAS} table\:

|    gap> t.CAS;
    [ rec(
          name := "m10",
          permchars := (3,5)(4,8,7,6),
          permclasses := (),
          text := [ 'n', 'a', 'm', 'e', 's', ':', ' ', ' ', ' ', ' ',
              ' ', 'm', '1', '0', '\n', 'o', 'r', 'd', 'e', 'r', ':',
              ' ', ' ', ' ', ' ', ' ', '2', '^', '4', '.', '3', '^', '2',
              '.', '5', ' ', '=', ' ', '7', '2', '0', '\n', 'n', 'u',
              'm', 'b', 'e', 'r', ' ', 'o', 'f', ' ', 'c', 'l', 'a', 's',
              's', 'e', 's', ':', ' ', '8', '\n', 's', 'o', 'u', 'r',
              'c', 'e', ':', ' ', ' ', ' ', ' ', 'c', 'a', 'm', 'b', 'r',
              'i', 'd', 'g', 'e', ' ', 'a', 't', 'l', 'a', 's', '\n',
              'c', 'o', 'm', 'm', 'e', 'n', 't', 's', ':', ' ', ' ', 'p',
              'o', 'i', 'n', 't', ' ', 's', 't', 'a', 'b', 'i', 'l', 'i',
              'z', 'e', 'r', ' ', 'o', 'f', ' ', 'm', 'a', 't', 'h', 'i',
              'e', 'u', '-', 'g', 'r', 'o', 'u', 'p', ' ', 'm', '1', '1',
              '\n', 't', 'e', 's', 't', ':', ' ', ' ', ' ', ' ', ' ',
              ' ', 'o', 'r', 't', 'h', ',', ' ', 'm', 'i', 'n', ',', ' ',
              's', 'y', 'm', '[', '3', ']', ' ', ' ', ' ', ' ', ' ', ' ',
              ' ', ' ', ' ', ' ', ' ', ' ', ' ', ' ', ' ', ' ', ' ', ' ',
              ' ', ' ', ' ', ' ', ' ', ' ', ' ', ' ', ' ', ' ', '\n' ] ) ]|

The subgroup  fusions were computed anew;  their record  component 'text'
tells if the  fusion is equal to that  in the {\CAS} library --of  course
modulo the permutation of classes.

*Note* that the fusions are  neither tested to  be consistent for any two
subgroups of a group and their  intersection, nor tested to be consistent
with respect to composition of maps.

%%%%%%%%%%%%%%%%%%%%%%%%%%%%%%%%%%%%%%%%%%%%%%%%%%%%%%%%%%%%%%%%%%%%%%%%%
\Section{Organization of the Table Libraries}

The *primary files* are 'TBLNAME/ctadmin.tbl' and 'TBLNAME/ctprimar.tbl'.
The former contains the evaluation function 'CharTableLibrary'
(see "CharTable") and some utilities, the latter contains the global
variable 'LIBLIST' which encodes all information where to find library
tables; the file 'TBLNAME/ctprimar.tbl' can be constructed from the data
files of the table libraries using the 'awk' script 'maketbl' in the
'etc' directory of the {\GAP} distribution.

Also the  *secondary  files* are  all stored in  the directory 'TBLNAME';
they are

|    clmelab.tbl   clmexsp.tbl   ctadmin.tbl   ctbalter.tbl  ctbatres.tbl
    ctbconja.tbl  ctbfisc1.tbl  ctbfisc2.tbl  ctbline1.tbl  ctbline2.tbl
    ctbline3.tbl  ctbline4.tbl  ctbline5.tbl  ctbmathi.tbl  ctbmonst.tbl
    ctborth1.tbl  ctborth2.tbl  ctborth3.tbl  ctbspora.tbl  ctbsympl.tbl
    ctbtwis1.tbl  ctbtwis2.tbl  ctbunit1.tbl  ctbunit2.tbl  ctbunit3.tbl
    ctbunit4.tbl  ctgeneri.tbl  ctoalter.tbl  ctoatres.tbl  ctocliff.tbl
    ctoconja.tbl  ctofisc1.tbl  ctofisc2.tbl  ctoholpl.tbl  ctoinert.tbl
    ctoline1.tbl  ctoline2.tbl  ctoline3.tbl  ctoline4.tbl  ctoline5.tbl
    ctoline6.tbl  ctomathi.tbl  ctomaxi1.tbl  ctomaxi2.tbl  ctomaxi3.tbl
    ctomaxi4.tbl  ctomaxi5.tbl  ctomaxi6.tbl  ctomisc1.tbl  ctomisc2.tbl
    ctomisc3.tbl  ctomisc4.tbl  ctomisc5.tbl  ctomisc6.tbl  ctomonst.tbl
    ctonews.tbl   ctoorth1.tbl  ctoorth2.tbl  ctoorth3.tbl  ctoorth4.tbl
    ctoorth5.tbl  ctospora.tbl  ctosylno.tbl  ctosympl.tbl  ctotwis1.tbl
    ctotwis2.tbl  ctounit1.tbl  ctounit2.tbl  ctounit3.tbl  ctounit4.tbl|

The names start with 'ct' for  ``character table\'\', followed by 'o' for
``ordinary\'\', 'b'  for ``Brauer\'\'\ or  'g' for ``generic\'\', then an
up  to  5  letter description of    the contents, e.g.,   'alter' for the
alternating groups, and the extension '.tbl'.

The file  'ctb<descr>.tbl'   contains    (at most)  the   Brauer   tables
corresponding to the ordinary tables in 'cto<descr>.tbl'.

The *format of library tables* is always like this\:

|    MOT(|<tblname>|,
          ...
               # here the data components are stored
          ... );|

Here <tblname> is the value of the 'identifier' component of the table,
e.g.\ '\"A5\"'.

For the contents of the table record, there are  three different ways how
tables are stored\:

*Full  tables* (like that  of $A_5$)  are stored similar  to the internal
format (see "Character  Table Records").   Lists of characters,  however,
will be abbreviated in the following way\:

For each subset of characters which differ  just by multiplication with a
linear character or by Galois conjugacy, only one is given by its values,
the  others are replaced by   '[TENSOR,[<i>,<j>]]' (which means that  the
character is the tensor  product of the <i>-th  and the <j>-th character)
or   '[GALOIS,[<i>,<j>]]' (which means that  the  character is the <j>-th
Galois conjugate of the <i>-th character.

*Brauer tables* (like that  of $A_5$ mod $2$) are  stored relative to the
corresponding ordinary table; instead of irreducible characters the files
contain decomposition matrices or Brauer  trees for the blocks of nonzero
defect  (see "Brauer Table Records"), and  components which can be got by
restriction to $p$--regular classes are not stored at all.

*Construction   tables*  (like that   of   $O_8^-(3)M7$) have a component
'construction'  that is a  function  of one  variable.   This function is
called by 'CharTable'  (see "CharTable") when  the table is  constructed,
i.e.\ *not* when the file containing the table is read.

The aim of this rather complicated way to store a character table is that
big  tables with a simple structure  (e.g. direct products) can be stored
in a very compact way.

Another  special case where  construction  tables are useful  is  that of
projective tables\:

In  their   component 'irreducibles'  they   do  not  contain irreducible
characters  but a list with   information about the factor groups\:\  Any
entry is a list of length 2 that  contains at position 1  the name of the
table  of the factor group, at  the  second position  a  list of integers
representing the Galois automorphisms to get  follower characters.  E.g.,
for $12.M_{22}$, the value of 'irreducibles' is

|    [["M22",[]],["2.M22",[]],
     ["3.M22",[-1,-13,-13,-1,23,23,-1,-1,-1,-1,-1]],
     ["4.M22",[-1,-1,15,15,23,23,-1,-1]],,
     ["6.M22",[-13,-13,-1,23,23,-1,-7,-7,-1,-1]],,,,,,
     ["12.M22",[[17,-17,-1],[17,-17,-1],[-55,-377,-433],[-55,-377,-433],
     [89,991,1079],[89,991,1079],[-7,7,-1]]]]|

Using this and the  'projectives' component of the  table of the smallest
nontrivial  factor  group,     "CharTable"  'CharTable'  constructs   the
irreducible  characters.     The  table  head,  however,   need    not be
constructed.

%%%%%%%%%%%%%%%%%%%%%%%%%%%%%%%%%%%%%%%%%%%%%%%%%%%%%%%%%%%%%%%%%%%%%%%%%
\Section{How to Extend a Table Library}\index{library tables!add}%
\index{tables!add to a library}\index{NotifyCharTable}%
\index{PrintToLib}

If you have some ordinary character tables which are  not (or not yet) in
a {\GAP}  table library, but  which you want  to treat as library tables,
e.g., assign  them  to variables  using "CharTable" 'CharTable',  you can
include these tables.  For that, two things must be done\:

First you must notify each table, i.e., tell  {\GAP} on which file it can
be found, and which names are admissible; this can be done using

'NotifyCharTable( <firstname>, <filename>, <othernames> )',

with strings <firstname> (the  'identifier'  component of the table)  and
<filename> (the name  of  the  file  containing the  table,  relative  to
'TBLNAME', and without extension  '.tbl'),   and a list <othernames>   of
strings which are other admissible names of the table (see "CharTable").

'NotifyCharTable' will add the necessary information to 'LIBLIST'.
A warning is printed for each table <libtbl> that was already accessible
by some of the names, and these names are ignored for the new tables.
Of course this affects only the value of 'LIBLIST' in the current {\GAP}
session, not that on the file.

*Note* that an error   is  raised if you   want  to notify a  table  with
<firstname> or  name  in <othernames> which  is already  the 'identifier'
component of a library table.

|    gap> Append( TBLNAME, ";tables/" );
    # tells {\GAP} that the directory 'tables' is a place to look for
    # library tables
    gap> NotifyCharTable( "Private", "mytables", [ "My" ] );
    # tells {\GAP} that the table with names '\"Private\"' and '\"My\"'
    # is stored on file 'mytables.tbl'
    gap> FirstNameCharTable( "My" );
    "Private"
    gap> FileNameCharTable( "My" );
    "mytables"|

The  second  condition is that each  file  must contain tables in library
format  as  described in  "Organization  of the Table Libraries";  in the
example, the contents of the file 'tables/mytables.tbl' may be this\:

|    SET_TABLEFILENAME("mytables");
    ALN:= Ignore;
    MOT("Private",
    [
    "my private character table"
    ],
    [2,2],
    [],
    [[1,1],[1,-1]],
    []);
    ALN("Private",["my"]);
    LIBTABLE.LOADSTATUS.("mytables"):="loaded";|

We simulate reading this file by explicitly assigning some of the
components.

|    gap> LIBTABLE.("mytables"):= rec(
    > Private:= rec( identifier:= "Private",
    >                centralizers:= [2,2],
    >                irreducibles:= [[1,1],[1,-1]] ) );;
    gap> LIBTABLE.LOADSTATUS.("mytables"):="loaded";;|

Now the private table is a library table\:

|    gap> CharTable( "My" );
    CharTable( "Private" ) |

To append the table <tbl> in library format to the file with name <file>,
use

'PrintToLib( <file>, <tbl> )'.

*Note*  that here <file> is the  absolute name of  the file, not the name
relative to 'TBLNAME'.  Thus the filename in the  row with the assignment
to 'LIBTABLE' must be adjusted to make the file a library file.

%%%%%%%%%%%%%%%%%%%%%%%%%%%%%%%%%%%%%%%%%%%%%%%%%%%%%%%%%%%%%%%%%%%%%%%%%
\Section{FirstNameCharTable}

'FirstNameCharTable( <name> )'

returns the  value of the 'identifier' component   of the character table
with admissible name <name>, if exists; otherwise 'false' is returned.

For each admissible name, also the lowercase string is admissible.

|    gap> FirstNameCharTable( "m22mod3" );
    "M22mod3"
    gap> FirstNameCharTable( "s5" );
    "A5.2"
    gap> FirstNameCharTable( "J5" );
    false|

%%%%%%%%%%%%%%%%%%%%%%%%%%%%%%%%%%%%%%%%%%%%%%%%%%%%%%%%%%%%%%%%%%%%%%%%%
\Section{FileNameCharTable}

'FileNameCharTable( <tblname> )'

returns  the value of the 'filename'  component of the information record
in 'LIBLIST' for the table with admissible name <tblname>, if exists;
otherwise 'false' is returned.

|    gap> FileNameCharTable( "M22mod3" );
    "ctbmathi"
    gap> FileNameCharTable( "J5" );
    false|



