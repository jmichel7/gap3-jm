%%%%%%%%%%%%%%%%%%%%%%%%%%%%%%%%%%%%%%%%%%%%%%%%%%%%%%%%%%%%%%%%%%%%%%%%%%%%%
%%
%A  chvsperm.tex       CHEVIE documentation       Jean Michel
%%
%Y  Copyright (C) 2018 - 2018  University  Paris VII.
%%
%%  This  file  contains functions to deal with signed permutations.
%%%%%%%%%%%%%%%%%%%%%%%%%%%%%%%%%%%%%%%%%%%%%%%%%%%%%%%%%%%%%%%%%%%%%%%%%

\Chapter{Signed permutations}

A   *signed  permutation*  of   |[1..n]|  is  a   permutation  of  the  set
$\{-n,\ldots,-1,1,\ldots,n\}$  which  preserves  the  pairs $[-i,i]$. It is
represented  internally  as  the  images  of  $[1..n]$.  It is printed as a
product of signed cycles.

A  signed permutation  can be  represented in  two other  ways which may be
convenient. The first way is to replace the integers
$\{1,\ldots,n,-n,\ldots,-1\}$   by  $\{1,3,\ldots,2n-1,2,4,\ldots,2n\}$  to
have   \GAP\  permutations,  which  form   the  hyperoctaedral  group  (see
"CoxeterGroupHyperoctaedralGroup").

The  second way is to represent the signed permutation by a monomial matrix
with  entries  '1'  or  '-1'.  If  such  a matrix <m> represents the signed
permutation <sp>, then |l*m| is the same as 'Permuted(<l>,<sp>)'.

%%%%%%%%%%%%%%%%%%%%%%%%%%%%%%%%%%%%%%%%%%%%%%%%%%%%%%%%%%%%%%%%%%%%%%%%%
\Section{Permuted for signed permutations}%
\index{Permuted for signed permutations}%

'Permuted( <l>, <sp>)'

'Permuted'  returns a new list  <n> that contains the  elements of the list
<l>  permuted according to the signed permutation <sp>. If <sp> is given as
a list, then '<n>[AbsInt(i\^sp)] = <l>[<i>]SignInt(i\^sp)'.

|    gap> p:=SignedPerm([-2,-1,-3]);
    (1,-2)(3,-3)
    gap> Permuted([20,30,40],p);
    [ -30, -20, -40 ]|

%%%%%%%%%%%%%%%%%%%%%%%%%%%%%%%%%%%%%%%%%%%%%%%%%%%%%%%%%%%%%%%%%%%%%%%%%
\Section{PermutationMat for signed permutations}%
\index{PermutationMat for signed permutations}%

'PermutationMat( <sp> [,<d>])'

This   function  returns  the  signed  permutation  matrix  of  the  signed
permutation  <sp>. This is a matrix  <m> such that |Permuted(l,sp)=l*m| for
any list of numbers l. If an additional argument <d> is given the matrix is
returned of that dimension.

|    gap> p:=SignedPerm([-2,-1,-3]);
    (1,-2)(3,-3)
    gap> PermutationMat(p);
    [ [ 0, -1, 0 ], [ -1, 0, 0 ], [ 0, 0, -1 ] ]|

%%%%%%%%%%%%%%%%%%%%%%%%%%%%%%%%%%%%%%%%%%%%%%%%%%%%%%%%%%%%%%%%%%%%%%%%%
\Section{SignedPerm}%
\index{SignedPerm}%

'SignedPerm( <sp> [,<d> or <sgns>])'

This  function converts to a  signed permutation a list,  an element of the
hyperoctaedral  group,  a  signed  permutation  matrix,  or  a  pair  of  a
permutation   and  of  a  list  of  signs.  If  given  an  element  of  the
hyperoctaedral  group,  the  rank  <d>  of  that  group  can be given as an
argument,   otherwise  a  representation  of  <sp>   as  a  list  is  given
corresponding to the smallest hyperoctaedral group to which it belongs.

|    gap> SignedPerm([[0,-1,0],[0,0,-1],[-1,0,0]]);
    (1,-2,3,-1,2,-3)
    gap> SignedPerm((1,4,5,2,3,6)); 
    (1,-2,3,-1,2,-3)
    gap> SignedPerm((1,2,3),[-1,-1,-1]);
    (1,-2,3,-1,2,-3)
    gap> SignedPerm([-2,-3,-1]);
    (1,-2,3,-1,2,-3)|

%%%%%%%%%%%%%%%%%%%%%%%%%%%%%%%%%%%%%%%%%%%%%%%%%%%%%%%%%%%%%%%%%%%%%%%%%
\Section{Cycles for signed permutations}%
\index{Cycles for signed permutations}%

'Cycles( <sp> )'

Returns   the  list   of  cycles   of  the   signed  permutation   <sp>  on
$\{-n,\ldots,-1,1,\ldots,n\}$.  If one  cycle is  the negative  of another,
only one of the two cycles is given.

|    gap> Cycles(SignedPerm([-2,-3,-1]));
    [ [ 1, -2, 3, -1, 2, -3 ] ]
    gap> Cycles(SignedPerm([-2,-1,-3]));
    [ [ 1, -2 ], [ 3, -3 ] ]
    gap> Cycles(SignedPerm([-2,-1,3]));
    [ [ 1, -2 ] ]|

%%%%%%%%%%%%%%%%%%%%%%%%%%%%%%%%%%%%%%%%%%%%%%%%%%%%%%%%%%%%%%%%%%%%%%%%%
\Section{SignedPermListList}%
\index{SignedPermListList}%

    'SignedPermListList( <list1>, <list2> )'

'SignedPermListList'  returns a signed  permutation that may  be applied to
<list1> to obtain <list2>, if there is one. Otherwise it returns 'false'.

|    gap> SignedPermListList([20,30,40],[-40,-20,-30]);
    (1,-2,3,-1,2,-3)
    gap> Permuted([20,30,40],last);
    [ -40, -20, -30 ]|

%%%%%%%%%%%%%%%%%%%%%%%%%%%%%%%%%%%%%%%%%%%%%%%%%%%%%%%%%%%%%%%%%%%%%%%%%
\Section{SignedMatStab}%
\index{SignedMatStab}%

'SignedMatStab(<M>[, <l>])'

Finds the stabilizer of <M> in the group of signed permutations.

|    gap> uc:=UnipotentCharacters(ComplexReflectionGroup(6));
    UnipotentCharacters( G6 )
    gap> SignedMatStab(Fourier(uc.families[2]));
    Group( (2,19)(4,-14)(5,20)(7,12), (1,-1)(2,-2)(3,-3)(4,-4)(5,-5)(6,-6)\
    (7,-7)(8,-8)(9,-9)(10,-10)(11,-11)(12,-12)(13,-13)(14,-14)(15,-15)(16,\
    -16)(17,-17)(18,-18)(19,-19)(20,-20)(21,-21)(22,-22), (1,3)(2,19)(4,-1\
    4)(5,-5)(6,-18)(7,-7)(8,10)(11,15)(12,-12)(13,22)(16,21)(17,-17)(20,-2\
    0), (1,6)(2,-19)(3,-18)(4,14)(8,16)(9,-9)(10,21)(11,-13)(15,-22), (1,1\
    1)(3,15)(4,14)(6,-13)(7,-12)(8,-10)(9,-9)(16,-21)(18,22) )
    gap> Size(last);
    32|

%%%%%%%%%%%%%%%%%%%%%%%%%%%%%%%%%%%%%%%%%%%%%%%%%%%%%%%%%%%%%%%%%%%%%%%%%
\Section{SignedPermMatMat}%
\index{SignedPermMatMat}%

'SignedPermMatMat( <M> , <N> [, <l1>, <l2>])'

<M>  and <N>  should be  symmetric matrices.  'PermMatMat' returns a signed
permutation <p> such that 'OnMatrices(M,p)=N' if such a permutation exists,
and  'false'  otherwise.  If  list  arguments  <l1> and <l2> are given, the
permutation <p> should also satisfy 'Permuted(l1,p)=l2'.

This  routine is  useful to  identify two  objects which are isomorphic but
with  different  labelings.  It  is  used  in  \CHEVIE\ to identify Lusztig
Fourier  transform matrices  with standard  (classified) data.  The program
uses  sophisticated  algorithms,  and  can  often  handle  matrices  up  to
$80\times 80$.

|    gap> f:=SubFamilyij(CHEVIE.families.X(12),1,3,(3+ER(-3))/2);
    Family("RZ/12^2[1,3]")
    gap> M:=Fourier(ComplexConjugate(f));;
    gap> uc:=UnipotentCharacters(ComplexReflectionGroup(6));
    UnipotentCharacters( G6 )
    gap> N:=Fourier(uc.families[2]);;
    gap> SignedPermMatMat(M,N);
    (1,13)(2,19,-2,-19)(3,22)(4,-4)(5,-5)(6,-11)(7,12)(8,21,-8,-21)(9,-9)(\
    10,16)(15,-18,-15,18)
    gap> OnMatrices(M,last)=N;
    true|

%%%%%%%%%%%%%%%%%%%%%%%%%%%%%%%%%%%%%%%%%%%%%%%%%%%%%%%%%%%%%%%%%%%%%%%%%
