%%%%%%%%%%%%%%%%%%%%%%%%%%%%%%%%%%%%%%%%%%%%%%%%%%%%%%%%%%%%%%%%%%%%%%%%%%%%%
%%
%A  chvsperm.tex       CHEVIE documentation       Jean Michel
%%
%Y  Copyright (C) 2018 - 2018  University  Paris VII.
%%
%%  This  file  contains functions to deal with signed permutations.
%%%%%%%%%%%%%%%%%%%%%%%%%%%%%%%%%%%%%%%%%%%%%%%%%%%%%%%%%%%%%%%%%%%%%%%%%

\Chapter{Signed permutations}

A   *signed  permutation*  of   |[1..n]|  is  a   permutation  of  the  set
$\{-n,\ldots,-1,1,\ldots,n\}$  which  preserves  the  pairs $[-i,i]$. It is
represented  as the images of $[1..n]$.

A  signed permutation  can be  represented in  two other  ways which may be
convenient.  The first way is to replace the integers $\{-n,\ldots,-1\}$ by
$\{n+1,\ldots,2n\}$   to   have   \GAP\   permutations,   which   form  the
hyperoctaedral group (see "CoxeterGroupHyperoctaedralGroup").

The  second way is to represent the signed permutation by a monomial matrix
with  entries  '1'  or  '-1'.  If  such  a matrix <m> represents the signed
permutation <sp>, then |l*m| is the same as 'SignPermuted(<l>,<sp>)'.

%%%%%%%%%%%%%%%%%%%%%%%%%%%%%%%%%%%%%%%%%%%%%%%%%%%%%%%%%%%%%%%%%%%%%%%%%
\Section{SignPermuted}%
\index{SignPermuted}%

'SignPermuted( <l>, <sp>)'

'SignPermuted'  returns a  new list  <n> that  contains the elements of the
list  <l> permuted  according to  the signed  permutation <sp>.  If <sp> is
given  as a  list, then  '<n>[AbsInt(sp[i])] = <l>[<i>]SignInt(sp[i])'. The
signed   permutation  <sp>  can  also  be   given  as  an  element  of  the
hyperoctaedral group (see the introduction of the chapter for definitions).

|    gap> SignPermuted([20,30,40],[-2,-1,-3]);
    [ -30, -20, -40 ]
    gap> W:=CoxeterGroupHyperoctaedralGroup(3);
    Group( (3,4), (2,3)(4,5), (1,2)(5,6) )
    gap> SignPermuted([20,30,40],W.3);
    [ 30, 20, 40 ]
    gap> SignPermuted([20,30,40],W.2);
    [ 20, 40, 30 ]
    gap> SignPermuted([20,30,40],W.1);
    [ 20, 30, -40 ]|

%%%%%%%%%%%%%%%%%%%%%%%%%%%%%%%%%%%%%%%%%%%%%%%%%%%%%%%%%%%%%%%%%%%%%%%%%
\Section{SignedPermutationMat}%
\index{SignedPermutationMat}%

'SignedPermutationMat( <sp> [,<d>])'

This   function  returns  the  signed  permutation  matrix  of  the  signed
permutation  <sp>, given as a  list or as an  element of the hyperoctaedral
group.  This is a matrix <m> such that |SignPermuted(l,sp)=l*m|. If <sp> is
an  element of hyperoctaedral  group, the matrix  is given of dimension the
rank  of the  smallest hyperoctaedral  group to  which <sp>  belongs. If an
additional argument <d> is given the matrix is returned of that dimension.

|    gap> m:=SignedPermutationMat([-2,-3,-1]);
    [ [ 0, -1, 0 ], [ 0, 0, -1 ], [ -1, 0, 0 ] ]
    gap> m=SignedPermutationMat((1,5,3,6,2,4));
    true
    gap> [20,30,40]*m;
    [ -40, -20, -30 ]
    gap> SignPermuted([20,30,40],[-2,-3,-1]);
    [ -40, -20, -30 ]|

%%%%%%%%%%%%%%%%%%%%%%%%%%%%%%%%%%%%%%%%%%%%%%%%%%%%%%%%%%%%%%%%%%%%%%%%%
\Section{SignedPerm}%
\index{SignedPerm}%

'SignedPerm( <sp> [,<d> or <sgns>])'

This  function converts to a signed permutation  given as a list, either an
element of the hyperoctaedral group, a signed permutation matrix, or a pair
of  a  permutation  and  of  a  list  of  signs. If given an element of the
hyperoctaedral  group,  the  rank  <d>  of  that  group  can be given as an
argument,   otherwise  a  representation  of  <sp>   as  a  list  is  given
corresponding to the smallest hyperoctaedral group to which it belongs.

Finally, if given a signed permutation as a list, this function returns
am element of the hyperoctaedral group.

|    gap> SignedPerm([[0,-1,0],[0,0,-1],[-1,0,0]]);
    [ -2, -3, -1 ]
    gap> SignedPerm((1,5,3,6,2,4));
    [ -2, -3, -1 ]
    gap> SignedPerm((1,2,3),[-1,-1,-1]);
    [ -2, -3, -1 ]
    gap> SignedPerm([-2,-3,-1]);
    (1,5,3,6,2,4)|

%%%%%%%%%%%%%%%%%%%%%%%%%%%%%%%%%%%%%%%%%%%%%%%%%%%%%%%%%%%%%%%%%%%%%%%%%
\Section{CyclesSignedPerm}%
\index{CyclesSignedPerm}%

'CyclesSignedPerm( <sp> )'

Returns   the  list   of  cycles   of  the   signed  permutation   <sp>  on
$\{-n,\ldots,-1,1,\ldots,n\}$,  given as  a list  or a  permutation. If one
cycle is the negative of another, only one of the two cycles is given.

|    gap> CyclesSignedPerm([-2,-3,-1]);
    [ [ 1, -2, 3, -1, 2, -3 ] ]
    gap> CyclesSignedPerm([-2,-1,-3]);
    [ [ 1, -2 ], [ 3, -3 ] ]
    gap> CyclesSignedPerm([-2,-1,3]);
    [ [ 1, -2 ] ]
    gap> CyclesSignedPerm((1,5,3,6,2,4));
    [ [ 1, -2, 3, -1, 2, -3 ] ]|

%%%%%%%%%%%%%%%%%%%%%%%%%%%%%%%%%%%%%%%%%%%%%%%%%%%%%%%%%%%%%%%%%%%%%%%%%
\Section{SignedPermListList}%
\index{SignedPermListList}%

    'SignedPermListList( <list1>, <list2> )'

'SignedPermListList'  returns a signed  permutation that may  be applied to
<list1> to obtain <list2>, if there is one. Otherwise it returns 'false'.

|    gap> SignedPermListList([20,30,40],[-40,-20,-30]);
    [ -2, -3, -1 ]
    gap> SignPermuted([20,30,40],[-2,-3,-1]);
    [ -40, -20, -30 ]|

%%%%%%%%%%%%%%%%%%%%%%%%%%%%%%%%%%%%%%%%%%%%%%%%%%%%%%%%%%%%%%%%%%%%%%%%%
