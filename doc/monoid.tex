%%%%%%%%%%%%%%%%%%%%%%%%%%%%%%%%%%%%%%%%%%%%%%%%%%%%%%%%%%%%%%%%%%%%%%%%%%%%%
%%
%A  GAP documentation                                   Goetz.Pfeiffer@UCG.IE
%%
%A  $Id: monoid.tex,v 2.2 1997/11/26 09:59:49 goetz Exp $
%%
%Y  Copyright (C) 1997, Mathematics Dept, University College Galway, Ireland.
%%
%%  This file documents the general functions for monoids & semigroups.
%%
\Chapter{Monoids and Semigroups}

Semigroups and, even  more,  monoids are not  far  away  from being  like
groups.  But, surprisingly, they have not  received much attention yet in
the form  of {\GAP} programs.  This  small collection of files and manual
chapters is an attempt to start closing this gap.

The only difference  between a semigroup  and a monoid is one  element\:\ 
the  identity.  Although this   may  lead to   subtle differences in  the
behavior of these structures the underlying  assumption of these programs
is that you can always, by means of adding an element with the properties
of an identity, turn a semigroup into a monoid.  So most of the functions
will only be available  for monoids and not  for semigroups.   The actual
process of adding an identity is also not supported at the moment.

The emphasis of  this package is on  transformation monoids (see  chapter
"Transformation  Monoids").   However, it seemed  to  be a  good idea to
provide  some of the framework for  general monoids (this chapter) before
concentrating on   the   special  case.    Separate   chapters  introduce
transformations (see chapter "Transformations") and binary relations (see
chapter "Binary Relations") as special types of monoid elements.  Another
chapter treats  several ways of how a  monoid can  act on certain domains
(see chapter "Actions of Monoids").

For  a general treatment  of  the theory  of  monoids and  transformation
monoids   see~\cite{Lallement79}  and   \cite{Howie95}.      A   detailed
description  of  this implementation and   the theory behind  it is given
in~\cite{LPRR1} and ~\cite{LPRR2}.

A semigroup is constructed by the  function 'SemiGroup' (see "SemiGroup")
and a monoid is constructed by the function 'Monoid' (see "Monoid").

Note that monoid elements usually  exist independently of a monoid, e.g.,
you can write down two transformations  and compute their product without
ever defining a monoid that contains them.

The chapter starts with a description  of monoid elements, i.e. all those
objects that  can be  element   of  a semigroup  or    of a monoid   (see
"Comparisons of Monoid  Elements", "Operations for Monoid  Elements", and
"IsMonoidElement").  Then the    functions which  construct   monoids and
semigroups  and the  functions which  test  whether  a given  object is a
monoid or a    semigroup are described   (see "SemiGroup", "IsSemiGroup",
"Monoid" and "IsMonoid").

Monoids  and semigroups are domains, so  every set theoretic function for
domains is applicable  to them (see  "Set Functions for Monoids").  There
are functions which  construct Green Classes of  various types as subsets
of   a  monoid (see "Green  Classes",    "RClass", "LClass", "DClass" and
"HClass"), functions  which test whether a given  object is a Green class
of   a certain    type (see   "IsRClass",    "IsLClass", "IsDClass"   and
"IsHClass"), and functions which determine the  list of all Green Classes
of some  given type of  a monoid (see  "RClasses", "LClasses", "DClasses"
and "HClasses").

The next sections describe how set functions are applied to Green classes
(see "Set Functions for Green Classes")  and how to compute various kinds
of Sch\accent127utzenberger groups (see "SchutzenbergerGroup").

The final sections describe how to determine the  idempotents of a monoid
(see   "Idempotents"), the lack  of  support for homomorphisms of monoids
(see "Monoid Homomorphisms") and how  monoids are represented by  records
in {\GAP} (see "Monoid Records and Semigroup Records").

The functions described here are in the file '\"monoid.g\"'.

%%%%%%%%%%%%%%%%%%%%%%%%%%%%%%%%%%%%%%%%%%%%%%%%%%%%%%%%%%%%%%%%%%%%%%%%%
\Section{Comparisons of Monoid Elements}%
\index{equality!of monoid elements}%
\index{ordering!of monoid elements}

'<s> = <t>' \\
'<s> \<> <t>'

The equality operator '=' evaluates to 'true'  if the monoid elements <s>
and <t> are   equal and to 'false'  otherwise.    The inequality operator
'\<>' evaluates to  'true'  if the monoid elements   <s> and <t>  are not
equal and to 'false' otherwise.

You can compare monoid  elements with objects of  other types.  Of course
they are never equal.  Standard monoid  elements are transformations (see
chapter "Transformations")   and binary relations   (see chapter  "Binary
Relations").

\vspace{5mm}
'<s> \<\ <t>' \\
'<s> \<= <t>' \\
'<s> >= <t>' \\
'<s> > <t>'

The operators '\<', '\<=', '>=' and '>'  evaluate to 'true' if the monoid
element <s> is strictly less than, less than or equal to, greater than or
equal to and strictly  greater than the monoid  element <t>.  There is no
general ordering on monoid elements.

Standard monoid   elements may be  compared  with objects of  other types
while generic monoid elements may disallow such a comparison.

%%%%%%%%%%%%%%%%%%%%%%%%%%%%%%%%%%%%%%%%%%%%%%%%%%%%%%%%%%%%%%%%%%%%%%%%%
\Section{Operations for Monoid Elements}

'<s> \*\ <t>'%
\index{product!of monoid elements}

The operator '\*' evaluates to the product of the two monoid elements <s>
and <t>.   The  operands must of  course lie  in a  common parent monoid,
otherwise an error is signaled.

\vspace{5mm}
'<s> \^\ <i>'%
\index{power!of monoid elements}

The powering  operator '\^' returns the  <i>-th power of a monoid element
<s> and a positive integer <i>.  If <i> is zero  the identity of a parent
monoid of <s> is returned.

\vspace{5mm}
'<list> \*\ <s>'%
\index{product!of list and monoid element}\\
'<s> \*\ <list>'

In this form the operator '\*' returns a new list where each entry is the
product of <s>  and the   corresponding  entry  of <list>.    Of   course
multiplication must be defined between <s> and each entry of <list>.

%%%%%%%%%%%%%%%%%%%%%%%%%%%%%%%%%%%%%%%%%%%%%%%%%%%%%%%%%%%%%%%%%%%%%%%%%
\Section{IsMonoidElement}%
\index{test!for monoid element}

'IsMonoidElement( <obj> )'

'IsMonoidElement' returns  'true' if the  object  <obj>, which may  be an
object of arbitrary type, is a monoid element, and 'false' otherwise.  It
will signal an error if <obj> is an unbound variable.

|    gap> IsMonoidElement( 10 );
    false
    gap> IsMonoidElement( Transformation( [ 1, 2, 1 ] ) );
    true |

%%%%%%%%%%%%%%%%%%%%%%%%%%%%%%%%%%%%%%%%%%%%%%%%%%%%%%%%%%%%%%%%%%%%%%%%%%%%%
\Section{SemiGroup}

'SemiGroup( <list> )'

'SemiGroup' returns  the semigroup   generated   by the list  <list>   of
semigroup elements.

|    gap> SemiGroup( [ Transformation( [ 1, 2, 1 ] ) ] );
    SemiGroup( [ Transformation( [ 1, 2, 1 ] ) ] ) |

\vspace{5mm}
'SemiGroup( <gen1>, <gen2>, ... )'

In this form 'SemiGroup' returns the   semigroup generated by the  semigroup  elements
<gen1>, <gen2>, ...

|    gap> SemiGroup( Transformation( [ 1, 2, 1 ] ) );                     
    SemiGroup( [ Transformation( [ 1, 2, 1 ] ) ] ) |

%%%%%%%%%%%%%%%%%%%%%%%%%%%%%%%%%%%%%%%%%%%%%%%%%%%%%%%%%%%%%%%%%%%%%%%%%
\Section{IsSemiGroup}%
\index{test!for semigroup}

'IsSemiGroup( <obj> )'

'IsSemiGroup' returns 'true' if the object <obj>,  which may be an object
of an  arbitrary  type, is a semigroup,  and  'false' otherwise.  It will
signal an error if <obj> is an unbound variable.

|    gap> IsSemiGroup( SemiGroup( Transformation( [ 1, 2, 1 ] ) ) );
    true
    gap> IsSemiGroup( Group( (2,3) ) ); 
    false |

%%%%%%%%%%%%%%%%%%%%%%%%%%%%%%%%%%%%%%%%%%%%%%%%%%%%%%%%%%%%%%%%%%%%%%%%%
\Section{Monoid}

'Monoid( <list> )'\\
'Monoid( <list> , <id> )'

'Monoid'  returns  the monoid generated  by  the  list <list>  of  monoid
elements.  If present, <id> must be the identity of this monoid.

|    gap> Monoid( [ Transformation( [ 1, 2, 1 ] ) ], 
    > IdentityTransformation( 3 ) );
    Monoid( [ Transformation( [ 1, 2, 1 ] ) ] ) |

\vspace{5mm}
'Monoid( <gen1>, <gen2>, ... )'

In this form 'Monoid' returns the monoid generated by the monoid elements
<gen1>, <gen2>, ...

|    gap> Monoid( Transformation( [ 1, 2, 1 ] ) );                     
    Monoid( [ Transformation( [ 1, 2, 1 ] ) ] ) |

%%%%%%%%%%%%%%%%%%%%%%%%%%%%%%%%%%%%%%%%%%%%%%%%%%%%%%%%%%%%%%%%%%%%%%%%%
\Section{IsMonoid}%
\index{test!for monoid}

'IsMonoid( <obj> )'

'IsMonoid' returns 'true' if the object <obj>,  which may be an object of
an arbitrary type, is a monoid, and 'false' otherwise.  It will signal an
error if <obj> is an unbound variable.

|    gap> IsMonoid( Monoid( Transformation( [ 1, 2, 1 ] ) ) );
    true
    gap> IsMonoid( Group( (2,3) ) ); 
    false |

%%%%%%%%%%%%%%%%%%%%%%%%%%%%%%%%%%%%%%%%%%%%%%%%%%%%%%%%%%%%%%%%%%%%%%%%%
\Section{Set Functions for Monoids}

Monoids and semigroups  are domains.   Thus  all set theoretic  functions
described  in chapter \"Domains\"\   should be  applicable  to   monoids. 
However, no generic method is  installed yet.  Of particular interest are
the functions 'Size'   and  'Elements' which will  have   special methods
depending on the kind of monoid being dealt with.

%%%%%%%%%%%%%%%%%%%%%%%%%%%%%%%%%%%%%%%%%%%%%%%%%%%%%%%%%%%%%%%%%%%%%%%%%
\Section{Green Classes}

Green classes  are special subsets of a  monoid.  In particular, they are
domains    so all  set  theoretic   functions  for   domains (see chapter
\"Domains\")  can be   applied to Green  classes.   This  is described in
section  "Set  Functions for   Green  Classes".  The  following  sections
describe how Green classes can be constructed.

%%%%%%%%%%%%%%%%%%%%%%%%%%%%%%%%%%%%%%%%%%%%%%%%%%%%%%%%%%%%%%%%%%%%%%%%%
\Section{RClass}

'RClass( <M>, <s> )'

'RClass' returns the R class of the element <s> in the monoid <M>.

|    gap> M:= Monoid( Transformation( [ 2, 1, 2 ] ), 
    > Transformation( [ 1, 2, 2 ] ) );;
    gap> M.name:= "M";;
    gap> RClass( M, Transformation( [ 1, 2, 2 ] ) );
    RClass( M, Transformation( [ 1, 2, 2 ] ) ) |

The  R class of  <s>  in <M> is  the  set of  all  elements of <M>  which
generate the same  right ideal in <M>, i.e.,  the set of  all $m$  in <M>
with $<s> <M> = m <M>$.

%%%%%%%%%%%%%%%%%%%%%%%%%%%%%%%%%%%%%%%%%%%%%%%%%%%%%%%%%%%%%%%%%%%%%%%%%
\Section{IsRClass}%
\index{test!for R class}

'IsRClass( <obj> )'

'IsRClass' returns 'true' if the object <obj>, which  may be an object of
arbitrary type, is an R class, and 'false' otherwise  (see "RClass").  It
will signal an error if <obj> is an unbound variable.

%%%%%%%%%%%%%%%%%%%%%%%%%%%%%%%%%%%%%%%%%%%%%%%%%%%%%%%%%%%%%%%%%%%%%%%%%
\Section{RClasses}

'RClasses( <M> )'\\
'RClasses( <dClass> )'

'RClasses' returns the list  of R classes the monoid  <M>.  In the second
form 'RClasses' returns the list of R classes in the D class <dClass>.

|    gap> M:= Monoid( Transformation( [ 2, 1, 2 ] ),
    > Transformation( [ 1, 2, 2 ] ) );;
    gap> M.name:= "M";;
    gap> RClasses( M );
    [ RClass( M, Transformation( [ 1, 2, 3 ] ) ), 
      RClass( M, Transformation( [ 2, 1, 2 ] ) ), 
      RClass( M, Transformation( [ 1, 2, 2 ] ) ) ] |

%%%%%%%%%%%%%%%%%%%%%%%%%%%%%%%%%%%%%%%%%%%%%%%%%%%%%%%%%%%%%%%%%%%%%%%%%
\Section{LClass}

'LClass( <M>, <s> )'

'LClass' returns the L class of the element <s> in the monoid <M>.

|    gap> M:= Monoid( Transformation( [ 2, 1, 2 ] ),
    > Transformation( [ 1, 2, 2 ] ) );;
    gap> M.name:= "M";;
    gap> LClass( M, Transformation( [ 1, 2, 2 ] ) );
    LClass( M, Transformation( [ 1, 2, 2 ] ) ) |

The  L class of  <s>  in <M> is  the  set of  all  elements of <M>  which
generate the same  left  ideal in <M>, i.e.,  the set of  all $m$  in <M>
with $<M> <s> = <M> m$.

%%%%%%%%%%%%%%%%%%%%%%%%%%%%%%%%%%%%%%%%%%%%%%%%%%%%%%%%%%%%%%%%%%%%%%%%%
\Section{IsLClass}%
\index{test!for L Class}

'IsLClass( <obj> )'

'IsLClass' returns 'true' if the object <obj>, which  may be an object of
arbitrary type, is an L class, and  'false' otherwise (see "LClass").  It
will signal an error if <obj> is an unbound variable.

%%%%%%%%%%%%%%%%%%%%%%%%%%%%%%%%%%%%%%%%%%%%%%%%%%%%%%%%%%%%%%%%%%%%%%%%%
\Section{LClasses}

'LClasses( <M> )'\\
'LClasses( <dClass> )'

'LClasses' returns the list  of L classes  the monoid <M>.  In the second
form 'LClasses' returns the list of L classes in the D class <dClass>.

|    gap> M:= Monoid( Transformation( [ 2, 1, 2 ] ),
    > Transformation( [ 1, 2, 2 ] ) );;
    gap> M.name:= "M";;
    gap> LClasses( M );
    [ LClass( M, Transformation( [ 1, 2, 3 ] ) ), 
      LClass( M, Transformation( [ 2, 1, 2 ] ) ) ] |

%%%%%%%%%%%%%%%%%%%%%%%%%%%%%%%%%%%%%%%%%%%%%%%%%%%%%%%%%%%%%%%%%%%%%%%%%
\Section{DClass}

'DClass( <M>, <s> )'

'DClass' returns the D class of the element <s> in the monoid <M>.

|    gap> M:= Monoid( Transformation( [ 2, 1, 2 ] ),
    > Transformation( [ 1, 2, 2 ] ) );;
    gap> M.name:= "M";;
    gap> DClass( M, Transformation( [ 1, 2, 2 ] ) );
    DClass( M, Transformation( [ 1, 2, 2 ] ) ) |

The D  class  of <s> in  <M>  is the set   of all elements  of  <M> which
generate the same ideal in <M>, i.e., the set of all $m$ in <M> with $<M>
<s> <M> = <M> m <M>$.

%%%%%%%%%%%%%%%%%%%%%%%%%%%%%%%%%%%%%%%%%%%%%%%%%%%%%%%%%%%%%%%%%%%%%%%%%
\Section{IsDClass}%
\index{test!for D class}

'IsDClass( <obj> )'

'IsDClass' returns 'true' if the object <obj>, which  may be an object of
arbitrary type, is  a D class, and 'false'  otherwise (see "DClass").  It
will signal an error if <obj> is an unbound variable.

%%%%%%%%%%%%%%%%%%%%%%%%%%%%%%%%%%%%%%%%%%%%%%%%%%%%%%%%%%%%%%%%%%%%%%%%%
\Section{DClasses}

'DClasses( <M> )'

'DClasses' returns the list of D classes the monoid <M>.

|    gap> M:= Monoid( Transformation( [ 2, 1, 2 ] ),
    > Transformation( [ 1, 2, 2 ] ) );;
    gap> M.name:= "M";;
    gap> DClasses( M );
    [ DClass( M, Transformation( [ 1, 2, 3 ] ) ), 
      DClass( M, Transformation( [ 2, 1, 2 ] ) ) ] |

%%%%%%%%%%%%%%%%%%%%%%%%%%%%%%%%%%%%%%%%%%%%%%%%%%%%%%%%%%%%%%%%%%%%%%%%%
\Section{HClass}

'HClass( <M>, <s> )'

'HClass' returns the H class of the element <s> in the monoid <M>.

|    gap> M:= Monoid( Transformation( [ 2, 1, 2 ] ),
    > Transformation( [ 1, 2, 2 ] ) );;
    gap> M.name:= "M";;
    gap> HClass( M, Transformation( [ 1, 2, 2 ] ) );
    HClass( M, Transformation( [ 1, 2, 2 ] ) ) |

The H class of <s>  in <M> is the intersection  of the R  class of <s> in
<M> and the L class of <s> in <M> (see "RClass" and "LClass").

%%%%%%%%%%%%%%%%%%%%%%%%%%%%%%%%%%%%%%%%%%%%%%%%%%%%%%%%%%%%%%%%%%%%%%%%%
\Section{IsHClass}%
\index{test!for H class}

'IsHClass( <obj> )'

'IsHClass' returns 'true' if the object <obj>,  which may be an object of
arbitrary type, is an H class, and 'false' otherwise  (see "HClass").  It
will signal an error if <obj> is an unbound variable.

%%%%%%%%%%%%%%%%%%%%%%%%%%%%%%%%%%%%%%%%%%%%%%%%%%%%%%%%%%%%%%%%%%%%%%%%%
\Section{HClasses}

'HClasses( <M> )'\\
'HClasses( <class> )'

'HClasses' returns the list of  H classes the  monoid <M>.  In the second
form 'HClasses'  returns  the list of  all   H classes in  <class>  where
<class> is an R class, an L class or a D class.

|    gap> M:= Monoid( Transformation( [ 2, 1, 2 ] ),
    > Transformation( [ 1, 2, 2 ] ) );;
    gap> M.name:= "M";;
    gap> HClasses( M );
    [ HClass( M, Transformation( [ 1, 2, 3 ] ) ), 
      HClass( M, Transformation( [ 2, 1, 2 ] ) ), 
      HClass( M, Transformation( [ 2, 1, 1 ] ) ) ] |

%%%%%%%%%%%%%%%%%%%%%%%%%%%%%%%%%%%%%%%%%%%%%%%%%%%%%%%%%%%%%%%%%%%%%%%%%
\Section{Set Functions for Green Classes}

Green classes are domains so all set theoretic  functions for domains can
be  applied to them.  Most  of the  set functions  will  work via default
methods once the following methods have been implemented.

'Size( <class> )'%
\index{Size!of a Green class}

determines the size of Green class <class>.

'Elements( <class> )'%
\index{Elements!of a Green class}

returns the set of all elements of the Green class <class>

'<obj> in <class>'%
\index{in!for Green classes}%
\index{membership test!for Green classes}

returns 'true' if  <obj>  is a member  of   the Green class <class>   and
'false' otherwise.

However, no generic methods are provided.

%%%%%%%%%%%%%%%%%%%%%%%%%%%%%%%%%%%%%%%%%%%%%%%%%%%%%%%%%%%%%%%%%%%%%%%%%
\Section{Green Class Records}

A Green class is represented  by a domain  record with the following  tag
components.

'isDomain': \\
        is always 'true'.

'isRClass', 'isLClass', 'isDClass', or 'isHClass': \\
        present and 'true' depending on what kind of Green class is being
        dealt with.

The Green class is determined by the following identity components, which
every Green class record must have.

'monoid': \\
        the monoid.

'representative': \\
        an element of the class.  Which one is unspecified.

In addition to these a Green class record may have the following optional
information components.

'elements': \\
        if present the proper set of elements of the class.

'size': \\
        if present the size of the class.

%%%%%%%%%%%%%%%%%%%%%%%%%%%%%%%%%%%%%%%%%%%%%%%%%%%%%%%%%%%%%%%%%%%%%%%%%
\Section{SchutzenbergerGroup}

'SchutzenbergerGroup( <M>, <s> )'\\
'SchutzenbergerGroup( <class> )'

'SchutzenbergerGroup'  returns  the Sch\accent127utzenberger group of the
element <s> in the monoid <M> as a group.

|    gap> M:= Monoid( Transformation( [ 2, 1, 2 ] ),
    > Transformation( [ 1, 2, 2 ] ) );;
    gap> SchutzenbergerGroup( M, Transformation( [ 2, 1, 2 ] ) );
    Group( (1,2) ) |

In   the     second      form      'SchutzenbergerGroup'   returns    the
Sch\accent127utzenberger group of the Green class <class> of a monoid.

%%%%%%%%%%%%%%%%%%%%%%%%%%%%%%%%%%%%%%%%%%%%%%%%%%%%%%%%%%%%%%%%%%%%%%%%%
\Section{Idempotents}

'Idempotents( <M> )'\\
'Idempotents( <class> )'

returns  the set of  idempotents  in the monoid <M>  or  in a Green class
<class>.

|    gap> M:= Monoid( Transformation( [ 2, 1, 2 ] ),
    > Transformation( [ 1, 2, 2 ] ) );;
    gap> Idempotents( M );
    [ Transformation( [ 1, 2, 1 ] ), Transformation( [ 1, 2, 2 ] ), 
      Transformation( [ 1, 2, 3 ] ) ] 
    gap> Idempotents( DClass( M, Transformation( [ 2, 1, 2 ] ) ) );
    [ Transformation( [ 1, 2, 1 ] ), Transformation( [ 1, 2, 2 ] ) ] |

%%%%%%%%%%%%%%%%%%%%%%%%%%%%%%%%%%%%%%%%%%%%%%%%%%%%%%%%%%%%%%%%%%%%%%%%%
\Section{Monoid Homomorphisms}%
\index{homomorphisms!of monoids}

The homomorphisms between monoids  are of interest  as soon as  there are
monoids.  However, no effort  has been  made to  provide any map  between
monoids.  Here certainly some work needs to be done.

%%%%%%%%%%%%%%%%%%%%%%%%%%%%%%%%%%%%%%%%%%%%%%%%%%%%%%%%%%%%%%%%%%%%%%%%%
\Section{Monoid Records and Semigroup Records}

Like  other domains  semigroups and monoids  are  represented by records. 
While it is possible to construct such a  record by hand it is preferable
to have  the  functions 'SemiGroup'  (see "SemiGroup")  or  'Monoid' (see
"Monoid") do this for you.

After such  a record is  created one can add  record components.  But you
may not alter the values of components which are already present.

A semigroup or monoid record has the following category components.

'isDomain': \\
        is always 'true' since a monoid or a semigroup is a domain.

'isSemiGroup':\\
        is always 'true' for semigroups.

'isMonoid': \\
        is always 'true' for monoids.
        
The following components are the identification components of a semigroup
or monoid record.

'generators': \\
        is   a  list of  generators  of   the monoid  or   the semigroup. 
        Duplicates are allowed in this list, but in  the case of a monoid
        none of the generators may be the identity.

'identity': \\
        is the indentity in the case of a monoid.

Other components which contain information about  the semigroup or monoid
may be present.

'size': \\
        is the size of the monoid or the semigroup (see \"Size\"). 

'elements': \\
        is the  set  of elements  of the  monoid  or the  semigroup  (see 
        \"Elements\").

%%%%%%%%%%%%%%%%%%%%%%%%%%%%%%%%%%%%%%%%%%%%%%%%%%%%%%%%%%%%%%%%%%%%%%%%%%%%%
%%
%E  Emacs . . . . . . . . . . . . . . . . . . . . . . . local emacs variables
%%
%%  Local Variables:
%%  mode:               LaTeX
%%  outline-regexp:     "\\\\Chapter\\|\\\\Section"
%%  fill-column:        73
%%  TeX-master:         "manual"
%%  eval:               (outline-minor-mode)
%%  End:
%%

