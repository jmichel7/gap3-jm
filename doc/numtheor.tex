%%%%%%%%%%%%%%%%%%%%%%%%%%%%%%%%%%%%%%%%%%%%%%%%%%%%%%%%%%%%%%%%%%%%%%%%%%%%%
%%
%A  numtheor.tex                GAP documentation            Martin Schoenert
%%
%A  @(#)$Id: numtheor.tex,v 1.1.1.1 1996/12/11 12:36:48 werner Exp $
%%
%Y  Copyright 1990-1992,  Lehrstuhl D fuer Mathematik,  RWTH Aachen,  Germany
%%
%%  This file describes those functions that mainly deal with prime residues.
%%
%H  $Log: numtheor.tex,v $
%H  Revision 1.1.1.1  1996/12/11 12:36:48  werner
%H  Preparing 3.4.4 for release
%H
%H  Revision 3.8  1993/02/07  13:26:29  felsch
%H  more examples fixed
%H
%H  Revision 3.7  1993/02/01  13:48:06  felsch
%H  examples fixed
%H
%H  Revision 3.6  1992/04/06  16:40:34  martin
%H  fixed some more typos
%H
%H  Revision 3.5  1992/01/24  08:20:03  martin
%H  fixed the description of 'Phi' and 'IsPrimitiveRootMod'
%H
%H  Revision 3.4  1991/12/27  16:07:04  martin
%H  revised everything for GAP 3.1 manual
%H
%H  Revision 3.3  1991/07/26  12:34:01  martin
%H  improved the index
%H
%H  Revision 3.2  1991/07/26  11:18:27  martin
%H  added provisions for the bibliography
%H
%H  Revision 3.1  1991/07/25  16:16:59  martin
%H  fixed some minor typos
%H
%H  Revision 3.0  1991/04/11  11:32:33  martin
%H  Initial revision under RCS
%H
%%
\Chapter{Number Theory}%
\index{prime residue group}

The integers  relatively prime to  $m$ form a  group under multiplication
modulo $m$, called the *prime residue group*.  This chapter describes the
functions that deal with this group.

The  first  section  describes the  function  that  computes  the  set of
representatives of the group (see "PrimeResidues").

The next sections describe  the functions that  compute the  size and the
exponent of the group (see "Phi" and "Lambda").

The next  section  describes the function that computes  the  order of an
element in the group (see "OrderMod").

The  next  section  describes the functions that test  whether  a residue
generates the group or computes a  generator of the group, provided it is
cyclic (see "IsPrimitiveRootMod", "PrimitiveRootMod").

The next section describes  the functions that test whether an element is
a square in the group (see "Jacobi" and "Legendre").

The next  sections  describe the functions that compute general  roots in
the group (see "RootMod" and "RootsUnityMod").

All these functions are in the file 'LIBNAME/\"numtheor.g\"'.

%%%%%%%%%%%%%%%%%%%%%%%%%%%%%%%%%%%%%%%%%%%%%%%%%%%%%%%%%%%%%%%%%%%%%%%%%
\Section{PrimeResidues}%
\index{prime residue group}

'PrimeResidues( <m> )'

'PrimeResidues' returns the set of integers  from the range $0..Abs(m)-1$
that are relatively prime to the integer <m>.

$Abs(m)$ must be less than $2^{28}$, otherwise the set would probably  be
too large anyhow.

The integers  relatively prime to $m$ form  a group  under multiplication
modulo $m$,  called the *prime residue group*.   $\phi(m)$ (see "Phi") is
the order  of this  group, $\lambda(m)$ (see "Lambda")  the exponent.  If
and only if $m$ is 2, 4, an odd prime power $p^e$, or twice an  odd prime
power $2 p^e$, this group is cyclic.  In this case the generators  of the
group, i.e., elements of order $\phi(m)$, are called primitive roots (see
"IsPrimitiveRootMod", "PrimitiveRootMod").

|    gap> PrimeResidues( 0 );
    [  ]
    gap> PrimeResidues( 1 );
    [ 0 ]
    gap> PrimeResidues( 20 );
    [ 1, 3, 7, 9, 11, 13, 17, 19 ] |

%%%%%%%%%%%%%%%%%%%%%%%%%%%%%%%%%%%%%%%%%%%%%%%%%%%%%%%%%%%%%%%%%%%%%%%%%
\Section{Phi}%
\index{Eulers totient function}%
\index{prime residue group!order}%
\index{order!of the prime residue group}

'Phi( <m> )'

'Phi' returns the value of the *Euler totient function* $\phi(m)$ for the
integer <m>.  $\phi(m)$   is defined as the  number  of positive integers
less than or equal to <m> that are relatively prime to <m>.

Suppose that $m = p_1^{e_1} p_2^{e_2} ...  p_k^{e_k}$.  Then $\phi(m)$ is
$p_1^{e_1-1} (p_1-1)  p_2^{e_2-1} (p_2-1)  ...  p_k^{e_k-1} (p_k-1)$.  It
follows that $m$ is a prime if and only if $\phi(m) = m - 1$.

The  integers relatively prime  to $m$ form  a group under multiplication
modulo $m$, called  the *prime residue  group*.  It can be  computed with
'PrimeResidues'  (see "PrimeResidues").  $\phi(m)$  is the order of  this
group, $\lambda(m)$  (see "Lambda") the exponent.  If  and only if $m$ is
2, 4, an odd prime power $p^e$, or twice an odd prime power $2 p^e$, this
group is  cyclic.  In  this  case  the generators   of the   group, i.e.,
elements   of    order $\phi(m)$,   are   called   primitive  roots  (see
"IsPrimitiveRootMod", "PrimitiveRootMod").

'Phi' usually spends most of its time factoring <m> (see "FactorsInt").

|    gap> Phi( 12 );
    4
    gap> Phi( 2^13-1 );
    8190        # which proves that $2^{13}-1$ is a prime
    gap> Phi( 2^15-1 );
    27000 |

%%%%%%%%%%%%%%%%%%%%%%%%%%%%%%%%%%%%%%%%%%%%%%%%%%%%%%%%%%%%%%%%%%%%%%%%%
\Section{Lambda}%
\index{Carmichaels lambda function}%
\index{prime residue group!exponent}%
\index{exponent!of the prime residue group}

'Lambda( <m> )'

'Lambda' returns the  exponent of the group  of relatively prime residues
modulo the integer <m>.

$\lambda(m)$ is the smallest positive integer $l$ such that for every $a$
relatively  prime  to $m$  we have  $a^l=1$  mod  $m$.  Fermat\'s theorem
asserts $a^{\phi(m)}=1$ mod $m$, thus $\lambda(m)$ divides $\phi(m)$ (see
"Phi").

Carmichael\'s theorem states that  $\lambda$ can  be computed as  follows
$\lambda(2)=1$, $\lambda(4)=2$ and $\lambda(2^e) = 2^{e-2}$ if $3 \<= e$,
$\lambda(p^e) = (p-1) p^{e-1}$  ($= \phi(p^e)$) if $p$ is  an  odd prime,
and  $\lambda(n m) = Lcm(\lambda(n),\lambda(m))$ if $n, m$ are relatively
prime.

Composites for which $\lambda(m)$ divides $m - 1$ are called Carmichaels.
If $6k+1$, $12k+1$ and $18k+1$ are primes their product is such a number.
It is believed but unproven that there are infinitely  many  Carmichaels.
There are only  1547  Carmichaels below $10^{10}$ but  455052511  primes.

The  integers  relatively prime to  $m$ form a group under multiplication
modulo $m$, called the *prime residue group*.   It can  be  computed with
'PrimeResidues'  (see   "PrimeResidues").   $\phi(m)$ (see "Phi")  is the
order of this group, $\lambda(m)$ the exponent.  If and only if $m$ is 2,
4, an odd prime  power $p^e$, or  twice an odd prime  power $2 p^e$, this
group is cyclic.   In   this  case   the  generators of the  group, i.e.,
elements   of  order   $\phi(m)$,   are   called  primitive   roots  (see
"IsPrimitiveRootMod", "PrimitiveRootMod").

'Lambda'    usually  spends  most   of    its  time factoring <m>    (see
"FactorsInt").

|    gap> Lambda( 10 );
    4
    gap> Lambda( 30 );
    4
    gap> Lambda( 561 );
    80        # 561 is the smallest Carmichael number |

%%%%%%%%%%%%%%%%%%%%%%%%%%%%%%%%%%%%%%%%%%%%%%%%%%%%%%%%%%%%%%%%%%%%%%%%%
\Section{OrderMod}%
\index{multiplicative order of an integer}

'OrderMod( <n>, <m> )'

'OrderMod' returns the multiplicative order of the integer <n> modulo the
positive integer <m>.   If <n> is  less than 0 or  larger than  <m> it is
replaced by its remainder.  If <n> and  <m>  are not relatively prime the
order of <n> is not defined and 'OrderMod' will return 0.

If  $n$ and  $m$ are  relatively prime  the  multiplicative order of  $n$
modulo $m$ is the smallest positive  integer $i$ such that  $n^i = 1$ mod
$m$.  Elements of maximal order are called primitive roots (see "Phi").

'OrderMod' usually spends  most of its  time  factoring <m> and $\phi(m)$
(see "FactorsInt").

|    gap> OrderMod( 2, 7 );
    3
    gap> OrderMod( 3, 7 );
    6        # 3 is a primitive root modulo 7 |

%%%%%%%%%%%%%%%%%%%%%%%%%%%%%%%%%%%%%%%%%%%%%%%%%%%%%%%%%%%%%%%%%%%%%%%%%
\Section{IsPrimitiveRootMod}%
\index{test!for a primitive root}%
\index{prime residue group!generator}%
\index{generator!of the prime residue group}

'IsPrimitiveRootMod( <r>, <m> )'

'IsPrimitiveRootMod' returns  'true' if  the integer  <r>  is a primitive
root modulo the positive integer <m> and  'false' otherwise.  If  <r>  is
less than 0 or larger than <m> it is replaced by its remainder.

The integers  relatively prime to $m$ form  a group  under multiplication
modulo  $m$, called the  prime  residue  group.  It  can be computed with
'PrimeResidues'  (see   "PrimeResidues").   $\phi(m)$ (see  "Phi") is the
order of this  group, $\lambda(m)$  (see "Lambda") the  exponent.  If and
only if $m$  is  2, 4, an  odd prime power $p^e$, or  twice  an odd prime
power $2 p^e$, this group is cyclic.  In  this case the generators of the
group, i.e., elements of  order $\phi(m)$,  are called  *primitive roots*
(see also "PrimitiveRootMod").

|    gap> IsPrimitiveRootMod( 2, 541 );
    true
    gap> IsPrimitiveRootMod( -539, 541 );
    true        # same computation as above
    gap> IsPrimitiveRootMod( 4, 541 );
    false
    gap> ForAny( [1..29], r -> IsPrimitiveRootMod( r, 30 ) );
    false        # there does not exist a primitive root modulo 30 |

%%%%%%%%%%%%%%%%%%%%%%%%%%%%%%%%%%%%%%%%%%%%%%%%%%%%%%%%%%%%%%%%%%%%%%%%%
\Section{PrimitiveRootMod}%
\index{primitive root modulo an integer}%
\index{prime residue group!generator}%
\index{generator!of the prime residue group}

'PrimitiveRootMod( <m> )' \\
'PrimitiveRootMod( <m>, <start> )'

'PrimitiveRootMod' returns    the  smallest  primitive   root modulo  the
positive integer <m>  and 'false' if  no such primitive  root exists.  If
the optional second integer argument  <start> is given 'PrimitiveRootMod'
returns the smallest primitive root that is strictly larger than <start>.

The integers  relatively prime to  $m$ form a group under  multiplication
modulo $m$,  called the  prime residue group.    It can be computed  with
'PrimeResidues'  (see  "PrimeResidues").  $\phi(m)$ (see   "Phi") is  the
order of this group,  $\lambda(m)$ (see "Lambda")  the exponent.   If and
only  if $m$  is 2, 4,  an odd prime  power $p^e$, or twice an  odd prime
power $2 p^e$, this group is cyclic.  In  this case the generators of the
group, i.e., elements  of  order  $\phi(m)$, are called *primitive roots*
(see also "IsPrimitiveRootMod").

|    gap> PrimitiveRootMod( 409 );
    21        # largest primitive root for a prime less than 2000
    gap> PrimitiveRootMod( 541, 2 );
    10
    gap> PrimitiveRootMod( 337, 327 );
    false        # 327 is the largest primitive root mod 337
    gap> PrimitiveRootMod( 30 );
    false        # the exists no primitive root modulo 30 |

%%%%%%%%%%%%%%%%%%%%%%%%%%%%%%%%%%%%%%%%%%%%%%%%%%%%%%%%%%%%%%%%%%%%%%%%%
\Section{Jacobi}%
\index{quadratic residue}\index{residue!quadratic}

'Jacobi( <n>, <m> )'

'Jacobi'  returns  the value of  the *Jacobi symbol*  of  the integer <n>
modulo the integer <m>.

Suppose that $m = p_1 p_2 .. p_k$ as a product of primes, not necessarily
distinct.   Then for $n$  relatively prime to $m$   the Jacobi  symbol is
defined by $J(n/m) =  L(n/p_1)  L(n/p_2) ..  L(n/p_k)$, where $L(n/p)$ is
the Legendre symbol (see  "Legendre").   By convention $J(n/1)  = 1$.  If
the gcd of $n$ and $m$ is larger than 1 we define $J(n/m) = 0$.

If $n$ is an *quadratic residue* modulo $m$, i.e., if there exists an $r$
such that  $r^2 =  n$ mod  $m$  then $J(n/m)  = 1$.  However $J(n/m) = 1$
implies the existence of such an $r$ only if $m$ is a prime.

'Jacobi' is very efficient, even for large values of <n> and <m>,  it  is
about as fast as the Euclidean algorithm (see "Gcd").

|    gap> Jacobi( 11, 35 );
    1         # $9^2 = 11$ mod $35$
    gap> Jacobi( 6, 35 );
    -1        # thus there is no $r$ such that $r^2 = 6$ mod $35$
    gap> Jacobi( 3, 35 );
    1         # even though there is no $r$ with $r^2 = 3$ mod $35$ |

%%%%%%%%%%%%%%%%%%%%%%%%%%%%%%%%%%%%%%%%%%%%%%%%%%%%%%%%%%%%%%%%%%%%%%%%%
\Section{Legendre}%
\index{quadratic residue}\index{residue!quadratic}

'Legendre( <n>, <m> )'

'Legendre' returns the value of the *Legendre symbol*  of the integer <n>
modulo the positive integer <m>.

The value  of  the Legendre  symbol $L(n/m)$ is 1 if  $n$ is a *quadratic
residue* modulo $m$, i.e., if there exists an  integer $r$ such that $r^2
= n$ mod $m$ and -1 otherwise.

If a root of <n> exists it can be found by 'RootMod' (see "RootMod").

While the value of the Legendre symbol usually  is only defined for <m> a
prime, we have extended the  definition to include composite moduli  too.
The  Jacobi  symbol  (see "Jacobi")  is    another generalization  of the
Legendre symbol for composite moduli that is  much  cheaper  to  compute,
because it does not need the factorization of <m> (see "FactorsInt").

|    gap> Legendre( 5, 11 );
    1         # $4^2 = 5$ mod $11$
    gap> Legendre( 6, 11 );
    -1        # thus there is no $r$ such that $r^2 = 6$ mod $11$
    gap> Legendre( 3, 35 );
    -1        # thus there is no $r$ such that $r^2 = 3$ mod $35$ |

%%%%%%%%%%%%%%%%%%%%%%%%%%%%%%%%%%%%%%%%%%%%%%%%%%%%%%%%%%%%%%%%%%%%%%%%%
\Section{RootMod}%
\index{quadratic residue}\index{residue!quadratic}%
\index{root!of an integer modulo another}

'RootMod( <n>, <m> )' \\
'RootMod( <n>, <k>, <m> )'

In the first  form 'RootMod'  computes a square  root of the  integer <n>
modulo the positive integer <m>, i.e., an integer <r> such that $r^2 = n$
mod <m>.  If no such root exists 'RootMod' returns 'false'.

A root of  <n> exists only if 'Legendre(<n>,<m>)  = 1' (see  "Legendre").
If <m> has  <k> different prime factors  then  there are $2^k$  different
roots of <n> mod  <m>.  It is  unspecified  which  one 'RootMod' returns.
You can, however, use  'RootsUnityMod' (see  "RootsUnityMod")  to compute
the full set of roots.

In the  second form 'RootMod'  computes a <k>th  root of the  integer <n>
modulo the positive integer <m>, i.e., an integer <r> such that $r^k = n$
mod <m>.  If no such root exists 'RootMod' returns 'false'.

In the current implementation <k> must be a prime.

'RootMod' is efficient even for large values of <m>,  actually  most time
is usually spent factoring <m> (see "FactorsInt").

|    gap> RootMod( 64, 1009 );
    1001        # note 'RootMod' does not return 8 in this case but -8
    gap> RootMod( 64, 3, 1009 );
    518
    gap> RootMod( 64, 5, 1009 );
    656
    gap> List( RootMod( 64, 1009 ) * RootsUnityMod( 1009 ),
    >          x -> x mod 1009 );
    [ 1001, 8 ]        # set of all square roots of 64 mod 1009 |

%%%%%%%%%%%%%%%%%%%%%%%%%%%%%%%%%%%%%%%%%%%%%%%%%%%%%%%%%%%%%%%%%%%%%%%%%
\Section{RootsUnityMod}%
\index{modular roots}%
\index{root!of 1 modulo an integer}

'RootsUnityMod( <m> )' \\
'RootsUnityMod( <k>, <m> )'

In the first form 'RootsUnityMod' computes the square  roots of  1 modulo
the integer  <m>, i.e.,  the set of all positive  integers <r> less  than
<n> such that $r^2 = 1$ mod <m>.

In the second form 'RootsUnityMod'  computes the <k>th  roots of 1 modulo
the integer  <m>, i.e., the  set of  all positive integers <r>  less than
<n> such that $r^k = 1$ mod <m>.

In  general  there are  $k^n$ such  roots if  the  modulus  <m>  has  <n>
different prime factors <p> such that $p  = 1$ mod $k$.  If $k^2$ divides
$m$ then there are $k^{n+1}$ such roots; and especially if $k = 2$  and 8
divides $m$ there are $2^{n+2}$ such roots.

If you are interested in the full set of roots  of another number instead
of 1 use 'RootsUnityMod' together with 'RootMod' (see "RootMod").

In the current implementation <k> must be a prime.

'RootsUnityMod' is efficient even for large values  of <m>, actually most
time is usually spent factoring <m> (see "FactorsInt").

|    gap> RootsUnityMod(7*31);
    [ 1, 92, 125, 216 ]
    gap> RootsUnityMod(3,7*31);
    [ 1, 25, 32, 36, 67, 149, 156, 191, 211 ]
    gap> RootsUnityMod(5,7*31);
    [ 1, 8, 64, 78, 190 ]
    gap> List( RootMod( 64, 1009 ) * RootsUnityMod( 1009 ),
    >          x -> x mod 1009 );
    [ 1001, 8 ]        # set of all square roots of 64 mod 1009 |

%%%%%%%%%%%%%%%%%%%%%%%%%%%%%%%%%%%%%%%%%%%%%%%%%%%%%%%%%%%%%%%%%%%%%%%%%
%%
%E  Emacs . . . . . . . . . . . . . . . . . . . . . local Emacs variables
%%
%%  Local Variables:
%%  mode:               outline
%%  outline-regexp:     "\\\\Chapter\\|\\\\Section"
%%  fill-column:        73
%%  eval:               (hide-body)
%%  End:
%%



