%%%%%%%%%%%%%%%%%%%%%%%%%%%%%%%%%%%%%%%%%%%%%%%%%%%%%%%%%%%%%%%%%%%%%%%%%%%%%
%%
%A  manual.tex                  GAP documentation            Martin Schoenert
%%
%Y  Copyright (C)  1995,  Lehrstuhl D fuer Mathematik,  RWTH Aachen,  Germany
%%
%%  This file contains the GAP manual.
%%
%%  To print it, you must format it with \LaTeX\ document preparation system.
%%
%%  This file must be accompanied by the files  in the \Include list below.
%%
%%  It should also be acommpanied  by  'manual.toc'  (the table of contents),
%%  'manual.bbl' (bibliography),  'manual.bib' (source for the bibliography),
%%  'manual.ind' (the index), and 'manual.idx' (the source for 'manual.ind').
%%  If they are missing you can  reproduce  them  again  with {\LaTeX}.  Read
%%  'install.tex' for the necessary steps.
%%
%%  Note that the file 'manual.idx' is also used by the on-screen help.
%%
%H  $Log: manual.tex,v $
%H  Revision 1.11  1997/04/14 10:01:29  gap
%H  Added Bettina as Author
%H
%H  Revision 1.7  1997/03/27 16:37:18  werner
%H  Incorporating the Chevie manual.
%H
%H  Revision 1.6  1997/03/27 13:44:09  gap
%H  Linked in documentation for KBMAG and PCQA share packages
%H
%H  Revision 1.4  1997/03/27 10:51:45  gap
%H  Added documentation for Specht, autag gliss and grim. Started smoothing
%H  and integrating.
%H
%H  Revision 1.2  1997/01/30 14:53:38  werner
%H  Added Michael Smith's automorphism package
%H
%H  Revision 3.40.1.8  1995/12/20  14:36:10  mschoene
%H  added DCE documentation
%H
%H  Revision 3.40.1.6  1995/11/09  08:26:21  fceller
%H  added \R for reals
%H
%H  Revision 3.40.1.4  1995/05/18  03:26:47  mschoene
%H  added Cohomology documentation
%H
%H  Revision 3.40.1.3  1995/05/18  02:24:48  mschoene
%H  added GUAVA documentation
%H
%H  Revision 3.40.1.2  1994/08/31  12:12:58  mschoene
%H  added chapter "Algebraic Extensions"
%H
%H  Revision 3.38  1994/06/03  12:21:20  mschoene
%H  added chapter about special ag groups
%H
%H  Revision 3.35  1994/05/22  12:40:16  fceller
%H  added smash package
%H
%H  Revision 3.32  1993/10/27  10:39:40  sam
%H  added info about 'sisyphos.tex'
%H
%H  Revision 3.30  1993/05/28  13:56:21  gap
%H  added GRAPE
%H
%H  Revision 3.25  1993/01/22  19:29:11  martin
%H  added "anupq" and "tom"
%H
%H  Revision 3.7  1991/07/29  13:40:29  fceller
%H  added chapter about finite polycyclic presentations
%H
%%
\documentclass{book}
\usepackage{amstext,amsfonts,amsbsy,amssymb}
\usepackage{tocloft}

%%%%%%%%%%%%%%%%%%%%%%%%%%%%%%%%%%%%%%%%%%%%%%%%%%%%%%%%%%%%%%%%%%%%%%%%%%%%%
%%
%%  With '\includeonly{<chapters>}' you can specify that you want  only  some
%%  chapters to be printed.  If all  the  '.aux'  files  are  there, chapter-
%%  section- and page-numbers will all be correct.
%%
%\includeonly{}

%%%%%%%%%%%%%%%%%%%%%%%%%%%%%%%%%%%%%%%%%%%%%%%%%%%%%%%%%%%%%%%%%%%%%%%%%%%%%
%%
%%  The following command instructs {\LaTeX} to stuff more on each  page  and
%%  to move each page towards to outer border.
%%
\topmargin 0 pt
\textheight 47\baselineskip
\advance\textheight by \topskip
\oddsidemargin  0.5 in
\evensidemargin  .25in
\textwidth 5.5in

%%%%%%%%%%%%%%%%%%%%%%%%%%%%%%%%%%%%%%%%%%%%%%%%%%%%%%%%%%%%%%%%%%%%%%%%%%%%%
%%
%%  The following commands instruct  {\LaTeX}  to  separate the paragraphs in
%%  this manual with a small space and to leave them unindented.
%%
\parskip 1.0ex plus 0.5ex minus 0.5ex
\parindent 0pt

%%%%%%%%%%%%%%%%%%%%%%%%%%%%%%%%%%%%%%%%%%%%%%%%%%%%%%%%%%%%%%%%%%%%%%%%%%%%%
%%
%%  'text'
%%
%%  'text' prints the text in  monospaced  typewriter  font  in  the  printed
%%  manual  and  is  displayed  unchanged  in  the  on-screen  documentation.
%%  It should be used for names of GAP variables and functions and other text
%%  that the user may actually enter into his computer and see on his screen.
%%  The text may contain all the usual characters  and |<name>| placeholders.
%%  |\'| can be used to enter a single  quote  character  into  the  text.
%%
\catcode`\'=13 \gdef'#1'{{\tt #1}}
\gdef\'{\char`'}

%%%%%%%%%%%%%%%%%%%%%%%%%%%%%%%%%%%%%%%%%%%%%%%%%%%%%%%%%%%%%%%%%%%%%%%%%%%%%
%%
%%  <text>
%%
%%  <text> prints  the text in  an italics font  in the printed manual and is
%%  displayed  unchanged in the  on-screen  documentation.  It should be used
%%  for arguments  in description of  GAP functions and   other placeholders.
%%  The  text should not contain any special characters.  |\<| can be used to
%%  enter a less than character into the text.
%%
\catcode`\<=13 \gdef<#1>{{\it #1\/}}
\gdef\<{\char`<}

%%%%%%%%%%%%%%%%%%%%%%%%%%%%%%%%%%%%%%%%%%%%%%%%%%%%%%%%%%%%%%%%%%%%%%%%%%%%%
%%
%%  *text*
%%
%%  *text*  prints the text  in boldface font   in the printed manual and  is
%%  displayed unchanged in the on-screen  documentation.  It  should  be used
%%  for the definition  of mathematical  keywords.   The text may contain all
%%  the usual characters.  |\*| can be used to enter a star into the text.
%%
\catcode`\*=13 \gdef*#1*{{\bf #1}}
\gdef\*{\char`*}
\gdef\^{\char`^}

%%%%%%%%%%%%%%%%%%%%%%%%%%%%%%%%%%%%%%%%%%%%%%%%%%%%%%%%%%%%%%%%%%%%%%%%%%%%%
%%
%%  |text|
%%
%%  |text| prints the text between the two  pipe  symbols in typewriter style
%%  obeying the   linebreaks and spaces  in  the   manual.   In the on-screen
%%  documentation it  remains unchanged, except  that the pipes are stripped.
%%  It should be used to  enter lengthy examples  into the text.  If the hash
%%  character '\#' appears in the example the text between it and  the end of
%%  the line  is set in  ordinary mode,  i.e., in  roman   font with  all the
%%  possibilities ordinary available.  |\|'\|' can be used to  enter  a  pipe
%%  symbol into the text.
%%
\catcode`\@=11

{\catcode`\ =\active\gdef\xvobeyspaces{\catcode`\ \active\let \xobeysp}}
\def\xobeysp{\leavevmode{} }

\catcode`\|=13
\gdef|{\leavevmode{}\hbox{}\begingroup
\def|{\endgroup}%
\catcode`\\=12\catcode`\{=12\catcode`\}=12\catcode`\$=12\catcode`\&=13
\catcode`\#=13\catcode`\^=12\catcode`\_=12\catcode`\ =12\catcode`\%=12
\catcode`\~=12\catcode`\'=12\catcode`\<=12\catcode`\"=12\catcode`\|=13
\catcode`\*=12\catcode`\:=12
\leftskip\@totalleftmargin\rightskip\z@
\parindent\z@\parfillskip\@flushglue\parskip\z@
\@tempswafalse\def\par{\if@tempswa\hbox{}\fi\@tempswatrue\@@par}%
\tt\obeylines\frenchspacing\xvobeyspaces\samepage}

\catcode`\@=12

{\catcode`\#=13
\gdef#{\begingroup
\catcode`\\=0 \catcode`\{=1 \catcode`\}=2 \catcode`\$=3 \catcode`\&=4
\catcode`\#=6 \catcode`\^=7 \catcode`\_=8 \catcode`\ =10\catcode`\%=14
\catcode`\~=13\catcode`\'=13\catcode`\<=13\catcode`\"=13\catcode`\|=13
\catcode`\*=13\catcode`\:=13
\catcode`\^^M=12 \Comment}}
{\catcode`\^^M=12
\gdef\Comment#1^^M{\rm \# #1 \endgroup \Newline}}
{\obeylines
\gdef\Newline{
}}

{\catcode`\&=13
\gdef&{\#}}

\gdef\|{\char`|}

%%%%%%%%%%%%%%%%%%%%%%%%%%%%%%%%%%%%%%%%%%%%%%%%%%%%%%%%%%%%%%%%%%%%%%%%%%%%%
%%
%%  <item>: <text>
%%
%%  This formats the  paragraph  <text>, i.e.,  everything between  the colon
%%  '\:' and the next  empty  line, indented 1 cm to the right in the printed
%%  manual and is displayed  unchanged in the  on-screen documentation.  This
%%  convention should be  used to format  a list or an enumeration.    <item>
%%  should be  a single  word  or a short phrase.  It  may contain  all usual
%%  characters and the usual formatting stuff.  <text> is a  normal paragraph
%%  and may contain everything.   \:  can be used  to enter a colon character
%%  into the text.  As  an example consider the  following description.  This
%%  will print quite similar in the printed manual.
%%
%%      A group is represented by a record that must have the components
%%
%%      'generators': \\
%%              a list of group elements that  generate  the  group  that  is
%%              given by the group record.
%%
%%      'identity': \\
%%              the identity element of the group that is given by the  group
%%              record.
%%
\catcode`\:=13
\gdef:{\hangafter=1\hangindent=1cm\hspace{1cm}{}}
\gdef\:{\char`:}

%%%%%%%%%%%%%%%%%%%%%%%%%%%%%%%%%%%%%%%%%%%%%%%%%%%%%%%%%%%%%%%%%%%%%%%%%%%%%
%%
%%  "reference"
%%
%%  "reference" prints the  number of the  chapter or section  in the printed
%%  manual and is  displayed unchanged  in the  on-screen  documentation.  It
%%  should be used when referring to other  chapters or  sections.   The text
%%  should  not contain any special characters.  \"  can be  used  to enter a
%%  double quote into the text.
%%
\catcode`\"=13 \gdef"#1"{\ref{#1}}

\gdef\"{\char`"}

%%%%%%%%%%%%%%%%%%%%%%%%%%%%%%%%%%%%%%%%%%%%%%%%%%%%%%%%%%%%%%%%%%%%%%%%%%%%%
%%
%%  \GAP
%%
%%  \GAP can be used to enter the *sans serif* GAP logo  into  the  text.  If
%%  this is followed by spaces it should be enclosed in curly  braces  as  in
%%  |{\GAP}| is wonderful.
%%
\newcommand\GAP{{\sf GAP3}}
\newcommand\CAS{{\sf CAS}}
\newcommand\ATLAS{{\sf ATLAS}}
\newcommand\CHEVIE{{\sf CHEVIE}}
\newcommand\MAPLE{{\sf MAPLE}}
\newcommand\C{{\mathbb C}}
\newcommand\F{{\mathbb F}}
\newcommand\Q{{\mathbb Q}}
\newcommand\R{{\mathbb R}}
\newcommand\Z{{\mathbb Z}}

%%%%%%%%%%%%%%%%%%%%%%%%%%%%%%%%%%%%%%%%%%%%%%%%%%%%%%%%%%%%%%%%%%%%%%%%%%%%%
%%
%%  \Include{<filename>}
%%
%%  |\Include| instructs \LaTeX\ to include the file with the name <filename>
%%  into a document.  It works basically like |\include| except that it  also
%%  writes a comment line to the index file where the on-screen help function
%%  may use it.
%%
\catcode`\@=11 \catcode`\%=12 \catcode`\~=14
\newcommand{\Include}[1]{\write\@indexfile{% #1.tex}\include{#1}}
\catcode`\@=12 \catcode`\%=14 \catcode`\~=13

%%%%%%%%%%%%%%%%%%%%%%%%%%%%%%%%%%%%%%%%%%%%%%%%%%%%%%%%%%%%%%%%%%%%%%%%%%%%%
%%
%%  \Chapter{<name>}
%%  \Section{<name>}
%%
%%  |\Chapter| and |\Section| begin a  new  chapter  or  section.  They  work
%%  basically like the ordinary |\chapter| and |\section| macros except  that
%%  they also create a label for <name>   and write a  comment  line  to  the
%%  index file where the on-screen help function may use it.
%%
\catcode`\@=11 \catcode`\%=12 \catcode`\~=14

\newcommand{\Chapter}[1]{{\chapter{#1}~
\label{#1}~
\write\@indexfile{% 1}\index{#1}}}

\newcommand{\Section}[1]{{\pagebreak[3]\section{#1}~
\label{#1}~
\write\@indexfile{% 2}\index{#1}}}

\catcode`\@=12 \catcode`\%=14 \catcode`\~=13

%%%%%%%%%%%%%%%%%%%%%%%%%%%%%%%%%%%%%%%%%%%%%%%%%%%%%%%%%%%%%%%%%%%%%%%%%%%%%
%%
%%  tell LaTeX to prepare an index
%%
\makeindex

%%%%%%%%%%%%%%%%%%%%%%%%%%%%%%%%%%%%%%%%%%%%%%%%%%%%%%%%%%%%%%%%%%%%%%%%%%%%%
%%
%%  make the title page
%%
\begin{document}

\title{GAP \\
Groups, Algorithms and Programming \\
version 3.4.4 distribution gap3-jm}
\author{
Martin Sch{\accent127 o}nert \\
with\\
Hans Ulrich Besche, Thomas Breuer, Frank Celler, Bettina Eick\\
Volkmar Felsch, Alexander Hulpke, J{\accent127 u}rgen Mnich, Werner Nickel \\
G{\accent127 o}tz Pfeiffer, Udo Polis, Heiko Thei\ss en \\
Lehrstuhl D f{\accent127 u}r Mathematik, RWTH Aachen \\
Alice Niemeyer \\
Department of Mathematics, University of Western Australia\\
\\
Jean Michel\\
Department of Mathematics, University Paris-Diderot, France
}
\date{GAP 3.4.4 of 20 Dec. 1995, gap3-jm of 19 Feb. 2018}

\maketitle

%%%%%%%%%%%%%%%%%%%%%%%%%%%%%%%%%%%%%%%%%%%%%%%%%%%%%%%%%%%%%%%%%%%%%%%%%%%%%
%%
%%  include the copyright notice and the preface
%%
%%%%%%%%%%%%%%%%%%%%%%%%%%%%%%%%%%%%%%%%%%%%%%%%%%%%%%%%%%%%%%%%%%%%%%%%%%%%%
%%
%A  copyrigh.tex                GAP documentation            Martin Schoenert
%%
%A  @(#)$Id: copyrigh.tex,v 1.1.1.1 1996/12/11 12:36:44 werner Exp $
%%
%Y  Copyright 1990-1992,  Lehrstuhl D fuer Mathematik,  RWTH Aachen,  Germany
%%
%%  This file contains the GAP copyright
%%
%H  $Log: copyrigh.tex,v $
%H  Revision 1.1.1.1  1996/12/11 12:36:44  werner
%H  Preparing 3.4.4 for release
%H
%H  Revision 3.3  1992/04/05  23:41:40  martin
%H  made a minor modification
%H
%H  Revision 3.2  1992/03/31  12:50:36  martin
%H  fixed several typos
%H
%H  Revision 3.1  1992/03/31  08:22:32  martin
%H  initial revision under RCS
%H
%%
\thispagestyle{empty}

{\large Copyright {\copyright} 1992 by Lehrstuhl D f{\accent127u}r Mathematik}

RWTH, Templergraben 64, D 5100 Aachen, Germany

{\GAP}  can  be  copied  and  distributed  freely for any  non-commercial
purpose.

If you copy {\GAP} for somebody else, you may ask this person for  refund
of your expenses.  This should cover cost of media, copying and shipping.
You are not allowed to ask for more than this.  In any case you must give
a copy of this copyright notice along with the program.

If you obtain {\GAP} please send us  a short notice to that effect, e.g.,
an  e-mail  message   to  the  address  'gap@samson.math.rwth-aachen.de',
containing your full  name and address.  This  allows us to keep track of
the number of {\GAP} users.

If you  publish  a mathematical result  that  was  partly obtained  using
{\GAP}, please cite {\GAP}, just as you would cite another paper that you
used.   Also we  would appreciate it if you could inform us about  such a
paper.

You  are permitted  to modify and  redistribute  {\GAP},  but you are not
allowed  to restrict further redistribution.  That is to say  proprietary
modifications will  not  be allowed.  We want all  versions  of {\GAP} to
remain free.

If you  modify any part of {\GAP} and redistribute it,  you must supply a
'README'  document.   This should specify what modifications you made  in
which  files.  We do  not  want  to take  credit  or  be blamed  for your
modifications.

Of course we are interested in all of your modifications.  In  particular
we would like to see bug-fixes, improvements and new functions.  So again
we would appreciate it if you would inform us about all modifications you
make.

{\GAP} is distributed by us without any warranty, to the extent permitted
by applicable state law.  We  distribute {\GAP} *as is* without  warranty
of any kind, either expressed or implied, including,  but not limited to,
the implied  warranties  of merchantability and  fitness for a particular
purpose.

The entire risk as to the quality and performance of the program is  with
you.  Should {\GAP} prove defective, you assume the cost of all necessary
servicing, repair or correction.

In no  case  unless  required by applicable law will we, and/or any other
party who  may  modify  and  redistribute  {\GAP}  as permitted above, be
liable  to you for damages, including lost profits, lost monies or  other
special, incidental or consequential damages  arising out  of the  use or
inability to use {\GAP}.

\include{preface}

%%%%%%%%%%%%%%%%%%%%%%%%%%%%%%%%%%%%%%%%%%%%%%%%%%%%%%%%%%%%%%%%%%%%%%%%%%%%%
%%
%%  make the table of chapters and the table of contents
%%
%%  To make the table of chapters we read the  table  of  contents  file  and
%%  make sure that section lines are ignored.  We also must  make  sure  that
%%  table  of contents  file is  not cleared, because  we want  to read  it a
%%  second time for the full table of contents.
%%
%%  In the full table of contents we have to make a little more room for  the
%%  section numbers, because we have so many sections in some chapters.
%%
\setlength{\cftchapnumwidth}{2em}
\newcommand{\ignoretwoarguments}[2]{}
\catcode`\@=11
\def\l@section{\ignoretwoarguments}
\@fileswfalse
\catcode`\@=12
\tableofcontents
\catcode`\@=11
\@fileswtrue
\catcode`\@=12

\catcode`\@=11
\def\@pnumwidth{2em}
\def\l@section{\@dottedtocline{1}{1.5em}{4em}}
\catcode`\@=12

\tableofcontents

%%%%%%%%%%%%%%%%%%%%%%%%%%%%%%%%%%%%%%%%%%%%%%%%%%%%%%%%%%%%%%%%%%%%%%%%%%%%%
%%
%%  and now the chapters
%%
\Include{aboutgap} %01
\Include{language} %02
\Include{environm} %03
\Include{domain}   %04
\Include{ring}     %05
\Include{field}    %06
\Include{group}    %07
\Include{operatio} %08
\Include{vecspace} %09
\Include{integer}  %10
\Include{numtheor} %11
\Include{rational} %12
\Include{cyclotom} %13
\Include{gaussian} %14
\Include{numfield} %15
\Include{algext}   %16
\Include{unknown}  %17
\Include{finfield} %18
\Include{polynom}  %19
\Include{permutat} %20
\Include{permgrp}  %21
\Include{word}     %22
\Include{fpgrp}    %23
\Include{agwords}  %24
\Include{aggroup}  %25
\Include{saggroup} %26
\Include{list}     %27
\Include{set}      %28
\Include{blister}  %29
\Include{string}   %30
\Include{range}    %31
\Include{vector}   %32
\Include{rowspace} %33
\Include{matrix}   %34
\Include{matint}   %30
\Include{matring}  %35
\Include{matgrp}   %36
\Include{grplib}   %37
\Include{algebra}  %38
\Include{algfp}    %39
\Include{algmat}   %40
\Include{module}   %41
\Include{mapping}  %42
\Include{homomorp} %43
\Include{boolean}  %44
\Include{record}   %45
\Include{combinat} %46
\Include{tom}      %47
\Include{chartabl} %48
\Include{gentable} %49
\Include{characte} %50
\Include{paramaps} %51
\Include{gettable} %52
\Include{classfun} %53
\Include{monomial} %54
\Include{install}  %55
\Include{share}    %56
\Include{anupq}    % anupq package
\Include{autag}    % autag package
\Include{cohomolo} % cohomolo package
\Include{cryst}    % cryst package
\Include{dce}      % dce package
\Include{gliss}    % gliss package
\Include{grape}    % grape package
\Include{grim}     % grim package
\Include{guava}    % guava package
\Include{kbmag}    % kbmag package
\Include{matpkg}   % matrix package
\Include{mtx}      % meataxe package
\Include{pcqa}     % pcqa package
\Include{sisyphos} % sisyphos package
\Include{specht}   % specht package
\Include{ve}       % ve package
\Include{arep}     % arep package
\Include{monoid}   %1 % monoid package
\Include{relation} %2
\Include{transfor} %3
\Include{monotran} %4
\Include{action}   %5 % end of monoid package
\Include{xmod}     % xmod package
\Include{chvintr}  %1 % chevie package
\Include{chvrefg}  %2
\Include{chvabsc}  %3
\Include{chvfref}  %4
\Include{chvcoxe}  %5
\Include{chvalgg}  %6
\Include{chvirr}   %7
\Include{chvrefl}  %8
\Include{chvbrai}  %9
\Include{chvchek}  %10
\Include{chvheck}  %11
\Include{chvhrep}  %12
\Include{chvkalu}  %13
\Include{chvhmod}  %14
\Include{chvrcos}  %15
\Include{chvccos}  %16
\Include{chvhcos}  %17
\Include{chvunic}  %18
\Include{chveig}   %19
\Include{chvucl}   %20
\Include{chvunir}  %21
\Include{chvaffn}  %22
\Include{chvutil}  %23
\Include{chvfmt}   %24
\Include{chvmat}   %25
\Include{chvcyc}   %26
\Include{chvsymb}  %27
\Include{chvsperm} %28
\Include{chvdec}   %29
\Include{chvposet} %30 % end of chevie package
\Include{curvintr} %1 % vkcurve package
\Include{curvmvp}  %2
\Include{curvalg}  %3
\Include{curvutil} %4 % end of vkcurve package
\Include{falgebra} % algebra package

%%%%%%%%%%%%%%%%%%%%%%%%%%%%%%%%%%%%%%%%%%%%%%%%%%%%%%%%%%%%%%%%%%%%%%%%%%%%%
%%
%%  the bibliography
%%
\begin{sloppypar}
\bibliographystyle{alpha}
\newcommand{\ignore}[1]{}
\catcode`\'=12 \catcode`\<=12 \catcode`\*=12
\catcode`\|=12 \catcode`\:=12 \catcode`\"=12
\bibliography{manual}
\end{sloppypar}
%%%%%%%%%%%%%%%%%%%%%%%%%%%%%%%%%%%%%%%%%%%%%%%%%%%%%%%%%%%%%%%%%%%%%%%%%%%%%
%%
%%  finally include the index
%%
\begin{sloppypar}
\raggedright
\begin{theindex}
\catcode`\@=11
\@input{\jobname.ind}
\catcode`\@=12
\end{theindex}
\end{sloppypar}
\end{document}
