%%%%%%%%%%%%%%%%%%%%%%%%%%%%%%%%%%%%%%%%%%%%%%%%%%%%%%%%%%%%%%%%%%%%%%%%%%%%%
%%
%A  matgrp.tex                  GAP documentation            Martin Schoenert
%%
%A  @(#)$Id: matgrp.tex,v 1.3 1997/04/16 09:38:28 gap Exp $
%%
%Y  Copyright 1990-1992,  Lehrstuhl D fuer Mathematik,  RWTH Aachen,  Germany
%%
%%  This file describes matrix group and their implementation.
%%
%H  $Log: matgrp.tex,v $
%H  Revision 1.3  1997/04/16 09:38:28  gap
%H  Typo
%H
%H  Revision 1.2  1997/04/14 11:21:27  gap
%H  hanged description of perm action
%H
%H  Revision 1.1.1.1  1996/12/11 12:36:47  werner
%H  Preparing 3.4.4 for release
%H
%H  Revision 3.5  1993/03/11  17:56:24  fceller
%H  added matrix group record
%H
%H  Revision 3.4  1993/02/19  10:48:42  gap
%H  adjustments in line length and spelling
%H
%H  Revision 3.3  1993/02/12  16:44:35  felsch
%H  examples adjusted to line length 72
%H
%H  Revision 3.2  1993/02/05  10:51:35  felsch
%H  example fixed
%H
%H  Revision 3.1  1992/04/04  15:56:52  martin
%H  initial revision under RCS
%H
%%
\Chapter{Matrix Groups}

A *matrix group* is  a group of invertable square  matrices (see  chapter
"Matrices").   In {\GAP}  you can define matrix  groups of matrices  over
each of the fields that {\GAP} supports, i.e., the  rationals, cyclotomic
extensions of the rationals, and finite fields (see chapters "Rationals",
"Cyclotomics", and "Finite Fields").

You  define a matrix  group in {\GAP} by calling  'Group'  (see  "Group")
passing the generating matrices as arguments.

|    gap> m1 := [ [ Z(3)^0, Z(3)^0,   Z(3) ],
    >            [   Z(3), 0*Z(3),   Z(3) ],
    >            [ 0*Z(3),   Z(3), 0*Z(3) ] ];;
    gap> m2 := [ [   Z(3),   Z(3), Z(3)^0 ],
    >            [   Z(3), 0*Z(3),   Z(3) ],
    >            [ Z(3)^0, 0*Z(3),   Z(3) ] ];;
    gap> m := Group( m1, m2 );
    Group( [ [ Z(3)^0, Z(3)^0, Z(3) ], [ Z(3), 0*Z(3), Z(3) ],
      [ 0*Z(3), Z(3), 0*Z(3) ] ],
    [ [ Z(3), Z(3), Z(3)^0 ], [ Z(3), 0*Z(3), Z(3) ],
      [ Z(3)^0, 0*Z(3), Z(3) ] ] )|

However,  currently {\GAP} can only  compute  with finite  matrix groups.
Also computations with large matrix groups are not done very efficiently.
We hope to improve this situation in the future, but currently you should
be careful not to try too large matrix groups.

Because  matrix  groups are just a special  case of  domains  all the set
theoretic functions such as 'Size' and  'Intersection'  are applicable to
matrix  groups  (see chapter  "Domains"  and  "Set  Functions for  Matrix
Groups").

Also matrix groups are  of course groups, so all the group functions such
as 'Centralizer' and 'DerivedSeries' are applicable to matrix groups (see
chapter "Groups" and "Group Functions for Matrix Groups").

%%%%%%%%%%%%%%%%%%%%%%%%%%%%%%%%%%%%%%%%%%%%%%%%%%%%%%%%%%%%%%%%%%%%%%%%%
\Section{Set Functions for Matrix Groups}
\index{in!for matrix groups}
\index{Size!for matrix groups}
\index{Intersection!for matrix groups}
\index{Random!for matrix groups}

As already  mentioned in the introduction of  this chapter matrix  groups
are   domains.    All   set  theoretic  functions  such   as  'Size'  and
'Intersections'  are  thus  applicable  to  matrix groups.   This section
describes  how  these  functions  are  implemented  for  matrix   groups.
Functions  not  mentioned here either  inherit the default  group methods
described  in "Set Functions for Groups" or the default  method mentioned
in the respective sections.

To compute with a matrix group <m>, {\GAP} computes  the operation of the
matrix group on the  underlying vector space (more precisely the union of
the orbits of the parent of <m>  on the standard basis vectors).  Then it
works with the  thus  defined permutation group  <p>,  which is of course
isomorphic  to  <m>, and  finally  translates  the  results back into the
matrix group.

\vspace{5mm}
'<obj> in <m>'

To test if an object <obj> lies in a matrix group <m>, {\GAP} first tests
whether <obj> is a invertable square matrix of the same dimensions as the
matrices of  <m>.   If  it is, {\GAP} tests  whether  <obj> permutes  the
vectors in  the union of the orbits of <m> on the standard basis vectors.
If it does,  {\GAP}  takes this permutation and tests whether it lies  in
<p>.

\vspace{5mm}
'Size( <m> )'

To compute the size of the matrix group <m>, {\GAP} computes the size  of
the isomorphic permutation group <p>.

\vspace{5mm}
'Intersection( <m1>, <m2> )'

To compute the intersection of two subgroups  <m1> and <m2> with a common
parent  matrix   group  <m>,   {\GAP}   intersects  the  images  of   the
corresponding  permutation subgroups  <p1>  and <p2>  of  <p>.   Then  it
translates  the  generators  of   the  intersection  of  the  permutation
subgroups  back to matrices.  The  intersection of  <m1> and  <m2> is the
subgroup  of  <m> generated  by those matrices.   If <m1> and <m2> do not
have a common parent group, or if only one  of them is a matrix group and
the  other  is a  set of  matrices,  the  default  method  is  used  (see
"Intersection").

%%%%%%%%%%%%%%%%%%%%%%%%%%%%%%%%%%%%%%%%%%%%%%%%%%%%%%%%%%%%%%%%%%%%%%%%%
\Section{Group Functions for Matrix Groups}
\index{Centralizer!for matrix groups}
\index{Normalizer!for matrix groups}
\index{SylowSubgroup!for matrix groups}
\index{ConjugacyClasses!for matrix groups}
\index{PermGroup!for matrix groups}
\index{Stabilizer!for matrix groups}
\index{RepresentativeOperation!for matrix groups}

As already mentioned  in  the introduction of this  chapter matrix groups
are  after  all group.  All group functions  such  as  'Centralizer'  and
'DerivedSeries'  are  thus  applicable  to  matrix groups.  This  section
describes  how  these  functions  are  implemented  for   matrix  groups.
Functions not mentioned here  either inherit  the  default  group methods
described in the respective sections.

To compute with a  matrix group <m>, {\GAP} computes the operation of the
matrix  group on the underlying vector space (more precisely, if the vector
space is small enough, it enumerates the space and acts on the whole space.
Otherwise it takes the union of
the orbits of the parent of <m> on the standard basis vectors).   Then it
works  with  the thus defined permutation  group <p>, which is of  course
isomorphic to  <m>, and  finally  translates the results  back  into  the
matrix group.

\vspace{5mm}
'Centralizer( <m>, <u> )' \\
'Normalizer( <m>, <u> )' \\
'SylowSubgroup( <m>, <p> )' \\
'ConjugacyClasses( <m> )'

This functions  all work by solving the  problem in the permutation group
<p> and translating the result back.

\vspace{5mm}
'PermGroup( <m> )'

This function simply returns the permutation group defined above.

\vspace{5mm}
'Stabilizer( <m>, <v> )'

The  stabilizer of  a vector <v> that lies in the  union of the orbits of
the parent of <m> on  the standard basis  vectors  is computed by finding
the stabilizer  of the corresponding point in  the permutation  group <p>
and  translating  this back.   Other stabilizers are  computed  with  the
default method (see "Stabilizer").

\vspace{5mm}
'RepresentativeOperation( <m>, <v1>, <v2> )'

If <v1> and <v2> are vectors that both lie in the union of the  orbits of
the   parent   group   of   <m>    on   the   standard   basis   vectors,
'RepresentativeOperation' finds a permutation in <p> that takes the point
corresponding  to <v1> to  the point corresponding to  <v2>.  If no  such
permutation  exists, it returns 'false'.   Otherwise  it  translates  the
permutation back to a matrix.

\vspace{5mm}
'RepresentativeOperation( <m>, <m1>, <m2> )'

If <m1> and <m2> are  matrices in <m>,  'RepresentativeOperation' finds a
permutation in <p> that conjugates the permutation corresponding to  <m1>
to the permutation corresponding to <m2>.  If no such permutation exists,
it returns 'false'.   Otherwise it  translates the permutation back to  a
matrix.

%%%%%%%%%%%%%%%%%%%%%%%%%%%%%%%%%%%%%%%%%%%%%%%%%%%%%%%%%%%%%%%%%%%%%%%%%
\Section{Matrix Group Records}

A group is  represented by  a record that  contains information about the
group.   A matrix  group  record  contains the  following  components  in
addition to those described in section "Group Records".

'isMatGroup': \\
    always 'true'.

If a  permutation representation  for a matrix group  <m> is known it  is
stored in the following components.

'permGroupP': \\
    contains the permutation group representation of <m>.

'permDomain': \\
    contains the union of the orbits of the parent of <m> on the standard
    basis vectors.

%%%%%%%%%%%%%%%%%%%%%%%%%%%%%%%%%%%%%%%%%%%%%%%%%%%%%%%%%%%%%%%%%%%%%%%%%
%%
%E  Emacs . . . . . . . . . . . . . . . . . . . . . local Emacs variables
%%
%%  Local Variables:
%%  mode:               outline
%%  outline-regexp:     "\\\\Chapter\\|\\\\Section"
%%  fill-column:        73
%%  eval:               (hide-body)
%%  End:
%%



