%%%%%%%%%%%%%%%%%%%%%%%%%%%%%%%%%%%%%%%%%%%%%%%%%%%%%%%%%%%%%%%%%%%%%%%%%%%%
%%
%A  chvmat.tex       CHEVIE documentation                      Jean Michel
%%
%Y  Copyright (C) 1996-2001  University  Paris VII.
%%
%%  This  file  documents matrix utility functions
%%%%%%%%%%%%%%%%%%%%%%%%%%%%%%%%%%%%%%%%%%%%%%%%%%%%%%%%%%%%%%%%%%%%%%%%%%%%

\Chapter{CHEVIE Matrix utility functions}

This  chapter documents various functions  which enhance \GAP\'s ability
to work with matrices.

%%%%%%%%%%%%%%%%%%%%%%%%%%%%%%%%%%%%%%%%%%%%%%%%%%%%%%%%%%%%%%%%%%%%%%%%%
\Section{DecomposedMat}%
\index{DecomposedMat}%
\index{Matrices!block decomposition of square}

'DecomposedMat( <mat> )'

Finds  if the  square matrix  <mat> with  zeroes (or  'false') in symmetric
positions admits a block decomposition.

Define  a  graph  <G>  with  vertices  '[1..Length(mat)]'  and with an edge
between  'i'  and  'j'  if  either  'mat[i][j]' or 'mat[j][i]' is non-zero.
'DecomposedMat' return a list of lists 'l' such that 'l[1],l[2]', etc.. are
the  vertices  in  each  connected  component  of  <G>. In other words, the
matrices  'mat\{l[1]\}\{l[1]\},mat\{l[2]\}\{l[2]\}',  etc...  are blocks of
the  matrix <mat>.  This function  may also  be applied to boolean matrices
where non-zero is then replaced by 'true'.

|    gap> m := [ [  0,  0,  0,  1 ],
    >           [  0,  0,  1,  0 ],
    >           [  0,  1,  0,  0 ],
    >           [  1,  0,  0,  0 ] ];;
    gap> DecomposedMat( m );
    [ [ 1, 4 ], [ 2, 3 ] ]
    gap> PrintArray( m{[ 1, 4 ]}{[ 1, 4 ]});
    [[0, 1],
     [1, 0]]|

%%%%%%%%%%%%%%%%%%%%%%%%%%%%%%%%%%%%%%%%%%%%%%%%%%%%%%%%%%%%%%%%%%%%%%%%%
\Section{BlocksMat}%
\index{BlocksMat}%
\index{Matrices!block decomposition of}

'Blocks( <M> )'

Finds  if the  matrix  <M> admits a block decomposition.

Define    a   bipartite   graph   <G>   with   vertices   '[1..Length(M)]',
'[1..Length(M[1])]'  and with an  edge between 'i'  and 'j' if 'M[i][j]' is
not  zero.  BlocksMat  returns  a  list  of  pairs  of  lists 'I' such that
'[I[1][1],I[1][2]]',  etc.. are the vertices in each connected component of
<G>.  In  other  words, |M{I[1][1]}{I[1][2]}|, |M{I[2][1]}{I[2][2]}|,etc...
are blocks of 'M'. 

This function  may also  be applied to boolean matrices
where non-zero is then replaced by 'true'.

|    gap> m:=[ [ 1, 0, 0, 0 ], [ 0, 1, 0, 0 ], [ 1, 0, 1, 0 ],
    >  [ 0, 0, 0, 1 ], [ 0, 0, 1, 0 ] ];;
    gap> BlocksMat(m);   
    [ [ [ 1, 3, 5 ], [ 1, 3 ] ], [ [ 2 ], [ 2 ] ], [ [ 4 ], [ 4 ] ] ]
    gap> PrintArray(m{[1,3,5]}{[1,3]}); 
    [[1, 0],
     [1, 1],
     [0, 1]]|

%%%%%%%%%%%%%%%%%%%%%%%%%%%%%%%%%%%%%%%%%%%%%%%%%%%%%%%%%%%%%%%%%%%%%%%%%
\Section{RepresentativeDiagonalConjugation}
\index{RepresentativeDiagonalConjugation}

'RepresentativeDiagonalConjugation( <M>, <N> )'

<M>  and <N> must be  square matrices. This function  returns a list <d>
such  that  'N=M\^DiagonalMat(d)'  if  such  a  list  exists,  and  false
otherwise.

|    gap> M:=[[1,2],[2,1]];
    [ [ 1, 2 ], [ 2, 1 ] ]
    gap> N:=[[1,4],[1,1]];
    [ [ 1, 4 ], [ 1, 1 ] ]
    gap> RepresentativeDiagonalConjugation(M,N);
    [ 1, 2 ]|

%%%%%%%%%%%%%%%%%%%%%%%%%%%%%%%%%%%%%%%%%%%%%%%%%%%%%%%%%%%%%%%%%%%%%%%%%
\Section{ProportionalityCoefficient}
\index{ProportionalityCoefficient}

'ProportionalityCoefficient( <v>, <w> )'

<v>  and <w>  should be  two vectors  of the  same length.  The function
returns  a scalar <c>  such that 'v=c\*w'  if such a  scalar exists, and
'false' otherwise.

|    gap> ProportionalityCoefficient([1,2],[2,4]);
    1/2
    gap> ProportionalityCoefficient([1,2],[2,3]);
    false|

%%%%%%%%%%%%%%%%%%%%%%%%%%%%%%%%%%%%%%%%%%%%%%%%%%%%%%%%%%%%%%%%%%%%%%%%%
\Section{ExteriorPower}
\index{ExteriorPower}

'ExteriorPower( <mat>, <n> )'

<mat>  should be  a square  matrix. The  function returns  the <n>-th exterior
power  of <mat>, in  the basis naturally  indexed by |Combinations([1..r],n)|,
where |r=Length(<mat>)|.

|    gap> M:=[[1,2,3,4],[2,3,4,1],[3,4,1,2],[4,1,2,3]];
    [ [ 1, 2, 3, 4 ], [ 2, 3, 4, 1 ], [ 3, 4, 1, 2 ], [ 4, 1, 2, 3 ] ]
    gap> ExteriorPower(M,2);
    [ [ -1, -2, -7, -1, -10, -13 ], [ -2, -8, -10, -10, -12, 2 ],
      [ -7, -10, -13, 1, 2, 1 ], [ -1, -10, 1, -13, 2, 7 ],
      [ -10, -12, 2, 2, 8, 10 ], [ -13, 2, 1, 7, 10, -1 ] ]|

%%%%%%%%%%%%%%%%%%%%%%%%%%%%%%%%%%%%%%%%%%%%%%%%%%%%%%%%%%%%%%%%%%%%%%%%%
\Section{SymmetricPower}
\index{SymmetricPower}

'SymmetricPower( <mat>, <n> )'

<mat>  should be  a square  matrix. The  function returns the <n>-th symmetric
power of <mat>, in the basis naturally indexed by |UnorderedTuples([1..r],n)|,
where |r=Length(<mat>)|.

|    gap> M:=[[1,2],[3,4]];
    [ [ 1, 2 ], [ 3, 4 ] ]
    gap> SymmetricPower(M,2);
    [ [ 1, 2, 4 ], [ 6, 10, 16 ], [ 9, 12, 16 ] ]|

%%%%%%%%%%%%%%%%%%%%%%%%%%%%%%%%%%%%%%%%%%%%%%%%%%%%%%%%%%%%%%%%%%%%%%%%%
\Section{SchurFunctor}
\index{SchurFunctor}

'SchurFunctor(<mat>,<l>)'

<mat>  should be  a square  matrix and  <l> a  partition. The result is the
Schur  functor  of  the  matrix  <mat>  corresponding to partition <l>; for
example,  if 'l=[n]' it returns the n-th symmetric power and if 'l=[1,1,1]'
it  returns the 3rd exterior power. The current algorithm (from Littlewood)
is rather inefficient so it is quite slow for partitions of $n$ where $n>6$.

|    gap> m:=CartanMat("A",3);
    [ [ 2, -1, 0 ], [ -1, 2, -1 ], [ 0, -1, 2 ] ]
    gap> SchurFunctor(m,[2,2]);
    [ [ 10, 12, -16, 16, -16, 12 ], [ 3/2, 9, -6, 4, -2, 1 ],
      [ -4, -12, 16, -16, 8, -4 ], [ 2, 4, -8, 16, -8, 4 ],
      [ -4, -4, 8, -16, 16, -12 ], [ 3/2, 1, -2, 4, -6, 9 ] ]|

%%%%%%%%%%%%%%%%%%%%%%%%%%%%%%%%%%%%%%%%%%%%%%%%%%%%%%%%%%%%%%%%%%%%%%%%%
\Section{IsNormalizing}
\index{IsNormalizing}

'IsNormalizing( <lst>, <mat> )'

returns true or  false according to whether  the matrix  <mat> leaves the
vectors in <lst> as a set invariant, i.e., 'Set(l \*\ M) = Set( l )'.

|    gap> a := [ [ 1, 2 ], [ 3, 1 ] ];;
    gap> l := [ [ 1, 0 ], [ 0, 1 ], [ 1, 1 ], [ 0, 0 ] ];;
    gap> l * a;
    [ [ 1, 2 ], [ 3, 1 ], [ 4, 3 ], [ 0, 0 ] ]
    gap> IsNormalizing( l, a );
    false |

%%%%%%%%%%%%%%%%%%%%%%%%%%%%%%%%%%%%%%%%%%%%%%%%%%%%%%%%%%%%%%%%%%%%%%%%%
\Section{IndependentLines}
\index{IndependentLines}

'IndependentLines( <M> )'

Returns the smallest (for lexicographic order) subset <I> of '[1..Length(M)]'
such that the rank of |M{I}| is equal to the rank of <M>.

|    gap> M:=CartanMat(ComplexReflectionGroup(31));
    [ [ 2, 1+E(4), 1-E(4), -E(4), 0 ], [ 1-E(4), 2, 1-E(4), -1, -1 ],
      [ 1+E(4), 1+E(4), 2, 0, -1 ], [ E(4), -1, 0, 2, 0 ], 
      [ 0, -1, -1, 0, 2 ] ]
    gap> IndependentLines(M);
    [ 1, 2, 4, 5 ]|

%%%%%%%%%%%%%%%%%%%%%%%%%%%%%%%%%%%%%%%%%%%%%%%%%%%%%%%%%%%%%%%%%%%%%%%%%
\Section{OnMatrices}%
\index{OnMatrices}%

'OnMatrices( <M> , <p>)'

Effects the simultaneous permutation of the lines and columns of the matrix
<M> specified by the permutation <p>.

|    gap> M:=DiagonalMat([1,2,3]);
    [ [ 1, 0, 0 ], [ 0, 2, 0 ], [ 0, 0, 3 ] ]
    gap> OnMatrices(M,(1,2,3));
    [ [ 3, 0, 0 ], [ 0, 1, 0 ], [ 0, 0, 2 ] ]|

%%%%%%%%%%%%%%%%%%%%%%%%%%%%%%%%%%%%%%%%%%%%%%%%%%%%%%%%%%%%%%%%%%%%%%%%%
\Section{PermMatMat}%
\index{PermMatMat}%

'PermMatMat( <M> , <N> [, <l1>, <l2>])'

<M>  and  <N>  should  be  symmetric  matrices.  'PermMatMat'  returns a
permutation  <p>  such  that  'OnMatrices(M,p)=N'  if such a permutation
exists,  and  'false'  otherwise.  If  list  arguments <l1> and <l2> are
given, the permutation <p> should also satisfy 'Permuted(l1,p)=l2'.

This  routine is useful to identify two objects which are isomorphic but
with  different labelings.  It is  used in  \CHEVIE\ to  identify Cartan
matrices   and   Lusztig   Fourier   transform  matrices  with  standard
(classified)  data. The  program uses  sophisticated algorithms, and can
often handle matrices up to $80\times 80$.

|    gap> M:=CartanMat("D",12);;
    gap> p:=Random(SymmetricGroup(12));
    ( 1,12, 7, 5, 9, 8, 3, 6)( 2,10)( 4,11)
    gap> N:=OnMatrices(M,p);;
    gap> PermMatMat(M,N);
    ( 1,12, 7, 5, 9, 8, 3, 6)( 2,10)( 4,11) |

%%%%%%%%%%%%%%%%%%%%%%%%%%%%%%%%%%%%%%%%%%%%%%%%%%%%%%%%%%%%%%%%%%%%%%%%%
\Section{BigCellDecomposition}%
\index{BigCellDecomposition}%

'BigCellDecomposition(M [, b])'

<M>  should be a square  matrix, and <b> specifies  a block structure for a
matrix  of  same  size  as  <M>  (it  is  a  list  of  lists whose union is
'[1..Length(M)]').  If  <b>  is  not  given,  the  trivial  block structure
'[[1],..,[Length(M)]]' is assumed.

The  function decomposes  <M> as  a product  $P_1 L  P$ where  <P> is upper
block-unitriangular   (with  identity  diagonal  blocks),  $P_1$  is  lower
block-unitriangular  and <L> is block-diagonal for the block structure <b>.
If  <M> is symmetric then $P_1$ is the  transposed of <P> and the result is
the  pair  '[P,L]';  else  the  result  is  the triple '[P1,L,P]'. The only
condition  for  this  decomposition  of  <M>  to  be  possible  is that the
principal minors according to the block structure be non-zero. This routine
is  used  when  computing  the  green  functions  and  the example below is
extracted from the computation of the Green functions for $G_2$.

|    gap> q:=X(Rationals);;q.name:="q";;
    gap> M:= [ [ q^6, q^0, q^3, q^3, q^5 + q, q^4 + q^2 ],
    > [ q^0, q^6, q^3, q^3, q^5 + q, q^4 + q^2 ],
    > [ q^3, q^3, q^6, q^0, q^4 + q^2, q^5 + q ],
    > [ q^3, q^3, q^0, q^6, q^4 + q^2, q^5 + q ],
    > [ q^5 + q, q^5 + q, q^4 + q^2, q^4 + q^2, q^6 + q^4 + q^2 + 1,
    >    q^5 + 2*q^3 + q ],
    >     [ q^4 + q^2, q^4 + q^2, q^5 + q, q^5 + q, q^5 + 2*q^3 + q,
    >    q^6 + q^4 + q^2 + 1 ] ];;
    gap> bb:=[ [ 2 ], [ 4 ], [ 6 ], [ 3, 5 ], [ 1 ] ];;
    gap> PL:=BigCellDecomposition(M,bb);
    [ [ [ q^0, 0*q^0, 0*q^0, 0*q^0, 0*q^0, 0*q^0 ],
          [ q^(-6), q^0, q^(-3), q^(-3), q^(-1) + q^(-5), q^(-2) + q^(-4)
             ], [ 0*q^0, 0*q^0, q^0, 0*q^0, 0*q^0, 0*q^0 ],
          [ q^(-3), 0*q^0, 0*q^0, q^0, q^(-2), q^(-1) ],
          [ q^(-1), 0*q^0, 0*q^0, 0*q^0, q^0, 0*q^0 ],
          [ q^(-2), 0*q^0, q^(-1), 0*q^0, q^(-1), q^0 ] ],
      [ [ q^6 - q^4 - 1 + q^(-2), 0*q^0, 0*q^0, 0*q^0, 0*q^0, 0*q^0 ],
          [ 0*q^0, q^6, 0*q^0, 0*q^0, 0*q^0, 0*q^0 ],
          [ 0*q^0, 0*q^0, q^6 - q^4 - 1 + q^(-2), 0*q^0, 0*q^0, 0*q^0 ],
          [ 0*q^0, 0*q^0, 0*q^0, q^6 - 1, 0*q^0, 0*q^0 ],
          [ 0*q^0, 0*q^0, 0*q^0, 0*q^0, q^6 - q^4 - 1 + q^(-2), 0*q^0 ],
          [ 0*q^0, 0*q^0, 0*q^0, 0*q^0, 0*q^0, q^6 - 1 ] ] ]
    gap> M=TransposedMat(PL[1])*PL[2]*PL[1];
    true|

%%%%%%%%%%%%%%%%%%%%%%%%%%%%%%%%%%%%%%%%%%%%%%%%%%%%%%%%%%%%%%%%%%%%%%%%%
