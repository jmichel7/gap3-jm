%%%%%%%%%%%%%%%%%%%%%%%%%%%%%%%%%%%%%%%%%%%%%%%%%%%%%%%%%%%%%%%%%%%%%%%%%%%%%
%%
%A  field.tex                   GAP documentation            Martin Schoenert
%%
%A  @(#)$Id: field.tex,v 1.1.1.1 1996/12/11 12:36:44 werner Exp $
%%
%Y  Copyright 1990-1992,  Lehrstuhl D fuer Mathematik,  RWTH Aachen,  Germany
%%
%%  This file describes the operators and functions of finite field elements.
%%
%H  $Log: field.tex,v $
%H  Revision 1.1.1.1  1996/12/11 12:36:44  werner
%H  Preparing 3.4.4 for release
%H
%H  Revision 3.12  1994/06/03  08:57:20  mschoene
%H  changed a few things to avoid LaTeX warnings
%H
%H  Revision 3.11  1994/03/18  14:22:47  ahulpke
%H  Fixed 'Conjugates' and its relatives for nonabelian extensions
%H
%H  Revision 3.10  1993/02/19  10:48:42  gap
%H  adjustments in line length and spelling
%H
%H  Revision 3.9  1993/02/15  09:49:57  felsch
%H  another example fixed
%H
%H  Revision 3.8  1993/02/13  08:49:49  felsch
%H  error messages in examples fixed
%H
%H  Revision 3.7  1993/02/12  17:19:25  felsch
%H  examples adjusted to line length 72
%H
%H  Revision 3.6  1992/05/25  18:13:25  martin
%H  fixed a typo in "Field Homomorphisms"
%H
%H  Revision 3.5  1992/04/06  15:19:57  martin
%H  fixed some more typos
%H
%H  Revision 3.4  1992/03/25  15:37:32  martin
%H  added new sections for field homomorphisms
%H
%H  Revision 3.3  1992/03/11  15:50:48  sam
%H  renamed chapter "Number Fields" to "Subfields of Cyclotomic Fields"
%H
%H  Revision 3.2  1991/12/30  12:07:53  martin
%H  fixed a few incorrect references
%H
%H  Revision 3.1  1991/12/30  11:45:29  martin
%H  changed incorrect reference to "CyclotomicField"
%H
%H  Revision 3.0  1991/12/27  16:10:27  martin
%H  initial revision under RCS
%H
%%
\Chapter{Fields}

Fields  are important algebraic  domains.  Mathematically a *field*  is a
commutative  ring  $F$  (see  chapter  "Rings"), such that  every element
except $0$ has a multiplicative inverse.  Thus $F$ has two operations '+'
and '\*' called addition  and multiplication.  $(F,+)$ must be an abelian
group,  whose identity  is called  $0_F$.   $(F-\{0_F\},\*)$  must  be an
abelian group, whose identity element is called $1_F$.

{\GAP} supports  the field  of rationals (see "Rationals"),  subfields of
cyclotomic  fields  (see  "Subfields of Cyclotomic Fields"),  and  finite
fields (see "Finite Fields").

This chapter begins with sections  that describe how  to  test whether  a
domain is a field (see "IsField"), how to find the smallest field and the
default  field  in  which  a  list  of  elements lies  (see  "Field"  and
"DefaultField"),  and how  to view a field  over  a subfield (see "Fields
over Subfields").

The next sections describes  the operation applicable  to field  elements
(see  "Comparisons  of  Field  Elements"   and   "Operations  for   Field
Elements").

The next sections  describe the  functions that are applicable  to fields
(see   "GaloisGroup")  and their   elements  (see   "Conjugates", "Norm",
"Trace", "CharPol",  and  "MinPol").

The  following  sections  describe  homomorphisms  of fields  (see "Field
Homomorphisms",     "IsFieldHomomorphism",     "KernelFieldHomomorphism",
"Mapping Functions for Field Homomorphisms").

The  last  section  describes how  fields are represented internally (see
"Field Records").

Fields  are domains, so all functions  that are applicable to all domains
are also applicable to fields (see chapter "Domains").

All functions for fields are in 'LIBNAME/\"field.g\"'.

%%%%%%%%%%%%%%%%%%%%%%%%%%%%%%%%%%%%%%%%%%%%%%%%%%%%%%%%%%%%%%%%%%%%%%%%%
\Section{IsField}

'IsField( <D> )'

'IsField'   returns  'true' if the   object <D>  is  a field  and 'false'
otherwise.

More precisely 'IsField' tests whether <D>  is a field record (see "Field
Records").  So, for example, a matrix group  may in fact be a field,  yet
'IsField' would return 'false'.

|    gap> IsField( GaloisField(16) );
    true
    gap> IsField( CyclotomicField(9) );
    true
    gap> IsField( rec( isDomain := true, isField := true ) );
    true    # it is possible to fool 'IsField'
    gap> IsField( AsRing( Rationals ) );
    false    # though this ring is, as a set, still 'Rationals' |

%%%%%%%%%%%%%%%%%%%%%%%%%%%%%%%%%%%%%%%%%%%%%%%%%%%%%%%%%%%%%%%%%%%%%%%%%
\Section{Field}

'Field( <z>,.. )'
'Field( <list> )'

In the first  form 'Field' returns  the smallest field  that contains all
the elements <z>,.. etc.  In the second form 'Field' returns the smallest
field that contains all the elements in  the list <list>.  If any element
is not an element of a  field or the  elements lie in no common  field an
error is raised.

|    gap> Field( Z(4) );
    GF(2^2)
    gap> Field( E(9) );
    CF(9)
    gap> Field( [ Z(4), Z(9) ] );
    Error, CharFFE: <z> must be a finite field element, vector, or matrix
    gap> Field( [ E(4), E(9) ] );
    CF(36) |

'Field' differs from  'DefaultField'   (see "DefaultField") in    that it
returns the     smallest  field  in  which   the   elements    lie, while
'DefaultField' may return a larger field if that makes sense.

%%%%%%%%%%%%%%%%%%%%%%%%%%%%%%%%%%%%%%%%%%%%%%%%%%%%%%%%%%%%%%%%%%%%%%%%%
\Section{DefaultField}

'DefaultField( <z>,.. )'
'DefaultField( <list> )'

In the first form 'DefaultField' returns the  default field that contains
all the elements <z>,.. etc.  In the  second form 'DefaultField' returns
the default field that contains all the elements in  the list <list>.  If
any element is not an element of a field or the elements lie in no common
field an error is raised.

The field returned by  'DefaultField'  need not be the  smallest field in
which the elements lie.  For example for elements  from cyclotomic fields
'DefaultField' may  return the  smallest  cyclotomic field  in  which the
elements lie, which need not be the  smallest field overall,  because the
elements may  in  fact lie  in  a  smaller number field which   is  not a
cyclotomic field.

For the exact   definition of  the default field  of a   certain type  of
elements read  the chapter describing  this type (see "Finite Fields" and
"Subfields of Cyclotomic Fields").

'DefaultField' is used  by 'Conjugates', 'Norm', 'Trace',  'CharPol', and
'MinPol' (see "Conjugates", "Norm",  "Trace", "CharPol", and "MinPol") if
no explicit field is given.

|    gap> DefaultField( Z(4) );
    GF(2^2)
    gap> DefaultField( E(9) );
    CF(9)
    gap> DefaultField( [ Z(4), Z(9) ] );
    Error, CharFFE: <z> must be a finite field element, vector, or matrix
    gap> DefaultField( [ E(4), E(9) ] );
    CF(36) |

'Field' (see "Field") differs from  'DefaultField' in that it returns the
smallest field in which the elements lie, while 'DefaultField' may return
a larger field if that makes sense.

%%%%%%%%%%%%%%%%%%%%%%%%%%%%%%%%%%%%%%%%%%%%%%%%%%%%%%%%%%%%%%%%%%%%%%%%%
\Section{Fields over Subfields}

'<F> / <G>'

The quotient operator '/' evaluates to a new  field  <H>.  This field has
the same elements as <F>, i.e., is a domain equal to <F>.  However <H> is
viewed as a field over the field <G>, which must be a subfield of <F>.

What subfield a field  is viewed over  determines  its Galois  group.  As
described in    "GaloisGroup" the Galois group  is   the group   of field
automorphisms  that  leave the subfield   fixed.  It also  influences the
results of "Norm",  "Trace",  "CharPol", and  "MinPol", because they  are
defined in terms of the Galois group.

|    gap> F := GF(2^12);
    GF(2^12)
    gap> G := GF(2^2);
    GF(2^2)
    gap> Q := F / G;
    GF(2^12)/GF(2^2)
    gap> Norm( F, Z(2^6) );
    Z(2)^0
    gap> Norm( Q, Z(2^6) );
    Z(2^2)^2 |

The operator '/' calls '<G>.operations./( <F>, <G> )'.

The default function called this  way is 'FieldOps./', which simply makes
a copy  of <F> and enters <G>  into the record component '<F>.field' (see
"Field Records").

%%%%%%%%%%%%%%%%%%%%%%%%%%%%%%%%%%%%%%%%%%%%%%%%%%%%%%%%%%%%%%%%%%%%%%%%%
\Section{Comparisons of Field Elements}

'<f> =   <g>' \\
'<f> \<> <g>'

The equality operator  '=' evaluates to 'true' if the  two field elements
<f> and <g> are equal, and to 'false' otherwise.  The inequality operator
'\<>' evaluates to 'true' if the two field  elements <f> and <g>  are not
equal, and to 'false' otherwise.  Note that any two field elements can be
compared, even  if they  do  not lie in compatible fields.   In this case
they cn, of course, never be equal.  For each type of fields the equality
of those field elements is given in the respective chapter.

Note that you can compare field elements with elements of other types; of
course they are never equal.

'<f> \<\ <g>' \\
'<f> \<= <g>' \\
'<f> >   <g>' \\
'<f> >=  <g>'

The  operators '\<', '\<=', '>', and '>=' evaluate to 'true' if the field
element <f> is less than, less than or equal to, greater than, or greater
than  or equal to the field  element  <g>.  For each  type of fields  the
definition of  the ordering of  those  field  elements is  given  in  the
respective  chapter.   The  ordering  of  field  elements is as  follows.
Rationals are smallest, next  are  cyclotomics, followed by  finite field
elements.

Note that  you can compare field elements with elements of  other  types;
they are smaller than everything else.

%%%%%%%%%%%%%%%%%%%%%%%%%%%%%%%%%%%%%%%%%%%%%%%%%%%%%%%%%%%%%%%%%%%%%%%%%
\Section{Operations for Field Elements}

The following operations  are always available for   field elements.   Of
course the operands must lie in compatible fields,  i.e., the fields must
be equal, or at least have a common superfield.

'<f> + <g>'

The operator '+' evaluates to  the sum of  the two field elements <f> and
<g>, which must lie in compatible fields.

'<f> - <g>'

The operator  '-'  evaluates to the difference of  the two field elements
<f> and <g>, which must lie in compatible fields.

'<f> \*\ <g>'

The operator '\*' evaluates to the product  of the two field elements <f>
and <g>, which must lie in compatible fields.

'<f> / <g>'

The operator '/' evaluates to the quotient of the two field  elements <f>
and <g>, which must  lie  in compatible fields.  If the  divisor is  0 an
error is signalled.

'<f> \^\ <n>'

The operator '\^' evaluates to the <n>-th power of the field element <f>.
If <n> is a  positive  integer  then  '<f>\^<n>'  is  '<f>\*<f>\*..\*<f>'
(<n> factors).  If <n> is a negative integer  '<f>\^<n>'  is  defined  as
$1 / {<f>^{-<n>}}$.   If 0 is  raised  to  a negative power   an error is
signalled.  Any field element, even 0, raised to the 0-th power yields 1.

For the precedence of the operators see "Operations".

%%%%%%%%%%%%%%%%%%%%%%%%%%%%%%%%%%%%%%%%%%%%%%%%%%%%%%%%%%%%%%%%%%%%%%%%%
\Section{GaloisGroup}
\index{Galois group!of a field}
\index{automorphism group!of a field}

'GaloisGroup( <F> )'

'GaloisGroup' returns the Galois group of  the field <F>  as a group (see
"Groups") of field automorphisms (see "Field Homomorphisms").

The Galois group of a field <F> over a subfield '<F>.field' is the  group
of automorphisms of <F> that  leave the subfield '<F>.field' fixed.  This
group can be interpreted as a permutation group  permuting  the zeroes of
the characteristic polynomial of a primitive  element of <F>.  The degree
of this group is equal to the number of zeroes, i.e., to the dimension of
<F>  as  a vector  space  over  the subfield  '<F>.field'.   It  operates
transitively on those  zeroes.   The normal divisors of the  Galois group
correspond to the subfields between '<F>' and '<F>.field'.

|    gap> G := GaloisGroup( GF(4096)/GF(4) );;
    gap> Size( G );
    6
    gap> IsCyclic( G );
    true    # the Galois group of every finite field is
            # generated by the Frobenius automorphism
    gap> H := GaloisGroup( CF(60) );;
    gap> Size( H );
    16
    gap> IsAbelian( H );
    true |

The default function 'FieldOps.GaloisGroup'  just raises  an error, since
there is no general method to compute the Galois  group of a field.  This
default function is overlaid by more specific functions for special types
of domains (see "Field Functions  for Finite Fields" and "GaloisGroup for
Number Fields").

%%%%%%%%%%%%%%%%%%%%%%%%%%%%%%%%%%%%%%%%%%%%%%%%%%%%%%%%%%%%%%%%%%%%%%%%%
\Section{MinPol}
\index{minimal polynom!of a field element}

'MinPol( <z> )' \\
'MinPol( <F>, <z> )'

In the first form 'MinPol' returns the coefficients of the minimal
polynomial  of the element
<z> in its default field over its prime field  (see  "DefaultField").  In
the  second form 'MinPol' returns  the coefficients of the minimal
polynomial of  the element
<z>  in the field <F>  over   the  subfield  '<F>.field'.

Let $F/S$ be a field extension and $L$ a minimal normal extension of $S$,
containing $F$.
The *minimal polynomial* of $z$ in $F$  over $S$  is the squarefree
polynomial
whose roots  are precisely  the  conjugates of  $z$ in $L$ (see
"Conjugates").  Because  the set of
conjugates  is  fixed  under  the Galois  group  of  $L$  over  $S$  (see
"GaloisGroup"),  so is the polynomial.  Thus all  the coefficients of the
minimal polynomial lie in $S$.

|    gap> MinPol( Z(2^6) );
    [ Z(2)^0, Z(2)^0, 0*Z(2), Z(2)^0, Z(2)^0, 0*Z(2), Z(2)^0 ]
    gap> MinPol( GF(2^12), Z(2^6) );
    [ Z(2)^0, Z(2)^0, 0*Z(2), Z(2)^0, Z(2)^0, 0*Z(2), Z(2)^0 ]
    gap> MinPol( GF(2^12)/GF(2^2), Z(2^6) );
    [ Z(2^2), Z(2)^0, Z(2)^0, Z(2)^0 ] |

The default function 'FieldOps.MinPol', which works only for extensions with
abelian Galois group, multiplies  the  linear factors $x - c$
with  <c>  ranging  over  the  set  of  conjugates  of  <z> in  <F>  (see
"Conjugates"). For generic algebraic extensions, it is overlayed by solving
a system of linear equations, given by the coefficients of powers of <z>
in respect to a given base.

%%%%%%%%%%%%%%%%%%%%%%%%%%%%%%%%%%%%%%%%%%%%%%%%%%%%%%%%%%%%%%%%%%%%%%%%%
\Section{CharPol}
\index{characteristic polynom!of a field element}

'CharPol( <z> )' \\
'CharPol( <F>, <z> )'

In the first form 'CharPol'  returns the coefficients of the characteristic
polynomial of the element  <z>   in   its  default  field   over  its
prime   field  (see "DefaultField").  In the second form 'CharPol' returns
the coefficients of the characteristic
polynomial of  the  element  <z>  in  the  field  <F>  over the  subfield
'<F>.field'.  The characteristic polynomial  is  returned  as  a  list of
coefficients, the <i>-th entry is the coefficient of $x^{i-1}$.

The *characteristic polynomial*  of an element  $z$ in a field $F$ over a
subfield $S$  is the $\frac{[F\:S]}{{\rm deg } \mu}$-th power of $\mu$, where
$\mu$ denotes the minimal polynomial of $z$ in $F$ over $S$. It is fixed under
the Galois group of the normal closure of $F$.
Thus all  the coefficients  of the characteristic polynomial
lie in  $S$.  The constant  term is
$(-1)^{F.degree/S.degree}=(-1)^{[F\:S]}$ times  the
norm of  $z$ (see  "Norm"), and the  coefficient  of  the  second highest
degree  term  is  the negative of the  trace of $z$  (see  "Trace").
The roots (including their multiplicities) in $F$ of the characteristic
polynomial of <z> in $F$ are the conjugates (see "Conjugates") of $z$ in
$F$.

|    gap> CharPol( Z(2^6) );
    [ Z(2)^0, Z(2)^0, 0*Z(2), Z(2)^0, Z(2)^0, 0*Z(2), Z(2)^0 ]
    gap> CharPol( GF(2^12), Z(2^6) );
    [ Z(2)^0, 0*Z(2), Z(2)^0, 0*Z(2), 0*Z(2), 0*Z(2), Z(2)^0, 0*Z(2), 
      Z(2)^0, 0*Z(2), 0*Z(2), 0*Z(2), Z(2)^0 ]
    gap> CharPol( GF(2^12)/GF(2^2), Z(2^6) );
    [ Z(2^2)^2, 0*Z(2), Z(2)^0, 0*Z(2), Z(2)^0, 0*Z(2), Z(2)^0 ] |

The default  function 'FieldOps.CharPol' multiplies the linear factors $x - c$
with <c>  ranging over  the conjugates of <z> in <F>  (see "Conjugates").
For nonabelian extensions, it is overlayed by a function, which computes the
appropriate power of the minimal polynomial.

%%%%%%%%%%%%%%%%%%%%%%%%%%%%%%%%%%%%%%%%%%%%%%%%%%%%%%%%%%%%%%%%%%%%%%%%%
\Section{Norm}
\index{norm!of a field element}

'Norm( <z> )' \\
'Norm( <F>, <z> )'

In the first form 'Norm' returns the norm of the field element <z> in its
default field over  its prime field  (see "DefaultField").  In the second
form 'Norm' returns the norm  of  <z> in the field  <F> over the subfield
'<F>.field'.

The  *norm* of an element $z$ in  a field $F$ over a subfield  $S$ is
$(-1)^{F.degree/S.degree}=(-1)^{[F\:S]}$  times  the  constant  term  of  the
characteristic polynomial of $z$ (see "CharPol"). Thus the norm lies in $S$.
The norm is the product of all conjugates of $z$ in the normal closure of
$F$ over $S$ (see "Conjugates").

|    gap> Norm( Z(2^6) );
    Z(2)^0
    gap> Norm( GF(2^12), Z(2^6) );
    Z(2)^0
    gap> Norm( GF(2^12)/GF(2^2), Z(2^6) );
    Z(2^2)^2 |

The default function 'FieldOps.Norm'  multiplies the conjugates of <z> in
<F>  (see  "Conjugates"). For nonabelian extensions, it is overlayed by a
function, which obtains the norm from the characteristic polynomial.

%%%%%%%%%%%%%%%%%%%%%%%%%%%%%%%%%%%%%%%%%%%%%%%%%%%%%%%%%%%%%%%%%%%%%%%%%
\Section{Trace}
\index{trace!of a field element}

'Trace( <z> )' \\
'Trace( <F>, <z> )'

In the first form 'Trace' returns the trace  of the  field element <z> in
its  default field over   its prime field (see  "DefaultField").   In the
second form 'Trace' returns the trace of the element <z> in the field <F>
over the subfield '<F>.field'.

The  *trace* of an element $z$ in a field $F$ over a  subfield $S$ is the
negative  of the coefficient of  the  second  highest degree
term of the characteristic polynomial of $z$ (see "CharPol").
Thus the trace lies in $S$. The trace is the
sum  over  all  conjugates  of $z$  in the normal closure of $F$  over $S$
(see "Conjugates").

|    gap> Trace( Z(2^6) );
    0*Z(2)
    gap> Trace( GF(2^12), Z(2^6) );
    0*Z(2)
    gap> Trace( GF(2^12)/GF(2^2), Z(2^6) );
    0*Z(2) |

The default function 'FieldOps.Trace' adds the conjugates  of <z>  in <F>
(see "Conjugates").  For nonabelian extensions, this is overlayed by a
function, which obtains the trace from the characteristic polynomial.

%%%%%%%%%%%%%%%%%%%%%%%%%%%%%%%%%%%%%%%%%%%%%%%%%%%%%%%%%%%%%%%%%%%%%%%%%
\Section{Conjugates}
\index{Galois conjugates!of a field element}
\index{conjugates!of a field element, Galois}

'Conjugates( <z> )' \\
'Conjugates( <F>, <z> )'

In the  first form 'Conjugates'   returns the list   of conjugates of the
field element  <z>  in  its  default  field   over its prime   field (see
"DefaultField").  In the second  form  'Conjugates'  returns the list  of
conjugates of  the field element <z>  in the field <F> over  the subfield
'<F>.field'.  In either case the list may contain  duplicates if <z> lies
in a proper subfield of its default field, respectively of <F>.

The *conjugates* of an element $z$ in a field  $F$ over a subfield $S$
are the roots in $F$ of the characteristic polynomial of $z$ in $F$ (see
"CharPol"). If $F$ is a normal extension of $S$, then the conjugates of $z$
are the images of $z$ under all elements of the Galois group of $F$
over $S$ (see "GaloisGroup"), i.e., under those automorphisms of $F$ that leave
$S$  fixed.  The number of  different conjugates of  $z$  is given by the
degree of the smallest extension of $S$ in which $z$ lies.\\
For a normal extension $F$,
'Norm' (see "Norm") computes the product,  'Trace' (see "Trace") the  sum
of  all  conjugates.  'CharPol' (see "CharPol")  computes the  polynomial
that has precisely the conjugates with their corresponding multiplicities
as roots,  'MinPol' (see  "MinPol") the  squarefree  polynomial  that has
precisely the conjugates as roots.

|    gap> Conjugates( Z(2^6) );
    [ Z(2^6), Z(2^6)^2, Z(2^6)^4, Z(2^6)^8, Z(2^6)^16, Z(2^6)^32 ]
    gap> Conjugates( GF(2^12), Z(2^6) );
    [ Z(2^6), Z(2^6)^2, Z(2^6)^4, Z(2^6)^8, Z(2^6)^16, Z(2^6)^32, Z(2^6),
      Z(2^6)^2, Z(2^6)^4, Z(2^6)^8, Z(2^6)^16, Z(2^6)^32 ]
    gap> Conjugates( GF(2^12)/GF(2^2), Z(2^6) );
    [ Z(2^6), Z(2^6)^4, Z(2^6)^16, Z(2^6), Z(2^6)^4, Z(2^6)^16 ] |

The  default  function 'FieldOps.Conjugates' applies the automorphisms of
the Galois group of <F> (see "GaloisGroup")  to  <z> and returns the list
of images. For nonabelian extensions, this is overlayed by a factorization
of the characteristic polynomial.

%%%%%%%%%%%%%%%%%%%%%%%%%%%%%%%%%%%%%%%%%%%%%%%%%%%%%%%%%%%%%%%%%%%%%%%%%
\Section{Field Homomorphisms}%
\index{homomorphisms!of fields}

Field homomorphisms  are an  important  class  of homomorphisms in {\GAP}
(see chapter "Homomorphisms").

A *field  homomorphism* $\phi$ is a mapping that maps  each element of  a
field $F$,  called the  source of $\phi$, to  an element of another field
$G$, called the range of $\phi$, such that for each pair  $x,y \in F$  we
have $(x+y)^\phi =  x^\phi + y^\phi$ and $(xy)^\phi = x^\phi y^\phi$.  We
also  require  that  $\phi$ maps the one of $F$ to  the  one of $G$ (that
$\phi$  maps the zero of $F$  to the zero of $G$  is implied by the above
relations).

An  Example of a field  homomorphism  is the Frobinius automorphism of  a
finite   field   (see   "FrobeniusAutomorphism").    Look   under  *field
homomorphisms*   in  the  index   for  a  list  of  all  available  field
homomorphisms.

Since field homomorphisms are just a  special case of  homomorphisms, all
functions described  in  chapter  "Homomorphisms"  are applicable  to all
field homomorphisms, e.g., the function to test if a homomorphism is a an
automorphism   (see   "IsAutomorphism").    More   general,  since  field
homomorphisms are just a special case of mappings all functions described
in chapter "Mappings" are also applicable,  e.g., the function to compute
the image of an element under a homomorphism (see "Image").

The following sections describe the functions that test whether a mapping
is a field homomorphism (see  "IsFieldHomomorphism"), compute the  kernel
of  a  field  homomorphism  (see  "KernelFieldHomomorphism"), and how the
general mapping functions are implemented for field homomorphisms.

%%%%%%%%%%%%%%%%%%%%%%%%%%%%%%%%%%%%%%%%%%%%%%%%%%%%%%%%%%%%%%%%%%%%%%%%%
\Section{IsFieldHomomorphism}%
\index{IsHomomorphism!for fields}

'IsFieldHomomorphism( <map> )'

'IsFieldHomomorphism' returns  'true'  if  the  mapping <map> is a  field
homomorphism and 'false' otherwise.  Signals an error if <map> is a multi
valued mapping.

A mapping $map$ is a field homomorphism if  its source  $F$ and range $G$
are both fields and  if  for each  pair of elements $x, y  \in F$ we have
$(x+y)^{map} = x^{map} + y^{map}$ and $(xy)^{map} = x^{map} y^{map}$.  We
also require that $1_F^{map} = 1_G$.

|    gap> f := GF( 16 );
    GF(2^4)
    gap> fun := FrobeniusAutomorphism( f );
    FrobeniusAutomorphism( GF(2^4) )
    gap> IsFieldHomomorphism( fun );
    true |

'IsFieldHomomorphism' first tests if the flag '<map>.isFieldHomomorphism'
is bound.  If the flag is bound, 'IsFieldHomomorphism' returns its value.
Otherwise it calls \\
'<map>.source.operations.IsFieldHomomorphism(  <map>  )',  remembers  the
returned value in '<map>.isFieldHomomorphism', and returns it.  Note that
of course all  functions  that create  field  homomorphism  set  the flag
'<map>.isFieldHomomorphism'  to 'true', so that no function is called for
those field homomorphisms.

The default function called this way is 'MappingOps.IsFieldHomomorphism'.
It computes all the elements of the source of <map> and  for each pair of
elements  $x, y$ tests whether $(x+y)^{map}  =  x^{map}  +  y^{map}$  and
$(xy)^{map} = x^{map} y^{map}$.  Look under *IsHomomorphism* in the index
to see for which mappings this function is overlaid.

%%%%%%%%%%%%%%%%%%%%%%%%%%%%%%%%%%%%%%%%%%%%%%%%%%%%%%%%%%%%%%%%%%%%%%%%%
\Section{KernelFieldHomomorphism}%
\index{Kernel!for fields}

'KernelFieldHomomorphism( <hom> )'

'KernelFieldHomomorphism' returns  the  kernel  of the field homomorphism
<hom>.

Because the kernel must be a  ideal in the source  and it can not  be the
full source  (because we  require that the one of the source is mapped to
the one of the range), it must be the trivial ideal.  Therefor the kernel
of  every field homomorphism is the set  containing  only the zero of the
source.

%%%%%%%%%%%%%%%%%%%%%%%%%%%%%%%%%%%%%%%%%%%%%%%%%%%%%%%%%%%%%%%%%%%%%%%%%
\Section{Mapping Functions for Field Homomorphisms}%
\index{IsInjective!for field homomorphisms}%
\index{IsSurjective!for field homomorphisms}%
\index{equality!for field homomorphisms}%
\index{Image!for field homomorphisms}%
\index{Images!for field homomorphisms}%
\index{PreImage!for field homomorphisms}%
\index{PreImages!for field homomorphisms}

This  section describes how  the mapping  functions  defined  in  chapter
"Mappings" are implemented for field homomorphisms.  Those  functions not
mentioned here are implemented by the default functions described in  the
respective sections.

\vspace{5mm}
'IsInjective( <hom> )'

Always returns 'true' (see "KernelFieldHomomorphism").

\vspace{5mm}
'IsSurjective( <hom> )'

The  field  homomorphism  <hom> is surjective if  the  size  of the image
'Size(  Image( <hom> ) )' is equal  to  the  size  of  the  range  'Size(
<hom>.range )'.

\vspace{5mm}
'<hom1> = <hom2>'

The  two field homomorphism <hom1> and <hom2> are are equal  if  the have
the  same  source and range and  if the images of the generators  of  the
source under <hom1> and <hom2> are equal.

\vspace{5mm}
'Image( <hom> )' \\
'Image( <hom>, <H> )' \\
'Images( <hom>, <H> )'

The  image  of  a  subfield  under a  field homomorphism  is computed  by
computing the images  of  a  set of  generators of the  subfield, and the
result is the subfield generated by those images.

\vspace{5mm}
'PreImage( <hom> )' \\
'PreImage( <hom>, <H> )' \\
'PreImages( <hom>, <H> )'

The preimages of a subfield  under a  field  homomorphism are computed by
computing the  preimages of all the generators  of  the subfield, and the
result is the subfield generated by those elements.

Look in the index under *IsInjective*, *IsSurjective*, *Image*, *Images*,
*PreImage*,   *PreImages*, and   *equality*   to   see for  which   field
homomorphisms these functions are overlaid.

%%%%%%%%%%%%%%%%%%%%%%%%%%%%%%%%%%%%%%%%%%%%%%%%%%%%%%%%%%%%%%%%%%%%%%%%%
\Section{Field Records}

A field is represented by  a record  that contains important  information
about this field.  The {\GAP} library predefines some  field records, for
example  'Rationals'  (see  "Rationals").  Field  constructors  construct
others,  for  example  'Field'  (see  "Field"),  and  'GaloisField'  (see
"GaloisField").  Of course you may also create such a record by hand.

All field  records contain the  components 'isDomain', 'isField', 'char',
'degree', 'generators', 'zero',  'one', 'field', 'base', and 'dimension'.
They  may  also   contain the  optional  components  'isFinite',  'size',
'galoisGroup'.   The  contents  of  all  components of   a field <F>  are
described below.

'isDomain': \\
        is always 'true'.  This indicates that <F> is a domain.

'isField': \\
        is always 'true'.  This indicates that <F> is a field.

'char': \\
        is the characteristic of <F>.  For finite fields this is always a
        prime, for infinite fields this is 0.

'degree': \\
        is the degree of  <F> *as extension of the  prime field*,  not as
        extension of  the subfield <S>.   For finite fields the  order of
        <F> is given by '<F>.char\^ <F>.degree'.

'generators': \\
        a list of elements  that together generate <F>.   That is  <F> is
        the smallest field over the prime  field given by '<F>.char' that
        contains the elements of '<F>.generators'.

'zero': \\
        is the additive neutral element of the finite field.

'one': \\
        is the multiplicative neutral element of the finite field.

'field': \\
        is  the subfield  <S> over which   <F> was constructed.  This  is
        either a field  record for <S>, or the  same value as '<F>.char',
        denoting the prime field (see "Fields over Subfields").

'base': \\
        is a list of elements  of <F> forming a  base  for <F> as  vector
        space over the subfield <S>.

'dimension': \\
        is the dimension of <F> as vector space over the subfield <S>.

'isFinite': \\
        if present this is 'true' if the field  <F> is finite and 'false'
        otherwise.

'size': \\
        if present this is the size of the field <F>.  If <F> is infinite
        this holds the string \"infinity\".

'galoisGroup': \\
        if    present   this holds  the     Galois    group of  <F>  (see
        "GaloisGroup").

%%%%%%%%%%%%%%%%%%%%%%%%%%%%%%%%%%%%%%%%%%%%%%%%%%%%%%%%%%%%%%%%%%%%%%%%%
%%
%E  Emacs . . . . . . . . . . . . . . . . . . . . . local Emacs variables
%%
%%  Local Variables:
%%  mode:               outline
%%  outline-regexp:     "\\\\Chapter\\|\\\\Section"
%%  fill-column:        73
%%  eval:               (hide-body)
%%  End:
%%



