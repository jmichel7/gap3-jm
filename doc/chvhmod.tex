%%%%%%%%%%%%%%%%%%%%%%%%%%%%%%%%%%%%%%%%%%%%%%%%%%%%%%%%%%%%%%%%%%%%%%%%%%%%%
%%
%A  chvhmod.tex       CHEVIE documentation       Meinolf Geck, Frank Luebeck,
%A                                                Jean Michel, G"otz Pfeiffer
%%
%Y  Copyright (C) 1992 - 1996  Lehrstuhl D f\"ur Mathematik, RWTH Aachen, IWR
%Y  der Universit\"at Heidelberg, University of St. Andrews, and   University
%Y  Paris VII.
%%
%%  This  file  contains  the  description  of  the  GAP functions of CHEVIE
%%  dealing with Hecke modules.
%%%%%%%%%%%%%%%%%%%%%%%%%%%%%%%%%%%%%%%%%%%%%%%%%%%%%%%%%%%%%%%%%%%%%%%%%

\Chapter{Parabolic modules for Iwahori-Hecke algebras}

Let $H$  be the  Hecke algebra  of the Coxeter  group $W$  with Coxeter
generating set  $S$, and let $I$  be a subset  of $S$. Let $\chi$  be a
one-dimensional  character of  the parabolic  subalgebra $H_I$  of $H$.
Then $H\otimes_{H_I}\chi$  (the induced  representation of  $\chi$ from
$H_I$  to  $H$)  is  naturally  a  $H$-module,  with  a  natural  basis
$MT_w=T_w\otimes 1$ indexed  by the reduced-$I$ elements  of $W$ (i.e.,
those elements $w$ such that $l(ws)>l(w)$ for any $s\in I$).

The module  action of  an generator  $T_s$ of  $H$ which  satisfies the
quadratic  relation  $(T_s-p_s)(T_s-q_s)=0$  is  given  in  this  basis
by\:  $$T_s\cdot MT_w=\left\{\begin{array}{cl}
\chi(T_{w^{-1}sw})MT_w,&
\text{if  $sw$ is  not reduced-$I$ (then $w^{-1}sw\in  I$).}\\
-p_s q_sMT_{sw}+(p_s+q_s)MT_w,&
\text{if $sw\<w$ is reduced-$I$.}\\
MT_{sw},& \text{if $sw>w$ is reduced-$I$.}\\
\end{array}\right.$$

Kazhdan-Lusztig bases of  an Hecke module are also defined  in the same
circumstances when Kazhdan-Lusztig bases of the algebra can be defined,
but only the case  of the base 'C\'' for $\chi$  the sign character has
been implemented for now.

%%%%%%%%%%%%%%%%%%%%%%%%%%%%%%%%%%%%%%%%%%%%%%%%%%%%%%%%%%%%%%%%%%%%%%%%%
\Section{Construction of Hecke module elements of the $MT$ basis}
\index{ModuleBasis}

'ModuleBasis( <H>, \"MT\"  [, <I> [,<chi>]] )'

<H> should  be an  Iwahori-Hecke algebra  of a  Coxeter group  <W> with
Coxeter generating set <S>, <I> should be a subset of <S> (specified by
a list of  the names of the  generators in <I>), and <chi>  should be a
one-dimensional character of the parabolic subalgebra of <H> determined
by <I>, represented by the list of its values on $\{T_s\}_{s\in I}$ (if
<chi>  takes the  same  value on  all  generators of  $H_I$  it can  be
represented by a single value).

The result is a function which can be used to make elements of the 'MT'
basis of the Hecke module associated to <I> and <chi>.

If  omitted,  <I>  is  assumed to  be  the  first  'W.semiSimpleRank-1'
generators of $W$ (this makes sense for an affine Weyl group where they
generate the corresponding linear Weyl group), and <chi> is taken to be
equal to $-1$ (which specifies the sign character of $H$).

It  is convenient  to assign  this function  with a  shorter name  when
computing with elements of the Hecke  module. In what follows we assume
that we have done the assignment\:\

|    gap> W:=CoxeterGroup("A",2);;Wa:=Affine(W);;
    gap> q:=X(Rationals);;q.name:="q";;
    gap> H:=Hecke(Wa,q);
    Hecke(~A2,q)
    gap> MT:=ModuleBasis(H,"MT");
    function ( arg ) ... end|

'MT( <w> )'

Here  <w> is  an element  of the  Coxeter group  'Group(H)'. The  basis
element  $MT_w$ is  returned if  $w$ is  reduced-<I>, and  otherwise an
error is signaled.


'MT( <elts>, <coeffs>)'

In this form, 'elts' is a list of elements of 'Group(H)' and 'coeffs' a
list  of coefficients  which  should be  of the  same  length 'k'.  The
element 'Sum([1..k],i->coeffs[i]\*MT(elts[i]))' is returned.

'MT( <list> )'

'MT( <s1>, .., <sn> )'

In  the above  two  forms, the  \GAP\  list <list>  or  the \GAP\  list
'[<s1>,..,<sn>]'  represents the  Coxeter word  for an  element <w>  of
'Group(H)'. The basis element $MT_w$ is returned if $w$ is reduced-<I>,
and otherwise an error is signaled.

The way elements of the Hecke module are printed depends on
'CHEVIE.PrintHecke'.   If   'CHEVIE.PrintHecke=rec(GAP:=true)',   they  are
printed  in a way which  can be input back  in \GAP. When you load \CHEVIE,
the 'PrintHecke' is initially set to 'rec()'.

%%%%%%%%%%%%%%%%%%%%%%%%%%%%%%%%%%%%%%%%%%%%%%%%%%%%%%%%%%%%%%%%%%%%%%%%%
\Section{Construction of Hecke module elements of the primed $MC$ basis}

'ModuleBasis( <H>, \"MC\'\" [, <I>] )'

<H> should be  an Iwahori-Hecke algebra with all parameters  a power of
the same indeterminate  of a Coxeter group <W>  with Coxeter generating
set <S> and <I>  should be a subset of <S> (specified by  a list of the
names of the  generators in <I>). The character <chi>  does not have to
be specified since in this case only <chi>'=-1' has been implemented.

If  omitted,  <I>  is  assumed to  be  the  first  'W.semiSimpleRank-1'
generators of $W$ (this makes sense for an affine Weyl group where they
generate the corresponding linear Weyl group).

The result  is a  function which can  be used to  make elements  of the
'MC\''  basis of  the  Hecke  module associated  to  <I>  and the  sign
character.  In  this particular  case,  the  $MC^\prime$ basis  can  be
defined for  an reduced-$I$ element $w$  in terms of the  $MT$ basis by
$MC^\prime_w=C^\prime_w MT_1$.

|    gap> H:=Hecke(Wa,q^2);
    Hecke(~A2,q^2)
    gap> MC:=ModuleBasis(H,"MC'");
    #warning: MC\' basis: q chosen as 2nd root of q\^2
    function ( arg ) ... end|

%%%%%%%%%%%%%%%%%%%%%%%%%%%%%%%%%%%%%%%%%%%%%%%%%%%%%%%%%%%%%%%%%%%%%

\Section{Operations for Hecke module elements}

'+', '-':\\ one can add or subtract two Hecke module elements.

'<Basis>(<x>)':\\ this  call will convert  Hecke module element  <x> to
basis  '<Basis>'. With  the  same initializations  as  in the  previous
sections, we have\:

|    gap> MT:=ModuleBasis(H,"MT");;
    gap> MC(MT(1,2,3));
    -MC'()+qMC'(3)-q^2MC'(1,3)-q^2MC'(2,3)+q^3MC'(1,2,3)|

'\*':\\  one  can  multiply  on   the  left  an  Hecke  module  element
by  an  element of  the  corresponding  Hecke  algebra. With  the  same
initializations as in the previous sections, we have\:

|    gap> H:=Hecke(Wa,q);
    Hecke(~A2,q)
    gap> MT:=ModuleBasis(H,"MT");;
    gap> T:=Basis(H,"T");
    function ( arg ) ... end
    gap> T(1)*MT(1,2,3);
    qMT(2,3)+(q-1)MT(1,2,3)|

%%%%%%%%%%%%%%%%%%%%%%%%%%%%%%%%%%%%%%%%%%%%%%%%%%%%%%%%%%%%%%%%%%%%%

\Section{CreateHeckeModuleBasis}
\index{CreateHeckeModuleBasis}

'CreateHeckeModuleBasis(<basis>, <ops>, <algebraops>)'

This function is completely parallel to the function 'CreateHeckeBasis'.
See the description  of this last function. The only  difference is that
it is not '<ops>.T' which is  required to be bound, but '<ops>.MT' which
should contain  a function which takes  an element in the  basis <basis>
and converts it to the 'MT' basis.

%%%%%%%%%%%%%%%%%%%%%%%%%%%%%%%%%%%%%%%%%%%%%%%%%%%%%%%%%%%%%%%%%%%%%%%%%
