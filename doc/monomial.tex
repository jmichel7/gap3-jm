%%%%%%%%%%%%%%%%%%%%%%%%%%%%%%%%%%%%%%%%%%%%%%%%%%%%%%%%%%%%%%%%%%%%%%%%%%%%%
%%
%A  monomial.tex                GAP documentation               Thomas Breuer
%A                                                         & Erzsebet Horvath
%%
%A  @(#)$Id: monomial.tex,v 1.2 1997/01/21 14:26:27 gap Exp $
%%
%Y  Copyright 1993-1995,  Lehrstuhl D fuer Mathematik,  RWTH Aachen,  Germany
%%
%H  $Log: monomial.tex,v $
%H  Revision 1.2  1997/01/21 14:26:27  gap
%H  vfelsch added an example
%H
%H  Revision 1.1.1.1  1996/12/11 12:36:48  werner
%H  Preparing 3.4.4 for release
%H
%H  Revision 3.1.1.3  1995/12/04  07:23:13  vfelsch
%H  index entries moved to their proper places
%H
%H  Revision 3.1.1.2  1995/05/04  08:02:10  sam
%H  mentioned ``open question'' is no longer open
%H
%H  Revision 3.1.1.1  1994/09/22  10:35:08  sam
%H  added documentation of 'IsMonomial' for group orders
%H
%H  Revision 3.1  1994/06/10  02:50:42  vfelsch
%H  updated examples
%H
%H  Revision 3.0  1994/04/06  06:43:06  sam
%H  Initial Revision under RCS
%H
%%
\Chapter{Monomiality Questions}

This chapter describes functions dealing with monomiality questions.

Section "More about Monomiality Questions" gives some hints how to use
the functions in the package.

The next sections (see "Alpha", "Delta", "BergerCondition") describe
functions that deal with character degrees and derived length.

The next sections describe tests for homogeneous restriction,
quasiprimitivity, and induction from a normal subgroup of a group character
(see "TestHomogeneous", "TestQuasiPrimitive", "IsPrimitive for Characters",
"TestInducedFromNormalSubgroup").

The next sections describe tests for subnormally monomiality, monomiality,
and relatively subnormally monomiality of a group or group character
(see "TestSubnormallyMonomial", "TestMonomialQuick", "TestMonomial",
"TestRelativelySM").

The final sections "IsMinimalNonmonomial" and "MinimalNonmonomialGroup"
describe functions that construct minimal nonmonomial groups, or check
whether a group is minimal nonmonomial.

\vspace{3mm}

All examples in this chapter use the symmetric group $S_4$ and the
special linear group $Sl(2,3)$.  For running the examples, you must first
define the groups.

|    gap> S4:= SolvableGroup( "S4" );;
    gap> Sl23:= SolvableGroup( "Sl(2,3)" );; |

%%%%%%%%%%%%%%%%%%%%%%%%%%%%%%%%%%%%%%%%%%%%%%%%%%%%%%%%%%%%%%%%%%%%%%%%%%%%%
\Section{More about Monomiality Questions}

*Group Characters*

All the functions in this package assume *characters* to be character records
as described in chapter "Class Functions".

\vspace{5mm}

*Property Tests*

When we ask whether a group character $\chi$ has a certain property, like
quasiprimitivity, we usually want more information than yes or no.
Often we are interested in the reason why a group character $\chi$ could be
proved to have a certain property, e.g., whether monomiality of $\chi$
was proved by the observation that the underlying group is nilpotent, or 
if it was necessary to construct a linear character of a subgroup from that
$\chi$ can be induced.  In the latter case we also may be interested in this
linear character.

Because of this the usual property checks of {\GAP} that return either 'true'
or 'false' are not sufficient for us.
Instead there are test functions that return a record with the possibly
useful information.
For example, the record returned by the function 'TestQuasiPrimitive'
(see "TestQuasiPrimitive") contains the component 'isQuasiPrimitive' which is
the known boolean property flag, a component 'comment' which is a string
telling the reason for the value of the 'isQuasiPrimitive' component,
and in the case that the argument $\chi$ was a not quasiprimitive character
the component 'character' which is an irreducible constituent of a
nonhomogeneous restriction of $\chi$ to a normal subgroup.

The results of these test functions are stored in the respective records,
in our example $\chi$ will have a component 'testQuasiPrimitive' after
the call of 'TestQuasiPrimitive'.

Besides these test functions there are also the known property checks,
e.g., the function 'IsQuasiPrimitive' which will call 'TestQuasiPrimitive'
and return the value of the 'isQuasiPrimitive' component of the result.

\vspace{5mm}

*Where one should be careful*

Monomiality questions usually involve computations in a lot of subgroups
and factor groups of a given group, and for these groups often expensive
calculations like that of the character table are necessary.
If it is probable that the character table of a group will occur at
a later stage again, one should try to store the group (with the character
table stored in the group record) and use this record later rather than
a new record that describes the same group.

An example\:\ Suppose you want to restrict a character to a normal subgroup
$N$ that was constructed as a normal closure of some group elements, and
suppose that you have already computed normal subgroups (by calls to
'NormalSubgroups' or 'MaximalNormalSubgroups') and their character tables.
Then you should look in the lists of known normal subgroups whether $N$ is
contained, and if yes you can use the known character table.

A mechanism that supports this for normal subgroups is described in
"Storing Subgroup Information".  The following hint may be useful in this
context.

If you know that sooner or later you will compute the character table of
a group $G$ then it may be advisable to do this as soon as possible.
For example if you need the normal subgroups of $G$ then they can be
computed more efficiently if the character table of $G$ is known, and they
can be stored compatibly to the contained $G$-conjugacy classes.
This correspondence of classes list and normal subgroup can be used very
often.

\vspace{5mm}

*Package Information*

Some of the functions print (perhaps useful) information if the
function 'InfoMonomial' is set to the value 'Print'.

%%%%%%%%%%%%%%%%%%%%%%%%%%%%%%%%%%%%%%%%%%%%%%%%%%%%%%%%%%%%%%%%%%%%%%%%%%%%%
\Section{Alpha}

'Alpha( <G> )'

returns for a solvable group <G> a list whose <i>-th entry is the maximal
derived length of groups $<G>  / \ker(\chi)$ for $\chi\in Irr(<G>)$ with
$\chi(1)$ at most the <i>-th irreducible degree of <G>.

The result is stored in the group record as '<G>.alpha'.

*Note* that calling this function will cause the computation of factor groups
of <G>, so it works efficiently only for AG groups.

|    gap> Alpha( Sl23 );
    [ 1, 3, 3 ]
    gap> Alpha( S4 );
    [ 1, 2, 3 ] |

%%%%%%%%%%%%%%%%%%%%%%%%%%%%%%%%%%%%%%%%%%%%%%%%%%%%%%%%%%%%%%%%%%%%%%%%%%%%%
\Section{Delta}

'Delta( <G> )'

returns for a solvable group <G> the list
'[ 1, alp[2]-alp[1], ..., alp[<n>]-alp[<n>-1] ]'
where 'alp = Alpha( <G> )' (see "Alpha").

|    gap> Delta( Sl23 );
    [ 1, 2, 0 ]
    gap> Delta( S4 );
    [ 1, 1, 1 ] |
    
%%%%%%%%%%%%%%%%%%%%%%%%%%%%%%%%%%%%%%%%%%%%%%%%%%%%%%%%%%%%%%%%%%%%%%%%%%%%%
\Section{BergerCondition}

'BergerCondition( <chi> )'\\
'BergerCondition(  <G>  )'

Called with an irreducible character <chi> of the group $G$ of degree $d$,
'BergerCondition' returns 'true' if <chi> satisfies
$M^{\prime} \leq \ker(\chi)$ for every normal subgroup $M$ of $G$ with the
property that $M \leq \ker(\psi)$ for all $\psi \in Irr(G)$ with
$\psi(1) \< \chi(1)$, and 'false' otherwise.

Called with a group <G>, 'BergerCondition' returns 'true' if all irreducible
characters of <G> satisfy the inequality above, and 'false' otherwise;
in the latter case 'InfoMonomial' tells about the smallest degree for that the
inequality is violated.

For groups of odd order the answer is always 'true' by a theorem of
T.~R.~Berger (see~\cite{Ber76}, Thm.~2.2).

|    gap> BergerCondition( S4 );
    true
    gap> BergerCondition( Sl23 );
    false
    gap> List( Irr( Sl23 ), BergerCondition );
    [ true, true, true, false, false, false, true ]
    gap> List( Irr( Sl23 ), Degree );
    [ 1, 1, 1, 2, 2, 2, 3 ] |

%%%%%%%%%%%%%%%%%%%%%%%%%%%%%%%%%%%%%%%%%%%%%%%%%%%%%%%%%%%%%%%%%%%%%%%%%%%%%
\Section{TestHomogeneous}

'TestHomogeneous( <chi>, <N> )'

returns a record with information whether the restriction of the character
<chi> of the group $G$ to the normal subgroup <N> of $G$ is homogeneous,
i.e., is a multiple of an irreducible character of <N>.

<N> may be given also as list of conjugacy class positions w.r. to $G$.

The components of the result are

'isHomogeneous': \\  'true' or 'false',

'comment':       \\  a string telling a reason for the value of the
                     'isHomogeneous' component,

'character':     \\  irreducible constituent of the restriction, only
                     bound if the restriction had to be checked,

'multiplicity':  \\  multiplicity of the 'character' component in the
                     restriction of <chi>.


|    gap> chi:= Irr( Sl23 )[4];
    Character( Sl(2,3), [ 2, -2, 0, -1, 1, -1, 1 ] )
    gap> n:= NormalSubgroupClasses( Sl23, [ 1, 2, 3 ] );
    Subgroup( Sl(2,3), [ b, c, d ] )
    gap> TestHomogeneous( chi, [ 1, 2, 3 ] );
    rec(
      isHomogeneous := true,
      comment := "restricts irreducibly" )
    gap> chi:= Irr( Sl23 )[7];
    Character( Sl(2,3), [ 3, 3, -1, 0, 0, 0, 0 ] )
    gap> TestHomogeneous( chi, n );
    &W  Warning: Group has no name
    rec(
      isHomogeneous := false,
      comment := "restriction checked",
      character := Character( Subgroup( Sl(2,3), [ b, c, d ] ),
        [ 1, 1, -1, 1, -1 ] ),
      multiplicity := 1 ) |

%%%%%%%%%%%%%%%%%%%%%%%%%%%%%%%%%%%%%%%%%%%%%%%%%%%%%%%%%%%%%%%%%%%%%%%%%%%%%
\Section{TestQuasiPrimitive}

'TestQuasiPrimitive( <chi> )'

returns a record with information about quasiprimitivity of the character
<chi> of the group $G$ (i.e., whether <chi> restricts homogeneously
to every normal subgroup of $G$).

The record contains the components

'isQuasiPrimitive':  \\ 'true' or 'false',

'comment':           \\  a string telling a reason for the value of the
                         'isQuasiPrimitive' component,

'character':  \\ an irreducible constituent of a nonhomogeneous
                 restriction of <chi>, bound only if <chi> is not
                 quasi-primitive.

\vspace{5mm}

'IsQuasiPrimitive( <chi> )'%
\index{IsQuasiPrimitive}

returns 'true' or 'false', depending on whether the character <chi> of the
group $G$ is quasiprimitive.

|    gap> chi:= Irr( Sl23 )[4];
    Character( Sl(2,3), [ 2, -2, 0, -1, 1, -1, 1 ] )
    gap> TestQuasiPrimitive( chi );
    &W  Warning: Group has no name
    rec(
      isQuasiPrimitive := true,
      comment := "all restrictions checked" )
    gap> chi:= Irr( Sl23 )[7];
    Character( Sl(2,3), [ 3, 3, -1, 0, 0, 0, 0 ] )
    gap> TestQuasiPrimitive( chi );
    rec(
      isQuasiPrimitive := false,
      comment := "restriction checked",
      character := Character( Subgroup( Sl(2,3), [ b, c, d ] ), 
        [ 1, 1, -1, 1, -1 ] ) ) |

%%%%%%%%%%%%%%%%%%%%%%%%%%%%%%%%%%%%%%%%%%%%%%%%%%%%%%%%%%%%%%%%%%%%%%%%%%%%%
\Section{IsPrimitive for Characters}

'IsPrimitive( <chi> )'

returns 'true' if the irreducible character <chi> of the solvable group $G$
is not induced from any proper subgroup of $G$, and 'false' otherwise.

*Note* that an irreducible character of a solvable group is primitive if and
only if it is quasi-primitive (see "TestQuasiPrimitive").

|    gap> IsPrimitive( Irr( Sl23 )[4] );
    true
    gap> IsPrimitive( Irr( Sl23 )[7] );
    false |

%%%%%%%%%%%%%%%%%%%%%%%%%%%%%%%%%%%%%%%%%%%%%%%%%%%%%%%%%%%%%%%%%%%%%%%%%%%%%
\Section{TestInducedFromNormalSubgroup}

'TestInducedFromNormalSubgroup( <chi>, <N> )'\\
'TestInducedFromNormalSubgroup( <chi> )'

returns a record with information about whether the irreducible character
<chi> of the group $G$ is induced from a *proper* normal subgroup of $G$.

If <chi> is the only argument then it is checked whether there is a
maximal normal subgroup of $G$ from that <chi> is induced.  If there
is a second argument <N>, a normal subgroup of $G$, then it is checked
whether <chi> is induced from <N>.
<N> may also be given as the list of positions of conjugacy classes
contained in the normal subgroup in question.

The result contains the components

'isInduced': \\ 'true' or 'false',

'comment':   \\ a string telling a reason for the value of the
                'isInduced' component,

'character': \\ if bound, a character of a maximal normal subgroup of $G$
                or of the argument <N> from that <chi> is induced.

\vspace{5mm}

'IsInducedFromNormalSubgroup( <chi> )'

returns 'true' if the group character <chi> is induced from a *proper*
normal subgroup of the group of <chi>, and 'false' otherwise.

|    gap> List( Irr( Sl23 ), IsInducedFromNormalSubgroup );
    [ false, false, false, false, false, false, true ]
    gap> List( Irr( S4 ){ [ 1, 3, 4 ] },
    >          TestInducedFromNormalSubgroup );
    &W  Warning: Group has no name
    [ rec(
          isInduced := false,
          comment := "linear character" ), rec(
          isInduced := true,
          comment := "induced from component '.character'",
          character := Character( Subgroup( S4, [ b, c, d ] ), 
            [ 1, 1, E(3), E(3)^2 ] ) ), rec(
          isInduced := false,
          comment := "all maximal normal subgroups checked" ) ] |

%%%%%%%%%%%%%%%%%%%%%%%%%%%%%%%%%%%%%%%%%%%%%%%%%%%%%%%%%%%%%%%%%%%%%%%%%%%%%
\Section{TestSubnormallyMonomial}

'TestSubnormallyMonomial( <G> )'\\
'TestSubnormallyMonomial( <chi> )'

returns a record with information whether the group <G> or the
irreducible group character <chi> of the group $G$ is subnormally
monomial.

The result contains the components

'isSubnormallyMonomial': \\ 'true' or 'false',

'comment':   \\ a string telling a reason for the value of the
                'isSubnormallyMonomial' component,

'character': \\ if bound, a character of <G> that is not
                subnormally monomial.

\vspace{5mm}

'IsSubnormallyMonomial( <G> )'%
\index{IsSubnormallyMonomial}\\
'IsSubnormallyMonomial( <chi> )'

returns 'true' if the group <G> or the group character <chi> is
subnormally monomial, and 'false' otherwise.

|    gap> TestSubnormallyMonomial( S4 );
    rec(
      isSubnormallyMonomial := false,
      character := Character( S4, [ 3, -1, 0, -1, 1 ] ),
      comment := "found not SM character" )
    gap> TestSubnormallyMonomial( Irr( S4 )[4] );
    rec(
      isSubnormallyMonomial := false,
      comment := "all subnormal subgroups checked" )
    gap> TestSubnormallyMonomial( SolvableGroup( "A4" ) );
    &W  Warning: Group has no name
    rec(
      isSubnormallyMonomial := true,
      comment := "all irreducibles checked" ) |

%%%%%%%%%%%%%%%%%%%%%%%%%%%%%%%%%%%%%%%%%%%%%%%%%%%%%%%%%%%%%%%%%%%%%%%%%%%%%
\Section{TestMonomialQuick}

'TestMonomialQuick( <chi> )'\\
'TestMonomialQuick(  <G>  )'

does some easy checks whether the irreducible character <chi> or the group
<G> are monomial.  'TestMonomialQuick' returns a record with components

'isMonomial': \\ either 'true' or 'false' or the string '\"?\"', depending
                 on whether (non)monomiality could be proved, and

'comment':    \\ a string telling the reason for the value of the
                 'isMonomial' component.

A group $G$ is proved to be monomial by 'TestMonomialQuick' if
its order is not divisible by the third power of a prime, or if $G$ is
nilpotent or Sylow abelian by supersolvable.  Nonsolvable groups are
proved to me nonmonomial by 'TestMonomialQuick'.

An irreducible character is proved to be monomial if it is linear, or if
its codegree is a prime power, or if its group knows to be monomial, or
if the factor group modulo the kernel can be proved to be monomial by
'TestMonomialQuick'.

|    gap> TestMonomialQuick( Irr( S4 )[3] );
    rec(
      isMonomial := true,
      comment := "kernel factor group is supersolvable" )
    gap> TestMonomialQuick( S4 );
    rec(
      isMonomial := true,
      comment := "abelian by supersolvable group" )
    gap> TestMonomialQuick( Sl23 );
    rec(
      isMonomial := "?",
      comment := "no decision by cheap tests" ) |

%%%%%%%%%%%%%%%%%%%%%%%%%%%%%%%%%%%%%%%%%%%%%%%%%%%%%%%%%%%%%%%%%%%%%%%%%%%%%
\Section{TestMonomial}

'TestMonomial( <chi> )'\\
'TestMonomial(  <G>  )'

returns a record containing information about monomiality of the group <G>
or the group character <chi> of a solvable group, respectively.

If a character <chi> is proved to be monomial the result contains
components 'isMonomial' (then 'true'), 'comment' (a string telling a
reason for monomiality), and if it was necessary to compute a linear
character from that <chi> is induced, also a component 'character'.

If <chi> or <G> is proved to be nonmonomial the component 'isMonomial'
is 'false', and in the case of <G> a nonmonomial character is contained
in the component 'character' if it had been necessary to compute it.

If the program cannot prove or disprove monomiality then the result
record contains the component 'isMonomial' with value '\"?\"'.

This case occurs in the call for a character <chi> if and only if
<chi> is not induced from the inertia subgroup of a component of any
reducible restriction to a normal subgroup.  It can happen that <chi>
is monomial in this situation.

For a group this case occurs if no irreducible character can be proved
to be nonmonomial, and if no decision is possible for at least one
irreducible character.

\vspace{3mm}

'IsMonomial( <G> )'%
\index{IsMonomial!for groups}\\
'IsMonomial( <chi> )'%
\index{IsMonomial!for characters}

returns 'true' if the group <G> or the character <chi> of a solvable group
can be proved to be monomial, 'false' if it can be proved to be nonmonomial,
and the string '\"?\"' otherwise.

|    gap> TestMonomial( S4 );
    rec(
      isMonomial := true,
      comment := "abelian by supersolvable group" )
    gap> TestMonomial( Sl23 );
    rec(
      isMonomial := false,
      comment := "list Delta( G ) contains entry > 1" ) |

\vspace{3mm}

'IsMonomial( <n> )'%
\index{IsMonomial!for group orders}

for a positive integer <n> returns 'true' if every solvable group of order
<n> is monomial, and 'false' otherwise.

|    gap> Filtered( [ 1 .. 111 ], x -> not IsMonomial( x ) );
    [ 24, 48, 72, 96, 108 ] |

%%%%%%%%%%%%%%%%%%%%%%%%%%%%%%%%%%%%%%%%%%%%%%%%%%%%%%%%%%%%%%%%%%%%%%%%%%%%%
\Section{TestRelativelySM}

'TestRelativelySM( <G> )'\\
'TestRelativelySM( <chi>, <N> )'

If the only argument is a SM group <G> then 'TestRelativelySM' returns
a record with information about whether <G> is relatively subnormally
monomial (relatively SM) with respect to every normal subgroup.

If there are two arguments, an irreducible character <chi> of a SM group $G$
and a normal subgroup <N> of $G$, then 'TestRelativelySM' returns a record
with information whether <chi> is relatively SM with respect to <N>, i.e,
whether there is a subnormal subgroup $H$ of $G$ that contains <N> such that
<chi> is induced from a character $\psi$ of $H$ where the restriction of
$\psi$ to <N> is irreducible.

The component 'isRelativelySM' is 'true' or 'false', the component
'comment' contains a string that describes the reason.
If the argument is <G>, and <G> is not relatively SM with respect to
a normal subgroup then the component 'character' contains a not
relatively SM character of such a normal subgroup.

*Note*\:\ It is not checked whether $G$ is SM.

|    gap> IsSubnormallyMonomial( SolvableGroup( "A4" ) );
    &W  Warning: Group has no name
    true
    gap> TestRelativelySM( SolvableGroup( "A4" ) );
    rec(
      isRelativelySM := true,
      comment :=
       "normal subgroups are abelian or have nilpotent factor group" ) |

%%%%%%%%%%%%%%%%%%%%%%%%%%%%%%%%%%%%%%%%%%%%%%%%%%%%%%%%%%%%%%%%%%%%%%%%%%%%%
\Section{IsMinimalNonmonomial}

'IsMinimalNonmonomial( <G> ) '

returns 'true' if the solvable group <G> is a minimal nonmonomial group,
and 'false' otherwise.
A group is called *minimal nonmonomial* if it is nonmonomial,
and all proper subgroups and factor groups are monomial.

The solvable minimal nonmonomial groups were classified by van der Waall
(see~\cite{vdW76}).

|    gap> IsMinimalNonmonomial( Sl23 );
    true
    gap> IsMinimalNonmonomial( S4 );
    false |

%%%%%%%%%%%%%%%%%%%%%%%%%%%%%%%%%%%%%%%%%%%%%%%%%%%%%%%%%%%%%%%%%%%%%%%%%%%%%
\Section{MinimalNonmonomialGroup}

'MinimalNonmonomialGroup( <p>, <factsize> )'

returns a minimal nonmonomial group described by the parameters
<factsize> and <p> if such a group exists, and 'false' otherwise.

Suppose that a required group $K$ exists.
<factsize> is the size of the Fitting factor $K / F(K)$; this value must
be 4, 8, an odd prime, twice an odd prime, or four times an odd prime.

In the case that <factsize> is twice an odd prime the centre $Z(K)$ iscyclic
of order $2^{p+1}$.  In all other cases <p> denotes the (unique) prime that
divides the order of $F(K)$.

The solvable minimal nonmonomial groups were classified by van der Waall
(see~\cite{vdW76}, the construction follows this article).

|    gap> MinimalNonmonomialGroup(  2,  3 ); # $SL_2(3)$
    2^(1+2):3
    gap> MinimalNonmonomialGroup(  3,  4 );
    3^(1+2):4
    gap> MinimalNonmonomialGroup(  5,  8 );
    5^(1+2):Q8
    gap> MinimalNonmonomialGroup( 13, 12 );
    13^(1+2):2.D6
    gap> MinimalNonmonomialGroup(  1, 14 );
    2^(1+6):D14
    gap> MinimalNonmonomialGroup(  2, 14 );
    (2^(1+6)Y4):D14 |

