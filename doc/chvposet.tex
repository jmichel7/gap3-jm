%%%%%%%%%%%%%%%%%%%%%%%%%%%%%%%%%%%%%%%%%%%%%%%%%%%%%%%%%%%%%%%%%%%%%%%%%%%%%
%%
%A  chvposet.tex       CHEVIE documentation       Jean Michel
%%
%Y  Copyright (C) 2010 - 1013  University  Paris VII.
%%
%%  This  file  contains  useful functions to deal with posets.
%%%%%%%%%%%%%%%%%%%%%%%%%%%%%%%%%%%%%%%%%%%%%%%%%%%%%%%%%%%%%%%%%%%%%%%%%

\Chapter{Posets and relations}

Posets are represented in \CHEVIE\ as records where at least one of the two
following fields is present\:

  |.incidence|:  a  boolean  matrix  such  that |.incidence[i][j]=true| iff
  |i<=j| in the poset.

  |.hasse|: a list representing the Hasse diagram of the poset\: the $i$-th
  entry  is the list of indices  of elements which are immediate successors
  (covers)  of the $i$-th element, that is  the list of $j$ such that |i<j|
  and such that there is no $k$ such that |i<k<j|.

If  only one field is present, the other  is computed on demand. Here is an
example of use;

|    gap> P:=BruhatPoset(CoxeterGroup("A",2));
    Poset with 6 elements
    gap> Display(P);
    <1,2<21,12<121
    gap> Hasse(P);
    [ [ 2, 3 ], [ 4, 5 ], [ 4, 5 ], [ 6 ], [ 6 ], [  ] ]
    gap> Incidence(P);   
    [ [ true, true, true, true, true, true ], 
      [ false, true, false, true, true, true ], 
      [ false, false, true, true, true, true ], 
      [ false, false, false, true, false, true ], 
      [ false, false, false, false, true, true ], 
      [ false, false, false, false, false, true ] ]|

%%%%%%%%%%%%%%%%%%%%%%%%%%%%%%%%%%%%%%%%%%%%%%%%%%%%%%%%%%%%%%%%%%%%%%%%%
\Section{TransitiveClosure of incidence matrix}%
\index{TransitiveClosure}

'TransitiveClosure(<M>)'

<M>  should be a  square boolean matrix  representing a relation; returns a
boolean  matrix representing the  transitive closure of  this relation. The
transitive  closure is computed  by the Floyd-Warshall  algorithm, which is
quite fast even for large matrices.

|    gap> M:=List([1..5],i->List([1..5],j->j-i in [0,1]));
    [ [ true, true, false, false, false ], 
      [ false, true, true, false, false ], 
      [ false, false, true, true, false ], 
      [ false, false, false, true, true ], 
      [ false, false, false, false, true ] ]
    gap> PrintArray(TransitiveClosure(M));
    [[ true,  true,  true,  true,  true],
     [false,  true,  true,  true,  true],
     [false, false,  true,  true,  true],
     [false, false, false,  true,  true],
     [false, false, false, false,  true]]|

%%%%%%%%%%%%%%%%%%%%%%%%%%%%%%%%%%%%%%%%%%%%%%%%%%%%%%%%%%%%%%%%%%%%%%%%%
\Section{LcmPartitions}%
\index{LcmPartitions}

'LcmPartitions(<p1>,...,<pn>)'
Each  argument is a partition of the same set 'S', represented by a list of
disjoint  subsets whose union is 'S'. Equivalently each argument represents
an equivalence relation on 'S'.

The result is the finest partition of 'S' such that each argument partition
refines it. It represents the 'or' of the equivalence relations represented
by the arguments.

|    gap> LcmPartitions([[1,2],[3,4],[5,6]],[[1],[2,5],[3],[4],[6]]);
    [ [ 1, 2, 5, 6 ], [ 3, 4 ] ]|

%%%%%%%%%%%%%%%%%%%%%%%%%%%%%%%%%%%%%%%%%%%%%%%%%%%%%%%%%%%%%%%%%%%%%%%%%
\Section{GcdPartitions}%
\index{GcdPartitions}

'GcdPartitions(<p1>,...,<pn>)'
Each  argument is a partition of the same set 'S', represented by a list of
disjoint  subsets whose union is 'S'. Equivalently each argument represents
an equivalence relation on 'S'.

The result is the coarsest partition which refines all argument partitions.
It  represents the  'and' of  the equivalence  relations represented by the
arguments.

|    gap> GcdPartitions([[1,2],[3,4],[5,6]],[[1],[2,5],[3],[4],[6]]);
    [ [ 1 ], [ 2 ], [ 3 ], [ 4 ], [ 5 ], [ 6 ] ]|

%%%%%%%%%%%%%%%%%%%%%%%%%%%%%%%%%%%%%%%%%%%%%%%%%%%%%%%%%%%%%%%%%%%%%%%%%
\Section{Poset}%
\index{Poset}

'Poset(<M>)'

'Poset(<H>)'

Creates   a  poset   from  either   an  incidence   matrix  <M>  such  that
|M[i][j]=true|  if and only if |i<=j| in  the poset, or a Hasse diagram <H>
given as a list whose $i$-th entry is the list of indices of elements which
are  immediate successors (covers) of the $i$-th element, that is |M[i]| is
the  list of $j$ such that |i<j| in the poset and such that there is no $k$
such that |i<k<j|.

'Poset(arg)'

In  this last form |arg[1]|  should be a record  with a field |.operations|
and the functions calls |ApplyFunc(arg[1].operations.Poset,arg)|.

A poset is represented as a record with the following fields.

  |.incidence|:  the incidence matrix.

  |.hasse|: the Hasse diagram.

Since the cost of computing one from the other is high, the above fields are
optional (only one of them needs to be present) and the other is computed on
demand.

  |.size|: the number of elements of the poset.

Finally,  an optional field |.label| may be given for formatting or display
purposes.  It should be  a function |label(P,i,opt)|  which returns a label
for the |i|-th element of the poset |P|, formatted according to the options
(if any) given in the options record |opt|.

%%%%%%%%%%%%%%%%%%%%%%%%%%%%%%%%%%%%%%%%%%%%%%%%%%%%%%%%%%%%%%%%%%%%%%%%%
\Section{Hasse}%
\index{Hasse}

'Hasse(<P>)'

returns the Hasse diagram of the poset <P>.

|    gap> p:=Poset(List([1..5],i->List([1..5],j->j mod i=0)));                  
    Poset with 5 elements
    gap> Hasse(p);
    [ [ 2, 3, 5 ], [ 4 ], [  ], [  ], [  ] ]|

%%%%%%%%%%%%%%%%%%%%%%%%%%%%%%%%%%%%%%%%%%%%%%%%%%%%%%%%%%%%%%%%%%%%%%%%%
\Section{Incidence}%
\index{Incidence}

'Incidence(<P>)'

returns the Incidence matrix of the poset <P>.

|    gap> p:=Poset(Concatenation(List([1..5],i->[i+1]),[[]]));
    Poset with 6 elements
    gap> Incidence(p);
    [ [ true, true, true, true, true, true ], 
      [ false, true, true, true, true, true ], 
      [ false, false, true, true, true, true ], 
      [ false, false, false, true, true, true ], 
      [ false, false, false, false, true, true ], 
      [ false, false, false, false, false, true ] ]|

%%%%%%%%%%%%%%%%%%%%%%%%%%%%%%%%%%%%%%%%%%%%%%%%%%%%%%%%%%%%%%%%%%%%%%%%%
\Section{LinearExtension}%
\index{LinearExtension}

'LinearExtension(<P>)'

returns  a linear extension of the poset <P>, that is a list |l| containing
a  permutation of  the integers  |[1..Size(P)]| such  that if |i<j| in <P>,
then |Position(l,i)<Position(l,j)|. This is also called a topological sort
of <P>.

|    gap> p:=Poset(List([1..5],i->List([1..5],j->j mod i=0))); 
    Poset with 5 elements
    gap> Display(p);
    1<2<4
    1<3,5
    gap> LinearExtension(p);
    [ 1, 2, 3, 5, 4 ]|

%%%%%%%%%%%%%%%%%%%%%%%%%%%%%%%%%%%%%%%%%%%%%%%%%%%%%%%%%%%%%%%%%%%%%%%%%
\Section{Functions for Posets}%

The function 'Size' returns the number of elements of the poset.

The functions 'String' and 'Print' just indicate the |Size| of the poset.

The  functions 'Format' and 'Display'  show the poset as  a list of maximal
covering chains, with formatting depending on their record of options. They
take  in account the  associated partition (see  "Partition for posets") to
give  a  more  compact  description  where  equivalent  elements are listed
together, separated by commas.

|    gap> p:=Poset(UnipotentClasses(ComplexReflectionGroup(28)));
    Poset with 16 elements
    gap> Display(p); 
    1<A1<~A1<A1+~A1<A2<A2+~A1<~A2+A1<C3(a1)<F4(a3)<C3,B3<F4(a2)<F4(a1)<F4
    A1+~A1<~A2<~A2+A1
    A2+~A1<B2<C3(a1)|

%%%%%%%%%%%%%%%%%%%%%%%%%%%%%%%%%%%%%%%%%%%%%%%%%%%%%%%%%%%%%%%%%%%%%%%%%
\Section{Partition for posets}%
\index{Partition}%

'Partition(<P>)'

returns  the  partition  of  |[1..Size(P)]|  determined  by the equivalence
relation  associated to <P>; that  is, |i| and |j|  are in the same part of
the  partition if the relations |i<k| and |j<k| as well are |k<i| and |k<j|
are equivalent for any |k| in the poset.

|    gap> p:=Poset(List([1..8],i->List([1..8],j->i=j or (i mod 4)<(j mod 4))));
    Poset with 8 elements
    gap> Display(p);
    4,8<1,5<2,6<3,7
    gap> Partition(p);
    [ [ 4, 8 ], [ 2, 6 ], [ 3, 7 ], [ 1, 5 ] ]|

%%%%%%%%%%%%%%%%%%%%%%%%%%%%%%%%%%%%%%%%%%%%%%%%%%%%%%%%%%%%%%%%%%%%%%%%%
\Section{Restricted for Posets}%
\index{Restricted}
\index{Poset!Restricted}

Restricted(<P>,<indices>)

returns the sub-poset of <P> determined by <indices>, which must be a sublist
of |[1..Size(P)]|.

|    gap> Display(p);
    4,8<1,5<2,6<3,7
    gap> Display(Restricted(p,[2..6]));
    3<4<1,5<2|

%%%%%%%%%%%%%%%%%%%%%%%%%%%%%%%%%%%%%%%%%%%%%%%%%%%%%%%%%%%%%%%%%%%%%%%%%
