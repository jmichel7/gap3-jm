%%%%%%%%%%%%%%%%%%%%%%%%%%%%%%%%%%%%%%%%%%%%%%%%%%%%%%%%%%%%%%%%%%%%%%%%%%%%%
%%
%A  saggroup.tex                GAP documentation                Bettina Eick
%%
%A  @(#)$Id: saggroup.tex,v 1.3 1997/01/21 15:40:19 gap Exp $
%%
%Y  Copyright (C)  1994,  Lehrstuhl D fuer Mathematik,  RWTH Aachen,  Germany
%%
%%  This file contains descriptions of special aggroups.
%%
%H  $Log: saggroup.tex,v $
%H  Revision 1.3  1997/01/21 15:40:19  gap
%H  vfelsch fixed file name
%H
%H  Revision 1.2  1997/01/21 09:00:54  gap
%H  Changed everything about SystemNormalizer
%H
%H  Revision 1.1.1.1  1996/12/11 12:36:50  werner
%H  Preparing 3.4.4 for release
%H
%H  Revision 3.2  1994/06/10  04:03:08  vfelsch
%H  updated examples
%H
%H  Revision 3.1  1994/06/03  12:19:36  mschoene
%H  initial revision under RCS
%H
%%


%%%%%%%%%%%%%%%%%%%%%%%%%%%%%%%%%%%%%%%%%%%%%%%%%%%%%%%%%%%%%%%%%%%%%%%%%
\Chapter{Special Ag Groups}

*Special ag groups* are a subcategory of ag groups (see "Finite
Polycyclic Groups").

Let $G$ be an ag group with PAG system $(g_1, \ldots, g_n)$.
Then $(g_1, \ldots, g_n)$ is a *special ag system* if it is an ag system
with some additional properties, which are described below.

In general a finite polycyclic group has several different ag systems and
at least one of this is a special ag system, but in {\GAP} an ag group is
defined by a fixed ag system and according to this an ag group is called
a special ag group if its ag system is a special ag system.

Special ag systems give more information about their corresponding group
than arbitrary ag systems do (see "More about Special Ag Groups") and
furthermore there are many algorithms, which are much more efficient for
a special ag group than for an arbitrary one. (See "Ag Group Functions
for Special Ag Groups")

The following sections describe the special ag system (see "More about
Special Ag Groups"), their construction in {\GAP} (see "Construction of
Special Ag Groups" and "Restricted Special Ag Groups") and their
additional record entries (see "Special Ag Group Records"). Then follow
two sections with functions which do only work for special ag groups (see
"MatGroupSagGroup" and "DualMatGroupSagGroup").

%%%%%%%%%%%%%%%%%%%%%%%%%%%%%%%%%%%%%%%%%%%%%%%%%%%%%%%%%%%%%%%%%%%%%%%%%
\Section{More about Special Ag Groups}

Now the properties of a special ag system are described.
First of all the *Leedham-Green series* will be introduced.

Let $G = G_1 > G_2 > \ldots > G_m > G_{m+1} = \{ 1 \}$ be the *lower
nilpotent series* of $G$, i.e., $G_i$ is the smallest normal subgroup of
$G_{i-1}$ such that $G_{i-1} / G_i$ is nilpotent.

To refine this series the *lower elementary abelian series* of a
nilpotent group $N$ will be constructed.
Let $N = P_1 \cdot \ldots \cdot P_l$ be the direct product of its
Sylow-subgroups such that $P_h$ is a $p_h$-group and
$p_1 \< p_2 \< \ldots \< p_l$ holds.  Let $\lambda_{j}(P_h)$ be the
$j$-th term of the $p_h$-central series of $P_h$ and let $k_h$ be the
length of this series (see "PCentralSeries").
Define $N_{j, p_h}$ as the subgroup of $N$ with
$$
  N_{j, p_h} = \lambda_{j+1}(P_1) \cdots \lambda_{j+1}(P_{h-1}) \cdot
               \lambda_j(P_h) \cdots \lambda_j(P_l).
$$
With $k = $max$\{k_1, \ldots, k_l\}$ the series
$$
  N = N_{1, p_1} \geq N_{1, p_2} \geq \ldots \geq N_{1,p_l}
      \geq N_{2, p_1} \geq \ldots \geq N_{k, p_l} = \{ 1 \}
$$
is obtained. Since the $p$-central series may have different lengths for
different primes, some subgroups might be equal.  The lower elementary
abelian series is obtained, if for all pairs of equal subgroups the one
with the lexicographically greater index is removed.  This series is a
characteristic central series with maximal elementary abelian factors.

To get the Leedham-Green series of $G$, each factor of the lower
nilpotent series of $G$ is refined by its lower elementary abelian
series. The subgroups of the Leedham-Green series are denoted by
$G_{i, j, p_{i, h}}$ such that $G_{i, j, p_{i, h}} / G_{i+1} =
(G_i / G_{i+1})_{j, p_{i,h}}$ for each prime $p_{i,h}$ dividing the order
of $G_i / G_{i+1}$.  The Leedham-Green series is a characteristic series
with elementary abelian factors.

A PAG system corresponds naturally to a composition series of its group.
The first additional property of a special ag system is that the
corresponding composition series refines the Leedham-Green series.

Secondly, all the elements of a special ag system are of prime-power
order, and furthermore, if a set of primes $\pi = \{q_1, \ldots, q_r\}$
is given, all elements of a special ag system which are of $q_h$-power
order for some $q_h$ in $\pi$ generate a Hall-$\pi$-subgroup of $G$. In
fact they form a canonical generating sequence of the
Hall-$\pi$-subgroup.  These Hall subgroups are called *public subgroups*,
since a subset of the PAG system is an induced generating set for the
subgroup. Note that the set of all public Sylow subgroups forms a Sylow
system of $G$.

The last property of the special ag systems is the existence of public
*local head complements*.  For a nilpotent group $N$, the group
$$
    \lambda_2(N) = \lambda_2(P_1) \cdots \lambda_2(P_l)
$$
is the Frattini subgroup of $N$.  The *local heads* of the group $G$ are
the factors
$$
  (G_i / G_{i+1}) / \lambda_2(G_i / G_{i+1}) = G_i / G_{i, 2,p_{i,1}}
$$
for each $i$.  A local head complement is a subgroup $K_i$ of $G$ such
that
$K_i / G_{i,2,p_{i,1}}$ is a complement  of  $G_i / G_{i,  2, p_{i  1}}$.
Now a special ag system has a public local head complement for each local
head. This complement is generated by the elements of the special ag
system which do not lie in $G_i \backslash G_{i,2,p_{i,1}}$. Note that
all complements of a local head are conjugate. The factors
$$
  \lambda_2(G_i / G_{i+1}) = G_{i, 2,p_{i,1}} / G_{i+1}
$$
are called  the *tails* of the group $G$. 

To handle the special ag system the *weights* are introduced.
Let $(g_1, \ldots, g_n)$ be a special ag system.
The triple $(w_1, w_2, w_3)$ is called the weight of the generator $g_i$
if $g_i$ lies in $G_{w_1, w_2, w_3}$ but not lower down in the
Leedham-Green series. That means $w_1$ corresponds to the subgroup in the
lower nilpotent series and $w_2$ to the subgroup in the
elementary-abelian series of this factor, and $w_3$ is the prime dividing
the order of $g_i$. Then $weight(g_i) = (w_1, w_2, w_3)$ and
$weight_j(g_i) = w_j$ for $j = 1,2,3$ is set.
With this definition $\{g_i \| weight_3(g_i) \in \pi\}$ is a
Hall-$\pi$-subgroup of $G$ and
$\{g_i \| weight(g_i) \neq (j, 1, p) \mbox{ for some } p \}$ is a local
head complement.

Now some advantages of a special ag system are summarized.

\begin{itemize}
\item[1.]
You have a characteristic series with elementary abelian factors of $G$
explicitly given in the ag system. This series is refined by the
composition series corresponding to the ag system.

\item[2.]
You can see whether $G$ is nilpotent or even a p-group, and if it is, you
have a central series explicitly given by the Leedham-Green
series. Analogously you can see whether the group is even elementary
abelian.

\item[3.]
You can easily calculate Hall-$\pi$-subgroups of $G$. Furthermore the set
of public Sylow subgroups forms a Sylow system.

\item[4.]
You get a smaller generating set of the group by taking only the elements
which correspond to local heads of the group.

\item[5.]
The collection with a special ag system may be faster than the collection
with an arbitrary ag system, since in the calculation of the public
subgroups of $G$ the commutators of the ag generators are shortened.

\item[6.]
Many algorithms are faster for special ag groups than for arbitrary ag
groups.

\end{itemize}

%%%%%%%%%%%%%%%%%%%%%%%%%%%%%%%%%%%%%%%%%%%%%%%%%%%%%%%%%%%%%%%%%%%%%%%%%%%%%
\Section{Construction of Special Ag Groups}

'SpecialAgGroup( <G> )' 

The function 'SpecialAgGroup' takes an ag group $G$ as input and
calculates a special ag group $H$, which is isomorphic to $G$.

To obtain the additional information of a special ag system see "Special
Ag Group Records".

%%%%%%%%%%%%%%%%%%%%%%%%%%%%%%%%%%%%%%%%%%%%%%%%%%%%%%%%%%%%%%%%%%%%%%%%%%%%%
\Section{Restricted Special Ag Groups}

If one is only interested in some of the information of special ag
systems then it is possible to suppress the calculation of one or all
types of the public subgroups by calling the function
'SpecialAgGroup( <G>, <flag> )', where <flag> is \"noHall\", \"noHead\" or 
\"noPublic\". 
With this options the algorithm takes less time. It calculates an ag
group $H$, which is isomorphic to $G$. But be careful, because the output
$H$ is not handled as a special ag group by {\GAP} but as an arbitrary ag
group.  Exspecially none of the functions listet in "Ag Group Functions
for Special Ag Groups" use the algorithms for special ag groups.

\vspace{.5cm}
'SpecialAgGroup( <G>, \"noPublic\" )' 

calculates an ag group $H$, which is isomorphic to $G$ and whose ag
system is corresponding to the Leedham-Green series.

\vspace{.5cm}
'SpecialAgGroup( <G>, \"noHall\" )' 

calculates an ag group $H$, which is isomorphic to $G$ and whose ag
system is corresponding to the Leedham-Green series and has public local
head complements.

\vspace{.5cm}
'SpecialAgGroup( <G>, \"noHead\" )' 

calculates an ag group $H$, which is isomorphic to $G$ and whose ag
system is corresponding to the Leedham-Green series and has public Hall
subgroups.

To obtain the additional information of a special ag system see "Special
Ag Group Records".

%%%%%%%%%%%%%%%%%%%%%%%%%%%%%%%%%%%%%%%%%%%%%%%%%%%%%%%%%%%%%%%%%%%%%%%%%
\Section{Special Ag Group Records}

In addition to the record components of ag groups (see "Finite Polycyclic
Groups") the following components are present in the group record of a
special ag group $H$.

'weights':  \\
    This is a list of weights such that the $i$-th entry gives the weight
    of  the element $h_i$, i.e., the triple  $(w_1,  w_2, w_3)$ when
    $h_i$  lies  in  $G_{w_1,  w_2,  w_3}$ but  not  lower  down  in  the
    Leedham-Green series (see "More about Special Ag Groups").

The entries 'layers', 'first', 'head' and 'tail' only depend on the
'weights'.  These entries are useful in many of the programs using the
special ag system.

'layers':  \\
    This  is  a list  of integers.  Assume that  the  subgroups of  the
    Leedham-Green series  are numbered beginning at $G$ and ending at the
    trivial group.  Then the $i$-th  entry gives the number of the
    subgroup in  the Leedham-Green  series to which $h_i$ corresponds  as
    described in 'weights'.

'first':  \\
    This is a list of integers, and 'first'[$j$] =  $i$ if $h_i$  is the
    first element of the $j$-th  layer.  Additionally the  last  entry of
    the list 'first' is always $n + 1$.

'head':  \\
    This is a list of integers, and 'head'[$j$] =  $i$ if $h_i$  is the
    first element of the $j$-th  local head.  Additionally the  last  
    entry of the list 'head' is always $n + 1$ (see "More about Special 
    Ag Groups").

'tail':  \\
    This is a list of integers, and 'tail'[$j$] =  $i$ if $h_{i-1}$ is 
    the last element of the $j$-th local head. In other words $h_i$ is either 
    the first element of the tail of the $j$-th layer in the lower nilpotent
    series, or in case this tail is trivial, then $h_i$ is the first 
    element of the $j+1$-st layer in the lower nilpotent series. 
    If the tail of the smallest nontrivial subgroup of the lower nilpotent
    series is trivial, then the last entry of the list 'tail' is $n+1$
    (see "More about Special Ag Groups").

'bijection':  \\ 
    This is the  isomorphism from $H$ to $G$ given through the  images of
    the generators of $H$.

The next four entries indicate if any <flag> and which one is used 
in the calculation of the special ag system (see "Construction of 
Special Ag Groups" and "Restricted Special Ag Groups").

'isHallSystem': \\  
    This entry is a Boolean. It is true if public Hall subgroups have
    been calculated, and false otherwise.

'isHeadSystem': \\ 
    This entry is a Boolean. It is true if public local head complements
    have been calculated, and false otherwise.

'isSagGroup': \\ 
    This entry is a Boolean. It is true if public Hall subgroups and
    public local head complements have been calculated, and false otherwise.

Note that in {\GAP} an ag group is called a special ag group if and only
if the record entry 'isSagGroup' is true.

\vspace{.5cm}
|    # construct a wreath product of a4 with s3 where s3 operates on 3 points.
    gap> s3 := SymmetricGroup( AgWords, 3 );;
    gap> a4 := AlternatingGroup( AgWords, 4 );;
    gap> a4wrs3 := WreathProduct(a4, s3, s3.bijection);
    Group( h1, h2, n1_1, n1_2, n1_3, n2_1, n2_2, n2_3, n3_1, n3_2, n3_3 )

    # now calculate the special ag group
    gap> S := SpecialAgGroup( a4wrs3 );
    Group( h1, n3_1, h2, n2_1, n1_1, n1_2, n1_3, n2_2, n2_3, n3_2, n3_3 )
    gap> S.weights;
    [ [ 1, 1, 2 ], [ 1, 1, 3 ], [ 2, 1, 3 ], [ 2, 1, 3 ], [ 2, 2, 3 ], 
      [ 3, 1, 2 ], [ 3, 1, 2 ], [ 3, 1, 2 ], [ 3, 1, 2 ], [ 3, 1, 2 ], 
      [ 3, 1, 2 ] ]
    gap> S.layers;
    [ 1, 2, 3, 3, 4, 5, 5, 5, 5, 5, 5 ]
    gap> S.first;
    [ 1, 2, 3, 5, 6, 12 ]
    gap> S.head;
    [ 1, 3, 6, 12 ]
    gap> S.tail;
    [ 3, 5, 12 ]
    gap> S.bijection;
    GroupHomomorphismByImages( Group( h1, n3_1, h2, n2_1, n1_1, n1_2,
    n1_3, n2_2, n2_3, n3_2, n3_3 ), Group( h1, h2, n1_1, n1_2, n1_3,
    n2_1, n2_2, n2_3, n3_1, n3_2, n3_3 ),
    [ h1, n3_1, h2, n2_1, n1_1, n1_2, n1_3, n2_2, n2_3, n3_2, n3_3 ],
    [ h1, n3_1, h2, n2_1*n3_1^2, n1_1*n2_1*n3_1, n1_2, n1_3, n2_2, n2_3,
      n3_2, n3_3 ] )
    gap> S.isHallSystem;
    true
    gap> S.isHeadSystem;
    true
    gap> S.isSagGroup;
    true |

In the next sections the functions which only apply to special ag groups
are described.

%%%%%%%%%%%%%%%%%%%%%%%%%%%%%%%%%%%%%%%%%%%%%%%%%%%%%%%%%%%%%%%%%%%%%%%%%
\Section{MatGroupSagGroup}

'MatGroupSagGroup( <H>, <i> )' 

'MatGroupSagGroup' calculates the matrix representation of $H$ on the
$i$-th layer of the Leed\-ham-Green series of $H$ (see "More about
Special Ag Groups").

See also 'MatGroupAgGroup'.

|    gap> S := SpecialAgGroup( a4wrs3 );;
    gap> S.weights;
    [ [ 1, 1, 2 ], [ 1, 1, 3 ], [ 2, 1, 3 ], [ 2, 1, 3 ], [ 2, 2, 3 ], 
      [ 3, 1, 2 ], [ 3, 1, 2 ], [ 3, 1, 2 ], [ 3, 1, 2 ], [ 3, 1, 2 ], 
      [ 3, 1, 2 ] ]
    gap> MatGroupSagGroup(S,3);
    Group( [ [ Z(3), 0*Z(3) ], [ 0*Z(3), Z(3) ] ], 
    [ [ Z(3)^0, Z(3)^0 ], [ 0*Z(3), Z(3)^0 ] ] )|

%%%%%%%%%%%%%%%%%%%%%%%%%%%%%%%%%%%%%%%%%%%%%%%%%%%%%%%%%%%%%%%%%%%%%%%%%
\Section{DualMatGroupSagGroup}

'DualMatGroupSagGroup( <H>, <i> )' 

'DualMatGroupSagGroup' calculates the dual matrix representation of $H$
on the $i$-th layer of the Leedham-Green series of $H$ (see "More about
Special Ag Groups").

Let $V$ be an $F H$-module for a field $F$. Then the dual module to $V$
is defined by
$V^\* \:= \{f \: V \rightarrow F \| f \mbox{ is linear }\}$.
This module is also an $F H$-module and the dual matrix representation is
the representation on the dual module.

|    gap> S := SpecialAgGroup( a4wrs3 );;
    gap> DualMatGroupSagGroup(S,3);
    Group( [ [ Z(3), 0*Z(3) ], [ 0*Z(3), Z(3) ] ], 
    [ [ Z(3)^0, 0*Z(3) ], [ Z(3)^0, Z(3)^0 ] ] ) |

%%%%%%%%%%%%%%%%%%%%%%%%%%%%%%%%%%%%%%%%%%%%%%%%%%%%%%%%%%%%%%%%%%%%%%%%%
\Section{Ag Group Functions for Special Ag Groups}

Since special ag groups are ag groups all functions for ag groups are
applicable to special ag groups. However certain of these functions use
special implementations to treat special ag groups, i.e. there exists
functions like SagGroupOps.FunctionName, which are called by the
corresponding general function in case a special ag group given.  If you
call one of these general functions with an arbitrary ag group, the
general function will not calculate the special ag group but use the
function for ag groups.  For the special implementations to treat special
ag groups note the following.

\vspace{.5cm}
'Centre( <H> )' \\
'MinimalGeneratingSet( <H> )' \\
'Intersection( <U>, <L>)' \\
'EulerianFunction( <H> )'
'MaximalSubgroups( <H> )' \\
'ConjugacyClassesMaximalSubgroups( <H> )' \\
'PrefrattiniSubgroup( <H> )' \\
'FrattiniSubgroup( <H> )' \\
'IsNilpotent( <H> )' \\
These functions are often faster and often use less space for special ag
groups.

\vspace{.5cm}
'ElementaryAbelianSeries( <H> )' \\
This function returns the Leedham-Green series (see "More about Special
Ag Groups").

\vspace{.5cm}
'IsElementaryAbelianSeries( <H> )' \\
Returns true. 

\vspace{.5cm}
'HallSubgroup( <H>, <primes> )' \\
'SylowSubgroup( <H>, <p> )' \\
'SylowSystem( <H> )' \\
These functions return the corresponding public subgroups (see "More
about Special Ag Groups").

\vspace{.5cm}
'Subgroup( <H>, <gens> )' \\
'AgSubgroup( <H>, <gens>, <bool> )' \\
These functions return an ag group which is not special, except if the
group itself is returned.

\vspace{.5cm}
All domain functions not mentioned here use no special treatments for
special ag groups. \\

Note also that there exists a package to compute formation theoretic
subgroups of special ag groups. This may be used to compute the
system normalizer of the public Sylow system, which is the $F$-normalizer
for the formation of nilpotent groups $F$. It is also possible to 
compute $F$-normalizers as well as $F$-covering subgroups and 
$F$-residuals of special ag groups for a number of saturated formations
$F$ which are given within the package or for self-defined saturated
formations $F$. 
