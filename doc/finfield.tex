%%%%%%%%%%%%%%%%%%%%%%%%%%%%%%%%%%%%%%%%%%%%%%%%%%%%%%%%%%%%%%%%%%%%%%%%%%%%%
%%
%A  finfield.tex                GAP documentation            Martin Schoenert
%%
%A  @(#)$Id: finfield.tex,v 1.1.1.1 1996/12/11 12:36:44 werner Exp $
%%
%Y  Copyright 1990-1992,  Lehrstuhl D fuer Mathematik,  RWTH Aachen,  Germany
%%
%%  This file describes the operators and functions of finite field elements.
%%
%H  $Log: finfield.tex,v $
%H  Revision 1.1.1.1  1996/12/11 12:36:44  werner
%H  Preparing 3.4.4 for release
%H
%H  Revision 3.8  1993/02/19  10:48:42  gap
%H  adjustments in line length and spelling
%H
%H  Revision 3.7  1993/02/10  10:30:19  felsch
%H  more examples fixed
%H
%H  Revision 3.6  1993/02/02  12:44:55  felsch
%H  long line fixed
%H
%H  Revision 3.5  1993/02/01  15:01:41  felsch
%H  examples fixed
%H
%H  Revision 3.4  1992/04/07  10:06:55  martin
%H  fixed some more typos
%H
%H  Revision 3.3  1992/04/02  21:06:23  martin
%H  changed *domain functions* to *set theoretic functions*
%H
%H  Revision 3.2  1992/03/25  15:37:32  martin
%H  added "FrobeniusAutomorphism"
%H
%H  Revision 3.1  1991/12/30  12:09:18  martin
%H  revised everything for GAP 3.1 manual
%H
%H  Revision 3.0  1991/04/11  11:30:55  martin
%H  Initial revision under RCS
%H
%%
\Chapter{Finite Fields}%
\index{field!finite}\index{galois field}\index{field!galois}

Finite fields comprise an important algebraic  domain.  The elements in a
field  form   an   additive  group   and  the  nonzero  elements  form  a
multiplicative  group.  For  every prime power $q$ there exists  a unique
field of  size  $q$ up to isomorphism.  {\GAP} supports finite fields  of
size at most $2^{16}$.

The first section in this chapter describes how you can enter elements of
finite fields and how {\GAP} prints them (see "Finite Field Elements").

The  next sections describe  the  operations applicable to  finite  field
elements (see "Comparisons of  Finite Field Elements" and "Operations for
Finite Field Elements").

The next section describes the function that tests whether an object is a
finite field element (see "IsFFE").

The  next sections describe   the functions  that give  basic information
about finite field elements (see "CharFFE", "DegreeFFE", and "OrderFFE").

The next  sections  describe  the functions  that compute  various  other
representations of finite field elements (see "IntFFE" and "LogFFE").

The next section  describes  the  function that constructs a finite field
(see "GaloisField").

Finite  fields  are  domains,  thus  all  set  theoretic   functions  are
applicable to them (see  chapter "Domains" and "Set Functions  for Finite
Fields").

Finite  fields  are  of course  fields,  thus  all  field  functions  are
applicable to them and to their elements (see chapter "Fields" and "Field
Functions for Finite Fields").

All functions are in 'LIBNAME/\"finfield.g\"'.

%%%%%%%%%%%%%%%%%%%%%%%%%%%%%%%%%%%%%%%%%%%%%%%%%%%%%%%%%%%%%%%%%%%%%%%%%
\Section{Finite Field Elements}%
\index{Z}

'Z( <p>\^<d> )'

The function 'Z' returns  the designated generator of  the multiplicative
group of the finite field with '<p>\^<d>' elements.  <p>  must be a prime
and '<p>\^<d>' must be less than or equal to $2^{16} = 65536$.

The  root returned by 'Z' is  a generator of  the multiplicative group of
the finite field with $p^d$ elements, which  is cyclic.  The order of the
element is  of course $p^d-1$.  The $p^d-1$  different powers of the root
are exactly the nonzero elements of the finite field.

Thus  all nonzero elements of the  finite field  with '<p>\^<d>' elements
can  be entered  as 'Z(<p>\^<d>)\^<i>'.  Note that this is  also the form
that {\GAP} uses to output those elements.

The additive neutral element  is '0\*Z(<p>)'.  It  is  different from the
integer '0' in subtle ways.  First 'IsInt( 0\*Z(<p>)  )' (see "IsInt") is
'false' and 'IsFFE( 0\*Z(<p>) )'  (see "IsFFE") is  'true', whereas it is
just the other way around for the integer 0.

The multiplicative neutral element is 'Z(<p>)\^0'.   It is different from
the integer '1' in subtle ways.  First 'IsInt( Z(<p>)\^0 )' (see "IsInt")
is 'false' and 'IsFFE( Z(<p>)\^0 )' (see  "IsFFE") is  'true', whereas it
is just the  other  way around for   the  integer 1.  Also '1+1'  is '2',
whereas, e.g., 'Z(2)\^0 + Z(2)\^0' is '0\*Z(2)'.

The  various  roots  returned  by  'Z'  for  finite  fields  of the  same
characteristic  are  compatible  in  the  following  sense.  If the field
$GF(p^n)$ is a  subfield of the  field  $GF(p^m)$, i.e., $n$ divides $m$,
then $Z(p^n) = Z(p^m)^{(p^m-1)/(p^n-1)}$.  Note that this is the simplest
relation that may  hold  between a generator of $GF(p^n)$ and  $GF(p^m)$,
since $Z(p^n)$ is an element of order $p^m-1$ and $Z(p^m)$  is an element
of order  $p^n-1$.  This is achieved  by choosing $Z(p)$ as  the smallest
primitive  root modulo $p$  and  $Z(p^n)$ as a root of the $n$-th  Conway
polynomial of  characteristic $p$.   Those polynomials  where  defined by
J.H.~Conway and computed by R.A.~Parker.

|    gap> z := Z(16);
    Z(2^4)
    gap> z*z;
    Z(2^4)^2 |

%%%%%%%%%%%%%%%%%%%%%%%%%%%%%%%%%%%%%%%%%%%%%%%%%%%%%%%%%%%%%%%%%%%%%%%%%
\Section{Comparisons of Finite Field Elements}%
\index{comparisons!of finite field elements}

'<z1> = <z2>'\\
'<z1> \<> <z2>'

The equality  operator  '=' evaluates to 'true'  if the two elements in a
finite  field <z1> and   <z2> are equal   and to 'false' otherwise.   The
inequality operator '\<>' evaluates to  'true' if  the two  elements in a
finite finite field <z1> and <z2> are not equal and to 'false' otherwise.

Note that the integer  0 is not equal to  the  zero element in any finite
field.  There comparisons '<z> = 0' will always evaluate to 'false'.  Use
'<z> = 0\*<z>' instead, or even better '<z> = <F>.zero', where <F> is the
field record for a finite field of the same characteristic.

|    gap> Z(2^4)^10 = Z(2^4)^25;
    true    # 'Z(2\^4)' has order 15
    gap> Z(2^4)^10 = Z(2^2)^2;
    true    # shows the embedding of 'GF(4)' into 'GF(16)'
    gap> Z(2^4)^10 = Z(3);
    false |

'<z1> \<\ <z2>'\\
'<z1> \<= <z2>'\\
'<z1>  >  <z2>'\\
'<z1>  >= <z2>'

The operators  '\<',  '\<=', '>',  and  '=>' evaluate to  'true'  if  the
element  in  a finite field <z1>  is  less  than, less than or equal  to,
greater than, and greater than or equal to the element  in a finite field
<z2>.

Elements in finite fields  are ordered as follows.   If the two  elements
lie in fields of different characteristics the one that lies in the field
with the  smaller characteristic is smaller.  If  the two elements lie in
different fields  of  the same characteristic  the  one that  lies in the
smaller field is smaller.  If the two elements lie  in the same field and
one is  the zero and the  other is not, the zero  element is smaller.  If
the  two elements lie  in  the same field and   both are nonzero, and are
represented as $Z(p^d)^{i_1}$  and $Z(p^d)^{i_2}$ respectively, then  the
one with the smaller $i$ is smaller.

You can  compare elements in a  finite field with objects of other types.
Integers, rationals, and  cyclotomics are smaller than elements in finite
fields, all other objects are larger.  Note especially that the integer 0
is smaller than the zero in every finite field.

|    gap> Z(2) < Z(3);
    true
    gap> Z(2) < Z(4);
    true
    gap> 0*Z(2) < Z(2);
    true
    gap> Z(4) < Z(4)^2;
    true
    gap> 0 < 0*Z(2);
    true
    gap> Z(4) < [ Z(4) ];
    true |

%%%%%%%%%%%%%%%%%%%%%%%%%%%%%%%%%%%%%%%%%%%%%%%%%%%%%%%%%%%%%%%%%%%%%%%%%
\Section{Operations for Finite Field Elements}%

'<z1>  +  <z2>'\\
'<z1>  -  <z2>'\\
'<z1> \*\ <z2>'\\
'<z1>  /  <z2>'

The  operators '+', '-',  '\*' and '/' evaluate to  the sum,  difference,
product,  and quotient of the two  finite field elements  <z1> and  <z2>,
which  must lie in fields  of the same characteristic.  For  the quotient
'/' <z2> must of course be nonzero.  The result  must of course lie  in a
finite field of size less than  or equal to  $2^{16}$, otherwise an error
is signalled.

Either operand may also be an integer <i>.  If <i> is zero it is taken as
the  zero  in the finite  field, i.e.,  '<F>.zero', where  <F> is a field
record for the finite  field in which the other  operand lies.  If <i> is
positive, it is  taken as <i>-fold  sum '<F>.one+<F>.one+..+<F>.one'.  If
<i> is negative it is taken as the additive inverse of '-<i>'.

|    gap> Z(8) + Z(8)^4;
    Z(2^3)^2
    gap> Z(8) - 1;
    Z(2^3)^3
    gap> Z(8) * Z(8)^6;
    Z(2)^0
    gap> Z(8) / Z(8)^6;
    Z(2^3)^2
    gap> -Z(9);
    Z(3^2)^5 |

'<z> \^\ <i>'

The powering operator '\^' returns the <i>-th power of the  element  in a
finite field <z>.  <i> must be an integer.  If  the exponent <i> is zero,
'<z>\^<i>' is  defined as the one  in the finite  field,  even if  <z> is
zero; if <i> is positive, '<z>\^<i>' is  defined as the  <i>-fold product
'<z>\*<z>\*..\*<z>'; finally, if  <i> is negative, '<z>\^<i>' is  defined
as '(1/<z>)\^-<i>'.  In this case <z> must of course be nonzero.

|    gap> Z(4)^2;
    Z(2^2)^2
    gap> Z(4)^3;
    Z(2)^0    # is in fact 1
    gap> (0*Z(4))^0;
    Z(2)^0 |

%%%%%%%%%%%%%%%%%%%%%%%%%%%%%%%%%%%%%%%%%%%%%%%%%%%%%%%%%%%%%%%%%%%%%%%%%
\Section{IsFFE}%
\index{test!for a finite field element}

'IsFFE( <obj> )'

'IsFFE' returns 'true' if <obj>, which may be  an object  of an arbitrary
type, is an element in a finite field and 'false' otherwise.  Will signal
an error if <obj> is an unbound variable.

Note that integers,  even though they can  be multiplied with elements in
finite fields, are not  considered themselves elements in  finite fields.
Therefore 'IsFFE' will return 'false' for integer arguments.

|    gap> IsFFE( Z(2^4)^7 );
    true
    gap> IsFFE( 5 );
    false |

%%%%%%%%%%%%%%%%%%%%%%%%%%%%%%%%%%%%%%%%%%%%%%%%%%%%%%%%%%%%%%%%%%%%%%%%%
\Section{CharFFE}%
\index{characteristic!of a finite field element}

'CharFFE( <z> )' or 'CharFFE( <vec> )' or 'CharFFE( <mat> )'

'CharFFE' returns the characteristic  of the finite  field <F> containing
the  element <z>, respectively  all  elements of the  vector <vec> over a
finite field (see "Vectors"), or matrix  <mat> over  a  finite field (see
"Matrices").

|    gap> CharFFE( Z(16)^7 );
    2
    gap> CharFFE( Z(16)^5 );
    2
    gap> CharFFE( [ Z(3), Z(27)^11, Z(9)^3 ] );
    3
    gap> CharFFE( [ [ Z(5), Z(125)^3 ], [ Z(625)^13, Z(5) ] ] );
    Error, CharFFE: <z> must be a finite field element, vector, or matrix
    # The smallest finite field which contains all four of these elements
    # is too large for {\GAP} |

%%%%%%%%%%%%%%%%%%%%%%%%%%%%%%%%%%%%%%%%%%%%%%%%%%%%%%%%%%%%%%%%%%%%%%%%%
\Section{DegreeFFE}%
\index{degree!of a finite field element}

'DegreeFFE( <z> )' or 'DegreeFFE( <vec> )' or 'DegreeFFE( <mat> )'

'DegreeFFE'  returns  the   degree of  the   smallest  finite field   <F>
containing the element <z>, respectively all elements of the vector <vec>
over a finite field (see "Vectors"), or matrix  <mat> over a finite field
(see "Matrices").  For vectors and matrices, an error is signalled if the
smallest finite field containing all elements of the vector or matrix has
size larger than $2^{16}$.

|    gap> DegreeFFE( Z(16)^7 );
    4
    gap> DegreeFFE( Z(16)^5 );
    2
    gap> DegreeFFE( [ Z(3), Z(27)^11, Z(9)^3 ] );
    6
    gap> DegreeFFE( [ [ Z(5), Z(125)^3 ], [ Z(625)^13, Z(5) ] ] );
    Error, DegreeFFE: <z> must be a finite field element, vector, or matrix
    # The smallest finite field which contains all four of these elements
    # is too large for {\GAP} |

%%%%%%%%%%%%%%%%%%%%%%%%%%%%%%%%%%%%%%%%%%%%%%%%%%%%%%%%%%%%%%%%%%%%%%%%%
\Section{OrderFFE}%
\index{order!of a finite field element}

'OrderFFE( <z> )'

'OrderFFE' returns the order of  the element <z> in  a finite field.  The
order  is the smallest positive integer <i> such  that  '<z>\^<i>'  is 1.
The order of the zero in a finite field is defined to be 0.

|    gap> OrderFFE( Z(16)^7 );
    15
    gap> OrderFFE( Z(16)^5 );
    3
    gap> OrderFFE( Z(27)^11 );
    26
    gap> OrderFFE( Z(625)^13 );
    48
    gap> OrderFFE( Z(211)^0 );
    1 |

%%%%%%%%%%%%%%%%%%%%%%%%%%%%%%%%%%%%%%%%%%%%%%%%%%%%%%%%%%%%%%%%%%%%%%%%%
\Section{IntFFE}
\index{convert!a finite field element to an integer}

'IntFFE( <z> )'

'IntFFE' returns the integer corresponding to the element <z>, which must
lie in  a finite  prime field.   That is  'IntFFE' returns  the  smallest
nonnegative integer <i> such that '<i> \*\ <z>\^ 0 = <z>'.

The  correspondence between   elements   from a finite   prime field   of
characteristic <p> and the integers between 0  and  '<p>-1' is defined by
choosing 'Z(<p>)'  the     smallest  primitive  root    mod   <p>    (see
"PrimitiveRootMod").

|    gap> IntFFE( Z(13) );
    2
    gap> PrimitiveRootMod( 13 );
    2
    gap> IntFFE( Z(409) );
    21
    gap> IntFFE( Z(409)^116 );
    311
    gap> 21^116 mod 409;
    311 |

%%%%%%%%%%%%%%%%%%%%%%%%%%%%%%%%%%%%%%%%%%%%%%%%%%%%%%%%%%%%%%%%%%%%%%%%%
\Section{LogFFE}
\index{logarithm!of a finite field element, discrete}
\index{discrete logarithm!of a finite field element}

'LogFFE( <z> )' \\
'LogFFE( <z>, <r> )'

In the first form 'LogFFE' returns  the discrete logarithm of the element
<z> in a finite field with respect to the  root 'FieldFFE(<z>).root'.  An
error is signalled if <z> is zero.

In the second form 'LogFFE' returns the discrete logarithm of the element
<z> in  a  finite  field with  respect  to  the  root <r>.   An  error is
signalled if <z> is zero, or if <z> is not a power of <r>.

The *discrete logarithm* of an element $z$ with  respect to a root $r$ is
the smallest nonnegative integer $i$ such that $r^i = z$.

|    gap> LogFFE( Z(409)^116 );
    116
    gap> LogFFE( Z(409)^116, Z(409)^2 );
    58 |

%%%%%%%%%%%%%%%%%%%%%%%%%%%%%%%%%%%%%%%%%%%%%%%%%%%%%%%%%%%%%%%%%%%%%%%%%
\Section{GaloisField}%
\index{GF}

'GaloisField( <p>\^<d> )' \\
'GF( <p>\^<d> )' \\
'GaloisField( <p>\|<S>, <d>\|<pol>\|<bas> )' \\
'GF( <p>\|<S>, <d>\|<pol>\|<bas> )'

'GaloisField' returns a  field record (see  "Field Records") for a finite
field.  It takes two arguments.  The form  'GaloisField(<p>,<d>)',  where
<p>,<d> are integers, can also be given as 'GaloisField(<p>\^<d>)'.  'GF'
is an abbreviation for 'GaloisField'.

The first argument  specifies the subfield <S>  over which the new  field
<F> is to be taken.  It can be  a prime or  a finite field record.  If it
is a prime <p>, the  subfield is the  prime field of this characteristic.
If it is a field record <S>, the subfield is the  field described by this
record.

The  second  argument specifies the extension.  It can be an integer,  an
irreducible polynomial, or a base.   If  it is an  integer  <d>, the  new
field  is  constructed  as  the  polynomial  extension  with  the  Conway
polynomial  of degree <d> over the subfield <S>.  If it is an irreducible
polynomial <pol>,  in which case the elements  of the list <pol> must all
lie  in  the  subfield <S>, the new field  is  constructed as  polynomial
extension  of the  subfield <S> with  this  polynomial.  If  it is a base
<bas>,  in which  case  the elements  of  the list <bas>  must be  linear
independently  over the  subfield  <S>,  the  new field is constructed as
a linear vector space over the subfield <S>.

Note that the subfield over which a field was constructed determines over
which field the Galois group, conjugates, norm,  trace,  minimal polynom,
and characteristic polynom are computed (see "GaloisGroup", "Conjugates",
"Norm", "Trace", "MinPol", "CharPol", and   "Field  Functions for  Finite
Fields").

|    gap> GF( 2^4 );
    GF(2^4)
    gap> GF( GF(2^4), 2 );
    GF(2^8)/GF(2^4) |

%%%%%%%%%%%%%%%%%%%%%%%%%%%%%%%%%%%%%%%%%%%%%%%%%%%%%%%%%%%%%%%%%%%%%%%%%
\Section{FrobeniusAutomorphism}%
\index{homomorphisms!Frobenius, field}%
\index{field homomorphisms!Frobenius}%
\index{Image!for Frobenius automorphisms}%
\index{CompositionMapping!for Frobenius automorphisms}

'FrobeniusAutomorphism( <F> )'

'FrobeniusAutomorphism'  returns the Frobenius automorphism of the finite
field <F> as a field homomorphism (see "Field Homomorphisms").

The  Frobenius automorphism $f$ of  a finite field $F$  of characteristic
$p$  is  the function that  takes  each element $z$ of  $F$ to its $p$-th
power.    Each  automorphism  of  $F$  is  a   power   of  the  Frobenius
automorphism.  Thus the  Frobenius  automorphism  is a generator  for the
Galois group of $F$ (and an appropriate power of it is a generator of the
Galois group of $F$ over a subfield $S$) (see "GaloisGroup").

|    gap> f := GF(16);
    GF(2^4)
    gap> x := FrobeniusAutomorphism( f );
    FrobeniusAutomorphism( GF(2^4) )
    gap> Z(16) ^ x;
    Z(2^4)^2 |

The image  of an  element  <z> under the  <i>-th  power  of the Frobenius
automorphism  <f> of a finite field  <F> of characteristic <p> is  simply
computed by computing the '<p>\^<i>'-th power of <z>.  The product of the
<i>-th power and the <j>-th  power of <f> is  the  <k>-th power  of  <f>,
where <k> is  '<i>\*<j> mod (Size(<F>)-1)'.   The zeroth  power of <f> is
printed as 'IdentityMapping( <F> )'.

%%%%%%%%%%%%%%%%%%%%%%%%%%%%%%%%%%%%%%%%%%%%%%%%%%%%%%%%%%%%%%%%%%%%%%%%%
\Section{Set Functions for Finite Fields}%
\index{Elements!for finite fields}%
\index{in!for finite fields}\index{membership test!for finite fields}%
\index{Random!for finite fields}
\index{Intersection!for finite fields}

Finite  fields are of course domains.  Thus all  set theoretic  functions
are  applicable  to finite fields (see chapter "Domains").   This section
gives  further  comments on the definitions and  implementations of those
functions for finite fields.  All  set theoretic  functions not mentioned
here are not treated specially for finite fields.

'Elements'

The elements  of a  finite field  are computed  using the  fact that  the
finite field is a vector space over its prime field.

'in'

The membership test is  of course  very simple,  we  just  have  to  test
whether the  element is a  finite field element with 'IsFFE',  whether it
has the  correct characteristic  with 'CharFFE', and  whether its  degree
divides  the  degree of the  finite field  with 'DegreeFFE' (see "IsFFE",
"CharFFE", and "DegreeFFE").

'Random'

A random element of $GF(p^n)$ is  computed by  computing a random integer
$i$  from  $[0..p^n-1]$    and returning  $0\*Z(p)$  if  $i   = 0$    and
$Z(p^n)^{i-1}$ otherwise.

'Intersection'

The  intersection  of  $GF(p^n)$  and  $GF(p^m)$   is the   finite  field
$GF(p^{Gcd(n,m)})$, and is returned as finite field record.

%%%%%%%%%%%%%%%%%%%%%%%%%%%%%%%%%%%%%%%%%%%%%%%%%%%%%%%%%%%%%%%%%%%%%%%%%
\Section{Field Functions for Finite Fields}%
\index{GaloisGroup!for finite fields}%
\index{Conjugates!for finite fields}%
\index{Norm!for finite fields}

Finite fields  are, as the name  already implies, fields.  Thus all field
functions are applicable to finite fields and their elements (see chapter
"Fields").  This section  gives further comments on  the  definitions and
implementations  of  those  functions  for   finite fields.   All  domain
functions not mentioned here are not treated specially for finite fields.

'Field' and 'DefaultField'

Both  'Field'   and 'DefaultField'  return  the   smallest   finite field
containing the arguments as  an extension of  the  prime field.

'GaloisGroup'

The Galois  group of a finite field $F$ of size $p^m$ over a subfield $S$
of size $q =  p^n$ is a cyclic  group of  size $m/n$.  It is generated by
the  *Frobenius automorphism*  that takes every  element of  $F$  to  its
$q$-th power.  This automorphism of  $F$  leaves exactly the subfield $S$
fixed.

'Conjugates'

According  to the above theorem about  the Galois  group, each element of
$F$ has  $m/n$ conjugates, $z,  z^q,  z^{q^2}, ..., z^{q^{m/n-1}}$.

'Norm'

The  norm is the  product of the conjugates, i.e., $z^{{p^m-1}/{p^n-1}}$.
Because  we  have $Z(p^n) =  Z(p^m)^{{p^m-1}/{p^n-1}}$, it  follows  that
$Norm( GF(p^m)/GF(p^n), Z(p^m)^i ) = Z(p^n)^i$.

%%%%%%%%%%%%%%%%%%%%%%%%%%%%%%%%%%%%%%%%%%%%%%%%%%%%%%%%%%%%%%%%%%%%%%%%%
%%
%E  Emacs . . . . . . . . . . . . . . . . . . . . . local Emacs variables
%%
%%  Local Variables:
%%  mode:               outline
%%  outline-regexp:     "\\\\Chapter\\|\\\\Section"
%%  fill-column:        73
%%  eval:               (hide-body)
%%  End:
%%



