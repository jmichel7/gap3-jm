%%%%%%%%%%%%%%%%%%%%%%%%%%%%%%%%%%%%%%%%%%%%%%%%%%%%%%%%%%%%%%%%%%%%%%%%%%%%%
%%
%A  chvrcos.tex       CHEVIE documentation                      Jean Michel
%%
%Y  Copyright (C) 2005-2010   University  Paris VII.
%%
%%  This  file  contains  the  description  of  the  GAP functions of CHEVIE
%%  dealing with Spetses.
%%%%%%%%%%%%%%%%%%%%%%%%%%%%%%%%%%%%%%%%%%%%%%%%%%%%%%%%%%%%%%%%%%%%%%%%%
\def\GL{{\text GL}}

\Chapter{Reflection cosets}
Let $W\subset\GL(V)$ be a complex reflection group on the vector space $V$.
Let  $\phi$ be an element of $\GL(V)$  which normalizes $W$. Then the coset
$W\phi$ is called a reflection coset.

A  reference for these cosets is  \cite{BMM99}; the main motivation is that
in  the case where $W$  is a rational reflection  group (a Weyl group) such
cosets,  that  we  will  call  *Weyl  cosets*, model rational structures on
finite reductive groups. Finally, when $W$ is a so-called *Spetsial* group,
they  are the basic object for the construction of a *Spetses*, which is an
object  attached to  a complex  reflection group  from which one can derive
combinatorially  some attributes shared with  finite reductive groups, like
unipotent degrees, etc$\ldots$.

We  say that  a reflection  coset is  irreducible if  $W$ is irreducible. A
general  coset is a direct  product of *descents of  scalars*, which is the
case  where $\phi$ is transitive on  the irreducible components of $W$. The
irreducible   cosets  have   been  classified   in  \cite{BMM99}\:\  up  to
multiplication  of $\phi$  by a  scalar, there  is usually  only one or two
possible cosets for a given irreducible group.

In  \CHEVIE\ we deal  only with *finite  order* cosets, that  is, we assume
there  is a  (minimal) integer  $\delta$ such  that $(W\phi)^\delta=W\phi$.
Then   the  group  generated  by  $W$   and  $\phi$  is  finite,  of  order
$\delta\|W\|$.

A  subset $C$  of a  $W\phi$ is  called a  *conjugacy class*  if one of the
following equivalent conditions is fulfilled\:

$\bullet$  $C$ is the orbit of an  element in $W\phi$ under the conjugation
action of $W$.

$\bullet$ $C$ is a conjugacy class of $\langle W,\phi \rangle$ contained in
$W\phi$.

$\bullet$  The set $\{w\in W\mid\;w\phi\in C\}$ is a $\phi$-conjugacy class
of $W$ (two elements $v,w\in W$ are called $\phi$-conjugate, if and only if
there exists $x\in W$ with $v = xw\phi(x^{-1})$).

An  irreducible  character  of  $\langle  W,\phi \rangle$ has some non-zero
values  on $W\phi$ if  and only if  its restriction to  $W$ is irreducible.
Further,  two characters $\chi_1$ and  $\chi_2$ which have same irreducible
restriction   to  $W$   differ  by   a  character   of  the   cyclic  group
$\langle\phi\rangle$  (which  identifies  to  the  quotient $\langle W,\phi
\rangle/W$).  A set containing one extension to $\langle W,\phi \rangle$ of
each  $\phi$-invariant character  of $W$  is called  a *set  of irreducible
characters  of $W\phi$*. Two such characters  are orthogonal for the scalar
product  on the  class functions  on $W\phi$  given by  $$\langle \chi,\psi
\rangle \:= \frac{1}{\mid W\mid}\sum_{w\in W}
\chi(w\phi)\overline{\psi(w\phi)}.$$  For  rational  groups  (Weyl groups),
Lusztig  has defined a canonical choice  of a set of irreducible characters
for  $W\phi$  (called  the  *preferred  extensions*),  but for more general
reflection cosets we have made some rather arbitrary choices, which however
have the property that their values lie in the smallest possible field.

The  *character  table*  of  $W\phi$  is  the  table  of values of a set of
irreducible characters on the conjugacy classes.

A  *subcoset* $Lw\phi$ of $W\phi$ is given  by a reflection subgroup $L$ of
$W$ and an element $w$ of $W$ such that $w\phi$ normalizes $L$.

We  then  have  a  natural  notion  of  *restriction* of class functions on
$W\phi$  to class functions  on $Lw\phi$ as  well as of  *induction* in the
other  direction. These maps are adjoint with respect to the scalar product
defined above (see \cite{BMM99}).

In  \CHEVIE\  the  most  general  construction  of a reflection coset is by
starting  from  a  reflection  datum,  and  giving  in  addition the matrix
'phiMat' of the map $\phi\:V\rightarrow V$ (see the command
'ReflectionCoset').   However,   at   present,   general  cosets  are  only
implemented for groups represented as permutation groups on a set of roots,
and  it is required that the automorphism  given preserves this set up to a
scalar  (it  is  allowed  that  these  scalars  depend  on  the  pair of an
irreducible  component and its image). If it also allowed to specify $\phi$
by the permutation it induces on the roots; in this case it is assumed that
$\phi$  acts trivially on the orthogonal of  the roots, but the roots could
be  those of a parent group, generating a larger space. Thus in any case we
have  a  permutation  representation  of  $\langle  W,\phi  \rangle$ and we
consider the coset to be a set of permutations.

Reflection cosets are implemented in \CHEVIE\ by a record which points to a
reflection  group record and has additional fields holding 'phiMat' and the
corresponding  permutation 'phi'. In the general case, on each component of
$W$ which is a descent of scalars, 'phiMat' will permute the components and
differ  by a scalar on each  component from an automorphism which preserves
the  roots. In this case, we have  a permutation 'phi' and a 'scalar' which
is stored for that component.

The  most common situation where cosets  with non-trivial 'phi' arise is as
sub-cosets  of reflection groups. Here is  an ``exotic\'\' example, see the
next chapter for more classical examples involving Coxeter groups.

|    gap> W:=ComplexReflectionGroup(14);
    ComplexReflectionGroup(14)
    gap> PrintDiagram(W);
    G14 1--8--2(3)
    gap> R:=ReflectionSubgroup(W,[2,4]);
    ReflectionSubgroup(ComplexReflectionGroup(14), [ 2, 4 ])
    gap> PrintDiagram(R);
    G5(ER(6)) 2(3)==4(3)
    gap> Rphi:=ReflectionCoset(R,W.1);
    2G5(ER(6))<2,4>
    gap> PrintDiagram(Rphi);
    phi acts as (2,4) on the component below
    G5(ER(6)) 2(3)==4(3)
    gap> ReflectionDegrees(Rphi);
    [ [ 6, 1 ], [ 12, -1 ] ]|

The  last line shows for each  reflection degree the corresponding *factor*
of the coset, which is the scalar by which $\phi$ acts on the corresponding
fundamental reflection invariant. The factors characterize the coset.

The  variable 'CHEVIE.PrintSpets'  determines if  a coset  is printed in an
abbreviated  form which  describes its  type, as  above ('G5' twisted by 2,
with  a Cartan matrix  which differs from  the standard one  by a factor of
$\sqrt  6$), or in  a form which  could be input  back in {\GAP}. The above
example  was for the default value 'CHEVIE.PrintSpets=rec()'. With the same
data we have\:

|    gap> CHEVIE.PrintSpets:=rec(GAP:=true);;
    gap> Rphi;
    Spets(ReflectionSubgroup(ComplexReflectionGroup(14), [ 2, 4 ]), (1,3)(\
    2,4)(5,9)(6,10)(7,11)(8,12)(13,21)(14,22)(15,23)(16,24)(17,25)(18,26)(\
    19,27)(20,28)(29,41)(30,42)(31,43)(32,44)(33,45)(34,46)(35,47)(36,48)(\
    37,49)(38,50)(39,51)(40,52)(53,71)(54,72)(55,73)(56,74)(57,75)(58,76)(\
    59,77)(60,78)(62,79)(64,80)(65,81)(66,82)(67,69)(68,70)(83,100)(84,101\
    )(85,102)(87,103)(89,99)(90,97)(91,98)(92,96)(93,104)(94,95)(105,113)(\
    106,114)(109,111)(110,112)(115,118)(116,117)(119,120))
    gap> CHEVIE.PrintSpets:=rec();;|

%%%%%%%%%%%%%%%%%%%%%%%%%%%%%%%%%%%%%%%%%%%%%%%%%%%%%%%%%%%%%%%%%%%%%%%%%%%%%
\Section{ReflectionCoset}
\index{ReflectionCoset}

'ReflectionCoset( <W>[, <phiMat> ] )'

'ReflectionCoset( <W>[, <phiPerm>] )'

This  function returns a  reflection coset as  a \GAP\ object. The argument
<W>  must  be  a  reflection  group  (created  by 'ComplexReflectionGroup',
'CoxeterGroup', 'PermRootGroup' or 'ReflectionSubgroup'). In the first form
the  argument <phiMat> must be an  invertible matrix with 'Rank(<W>)' rows,
which  normalizes the parent of <W> (if any)  as well as <W>. In the second
form  <phiPerm> is  a permutation  which describes  the images of the roots
under  $phi$ (only the  image of the  roots corresponding to the generating
reflections need be given, since they already determine a unique <phiMat>).
This  second form is only allowed if  the semisimple rank of <W> equals the
rank  (i.e., the roots are a basis of  $V$). If there is no second argument
the  default for  <phiMat> is  the identity  matrix, so  the result  is the
trivial coset equal to $W$ itself.

'ReflectionCoset'  returns a  record from  which we  document the following
components\:

'isDomain', 'isFinite':\\
        true

'group':\\
        the group <W>

'phiMat':\\
        the matrix acting on $V$ which represents $\phi$.

'phi':\\
        the permutation on the roots of <W> induced by 'phiMat'.

|    gap> W := CoxeterGroup("A",3);;
    gap> Wphi := ReflectionCoset( W, (1,3));
    2A3
    gap> m:=MatXPerm(W,(1,3));
    [ [ 0, 0, 1 ], [ 0, 1, 0 ], [ 1, 0, 0 ] ]
    gap> ReflectionCoset( W,m);
    2A3|

%%%%%%%%%%%%%%%%%%%%%%%%%%%%%%%%%%%%%%%%%%%%%%%%%%%%%%%%%%%%%%%%%%%%%%%%%%%%%
\Section{Spets}
\index{Spets}

'Spets' is a synonym for 'ReflectionCoset'. See "ReflectionCoset".

%%%%%%%%%%%%%%%%%%%%%%%%%%%%%%%%%%%%%%%%%%%%%%%%%%%%%%%%%%%%%%%%%%%%%%%%%
\Section{ReflectionSubCoset}
\index{ReflectionSubCoset}

'ReflectionSubCoset( <WF>, <r>, [<w>] )'

Returns  the reflection subcoset of the  reflection coset <WF> generated by
the  reflections specified by  <r>. <r> is  a list of  indices specifying a
subset of the roots of <W> where <W> is the reflection group 'Group(<WF>)'.
If  specified,  <w>  must  be  an  element  of  $W$  such  that 'w\*WF.phi'
normalizes  up to scalars  the subroot system  generated by <r>. If absent,
the  default value for <w> is '()'. If the subroot system is not normalized
then 'false' is returned, with a warning message if 'InfoChevie=Print'.

|    gap> W:=ComplexReflectionGroup(14);
    ComplexReflectionGroup(14)
    gap> Wphi:=ReflectionCoset(W);
    G14
    gap> ReflectionSubCoset(Wphi,[2,4],W.1);
    2G5(ER(6))<2,4>
    gap> WF:=ReflectionCoset(CoxeterGroup("A",4),(1,4)(2,3));
    2A4
    gap> ReflectionSubCoset(WF,[2,3]);
    2A2<2,3>.(q-1)(q+1)
    gap> ReflectionSubCoset(WF,[1,2]);
    #I permutation for F0 must normalize set of roots.
    false|

%%%%%%%%%%%%%%%%%%%%%%%%%%%%%%%%%%%%%%%%%%%%%%%%%%%%%%%%%%%%%%%%%%%%%%%%%%%%%
\Section{SubSpets}
\index{SubSpets}

'SubSpets' is a synonym for 'ReflectionSubCoset'. See "ReflectionSubCoset".

%%%%%%%%%%%%%%%%%%%%%%%%%%%%%%%%%%%%%%%%%%%%%%%%%%%%%%%%%%%%%%%%%%%%%%%%%
\Section{Functions for Reflection cosets}

'Group( <WF> )':\\
     returns the reflection group of which <WF> is a coset.

Quite a few functions defined for domains, permutation groups or reflection
groups have been implemented to work with reflection Cosets.

'Size', 'Rank', 'SemisimpleRank':\\ these functions use the corresponding
     functions for 'Group( <WF> )'.
'Elements', 'Random', 'Representative',  'in':\\  these functions
     use the corresponding functions for 'Group( <WF> )' and multiply the
     result by |WF.phi|.

\index{ConjugacyClasses}
'ConjugacyClasses(  <WF>  )':\\ returns  the   conjugacy classes  of  the
  coset   <WF>  (see  the  introduction  of   this  Chapter).  Let  <W>  be
  'Group(<WF>)'.  Then the classes  are defined to  be the $W$-orbits on $W
  \phi$,  where  $W$  acts  by  conjugation  (they  coincide  with  the  $W
  \phi$-orbits,  $W \phi$ acting by the conjugation); by the translation $w
  \mapsto w\phi^{-1}$ they are sent to the $\phi$-conjugacy classes of $W$.

\index{PositionClass}
'PositionClass( <WF> , <x> )':\\ for any element <x> in <WF> this returns
     the  number   'i'   such      that  <x>    is  an     element     of
     'ConjugacyClasses(<WF>)[i]'  (to  work  fast,  the classification of
     reflection groups is used).

\index{FusionConjugacyClasses}
'FusionConjugacyClasses( <WF1>, <WF>  )':\\ works in the  same way as for
     groups. See the section 'ReflectionSubCoset'.

'Print( <WF> )':\\ if  '<WF>.name' is bound  then  this is printed,  else
     this function prints  the coset  in a form  which can  be input back
     into \GAP.

\index{InductionTable}
'InductionTable( <HF>, <WF> )':\\
     works in the same way as for groups. It gives the induction table from
     the  Reflection  subcoset  <HF>  to  the  Reflection coset <WF>. If $H
     w\phi$ is a Reflection subcoset of $W \phi$, restriction of characters
     is defined as restriction of functions from $W \phi$ to $H w\phi$, and
     induction  as the adjoint map for  the natural scalar product $\langle
     f,  g\rangle =\frac1{\mid  W\mid}\sum_{v\in W}  f(v \phi)\overline g(v
     \phi)$.

|    gap> W := CoxeterGroup( "A", 4 );;
    gap> Wphi := ReflectionCoset( W, (1,4)(2,3) );
    2A4
    gap> Display(InductionTable(ReflectionSubCoset(Wphi,[2,3 ]),Wphi));
    Induction from 2A2<2,3>.(q-1)(q+1) to 2A4
          |'\|'|111 21 3
    ________________
    11111 |'\|'|  1  . .
    2111  |'\|'|  .  1 .
    221   |'\|'|  1  . .
    311   |'\|'|  1  . 1
    32    |'\|'|  .  . 1
    41    |'\|'|  .  1 .
    5     |'\|'|  .  . 1|

    'InductionTable' and 'FusionConjugacyClasses' work only between cosets.
    If the parent coset is the trivial coset it should still be given as
    a coset and not as a group\:

|    gap> Wphi:=ReflectionCoset(W);
    A4
    gap> L:=ReflectionSubCoset(Wphi,[2,3],LongestCoxeterElement(W));
    A2<2,3>.(q-1)(q+1)
    gap> InductionTable(L,W);
    Error, A2<2,3>.(q-1)(q+1) is a coset but CoxeterGroup("A",4) is not in
    S.operations.FusionConjugacyClasses( S, R ) called from
    FusionConjugacyClasses( u, g ) called from
    InductionTable( L, W ) called from
    main loop
    brk>
    gap> InductionTable(L,Wphi);
    InductionTable(A2<2,3>.(q-1)(q+1), A4)|

\index{ReflectionName}
'ReflectionName( <WF> )':\\ returns a string which describes the isomorphism
   type  of the  group $W\rtimes\langle  F\rangle$, associated  to <WF>, as
   described   in   the   introduction   of   this  Chapter.  An  orbit  of
   $\phi=$<WF>.'phi'  on the components is put in brackets if of length $k$
   greater than $1$, and is preceded by the order of $phi^k$ on it, if this
   is  not $1$.  For example  '\"2(A2xA2)\"' denotes  2 components  of type
   $A_2$  permuted by $\phi$, and such that $phi^2$ induces the non-trivial
   diagram  automorphism on  any of  them, while  '3D4' denotes an orbit of
   length 1 on which $phi$ is of order 3.

|    gap> W:=ReflectionCoset(CoxeterGroup("A",2,"G",2,"A",2),(1,5,2,6));
    2(A2xA2)<1,2,5,6>xG2<3,4>
    gap> ReflectionName( W );
    "2(A2xA2)<1,2,5,6>xG2<3,4>"|

\index{PrintDiagram}
'PrintDiagram( <WF> )':\\ this is  a purely  descriptive routine  (as was
     already  the case for finite  Reflection groups themselves). It prints
     the  Dynkin  diagram  of  'ReflectionGroup(<WF>)'  together  with  the
     information how '<WF>.phi' acts on it. Going from the above example\:

|    gap> PrintDiagram( W );
    phi permutes the next 2 components
    phi^2 acts as (1,2) on the component below
    A2 1 - 2
    A2 5 - 6
    G2 3 >>> 4|

'ChevieClassInfo(   <WF>   )',   see   the   explicit   description   in
"ChevieClassInfo for Reflection cosets".

\index{CharParams}
'CharParams(  <WF> )':\\ This returns appropriate labels for the characters
of the ReflectionCoset. 'CharName' has also a special version for cosets.

\index{GenericOrder}
'GenericOrder(  <WF>, <q> )':\\ Returns the generic order of the associated
     algebraic  group (for a Weyl coset)  or Spetses, using the generalized
     reflection  degrees. We  also have  'TorusOrder(WF,i,q)' which  is the
     same as
     'GenericOrder(SubSpets(WF,[],Representative(ConjugacyClasses(WF)[i])))'.

Note  that  some  functions  for   elements  of  a  Reflection  group  work
naturally   for   elements   of  a   Reflection   coset\:\
'EltWord',    'ReflectionLength', 'ReducedInRightCoset', etc$\ldots$

%%%%%%%%%%%%%%%%%%%%%%%%%%%%%%%%%%%%%%%%%%%%%%%%%%%%%%%%%%%%%%%%%%%%%%%%%
\Section{ChevieCharInfo for reflection cosets}
\index{ChevieCharInfo}

'ChevieCharInfo( <WF> )'

'ChevieCharInfo' gives for a reflection coset <WF> a record similar to what
it  gives for the corresponding group  <W>, excepted that some fields which
do  not  make  sense  are  omitted,  and that two fields record information
allowing to relate characters of the coset to that of the group\:

'charRestriction':\\ records for each character of <WF> the index of the
   character of <W> of which it is an extension.

'nrGroupClasses':\\ records 'NrConjugacyClasses(Group(WF))'.

|    gap> ChevieCharInfo(RootDatum("3D4"));
    rec(
      extRefl := [ 1, 5, 4, 6, 2 ],
      charparams := [ [ [ [  ], [ 4 ] ] ], [ [ [  ], [ 1, 1, 1, 1 ] ] ],
          [ [ [  ], [ 2, 2 ] ] ], [ [ [ 1, 1 ], [ 2 ] ] ],
          [ [ [ 1 ], [ 3 ] ] ], [ [ [ 1 ], [ 1, 1, 1 ] ] ],
          [ [ [ 1 ], [ 2, 1 ] ] ] ],
      charRestrictions := [ 13, 4, 10, 5, 11, 3, 6 ],
      nrGroupClasses := 13,
      b := [ 0, 12, 4, 4, 1, 7, 3 ],
      B := [ 0, 12, 8, 8, 5, 11, 9 ],
      charnames := [ ".4", ".1111", ".22", "11.2", "1.3", "1.111", "1.21"
         ],
      positionId := 1,
      positionDet := 2,
      a := [ 0, 12, 7, 1, 3, 3 ],
      A := [ 0, 12, 11, 5, 9, 9 ] )|

%%%%%%%%%%%%%%%%%%%%%%%%%%%%%%%%%%%%%%%%%%%%%%%%%%%%%%%%%%%%%%%%%%%%%%%%%
\Section{ReflectionType for reflection cosets}
\index{ReflectionType}

'ReflectionType( <WF> )'

returns  the type of the Reflection coset  <WF>. This consists of a list of
records,  one for each orbit of '<WF>.phi' on the irreducible components of
the Dynkin diagram of 'Group(<WF>)', which have two fields\:\\

'orbit':\\ is a list of types of the irreducible components in the orbit.
   These  types are the  same as returned  by the function 'ReflectionType'
   for  an  irreducible  untwisted  reflection  group.  The  components are
   ordered  according to the  action of '<WF>.phi',  so '<WF>.phi' maps the
   generating permutations with indices in the first type to indices in the
   second type in the same order as stored in the type, etc $\ldots$\\

'phi':\\ if $k$ is the number  of  irreducible  components  in  the  orbit,
   this  is the permutation which describes the action of '<WF>.phi'$^k$ on
   the simple roots of the first irreducible component in the orbit.

|    gap> W:=ReflectionCoset(CoxeterGroup("A",2,"A",2), (1,3,2,4));
    2(A2xA2)
    gap> ReflectionType( W );
    [ rec(orbit := [ rec(rank    := 2,
          series  := "A",
          indices := [ 1, 2 ]), rec(rank    := 2,
          series  := "A",
          indices := [ 3, 4 ]) ],
          twist := (1,2)) ]|

%%%%%%%%%%%%%%%%%%%%%%%%%%%%%%%%%%%%%%%%%%%%%%%%%%%%%%%%%%%%%%%%%%%%%%%%%
\Section{ReflectionDegrees for reflection cosets}
\index{ReflectionDegrees}

'ReflectionDegrees( <WF> )'

Let <W> be the Reflection group corresponding to the Reflection coset <WF>,
and  let $V$ be the vector space of dimension 'W.rank' on which <W> acts as
a  reflection group. Let $f_1,\ldots,f_n$ be the basic invariants of <W> on
the  symmetric  algebra  $SV$  of  $V$;  they  can  be  chosen  so they are
eigenvectors  of the matrix 'WF.phiMat'.  The corresponding eigenvalues are
called  the *factors* of $\phi$ acting  on $V$; they characterize the coset
--- they are equal to 1 for the trivial coset. The *generalized degrees* of
<WF>  are the pairs formed of  the reflection degrees and the corresponding
factor.

|    gap> W := CoxeterGroup( "E", 6 );; WF := ReflectionCoset( W );
    E6
    gap> phi := EltWord( W,[6,5,4,2,3,1,4,3,5,4,2,6,5,4,3,1]);;
    gap> HF := ReflectionSubCoset( WF, [ 2..5 ], phi );;
    gap> PrintDiagram( HF );
    phi acts as (2,3,5) on the component below
    D4 2
        \
         4 - 5
        /
       3
    gap> ReflectionDegrees( HF );
    [ [ 1, E(3) ], [ 1, E(3)^2 ], [ 2, 1 ], [ 4, E(3) ], [ 6, 1 ],
      [ 4, E(3)^2 ] ]|

%%%%%%%%%%%%%%%%%%%%%%%%%%%%%%%%%%%%%%%%%%%%%%%%%%%%%%%%%%%%%%%%%%%%%%%%%
\Section{Twistings}
\index{Twistings}

'Twistings( <W>, <L> )'

<W>  should be a Reflection group record  or a Reflection coset record, and
<L> should be a reflection subgroup of <W> (or of 'Group(<W>)' for a coset),
or  a sublist of  the generating reflections  of <W> (resp. 'Group(W)'), in
which case the call is the same as
'Twistings(<W>,ReflectionSubgroup(<W>,<L>))' (resp.
'Twistings(<W>,ReflectionSubgroup(Group(<W>),<L>))').

The  function returns  a list  of representatives,  up to <W>-conjugacy, of
reflection sub-cosets of <W> whose reflection group is <L>.

|    gap> W:=ComplexReflectionGroup(3,3,4);
    ComplexReflectionGroup(3,3,4)
    gap> Twistings(W,[1..3]);
    [ G333.(q-1), 3'G333<1,2,3,76>.(q-E3^2), 3G333<1,2,3,76>.(q-E3) ]|

%%%%%%%%%%%%%%%%%%%%%%%%%%%%%%%%%%%%%%%%%%%%%%%%%%%%%%%%%%%%%%%%%%%%%%%%%
\Section{ChevieClassInfo for Reflection cosets}
\index{ChevieClassInfo}

'ChevieClassInfo( <WF> )'

returns  information about  the conjugacy  classes of  the Reflection coset
<WF>.  The result is a record with three components\:\ 'classtext' contains
a list of reduced words for the representatives in
'ConjugacyClasses(<WF>)', 'classnames' contains corresponding names for the
classes, and 'classparams' gives corresponding parameters for the classes.

|    gap> W:=ReflectionCoset(ComplexReflectionGroup(14));
    G14
    gap> Rphi:=ReflectionSubCoset(W,[2,4],Group(W).1);
    2G5(ER(6))<2,4>
    gap> ChevieClassInfo(Rphi);
    rec(
      classtext :=
       [ [  ], [ 2, 4, 4, 2, 4, 4, 2 ], [ 2, 4, 4, 2, 4, 4, 2, 2 ],
          [ 2 ], [ 2, 2, 4, 2, 2 ], [ 2, 2, 4, 4, 2, 2 ], [ 2, 4 ],
          [ 2, 4, 2 ], [ 4, 2, 4, 2 ] ],
      classes := [ 12, 6, 6, 6, 12, 6, 6, 6, 12 ],
      orders := [ 2, 24, 24, 24, 6, 8, 24, 8, 6 ],
      classnames := [ "", "1221221", "12212211", "1", "11211", "112211",
          "12", "121", "2121" ] )|

%%%%%%%%%%%%%%%%%%%%%%%%%%%%%%%%%%%%%%%%%%%%%%%%%%%%%%%%%%%%%%%%%%%%%%%%%
\Section{CharTable for Reflection cosets}
\index{CharTable}

'CharTable(  <WF> )'

This function returns the character table of the Reflection coset <WF> (see
also  the introduction  of this  Chapter). We  call ``characters\'\' of the
Reflection   coset  $WF$  with  corresponding   Reflection  group  $W$  the
restriction  to  $W  \phi$  of  a  set  containing  one  extension  of each
$\phi$-invariant character of $W$ to the semidirect product of $W$ with the
cyclic  group generated  by $\phi$.  The choice  of extension is always the
same for a given coset, but rather arbitrary in general; for Weyl cosets it
is the \"preferred extension\" of Lusztig.

The returned record contains almost all components present in the character
table of a Reflection group. But if $\phi$ is not trivial then there are no
components 'powermap' (since powers of elements in the coset need not be in
the  coset)  and  'orders'  (if  you  really  need  them, use 'MatXPerm' to
determine the order of elements in the coset).

|    gap> W := ReflectionCoset( CoxeterGroup( "D", 4 ), (1,2,4) );
    3D4
    gap> Display( CharTable( W ) );
    3D4

           2  2   2     2      2  2      3      3
           3  1   1     1      .  .      1      1

             C3 ~A2 C3+A1 ~A2+A1 F4 ~A2+A2 F4(a1)

    .4        1   1     1      1  1      1      1
    .1111    -1   1     1     -1  1      1      1
    .22       .   2     2      . -1     -1     -1
    11.2      .   .     .      . -1      3      3
    1.3       1   1    -1     -1  .     -2      2
    1.111    -1   1    -1      1  .     -2      2
    1.21      .   2    -2      .  .      2     -2
    |

%%%%%%%%%%%%%%%%%%%%%%%%%%%%%%%%%%%%%%%%%%%%%%%%%%%%%%%%%%%%%%%%%%%%%%%%%
