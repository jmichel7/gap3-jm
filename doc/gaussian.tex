%%%%%%%%%%%%%%%%%%%%%%%%%%%%%%%%%%%%%%%%%%%%%%%%%%%%%%%%%%%%%%%%%%%%%%%%%%%%%
%%
%A  gaussian.tex                GAP documentation            Martin Schoenert
%%
%A  @(#)$Id: gaussian.tex,v 1.1.1.1 1996/12/11 12:36:45 werner Exp $
%%
%Y  Copyright 1990-1992,  Lehrstuhl D fuer Mathematik,  RWTH Aachen,  Germany
%%
%%  This file describes Gaussian integers and rationals and their  functions.
%%
%H  $Log: gaussian.tex,v $
%H  Revision 1.1.1.1  1996/12/11 12:36:45  werner
%H  Preparing 3.4.4 for release
%H
%H  Revision 3.6  1993/03/11  10:33:40  fceller
%H  added 'EuclideanQuotient', 'EuclideanRemainder' and 'QuotientRemainder'
%H
%H  Revision 3.5  1993/02/01  13:25:57  felsch
%H  example fixed
%H
%H  Revision 3.4  1992/04/06  16:54:53  martin
%H  fixed some more typos
%H
%H  Revision 3.3  1992/04/02  21:06:23  martin
%H  changed *domain functions* to *set theoretic functions*
%H
%H  Revision 3.2  1992/03/11  15:51:46  sam
%H  renamed chapter "Number Fields" to "Subfields of Cyclotomic Fields"
%H
%H  Revision 3.1  1992/01/08  11:41:10  martin
%H  initial revision under RCS
%H
%%
\Chapter{Gaussians}%
\index{type!gaussian rationals}
\index{type!gaussian integers}

If we adjoin a square root of -1, usually denoted by $i$, to the field of
rationals we obtain a field that is an extension of degree 2.  This field
is called the *Gaussian rationals* and its ring of integers is called the
*Gaussian integers*, because C.F. Gauss was the first to study them.

In {\GAP} Gaussian rationals are written in the  form '<a>  + <b>\*E(4)',
where <a> and  <b> are rationals,  because 'E(4)'  is {\GAP}\'s name  for
$i$.  Because 1 and $i$ form an integral base  the Gaussian  integers are
written in the form '<a> + <b>\*E(4)', where <a> and <b> are integers.

The first sections in this  chapter describe the operations applicable to
Gaussian rationals (see "Comparisons  of  Gaussians" and "Operations  for
Gaussians").

The next sections describe the functions that test whether an object is a
Gaussian rational or integer (see "IsGaussRat" and "IsGaussInt").

The {\GAP} object 'GaussianRationals' is the field domain of all Gaussian
rationals,  and the object 'GaussianIntegers'  is the ring domain of  all
Gaussian  integers.  All  set theoretic functions are applicable to those
two domains (see chapter "Domains" and "Set Functions for Gaussians").

The Gaussian rationals form a field so all field functions, e.g., 'Norm',
are  applicable to the domain  'GaussianRationals'  and its elements (see
chapter "Fields" and "Field Functions for Gaussian Rationals").

The Gaussian integers  form a Euclidean ring so all ring functions, e.g.,
'Factors', are applicable to  'GaussianIntegers'  and its  elements  (see
chapter   "Rings",   "Ring   Functions  for   Gaussian   Integers",   and
"TwoSquares").

The field of Gaussian  rationals is  just a  special  case of  cyclotomic
fields,  so everything that applies to  those fields also  applies  to it
(see chapters "Cyclotomics" and "Subfields of Cyclotomic Fields").

All functions are in the library file 'LIBNAME/\"gaussian.g\"'.

%%%%%%%%%%%%%%%%%%%%%%%%%%%%%%%%%%%%%%%%%%%%%%%%%%%%%%%%%%%%%%%%%%%%%%%%%
\Section{Comparisons of Gaussians}%
\index{equality!of gaussians}%
\index{ordering!of gaussians}

'<x> = <y>' \\
'<x> \<> <y>'

The  equality operator evaluates  to 'true' if  the two Gaussians <x> and
<y> are  equal, and to 'false' otherwise.   The inequality operator '\<>'
evaluates to 'true' if the  two Gaussians <x> and <y>  are not equal, and
to 'false' otherwise.  It is also possible  to compare a Gaussian with an
object of another type, of course they are never equal.

Two Gaussians '<a>  +  <b>\*E(4)' and  '<c> + <d>\*E(4)'   are considered
equal if '<a> = <c>' and '<b> = <d>'.

|    gap> 1 + E(4) = 2 / (1 - E(4));
    true
    gap> 1 + E(4) = 1 - E(4);
    false
    gap> 1 + E(4) = E(6);
    false |

'<x> \< <y>' \\
'<x> \<= <y>' \\
'<x> > <y>' \\
'<x> >= <y>'

The operators  '\<',  '\<=', '>',  and  '>='  evaluate to  'true' if  the
Gaussian  <x> is  less  than, less  than or equal   to, greater than, and
greater  than or equal to  the  Gaussian <y>,  and  to 'false' otherwise.
Gaussians can   also be compared  to objects   of other types,  they  are
smaller than anything else, except other cyclotomics (see "Comparisons of
Cyclotomics").

A Gaussian '<a>  +  <b>\*E(4)' is considered  less  than another Gaussian
'<c> + <d>\*E(4)' if <a> is less than <c>, or if <a> is  equal to <c> and
<b> is less than <d>.

|    gap> 1 + E(4) < 2 + E(4);
    true
    gap> 1 + E(4) < 1 - E(4);
    false
    gap> 1 + E(4) < 1/2;
    false |

%%%%%%%%%%%%%%%%%%%%%%%%%%%%%%%%%%%%%%%%%%%%%%%%%%%%%%%%%%%%%%%%%%%%%%%%%
\Section{Operations for Gaussians}%
\index{sum!of gaussians}\index{difference!of gaussians}%
\index{product!of gaussians}\index{quotient!of gaussians}%
\index{power!of gaussians}

'<x>  +  <y>' \\
'<x>  -  <y>' \\
'<x> \*\ <y>' \\
'<x>  /  <y>'

The  operators '+', '-', '\*', and  '/' evaluate  to the sum, difference,
product, and quotient of the two Gaussians <x> and <y>.  Of course either
operand  may also be an  ordinary rational (see "Rationals"), because the
rationals are embedded  into the Gaussian rationals.   On  the other hand
the Gaussian  rationals  are embedded  into other   cyclotomic fields, so
either operand may also be a cyclotomic (see "Cyclotomics").  Division by
0 is as usual an error.

'<x> \^\ <n>'

The operator '\^' evaluates to the  <n>-th power of the Gaussian rational
<x>.  If  <n> is positive, the power  is defined as  the <n>-fold product
'<x>\*<x>\*...<x>';   if <n> is    negative,   the power is  defined   as
'(1/<x>)\^(-<n>)'; and if <n> is zero, the power is 1, even if <x> is 0.

|    gap> (1 + E(4)) * (E(4) - 1);
    -2 |

%%%%%%%%%%%%%%%%%%%%%%%%%%%%%%%%%%%%%%%%%%%%%%%%%%%%%%%%%%%%%%%%%%%%%%%%%
\Section{IsGaussRat}%
\index{test!for gaussian rational}

'IsGaussRat( <obj> )'

'IsGaussRat' returns 'true' if <obj>, which may be an object of arbitrary
type, is a Gaussian rational and 'false' otherwise.  Will signal an error
if <obj> is an unbound variable.

|    gap> IsGaussRat( 1/2 );
    true
    gap> IsGaussRat( E(4) );
    true
    gap> IsGaussRat( E(6) );
    false
    gap> IsGaussRat( true );
    false |

'IsGaussInt' can be used to test whether an object is a  Gaussian integer
(see "IsGaussInt").

%%%%%%%%%%%%%%%%%%%%%%%%%%%%%%%%%%%%%%%%%%%%%%%%%%%%%%%%%%%%%%%%%%%%%%%%%
\Section{IsGaussInt}%
\index{test!for gaussian integer}

'IsGaussInt( <obj> )'

'IsGaussInt' returns 'true' if <obj>, which may be an object of arbitrary
type, is  a Gaussian integer, and  false otherwise.  Will signal an error
if <obj> is an unbound variable.

|    gap> IsGaussInt( 1 );
    true
    gap> IsGaussInt( E(4) );
    true
    gap> IsGaussInt( 1/2 + 1/2*E(4) );
    false
    gap> IsGaussInt( E(6) );
    false |

'IsGaussRat' can be used to test whether an object is a Gaussian rational
(see "IsGaussRat").

%%%%%%%%%%%%%%%%%%%%%%%%%%%%%%%%%%%%%%%%%%%%%%%%%%%%%%%%%%%%%%%%%%%%%%%%%
\Section{Set Functions for Gaussians}%
\index{membership test!for gaussians}\index{in!for gaussians}%
\index{Random!for gaussians}%

As  already mentioned in the  introduction of  this  chapter the  objects
'GaussianRationals'  and 'GaussianIntegers' are the  domains of  Gaussian
rationals and integers respectively.  All  set theoretic functions, i.e.,
'Size' and 'Intersection',  are applicable to  these  domains  and  their
elements (see chapter "Domains").  There does not seem to be an important
use of  this however.  All functions not  mentioned here are  not treated
specially, i.e., they are implemented by the  default function  mentioned
in the respective section.

\vspace{5mm}
'in'

The  membership test   for     Gaussian rationals  is  implemented    via
'IsGaussRat' ("IsGaussRat").  The  membership test for  Gaussian integers
is implemented via 'IsGaussInt' (see "IsGaussInt").

\vspace{5mm}
'Random'

A random Gaussian rational '<a> + <b>\*E(4)' is computed by combining two
random  rationals  <a>  and  <b>  (see  "Set  Functions  for Rationals").
Likewise  a  random  Gaussian  integer '<a> +  <b>\*E(4)' is  computed by
combining  two random  integers  <a>  and  <b> (see  "Set  Functions  for
Integers").

|    gap> Size( GaussianRationals );
    "infinity"
    gap> Intersection( GaussianIntegers, [1,1/2,E(4),-E(6),E(4)/3] );
    [ 1, E(4) ] |

%%%%%%%%%%%%%%%%%%%%%%%%%%%%%%%%%%%%%%%%%%%%%%%%%%%%%%%%%%%%%%%%%%%%%%%%%
\Section{Field Functions for Gaussian Rationals}%
\index{Conjugates!for gaussians}%
\index{Norm!for gaussians}%
\index{Trace!for gaussians}

As already mentioned  in the introduction of this chapter,  the domain of
Gaussian  rationals  is a  field.   Therefore  all  field  functions  are
applicable to this domain and  its elements (see chapter "Fields").  This
section gives  further comments on the definitions and implementations of
those  functions  for  the the  Gaussian  rationals.   All  functions not
mentioned here are not  treated specially, i.e., they are implemented  by
the default function mentioned in the respective section.

\vspace{5mm}
'Conjugates'

The  field  of Gaussian  rationals  is  an  extension of degree 2 of  the
rationals,  its prime field.  Therefore there is one further conjugate of
every element '<a> + <b>\*E(4)', namely '<a> - <b>\*E(4)'.

\vspace{5mm}
'Norm', 'Trace'

According to the definition  of conjugates above, the  norm of a Gaussian
rational    '<a> + <b>\*E(4)'  is  '<a>\^2  + <b>\^2'   and  the trace is
'2\*<a>'.

%%%%%%%%%%%%%%%%%%%%%%%%%%%%%%%%%%%%%%%%%%%%%%%%%%%%%%%%%%%%%%%%%%%%%%%%%
\Section{Ring Functions for Gaussian Integers}%
\index{IsUnit!for gaussians}\index{Units!for gaussians}%
\index{IsAssociated!for gaussians}\index{Associates!for gaussians}%
\index{StandardAssociate!for gaussians}%
\index{EuclideanDegree!for gaussians}%
\index{EuclideanRemainder!for gaussians}%
\index{EuclideanQuotient!for gaussians}%
\index{QuotientRemainder!for gaussians}%
\index{IsPrime!for gaussians}\index{IsIrreducible!for gaussians}%
\index{Factors!for gaussians}

As already mentioned in  the introduction to  this chapter,  the  ring of
Gaussian integers  is a Euclidean ring.  Therefore all ring functions are
applicable to this  ring and  its  elements (see chapter "Rings").   This
section gives further comments on  the definitions and implementations of
those functions for the Gaussian integers.   All functions  not mentioned
here are not treated specially, i.e., they are implemented by the default
function mentioned in the respective section.

\vspace{5mm}
'IsUnit', 'Units', 'IsAssociated', 'Associates'

The units of 'GaussianIntegers' are '[ 1, E(4), -1, -E(4) ]'.

\vspace{5mm}
'StandardAssociate'

The standard associate  of  a  Gaussian  integer  <x> is the   associated
element <y> of <x> that lies in the first  quadrant of the complex plane.
That  is <y>  is  that element  from '<x> \*\ [1,-1,E(4),-E(4)]' that has
positive real part and nonnegative imaginary part.

\vspace{5mm}
'EuclideanDegree'

The Euclidean degree of a Gaussian  integer <x> is the product of <x> and
its complex conjugate.

\vspace{5mm}
'EuclideanRemainder'

Define the integer part <i> of the quotient of  <x>  and <y> as the point
of  the lattice spanned by 1  and 'E(4)' that  lies next to the  rational
quotient of <x> and <y>, rounding towards the origin if there are several
such  points.  Then 'EuclideanRemainder( <x>, <y> )' is defined as '<x> -
<i> \*\ <y>'.  With this definition the ordinary Euclidean algorithm  for
the greatest common divisor works, whereas it does not work if you always
round towards the origin.

\vspace{5mm}
'EuclideanQuotient'

The  Euclidean quotient of two  Gaussian integers  <x>  and  <y>  is  the
quotient of $w$ and <y>, where $w$  is the difference between <x> and the
Euclidean remainder of <x> and <y>.

\vspace{5mm}
'QuotientRemainder'

'QuotientRemainder' uses 'EuclideanRemainder' and 'EuclideanQuotient'.

\vspace{5mm}
'IsPrime', 'IsIrreducible'

Since the Gaussian integers are a Euclidean ring, primes and irreducibles
are equivalent.  The primes are the elements '1 + E(4)' and '1 - E(4)' of
norm 2, the elements  '<a> + <b>\*E(4)'  and  '<a> - <b>\*E(4)'  of  norm
'<p> = <a>\^2 + <b>\^2' with <p> a  rational prime congruent  to 1 mod 4,
and the elements <p> of norm '<p>\^2' with <p> a rational prime congruent
to 3 mod 4.

\vspace{5mm}
'Factors'

The list returned by 'Factors'  is sorted according to  the norms of  the
primes, and among those of equal norm with respect to '\<'.  All elements
in  the  list  are  standard  associates,  except  the  first,  which  is
multiplied by a unit as necessary.

The  above characterization already shows  how one  can factor a Gaussian
integer.  First  compute the norm of the  element, factor this  norm over
the rational integers and then split 2 and the  primes congruent to 1 mod
4 with 'TwoSquares' (see "TwoSquares").

|    gap> Factors( GaussianIntegers, 30 );
    [ -1-E(4), 1+E(4), 3, 1+2*E(4), 2+E(4) ] |

%%%%%%%%%%%%%%%%%%%%%%%%%%%%%%%%%%%%%%%%%%%%%%%%%%%%%%%%%%%%%%%%%%%%%%%%%
\Section{TwoSquares}%
\index{representation!as a sum of two squares}

'TwoSquares( <n> )'

'TwoSquares' returns a list of two integers $x\<=y$  such that the sum of
the squares of $x$ and $y$ is equal to the nonnegative integer <n>, i.e.,
$n = x^2+y^2$.  If no such representation exists 'TwoSquares' will return
'false'.  'TwoSquares' will return a representation  for which the gcd of
$x$ and  $y$  is  as small  as  possible.    If there are    several such
representations, it is not specified which one 'TwoSquares' returns.

Let $a$ be the product of all maximal powers of primes of the form $4k+3$
dividing  $n$.  A representation of $n$ as a sum of two squares exists if
and only if $a$ is a perfect square.  Let $b$ be the maximal power of $2$
dividing  $n$, or  its  half, whichever is a  perfect  square.   Then the
minimal possible gcd of $x$ and $y$ is the square root $c$ of $a b$.  The
number  of  different minimal representations with $x\<=y$  is $2^{l-1}$,
where $l$ is the number of different prime factors  of the form $4k+1$ of
$n$.

|    gap> TwoSquares( 5 );
    [ 1, 2 ]
    gap> TwoSquares( 11 );
    false        # no representation exists
    gap> TwoSquares( 16 );
    [ 0, 4 ]
    gap> TwoSquares( 45 );
    [ 3, 6 ]        # 3 is the minimal possible gcd because 9 divides 45
    gap> TwoSquares( 125 );
    [ 2, 11 ]        # not [ 5, 10 ] because this has not minimal gcd
    gap> TwoSquares( 13*17 );
    [ 5, 14 ]        # [10,11] would be the other possible representation
    gap> TwoSquares( 848654483879497562821 );
    [ 6305894639, 28440994650 ]        # 848654483879497562821 is prime |

%%%%%%%%%%%%%%%%%%%%%%%%%%%%%%%%%%%%%%%%%%%%%%%%%%%%%%%%%%%%%%%%%%%%%%%%%
%%
%E  Emacs . . . . . . . . . . . . . . . . . . . . . local Emacs variables
%%
%%  Local Variables:
%%  mode:               outline
%%  outline-regexp:     "\\\\Chapter\\|\\\\Section"
%%  fill-column:        73
%%  eval:               (hide-body)
%%  End:
%%



