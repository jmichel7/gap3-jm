%%%%%%%%%%%%%%%%%%%%%%%%%%%%%%%%%%%%%%%%%%%%%%%%%%%%%%%%%%%%%%%%%%%%%%%%%%%%%
%%
%A  cyclotom.tex                GAP documentation               Thomas Breuer
%%
%A  @(#)$Id: cyclotom.tex,v 1.2 1997/02/24 10:33:19 gap Exp $
%%
%Y  Copyright 1990-1992,  Lehrstuhl D fuer Mathematik,  RWTH Aachen,  Germany
%%
%H  $Log: cyclotom.tex,v $
%H  Revision 1.2  1997/02/24 10:33:19  gap
%H  vfelsch documented 'ER(<N>)' for negative <N>
%H
%H  Revision 1.1.1.1  1996/12/11 12:36:44  werner
%H  Preparing 3.4.4 for release
%H
%H  Revision 3.11  1993/02/19  10:48:42  gap
%H  adjustments in line length and spelling
%H
%H  Revision 3.10  1993/02/12  15:45:43  felsch
%H  more examples fixed
%H
%H  Revision 3.9  1993/02/01  13:16:16  felsch
%H  examples fixed
%H
%H  Revision 3.8  1992/04/07  23:05:55  martin
%H  changed the author line
%H
%H  Revision 3.7  1992/03/27  13:56:54  sam
%H  removed "'" in index entries
%H
%H  Revision 3.6  1992/03/11  15:48:05  sam
%H  renamed chapter "Number Fields" to "Subfields of Cyclotomic Fields"
%H
%H  Revision 3.5  1992/02/13  15:37:26  sam
%H  renamed 'Automorphisms' to 'GaloisGroup'
%H
%H  Revision 3.4  1992/01/14  14:37:27  martin
%H  changed an overlong line
%H
%H  Revision 3.3  1992/01/14  14:01:12  sam
%H  adjusted citations
%H
%H  Revision 3.2  1991/12/30  09:37:48  martin
%H  changed two section titles slightly
%H
%H  Revision 3.1  1991/12/30  08:07:00  sam
%H  initial revision under RCS
%H
%%
\Chapter{Cyclotomics}\index{type!cyclotomic}\index{irrationalities}

{\GAP} allows computations in abelian extension fields of the rational
field  $Q$, i.e., fields with  abelian Galois group  over $Q$.   These
fields  are  described  in  chapter "Subfields of Cyclotomic  Fields".
They are subfields  of *cyclotomic fields* $Q_n = Q(e_n)$ where $e_n =
e^{\frac{2\pi i}{n}}$  is a primitive  $n$--th root  of  unity.  Their
elements are called *cyclotomics*.

The  internal representation  of a  cyclotomic  does not refer  to the
smallest number field but the smallest cyclotomic  field containing it
(the so--called *conductor*).  This is because it is easy to embed two
cyclotomic fields in a larger one that contains both, i.e., there is a
natural way to get the sum or the product of two arbitrary cyclotomics
as  element  of a cyclotomic field.   The  disadvantage  is  that  the
arithmetical operations are too expensive to do arithmetics  in number
fields, e.g.,  calculations in a matrix ring over a number field.  But
it  suffices to deal  with irrationalities  in  character  tables (see
"Character Tables").  (And in fact, the comfortability of working with
the natural embeddings is used  there in many situations which did not
actually afford it \ldots)

All functions that take  a field  extension as ---possibly optional---
argument, e.g., 'Trace' or 'Coefficients'  (see chapter "Fields"), are
described in chapter "Subfields of Cyclotomic Fields".

This chapter informs about:\\
    the representation of cyclotomics in {\GAP} (see
       "More about Cyclotomics"),\\
    access to the internal data (see "NofCyc", "CoeffsCyc")\\
    integral elements of number fields (see "Cyclotomic Integers",
       "IntCyc", "RoundCyc"),\\
    characteristic functions (see "IsCyc", "IsCycInt"),\\
    comparison and arithmetical operations of cyclotomics (see
    "Comparisons of Cyclotomics", "Operations for Cyclotomics"),\\
    functions concerning Galois conjugacy of cyclotomics (see "GaloisCyc",
       "StarCyc"), or lists of them (see "GaloisMat", "RationalizedMat"),\\
    some special cyclotomics, as defined in \cite{CCN85}
       (see "ATLAS irrationalities", "Quadratic")

The external functions are in the file 'LIBNAME/\"cyclotom.g\"'.

%%%%%%%%%%%%%%%%%%%%%%%%%%%%%%%%%%%%%%%%%%%%%%%%%%%%%%%%%%%%%%%%%%%%%%%%%
\Section{More about Cyclotomics}\index{cyclotomic field elements}
\index{E}\index{roots of unity}

Elements  of  number  fields  (see  chapter  "Subfields of  Cyclotomic
Fields"),  cyclotomics  for  short,  are  arithmetical  objects   like
rationals  and  finite  field  elements;  they are not implemented  as
records ---like  groups--- or e.g. with respect  to a character  table
(although  character  tables  may be the main  interest for cyclotomic
arithmetics).

'E( <n> )'

returns  the  primitive <n>-th  root  of unity  $e_n  =  e^{\frac{2\pi
i}{n}}$.   Cyclotomics  are   usually  entered   as   (and  irrational
cyclotomics  are  always displayed  as) sums of  roots of  unity  with
rational   coefficients.    (For  special   cyclotomics,   see  "ATLAS
irrationalities".)

|    gap> E(9); E(9)^3; E(6); E(12) / 3; 
    -E(9)^4-E(9)^7    # the root needs not to be an element of the base
    E(3)
    -E(3)^2
    -1/3*E(12)^7|

For the  representation  of  cyclotomics  one  has  to recall that the
cyclotomic  field  $Q_n  =  Q(e_n)$  is a  vector  space  of dimension
$\varphi(n)$  over  the rationals  where  $\varphi$  denotes  Euler\'s
phi-function (see "Phi").

Note that the set  of all $n$-th  roots of unity is linearly dependent
for $n > 1$, so  multiplication is not the multiplication of the group
ring $Q\langle e_n \rangle$; given a $Q$-basis of $Q_n$ the  result of
the  multiplication  (computed  as  multiplication of  polynomials  in
$e_n$, using $(e_n)^n = 1$) will be converted to the base.

|    gap> E(5) * E(5)^2; ( E(5) + E(5)^4 ) * E(5)^2;
    E(5)^3
    E(5)+E(5)^3
    gap> ( E(5) + E(5)^4 ) * E(5);
    -E(5)-E(5)^3-E(5)^4|

Cyclotomics  are always  represented in  the smallest cyclotomic field
they are contained in.  Together with  the choice of a fixed base this
means that two cyclotomics are equal if and only if  they are  equally
represented.

Addition  and multiplication of  two  cyclotomics represented in $Q_n$
and $Q_m$, respectively,  is computed in the smallest cyclotomic field
containing  both\:\ $Q_{'Lcm'(n,m)}$.  Conversely,  if  the result  is
contained in a smaller cyclotomic field the representation  is reduced
to the minimal such field.

The  base,  the  base conversion  and the  reduction  to  the  minimal
cyclotomic field  are  described in~\cite{Zum89},  more about the base
can be found in "ZumbroichBase".

Since $n$  must  be  a 'short integer',  the  maximal cyclotomic field
implemented in {\GAP} is not really the  field $Q^{ab}$.  The  biggest
allowed (though not very useful) $n$ is 65535.

There  is  a  global  variable 'Cyclotomics'  in {\GAP}, a record that
stands for  the domain of  all  cyclotomics (see chapter "Subfields of
Cyclotomic Fields").

%%%%%%%%%%%%%%%%%%%%%%%%%%%%%%%%%%%%%%%%%%%%%%%%%%%%%%%%%%%%%%%%%%%%%%%%%
\Section{Cyclotomic Integers}

A  cyclotomic is called  *integral*  or  *cyclotomic integer*  if  all
coefficients  of its minimal  polynomial are integers. Since  the base
used  is  an  integral  base  (see "ZumbroichBase"),  the  subring  of
cyclotomic  integers  in  a  cyclotomic  field  is  formed   by  those
cyclotomics  which have not only rational but integral coefficients in
their representation as  sums of roots of unity.   For example, square
roots   of   integers   are   cyclotomic    integers    (see    "ATLAS
irrationalities"),  any  root  of  unity   is  a  cyclotomic  integer,
character values  are always cyclotomic  integers,  but  all rationals
which are not integers are not cyclotomic integers.  (See "IsCycInt")

|    gap> ER( 5 );                # The square root of 5 is a cyclotomic
    E(5)-E(5)^2-E(5)^3+E(5)^4    # integer, it has integral coefficients.
    gap> 1/2 * ER( 5 );          # This is not a cyclotomic integer, \ldots
    1/2*E(5)-1/2*E(5)^2-1/2*E(5)^3+1/2*E(5)^4
    gap> 1/2 * ER( 5 ) - 1/2;    # \ldots but this is one.
    E(5)+E(5)^4|

%%%%%%%%%%%%%%%%%%%%%%%%%%%%%%%%%%%%%%%%%%%%%%%%%%%%%%%%%%%%%%%%%%%%%%%%%
\Section{IntCyc}

'IntCyc( <z> )'

returns  the  cyclotomic  integer  (see  "Cyclotomic  Integers")  with
Zumbroich base coefficients  (see "ZumbroichBase") 'List( <zumb>, x ->
Int( x ) )' where <zumb> is the  vector of Zumbroich base coefficients
of the cyclotomic <z>; see also "RoundCyc".

|    gap> IntCyc( E(5)+1/2*E(5)^2 ); IntCyc( 2/3*E(7)+3/2*E(4) );
    E(5)
    E(4)|

%%%%%%%%%%%%%%%%%%%%%%%%%%%%%%%%%%%%%%%%%%%%%%%%%%%%%%%%%%%%%%%%%%%%%%%%%
\Section{RoundCyc}

'RoundCyc( <z> )'

returns  the  cyclotomic  integer  (see  "Cyclotomic  Integers")  with
Zumbroich base coefficients (see "ZumbroichBase") 'List( <zumb>,  x ->
Int(  x+1/2  )  )' where  <zumb>  is  the  vector  of  Zumbroich  base
coefficients of the cyclotomic <z>; see also "IntCyc".

|    gap> RoundCyc( E(5)+1/2*E(5)^2 ); RoundCyc( 2/3*E(7)+3/2*E(4) );
    E(5)+E(5)^2
    -2*E(28)^3+E(28)^4-2*E(28)^11-2*E(28)^15-2*E(28)^19-2*E(28)^23
     -2*E(28)^27|

%%%%%%%%%%%%%%%%%%%%%%%%%%%%%%%%%%%%%%%%%%%%%%%%%%%%%%%%%%%%%%%%%%%%%%%%%%%%%
\Section{IsCyc}\index{test!for a cyclotomic}

'IsCyc( <obj> )'

returns 'true' if <obj> is a cyclotomic, and 'false'  otherwise.  Will
signal an error if <obj> is an unbound variable.

|    gap> IsCyc( 0 ); IsCyc( E(3) ); IsCyc( 1/2 * E(3) ); IsCyc( IsCyc );
    true
    true
    true
    false|

'IsCyc' is an internal function.

%%%%%%%%%%%%%%%%%%%%%%%%%%%%%%%%%%%%%%%%%%%%%%%%%%%%%%%%%%%%%%%%%%%%%%%%%%%%%
\Section{IsCycInt}\index{test!for a cyclotomic integer}

'IsCycInt( <obj> )'

returns  'true'  if  <obj>  is a cyclotomic integer  (see  "Cyclotomic
Integers"), 'false'  otherwise.  Will signal  an  error if <obj> is an
unbound variable.

|    gap> IsCycInt( 0 ); IsCycInt( E(3) ); IsCycInt( 1/2 * E(3) );
    true
    true
    false|

'IsCycInt' is an internal function.

%%%%%%%%%%%%%%%%%%%%%%%%%%%%%%%%%%%%%%%%%%%%%%%%%%%%%%%%%%%%%%%%%%%%%%%%%%%%%
\Section{NofCyc}

'NofCyc( <z> )'\\
'NofCyc( <list> )'

returns the smallest positive integer $n$ for which the cyclotomic <z>
is resp.\ for  which all  cyclotomics in the list <list> are contained
in $Q_n = Q( e^{\frac{2 \pi i}{n}} ) = Q( 'E(<n>)' )$.

|    gap> NofCyc( 0 ); NofCyc( E(10) ); NofCyc( E(12) );
    1
    5
    12|

'NofCyc' is an internal function.

%%%%%%%%%%%%%%%%%%%%%%%%%%%%%%%%%%%%%%%%%%%%%%%%%%%%%%%%%%%%%%%%%%%%%%%%%%%%%
\Section{CoeffsCyc}\index{coefficients!for cyclotomics}

'CoeffsCyc( <z>, <n> )'

If <z> is a cyclotomic which is contained in  $Q_n$, 'CoeffsCyc(  <z>,
<n> )' returns a list <cfs> of length  <n> where the entry at position
<i>  is  the   coefficient   of   $'E(<n>)'^{i-1}$   in  the  internal
representation  of  <z> as element  of the cyclotomic field $Q_n$ (see
"More  about  Cyclotomics",  "ZumbroichBase")\:\
$<z> = <cfs>[1] + <cfs>[2]\ 'E(<n>)'^1 + \ldots + <cfs>[n]\
'E(<n>)'^{n-1}$.

*Note*  that  all  positions which do not belong to  base  elements of
$Q_n$ contain zeroes.

|    gap> CoeffsCyc( E(5), 5 ); CoeffsCyc( E(5), 15 );
    [ 0, 1, 0, 0, 0 ]
    [ 0, 0, 0, 0, 0, 0, 0, 0, -1, 0, 0, 0, 0, -1, 0 ]
    gap> CoeffsCyc( 1+E(3), 9 ); CoeffsCyc( E(5), 7 );
    [ 0, 0, 0, 0, 0, 0, -1, 0, 0 ]
    Error, no representation of <z> in 7th roots of unity|

'CoeffsCyc' calls the internal function 'COEFFSCYC'\:

'COEFFSCYC( <z> )'

is equivalent to 'CoeffsCyc( <z>, NofCyc( <z> ) )', see "NofCyc".

%%%%%%%%%%%%%%%%%%%%%%%%%%%%%%%%%%%%%%%%%%%%%%%%%%%%%%%%%%%%%%%%%%%%%%%%%%%%%
\Section{Comparisons of Cyclotomics}\index{operators!for cyclotomics}

To compare  cyclotomics, the operators '\<', '\<=', '=', '>=', '>' and
'\<>' can  be used, the result will be  'true' if the first operand is
smaller, smaller or equal, equal, larger or equal, larger, or inequal,
respectively, and 'false' otherwise.

Cyclotomics are ordered as follows\:\ 
The relation between rationals is as usual,  and rationals are smaller
than irrational cyclotomics. For two irrational cyclotomics <z1>, <z2>
which  lie in different  minimal cyclotomic  fields,  we have $<z1> \<
<z2>$  if  and   only  if   $'NofCyc(<z1>)'  \<  'NofCyc(<z2>)'$);  if
$'NofCyc(<z1>)'  = 'NofCyc(<z2>)'$), that one  is smaller that has the
smaller coefficient vector, i.e.,  $<z1>  \leq <z2>$  if  and  only if
$'COEFFSCYC(<z1>)' \leq 'COEFFSCYC(<z2>)'$.

You  can compare  cyclotomics with objects of other types; all objects
which are not cyclotomics are larger than cyclotomics.

|    gap> E(5) < E(6);     # the latter value lies in $Q_3$
    false
    gap> E(3) < E(3)^2;    # both lie in $Q_3$, so compare coefficients
    false
    gap> 3 < E(3); E(5) < E(7);
    true
    true
    gap> E(728) < (1,2);
    true|

%%%%%%%%%%%%%%%%%%%%%%%%%%%%%%%%%%%%%%%%%%%%%%%%%%%%%%%%%%%%%%%%%%%%%%%%%%%%%
\Section{Operations for Cyclotomics}
\index{operators!for cyclotomics}

The  operators '+', '-', '\*', '/' are used for addition, subtraction,
multiplication and division of two cyclotomics; note that  division by
0 causes an error.

'+' and '-' can also be used as unary operators;

'\^' is used for exponentiation of a cyclotomic with an integer;
     this is in general *not* equal to Galois conjugation.

|    gap> E(5) + E(3); (E(5) + E(5)^4) ^ 2; E(5) / E(3); E(5) * E(3);
    -E(15)^2-2*E(15)^8-E(15)^11-E(15)^13-E(15)^14
    -2*E(5)-E(5)^2-E(5)^3-2*E(5)^4
    E(15)^13
    E(15)^8|

%%%%%%%%%%%%%%%%%%%%%%%%%%%%%%%%%%%%%%%%%%%%%%%%%%%%%%%%%%%%%%%%%%%%%%%%%%%%%
\Section{GaloisCyc}\index{galois conjugation}\index{galois automorphism}

'GaloisCyc( <z>, <k> )'

returns  the cyclotomic obtained on raising the roots  of unity in the
representation of  the cyclotomic <z> to  the <k>-th power.  If <z> is
represented in the  field $Q_n$ and <k>  is  a  fixed integer relative
prime to <n>, 'GaloisCyc( ., <k> )'  acts as a  Galois automorphism of
$Q_n$  (see   "GaloisGroup  for   Number  Fields");   to   get  Galois
automorphisms as functions, use "GaloisGroup" 'GaloisGroup'.

|    gap> GaloisCyc( E(5) + E(5)^4, 2 );
    E(5)^2+E(5)^3
    gap> GaloisCyc( E(5), -1 );           # the complex conjugate
    E(5)^4
    gap> GaloisCyc( E(5) + E(5)^4, -1 );  # this value is real
    E(5)+E(5)^4
    gap> GaloisCyc( E(15) + E(15)^4, 3 );
    E(5)+E(5)^4|

'GaloisCyc' is an internal function.

%%%%%%%%%%%%%%%%%%%%%%%%%%%%%%%%%%%%%%%%%%%%%%%%%%%%%%%%%%%%%%%%%%%%%%%%%
\Section{ATLAS irrationalities}\index{$b_N$}\index{$c_N$}\index{$d_N$}
\index{$e_N$}\index{$f_N$}\index{$g_N$}\index{$h_N$}\index{$i_N$}
\index{$j_N$}\index{$k_N$}\index{$l_N$}\index{$m_N$}\index{$r_N$}
\index{$s_N$}\index{$t_N$}\index{$u_N$}\index{$v_N$}\index{$w_N$}
\index{$x_N$}\index{$y_N$}\index{$n_k$}\index{atomic irrationalities}
\index{EB}\index{EC}\index{ED}\index{EE}\index{EF}\index{EG}\index{EH}
\index{EI}\index{EJ}\index{EK}\index{EL}\index{EM}\index{ER}\index{ES}
\index{ET}\index{EU}\index{EV}\index{EW}\index{EX}\index{EY}\index{NK}

'EB( <N> )', 'EC( <N> )', \ldots, 'EH( <N> )',\\
'EI( <N> )', 'ER( <N> )',\\
'EJ( <N> )', 'EK( <N> )', 'EL( <N> )', 'EM( <N> )',\\
'EJ( <N>, <d> )', 'EK( <N>, <d> )', 'EL( <N>, <d> )', 'EM( <N>, <d> )',\\
'ES( <N> )', 'ET( <N> )', \ldots, 'EY( <N> )',\\
'ES( <N>, <d> )', 'ET( <N>, <d> )', \ldots, 'EY( <N>, <d> )',\\
'NK( <N>, <k>, <d> )'

For $N$ a positive integer, let $z = 'E(<N>)' = e^{2 \pi i / N}$.  The
following   so-called  atomic  irrationalities  (see~\cite[Chapter  7,
Section 10]{CCN85}) can be entered by functions (Note that the  values
are not necessary irrational.)\:

\[\begin{array}{llllll}
'EB(<N>)' & = & b_N & = & \frac{1}{2}\sum_{j=1}^{N-1}z^{j^{2}} &
 (N\equiv 1\bmod 2)\\

'EC(<N>)' & = & c_N & = & \frac{1}{3}\sum_{j=1}^{N-1}z^{j^{3}} &
 (N\equiv 1\bmod 3)\\

'ED(<N>)' & = & d_N & = & \frac{1}{4}\sum_{j=1}^{N-1}z^{j^{4}} &
 (N\equiv 1\bmod 4)\\

'EE(<N>)' & = & e_N & = & \frac{1}{5}\sum_{j=1}^{N-1}z^{j^{5}} &
 (N\equiv 1\bmod 5)\\

'EF(<N>)' & = & f_N & = & \frac{1}{6}\sum_{j=1}^{N-1}z^{j^{6}} &
 (N\equiv 1\bmod 6)\\

'EG(<N>)' & = & g_N & = & \frac{1}{7}\sum_{j=1}^{N-1}z^{j^{7}} &
 (N\equiv 1\bmod 7)\\

'EH(<N>)' & = & h_N & = & \frac{1}{8}\sum_{j=1}^{N-1}z^{j^{8}} &
 (N\equiv 1\bmod 8)
\end{array}\]

(Note that in $c_N, \ldots, h_N$, <N> must be a prime.)

\[\begin{array}{lllll}
'ER(<N>)' & = & \sqrt{N}\\
'EI(<N>)' & = & i \sqrt{N} & = & \sqrt{-N}\\
\end{array}\]

From a theorem of Gauss we know that
\[ b_N = \left\{ \begin{array}{llll}
          \frac{1}{2}(-1+\sqrt{N}) & {\rm if} & N\equiv 1 & \bmod 4 \\
          \frac{1}{2}(-1+i\sqrt{N}) & {\rm if} & N\equiv -1 & \bmod 4
          \end{array}\right. ,\]

so $\sqrt{N}$ can be (and in fact is) computed from $b_N$. If <N> is a
negative integer then 'ER(<N>) = EI(-<N>)'.

For  given  <N>,  let  $n_k  =  n_k(N)$  be  the  first  integer  with
multiplicative order  exactly <k>  modulo <N>, chosen  in the order of
preference
\[ 1, -1, 2, -2, 3, -3, 4, -4, \ldots\ .\]

We have
\[\begin{array}{llllll}
'EY(<N>)' & = & y_n & = & z+z^n &(n = n_2)\\
'EX(<N>)' & = & x_n & = & z+z^n+z^{n^2} &(n=n_3)\\
'EW(<N>)' & = & w_n & = & z+z^n+z^{n^2}+z^{n^3} &(n=n_4)\\
'EV(<N>)' & = & v_n & = & z+z^n+z^{n^2}+z^{n^3}+z^{n^4} &(n=n_5)\\
'EU(<N>)' & = & u_n & = & z+z^n+z^{n^2}+\ldots +z^{n^5} &(n=n_6)\\
'ET(<N>)' & = & t_n & = & z+z^n+z^{n^2}+\ldots +z^{n^6} &(n=n_7)\\
'ES(<N>)' & = & s_n & = & z+z^n+z^{n^2}+\ldots +z^{n^7} &(n=n_8)
\end{array}\]

\[\begin{array}{llllll}
'EM(<N>)' & = & m_n & = & z-z^n &(n=n_2)\\
'EL(<N>)' & = & l_n & = & z-z^n+z^{n^2}-z^{n^3} &(n=n_4)\\
'EK(<N>)' & = & k_n & = & z-z^n+\ldots -z^{n^5} &(n=n_6)\\
'EJ(<N>)' & = & j_n & = & z-z^n+\ldots -z^{n^7} &(n=n_8)
\end{array}\]

Let  $n_k^{(d)}  =  n_k^{(d)}(N)$  be   the   $d+1$-th  integer   with
multiplicative order exactly <k> modulo  <N>,  chosen  in the order of
preference          defined          above;          we          write
$n_k=n_k^{(0)},n_k^{\prime}=n_k^{(1)}, n_k^{\prime\prime} = n_k^{(2)}$
and  so  on.  These  values can be  computed  as 'NK(<N>,<k>,<d>)'$  =
n_k^{(d)}(N)$; if there is no integer with the required multiplicative
order, 'NK' will return 'false'.

The algebraic numbers
\[y_N^{\prime}=y_N^{(1)},y_N^{\prime\prime}=y_N^{(2)},\ldots,
x_N^{\prime},x_N^{\prime\prime},\ldots,
j_N^{\prime},j_N^{\prime\prime},\ldots\]
are obtained on replacing $n_k$ in the above
definitions by $n_k^{\prime},n_k^{\prime\prime},\ldots$; they
can be entered as

\[\begin{array}{lll}
'EY(<N>,<d>)' & = & y_N^{(d)}\\
'EX(<N>,<d>)' & = & x_N^{(d)}\\
 & \vdots \\
'EJ(<N>,<d>)' & = & j_n^{(d)}
\end{array}\]

|    gap> EW(16,3); EW(17,2); ER(3); EI(3); EY(5); EB(9);
    0
    E(17)+E(17)^4+E(17)^13+E(17)^16
    -E(12)^7+E(12)^11
    E(3)-E(3)^2
    E(5)+E(5)^4
    1|

%%%%%%%%%%%%%%%%%%%%%%%%%%%%%%%%%%%%%%%%%%%%%%%%%%%%%%%%%%%%%%%%%%%%%%%%%%%%%
\Section{StarCyc}\index{galois conjugate!unique}

'StarCyc( <z> )'

If <z> is an irrational  element of a  quadratic number field (i.e. if
<z> is a quadratic irrationality), 'StarCyc( <z> )' returns the unique
Galois  conjugate  of <z> that is different from  <z>;  this is  often
called  $<z>\ast$   (see  "DisplayCharTable").  Otherwise  'false'  is
returned.

|    gap> StarCyc( EB(5) ); StarCyc( E(5) );
    E(5)^2+E(5)^3
    false|

%%%%%%%%%%%%%%%%%%%%%%%%%%%%%%%%%%%%%%%%%%%%%%%%%%%%%%%%%%%%%%%%%%%%%%%%%%%%%
\Section{Quadratic}\index{quadratic irrationalities}
\index{quadratic number fields}

'Quadratic( <z> )'

If <z> is a cyclotomic integer that is contained in a quadratic number
field over  the rationals, it can  be written as $<z> = \frac{  a +  b
\sqrt{n} }{d}$  with integers  $a$, $b$, $n$  and $d$,  where  $d$  is
either 1 or 2.  In this case 'Quadratic( <z>  )' returns a record with
fields 'a', 'b', 'root', 'd' and 'ATLAS' where the first four mean the
integers mentioned above, and the last one is a string that is a  (not
necessarily  shortest)  representation of <z> by $b_m$, $i_m$ or $r_m$
for $m = '\|root\|'$ (see "ATLAS irrationalities").

If <z>  is not a quadratic irrationality  or not a cyclotomic integer,
'false' is returned.

|    gap> Quadratic( EB(5) ); Quadratic( EB(27) );
    rec(
      a := -1,
      b := 1,
      root := 5,
      d := 2,
      ATLAS := "b5" )
    rec(
      a := -1,
      b := 3,
      root := -3,
      d := 2,
      ATLAS := "1+3b3" )
    gap> Quadratic(0); Quadratic( E(5) );
    rec( 
      a := 0,
      b := 0,
      root := 1,
      d := 1,
      ATLAS := "0" )
    false|

%%%%%%%%%%%%%%%%%%%%%%%%%%%%%%%%%%%%%%%%%%%%%%%%%%%%%%%%%%%%%%%%%%%%%%%%%%%%%
\Section{GaloisMat}\index{galois conjugate characters}

'GaloisMat( <mat> )'

<mat>  must  be  a matrix of  cyclotomics (or  possibly unknowns,  see
"Unknown").  The conjugate of a row in <mat> under a particular Galois
automorphism is  defined  pointwise.  If <mat> consists of full orbits
under this action then the Galois group of its entries acts  on  <mat>
as a permutation group, otherwise the orbits must be completed before.

'GaloisMat( <mat> )' returns a record  with fields 'mat', 'galoisfams'
and 'generators'\:

'mat':\\
      a list with initial  segment <mat> (*not* a  copy of <mat>); the
      list  consists of  full  orbits under the  action of  the Galois
      group  of the entries of  <mat> defined above. The last  entries
      are those rows  which had to be added to complete the orbits; so
      if  they were already complete, <mat> and 'mat'  have  identical
      entries.

'galoisfams':\\
      a list that has the same length as 'mat'; its entries are either
      1, 0, -1 or lists\:\\ $'galoisfams[i]'  = 1$ means that 'mat[i]'
      consists of rationals,  i.e. $[ 'mat[i]'  ]$  forms  an orbit.\\
      $'galoisfams[i]' =-1$  means that 'mat[i]' contains unknowns; in
      this case $[ 'mat[i]'  ]$ is regarded as  an orbit, too, even if
      'mat[i]' contains irrational entries.\\ If $'galoisfams[i]' =  [
      l_1, l_2 ]$ is  a list then 'mat[i]' is the first element of its
      orbit in  'mat'; $l_1$  is  the list of positions  of rows which
      form  the orbit, and $l_2$ is the  list of  corresponding Galois
      automorphisms  (as  exponents,  not  as  functions); so  we have
      $'mat'[  l_1[j] ][k]  =  'GaloisCyc'(  'mat'[i][k], l_2[j] )$.\\
      $'galoisfams[i]' =  0$  means that  'mat[i]'  is an element of a
      nontrivial orbit but not the first element of it.

'generators':\\
      a  list  of  permutations  generating  the   permutation   group
      corresponding  to the  action of the Galois group on the rows of
      'mat'.

Note  that <mat> should be a set,  i.e. no  two rows should be  equal.
Otherwise only the first row of some equal rows is  considered for the
permutations, and a warning is printed.

|    gap> GaloisMat( [ [ E(3), E(4) ] ] );
    rec(
      mat := [ [ E(3), E(4) ], [ E(3), -E(4) ], [ E(3)^2, E(4) ], 
          [ E(3)^2, -E(4) ] ],
      galoisfams := [ [ [ 1, 2, 3, 4 ], [ 1, 7, 5, 11 ] ], 0, 0, 0 ],
      generators := [ (1,2)(3,4), (1,3)(2,4) ] )
    gap> GaloisMat( [ [ 1, 1, 1 ], [ 1, E(3), E(3)^2 ] ] );
    rec(
      mat := [ [ 1, 1, 1 ], [ 1, E(3), E(3)^2 ], [ 1, E(3)^2, E(3) ] ],
      galoisfams := [ 1, [ [ 2, 3 ], [ 1, 2 ] ], 0 ],
      generators := [ (2,3) ] )|

%%%%%%%%%%%%%%%%%%%%%%%%%%%%%%%%%%%%%%%%%%%%%%%%%%%%%%%%%%%%%%%%%%%%%%%%%%%%%
\Section{RationalizedMat}\index{rational characters}

'RationalizedMat( <mat> )'

returns the set  of rationalized rows of <mat>, i.e. the  set  of sums
over  orbits under the action  of the Galois  group of the elements of
<mat> (see "GaloisMat").

This may be viewed as a kind of trace operation for the rows.

Note that <mat> should be a set, i.e. no two rows should be equal.

|    gap> mat:= CharTable( "A5" ).irreducibles;
    [ [ 1, 1, 1, 1, 1 ], [ 3, -1, 0, -E(5)-E(5)^4, -E(5)^2-E(5)^3 ], 
      [ 3, -1, 0, -E(5)^2-E(5)^3, -E(5)-E(5)^4 ], [ 4, 0, 1, -1, -1 ], 
      [ 5, 1, -1, 0, 0 ] ]
    gap> RationalizedMat( mat );
    [ [ 1, 1, 1, 1, 1 ], [ 6, -2, 0, 1, 1 ], [ 4, 0, 1, -1, -1 ], 
      [ 5, 1, -1, 0, 0 ] ]|

