%%%%%%%%%%%%%%%%%%%%%%%%%%%%%%%%%%%%%%%%%%%%%%%%%%%%%%%%%%%%%%%%%%%%%%%%%%%%%
%%
%A  range.tex                   GAP documentation            Martin Schoenert
%%
%A  @(#)$Id: range.tex,v 1.1.1.1 1996/12/11 12:36:49 werner Exp $
%%
%Y  Copyright 1990-1992,  Lehrstuhl D fuer Mathematik,  RWTH Aachen,  Germany
%%
%%  This file describes those functions that mainly deal with ranges.
%%
%H  $Log: range.tex,v $
%H  Revision 1.1.1.1  1996/12/11 12:36:49  werner
%H  Preparing 3.4.4 for release
%H
%H  Revision 3.8  1994/06/09  09:29:42  vfelsch
%H  updated output of examples
%H
%H  Revision 3.7  1993/02/19  10:48:42  gap
%H  adjustments in line length and spelling
%H
%H  Revision 3.6  1993/02/12  11:39:35  felsch
%H  new example fixed
%H
%H  Revision 3.5  1993/02/10  18:56:42  martin
%H  changed the description for extended ranges
%H
%H  Revision 3.4  1993/02/05  08:07:20  felsch
%H  examples fixed
%H
%H  Revision 3.3  1991/12/27  16:07:04  martin
%H  revised everything for GAP 3.1 manual
%H
%H  Revision 3.2  1991/07/26  09:01:07  martin
%H  changed |\GAP\ | to |{\GAP}|
%H
%H  Revision 3.1  1991/07/25  16:16:59  martin
%H  fixed some minor typos
%H
%H  Revision 3.0  1991/04/11  11:32:53  martin
%H  Initial revision under RCS
%H
%%
\Chapter{Ranges}

A *range* is a dense list of integers,  such  that the difference between
consecutive  elements is a nonzero  constant.   Ranges can be abbreviated
with the syntactic construct '[ <first>, <second> .. <last> ]' or, if the
difference between consecutive elements is 1, as '[ <first> .. <last> ]'.

If  '<first> > <last>',  '[<first>,<second>..<last>]' is  the empty list,
which   by  definition  is   also   a  range.    If   <first>  =  <last>,
'[<first>,<second>..<last>]' is a singleton list,  which is a range  too.
Note that '<last> - <first>' must be divisible by the increment '<second>
- <first>', otherwise an error is signalled.

Note  that a range is  just a special case of a list.  So everything that
is  possible for lists (see "Lists") is also  possible for ranges.   Thus
you can  access elements in such a range (see "List Elements"),  test for
membership  (see  "In"), etc.   You can even assign to such a  range (see
"List   Assignment").   Of   course,  unless   you   assign   '<last>   +
<second>-<first>'  to   the   entry   '<range>[Length(<range>)+1]',   the
resulting list will no longer be a range.

Most often ranges are used in connection with the 'for'-loop (see "For").
Here  the construct \\
'for <var>  in [<first>..<last>]  do <statements>  od' replaces the \\
'for <var>  from <first>  to <last>  do <statements>  od',  which is more
usual in other programming languages.

Note that a range is at the same time also a set (see "Sets"), because it
contains no  holes or duplicates  and is sorted, and  also a  vector (see
"Vectors"), because it contains no holes and all elements are integers.

|    gap> r := [10..20];
    [ 10 .. 20 ]
    gap> Length( r );
    11
    gap> r[3];
    12
    gap> 17 in r;
    true
    gap> r[12] := 25;; r;
    [ 10, 11, 12, 13, 14, 15, 16, 17, 18, 19, 20, 25 ]
    gap> r := [1,3..17];
    [ 1, 3 .. 17 ]
    gap> Length( r );
    9
    gap> r[4];
    7
    gap> r := [0,-1..-9];
    [ 0, -1 .. -9 ]
    gap> r[5];
    -4
    gap> r := [ 1, 4 .. 32 ];
    Error, Range: <high>-<low> must be divisible by <inc>
    gap> s := [];;  for i  in [10..20]  do Add( s, i^2 );  od;  s;
    [ 100, 121, 144, 169, 196, 225, 256, 289, 324, 361, 400 ] |

The first section in this chapter describes  the function that tests if a
list is a range (see "IsRange").

The  other section tells  you more  about  the internal representation of
ranges (see "More about Ranges").

%%%%%%%%%%%%%%%%%%%%%%%%%%%%%%%%%%%%%%%%%%%%%%%%%%%%%%%%%%%%%%%%%%%%%%%%%
\Section{IsRange}\index{test!for a range}

'IsRange( <obj> )'

'IsRange' returns 'true' if <obj>, which may be an object of any type, is
a range and 'false' otherwise.  A range is a list without holes such that
the elements are integers with a constant increment.  Will cause an error
if <obj> is an unassigned variable.

|    gap> IsRange( [1,2,3] );
    true    # this list is a range
    gap> IsRange( [7,5,3,1] );
    true    # this list is a range
    gap> IsRange( [1,2,4,5] );
    false    # this list is a set and a vector, but not a range
    gap> IsRange( [1,,3,,5,,7] );
    false    # this list contains holes
    gap> IsRange( 1 );
    false    # is not even a list
    gap> IsRange( [] );
    true    # the empty list is a range by definition
    gap> IsRange( [1] );
    true    # singleton lists are a range by definition too |

%%%%%%%%%%%%%%%%%%%%%%%%%%%%%%%%%%%%%%%%%%%%%%%%%%%%%%%%%%%%%%%%%%%%%%%%%
\Section{More about Ranges}

For some  lists the kernel knows   that they are  in fact  ranges.  Those
lists are represented internally in a compact way instead of the ordinary
way.  This is important since this representation needs only 12 bytes for
the entire list while the ordinary representation needs $4 length$ bytes.

Note that a list that is represented in the ordinary way might still be a
range.  It is just that {\GAP} does not  know this.   This  section tells
you under which circumstances a range is represented  in the compact way,
so you can write your  program in such  a  way that you make best  use of
this compact representation for ranges.

Lists  created by the syntactic construct '[ <first>, <second>  .. <last>
]' are  of  course known to be ranges and  are represented in the compact
way.

If  you call  'IsRange' for a  list represented the ordinary  way that is
indeed a range, 'IsRange' will note this,  change the representation from
the ordinary to the compact representation, and then return 'true';


If   you change a   range  that is  represented   in the compact  way, by
assignment, 'Add'   or 'Append', the   range will  be   converted to  the
ordinary representation, even  if the change is  such that the  resulting
list is still a proper range.

Suppose  you have   built a proper   range  in  such  a way   that  it is
represented in the  ordinary way and that you  now want to convert  it to
the compact representation to save space.  Then you should call 'IsRange'
with that list as an argument.  If it is indeed a proper range, 'IsRange'
will convert it to the compact representation.  You can think of the call
to 'IsRange' as a hint to {\GAP} that this list is a proper range.

%%%%%%%%%%%%%%%%%%%%%%%%%%%%%%%%%%%%%%%%%%%%%%%%%%%%%%%%%%%%%%%%%%%%%%%%%
%%
%E  Emacs . . . . . . . . . . . . . . . . . . . . . local Emacs variables
%%
%%  Local Variables:
%%  mode:               outline
%%  outline-regexp:     "\\\\Chapter\\|\\\\Section"
%%  fill-column:        73
%%  eval:               (hide-body)
%%  End:
%%



