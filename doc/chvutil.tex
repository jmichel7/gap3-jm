%%%%%%%%%%%%%%%%%%%%%%%%%%%%%%%%%%%%%%%%%%%%%%%%%%%%%%%%%%%%%%%%%%%%%%%%%%%%%
%%
%A  chvutil.tex       CHEVIE documentation       Meinolf Geck, Frank Luebeck,
%A                                                Jean Michel, G"otz Pfeiffer
%%
%Y  Copyright (C) 1992 - 1996  Lehrstuhl D f\"ur Mathematik, RWTH Aachen, IWR
%Y  der Universit\"at Heidelberg, University of St. Andrews, and   University
%Y  Paris VII.
%%
%%  This  file  contains  utility functions.
%%%%%%%%%%%%%%%%%%%%%%%%%%%%%%%%%%%%%%%%%%%%%%%%%%%%%%%%%%%%%%%%%%%%%%%%%

\Chapter{CHEVIE utility functions}

The  functions described below, used   in various parts  of the  \CHEVIE\
package, are of a general nature  and should really  be included in other
parts of the \GAP\ library.  We include them here for  the moment for the
commodity of the reader.

%%%%%%%%%%%%%%%%%%%%%%%%%%%%%%%%%%%%%%%%%%%%%%%%%%%%%%%%%%%%%%%%%%%%%%%%%
\Section{SymmetricDifference}%
\index{SymmetricDifference}%

'SymmetricDifference( <S>, <T>)'

This  function returns  the symmetric  difference of  the sets <S> and <T>,
which can be written in \GAP\ as
'Difference(Union(x,y),IntersectionSet(x,y)'.

|    gap> SymmetricDifference([1,2],[2,3]);
    [ 1, 3 ]|

%%%%%%%%%%%%%%%%%%%%%%%%%%%%%%%%%%%%%%%%%%%%%%%%%%%%%%%%%%%%%%%%%%%%%%%%%
\Section{DifferenceMultiSet}%
\index{DifferenceMultiSet}%

'DifferenceMultiSet( <l>, <s> )'

This function returns the difference of the multisets <l> and <s>. That is,
<l>  and <s> are lists  which may contain several  times the same item. The
result is a list which is like <l>, excepted if an item occurs $a$ times in
<s>,  the first $a$ occurrences  of this item in  <l> have been deleted (all
the occurrences if $a$ is greater than the times it occurred in <l>).

|    gap> DifferenceMultiSet("ababcbadce","edbca");
    "abbac"|

%%%%%%%%%%%%%%%%%%%%%%%%%%%%%%%%%%%%%%%%%%%%%%%%%%%%%%%%%%%%%%%%%%%%%%%%%
\Section{Rotation}%
\index{Rotation}%

'Rotation(<l>, <i>)'

This function returns <l> rotated <i> steps.

|    gap> l:=[1..5];;
    gap> Rotation(l,1);
    [ 2, 3, 4, 5, 1 ]
    gap> Rotation(l,0);
    [ 1, 2, 3, 4, 5 ]
    gap> Rotation(l,-1);
    [ 5, 1, 2, 3, 4 ]|

%%%%%%%%%%%%%%%%%%%%%%%%%%%%%%%%%%%%%%%%%%%%%%%%%%%%%%%%%%%%%%%%%%%%%%%%%
\Section{Rotations}%
\index{Rotations}%

'Rotations(<l>)'

This function returns the list of rotations of the list <l>.

|    gap> Rotations("abcd");
    [ "abcd", "bcda", "cdab", "dabc" ]
    gap> Rotations([1,0,1,0]);
    [ [ 1, 0, 1, 0 ], [ 0, 1, 0, 1 ], [ 1, 0, 1, 0 ], [ 0, 1, 0, 1 ] ]|

%%%%%%%%%%%%%%%%%%%%%%%%%%%%%%%%%%%%%%%%%%%%%%%%%%%%%%%%%%%%%%%%%%%%%%%%%
\Section{Inherit}%
\index{Inherit}%

'Inherit(<rec1>,<rec2>[,<fields>])'

This  functions copies to  the record <rec1>  all the fields  of the record
<rec2>. If an additional argument <fields> is given, it should be a list of
field names, and then only the fields specified by <fields> are copied. The
function returns the modified <rec1>.

|    gap> r:=rec(a:=1,b:=2);
    rec(
      a := 1,
      b := 2 )
    gap> s:=rec(c:=3,d:=4);
    rec(
      c := 3,
      d := 4 )
    gap> Inherit(r,s);
    rec(
      a := 1,
      b := 2,
      c := 3,
      d := 4 )
    gap> r:=rec(a:=1,b:=2);
    rec(
      a := 1,
      b := 2 )
    gap> Inherit(r,s,["d"]);
    rec(
      a := 1,
      b := 2,
      d := 4 )|

%%%%%%%%%%%%%%%%%%%%%%%%%%%%%%%%%%%%%%%%%%%%%%%%%%%%%%%%%%%%%%%%%%%%%%%%%
\Section{Dictionary}%
\index{Dictionary}%

'Dictionary()'

This  function  creates  a  dictionary  data  type. The created object is a
record  with two functions\:

'Get(k)':  get the value associated to key 'k'; it returns 'false' if there
is no such key.

'Insert(k,v)'  sets in the dictionary the value associated to key 'k' to be
'v'.

The main advantage compared to records is that keys may be of any type.

|    gap> d:=Dictionary();
    Dictionary with 0 entries
    gap> d.Insert("a",1);
    1
    gap> d.Insert("b",2);
    2
    gap> d.Get("a");
    1
    gap> d.Get("c");
    false
    gap> d;
    Dictionary with 2 entries|

%%%%%%%%%%%%%%%%%%%%%%%%%%%%%%%%%%%%%%%%%%%%%%%%%%%%%%%%%%%%%%%%%%%%%%%%%
\Section{GetRoot}%
\index{GetRoot}%

'GetRoot( <x>, <n> [, <msg>])'

<n> must  be a  positive integer.  'GetRoot' returns  an <n>-th  root of
<x>  when possible,  else  signals an  error. If  <msg>  is present  and
'InfoChevie=Print' a  warning message is  printed about which  choice of
root has been made, after printing <msg>.

In the  current implementation, it  is possible  to find an  <n>-th root
when <x> is one of the following \GAP\ objects\:

1- a monomial of the form 'a\*q\^(b\*n)' when we know how to find a root
of $a$. The root chosen is 'GetRoot(a,n)\*q\^b'.

2-  a  root  of  unity  of  the  form  'E(a)\^i'.  The  root  chosen  is
'E(a\*n)\^i'.

3- an integer, when <n>=2 (the root  chosen is 'ER(x)') or when <x> is a
perfect <n>-th power of <a> (the root chosen is <a>).

4- a product of an <x> of form 2- by an <x> of form 3-.

5- when <x> is a record and has a method 'x.operations.GetRoot' the work
is delegated to that method.

|    gap> q:=X(Cyclotomics);;q.name:="q";;
    gap> GetRoot(E(3)*q^2,2,"test");
    |\#|warning: test: E3^2q chosen as 2nd root of (E(3))*q^2
    (E(3)^2)*q
    gap> GetRoot(1,2,"test");
    |\#|warning: test: 1 chosen as 2nd root of 1
    1|

The  example  above   shows  that  'GetRoot'  is   not  compatible  with
specialization\:\  'E(3)\*q\^2' evaluated  at 'E(3)'  is '1'  whose root
chosen by 'GetRoot' is '1', while '(-E(3)\^2)\*q' evaluated at 'E(3)' is
'-1'. Actually it can be shown that it is not possible mathematically to
define a function 'GetRoot' compatible with specializations. This is why
there is a provision in functions for character tables of Hecke algebras
to provide explicit roots.

|    gap> GetRoot(8,3);
    2
    gap> GetRoot(7,3);
    Error, : unable to compute 3-th root of 7
     in
    GetRoot( 7, 3 ) called from
    main loop
    brk>|

%%%%%%%%%%%%%%%%%%%%%%%%%%%%%%%%%%%%%%%%%%%%%%%%%%%%%%%%%%%%%%%%%%%%%%%%%
\Section{CharParams}%
\index{CharParams}%

'CharParams(<G>)'

'G'  should   be  a  group  or   another  object  which  has   a  method
'CharTable', or  a character table.  The function 'CharParams'  tries to
determine a  list of  labels for  the characters  of 'G'.  If 'G'  has a
method  'CharParams'  this  is  called.  Otherwise,  if  'G'  is  not  a
character table,  its 'CharTable' is  called. If  the table has  a field
'.charparam'  in  '.irredinfo' this  is  returned.  Otherwise, the  list
'[1..Length(G.irreducibles)]' is returned.

|    gap> CharParams(CoxeterGroup("A",2));
    [ [ [ 1, 1, 1 ] ], [ [ 2, 1 ] ], [ [ 3 ] ] ]
    gap> CharParams(Group((1,2),(2,3)));
    #W  Warning: Group has no name
    [ 1 .. 3 ]|

%%%%%%%%%%%%%%%%%%%%%%%%%%%%%%%%%%%%%%%%%%%%%%%%%%%%%%%%%%%%%%%%%%%%%%%%%
\Section{CharName}%
\index{CharName}%

'CharName(<G>, <param>)'

'G' should  be a group  and 'param' a parameter  of a character  of that
group (as returned by 'CharParams'). If 'G' has a method 'CharName', the
function returns  the result  of that  method, which  is a  string which
displays nicely  'param' (this is  used by  \CHEVIE\ to nicely  fill the
'.charNames' in a 'CharTable' --  all finite reflection groups have such
methods 'CharName').

|    gap> G:=CoxeterGroup("G", 2);
    CoxeterGroup("G",2)
    gap> CharParams(G);
    [ [ [ 1, 0 ] ], [ [ 1, 6 ] ], [ [ 1, 3, 1 ] ], [ [ 1, 3, 2 ] ],
      [ [ 2, 1 ] ], [ [ 2, 2 ] ] ]
    gap>  List(last,x->CharName(G,x));
    [ "phi{1,0}", "phi{1,6}", "phi{1,3}'", "phi{1,3}''", "phi{2,1}",
      "phi{2,2}" ]|

%%%%%%%%%%%%%%%%%%%%%%%%%%%%%%%%%%%%%%%%%%%%%%%%%%%%%%%%%%%%%%%%%%%%%%%%%
\Section{PositionId}
\index{PositionId}

'PositionId( <G> )'

<G>  should be  a group,  a character  table, an  Hecke algebra  or another
object  which  has  characters.  'PositionId'  returns  the position of the
identity character in the character table of <G>.

|    gap> W := CoxeterGroup( "D", 4 );;
    gap> PositionId( W );
    13|

%%%%%%%%%%%%%%%%%%%%%%%%%%%%%%%%%%%%%%%%%%%%%%%%%%%%%%%%%%%%%%%%%%%%%%%%%
\Section{InductionTable}%
\index{InductionTable}

'InductionTable( <S>, <G> )'

'InductionTable'  computes the decomposition of the induced characters from
the subgroup <S> into irreducible characters of <G>. The rows correspond to
the  characters of the parent group, the  columns to those of the subgroup.
What  is returned  is actually  a record  with several  fields\:\ '.scalar'
contains  the induction table proper, and  there are 'Display' and 'Format'
methods.  The  other  fields  contain  labeling  information taken from the
character tables of <S> and <G> when it exists.

|    gap> G := Group( [ (1,2), (2,3), (3,4) ], () );
    Group( (1,2), (2,3), (3,4) )
    gap> S:=Subgroup( G, [ (1,2), (3,4) ] );
    Subgroup( Group( (1,2), (2,3), (3,4) ), [ (1,2), (3,4) ] )
    gap> G.name := "G";; S.name := "S";; # to avoid warnings
    gap> Display( InductionTable( S, G ) );
    Induction from S to G
        |'\|'|X.1 X.2 X.3 X.4
    _____________________
    X.1 |'\|'|  1   .   .   .
    X.2 |'\|'|  .   .   .   1
    X.3 |'\|'|  1   .   .   1
    X.4 |'\|'|  .   1   1   1
    X.5 |'\|'|  1   1   1   .|

|    gap> G := CoxeterGroup( "G", 2 );;
    gap> S := ReflectionSubgroup( G, [ 1, 4 ] );
    ReflectionSubgroup(CoxeterGroup("G",2), [ 1, 4 ])
    gap> t := InductionTable( S, G );
    InductionTable(ReflectionSubgroup(CoxeterGroup("G",2), [ 1, 4 ]), Coxe\
    terGroup("G",2))
    gap> Display( t );
    Induction from A1x~A1 to G2
               |'\|'|11,11 11,2 2,11 2,2
    ________________________________
    phi{1,0}   |'\|'|    .    .    .   1
    phi{1,6}   |'\|'|    1    .    .   .
    phi{1,3}'  |'\|'|    .    1    .   .
    phi{1,3}'' |'\|'|    .    .    1   .
    phi{2,1}   |'\|'|    .    1    1   .
    phi{2,2}   |'\|'|    1    .    .   1|

The  'Display'  and  'Format'  methods  take  the  same  arguments  as  the
'FormatTable'  method. For instance to select a subset of the characters of
the subgroup and of the parent group, one can call

|    gap> Display( t,rec( rows := [5], columns := [3,2] ) );
    Induction from A1x~A1 to G2
             |'\|'|2,11 11,2
    ____________________
    phi{2,1} |'\|'|   1    1|

It is also possible to get TeX and LaTeX output, see "FormatTable".

%%%%%%%%%%%%%%%%%%%%%%%%%%%%%%%%%%%%%%%%%%%%%%%%%%%%%%%%%%%%%%%%%%%%%%%%%%%%%
\Section{CharRepresentationWords}
\index{CharRepresentationWords}

'CharRepresentationWords( <rep> , <elts> )'

given  a list <rep>  of matrices  corresponding to generators  and a list
<elts>  of words in  the generators it returns  the list of traces of the
corresponding representation on the elements in <elts>.

|    gap> H := Hecke(CoxeterGroup( "F", 4 ));;
    gap> r := ChevieClassInfo( Group( H ) ).classtext;;
    gap> t := HeckeReflectionRepresentation( H );;
    gap> CharRepresentationWords( t, r );
    [ 4, -4, 0, 1, -1, 0, 1, -1, -2, 2, 0, 2, -2, -1, 1, 0, 2, -2, -1, 1,
      0, 0, 2, -2, 0 ]|

%%%%%%%%%%%%%%%%%%%%%%%%%%%%%%%%%%%%%%%%%%%%%%%%%%%%%%%%%%%%%%%%%%%%%%%%%
\Section{PointsAndRepresentativesOrbits}%
\index{PointsAndRepresentativesOrbits}%

'PointsAndRepresentativesOrbits( <G>[, <m>] )'

returns a  pair <[orb, rep]> where  <orb> is a list of  the orbits of the
permutation group <G> on '[ 1..LargestMovedPoint( <G> ) ]' and <rep> is a
list  of   list of elements   of  <G>  such  that 'rep[i][j]'  applied to
'orb[i][1]'  yields 'orb[i][j]' for all $i,j$.  If  the optional argument
<m> is given, then 'LargestMovedPoint( <G> )' is  replaced by the integer
<m>.

|    gap> G := Group( (1,7)(2,3)(5,6)(8,9)(11,12),
    >                (1,5)(2,8)(3,4)(7,11)(9,10) );;
    gap> PointsAndRepresentativesOrbits( G );
    [ [ [ 1, 7, 5, 11, 6, 12 ], [ 2, 3, 8, 4, 9, 10 ] ],
      [ [ (), ( 1, 7)( 2, 3)( 5, 6)( 8, 9)(11,12),
              ( 1, 5)( 2, 8)( 3, 4)( 7,11)( 9,10),
              ( 1,11,12, 7, 5, 6)( 2, 4, 3, 8,10, 9),
              ( 1, 6, 5, 7,12,11)( 2, 9,10, 8, 3, 4),
              ( 1,12)( 2, 4)( 3, 9)( 6, 7)( 8,10) ],
          [ (), ( 1, 7)( 2, 3)( 5, 6)( 8, 9)(11,12),
              ( 1, 5)( 2, 8)( 3, 4)( 7,11)( 9,10),
              ( 1,11,12, 7, 5, 6)( 2, 4, 3, 8,10, 9),
              ( 1, 6, 5, 7,12,11)( 2, 9,10, 8, 3, 4),
              ( 1, 6)( 2,10)( 4, 8)( 5,11)( 7,12) ] ] ]|

%%%%%%%%%%%%%%%%%%%%%%%%%%%%%%%%%%%%%%%%%%%%%%%%%%%%%%%%%%%%%%%%%%%%%%%%%
\Section{AbelianGenerators}%
\index{AbelianGenerators}%

'AbelianGenerators( <l>)'

<l>  should be a list of elements generating an abelian group. The function
returns a list of generators for the generated group
'G:=ApplyFunc(Group,l)'  which is optimal  in the sense  that they generate
the cyclic groups whose orders are given by 'AbelianInvariants(G)'.

%%%%%%%%%%%%%%%%%%%%%%%%%%%%%%%%%%%%%%%%%%%%%%%%%%%%%%%%%%%%%%%%%%%%%%%%%
