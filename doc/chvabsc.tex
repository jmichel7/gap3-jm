%%%%%%%%%%%%%%%%%%%%%%%%%%%%%%%%%%%%%%%%%%%%%%%%%%%%%%%%%%%%%%%%%%%%%%%%%%%%%
%%
%A  chvabsc.tex       CHEVIE documentation                        Jean Michel
%%
%A  $Id: chvabsc.tex,v 1.1 1997/03/27 16:31:29 werner Exp $
%%
%Y  Copyright (C) 1992 - 1996  Lehrstuhl D f\"ur Mathematik, RWTH Aachen
%Y  and  University  Paris VII.
%%
%%  This file contains  the description of the GAP  functions of CHEVIE
%%  dealing with abstract Coxeter groups.
%%
%%%%%%%%%%%%%%%%%%%%%%%%%%%%%%%%%%%%%%%%%%%%%%%%%%%%%%%%%%%%%%%%%%%%%%%%%

\Chapter{Coxeter groups}

In this chapter  we describe functions for dealing  with general Coxeter
groups.

A suitable reference for the general  theory is, for example, the volume
of Bourbaki \cite{Bou68}.

A  *Coxeter group* is  a group which  has the presentation $W= \left\langle
s_1,  \ldots,  s_n\  \|\  (s_i  s_j)^{m(i,j)}=1  \right\rangle  $  for some
symmetric  integer  matrix  $m(i,j)$  called  the  *Coxeter  matrix*, where
$m(i,j)>1$  for $i  \ne j$  and $m(i,i)=1$.  It is  true (but a non-trivial
theorem)  that  in  a  Coxeter  group  the  order  of  $s_i s_j$ is exactly
$m(i,j)$, thus a Coxeter group is the same as a *Coxeter system*, that is a
pair   $(W,S)$  of  a  group  $W$   and  a  set  $S=\{s_1,\ldots,s_n\}$  of
involutions,  such that the group is  presented by relations describing the
order  of the  product of  two generating  elements. A  Coxeter group has a
natural  representation on a real vector  space $V$ of dimension the number
of  generators, where each generator acts  as a reflection, its *reflection
representation*  (see  'CoxeterGroupByCoxeterMatrix');  the faithfulness of
this  representation in the main  argument to prove that  the order of $s_i
s_j$  is exactly $m(i,j)$. Thus Coxeter  groups are real reflection groups.
The  converse need not be true if the set of reflecting hyperplanes has bad
topological properties, but it turns out that finite Coxeter groups are the
same  as finite real  reflection groups. The  possible Coxeter matrices for
finite  Coxeter groups  have been  completely classified; the corresponding
finite groups play a deep role in several areas of mathematics.

Coxeter  groups  have  a  nice  solution  to the word problem. The *length*
$l(w)$ of an element $w$ of $W$ is the minimum number of elements of $S$ of
which it is a product (since the elements of $S$ are involutions, we do not
need inverses). An expression of $w$ of minimal length is called a *reduced
word*  for $w$. The main property of  reduced words is the *exchange lemma*
which  states that if $s_1\ldots s_k$ is a reduced word for $w$ where $k=l(w)$
and $s\in S$ is such that $l(sw)\le l(w)$ then one of the $s_i$ in the word
for  $w$ can be deleted to obtain a reduced word for $sw$. Thus given $s\in
S$ and $w\in W$, either $l(sw)=l(w)+1$ or $l(sw)=l(w)-1$ and we say in this
last  case  that  $s$  belongs  to  the  *left  descent  set*  of  $w$. The
computation  of a reduced word for an element, and other word problems, are
easily  done if we know the left descent sets. For most Coxeter groups that
we  will be able to  build in \CHEVIE, this  left descent set can be easily
determined  (see e.g. 'CoxeterGroupSymmetricGroup' below), so this suggests
how  to deal with Coxeter groups  in {\CHEVIE}. They are reflection groups,
so the following fields are defined in the group record\:

'.nbGeneratingReflections':\\ the size of $S$

'.reflections':\\ a list of elements of <W>, such
     that 'W.reflections{[1..W.nbGeneratingReflections]}' is the set $S$.

the above names  are used instead of names  like 'CoxeterGenerators' and
'CoxeterRank' since  the Coxeter groups  *are* reflection groups  and we
want the functions for reflection  groups applicable to them (similarly,
if you have  read the chapter on reflections and  reflection groups, you
will realize  that there is also  a field '.OrdersGeneratingReflections'
which contains only 2\'s). The  main additional function which allows to
compute within Coxeter groups is\:

'.operations.IsLeftDescending(W,w,i)':\\ returns true if and only if the
      $i$-th element of $S$ is in the left descending set of $w$.

For Coxeter groups constructed in {\CHEVIE} an 'IsLeftDescending' operation
is  provided, but you can construct your own Coxeter groups just by filling
the  above fields (see the  function 'CoxeterGroupSymmetricGroup' below for
an  example). It  should be  noted than  you can  make into a Coxeter group
*any*  kind of  group\:\ finitely  presented groups,  permutation groups or
matrix  groups, if you  fill appropriately the  above fields; and the given
generating  reflection do not  have to be  'W.generators' --- all functions
for Coxeter group and Hecke algebras will then work for your Coxeter groups
(using your function 'IsLeftDescending').

A  common occurrence in \CHEVIE\ code for Coxeter groups is a loop like\:

'First([1..W.semisimpleRank],x->IsLeftDescending(W,w,x))'

which for a reflection subgroup becomes

'First(W.rootRestriction\{[1..W.semisimpleRank]\},x->IsLeftDescending(W,w,x))'

where  the overhead is quite large, since  dispatching on the group type is
done in 'IsLeftDescending'. To improve this code, if you provide a function
'FirstLeftDescending(W,w)'  it will be called instead of the above loop (if
you  do  not  provide  one  the  above  loop will be used). Such a function
provided  by \CHEVIE\ for finite  Coxeter groups represented as permutation
groups of the roots is 3 times more efficient than the above loop.

Because of  the easy  solution of  the word  problem in  Coxeter groups,
a  convenient  way  to  represent  their elements is  as  words  in  the
Coxeter  generators.  They are  represented  in  {\CHEVIE} as  lists  of
labels  for  the  generators.  By  default these  labels  are  given  as
the  index  of  a  generator  in  $S$, so  a  Coxeter  word  is  just  a
list  of  integers  which  run  from  1  to  the  length  of  $S$.  This
can  be  changed  to  reflect  a more  conventional  notation  for  some
groups, by changing the field  '.reflectionsLabels' of the Coxeter group
which contains  the labels  used for  the Coxeter  words (by  default it
contains '[1..W.nbGeneratingReflections]').  For a Coxeter group  with 2
generators, you  could for instance  set this  field to '\"st\"'  to use
words such as '\"sts\"' instead  of '[1,2,1]'. For reflection subgroups,
this  is used  in {\CHEVIE}  by setting  the reflection  labels to  the
indices of the generators  in the set $S$ of the  parent group (which is
given by '.rootInclusion').

The functions 'CoxeterWord' and 'EltWord' will do the conversion between
Coxeter words and elements of the group.

|    gap> W := CoxeterGroup( "D", 4 );;
    gap> p := EltWord( W, [ 1, 3, 2, 1, 3 ] );
    ( 1,14,13, 2)( 3,17, 8,18)( 4,12)( 5,20, 6,15)( 7,10,11, 9)(16,24)
    (19,22,23,21)
    gap> CoxeterWord( W, p );
    [ 1, 3, 1, 2, 3 ]
    gap> W.reflectionsLabels:="stuv";
    "stuv"
    gap> CoxeterWord(W,p);
    "sustu"|

We  notice  that  the  word  we   started  with  and  the  one  that  we
ended  up  with,  are  not  the same.  But  of  course,  they  represent
the  same  element of  $W$.  The  reason  for  this difference  is  that
the  function 'CoxeterWord'  always  computes a  reduced  word which  is
the  lexicographically smallest  among  all possible  expressions of  an
element of  $W$ as a word  in the fundamental reflections.  The function
'ReducedCoxeterWord' does the same but with a word as input (rather than
an element  of the  group). Below are  some other  possible computations
with the same Coxeter group as above\:

|    gap> LongestCoxeterWord( W );  # the (unique) longest element in W
    [ 1, 2, 3, 1, 2, 3, 4, 3, 1, 2, 3, 4 ]
    gap> w0 := LongestCoxeterElement( W ); # = EltWord( W, last )
    ( 1,13)( 2,14)( 3,15)( 4,16)( 5,17)( 6,18)( 7,19)( 8,20)( 9,21)(10,22)
    (11,23)(12,24)
    gap> CoxeterLength( W, w0 );
    12
    gap> List( Reflections( W ), i -> CoxeterWord( W, i ) );
    [ "s", "t", "u", "v", "sus", "tut", "uvu", "stust", "suvus", "tuvut",
      "stuvust", "ustuvustu" ]
    gap> l := List( [0 .. W.N], x -> CoxeterElements( W, x ) );;
    gap> List( l, Length );
    [ 1, 4, 9, 16, 23, 28, 30, 28, 23, 16, 9, 4, 1 ] |

The above line tells us that there is 1 element of length 0, there are 4
elements of length 4, etc.

For  many basic  functions (like  'Bruhat', 'CoxeterLength',  etc.) we have
chosen  the convention  that the  input is  an element  of a  Coxeter group
(rather  than a Coxeter word). The reason is that for a Coxeter group which
is  a permutation  group, if  in some  application one  has to  do a lot of
computations  with Coxeter group  elements then using  the low level {\GAP}
functions  for permutations is usually  much faster than manipulating lists
of reduced expressions.

Before  describing functions  applicable to  Coxeter groups  and Coxeter
words we describe functions which  build two familiar examples.

%%%%%%%%%%%%%%%%%%%%%%%%%%%%%%%%%%%%%%%%%%%%%%%%%%%%%%%%%%%%%%%%%%%%%%%%
\Section{CoxeterGroupSymmetricGroup}
\index{CoxeterGroupSymmetricGroup}

'CoxeterGroupSymmetricGroup( <n> )'

returns the symmetric group on <n> letters as a Coxeter group.   We  give
the  code  of  this  function  as  it  is a good example on how to make a
Coxeter group\:

|    gap> CoxeterGroupSymmetricGroup := function ( n )
    > local  W;
    > W := SymmetricGroup( n );
    > W.reflections := List( [ 1 .. n - 1 ], i->(i,i + 1) );
    > W.operations.IsLeftDescending := function ( W, w, i )
    >       return i ^ w > (i + 1) ^ w;
    >   end;
    > AbsCoxOps.CompleteCoxeterGroupRecord( W );
    > return W;
    > end;
    function ( n ) ... end|

In   the   above,  we   first   set   the  generating   reflections   of
$W$   to  be   the  elementary   transpositions  '(i,i+1)'   (which  are
reflections  in  the  natural  representation  of  the  symmetric  group
permuting  the  standard  basis  of an  $n$-dimensional  vector  space),
then   give  the   'IsLeftDescending'   function   (which  just   checks
if   '(i,i+1)'   is  an   inversion   of   the  permutation).   Finally,
'AbsCoxOps.CompleteCoxeterGroupRecord' is a  service routine which fills
other fields from the ones we gave. We can see what it did by doing\:

|    gap> PrintRec(CoxeterGroupSymmetricGroup(3));
    rec(
      isDomain                        := true,
      isGroup                         := true,
      identity                        := (),
      generators                      := [ (1,3), (2,3) ],
      operations                      := HasTypeOps,
      isPermGroup                     := true,
      isFinite                        := true,
      1                               := (1,3),
      2                               := (2,3),
      degree                          := 3,
      reflections                     := [ (1,2), (2,3) ],
      nbGeneratingReflections         := 2,
      generatingReflections           := [ 1 .. 2 ],
      EigenvaluesGeneratingReflections:= [ 1/2, 1/2 ],
      isCoxeterGroup                  := true,
      reflectionsLabels               := [ 1 .. 2 ],
      coxeterMat                      := [ [ 1, 3 ], [ 3, 1 ] ],
      orbitRepresentative             := [ 1, 1 ],
      longestElm                      := (1,3),
      longestCoxeterWord              := [ 1, 2, 1 ],
      N                               := 3 )|

We  do   not  indicate  all  the   fields  here.  Some  are   there  for
technical  reasons   and  may   change  from   version  to   version  of
\CHEVIE.  Among  the  added  fields,  we  see  'nbGeneratingReflections'
(taken  to   be  'Length(W.reflections)'   if  we   do  not   give  it),
'.OrdersGeneratingReflections',  the  Coxeter  matrix  '.coxeterMat',  a
description of conjugacy classes of  the generating reflections given in
'.orbitRepresentative' (whose  $i$-th entry is  the smallest index  of a
reflection  conjugate to  '.reflections[i]'), '.reflectionsLabels'  (the
default labels used for Coxeter word). At the end are 3 fields which are
computed only  for finite Coxeter  groups\:\ the longest element,  as an
element and as a Coxeter word, and in 'W.N' the number of reflections in
$W$ (which is also the length of the longest Coxeter word).

%%%%%%%%%%%%%%%%%%%%%%%%%%%%%%%%%%%%%%%%%%%%%%%%%%%%%%%%%%%%%%%%%%%%%%%%
\Section{CoxeterGroupHyperoctaedralGroup}
\index{CoxeterGroupHyperoctaedralGroup}

'CoxeterGroupHyperoctaedralGroup( <n> )'

returns the hyperoctaedral group of rank <n> as a Coxeter group. It is
given as a permutation group on $2n$ letters, with Coxeter generators
the permutations '(i,i+1)(2n+1-i,2n-i)' and '(n,n+1)'.

|    gap> CoxeterGroupHyperoctaedralGroup(2);
    Group( (2,3), (1,2)(3,4) )|

%%%%%%%%%%%%%%%%%%%%%%%%%%%%%%%%%%%%%%%%%%%%%%%%%%%%%%%%%%%%%%%%%%%%%%%%
\Section{CoxeterMatrix}
\index{CoxeterMatrix}

'CoxeterMatrix( <W> )'

return the Coxeter  matrix of the Coxeter group <W>,  that is the matrix
whose entry  'm[i][j]' contains the  order of $g_i\*g_j$ where  $g_i$ is
the $i$-th Coxeter generator of <W>. An infinite order is represented by
the entry 0.

|    gap> W:=CoxeterGroupSymmetricGroup(4);
    Group( (1,4), (2,4), (3,4) )
    gap> CoxeterMatrix(W);
    [ [ 1, 3, 2 ], [ 3, 1, 3 ], [ 2, 3, 1 ] ]|

%%%%%%%%%%%%%%%%%%%%%%%%%%%%%%%%%%%%%%%%%%%%%%%%%%%%%%%%%%%%%%%%%%%%%%%%
\Section{CoxeterGroupByCoxeterMatrix}
\index{CoxeterGroupByCoxeterMatrix}

'CoxeterGroupByCoxeterMatrix( <m> )'

returns  the  Coxeter  group  whose  Coxeter  matrix  is  <m>.

The  matrix <m> should  be a symmetric  integer matrix such that 'm[i,i]=1'
and 'm[i,j]>=2' (or 'm[i,j]=0' to represent an infinite entry).

The  group is constructed as a  matrix group, using the standard reflection
representation  for Coxeter  groups. This  is the  representation on a real
vector  space $V$ of dimension ' Length(m)' defined as follows \:\ if $e_s$
is  a  basis  of  $V$  indexed  by  the  lines  of  $m$, we make the $s$-th
reflection act by $s(x)=x-2\langle x, e_s\rangle e_s$ where
$\langle,\rangle$   is  the  bilinear  form  on  $V$  defined  by  $\langle
e_s,e_t\rangle=-\cos(\pi/m[s,t])$  (where  by  convention $\pi/m[s,t]=0$ if
$m[s,t]=\infty$,   which   is   represented   in   {\CHEVIE}   by   setting
'm[s,t]\:=0').  In the example below the  affine Weyl group $\tilde A_2$ is
constructed, and then $\tilde A_1$.

|    gap> m:=[[1,3,3],[3,1,3],[3,3,1]];;
    gap> W:=CoxeterGroupByCoxeterMatrix(m);
    CoxeterGroupByCoxeterMatrix([ [ 1, 3, 3 ], [ 3, 1, 3 ], [ 3, 3, 1 ] ])
    gap> CoxeterWords(W,3);
    [ [ 1, 3, 2 ], [ 1, 2, 3 ], [ 1, 2, 1 ], [ 1, 3, 1 ], [ 2, 1, 3 ],
      [ 3, 1, 2 ], [ 2, 3, 2 ], [ 2, 3, 1 ], [ 3, 2, 1 ] ]
    gap> CoxeterGroupByCoxeterMatrix([[1,0],[0,1]]);
    CoxeterGroupByCoxeterMatrix([[1,0],[0,1]])|

%%%%%%%%%%%%%%%%%%%%%%%%%%%%%%%%%%%%%%%%%%%%%%%%%%%%%%%%%%%%%%%%%%%%%%%%
\Section{CoxeterGroupByCartanMatrix}
\index{CoxeterGroupByCartanMatrix}

'CoxeterGroupByCartanMatrix( <m> )'

<m>  should be a square  matrix of real cyclotomic  numbers. It returns the
reflection  group  whose  Cartan  matrix  is  <m>.  This  is a matrix group
constructed  as follows.  Let $V$  be a  real vector  space of  dimension '
Length(m)',  and  let  $\langle,\rangle$  be  the  bilinear form defined by
$\langle e_i, e_j\rangle=m[i,j]$ where $e_i$ is the canonical basis of $V$.
Then   the  result  is  the  matrix  group  generated  by  the  reflections
$s_i(x)=x-2\langle x, e_i\rangle e_i$.

This  function  is  used  in  'CoxeterGroupByCoxeterMatrix', using also the
function 'CartanMatFromCoxeterMatrix'.

|    gap> CartanMatFromCoxeterMatrix([[1,0],[0,1]]);
    [ [ 2, -2 ], [ -2, 2 ] ]
    gap> CoxeterGroupByCartanMatrix(last);
    CoxeterGroupByCartanMatrix([[2,-2],[-2,2]])|

Above is another way to construct $\tilde A_1$.

%%%%%%%%%%%%%%%%%%%%%%%%%%%%%%%%%%%%%%%%%%%%%%%%%%%%%%%%%%%%%%%%%%%%%%%%
\Section{CartanMatFromCoxeterMatrix}
\index{CartanMatFromCoxeterMatrix}

'CartanMatFromCoxeterMatrix( <m> )'

The  argument is  a CoxeterMatrix  for a  finite Coxeter  group <W> and the
result is a Cartan Matrix for the standard reflection representation of <W>
(see  "CartanMat"). Its diagonal terms are  $2$ and the coefficient between
two  generating reflections $s$  and $t$ is  $-2\cos(\pi/m[s,t])$ (where by
convention  $\pi/m[s,t]=0$  if  $m[s,t]=\infty$,  which  is  represented in
{\CHEVIE} by setting 'm[s,t]\:=0').

|    gap> m:=[[1,3],[3,1]];
    [ [ 1, 3 ], [ 3, 1 ] ]
    gap> CartanMatFromCoxeterMatrix(m);
    [ [ 2, -1 ], [ -1, 2 ] ]|

%%%%%%%%%%%%%%%%%%%%%%%%%%%%%%%%%%%%%%%%%%%%%%%%%%%%%%%%%%%%%%%%%%%%%%%%
\Section{Functions for general Coxeter groups}

Some functions take advantage of the fact a group is a Coxeter group to
use a better algorithm. A typical example is\:

'Elements(<W>)'
\index{Elements}

For  finite Coxeter  groups, uses  a  recursive algorithm  based on  the
construction of elements of a chain of parabolic subgroups

\index{ReflectionSubgroup}
'ReflectionSubgroup(<W>, <J>)'

When  <I>  is  a  subset of  |[1..W.nbGeneratingReflections]|  then  the
reflection  subgroup  of  <W>   generated  by  'W.reflections\{I\}'  can
be  generated  abstractly  (without  any specific  knowledge  about  the
representation of  <W>) as a  Coxeter group.  This is what  this routine
does\:\  implement a  special case  of 'ReflectionSubgroup'  which works
for  arbitrary Coxeter  groups  (see  "ReflectionSubgroup"). The  actual
argument <J>  should be reflection labels  for <W>, i.e. be  a subset of
'W.reflectionsLabels'.

Similarly,     the     functions     'ReducedRightCosetRepresentatives',
'PermCosetsSubgroup', work  for reflection subgroups of  the above form.
See  the chapter  on reflection  subgroups  for a  description of  these
functions.

\index{CartanMat}
'CartanMat(<W>)'

Returns       'CartanMatFromCoxeterMatrix(CoxeterMatrix(<W>))'      (see
"CartanMatFromCoxeterMatrix").

The  functions  'ReflectionType',  'ReflectionName'  and  all  functions
depending on the classification of finite Coxeter groups work for finite
Coxeter groups. See  the chapter on reflection groups  for a description
of these functions.

\index{BraidRelations}
'BraidRelations(<W>)'

returns the braid relations implied by the Coxeter matrix of <W>.

%%%%%%%%%%%%%%%%%%%%%%%%%%%%%%%%%%%%%%%%%%%%%%%%%%%%%%%%%%%%%%%%%%%%%%%%
\Section{IsLeftDescending}
\index{IsLeftDescending}

'IsLeftDescending( <W> , <w>, <i> )'

returns  'true'  if  and  only   if  the  $i$-th  generating  reflection
'W.reflections[i]' is in the left descent set of the element <w> of <W>.

|    gap> W:=CoxeterGroupSymmetricGroup(3);
    Group( (1,3), (2,3) )
    gap> IsLeftDescending(W,(1,2),1);
    true|

%%%%%%%%%%%%%%%%%%%%%%%%%%%%%%%%%%%%%%%%%%%%%%%%%%%%%%%%%%%%%%%%%%%%%%%%%

\Section{FirstLeftDescending}
\index{FirstLeftDescending}

'FirstLeftDescending( <W> , <w> )'

returns the  index in the list  of generating reflections of  <W> of the
first element of the  left descent set of the element  <w> of <W> (i.e.,
the first <i> such that  if 's=W.reflections[i]' then $l(sw)\<l(w)$). It
is quite  important to think  of using  this function rather  than write
a  loop  like  'First([1..W.nbGeneratingReflections],IsLeftDescending)',
since for particular classes of  groups (e.g. finite Coxeter groups) the
function is much optimized compared to such a loop.

|    gap> W:=CoxeterGroupSymmetricGroup(3);
    Group( (1,3), (2,3) )
    gap> FirstLeftDescending(W,(2,3));
    2|

%%%%%%%%%%%%%%%%%%%%%%%%%%%%%%%%%%%%%%%%%%%%%%%%%%%%%%%%%%%%%%%%%%%%%%%%%
\Section{LeftDescentSet}
\index{LeftDescentSet}

'LeftDescentSet( <W>, <w> )'

The set of generators $s$ such that $l(sw)\<l(w)$, given  as  a  list  of
labels for the corresponding generating reflections.

|    gap> W:=CoxeterGroupSymmetricGroup(3);
     Group( (1,3), (2,3) )
    gap> LeftDescentSet( W, (1,3));
    [ 1, 2 ] |

See also "RightDescentSet".

%%%%%%%%%%%%%%%%%%%%%%%%%%%%%%%%%%%%%%%%%%%%%%%%%%%%%%%%%%%%%%%%%%%%%%%%%
\Section{RightDescentSet}
\index{RightDescentSet}

'RightDescentSet( <W>, <w> )'

The set of generators $s$ such that $l(ws)\< l(w)$, given as  a  list  of
labels for the corresponding generating reflections.

|    gap> W := CoxeterGroup( "A", 2 );;
    gap> w := EltWord( W, [ 1, 2 ] );;
    gap> RightDescentSet( W, w );
    [ 2 ] |

See also "LeftDescentSet".

%%%%%%%%%%%%%%%%%%%%%%%%%%%%%%%%%%%%%%%%%%%%%%%%%%%%%%%%%%%%%%%%%%%%%%%%%

\Section{EltWord}
\index{EltWord}

'EltWord( <W> , <w> )'

returns  the element of <W> which corresponds to the Coxeter word <w>. Thus
it  returns a permutation if <W> is a permutation group (the usual case for
finite  Coxeter  groups)  and  a  matrix  for matrix groups (such as affine
Coxeter  groups).

|    gap> W:=CoxeterGroupSymmetricGroup(4);
    Group( (1,4), (2,4), (3,4) )
    gap> EltWord(W,[1,2,3]);
    (1,4,3,2)|

See also "CoxeterWord".

%%%%%%%%%%%%%%%%%%%%%%%%%%%%%%%%%%%%%%%%%%%%%%%%%%%%%%%%%%%%%%%%%%%%%%%%%
\Section{CoxeterWord}
\index{CoxeterWord}

'CoxeterWord( <W> , <w> )'

returns a reduced  word in the standard generators of  the Coxeter group
<W>  for  the  element  <w>  (represented as  the  {\GAP}  list  of  the
corresponding reflection labels).

|    gap> W := CoxeterGroup( "A", 3 );;
    gap> w := ( 1,11)( 3,10)( 4, 9)( 5, 7)( 6,12);;
    gap> w in W;
    true
    gap> CoxeterWord( W, w );
    [ 1, 2, 3, 2, 1 ] |

The result  of 'CoxeterWord'  is the lexicographically  smallest reduced
word  for~<w> (for  the  ordering  of the  Coxeter  generators given  by
'W.reflections').

See also "EltWord" and "ReducedCoxeterWord".

%%%%%%%%%%%%%%%%%%%%%%%%%%%%%%%%%%%%%%%%%%%%%%%%%%%%%%%%%%%%%%%%%%%%%%%%%
\Section{CoxeterLength}
\index{CoxeterLength}

'CoxeterLength( <W> , <w> )'

returns the length of the element <w>  of <W> as a reduced expression in
the standard generators.

|    gap> W := CoxeterGroup( "F", 4 );;
    gap> p := EltWord( W, [ 1, 2, 3, 4, 2 ] );
    ( 1,44,38,25,20,14)( 2, 5,40,47,48,35)( 3, 7,13,21,19,15)
    ( 4, 6,12,28,30,36)( 8,34,41,32,10,17)( 9,18)(11,26,29,16,23,24)
    (27,31,37,45,43,39)(33,42)
    gap> CoxeterLength( W, p );
    5
    gap> CoxeterWord( W, p );
    [ 1, 2, 3, 2, 4 ] |

%%%%%%%%%%%%%%%%%%%%%%%%%%%%%%%%%%%%%%%%%%%%%%%%%%%%%%%%%%%%%%%%%%%%%%%%%
\Section{ReducedCoxeterWord}
\index{ReducedCoxeterWord}

'ReducedCoxeterWord( <W> , <w> )'

returns a  reduced expression for an  element of the Coxeter  group <W>,
which is  given as a {\GAP}  list of reflection labels  for the standard
generators of <W>.

|    gap> W := CoxeterGroup( "E", 6 );;
    gap> ReducedCoxeterWord( W, [ 1, 1, 1, 1, 1, 2, 2, 2, 3 ] );
    [ 1, 2, 3 ] |

%%%%%%%%%%%%%%%%%%%%%%%%%%%%%%%%%%%%%%%%%%%%%%%%%%%%%%%%%%%%%%%%%%%%%%%%%
\Section{BrieskornNormalForm}
\index{BrieskornNormalForm}

'BrieskornNormalForm( <W> , <w> )'

Brieskorn \cite{Bri71}  has noticed that  if $L(w)$ is the  left descent
set  of $w$  (see "LeftDescentSet"),  and if  $w_{L(w)}$ is  the longest
Coxeter element (see "LongestCoxeterElement") of the reflection subgroup
$W_{L(w)}$ (note that this  element is  an involution),  then $w_{L(w)}$
divides  $w$, in  the  sense that  $l(w_{L(w)})+l(w_{L(w)}^{-1}w)=l(w)$.
We  can  now  divide  $w$   by  $w_{L(w)}$  and  continue  this  process
with  the  quotient.  In  this  way,  we  obtain  a  reduced  expression
$w=w_{L_1}  \cdots w_{L_r}$  where $L_i=L(w_{L_i}  \cdots w_{L_r})$  for
all  $i$,  which  we  call  the *Brieskorn  normal  form*  of  $w$.  The
function 'BrieskornNormalForm'  will return a description  of this form,
by  returning  the  list  of   sets  $L(w)$  which  describe  the  above
decomposition.

|    gap> W:=CoxeterGroup("E",8);
    CoxeterGroup("E",8)
    gap> w:=[ 2, 3, 4, 2, 3, 4, 5, 4, 2, 3, 4, 5, 6, 5, 4, 2, 3, 4,
    >   5, 6, 7, 6, 5, 4, 2, 3, 4, 5, 6, 7, 8 ];;
    gap> BrieskornNormalForm(W,EltWord(W,w));
        [ [ 2, 3, 4, 5, 6, 7 ], [ 8 ] ]
    gap> EltWord(W,w)=Product(last,x->LongestCoxeterElement(W,x));
    true|

%%%%%%%%%%%%%%%%%%%%%%%%%%%%%%%%%%%%%%%%%%%%%%%%%%%%%%%%%%%%%%%%%%%%%%%%%

\Section{LongestCoxeterElement}
\index{LongestCoxeterElement}

'LongestCoxeterElement( <W> [,<I>])'

If <W>  is finite, returns the  unique element of maximal  length of the
Coxeter group <W>. May loop infinitely otherwise.

|    gap> LongestCoxeterElement( CoxeterGroupSymmetricGroup( 4 ) );
    (1,4)(2,3)|

If  a second  argument <I>  is given,  returns the  longest element  of the
parabolic  subgroup generated by the reflections in <I> (where <I> is given
as '.reflectionsLabels').

|    gap> LongestCoxeterElement(CoxeterGroupSymmetricGroup(4),[2,3]);
    (2,4)|

%%%%%%%%%%%%%%%%%%%%%%%%%%%%%%%%%%%%%%%%%%%%%%%%%%%%%%%%%%%%%%%%%%%%%%%%%
\Section{LongestCoxeterWord}
\index{LongestCoxeterWord}

'LongestCoxeterWord( <W> )'

If  <W>  is  finite,  returns  a  reduced  expression  in  the  standard
generators for the unique element of maximal length of the Coxeter group
<W>. May loop infinitely otherwise.

|    gap> LongestCoxeterWord( CoxeterGroupSymmetricGroup( 5 ) );
    [ 1, 2, 1, 3, 2, 1, 4, 3, 2, 1 ] |

%%%%%%%%%%%%%%%%%%%%%%%%%%%%%%%%%%%%%%%%%%%%%%%%%%%%%%%%%%%%%%%%%%%%%%%%%
\Section{CoxeterElements}
\index{CoxeterElements}

'CoxeterElements( <W>[, <l>] )'

With  one argument this is equivalent  to 'Elements(W)' --- this works only
if  <W> is finite.  The returned elements  are sorted by increasing Coxeter
length.  If the second argument is an  integer <l>, the elements of Coxeter
length  <l>  are  returned.  The  second  argument  can  also  be a list of
integers, and the result is a list of same length as <l> of lists where the
<i>-th list contains the elements of Coxeter length 'l[i]'.

|    gap> W := CoxeterGroup( "G", 2 );;
    gap> e := CoxeterElements( W, 6 );
    [ ( 1, 7)( 2, 8)( 3, 9)( 4,10)( 5,11)( 6,12) ]
    gap> e[1] = LongestCoxeterElement( W );
    true |

After  the call to  'CoxeterElements(W,l)', the list  of elements of 'W' of
Coxeter  length 'l' is stored in the component 'elts[l+1]' of the record of
$W$.  There are  a number  of programs  (like "BruhatPoset")  which use the
lists 'W.elts'.

%%%%%%%%%%%%%%%%%%%%%%%%%%%%%%%%%%%%%%%%%%%%%%%%%%%%%%%%%%%%%%%%%%%%%%%%%
\Section{CoxeterWords}
\index{CoxeterWords}

'CoxeterWords( <W>[, <l>] )'

With second argument the integer <l> returns the list of 'CoxeterWord's for
all elements of 'CoxeterLength' <l> in the Coxeter group <W>.

If  only one  argument is  given, returns  all elements  of <W>  as Coxeter
words, in the same order as

|Concatenation(List([0..W.N],i->CoxeterWords(W,i)))|

this only makes sense for finite Coxeter groups.

|    gap> CoxeterWords( CoxeterGroup( "G", 2 ) );
    [ [  ], [ 2 ], [ 1 ], [ 2, 1 ], [ 1, 2 ], [ 2, 1, 2 ], [ 1, 2, 1 ],
      [ 2, 1, 2, 1 ], [ 1, 2, 1, 2 ], [ 2, 1, 2, 1, 2 ],
      [ 1, 2, 1, 2, 1 ], [ 1, 2, 1, 2, 1, 2 ] ] |

%%%%%%%%%%%%%%%%%%%%%%%%%%%%%%%%%%%%%%%%%%%%%%%%%%%%%%%%%%%%%%%%%%%%%%%%%
\Section{Bruhat}
\index{Bruhat}

'Bruhat( <W>, <y>, <w> )'

returns 'true', if the element <y> is less than or equal to the element <w>
of  the Coxeter group <W> for the  Bruhat order, and 'false' otherwise (<y>
is  less than <w> if a reduced expression for <y> can be extracted from one
for  <w>). See \cite[(5.9) and (5.10)]{Hum90}  for properties of the Bruhat
order.

|    gap> W := CoxeterGroup( "H", 3 );;
    gap> w := EltWord( W, [ 1, 2, 1, 3 ] );;
    gap> b := Filtered( Elements( W ), x -> Bruhat( W, x, w) );;
    gap> List( b, x -> CoxeterWord( W, x ) );
    [ [  ], [ 3 ], [ 2 ], [ 1 ], [ 2, 1 ], [ 2, 3 ], [ 1, 3 ], [ 1, 2 ],
      [ 2, 1, 3 ], [ 1, 2, 1 ], [ 1, 2, 3 ], [ 1, 2, 1, 3 ] ]|

%%%%%%%%%%%%%%%%%%%%%%%%%%%%%%%%%%%%%%%%%%%%%%%%%%%%%%%%%%%%%%%%%%%%%%%%%
\Section{BruhatSmaller}
\index{BruhatSmaller}

'BruhatSmaller( <W>, <w>)'

Returns  a list  whose $i$-th  element is  the list  of elements  of <W>
smaller for  the Bruhat  order that  <w> and of  Length $i-1$.  Thus the
first element  of the returned  list contains only 'W.identity'  and the
'CoxeterLength(W,w)'-th element contains only <w>.

|    gap> W:=CoxeterGroupSymmetricGroup(3);
    Group( (1,3), (2,3) )
    gap> BruhatSmaller(W,(1,3));
    [ [ () ], [ (2,3), (1,2) ], [ (1,2,3), (1,3,2) ], [ (1,3) ] ]|

%%%%%%%%%%%%%%%%%%%%%%%%%%%%%%%%%%%%%%%%%%%%%%%%%%%%%%%%%%%%%%%%%%%%%%%%%
\Section{BruhatPoset}
\index{BruhatPoset}

'BruhatPoset( <W> [, <w>])'

Returns as a poset (see "Poset") the Bruhat poset of <W>. If an element <w>
is given, only the poset of the elements smaller than <w> is given.

|    gap> W:=CoxeterGroup("A",2);          
    CoxeterGroup("A",2)
    gap> BruhatPoset(W);
    Poset with 6 elements
    gap> Display(last);
    <1,2<21,12<121
    gap> W:=CoxeterGroup("A",3);
    CoxeterGroup("A",3)
    gap> BruhatPoset(W,EltWord(W,[1,3]));
    Poset with 4 elements
    gap> Display(last);
    <3,1<13|

%%%%%%%%%%%%%%%%%%%%%%%%%%%%%%%%%%%%%%%%%%%%%%%%%%%%%%%%%%%%%%%%%%%%%%%%%
\Section{ReducedInRightCoset}
\index{ReducedInRightCoset}

'ReducedInRightCoset( <W>, <w>)'

Let  <w>  be an  element  of  a parent  group  of  <W> whose  action  by
conjugation induces  an automorphism of  Coxeter groups on <W>,  that is
sends the Coxeter generators of <W> to a conjugate set (but may not send
the  tuple of  generators to  a conjugate  tuple). 'ReducedInRightCoset'
returns  the  unique   element  in  the  right  coset   $W.w$  which  is
$W$-reduced, that  is which preserves  the set of Coxeter  generators of
$W$.

|    gap> W:=CoxeterGroupSymmetricGroup(6);
    Group( (1,6), (2,6), (3,6), (4,6), (5,6) )
    gap> H:=ReflectionSubgroup(W,[2..4]);
    Subgroup( Group( (1,6), (2,6), (3,6), (4,6), (5,6) ),
    [ (2,3), (3,4), (4,5) ] )
    gap> ReducedInRightCoset(H,(1,6)(2,4)(3,5));
    (1,6)|

%%%%%%%%%%%%%%%%%%%%%%%%%%%%%%%%%%%%%%%%%%%%%%%%%%%%%%%%%%%%%%%%%%%%%%%%%
\Section{ForEachElement}
\index{ForEachElement}

'ForEachElement( <W>, <f>)'

This  functions calls  'f(x)' for  each element  <x> of  the finite Coxeter
group  <W>. It is quite useful when the 'Size' of <W> would make impossible
to call 'Elements(W)'. For example,

|    gap> i:=0;;
    gap> W:=CoxeterGroup("E",7);;
    gap> ForEachElement(W,function(x)i:=i+1;
    > if i mod 1000000=0 then Print("*\c");fi;
    > end);Print("\n");
    **|

prints  a |*| about every second on  a 3Ghz computer, so enumerates 1000000
elements per second.

%%%%%%%%%%%%%%%%%%%%%%%%%%%%%%%%%%%%%%%%%%%%%%%%%%%%%%%%%%%%%%%%%%%%%%%%%
\Section{ForEachCoxeterWord}
\index{ForEachCoxeterWord}

'ForEachCoxeterWord( <W>, <f>)'

This functions calls 'f(x)' for each coxeter word <x> of the finite Coxeter
group  <W>. It is quite useful when the 'Size' of <W> would make impossible
to call 'CoxeterWords(W)'. For example,

|    gap> i:=0;;
    gap> W:=CoxeterGroup("E",7);;
    gap> ForEachCoxeterWord(W,function(x)i:=i+1;
    > if i mod 1000000=0 then Print("*\c");fi;
    > end);Print("\n");
    **|

prints  a |*| about every second on  a 3Ghz computer, so enumerates 1000000
elements per second.

%%%%%%%%%%%%%%%%%%%%%%%%%%%%%%%%%%%%%%%%%%%%%%%%%%%%%%%%%%%%%%%%%%%%%%%%%
\Section{StandardParabolicClass}
\index{StandardParabolicClass}

'StandardParabolicClass( <W>, <I>)'

<I>  should be a subset of 'W.reflectionsLabels' describing a subset of the
generating reflections for <W>. The function returns the list of subsets of
'W.reflectionsLabels' corresponding to sets of reflections conjugate to the
given subset.

|    gap> StandardParabolicClass(CoxeterGroup("E",8),[7,8]);
    [ [ 1, 3 ], [ 2, 4 ], [ 3, 4 ], [ 4, 5 ], [ 5, 6 ], [ 6, 7 ],
      [ 7, 8 ] ]|

%%%%%%%%%%%%%%%%%%%%%%%%%%%%%%%%%%%%%%%%%%%%%%%%%%%%%%%%%%%%%%%%%%%%%%%%%
\Section{ParabolicRepresentatives}
\index{ParabolicRepresentatives}

'ParabolicRepresentatives(<W> [, <r>])'

Returns   a   list   of   subsets   of   'W.reflectionsLabels'   describing
representatives  of orbits of parabolic subgroups under conjugation by $W$.
If  a second  argument <r>  is given,  returns only  representatives of the
parabolic subgroups of semisimple rank <r>.

|    gap> ParabolicRepresentatives(Affine(CoxeterGroup("A",3)));
    [ [  ], [ 1 ], [ 1, 2 ], [ 1, 2, 3 ], [ 1, 2, 3, 4 ], [ 1, 2, 4 ],
      [ 1, 3 ], [ 1, 3, 4 ], [ 2, 3, 4 ], [ 2, 4 ] ]
    gap> ParabolicRepresentatives(Affine(CoxeterGroup("A",3)),2);
    [ [ 1, 2 ], [ 1, 3 ], [ 2, 4 ] ]|

%%%%%%%%%%%%%%%%%%%%%%%%%%%%%%%%%%%%%%%%%%%%%%%%%%%%%%%%%%%%%%%%%%%%%%%%%
\Section{ReducedExpressions}
\index{ReducedExpressions}

'ReducedExpressions(<W> , <w>)'

Returns  the list  of all  reduced expressions  of the  element <w>  of the
Coxeter group <W>.

|    gap> W:=CoxeterGroup("A",3);
    CoxeterGroup("A",3)
    gap> ReducedExpressions(W,LongestCoxeterElement(W));
    [ [ 1, 2, 1, 3, 2, 1 ], [ 1, 2, 3, 1, 2, 1 ], [ 1, 2, 3, 2, 1, 2 ], 
      [ 1, 3, 2, 1, 3, 2 ], [ 1, 3, 2, 3, 1, 2 ], [ 2, 1, 2, 3, 2, 1 ], 
      [ 2, 1, 3, 2, 1, 3 ], [ 2, 1, 3, 2, 3, 1 ], [ 2, 3, 1, 2, 1, 3 ], 
      [ 2, 3, 1, 2, 3, 1 ], [ 2, 3, 2, 1, 2, 3 ], [ 3, 1, 2, 1, 3, 2 ], 
      [ 3, 1, 2, 3, 1, 2 ], [ 3, 2, 1, 2, 3, 2 ], [ 3, 2, 1, 3, 2, 3 ], 
      [ 3, 2, 3, 1, 2, 3 ] ]|

%%%%%%%%%%%%%%%%%%%%%%%%%%%%%%%%%%%%%%%%%%%%%%%%%%%%%%%%%%%%%%%%%%%%%%%%%
