%%%%%%%%%%%%%%%%%%%%%%%%%%%%%%%%%%%%%%%%%%%%%%%%%%%%%%%%%%%%%%%%%%%%%%%%%%%%%
%%
%A  Cohomology package documentation                    Derek Holt
%%
%Y  Copyright 1995 -- Mathematics Institute, Univ. of Warwick, UK.
%%
%%  This file describes certain cohomology functions.
%%  They call external C programs in the cohomology package.
%%
%%
\Chapter{Cohomology}%

This chapter describes functions which may be used to perform certain
cohomological calculations on a finite group <G>.

These include\:\\
:(i) The <p>-part $Mul_p$ of the Schur multiplier <Mul> of <G>,
and a presentation
of a covering extension of $Mul_p$ by <G>, for a specified prime <p>;\\
(ii) The dimensions of the first and second cohomology groups of <G> acting on
a finite dimensional <KG> module <M>, where <K> is a field of prime order; and\\
(iii) Presentations of split and nonsplit extensions of <M> by <G>.

All of these functions require <G> to be defined as a finite permutation
group. The functions which compute presentations require, in addition, a
presentation of <G>. Finally, the functions which operate on a module <M>
require the module to be defined by a list of matrices over <K>. This
situation is handled by first defining a ${GAP}$ record, which contains the
required information. This is done using the function 'CHR', which must be
called before any of the other functions. The remaining functions operate
on this record.

If no presentation of the permutation group <G> is known, and <G> has
order at most 32767, then a presentation can be computed using the
function 'CalcPres'. On the other hand, if you start with a finitely
presented group, then you can create a permutation representation
with the function 'PermRep' (although there is no guarantee that the
representation will be faithful ingeneral).

The functions all compute and make use of a descending sequence of subgroups
of <G>, starting at <G> and ending with a Sylow <p>-subgroup of <G>, and
it is usually most efficient to have the indices of the subgroups in this
chain as small as possible. If you get a warning message, and one of
the function fails because the indices in the chain computed are too large,
then you can try to remedy matters by supplying your own chain. See
Section "Further Information" for more details, and an example.

If you set the external variable 'InfoCohomolofy' to the value 'Print',
then a small amount of information will be printed, indicating what is
happening. If <chr> is the cohomology record you are working with, and
you set the field <chr>'.verbose' to the value 'true', then you will
see all the output of the external programs.

%%%%%%%%%%%%%%%%%%%%%%%%%%%%%%%%%%%%%%%%%%%%%%%%%%%%%%%%%%%%%%%%%%%%%%%%%%%%%
\Section{CHR}
'CHR(<G>, <p>, [<F>], [<mats>] )'

'CHR'  constructs  a  cohomology-record, which is used as a parameter for
all of the other functions in this chapter. <G> must be  a finite
permutation group, and <p> a prime number. If present, <F> must either be
zero or a finitely presented group with the same number of generators as
<G>, of which the relators are satisfied by the generators of <G>.
In fact, to obtain meaningful results, <F> should almost certainly be
isomorphic to <G>. If present, <mats> should be a list of invertible matrices
over the finite field <K> = <GF(p)>. The list should have the same length as the
number of generators of <G>, and the matrices should correspond to these
generators, and define a <GF(p)G>-module, which we will denote by <M>.

%%%%%%%%%%%%%%%%%%%%%%%%%%%%%%%%%%%%%%%%%%%%%%%%%%%%%%%%%%%%%%%%%%%%%%%%%%%%%
\Section{SchurMultiplier}
'SchurMultiplier(<chr>)'

<chr> must be a cohomology-record that was created by a call of
'CHR(<G>,<p>,[<F>],[<mats>])'. 'SchurMultiplier' calculates
the <p>-part $Mul_p$ of the Schur multiplier <Mul> of <G>.
The result is returned as a list of integers, which are the
abelian invariants of $Mul_p$. If the list is empty, then
$Mul_p$ is trivial.

%%%%%%%%%%%%%%%%%%%%%%%%%%%%%%%%%%%%%%%%%%%%%%%%%%%%%%%%%%%%%%%%%%%%%%%%%%%%%
\Section{CoveringGroup}
'CoveringGroup(<chr>)'

<chr> must be a cohomology-record, created by a call of
'CHR(<G>,<p>,<F>,[<mats>])', where <F> is a finitely presented group.
'CoveringGroup' calculates a presentation of a covering extension of $Mul_p$
by <G>, where $Mul_p$ is the <p>-part of the Schur multiplier <Mul> of <G>.
The set of generators of the finitely presented group that is returned
is a union of two sets, which are in one-one correspondence with the
generators of <F> and of $Mul_p$, respectively. 

The relators fall into three classes\:\\
:a) Those that specify the orders of the generators of $Mul_p$;\\
b) Those that say that the generators of $Mul_p$ are central; and\\
c) Those that give the values of the relators of <F> as elements of $Mul_p$.

%%%%%%%%%%%%%%%%%%%%%%%%%%%%%%%%%%%%%%%%%%%%%%%%%%%%%%%%%%%%%%%%%%%%%%%%%%%%%
\Section{FirstCohomologyDimension}
'FirstCohomologyDimension(<chr>)'

<chr> must be a cohomology-record, created by a call of
'CHR(<G>,<p>,<F>,<mats>)'. (If there is no finitely presented group <F> involved,
then the third parameter of 'CHR' should be given as 0.)
'FirstCohomologyDimension' calculates and returns the dimension
over <K> = <GF(p)> of the first cohomology group $H^1(G,M)$ of the group
<G> in its action on the module <M> defined by the matrices <mats>.

%%%%%%%%%%%%%%%%%%%%%%%%%%%%%%%%%%%%%%%%%%%%%%%%%%%%%%%%%%%%%%%%%%%%%%%%%%%%%
\Section{SecondCohomologyDimension}
'SecondCohomologyDimension(<chr>)'

<chr> must be a cohomology-record, created by a call of
'CHR(<G>,<p>,<F>,<mats>)'.
(If there is no finitely presented group <F> involved,
then the third parameter of 'CHR' should be given as 0.)
'SecondCohomologyDimension' calculates and returns the dimension
over <K> = <GF(p)> of the second cohomology group $H^2(G,M)$ of the group
<G> in its action on the module <M> defined by the matrices <mats>.

%%%%%%%%%%%%%%%%%%%%%%%%%%%%%%%%%%%%%%%%%%%%%%%%%%%%%%%%%%%%%%%%%%%%%%%%%%%%%
\Section{SplitExtension}
'SplitExtension(<chr>)'

<chr> must be a cohomology-record, created by a call of
'CHR(<G>,<p>,<F>,<mats>)', where <F> is a finitely presented group.
'SplitExtension' returns a presentation of the split extension of the module
<M> defined by the matrices <mats> by the group <G>.
This is a straightforward calculation, and involves no call of the
external cohomology programs. It is provided here for convenience.

%%%%%%%%%%%%%%%%%%%%%%%%%%%%%%%%%%%%%%%%%%%%%%%%%%%%%%%%%%%%%%%%%%%%%%%%%%%%%
\Section{NonsplitExtension}
'NonsplitExtension(<chr>, [<vec>] )'

<chr> must be a cohomology-record, created by a call of
'CHR(<G>,<p>,<F>,<mats>)', where <F> is a finitely presented group.
If present, <vec> must be a list of integers of length equal to the
dimension over <K> = <GF(p)> of the second cohomology group $H^2(G,M)$ of the
group <G> in its action on the module <M> defined by the matrices <mats>.
'NonsplitExtension' calculates and returns a presentation of a nonsplit
extension of <M> by <G>. Since there may be many such extensions, and
the equivalence classes of these extensions are in one-one correspondence
with the nonzero elements of $H^2(G,M)$, the optional second parameter
can be used to specify an element of $H^2(G,M)$ as a vector.
The default value of this vector is '[1,0,...,0]'.
The set of generators of the finitely presented group that is returned
is a union of two sets, which are in one-one correspondence with the
generators of <F> and of <M> (as an abelian group), respectively. 

The relators fall into three classes\:\\
:a) Those that say that <M> is an abelian group of exponent <p>;\\
b) Those that define the action of the generators of <F> on
those of <M>; and\\
c) Those that give the values of the relators of <F> as elements of $M$.

(<Note>\: It is not particularly efficient to call 'SecondCohomologyDimension'
first to calculate the dimension of $H^2(G,M)$, which must of course be known
if the second parameter is to be given; it is preferable to call
'NonsplitExtension' immediately without the second parameter (which will
return one nonsplit extension), and then to call 'SecondCohomologyDimension',
which will at that stage return the required dimension immediately -
all subsequent calls of 'NonsplitExtension' on <chr> will also yield
immediate results.)

%%%%%%%%%%%%%%%%%%%%%%%%%%%%%%%%%%%%%%%%%%%%%%%%%%%%%%%%%%%%%%%%%%%%%%%%%%%%%
\Section{CalcPres}
'CalcPres(<chr>)'

'CalcPres'  computes a presentation of the permutation group <chr>'.permgp'
on the same generators as <chr>'.permgp', and stores it as <chr>'.fpgp'.
It currently only works for groups of order up to 32767, although that
could easily be increased if required.

%%%%%%%%%%%%%%%%%%%%%%%%%%%%%%%%%%%%%%%%%%%%%%%%%%%%%%%%%%%%%%%%%%%%%%%%%%%%%
\Section{PermRep}
'PermRep(<G>, <K>)'

'PermRep' calculates the permutation representation of the finitely
presented group <F> on the right cosets of the subgroup <K>, and returns
it as a permutation group of which the generators correspond to those
of <F>. It simply calls the {\GAP} Todd-Coxeter function. Of course,
there is no guarantee in general that this representation will be
faithful.

%%%%%%%%%%%%%%%%%%%%%%%%%%%%%%%%%%%%%%%%%%%%%%%%%%%%%%%%%%%%%%%%%%%%%%%%%%%%%
\Section{Further Information}

Suppose, as usual, that the cohomology record <chr> was constructed
with the call 'CHR(<G>,<p>,\ [<F>],[<mats>])'.
All of the functions make use of a strictly decreasing chain of subgroups
of the permutation group <G> starting with <G> itself and ending with a
Sylow <p>-subgroup <P> of <G>. In general, the programs run most efficiently
if the indices between successive terms in this sequence are as small as
possible. By default, {\GAP} will attempt to find a suitable chain, when
you call the first cohomology function on <chr>. However, you may be able
to construct a better chain yourself. If so, then you can do this
by assigning the record field <chr>'.chain' to the list <L> of subgroups
that you wish to use. You should do that before calling any of the
cohomology functions. Remeber that the first term in the list must be
<G> itself, the sequence of subgroups must be strictly decreasing,
and the last term must be equal to the Sylow subgroup stored as
<chr>'.sylow'. (You can change <chr>'.sylow' to a different Sylow
<p>-subgroup if you like.) Here is a slightly contrived example of this
process.

|    gap> RequirePackage( "cohomolo" );
    gap> G:=AlternatingGroup(16);;
    gap> chr:=CHR(G,2);;
    gap> InfoCohomology:=Print;;
    gap> SchurMultiplier(chr);
    &Indices in the subgroup chain are:  2027025 315 
    &WARNING: An index in the subgroup chain found is larger than 50000.
    &This calculation may fail. See manual for possible remedies.
    &I  Cohomology package: Calling external program.
    &I  External program complete.
    Error, 'Cohomology' failed for some reason.
     in
    Cohomology( chr, true, false, false, TmpName(  ) ) called from
    SchurMultiplier( chr ) called from
    main loop
    brk> quit;|

The first index in the chain found by GAP was hopelessly large. Let\'s try
and do better.

|    gap> P:=chr.sylow;;
    gap> H1:=Subgroup(G, [(1,2)(9,10), (2,3,4,5,6,7,8),
    >        (1,9)(2,10)(3,11)(4,12)(5,13)(6,14)(7,15)(8,16)]);;
    gap> Index(G,H1);
    6435
    gap> H2:=Subgroup(H1, [(1,2)(5,6), (1,2)(9,10), (2,3,4),
    >        (1,5)(2,6)(3,7)(4,8),
    >        (1,9)(2,10)(3,11)(4,12)(5,13)(6,14)(7,15)(8,16)]);;
    gap> Index(H1,H2);
    1225
    gap> IsSubgroup(H2,P);
    true|

If that had been false, we could have replaced chr.sylow by a Sylow
2-subgroup of H2. As it is true, we just continue.

|    gap> Index(H2,P);
    81
    gap> chr.chain := [G,H1,H2,P];;
    gap> SchurMultiplier(chr);
    &I  Cohomology package: Calling external program.
    &I  External program complete.
    &I  Removing temporary files.
    [ 2 ]|

