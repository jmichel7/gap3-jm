%%%%%%%%%%%%%%%%%%%%%%%%%%%%%%%%%%%%%%%%%%%%%%%%%%%%%%%%%%%%%%%%%%%%%%%%%%%%%
%%
%A  algext.tex                  GAP documentation            Alexander Hulpke
%%
%A  @(#)$Id: algext.tex,v 1.2 1997/01/21 14:33:34 gap Exp $
%%
%Y  Copyright (C)  1994,  Lehrstuhl D fuer Mathematik,  RWTH Aachen,  Germany
%%
%%  This file describes algebraic extensions.
%%
%H  $Log: algext.tex,v $
%H  Revision 1.2  1997/01/21 14:33:34  gap
%H  vfelsch removed mentioning degree 15 and
%H  fixed some examples, in particular for DecomPoly
%H
%H  Revision 1.1.1.1  1996/12/11 12:36:42  werner
%H  Preparing 3.4.4 for release
%H
%H  Revision 3.1  1994/08/31  12:14:02  mschoene
%H  initial revision under RCS
%H
%%
\Chapter{Algebraic extensions of fields}
\index{Field Extensions}

If we adjoin a root $\alpha$ of an irreducible polynomial $p \in K[x]$ to
the field $K$ we get an *algebraic extension* $K(\alpha)$, which is again
a field.  By Kronecker\'s construction, we may identify $K(\alpha)$ with
the factor ring $K[x]/(p)$, an identification that also provides a method
for computing in these extension fields.

Currently \GAP\ only allows extension fields of fields $K$, when $K$
itself is not an extension field.

As it is planned to modify the representation of field extensions to
unify vector space structures and to speed up computations, {\bf All
information in this chapter is subject to change in future versions}.

%%%%%%%%%%%%%%%%%%%%%%%%%%%%%%%%%%%%%%%%%%%%%%%%%%%%%%%%%%%%%%%%%%%%%%%%%
\Section{AlgebraicExtension}

'AlgebraicExtension( <pol> )'

constructs the algebraic extension <L> corresponding to the polynomial
<pol>.  <pol> must be an irreducible polynomial defined over a
``defining\'\' field <K>.  The elements of <K> are embedded into <L> in
the canonical way.  As <L> is a field, all field functions are applicable
to <L>.  Similarly, all field element functions apply to the elements of
<L>.

<L> is considered implicitely to be a field over the subfield <K>.  This
means, that functions like 'Trace' and 'Norm' relative to subfields are
not supported.

|    gap> x:=X(Rationals);;x.name:="x";;
    gap> p:=x^4+3*x^2+1;
    x^4 + 3*x^2 + 1
    gap> e:=AlgebraicExtension(p);
    AlgebraicExtension(Rationals,x^4 + 3*x^2 + 1)
    gap> e.name:="e";;
    gap> IsField(e);
    true
    gap> y:=X(GF(2));;y.name:="y";;
    gap> q:=y^2+y+1;
    Z(2)^0*(y^2 + y + 1)
    gap> f:=AlgebraicExtension(q);
    AlgebraicExtension(GF(2),Z(2)^0*(y^2 + y + 1))|

%%%%%%%%%%%%%%%%%%%%%%%%%%%%%%%%%%%%%%%%%%%%%%%%%%%%%%%%%%%%%%%%%%%%%%%%%
\Section{IsAlgebraicExtension}
\index{test!for algebraic extension}

'IsAlgebraicExtension( <D> )'

'IsAlgebraicExtension' returns 'true' if the object <D> is an algebraic
field extension and 'false' otherwise.

More  precisely, 'IsAlgebraicExtension'  tests whether  <D> is an algebraic
field  extension  record  (see  "Algebraic  Extension  Records").  So,  for
example,   a  matrix   ring  may   in  fact   be  a  field  extension,  yet
'IsAlgebraicExtension' would return 'false'.

|    gap> IsAlgebraicExtension(e);
    true
    gap> IsAlgebraicExtension(Rationals);
    false|

%%%%%%%%%%%%%%%%%%%%%%%%%%%%%%%%%%%%%%%%%%%%%%%%%%%%%%%%%%%%%%%%%%%%%%%%%
\Section{RootOf}
\index{primitive element}

'RootOf( <pol> )'

returns a root of the irreducible polynomial <pol> as element of the
corresponding extension field 'AlgebraicExtension(<pol>)'.  This root is
called the *primitive element* of this extension.

% No assumptions on *which* root (as a complex number) is selected can
% be made.

|    gap> r:=RootOf(p);
    RootOf(x^4 + 3*x^2 + 1)
    gap> r.name:="alpha";;|

%%%%%%%%%%%%%%%%%%%%%%%%%%%%%%%%%%%%%%%%%%%%%%%%%%%%%%%%%%%%%%%%%%%%%%%%%
\Section{Algebraic Extension Elements}
\index{type!algebraic elements}
\index{Operations for algebraic elements}

According  to  Kronecker\'s  construction,  the  elements  of  an algebraic
extension are considered to be polynomials in the primitive element. Unless
they  are already in the defining field (in which case they are represented
as  elements of this field), they are  represented by records in \GAP\ (see
"Extension  Element  Records").  These  records  contain a representation a
polynomial  in the primitive  element. The extension  corresponding to this
primitive element is the default field for the algebraic element.

The usual field operations are applicable to algebraic elements.

|    gap> r^3/(r^2+1);
    -1*alpha^3-1*alpha
    gap> DefaultField(r^2);
    e|


%%%%%%%%%%%%%%%%%%%%%%%%%%%%%%%%%%%%%%%%%%%%%%%%%%%%%%%%%%%%%%%%%%%%%%%%%
\Section{Set functions for Algebraic Extensions}
\index{membership test!for algebraic extensions}
\index{in!for algebraic Elements}
\index{Random!for algebraic Extensions}

As algebraic extensions are fields, all set theoretic functions are
applicable to algebraic elements.  The following two routines are treated
specially\:

\vspace{5mm}
'in'

tests,  whether a given object is  contained in an algebraic extension. The
base  field  is  embedded  in  the  natural  way  into  the  extension. Two
extensions are considered to be distinct, even if the minimal polynomial of
one has a root in the other one.

|    gap> r in e;5 in e;
    true
    true
    gap> p1:=Polynomial(Rationals,MinPol(r^2));
    x^2 + 3*x + 1
    gap> r2:=RootOf(p1);
    RootOf(x^2 + 3*x + 1)
    gap> r2 in e;
    false|

\vspace{5mm}
'Random'

A random algebraic element is computed by taking a linear combination of the
powers of the primitive element with random coefficients from the ground
field.

|    gap> ran:=Random(e);
    -1*alpha^3-4*alpha^2|

%%%%%%%%%%%%%%%%%%%%%%%%%%%%%%%%%%%%%%%%%%%%%%%%%%%%%%%%%%%%%%%%%%%%%%%%%
\Section{IsNormalExtension}
\index{test!for normal extension}

'IsNormalExtension(<L>)'
%'IsNormalExtension(<L>,<K>)'

An algebraic extension field is called a *normal extension*, if it is a
splitting field of the defining polynomial.  The second version returns
whether <L> is a normal extension of <K>.  The first version returns
whether <L> is a normal extension of its definition field.

|    gap> IsNormalExtension(e);
    true
    gap> p2:=x^4+x+1;;
    gap> e2:=AlgebraicExtension(p2);
    AlgebraicExtension(Rationals,x^4 + x + 1)
    gap> IsNormalExtension(e2);
    false|

%%%%%%%%%%%%%%%%%%%%%%%%%%%%%%%%%%%%%%%%%%%%%%%%%%%%%%%%%%%%%%%%%%%%%%%%%
\Section{MinpolFactors}

'MinpolFactors( <L> )'

returns the factorization of the defining polynomial of <L> over <L>.

|    gap> X(e).name:="X";;
    gap> MinpolFactors(e);
    [ X + (-1*alpha), X + (-1*alpha^3-3*alpha), X + (alpha), 
      X + (alpha^3+3*alpha) ]|

%%%%%%%%%%%%%%%%%%%%%%%%%%%%%%%%%%%%%%%%%%%%%%%%%%%%%%%%%%%%%%%%%%%%%%%%%
\Section{GaloisGroup for Extension Fields}
\index{Galois group!of an extension field}
\index{automorphism group!of an extension field}

'GaloisGroup( <L> )'

returns the Galois group of the field <L> if <L> is a normal extension
and issues an error if not.  The Galois group is a group of extension
automorphisms (see "ExtensionAutomorphism").

The computation of a Galois group is computationally relatively hard,
and can take significant time.

|    gap> g:=GaloisGroup(f);
    Group( ExtensionAutomorphism(AlgebraicExtension(GF(2),Z(2)^0*(y^
    2 + y + 1)),RootOf(Z(2)^0*(y^2 + y + 1))+Z(2)^0) )
    gap> h:=GaloisGroup(e);
    Group( ExtensionAutomorphism(e,alpha^3+
    3*alpha), ExtensionAutomorphism(e,-1*alpha), ExtensionAutomorphism(e,
    -1*alpha^3-3*alpha) )
    gap> Size(h);
    4
    gap> AbelianInvariants(h);
    [ 2, 2 ]|

%%%%%%%%%%%%%%%%%%%%%%%%%%%%%%%%%%%%%%%%%%%%%%%%%%%%%%%%%%%%%%%%%%%%%%%%%
\Section{ExtensionAutomorphism}
\index{field homomorphisms!of algebraic extensions}

'ExtensionAutomorphism( <L>, <img> )'

is the automorphism of the extension <L>, that maps the primitive root of
<L> to <img>.  As it is a field automorphism, section "Field
Homomorphisms" applies.

%%%%%%%%%%%%%%%%%%%%%%%%%%%%%%%%%%%%%%%%%%%%%%%%%%%%%%%%%%%%%%%%%%%%%%%%%
\Section{Field functions for Algebraic Extensions}
\index{Norm!for algebraic extensions}
\index{Trace!for algebraic extensions}
\index{DefaultField!for algebraic extensions}
\index{Field!for algebraic extensions}

As already mentioned, algebraic extensions are fields.  Thus all field
functions like 'Norm' and 'Trace' are applicable.

|    gap> Trace(r^4+2*r);
    14
    gap> Norm(ran);
    305|

'DefaultField' always returns the algebraic extension, which contains the
primitive element by which the number is represented, see "Algebraic
Extension Elements".

|    gap> DefaultField(r^2);
    e|

As subfields are not yet supported, 'Field' will issue an error, if
several elements are given, or if the element is not a primitive element
for its default field.

You can create a polynomial ring over an algebraic extension to which all
functions described in "Ring Functions for Polynomial Rings" can be
applied, for example you can factor polynomials.  Factorization is done
--- depending on the polynomial --- by factoring the squarefree norem or
using a hensel lift (with possibly added lattice reduction) as described
in \cite{Abb89}, using bounds from \cite{BTW93}.

|    gap> X(e).name:="X";;
    gap> p1:=EmbeddedPolynomial(PolynomialRing(e),p1);
    X^2 + 3*X + 1
    gap> Factors(p1);
    [ X + (-1*alpha^2), X + (alpha^2+3) ]|

%%%%%%%%%%%%%%%%%%%%%%%%%%%%%%%%%%%%%%%%%%%%%%%%%%%%%%%%%%%%%%%%%%%%%%%%%
\Section{Algebraic Extension Records}
\index{record fields!for algebraic extension fields}

Since every algebraic extension is a field, it is represented as a
record.  This record contains all components, a field record will contain
(see "Field Records").  Additionally, it contains the components
'isAlgebraicExtension', 'minpol', 'primitiveElm' and may contain the
components 'isNormalExtension', 'minpolFactors' and 'galoisType'.

'isAlgebraicExtension': \\
    is always 'true'. This indicates that <F> is an algebraic extension.

'minpol': \\
    is the defining polynomial of <F>.

'primitiveElm': \\
    contains 'RootOf(<F>.minpol)'.

'isNormalExtension': \\
     indicates, whether <F> is a normal extension field.

'minpolFactors': \\
    contains a factorization of '<F>.minpol' over <F>.

'galoisType': \\
    contains the Galois type of the normal closure of <F>.
    See section "GaloisType".

%%%%%%%%%%%%%%%%%%%%%%%%%%%%%%%%%%%%%%%%%%%%%%%%%%%%%%%%%%%%%%%%%%%%%%%%%
\Section{Extension Element Records}
\index{record fields!for extension elements}

Elements of an algebraic extension are represented by a record.  The
record for the element <e> of <L> contains the components
'isAlgebraicElement', 'domain' and 'coefficients'\:

'isAlgebraicElement': \\
    is always 'true', and indicates, that <e> is an algebraic element.

'domain': \\
    contains <L>.

'coefficients': \\
    contains the coefficients of <e> as a polynomial in the primitive
    root of <L>.

%%%%%%%%%%%%%%%%%%%%%%%%%%%%%%%%%%%%%%%%%%%%%%%%%%%%%%%%%%%%%%%%%%%%%%%%%
\Section{IsAlgebraicElement}
\index{test!for algebraic element}

'IsAlgebraicElement( <obj> )'

returns 'true' if obj is an algebraic element, i.e., an element of an
algebraic extension, that is not in the defining field, and 'false'
otherwise.

|    gap> IsAlgebraicElement(r);
    true
    gap> IsAlgebraicElement(3);
    false|

%%%%%%%%%%%%%%%%%%%%%%%%%%%%%%%%%%%%%%%%%%%%%%%%%%%%%%%%%%%%%%%%%%%%%%%%%
\Section{Algebraic extensions of the Rationals}

The following sections describe functions that are specific to algebraic
extensions of ${Q\mskip-11mu\prime\,\,}$.

%%%%%%%%%%%%%%%%%%%%%%%%%%%%%%%%%%%%%%%%%%%%%%%%%%%%%%%%%%%%%%%%%%%%%%%%%
\Section{DefectApproximation}
\index{defect}

'DefectApproximation( <L> )'

computes a multiple of the defect of the basis of <L>, given by the
powers of the primitive element.  The *defect* indicates, which
denominator is necessary in the coefficients, to express algebraic
integers in <L> as a linear combination of the base of <L>.
'DefectApproximation' takes the maximal square in the discriminant as a
first approximation, and then uses Berwicks and Hesses method (see
\cite{Bra89}) to improve this approximation.  The number returned is not
neccessarily the defect, but may be a proper multiple of it.

|    gap> DefectApproximation(e);
    1|

%%%%%%%%%%%%%%%%%%%%%%%%%%%%%%%%%%%%%%%%%%%%%%%%%%%%%%%%%%%%%%%%%%%%%%%%%
\Section{GaloisType}
\index{Galois}

'GaloisType( <L> )' \\
'Galois( <f> )'

The first version returns the number of the permutation isomorphism type
of the Galois group of the normal closure of <L>, considered as a
transitive permutation group of the roots of the defining polynomial (see
"The Transitive Groups Library").  The second version returns the Galois
type of the splitting field of <f>.  Identification is done by factoring
appropriate Galois resolvents as proposed in \cite{MS85}.  This function
is provided for rational polynomials of degree up to 15.  However, it may
be not feasible to call this function for polynomials of degree 14 or 15,
as the involved computations may be enormous.  For some polynomials of
degree 14, a complete discrimination is not yet possible, as it would
require computations, that are not feasible with current factoring
methods.

|    gap> GaloisType(e);
    2
    gap> TransitiveGroup(e.degree,2);
    E(4) = 2[x]2|

%%%%%%%%%%%%%%%%%%%%%%%%%%%%%%%%%%%%%%%%%%%%%%%%%%%%%%%%%%%%%%%%%%%%%%%%%
\Section{ProbabilityShapes}

'ProbabilityShapes( <pol> )'

returns a list of numbers, which contains most likely the isomorphism
type of the galois group of <pol> (see "GaloisType").  This routine only
applies the cycle structure test according to Tschebotareff\'s theorem.
Accordingly, it is very fast, but the result is not guaranteed to be
correct.

|    gap> ProbabilityShapes(e.minpol);
    [ 2 ]|

%%%%%%%%%%%%%%%%%%%%%%%%%%%%%%%%%%%%%%%%%%%%%%%%%%%%%%%%%%%%%%%%%%%%%%%%%
\Section{DecomPoly}
\index{decomposition!of polynomials}
\index{ideal decomposition}

'DecomPoly( <pol> )'\\
'DecomPoly( <pol>, \"all\" )'

returns an ideal decomposition of the polynomial <pol>.  An ideal
decomposition is given by two polynomials <g> and <h>, such that $pol$
divides $(g\circ h)$.  By the Galois correspondence any ideal
decomposition corresponds to a block system of the Galois group.  The
polynomial <g> defines a subfield $K(\beta)$ of $K(\alpha)$ with
$h(\alpha)=\beta$.  The first form finds one ideal decomposition, while
the second form finds all possible different ideal decompositions
(i.e. all subfields).

|    gap> d:=DecomPoly(e.minpol);
    [ x^2 + 5, x^3 + 4*x ]
    gap> p:=x^6+108;;
    gap> d:=DecomPoly(p,"all");
    [ [ x^2 + 108, x^3 ], [ x^3 + 108, x^2 ], 
      [ x^3 - 186624, x^5 + 6*x^2 ], [ x^3 + 186624, x^5 - 6*x^2 ] ]
    gap> Value(d[1][1],d[1][2]);
    x^6 + 108|



