%%%%%%%%%%%%%%%%%%%%%%%%%%%%%%%%%%%%%%%%%%%%%%%%%%%%%%%%%%%%%%%%%%%%%%%%%%%%%
%%
%A  vector.tex                  GAP documentation            Martin Schoenert
%%
%A  @(#)$Id: vector.tex,v 1.1.1.1 1996/12/11 12:36:51 werner Exp $
%%
%Y  Copyright 1990-1992,  Lehrstuhl D fuer Mathematik,  RWTH Aachen,  Germany
%%
%%  This file describes  those  functions  that  mainly  deal  with  vectors.
%%
%H  $Log: vector.tex,v $
%H  Revision 1.1.1.1  1996/12/11 12:36:51  werner
%H  Preparing 3.4.4 for release
%H
%H  Revision 3.7  1993/05/04  11:42:30  fceller
%H  fixed a spelling error
%H
%H  Revision 3.6  1993/02/19  10:48:42  gap
%H  adjustments in line length and spelling
%H
%H  Revision 3.5  1993/02/11  17:14:10  martin
%H  vectors may now contain records
%H
%H  Revision 3.4  1993/02/05  08:17:44  felsch
%H  examples fixed
%H
%H  Revision 3.3  1991/12/27  16:07:04  martin
%H  revised everything for GAP 3.1 manual
%H
%H  Revision 3.2  1991/07/26  09:01:07  martin
%H  changed |\GAP\ | to |{\GAP}|
%H
%H  Revision 3.1  1991/07/25  16:16:59  martin
%H  fixed some minor typos
%H
%H  Revision 3.0  1991/04/11  11:36:02  martin
%H  Initial revision under RCS
%H
%%
\Chapter{Vectors}%
\index{scalars}\index{base!of vector space}\index{dimension!of vector space}

A important concept in algebra  is the vector  space over a field $F$.  A
*vector space* $V$  is a set of *vectors*, for which an addition $u +  v$
and a multiplication by *scalars*, i.e., elements from $F$, $s v$ must be
defined.  A *base* of $V$ is a list of vectors, such that every vector in
$V$ can be uniquely written as  linear combination  of the  base vectors.
If the base if finite, its size is called  the *dimension* of $V$.  Using
a base it  can be  shown that $V$ is isomorphic to the set $n$-tuples  of
elements with the componentwise addition and multiplication.

This comment suggests the representation that is actually used in {\GAP}.
A {\GAP}  vector is a  list without holes  whose elements all come from a
common  field.   We call  the length   of the list  the dimension  of the
vector.  This is a little bit lax, because the dimension is a property of
the vector space, not of the vector, but should seldom cause confusion.

The first possibility for this field are the rationals (see "Rationals").
We call a list without holes whose elements  are all rationals a rational
vector, which is a bit lax too, but should  again  cause  no   confusion.
For example '[ 1/2, 0, -1/3, 2 ]' is a rational vector of dimension 4.

The second possibility  are  cyclotomics (see "Cyclotomics").   Note that
the  rationals  are the  prime field  of  cyclotomic fields and therefore
rational  vectors are  just a  special  case of  cyclotomic vectors.   An
example of a cyclotomic vector is '[ E(3)+E(3)\^2, 1, E(15) ]'.

Third the common field may be a finite field (see "Finite Fields").  Note
that it is not enough that  all elements are finite field elements of the
same characteristic, the common finite field containing all elements must
be representable  in  {\GAP}, i.e., must have  at most $2^{16}$ elements.
An  example of such  a vector over the finite  field  $GF(3^4)$  with  81
elements is '[ Z(3\^4)\^3, Z(3\^2)\^5, Z(3\^4)\^11 ]'.

Finally  a list  all of whose elements are  records is also considered  a
vector.  In that case the records should all have  an 'operations' record
with  the necessary  functions '+', '-',  '\*',  '\^'.   This allows  for
vectors  over  library  and/or user defined fields  (or rings)  such as a
polynomial ring (see "Polynomials").

The first section in this  chapter describes the operations applicable to
vectors (see "Operations for Vectors").

The  next section describes  the function that   tests if an  object is a
vector (see "IsVector").

The next section describes the function that returns a canonical multiple
of a vector (see "NormedVector").

The  last section  tells you  more  about the internal representation  of
vectors (see "More about Vectors").

Because vectors are just a special case of lists,  all the operations and
functions for lists are applicable to vectors also (see chapter "Lists").
This especially includes   accessing elements  of   a vector (see   "List
Elements"), changing elements of  a  vector (see "List  Assignment"), and
comparing vectors (see "Comparisons of Lists").

Vectorspaces  are a special  category   of domains and  are described  by
vectorspace records (see chapter "Vector Spaces").

Vectors  play  an important role  for  matrices (see chapter "Matrices"),
which are implemented as lists of vectors.

%%%%%%%%%%%%%%%%%%%%%%%%%%%%%%%%%%%%%%%%%%%%%%%%%%%%%%%%%%%%%%%%%%%%%%%%%
\Section{Operations for Vectors}%
\index{operations!for vectors}

'<vec1> + <vec2>'

In this form the addition operator '+'  evaluates to  the sum  of the two
vectors <vec1> and <vec2>, which must have the same  dimension and lie in
a common field.  The sum is a new  vector where each entry  is the sum of
the corresponding entries  of the vectors.  As an  exception it  is  also
possible to add an integer vector to a finite field vector, in which case
the integers are interpreted as '<scalar> \*\ <GF>.one'.

'<scalar> + <vec>' \\
'<vec> + <scalar>'

In  this form '+' evaluates  to the  sum of the   scalar <scalar> and the
vector <vec>, which must lie in a common field.  The sum is  a new vector
where each entry is the sum of the scalar and the  corresponding entry of
the vector.  As an exception it is also possible to add an integer scalar
to a finite field  vector, in  which case  the integer  is interpreted as
'<scalar> \*\ <GF>.one'.

|    gap> [ 1, 2, 3 ] + [ 1/2, 1/3, 1/4 ];
    [ 3/2, 7/3, 13/4 ]
    gap> [ 1/2, 3/2, 1/2 ] + 1/2;
    [ 1, 2, 1 ] |

'<vec1> - <vec2>'  \\
'<scalar> - <vec>' \\
'<vec> - <scalar>'

The difference operator '-'  returns the componentwise difference of  its
two operands and is defined subject to the same restrictions as '+'.

|    gap> [ 1, 2, 3 ] - [ 1/2, 1/3, 1/4 ];
    [ 1/2, 5/3, 11/4 ]
    gap> [ 1/2, 3/2, 1/2 ] - 1/2;
    [ 0, 1, 0 ] |

'<vec1> \*\ <vec2>'

In this form the multiplication operator '\*' evaluates to the product of
the two vectors <vec1> and <vec2>, which must have the same dimension and
lie  in a common field.   The product is  the sum of  the products of the
corresponding entries  of  the vectors.   As  an exception  it   is  also
possible to multiply an integer vector to a finite field vector, in which
case the integers are interpreted as '<scalar> \*\ <GF>.one'.

'<scalar> \*\ <vec>' \\
'<vec> \*\ <scalar>'

In this form '\*' evaluates to the product of the scalar <scalar> and the
vector <vec>, which must lie  in a common field.   The  product  is a new
vector  where   each entry  is   the  product  of   the  scalar  and  the
corresponding entry of the vector.  As an  exception it  is also possible
to multiply an integer scalar to a finite field vector, in which case the
integer is interpreted as '<scalar> \* <GF>.one'.

|    gap> [ 1, 2, 3 ] * [ 1/2, 1/3, 1/4 ];
    23/12
    gap> [ 1/2, 3/2, 1/2 ] * 2;
    [ 1, 3, 1 ] |

Further operations  with vectors  as operands  are defined  by the matrix
operations (see "Operations for Matrices").

%%%%%%%%%%%%%%%%%%%%%%%%%%%%%%%%%%%%%%%%%%%%%%%%%%%%%%%%%%%%%%%%%%%%%%%%%
\Section{IsVector}%
\index{test!for a vector}

'IsVector( <obj> )'

'IsVector' returns  'true' if <obj>, which may  be an object of arbitrary
type,  is a vector and 'false'  else.  A vector  is a list without holes,
whose elements all come from a common field.

|    gap> IsVector( [ 0, -3, -2, 0, 6 ] );
    true
    gap> IsVector( [ Z(3^4)^3, Z(3^2)^5, Z(3^4)^13 ] );
    true
    gap> IsVector( [ 0, Z(2^3)^3, Z(2^3) ] );
    false    # integers are not finite field elements
    gap> IsVector( [ , 2, 3,, 5,, 7 ] );
    false    # list that have holes are not vectors
    gap> IsVector( 0 );
    false    # not even a list |

%%%%%%%%%%%%%%%%%%%%%%%%%%%%%%%%%%%%%%%%%%%%%%%%%%%%%%%%%%%%%%%%%%%%%%%%%
\Section{NormedVector}%

'NormedVector( <vec> )'

'NormedVector' returns the scalar  multiple of <vec>  such that the first
nonzero entry of <vec> is the one from the field over which the vector is
defined.  If <vec> contains only zeroes a copy of it is returned.

|    gap> NormedVector( [ 0, -3, -2, 0, 6 ] );
    [ 0, 1, 2/3, 0, -2 ]
    gap> NormedVector( [ 0, 0 ] );
    [ 0, 0 ]
    gap> NormedVector( [ Z(3^4)^3, Z(3^2)^5, Z(3^4)^13 ] );
    [ Z(3)^0, Z(3^4)^47, Z(3^2) ] |

%%%%%%%%%%%%%%%%%%%%%%%%%%%%%%%%%%%%%%%%%%%%%%%%%%%%%%%%%%%%%%%%%%%%%%%%%
\Section{More about Vectors}

In  the  first section of this  chapter  we defined  a  vector  as a list
without  holes   whose  elements  all come  from  a   common field.  This
representation is quite nice to  use.  However, suppose that {\GAP} would
have to check that a list is a  vector every time  this vector appears as
operand in a addition or multiplication.  This would be quite wasteful.

To avoid this a list that is a vector may, but need not, have an internal
flag set that  tells the operations that  this  list is  indeed a vector.
Then this operations do not  have to check  this operand and can  perform
the operation  right away.  This section  tells you when a vector obtains
this flag, so  you can write your  functions in such  a way that you make
best use of this feature.

The results of  vector  operations, i.e., binary operations  that involve
vectors, are known by construction to be  vectors, and thus have the flag
set upon creation.

If the operand of one of the binary operation is a list that does not yet
have the   flag set, those operations will   check  that this  operand is
indeed a vector and set the flag if it is.  If it is not a vector and not
a matrix an error is signalled.

If the argument to 'IsVector' is a list that does not  yet have this flag
set, 'IsVector' will  test if all elements come  from a common field.  If
they do, 'IsVector'  will set the flag.   Thus on the one hand 'IsVector'
is a test whether the argument is a vector.  On the other hand 'IsVector'
can be used as a hint to {\GAP} that a certain list is indeed a vector.

If you  change  a vector, that does   have this flag set,  by assignment,
'Add', or  'Append', the vectors will loose  its flag, even if the change
is such that the resulting list is still a vector.  However if the vector
is a vector over a finite  field and you assign an  element from the same
finite field  the vector will keep its  flag.  Note that changing  a list
that is not a vector will never set the  flag, even if the resulting list
is  a vector.  Such a vector  will obtain the flag  only if it appears as
operand in a binary operation, or is passed to 'IsVector'.

Vectors over  finite fields  have  one additional feature.   If  they are
known  to be  vectors, not only do  they have  the flag set, but also are
they represented differently.  This representation is much  more compact.
Instead of storing every element separately and storing for every element
separately in which field it lies, the field is only  stored once.   This
representation takes up to 10 times less memory.

%%%%%%%%%%%%%%%%%%%%%%%%%%%%%%%%%%%%%%%%%%%%%%%%%%%%%%%%%%%%%%%%%%%%%%%%%
%%
%E  Emacs . . . . . . . . . . . . . . . . . . . . . local Emacs variables
%%
%%  Local Variables:
%%  mode:               outline
%%  outline-regexp:     "\\\\Chapter\\|\\\\Section"
%%  fill-column:        73
%%  eval:               (hide-body)
%%  End:
%%



