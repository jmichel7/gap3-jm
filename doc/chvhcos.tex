%%%%%%%%%%%%%%%%%%%%%%%%%%%%%%%%%%%%%%%%%%%%%%%%%%%%%%%%%%%%%%%%%%%%%%%%%%%%%
%%
%A  chvhcos.tex       CHEVIE documentation       Meinolf Geck, Frank Luebeck,
%A                                                Jean Michel, G"otz Pfeiffer
%%
%Y  Copyright (C) 1992 - 1996  Lehrstuhl D f\"ur Mathematik, RWTH Aachen, IWR
%Y  der Universit\"at Heidelberg, University of St. Andrews, and   University
%Y  Paris VII.
%%
%%  This  file  contains  the  description  of  the  GAP functions of CHEVIE
%%  dealing with Hecke cosets.
%%%%%%%%%%%%%%%%%%%%%%%%%%%%%%%%%%%%%%%%%%%%%%%%%%%%%%%%%%%%%%%%%%%%%%%%%

\Chapter{Hecke cosets}

``Hecke  cosets\"\ are  $H\phi$  where $H$  is a   Hecke algebra  of some
Coxeter   group $W$  on   which   the  reduced  element $\phi$   acts  by
$\phi(T_w)=T_{\phi(w)}$.  This corresponds to the action of the Frobenius
automorphism on  the commuting  algebra  of  the induced of  the  trivial
representation from the rational points of some $F$-stable Borel subgroup
to ${\bf G}^F$.

|    gap> W := CoxeterGroup( "A", 2 );;
    gap> q := X( Rationals );; q.name := "q";;
    gap> HF := Hecke( CoxeterCoset( W, (1,2) ), q^2, q );
    Hecke(2A2,q^2,q)
    gap> Display( CharTable( HF ) );
    H(2A2)
    
         2     1   1   .
         3     1   .   1
    
             111  21   3
        2P   111 111   3
        3P   111  21 111
    
    111       -1   1  -1
    21     -2q^3   .   q
    3        q^6   1 q^2
    |

Thanks  to the work of Xuhua  He and Sian Nie, 'HeckeClassPolynomials' also
make sense for these cosets. This is used to compute such character tables.

%%%%%%%%%%%%%%%%%%%%%%%%%%%%%%%%%%%%%%%%%%%%%%%%%%%%%%%%%%%%%%%%%%%%%%%%%
\Section{Hecke for Coxeter cosets}
\index{Hecke}

'Hecke( <WF>, <H> )'

'Hecke( <WF>, <params> )'

Construct  a  Hecke coset  a Coxeter   coset  <WF> and  an Hecke  algebra
associated to the CoxeterGroup of <WF>. The  second form is equivalent to
'Hecke( <WF>, Hecke(CoxeterGroup(<WF>), <params>))'.

%%%%%%%%%%%%%%%%%%%%%%%%%%%%%%%%%%%%%%%%%%%%%%%%%%%%%%%%%%%%%%%%%%%%%

\Section{Operations and functions for Hecke cosets}

'Hecke':\\ returns the untwisted Hecke algebra corresponding to the Hecke
     coset.

\index{CoxeterCoset}
'CoxeterCoset':\\  returns the Coxeter coset   corresponding to the Hecke
     coset.

\index{CoxeterGroup}
'CoxeterGroup':\\ returns the  untwisted  Coxeter group corresponding  to
     the Hecke coset.

'Print':\\ prints the Hecke  coset in a form  which can be read back into
     \GAP.

\index{CharTable}
'CharTable':\\ returns the character table of the Hecke coset.

\index{Basis}
|Basis(H,"T")|:\\ the 'T' basis. 

%%%%%%%%%%%%%%%%%%%%%%%%%%%%%%%%%%%%%%%%%%%%%%%%%%%%%%%%%%%%%%%%%%%%%%%%%
