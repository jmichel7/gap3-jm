%%%%%%%%%%%%%%%%%%%%%%%%%%%%%%%%%%%%%%%%%%%%%%%%%%%%%%%%%%%%%%%%%%%%%%%%%%%%%
%%
%A  algmat.tex                  GAP documentation               Thomas Breuer
%%
%A  @(#)$Id: algmat.tex,v 1.1.1.1 1996/12/11 12:36:42 werner Exp $
%%
%Y  Copyright 1994-1995,  Lehrstuhl D fuer Mathematik,  RWTH Aachen,  Germany
%%
%%  This file contains the description of matrix algebras.
%%
%H  $Log: algmat.tex,v $
%H  Revision 1.1.1.1  1996/12/11 12:36:42  werner
%H  Preparing 3.4.4 for release
%H
%H  Revision 3.0  1994/05/19  14:00:16  sam
%H  Initial Revision under RCS
%H
%%
\Chapter{Matrix Algebras}

This chapter describes the data structures and functions for matrix
algebras in {\GAP}.  See chapter "Algebras" for the description of all
those aspects that concern general algebras.

First the objects of interest in this chapter are introduced (see "More
about Matrix Algebras", "Bases for Matrix Algebras").

The next sections describe functions for matrix algebras, first those
that can be applied not only for matrix algebras (see "IsMatAlgebra",
"Zero and One for Matrix Algebras", "Functions for Matrix Algebras", 
"Algebra Functions for Matrix Algebras", "RepresentativeOperation for
Matrix Algebras"), and then specific matrix algebra functions (see
"MatAlgebra", "NullAlgebra", "Fingerprint", "NaturalModule").

%%%%%%%%%%%%%%%%%%%%%%%%%%%%%%%%%%%%%%%%%%%%%%%%%%%%%%%%%%%%%%%%%%%%%%%%%
\Section{More about Matrix Algebras}

A *matrix algebra* is an algebra (see "More about Algebras") the elements
of which are matrices.

There is a canonical isomorphism of a matrix algebra onto a row space
(see chapter "Row Spaces") that maps a matrix to the concatenation of its
rows.  This makes all computations with matrix algebras that use its
vector space structure as efficient as the corresponding computation with
a row space.  For example the computation of a vector space basis, of
coefficients with respect to such a basis, and of representatives under
the action on a vector space by right multiplication.

If one is interested in matrix algebras as domains themselves then one
should think of this algebra as of a row space that admits a
multiplication.  For example, the convention for row spaces that the
coefficients field must contain the field of the vector elements also
applies to matrix algebras.  And the concept of vector space bases is
the same as that for row spaces (see "Bases for Matrix Algebras").

In the chapter about modules (see chapter "Modules") it is stated that
modules are of interest mainly as operation domains of algebras.  Here
we can state that matrix algebras are of interest mainly because they
describe modules.  For some of the functions it is not obvious whether
they are functions for modules or for algebras or for the matrices that
generate an algebra.  For example, one usually talks about the
fingerprint of an $A$-module $M$, but this is in fact computed as the
list of nullspace dimensions of generators of a certain matrix algebra,
namely the induced action of $A$ on $M$ as is computed using
'Operation( <A>, <M> )' (see "Fingerprint", "Operation for Algebras").

%%%%%%%%%%%%%%%%%%%%%%%%%%%%%%%%%%%%%%%%%%%%%%%%%%%%%%%%%%%%%%%%%%%%%%%%%
\Section{Bases for Matrix Algebras}

As stated in section "More about Matrix Algebras", the implementation of
bases for matrix algebras follows that of row space bases, see "Row Space
Bases" for the details.  Consequently there are two types of bases,
arbitrary bases and semi-echelonized bases, where the latter type can be
defined as follows.  Let $\varphi$ be the vector space homomorphism that
maps a matrix in the algebra $A$ to the concatenation of its rows, and
let $B = (b_1, b_2, \ldots, b_n)$ be a vector space basis of $A$, then
$B$ is called *semi-echelonized* if and only if the row space basis
$(\varphi(b_1), \varphi(b_2), \ldots, \varphi(b_n))$ is semi-echelonized,
in the sense of "Row Space Bases".  The *canonical basis* is defined
analogeously.

Due to the multiplicative structure that allows to view a matrix algebra
$A$ as an $A$-module with action via multiplication from the right,
there is additionally the notion of a *standard basis* for $A$, which is
essentially described in "StandardBasis for Row Modules".  The default way
to compute a vector space basis of a matrix algebra from a set of
generating matrices is to compute this standard basis and a
semi-echelonized basis in parallel.

If the matrix algebra $A$ is unital then every semi-echelonized basis and
also the standard basis have 'One( <A> )' as first basis vector.

%%%%%%%%%%%%%%%%%%%%%%%%%%%%%%%%%%%%%%%%%%%%%%%%%%%%%%%%%%%%%%%%%%%%%%%%%
\Section{IsMatAlgebra}

'IsMatAlgebra( <obj> )'

returns 'true' if <obj>, which may be an object of arbitrary type, is a
matrix algebra and 'false' otherwise.

|    gap> IsMatAlgebra( FreeAlgebra( GF(2), 0 ) );
    false
    gap> IsMatAlgebra( Algebra( Rationals, [[[1]]] ) );
    true |

%%%%%%%%%%%%%%%%%%%%%%%%%%%%%%%%%%%%%%%%%%%%%%%%%%%%%%%%%%%%%%%%%%%%%%%%%
\Section{Zero and One for Matrix Algebras}

'Zero( <A> )' : \\
    returns the square zero matrix of the same dimension and
    characteristic as the elements of <A>.  This matrix is thought only
    for testing whether a matrix is zero, usually all its rows will be
    *identical* in order to save space.  So you should *not* use this
    zero matrix for other purposes; use "NullMat" 'NullMat' instead.

'One( <A> )' : \\
    returns for a unital matrix algebra <A> the identity matrix of the
    same dimension and characteristic as the elements of <A>; for a not
    unital matrix algebra <A> the (left and right) multiplicative
    neutral element (if exists) is computed by solving a linear equation
    system.

%%%%%%%%%%%%%%%%%%%%%%%%%%%%%%%%%%%%%%%%%%%%%%%%%%%%%%%%%%%%%%%%%%%%%%%%%
\Section{Functions for Matrix Algebras}

'Closure', 'Elements', 'IsFinite', and 'Size' are the only *set theoretic
functions* that are overlaid in the operations records for matrix
algebras and unital matrix algebras.
See "Set Theoretic Functions for Algebras" for an overview of set
theoretic functions for general algebras.

No *vector space functions* are overlaid in the operations records for
matrix algebras and unital matrix algebras.
The *functions for vector space bases* are mainly the same as those for
row space bases (see "Bases for Matrix Algebras").

For other functions for matrix algebras, see "Algebra Functions for
Matrix Algebras".

%%%%%%%%%%%%%%%%%%%%%%%%%%%%%%%%%%%%%%%%%%%%%%%%%%%%%%%%%%%%%%%%%%%%%%%%%
\Section{Algebra Functions for Matrix Algebras}

'Centralizer( <A>, <a> )' \\
'Centralizer( <A>, <S> )' : \\
    returns the element or subalgebra centralizer in the matrix algebra
    <A>.  Centralizers in matrix algebras are computed by solving a
    linear equation system.

'Centre( <A> )' : \\
    returns the centre of the matrix algebra <A>, which is computed by
    solving a linear equation system.

'FpAlgebra( <A> )' : \\
    returns a finitely presented algebra that is isomorphic to <A>.
    The presentation is computed using the structure constants, thus
    a vector space basis of <A> has to be computed.
    If <A> contains no multiplicative neutral element (see "Zero and One
    for Matrix Algebras") an error is signalled.
    (At the moment the implementation is really simpleminded.)

|    gap> a:= UnitalAlgebra( Rationals, [[[0,1],[0,0]]] );
    UnitalAlgebra( Rationals, [ [ [ 0, 1 ], [ 0, 0 ] ] ] )
    gap> FpAlgebra( a );
    UnitalAlgebra( Rationals, [ a.1 ] )
    gap> last.relators;
    [ a.1^2 ] |

%%%%%%%%%%%%%%%%%%%%%%%%%%%%%%%%%%%%%%%%%%%%%%%%%%%%%%%%%%%%%%%%%%%%%%%%%
\Section{RepresentativeOperation for Matrix Algebras}

'RepresentativeOperation( <A>, <v1>, <v2> )'

returns the element in the matrix algebra <A> that maps <v1> to <v2>
via right multiplication if such an element exists, and 'false'
otherwise.  <v1> and <v2> may be vectors or matrices of same dimension.

|    gap> a:= MatAlgebra( GF(2), 2 );
    UnitalAlgebra( GF(2), [ [ [ Z(2)^0, 0*Z(2) ], [ 0*Z(2), 0*Z(2) ] ], 
      [ [ 0*Z(2), Z(2)^0 ], [ Z(2)^0, 0*Z(2) ] ] ] )
    gap> v1:= [ 1, 0 ] * Z(2);; v2:= [ 1, 1 ] * Z(2);;
    gap> RepresentativeOperation( a, v1, v2 );
    [ [ Z(2)^0, Z(2)^0 ], [ Z(2)^0, Z(2)^0 ] ]
    gap> t:= TrivialSubalgebra( a );;
    gap> RepresentativeOperation( t, v1, v2 );
    false |

%%%%%%%%%%%%%%%%%%%%%%%%%%%%%%%%%%%%%%%%%%%%%%%%%%%%%%%%%%%%%%%%%%%%%%%%%
\Section{MatAlgebra}

'MatAlgebra( <F>, <n> )'

returns the full matrix algebra of <n> by <n> matrices over the field <F>.

|    gap> a:= MatAlgebra( GF(2), 2 );
    UnitalAlgebra( GF(2), [ [ [ Z(2)^0, 0*Z(2) ], [ 0*Z(2), 0*Z(2) ] ], 
      [ [ 0*Z(2), Z(2)^0 ], [ Z(2)^0, 0*Z(2) ] ] ] )
    gap> Size( a );
    16 |

%%%%%%%%%%%%%%%%%%%%%%%%%%%%%%%%%%%%%%%%%%%%%%%%%%%%%%%%%%%%%%%%%%%%%%%%%
\Section{NullAlgebra}

'NullAlgebra( <F> )'

returns a trivial algebra (that is, it contains only the zero element)
over the field <F>.
This occurs in a natural way whenever 'Operation' (see "Operation for
Algebras") constructs a faithful representation of the zero module.

Here we meet the strange situation that an operation algebra does not
consist of matrices, since in {\GAP} a matrix always has a positive
number of rows and columns.  The element of a 'NullAlgebra( <F> )' is
the object 'EmptyMat' that acts (trivially) on empty lists via right
multiplication.

|    gap> a:= NullAlgebra( GF(2) );
    NullAlgebra( GF(2) )
    gap> Size( a );
    1
    gap> Elements( a );
    [ EmptyMat ]
    gap> [] * EmptyMat;
    [  ]
    gap> IsAlgebra( a );
    true |

%%%%%%%%%%%%%%%%%%%%%%%%%%%%%%%%%%%%%%%%%%%%%%%%%%%%%%%%%%%%%%%%%%%%%%%%%%%%%
\Section{Fingerprint}

'Fingerprint( <A> )'\\
'Fingerprint( <A>, <list> )'

returns the fingerprint of the matrix algebra <A>, i.e., a list of
nullities of six ``standard\'\'\ words in <A> (for 2-generator
algebras only) or of the words with numbers in <list>.

|    gap> m1:= PermutationMat( (1,2,3,4,5), 5, GF(2) );;
    gap> m2:= PermutationMat( (1,2)      , 5, GF(2) );;
    gap> a:= Algebra( GF(2), [ m1, m2 ] );;
    gap> Fingerprint( a );
    [ 1, 1, 1, 3, 0, 4 ] |

Let $a$ and $b$ be the generators of a 2-generator matix algebra.
The six standard words used by 'Fingerprint' are $w_1, w_2, \ldots, w_6$
where
\[ \begin{array}{llllll} w_1 & = & a b + a + b, &
                         w_2 & = & w_1 + a b^2, \\
                         w_3 & = & a + b w_2,   &
                         w_4 & = & b + w_3,     \\
                         w_5 & = & a b + w_4,   &
                         w_6 & = & a + w_5     \end{array} \]

%%%%%%%%%%%%%%%%%%%%%%%%%%%%%%%%%%%%%%%%%%%%%%%%%%%%%%%%%%%%%%%%%%%%%%%%%%%%%
\Section{NaturalModule}

'NaturalModule( <A> )'

returns the *natural module* $M$ of the matrix algebra <A>.
If <A> consists of $n$ by $n$ matrices, and $F$ is the coefficients field of
<A> then $M$ is an $n$-dimensional row space over the field $F$, viewed as
<A>-right module (see "Module").

|    gap> a:= MatAlgebra( GF(2), 2 );;
    gap> a.name:= "a";;
    gap> m:= NaturalModule( a );
    Module( a, [ [ Z(2)^0, 0*Z(2) ], [ 0*Z(2), Z(2)^0 ] ] ) |

%%%%%%%%%%%%%%%%%%%%%%%%%%%%%%%%%%%%%%%%%%%%%%%%%%%%%%%%%%%%%%%%%%%%%%%%%%%%%
%%
%E  Emacs . . . . . . . . . . . . . . . . . . . . . . . local emacs variables
%%
%%  Local Variables:
%%  mode:               outline
%%  outline-regexp:     "%F\\|%V\\|%E"
%%  fill-column:        73
%%  fill-prefix:        "%%  "
%%  eval:               (hide-body)
%%  End:
