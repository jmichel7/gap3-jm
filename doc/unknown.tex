%%%%%%%%%%%%%%%%%%%%%%%%%%%%%%%%%%%%%%%%%%%%%%%%%%%%%%%%%%%%%%%%%%%%%%%%%%%%%
%%
%A  unknown.tex                 GAP documentation               Thomas Breuer
%%
%A  @(#)$Id: unknown.tex,v 1.1.1.1 1996/12/11 12:36:51 werner Exp $
%%
%Y  Copyright 1990-1992,  Lehrstuhl D fuer Mathematik,  RWTH Aachen,  Germany
%%
%H  $Log: unknown.tex,v $
%H  Revision 1.1.1.1  1996/12/11 12:36:51  werner
%H  Preparing 3.4.4 for release
%H
%H  Revision 3.7  1993/02/18  13:40:55  felsch
%H  more examples fixed
%H
%H  Revision 3.6  1993/02/12  15:50:48  felsch
%H  examples adjusted to line length 72
%H
%H  Revision 3.5  1993/02/11  17:46:09  martin
%H  changed '@' to '&'
%H
%H  Revision 3.4  1993/02/07  13:23:35  felsch
%H  more examples fixed
%H
%H  Revision 3.3  1993/02/01  13:58:52  felsch
%H  examples fixed
%H
%H  Revision 3.2  1992/11/30  15:36:30  fceller
%H  changed text aligment so that the online help works
%H
%H  Revision 3.1  1991/12/27  16:46:58  sam
%H  initial revision under RCS
%%
\Chapter{Unknowns}%
\index{data type!unknown}

Sometimes  the  result  of  an   operation  does  not  allow   further
computations with it.  In many cases, then an error is signalled,  and
the computation is stopped.

This is not  appropriate  for some applications  in  character theory.
For example,  if a  character shall be induced  up (see "Induced") but
the subgroup fusion is only a parametrized map  (see chapter "Maps and
Parametrized  Maps"), there  are  positions where  the  value  of  the
induced character are not known, and other values which are determined
by the fusion map\:

|    gap> m11:= CharTable( "M11" );; m12:= CharTable( "M12" );;
    gap> fus:= InitFusion( m11, m12 );
    [ 1, [ 2, 3 ], [ 4, 5 ], [ 6, 7 ], 8, [ 9, 10 ], [ 11, 12 ],
      [ 11, 12 ], [ 14, 15 ], [ 14, 15 ] ]
    gap> Induced(m11,m12,Sublist(m11.irreducibles,[ 6 .. 9 ]),fus);
    &I Induced: subgroup order not dividing sum in character 1 at class 4
    &I Induced: subgroup order not dividing sum in character 1 at class 5
    &I Induced: subgroup order not dividing sum in character 1 at class 14
    &I Induced: subgroup order not dividing sum in character 1 at class 15
    &I Induced: subgroup order not dividing sum in character 2 at class 4
    &I Induced: subgroup order not dividing sum in character 2 at class 5
    &I Induced: subgroup order not dividing sum in character 2 at class 14
    &I Induced: subgroup order not dividing sum in character 2 at class 15
    &I Induced: subgroup order not dividing sum in character 3 at class 2
    &I Induced: subgroup order not dividing sum in character 3 at class 3
    &I Induced: subgroup order not dividing sum in character 3 at class 4
    &I Induced: subgroup order not dividing sum in character 3 at class 5
    &I Induced: subgroup order not dividing sum in character 3 at class 9
    &I Induced: subgroup order not dividing sum in character 3 at class 10
    &I Induced: subgroup order not dividing sum in character 4 at class 2
    &I Induced: subgroup order not dividing sum in character 4 at class 3
    &I Induced: subgroup order not dividing sum in character 4 at class 6
    &I Induced: subgroup order not dividing sum in character 4 at class 7
    &I Induced: subgroup order not dividing sum in character 4 at class 11
    &I Induced: subgroup order not dividing sum in character 4 at class 12
    &I Induced: subgroup order not dividing sum in character 4 at class 14
    &I Induced: subgroup order not dividing sum in character 4 at class 15
    [ [ 192, 0, 0, Unknown(9), Unknown(12), 0, 0, 2, 0, 0, 0, 0, 0,
          Unknown(15), Unknown(18) ], 
      [ 192, 0, 0, Unknown(27), Unknown(30), 0, 0, 2, 0, 0, 0, 0, 0,
          Unknown(33), Unknown(36) ], 
      [ 528, Unknown(45), Unknown(48), Unknown(51), Unknown(54), 0, 0, 
          -2, Unknown(57), Unknown(60), 0, 0, 0, 0, 0 ], 
      [ 540, Unknown(75), Unknown(78), 0, 0, Unknown(81), Unknown(84), 0, 
          0, 0, Unknown(87), Unknown(90), 0, Unknown(93), Unknown(96) ] ]|

For this  and other  situations,  in  \GAP\ there  is  the  data  type
*unknown*.   Objects  of this type, further on  called *unknowns*, may
stand for any cyclotomic (see "Cyclotomics").

Unknowns are  parametrized by positive integers.  When a \GAP\ session
is started, no unknowns do exist.

The only ways to create unknowns are to call  "Unknown" 'Unknown' or a
function that calls it, or to do arithmetical operations with unknowns
(see "Operations for Unknowns").

Two properties should be noted\:

Lists  of cyclotomics and unknowns are  no vectors, so cannot be added
or multiplied like vectors; as a  consequence, unknowns never occur in
matrices.

\GAP\ objects  which are printed to files will contain fixed unknowns,
i.e., function calls 'Unknown(  <n> )'  instead of 'Unknown()',  so be
careful to read files printed in different  sessions, since  there may
be the same unknown at different places.

The rest  of this  chapter  contains  informations  about the  unknown
constructor  (see   "Unknown"),  the  characteristic   function   (see
"IsUnknown"),  and  comparison  of  and  arithmetical  operations  for
unknowns  (see "Comparisons  of Unknowns", "Operations for Unknowns");
more is not yet known about unknowns.

%%%%%%%%%%%%%%%%%%%%%%%%%%%%%%%%%%%%%%%%%%%%%%%%%%%%%%%%%%%%%%%%%%%%%%%%%%%%%
\Section{Unknown}

'Unknown()'\\
'Unknown( <n> )'

'Unknown()'  returns  a new unknown value, i.e. the first  one that is
larger than all unknowns which exist in the actual \GAP\ session.

'Unknown(  <n>  )' returns the  <n>-th  unknown;  if it did not  exist
already, it is created.

|    gap> Unknown(); Unknown(2000); Unknown();
    Unknown(97)     # There were created already 96 unknowns.
    Unknown(2000)
    Unknown(2001)|

%%%%%%%%%%%%%%%%%%%%%%%%%%%%%%%%%%%%%%%%%%%%%%%%%%%%%%%%%%%%%%%%%%%%%%%%%%%%%
\Section{IsUnknown}

'IsUnknown( <obj> )'

returns  'true' if  <obj>  is  an  object of type unknown, and 'false'
otherwise.  Will signal an error if <obj> is an unbound variable.

|    gap> IsUnknown( Unknown ); IsUnknown( Unknown() );
    false
    true
    gap> IsUnknown( Unknown(2) );
    true|

%%%%%%%%%%%%%%%%%%%%%%%%%%%%%%%%%%%%%%%%%%%%%%%%%%%%%%%%%%%%%%%%%%%%%%%%%%%%%
\Section{Comparisons of Unknowns}

To compare  unknowns with  other objects,  the operators  '\<', '\<=',
'=', '>=',  '>' and '\<>' can be used.  The  result will  be 'true' if
the first  operand is  smaller,  smaller or equal,  equal,  larger  or
equal, larger, or inequal, respectively, and 'false' otherwise.

We have 'Unknown( <n> ) >= Unknown( <m> )' if and only if '<n> >= <m>'
holds; unknowns are larger than cyclotomics and finite field elements,
unknowns  are  smaller than  all  objects  which are not  cyclotomics,
finite field elements or unknowns.

|    gap> Unknown() >= Unknown();
    false
    gap> Unknown(2) < Unknown(3);
    true
    gap> Unknown() > 3;
    true
    gap> Unknown() > Z(8);
    false
    gap> Unknown() > E(3);
    true
    gap>  Unknown() > [];
    false|

%%%%%%%%%%%%%%%%%%%%%%%%%%%%%%%%%%%%%%%%%%%%%%%%%%%%%%%%%%%%%%%%%%%%%%%%%%%%%
\Section{Operations for Unknowns}

The  operators  '+',  '-',  '\*'  and   '/'  are  used  for  addition,
subtraction, multiplication and division of  unknowns and cyclotomics.
The  result  will be  a  new unknown  except  in one of  the following
cases\:

Multiplication  with zero yields zero, and multiplication  with one or
addition of zero yields the old unknown.

|    gap> Unknown() + 1; Unknown(2) + 0; last * 3; last * 1; last * 0;
    Unknown(2010)
    Unknown(2)
    Unknown(2011)
    Unknown(2011)
    0|

*Note* that division by an unknown causes  an error, since an  unknown
might stand for zero.


%%%%%%%%%%%%%%%%%%%%%%%%%%%%%%%%%%%%%%%%%%%%%%%%%%%%%%%%%%%%%%%%%%%%%%%%%%%%%
%E  Emacs . . . . . . . . . . . . . . . . . . . . . . . local Emacs variables
%%
%%  Local Variables:
%%  mode:               outline
%%  outline-regexp:     "\\\\Chapter\\|\\\\Section\\|%E"
%%  fill-column:        73
%%  eval:               (hide-body)
%%  End:
%%
