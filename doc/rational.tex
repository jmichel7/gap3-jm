%%%%%%%%%%%%%%%%%%%%%%%%%%%%%%%%%%%%%%%%%%%%%%%%%%%%%%%%%%%%%%%%%%%%%%%%%%%%%
%%
%A  rational.tex                GAP documentation            Martin Schoenert
%%
%A  @(#)$Id: rational.tex,v 1.1.1.1 1996/12/11 12:36:49 werner Exp $
%%
%Y  Copyright 1990-1992,  Lehrstuhl D fuer Mathematik,  RWTH Aachen,  Germany
%%
%%  This file describes the rational datatype,  its operations and functions.
%%
%H  $Log: rational.tex,v $
%H  Revision 1.1.1.1  1996/12/11 12:36:49  werner
%H  Preparing 3.4.4 for release
%H
%H  Revision 3.6  1993/02/19  10:48:42  gap
%H  adjustments in line length and spelling
%H
%H  Revision 3.5  1992/04/06  16:40:34  martin
%H  fixed some more typos
%H
%H  Revision 3.4  1992/04/02  21:06:23  martin
%H  changed *domain functions* to *set theoretic functions*
%H
%H  Revision 3.3  1992/03/19  15:57:34  martin
%H  renamed chapter "Number Fields" to "Subfields of Cyclotomic Fields"
%H
%H  Revision 3.2  1991/12/27  16:07:04  martin
%H  revised everything for GAP 3.1 manual
%H
%H  Revision 3.1  1991/07/25  16:16:59  martin
%H  fixed some minor typos
%H
%H  Revision 3.0  1991/04/11  11:33:12  martin
%H  Initial revision under RCS
%H
%%
\Chapter{Rationals}%
\index{type!rationals}

The *rationals*  form  a very important field.  On the one hand it is the
quotient field of the integers (see "Integers").  On the other hand it is
the prime field of the fields of characteristic zero  (see "Subfields  of
Cyclotomic Fields").

The former comment suggests the representation actually used.  A rational
is  represented   as   a   pair   of  integers,  called  *numerator*  and
*denominator*.   Numerator  and  denominator are  *reduced*, i.e.,  their
greatest common divisor is 1.  If  the denominator  is 1, the rational is
in fact an integer and is represented as such.   The numerator holds  the
sign of the rational, thus the denominator is always positive.

Because the underlying integer arithmetic can compute with arbitrary size
integers, the  rational arithmetic is  always  exact, even  for rationals
whose numerators and denominators have thousands of digits.

|    gap> 2/3;
    2/3
    gap> 66/123;
    22/41    # numerator and denominator are made relatively prime
    gap> 17/-13;
    -17/13    # the numerator carries the sign
    gap> 121/11;
    11    # rationals with denominator 1 (after cancelling) are integers|

The first sections of  this chapter  describe  the  functions  that  test
whether an object  is a rational (see "IsRat"), and select the  numerator
and denominator of a rational (see "Numerator", "Denominator").

The next sections describe  the rational operations (see "Comparisons  of
Rationals", and "Operations for Rationals").

The {\GAP} object 'Rationals' is the field domain  of all rationals.  All
set  theoretic  functions  are applicable  to  this  domain (see  chapter
"Domains"  and "Set  Functions for  Rationals").  Since 'Rationals' is  a
field all  field  functions are also  applicable  to this domain  and its
elements (see chapter "Fields" and "Field Functions for Rationals").

All external functions are defined in the file '\"LIBNAME/rational.g\"'.

%%%%%%%%%%%%%%%%%%%%%%%%%%%%%%%%%%%%%%%%%%%%%%%%%%%%%%%%%%%%%%%%%%%%%%%%%
\Section{IsRat}%
\index{test!for a rational}

'IsRat( <obj> )'

'IsRat' returns 'true' if <obj>,  which can be  an arbitrary object, is a
rational and 'false' otherwise.   Integers are rationals with denominator
1,   thus 'IsRat' returns  'true' for  integers.   'IsRat' will signal an
error if <obj> is an unbound variable or a procedure call.

|    gap> IsRat( 2/3 );
    true
    gap> IsRat( 17/-13 );
    true
    gap> IsRat( 11 );
    true
    gap> IsRat( IsRat );
    false    # 'IsRat' is a function, not a rational |

%%%%%%%%%%%%%%%%%%%%%%%%%%%%%%%%%%%%%%%%%%%%%%%%%%%%%%%%%%%%%%%%%%%%%%%%%
\Section{Numerator}%
\index{numerator!of a rational}

'Numerator( <rat> )'

'Numerator'  returns  the numerator of the rational <rat>.   Because  the
numerator holds the sign of the rational it may be any integer.  Integers
are rationals  with  denominator  1, thus  'Numerator'  is  the  identity
function for integers.

|    gap> Numerator( 2/3 );
    2
    gap> Numerator( 66/123 );
    22    # numerator and denominator are made relatively prime
    gap> Numerator( 17/-13 );
    -17    # the numerator holds the sign of the rational
    gap> Numerator( 11 );
    11    # integers are rationals with denominator 1 |

'Denominator' (see "Denominator") is the counterpart to 'Numerator'.

%%%%%%%%%%%%%%%%%%%%%%%%%%%%%%%%%%%%%%%%%%%%%%%%%%%%%%%%%%%%%%%%%%%%%%%%%
\Section{Denominator}%
\index{denominator!of a rational}

'Denominator( <rat> )'

'Denominator' returns the denominator of the rational <rat>.  Because the
numerator  holds the  sign of  the rational  the denominator  is always a
positive integer.   Integers are rationals  with  the denominator 1, thus
'Denominator' returns 1 for integers.

|    gap> Denominator( 2/3 );
    3
    gap> Denominator( 66/123 );
    41    # numerator and denominator are made relatively prime
    gap> Denominator( 17/-13 );
    13    # the denominator holds the sign of the rational
    gap> Denominator( 11 );
    1    # integers are rationals with denominator 1 |

'Numerator' (see "Numerator") is the counterpart to 'Denominator'.

%%%%%%%%%%%%%%%%%%%%%%%%%%%%%%%%%%%%%%%%%%%%%%%%%%%%%%%%%%%%%%%%%%%%%%%%%
\Section{Floor}
\index{Floor}

'Floor(<r>)'

This function returns the largest integer smaller or equal to <r>.

|    gap> Floor(-2/3);
    -1
    gap> Floor(2/3);
    0|

%%%%%%%%%%%%%%%%%%%%%%%%%%%%%%%%%%%%%%%%%%%%%%%%%%%%%%%%%%%%%%%%%%%%%%%%%
\Section{Mod1}
\index{Mod1}

'Mod1(<r>)'

The  argument should  be a  rational or  a list.  If <r>  is a rational, it
returns '(Numerator(<r>) mod Denominator(<r>))/Denominator(r)'. If <r> is a
list, it returns 'List(<r>,Mod1)'. This function is very useful for working
in $\Q/\Z$.

|    gap> Mod1([-2/3,-1,7/4,3]);
    [ 1/3, 0, 3/4, 0 ]|

%%%%%%%%%%%%%%%%%%%%%%%%%%%%%%%%%%%%%%%%%%%%%%%%%%%%%%%%%%%%%%%%%%%%%%%%%
\Section{Comparisons of Rationals}%
\index{comparisons!of rationals}

'<q1> = <q2>' \\
'<q1> \<> <q2>'

The  equality operator '=' evaluates to  'true' if the two rationals <q1>
and <q2> are  equal  and to 'false'  otherwise.  The  inequality operator
'\<>' evaluates to   'true' if the  two rationals  <q1> and  <q2> are not
equal and to 'false' otherwise.

|    gap> 2/3 = -4/-6;
    true
    gap> 66/123 <> 22/41;
    false
    gap> 17/13 = 11;
    false |

'<q1> \<\ <q2>' \\
'<q1> \<= <q2>' \\
'<q1> > <q2>'   \\
'<q1> >= <q2>'

The operators  '\<',  '\<=', '>',  and  '=>' evaluate  to  'true' if  the
rational  <q1> is  less  than, less than or equal to,  greater than,  and
greater than or equal to the rational <q2> and to 'false' otherwise.

One rational $q_1 = n_1/d_1$ is less than  another $q_2 = n_2/d_2$ if and
only if  $n_1 d_2 \< n_2 d_2$.  This  definition is of  course only valid
because the  denominator of rationals  is always  defined to be positive.
This   definition   also extends   to the comparison   of  rationals with
integers, which are   interpreted   as  rationals with    denominator  1.
Rationals can also  be compared  with objects of  other  types.  They are
smaller than objects of any other type by definition.

|    gap> 2/3 < 22/41;
    false
    gap> -17/13 < 11;
    true |

%%%%%%%%%%%%%%%%%%%%%%%%%%%%%%%%%%%%%%%%%%%%%%%%%%%%%%%%%%%%%%%%%%%%%%%%%
\Section{Operations for Rationals}%
\index{operations!for rationals}

'<q1> + <q2>'   \\
'<q1> - <q2>'   \\
'<q1> \*\ <q2>' \\
'<q1> / <q2>'

The operators '+',  '-',  '\*' and '/' evaluate to the  sum,  difference,
product, and  quotient  of  the  two  rationals  <q1>  and <q2>.  For the
quotient  '/'  <q2>  must of course be  nonzero, otherwise  an  error  is
signalled.   Either  operand  may  also  be  an  integer  <i>,  which  is
interpreted  as a  rational  with  denominator  1.   The result of  those
operations is always reduced.  If, after  the  reduction, the denominator
is 1, the rational is in fact an integer, and is represented as such.

|    gap> 2/3 + 4/5;
    22/15
    gap> 7/6 * 2/3;
    7/9    # note how the result is cancelled
    gap> 67/6 - 1/6;
    11    # the result is an integer |

'<q> \^\ <i>'

The powering operator '\^' returns the <i>-th power  of the rational <q>.
<i> must  be an integer.   If the  exponent <i>  is  zero, '<q>\^<i>'  is
defined as 1; if <i>  is positive, '<q>\^<i>' is  defined as the <i>-fold
product '<q>\*<q>\*..\*<q>'; finally, if  <i> is negative, '<q>\^<i>'  is
defined as '(1/<q>)\^-<i>'.  In this case <q> must of course be nonzero.

|    gap> (2/3) ^ 3;
    8/27
    gap> (-17/13) ^ -1;
    -13/17    # note how the sign switched
    gap> (1/2) ^ -2;
    4 |

%%%%%%%%%%%%%%%%%%%%%%%%%%%%%%%%%%%%%%%%%%%%%%%%%%%%%%%%%%%%%%%%%%%%%%%%%
\Section{Set Functions for Rationals}%

As  was already mentioned in the introduction of this  chapter the {\GAP}
object 'Rationals' is the  domain of all  rationals.  All  set  theoretic
functions,  e.g.,  'Intersection'  and  'Size',  are  applicable to  this
domain.

|    gap> Intersection( Rationals, [ E(4)^0, E(4)^1, E(4)^2, E(4)^3 ] );
    [ -1, 1 ]    # 'E(4)' is the complex square root of -1
    gap> Size( Rationals );
    "infinity" |

This does not seem to be very useful.

%%%%%%%%%%%%%%%%%%%%%%%%%%%%%%%%%%%%%%%%%%%%%%%%%%%%%%%%%%%%%%%%%%%%%%%%%
\Section{Field Functions for Rationals}%

As was already mentioned  in the introduction  of this chapter the {\GAP}
object 'Rationals'  is the field of all  rationals.  All field functions,
e.g., 'Norm' and 'MinPol' are applicable to this domain and its elements.
However, since  the field of  rationals  is the  prime  field, all  those
functions are trivial.  Therefore, 'Conjugates( Rationals, <q> )' returns
'[ <q> ]', 'Norm( Rationals, <q> )' and 'Trace(  Rationals, <q> )' return
<q>, and 'CharPol( Rationals, <q> )' and  'MinPol( Rationals, <q> )' both
return '[ -<q>, 1 ]'.

%%%%%%%%%%%%%%%%%%%%%%%%%%%%%%%%%%%%%%%%%%%%%%%%%%%%%%%%%%%%%%%%%%%%%%%%%
%%
%E  Emacs . . . . . . . . . . . . . . . . . . . . . local Emacs variables
%%
%%  Local Variables:
%%  mode:               outline
%%  outline-regexp:     "\\\\Chapter\\|\\\\Section"
%%  fill-column:        73
%%  eval:               (hide-body)
%%  End:
%%



