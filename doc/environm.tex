%%%%%%%%%%%%%%%%%%%%%%%%%%%%%%%%%%%%%%%%%%%%%%%%%%%%%%%%%%%%%%%%%%%%%%%%%%%%%
%%
%A  environm.tex                GAP documentation            Martin Schoenert
%%
%A  @(#)$Id: environm.tex,v 1.1.1.1 1996/12/11 12:36:44 werner Exp $
%%
%Y  Copyright 1990-1992,  Lehrstuhl D fuer Mathematik,  RWTH Aachen,  Germany
%%
%%  This chapter describes the functions and facilites of {\GAP} environment.
%%
%H  $Log: environm.tex,v $
%H  Revision 1.1.1.1  1996/12/11 12:36:44  werner
%H  Preparing 3.4.4 for release
%H
%H  Revision 3.12  1994/05/02  13:49:49  sam
%H  fixed a typo (found by K. Lux)
%H
%H  Revision 3.11  1994/01/10  19:24:36  mschoene
%H  fixed trivial typo (noted by mfn)
%H
%H  Revision 3.10  1993/10/11  11:30:57  fceller
%H  clarified with variable to use in 'Edit'
%H
%H  Revision 3.9  1993/05/05  13:59:57  fceller
%H  added 'LogInputTo'
%H
%H  Revision 3.8  1993/02/19  10:48:42  gap
%H  adjustments in line length and spelling
%H
%H  Revision 3.7  1993/02/18  09:19:03  felsch
%H  example fixed
%H
%H  Revision 3.6  1993/02/11  12:20:47  martin
%H  changed reference "Strings" to "Strings and Characters"
%H
%H  Revision 3.5  1993/02/10  20:49:08  martin
%H  fixed some typos
%H
%H  Revision 3.4  1992/04/02  15:43:12  martin
%H  replaced 'Linelength' by 'SizeScreen'
%H
%H  Revision 3.3  1992/03/12  16:18:31  martin
%H  fixed several typos
%H
%H  Revision 3.2  1992/01/23  13:44:51  martin
%H  added 'LogTo' and 'Linelength'
%H
%H  Revision 3.1  1992/01/11  18:32:46  martin
%H  changed the sections 'Help' and 'Reading Sections' to fit on one screen
%H
%H  Revision 3.0  1991/12/27  16:10:27  martin
%H  initial revision under RCS
%H
%%
\Chapter{Environment}

This  chapter describes the  interactive  environment in  which  you  use
{\GAP}.

The first sections describe  the main read  eval print loop and the break
loop (see "Main Loop", "Break Loops", and "Error").

The next section describes the commands you  can use to edit  the current
input line (see "Line Editing").

The next sections describe the {\GAP} help  system (see  "Help", "Reading
Sections",  "Format of Sections",    "Browsing through   the   Sections",
"Redisplaying a Section", "Abbreviating Section Names", "Help Index").

The next sections  describe  the input and output functions  (see "Read",
"ReadLib", "Print",  "PrintTo",  "AppendTo",  "LogTo", "LogInputTo",  and
"SizeScreen").

The  next  sections describe  the  functions that  allow you to   collect
statistics about a computation (see "Runtime", "Profile").

The last sections describe the functions that allow  you to execute other
programs as subprocesses from within {\GAP} (see "Exec" and "Edit").

%%%%%%%%%%%%%%%%%%%%%%%%%%%%%%%%%%%%%%%%%%%%%%%%%%%%%%%%%%%%%%%%%%%%%%%%%
\Section{Main Loop}%
\index{read eval print loop}\index{loop!read eval print}%
\index{prompt}\index{prompt!partial}%
\index{syntax errors}\index{errors!syntax}%
\index{output!suppressing}%
\index{last}\index{previous result}%

The normal  interaction with {\GAP} happens in  the so--called *read eval
print* loop.  This means that you type an input, {\GAP}  first reads  it,
evaluates it, and prints the result.  The exact sequence is as follows.

To show you that  it is ready  to accept your  input, {\GAP} displays the
*prompt* 'gap> '.  When you see this, you know that {\GAP} is waiting for
your input.

Note that every statement must be terminated by a  semicolon.   You  must
also enter <return> before {\GAP} starts to read and evaluate your input.
Because  {\GAP} does not  do anything until  you enter  <return>, you can
edit  your input to fix typos and only when  everything is  correct enter
<return> and have {\GAP}  take a  look at it (see "Line Editing").  It is
also possible to enter several statements as  input on a single line.  Of
course each statement must be terminated by a semicolon.

It is absolutely acceptable to enter a single statement on several lines.
When you have entered the beginning of a  statement, but the statement is
not  yet  complete, and  you  enter  <return>, {\GAP}  will  display  the
*partial prompt*   '> '.   When you see   this, you  know that {\GAP}  is
waiting for the rest of the statement.  This happens also when you forget
the semicolon ';' that terminates every {\GAP} statement.

When you enter <return>, {\GAP} first checks your  input to see  if it is
syntactically  correct (see chapter  "The Programming  Language"  for the
definition  of syntactically correct).   If it is  not, {\GAP}  prints an
error message of the following form

|    gap> 1 * ;
    Syntax error: expression expected
    1 * ;
        ^ |

The first line tells you what is wrong about  the input, in this case the
'\*' operator takes  two expressions as  operands, so obviously the right
one is missing.  If the  input came from a  file (see "Read"), this  line
will also contain the filename and the line number.  The second line is a
copy of the input.  And the third  line contains a  caret pointing to the
place in the previous line where {\GAP} realized that something is wrong.
This need not be  the exact place where  the error is,  but it is usually
quite close.

Sometimes, you will also  see a partial prompt after  you have entered an
input  that is  syntactically incorrect.   This is  because  {\GAP} is so
confused by your  input, that it thinks that  there is still something to
follow.  In this case you  should enter ';<return>' repeatedly,  ignoring
further error messages,  until you see the full  prompt  again.  When you
see the full prompt, you know that {\GAP} forgave you and is now ready to
accept your next -- hopefully correct -- input.

If your input is syntactically correct, {\GAP} evaluates  or executes it,
i.e., performs the  required computations  (see chapter "The  Programming
Language" for the definition of the evaluation).

If you do not see a prompt, you know that {\GAP} is still working on your
last input.   Of  course,  you  can  *type ahead*,   i.e., already  start
entering new input, but it  will not be  accepted by {\GAP} until  {\GAP}
has completed the ongoing computation.

When {\GAP} is ready it will usually print the result of the computation,
i.e., the value computed.  Note that not  all statements produce a value,
for example, if you enter a 'for'  loop, nothing will be printed, because
the 'for' loop does not produce a value that could be printed.

Also  sometimes you do  not want to see the   result.  For example if you
have  computed a value and now  want to assign the  result to a variable,
you  probably do not   want to see the   value again.  You  can terminate
statements by *two* semicolons to suppress the printing of the result.

If you have entered several statements on a single line {\GAP} will first
read, evaluate, and  print the first one,  then read evaluate, and  print
the second one, and so on.  This means that the second statement will not
even be checked  for  syntactical correctness until {\GAP}  has completed
the first computation.

After the result has been printed {\GAP} will display another prompt, and
wait for your next input.  And  the whole process  starts all over again.
Note that a  new  prompt will  only  be  printed after  {\GAP} has  read,
evaluated, and printed the  last  statement if  you have entered  several
statements on a single line.

In each statement  that you  enter the  result of  the previous statement
that produced  a value is available in  the variable 'last'.  The next to
previous result is available in  'last2' and  the result produced  before
that is available in 'last3'.

|    gap> 1; 2; 3;
    1
    2
    3
    gap> last3 + last2 * last;
    7 |

Also in each statement the time spent by the  last  statement, whether it
produced a value or not, is available in the variable 'time'.  This is an
integer that holds the number of milliseconds.

%%%%%%%%%%%%%%%%%%%%%%%%%%%%%%%%%%%%%%%%%%%%%%%%%%%%%%%%%%%%%%%%%%%%%%%%%
\Section{Break Loops}%

When  an  error  has  occurred  or when you  interrupt {\GAP}, usually by
hitting <ctr>-'C', {\GAP} enters a break loop,  that is in most  respects
like the main read  eval print loop  (see "Main Loop").  That is, you can
enter  statements,  {\GAP} reads  them, evaluates  them,  and prints  the
result if  any.  However those  evaluations  happen within the context in
which the error  occurred.   So you can look at the  arguments and  local
variables of the functions that were active  when the error happened  and
even change  them.   The  prompt is changed  from 'gap> ' to  'brk>  ' to
indicate that you are in a break loop.

There are two ways to leave a break loop.

The first is to quit the break loop and continue in the main loop.  To do
this  you enter 'quit;'  or hit the <eof>  (*e*nd *o*f *f*ile) character,
which is usually <ctr>-'D'.   In this  case  control returns to  the main
loop, and you can enter new statements.

The other way is  to return  from a  break loop.  To  do this  you  enter
'return;' or 'return <expr>;'.  If the break loop was entered because you
interrupted {\GAP}, then you can continue by entering  'return;'.  If the
break loop was  entered due to  an error, you  usually have  to  return a
value to continue the computation.   For example,  if the break  loop was
entered because a  variable  had no assigned  value, you  must return the
value that this variable should have to continue the computation.

%%%%%%%%%%%%%%%%%%%%%%%%%%%%%%%%%%%%%%%%%%%%%%%%%%%%%%%%%%%%%%%%%%%%%%%%%
\Section{Error}%

'Error( <messages>... )'

'Error' signals  an  error.  First  the messages <messages> are  printed,
this  is done exactly as if 'Print' (see "Print") were  called with these
arguments.  Then a break loop (see "Break Loops") is  entered, unless the
standard error output is not connected to a terminal.  You can leave this
break  loop  with 'return;'  to  continue execution  with  the  statement
following the call to 'Error'.

%%%%%%%%%%%%%%%%%%%%%%%%%%%%%%%%%%%%%%%%%%%%%%%%%%%%%%%%%%%%%%%%%%%%%%%%%
\Section{Line Editing}%

{\GAP} allows you to edit the current input line with a number of editing
commands.  Those commands  are accessible either as  *control keys* or as
*escape keys*.  You enter a control key  by  pressing the <ctr> key, and,
while still holding the <ctr> key  down, hitting another  key <key>.  You
enter an escape key by hitting <esc> and then hitting  another key <key>.
Below   we denote control  keys   by  <ctr>-<key>  and  escape  keys   by
<esc>-<key>.  The case  of <key>  does   not matter, i.e., <ctr>-'A'  and
<ctr>-'a' are equivalent.

Characters  not mentioned below always  insert  themselves at the current
cursor position.

The first few commands allow you to move the cursor on the current line.

<ctr>-'A'       move the cursor to the beginning of the line. \\
<esc>-'B'       move the cursor to the beginning of the previous word. \\
<ctr>-'B'       move the cursor backward one character. \\
<ctr>-'F'       move the cursor forward  one character. \\
<esc>-'F'       move the cursor to the end of the next word. \\
<ctr>-'E'       move the cursor to the end of the line.

The next commands delete or  kill  text.  The  last killed   text can  be
reinserted, possibly at a different position with the yank command.

<ctr>-'H' or <del> delete the character left of the cursor. \\
<ctr>-'D'       delete the character under the cursor. \\
<ctr>-'K'       kill up to the end of the line. \\
<esc>-'D'       kill forward to the end of the next word. \\
<esc>-<del>     kill backward to the beginning of the last word. \\
<ctr>-'X'       kill entire input line, and discard all pending input. \\
<ctr>-'Y'       insert (yank) a just killed text.

The next commands allow you to change the input.

<ctr>-'T'       exchange (twiddle) current and previous character. \\
<esc>-'U'       uppercase next word. \\
<esc>-'L'       lowercase next word. \\
<esc>-'C'       capitalize next word.

The <tab> character, which is in fact the control key <ctr>-'I', looks at
the characters before the cursor, interprets them as the  beginning of an
identifier and  tries to complete this identifier.  If there is more than
one possible completion, it completes to the longest common prefix of all
those completions.   If  the  characters  to the  left  of the cursor are
already  the longest  common  prefix of all completions  hitting <tab>  a
second time will display all possible completions.

<tab>           complete the identifier before the cursor.

The next  commands allow  you to fetch  previous lines, e.g.,  to correct
typos, etc.  This history is limited to about 8000 characters.

<ctr>-'L'       insert last input line before current character.\\
<ctr>-'P'       redisplay  the last input   line, another  <ctr>-'P' will
                redisplay the line  before that, etc.   If  the cursor is
                not in the first column  only the lines starting with the
                string to the left  of the cursor  are taken.\\
<ctr>-'N'       Like <ctr>-'P' but goes the other way round  through  the
                history.\\
<esc>-'\<'      goes to the beginning of the history.\\
<esc>-'>'       goes to the end of the history.\\
<ctr>-'O'       accepts this line and perform a <ctr>-'N'.

Finally there are a few miscellaneous commands.

<ctr>-'V'       enter next character literally, i.e., enter it even if it
                is one of the control keys.\\
<ctr>-'U'       execute the next command 4 times.\\
<esc>-<num>     execute the next command <num> times.\\
<esc>-<ctr>-'L'  repaint input line.

%%%%%%%%%%%%%%%%%%%%%%%%%%%%%%%%%%%%%%%%%%%%%%%%%%%%%%%%%%%%%%%%%%%%%%%%%
\Section{Help}%

This  section describes  together with  the following sections the {\GAP}
help system.  The help system lets you read the manual interactively.

'?<section>'

The help command '?' displays the section  with the name <section> on the
screen.  For  example '?Help'  will display this   section on the screen.
You should not  type in  the single quotes,   they are only  used in help
sections to delimit text that you should enter into {\GAP} or that {\GAP}
prints in response.  When the whole section has been displayed the normal
{\GAP} prompt 'gap>' is shown and normal {\GAP} interaction resumes.

The section "Reading  Sections"  tells  you what actions  you can perform
while  you are reading  a  section.   You  command {\GAP} to display this
section  by  entering '?Reading Sections', without quotes.   The  section
"Format of Sections" describes the format of sections and the conventions
used, "Browsing through the  Sections" lists the commands you use to flip
through  sections, "Redisplaying  a  Section" describes  how  to  read  a
section again, "Abbreviating Section Names" tells you how to avoid typing
the long section names, and "Help Index" describes the index command.

%%%%%%%%%%%%%%%%%%%%%%%%%%%%%%%%%%%%%%%%%%%%%%%%%%%%%%%%%%%%%%%%%%%%%%%%%
\Section{Reading Sections}%
\index{help!scrolling}

If the  section is longer than 24  lines {\GAP} stops  after 24 lines and
displays

|    -- <space> for more --|

If you press <space> {\GAP} displays the next 24 lines of the section and
then stops again.   This  goes on   until  the whole   section  has  been
displayed, at which  point {\GAP}  will  return immediately to  the  main
{\GAP} loop.  Pressing 'f' has the same effect as <space>.

You can also press 'b' or the key labeled <del> which will scroll back to
the *previous* 24 lines of  the section.  If  you press 'b' or <del> when
{\GAP} is displaying the top of a section {\GAP} will ring the bell.

You can  also press 'q'  to quit and  return immediately back to the main
{\GAP} loop without reading the rest of the section.

Actually the 24 is only a default,  if you have a larger screen  that can
display more lines of text you  may want to tell this to {\GAP} with  the
'-y <rows>' option when you start {\GAP}.

%%%%%%%%%%%%%%%%%%%%%%%%%%%%%%%%%%%%%%%%%%%%%%%%%%%%%%%%%%%%%%%%%%%%%%%%%
\Section{Format of Sections}%
\index{help!format}

This section describes the format of sections when they are displayed  on
the screen and the special conventions used.

As you can see {\GAP} indents sections 4 spaces and  prints a header line
containing  the  name of the  section  on the  left and  the  name of the
chapter on the right.

|<text>|

Text enclosed in angle brackets is used for arguments in the descriptions
of  functions and for other placeholders.   It  means that you should not
actually  enter this text into  {\GAP} but replace   it by an appropriate
text depending on what you  want to do.   For example when we write  that
you should enter '?<section>' to see the section with the name <section>,
<section> servers as  a placeholder, indicating  that  you can enter  the
name of the section that you want to  see at this  place.  In the printed
manual such text is printed in italics.

|'text'|

Text  enclosed  in single  quotes  is used  for  names of  variables  and
functions and other  text that you may actually enter into  your computer
and see on  your  screen.  The text enclosed in single quotes may contain
placeholders enclosed in angle brackets as described above.   For example
when the  help  text  for 'IsPrime'  says  that the form of  the  call is
|'IsPrime( <n> )'| this means that you should actually enter the IsPrime(
and  ),  without the  quotes,  but  replace the  <n> with the  number (or
expression) that  you  want  to test.  In the printed manual this text is
printed in a monospaced  (all characters have  the same width) typewriter
font.

|"text"|

Text enclosed  in  double  quotes  is  used for cross references to other
parts of the manual.  So the text inside the double quotes is the name of
another  section of  the manual.  This is  used  to  direct you  to other
sections that describe a topic or  a  function used  in this section.  So
for example "Abbreviating Section Names" is a cross reference to the next
section.  In the printed manual the text is replaced by the number of the
section.

|_| and |^|

In mathematical formulas the underscore and  the caret are used to denote
subscription  and superscription.  Ordinarily they apply only to the very
next  character  following,   unless  a   whole  expression  enclosed  in
parentheses follows.  So for example |x_1^(i+1)| denotes the variable 'x'
with  subscript  1 raised  to the  'i+1'  power.  In  the  printed manual
mathematical  formulas are typeset in italics (actually mathitalics)  and
subscripts and superscripts are actually lowered and raised.

Longer examples are usually  paragraphs of their  own that are indented 8
spaces from the left margin, i.e.  4 spaces further than  the surrounding
text.  Everything  on the lines with  the prompts 'gap>' and '>',  except
the  prompts  themselves  of course,  is the  input  you  have  to  type,
everything else is {\GAP}\'s response.   In the printed  manual  examples
are also indented 4 spaces  and are  printed  in  a monospaced typewriter
font.

|    gap> ?Format of Sections
    Format of Sections ______________________________________ Environment

    This section describes the format of sections when they are displayed
    on the screen and the special conventions used.

    ... |

%%%%%%%%%%%%%%%%%%%%%%%%%%%%%%%%%%%%%%%%%%%%%%%%%%%%%%%%%%%%%%%%%%%%%%%%%
\Section{Browsing through the Sections}%
\index{help!browsing}

The help  sections are organized like  a book into chapters.  This should
not  surprise you, since   the same source  is used  both for the printed
manual and the online help.  Just as you can flip  through the pages of a
book there are special commands to browse through the help sections.

'?>' \\
'?\<'

The two help commands '?\<' and '?>' correspond to the flipping of pages.
'?\<' takes you to the section preceding the current section and displays
it, and '?>' takes you to the section following the current section.

'?\<\<' \\
'?>>'

'?\<\<' is like '?\<',  only  more so.  It   takes you back to  the first
section of the current  chapter, which gives an  overview of the sections
described in  this chapter.  If you  are already in  this section '?\<\<'
takes you to the first section of  the previous chapter.  '?>>' takes you
to the first section of the next chapter.

'?-'\\
'?+'

{\GAP} remembers the sections that you have read.  '?-'  takes you to the
one that you  have read before  the  current one, and  displays it again.
Further '?-' takes you further back  in this history.  '?+' reverses this
process, i.e.,  it  takes you  back to   the section that   you have read
*after* the current one.  It is important to  note, that '?-' and '?+' do
*not* alter the history like the other help commands.

%%%%%%%%%%%%%%%%%%%%%%%%%%%%%%%%%%%%%%%%%%%%%%%%%%%%%%%%%%%%%%%%%%%%%%%%%
\Section{Redisplaying a Section}%
\index{help!redisplaying}

'?'

The help command '?' followed by no section name redisplays the last help
section again.  So  if you  reach  the bottom of  a long help section and
already forgot what  was mentioned at the beginning, or, for example, the
examples  do  not  seem  to  agree  with  your   interpretation  of   the
explanations, use '?' to read the whole section again from the beginning.

When  '?' is used before any section  has  been read {\GAP}  displays the
section 'Welcome to GAP'.

%%%%%%%%%%%%%%%%%%%%%%%%%%%%%%%%%%%%%%%%%%%%%%%%%%%%%%%%%%%%%%%%%%%%%%%%%
\Section{Abbreviating Section Names}%
\index{help!abbreviating}

Upper and lower case in <section> are not distinguished, so typing either
'?Abbreviating Section Names' or '?abbreviating section names' will  show
this very section.

Each  word  in  <section>  may be  abbreviated.   So  instead  of  typing
'?abbreviating section names'  you may also type '?abb sec nam',  or even
'?a s n'.  You  must not omit the spaces separating  the words.  For each
word in the section name you must  give at least the first character.  As
another example you may  type '?oper for int' instead of '?operations for
integers', which is especially handy when you can not remember whether it
was 'operations' or 'operators'.

If an abbreviation matches multiple   section names a   list of all these
section names is displayed.

%%%%%%%%%%%%%%%%%%%%%%%%%%%%%%%%%%%%%%%%%%%%%%%%%%%%%%%%%%%%%%%%%%%%%%%%%
\Section{Help Index}%
\index{help!index}

'??<topic>'

'??' looks up <topic> in {\GAP}\'s index and prints all the index entries
that contain the substring <topic>.  Then you can decide which section is
the one you are actually interested in and request this one.

|    gap> ??help
        help ______________________________________________________ Index
        Help
        Reading Sections (help!scrolling)
        Format of the Sections (help!format)
        Browsing through the Sections (help!browsing)
        Redisplaying a Section (help!redisplaying)
        Abbreviating Section Names (help!abbreviating)
        Help Index
    gap> |

The first thing on each line is the name of the  section.  If the name of
the section matches <topic> nothing more is printed.  Otherwise the index
entry  that matched   <topic>  is printed   in  parentheses following the
section name.  For  each section only  the first matching index entry  is
printed.  The  order  of the sections corresponds  to  their order in the
{\GAP} manual, so that related sections should be adjacent.

%%%%%%%%%%%%%%%%%%%%%%%%%%%%%%%%%%%%%%%%%%%%%%%%%%%%%%%%%%%%%%%%%%%%%%%%%
\Section{Read}%
\index{read!a file}\index{load!a file}%
\index{file!read a}\index{file!load a}

'Read( <filename> )'

'Read' reads the input from the  file with the filename <filename>, which
must be a string.

'Read' first opens the file <filename>.   If the file  does not exist, or
if {\GAP} can not open it, e.g., because of access restrictions, an error
is signalled.

Then the contents of the file are read and evaluated, but the results are
not printed.  The  reading and printing happens exactly  as described for
the main loop (see "Main Loop").

If an   input in  the file contains   a syntactical  error, a message  is
printed, and the rest of  this statement is ignored,  but the rest of the
file is read.

If a statement in the file  causes an error  a break loop is entered (see
"Break Loops").   The  input for this break  loop  is not taken  from the
file, but from the input connected to the 'stderr'  output of {\GAP}.  If
'stderr'  is not connected to  a terminal, no  break loop is entered.  If
this break loop is left with 'quit' (or <ctr>-'D') the file is closed and
{\GAP} does not continue to read from it.

Note that a statement may not begin in one file and end in another, i.e.,
<eof> (*e*nd *o*f *f*ile) is not treated as  whitespace, but as a special
symbol that must not appear inside any statement.

Note that one file may very well contain a read statement causing another
file to  be read, before input is again taken from the first file.  There
is an operating  system dependent maximum on the number of files that may
be open at once, usually it is 15.

The special  file name '\"\*stdin\*\"'  denotes the standard input, i.e.,
the stream through  which the user  enters commands to {\GAP}.  The exact
behaviour of  'Read( \"\*stdin\*\" )'  is operating system dependent, but
usually  the  following happens.   If {\GAP}   was started  with no input
redirection, statements are read from  the terminal stream until the user
enters the end of file character,  which is usually <ctr>-'D'.  Note that
terminal  streams are special,  in that  they  may  yield ordinary  input
*after* an end of file.  Thus when control returns to the main read eval
print loop the user can continue with {\GAP}.  If {\GAP} was started with
an input  redirection, statements are read  from the current  position in
the input file up  to the end of the  file.  When control returns  to the
main read eval print loop the input stream will still return end of file,
and {\GAP} will terminate.  The special file name '\"\*errin\*\"' denotes
the stream  connected with the 'stderr' output.   This stream  is usually
connected  to the terminal,  even if  the standard  input was redirected,
unless  the  standard error  stream was  also  redirected, in  which case
opening of '\"\*errin\*\"' fails, and 'Read' will signal an error.

'Read'  is implemented  in  terms of the  function 'READ', which  behaves
exactly like 'Read', except that 'READ' does not signal an  error when it
can not open  the file.  Instead it returns 'true' or 'false' to indicate
whether opening the file was successful or not.

%%%%%%%%%%%%%%%%%%%%%%%%%%%%%%%%%%%%%%%%%%%%%%%%%%%%%%%%%%%%%%%%%%%%%%%%%
\Section{ReadLib}%
\index{read!a library file}\index{load!a library file}%
\index{file!read a library}\index{file!load a library}

'ReadLib( <name> )'

'ReadLib'  reads   input from  the  library  file  with  the name <name>.
'ReadLib' prefixes  <name> with the value of  the variable  'LIBNAME' and
appends the string '\".g\"' and calls 'Read'  (see "Read") with this file
name.

%%%%%%%%%%%%%%%%%%%%%%%%%%%%%%%%%%%%%%%%%%%%%%%%%%%%%%%%%%%%%%%%%%%%%%%%%
\Section{Print}%

'Print( <obj1>, <obj2>... )'

'Print' prints the objects <obj1>, <obj2>... etc. to the standard output.
The output looks exactly like  the printed  representation of the objects
printed by the main loop.  The exception are strings,  which are  printed
without  the enclosing  quotes  and  a  few  other  transformations  (see
"Strings  and Characters").   Note that  no  space  or newline is printed
between  the  objects.  'PrintTo'  can  be used  to print to a  file (see
"PrintTo").

|    gap> for i  in [1..5]  do
    >     Print( i, " ", i^2, " ", i^3, "\n" );
    > od;
    1 1 1
    2 4 8
    3 9 27
    4 16 64
    5 25 125 |

%%%%%%%%%%%%%%%%%%%%%%%%%%%%%%%%%%%%%%%%%%%%%%%%%%%%%%%%%%%%%%%%%%%%%%%%%
\Section{PrintTo}%
\index{print!to a file}%
\index{file!print to a}

'PrintTo( <filename>, <obj1>, <obj2>... )'

'PrintTo' works  like  'Print', except that  the output is printed to the
file with the name <filename> instead of the  standard output.  This file
must of  course be writable by {\GAP},  otherwise  an error is signalled.
Note that 'PrintTo' will overwrite the previous contents  of this file if
it  already existed.  'AppendTo' can be used  to append  to  a file  (see
"AppendTo").

The  special  file  name  '\"\*stdout\*\"' can  be  used to print  to the
standard output.   This is equivalent to a  plain 'Print', except  that a
plain  'Print'  that is  executed  while   evaluating an argument    to a
'PrintTo' call will  also  print to  the output  file opened by  the last
'PrintTo'  call,  while  'PrintTo(  \"\*stdout\*\", <obj1>,  <obj2>... )'
always   prints to     the standard output.    The  special   file   name
'\"\*errout\*\"' can be used to print to the standard error  output file,
which is usually connected with the terminal, even if the standard output
was redirected.

There is an  operating  system dependent maximum to the number  of output
files that may be open at once, usually this is 14.

%%%%%%%%%%%%%%%%%%%%%%%%%%%%%%%%%%%%%%%%%%%%%%%%%%%%%%%%%%%%%%%%%%%%%%%%%
\Section{AppendTo}%
\index{append!to a file}%
\index{file!append to a}

'AppendTo( <filename>, <obj1>, <obj2>... )'

'AppendTo' works like  'PrintTo'  (see "PrintTo"), except that the output
does not overwrite the previous contents of the  file, but is appended to
the file.

%%%%%%%%%%%%%%%%%%%%%%%%%%%%%%%%%%%%%%%%%%%%%%%%%%%%%%%%%%%%%%%%%%%%%%%%%
\Section{LogTo}%
\index{log!to a file}%
\index{file!log to a}

'LogTo( <filename> )'

'LogTo' causes the subsequent interaction  to be logged to the file  with
the name <filename>, i.e., everything you see on your  terminal will also
appear in this  file.   This file must  of course be  writable by {\GAP},
otherwise  an error  is  signalled.  Note that 'LogTo' will overwrite the
previous contents of this file if it already existed.

'LogTo()'

In this form 'LogTo' stops logging again.

%%%%%%%%%%%%%%%%%%%%%%%%%%%%%%%%%%%%%%%%%%%%%%%%%%%%%%%%%%%%%%%%%%%%%%%%%
\Section{LogInputTo}%
\index{log input!to a file}%
\index{file!log input to a}

'LogInputTo( <filename> )'

'LogInputTo' causes  the subsequent input lines to be logged to the  file
with the name <filename>, i.e., every line  you type will also appear  in
this file.  This file must  of course be writable by {\GAP}, otherwise an
error is signalled.  Note that 'LogInputTo'  will  overwrite the previous
contents of this file if it already existed.

'LogInputTo()'

In this form 'LogInputTo' stops logging again.

%%%%%%%%%%%%%%%%%%%%%%%%%%%%%%%%%%%%%%%%%%%%%%%%%%%%%%%%%%%%%%%%%%%%%%%%%
\Section{SizeScreen}%

'SizeScreen()'

In this form 'SizeScreen' returns the size of the screen as  a list  with
two entries.  The first is the length of each line,  the  second  is  the
number of lines.

'SizeScreen( [ <x>, <y> ] )'

In this form 'SizeScreen' sets the size of the screen.  <x> is the length
of  each line, <y> is the number of  lines.  Either value may be missing,
to leave  this value unaffected.  Note that those parameters can  also be
set with the command line options '-x <x>' and '-y <y>' (see "Getting and
Installing GAP").

%%%%%%%%%%%%%%%%%%%%%%%%%%%%%%%%%%%%%%%%%%%%%%%%%%%%%%%%%%%%%%%%%%%%%%%%%
\Section{Runtime}%

'Runtime()'

'Runtime' returns the time spent by {\GAP} in milliseconds as an integer.
This is usually the cpu time,  i.e., not the  wall clock time.  Also time
spent by subprocesses of {\GAP} (see "Exec") is not counted.

%%%%%%%%%%%%%%%%%%%%%%%%%%%%%%%%%%%%%%%%%%%%%%%%%%%%%%%%%%%%%%%%%%%%%%%%%
\Section{Profile}%

'Profile( true )'

In this form 'Profile' turns  the  profiling on.  Subsequent computations
will record the time spent by each function and the number of  times each
function was called.  Old profiling information is cleared.

'Profile( false )'

In  this   form 'Profile' turns    the  profiling off  again.    Recorded
information is still kept, so you can  display it even  after turning the
profiling off.

'Profile()'

In   this  form 'Profile'  displays  the  collected  information   in the
following format.

|    gap> Factors( 10^21+1 );;    # make sure that the library is loaded
    gap> Profile( true );
    gap> Factors( 10^42+1 );
    [ 29, 101, 281, 9901, 226549, 121499449, 4458192223320340849 ]
    gap> Profile( false );
    gap> Profile();
     count    time percent time/call child function
         4    1811      76     452    2324  FactorsRho
        18     171       7       9     237  PowerModInt
       127      94       3       0      94  GcdInt
        41      83       3       2     415  IsPrimeInt
        91      59       2       0      59  TraceModQF
       511      47       1       0      39  QuoInt
        22      23       0       1      23  Jacobi
       116      20       0       0      31  log
         3      20       0       6      70  SmallestRootInt
         1      19       0      19    2370  FactorsInt
        26      15       0       0      39  LogInt
         4       4       0       1       4  Concatenation
         5       4       0       0      20  RootInt
         7       0       0       0       0  Add
        26       0       0       0       0  Length
        13       0       0       0       0  NextPrimeInt
         4       0       0       0       0  AddSet
         4       0       0       0       0  IsList
         4       0       0       0       0  Sort
         8       0       0       0       0  Append
              2369     100                  TOTAL |

The last  column contains the  name  of the function.   The  first column
contains the number of times each function was called.  The second column
contains the time spent in this function.  The third column  contains the
percentage of the total  time spent in this  function.  The fourth column
contains the time per call, i.e., the quotient of the second by the first
number.  The fifth column  contains the time  spent  in this function and
all other functions called, directly or indirectly, by this function.

%%%%%%%%%%%%%%%%%%%%%%%%%%%%%%%%%%%%%%%%%%%%%%%%%%%%%%%%%%%%%%%%%%%%%%%%%
\Section{Exec}%

'Exec( <command> )'

'Exec'  executes  the  command  given by  the  string <command>   in  the
operating system.  How this happens is operating system dependent.  Under
UNIX, for example, a new  shell is started  and <command>  is passed as a
command to this shell.

|    gap> Exec( "date" );
    Fri Dec 13 17:00:29 MET 1991 |

'Edit' (see "Edit") should be used to call an editor from within {\GAP}.

%%%%%%%%%%%%%%%%%%%%%%%%%%%%%%%%%%%%%%%%%%%%%%%%%%%%%%%%%%%%%%%%%%%%%%%%%
\Section{Edit}%

'Edit( <filename> )'

'Edit' starts an editor  with  the file  whose  filename is given by  the
string <filename>, and reads the file back into  {\GAP} when you exit the
editor again.  You should set the {\GAP} variable 'EDITOR' to the name of
the  editor  that you  usually use,  e.g.,  '/usr/ucb/vi'.   This can for
example  be done in   your '.gaprc' file  (see  the sections on operating
system dependent features in chapter "Getting and Installing GAP").

