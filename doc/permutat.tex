%%%%%%%%%%%%%%%%%%%%%%%%%%%%%%%%%%%%%%%%%%%%%%%%%%%%%%%%%%%%%%%%%%%%%%%%%%%%%
%%
%A  permutat.tex                GAP documentation                   Udo Polis
%%
%A  @(#)$Id: permutat.tex,v 1.1.1.1 1996/12/11 12:36:49 werner Exp $
%%
%Y  Copyright 1990-1992,  Lehrstuhl D fuer Mathematik,  RWTH Aachen,  Germany
%%
%%  This file describes those functions that deal with permutation groups
%%
%H  $Log: permutat.tex,v $
%H  Revision 1.1.1.1  1996/12/11 12:36:49  werner
%H  Preparing 3.4.4 for release
%H
%H  Revision 3.6  1993/03/11  17:48:26  fceller
%H  removed reference to bound of domain
%H
%H  Revision 3.5  1993/02/09  16:29:11  felsch
%H  another example adjusted
%H
%H  Revision 3.4  1993/02/01  16:28:08  felsch
%H  example fixed
%H
%H  Revision 3.3  1992/04/07  13:07:55  martin
%H  fixed some more typos
%H
%H  Revision 3.2  1992/03/13  10:44:47  martin
%H  changed two section titles
%H
%H  Revision 3.1  1992/01/15  14:09:18  martin
%H  initial revision under RCS
%H
%%
\Chapter{Permutations}

{\GAP} is a system  especially  designed for the computations  in groups.
Permutation groups are a very important class of groups and {\GAP} offers
a data type *permutation* to describe the elements of permutation groups.

Permutations  in  {\GAP}  operate on *positive integers*.  Whenever group
elements  operate  on  a domain  we  call  the elements  of  this  domain
*points*.  Thus in this chapter we often  call  positive integers points,
if we want to  emphasize that a permutation operates on them.  An integer
$i$ is said to  be *moved* by a permutation $p$ if the image $i^p$ of $i$
under $p$ is not  $i$.  The largest  integer moved by any permutation may
not be larger  than  $2^{28}-1$.

Note that permutations  do  not belong to  a specific group.   That means
that you can work  with permutations without defining a permutation group
that contains them.  This is  just like  it is  with integers, with which
you can compute without caring about the domain 'Integers' that  contains
them.  It also means that you can multiply any two permutations.

Permutations are entered and displayed in cycle notation.

|    gap> (1,2,3);
    (1,2,3)
    gap> (1,2,3) * (2,3,4);
    (1,3)(2,4) |

The first  sections  in  this chapter describe  the  operations that  are
available  for  permutations  (see  "Comparisons  of  Permutations"   and
"Operations for Permutations").   The next section describes the function
that  tests whether an object is a permutation (see "IsPerm").   The next
sections describe the functions that find the largest and smallest  point
moved    by    a    permutation    (see    "LargestMovedPointPerm"    and
"SmallestMovedPointPerm").  The next section describes  the function that
computes  the sign  of a permutation (see "SignPerm").  The next  section
describes the  function  that  computes  the  smallest  permutation  that
generates  the   same  cyclic  subgroup  as  a  given  permutation   (see
"SmallestGeneratorPerm").  The final sections describe the functions that
convert  between  lists and  permutations  (see  "ListPerm",  "PermList",
"RestrictedPerm", and "MappingPermListList").

Permutations are  elements  of groups operating on positive integers in a
natural way, thus see chapter "Groups"  and chapter "Operations" for more
functions.

The external functions are in the file 'LIBNAME/\"permutat.g\"'.

%%%%%%%%%%%%%%%%%%%%%%%%%%%%%%%%%%%%%%%%%%%%%%%%%%%%%%%%%%%%%%%%%%%%%%%%%
\Section{Comparisons of Permutations}%

'<p1> = <p2>' \\
'<p1> \<> <p2>'

The  equality operator  '=' evaluates to  'true'  if the two permutations
<p1> and  <p2> are equal,  and  to  'false'  otherwise.   The  inequality
operator '\<>' evaluates to 'true' if the two permutations <p1>  and <p2>
are  not  equal,  and  to  'false'   otherwise.   You  can  also  compare
permutations with objects of other types, of course they are never equal.

Two permutations are considered equal  if and  only if they move the same
points and if  the images  of the  moved  points are  the  same under the
operation of both permutations.

|    gap> (1,2,3) = (2,3,1);
    true
    gap> (1,2,3) * (2,3,4) = (1,3)(2,4);
    true |

'<p1> \<  <p2>' \\
'<p1> \<= <p2>' \\
'<p1>  >  <p2>' \\
'<p1>  >= <p2>'

The operators '\<',  '\<=',  '>',  and  '>='  evaluate  to 'true'  if the
permutation <p1> is less  than,  less than or  equal to, greater than, or
greater than or equal to the permutation <p2>, and to 'false' otherwise.

Let $p_1$ and $p_2$ be two  permutations that are  not equal.  Then there
exists  at least one point  $i$ such that $i^{p_1} \<> i^{p_2}$.  Let $k$
be the  smallest such point.  Then $p_1$ is considered smaller than $p_2$
if  and only  if $k^{p_1} \<\ k^{p_2}$.   Note that this implies that the
identity permutation is the smallest permutation.

You can also compare permutations with objects of other types.  Integers,
rationals, cyclotomics, unknowns, and  finite field  elements are smaller
than permutations.  Everything else is larger.

|    gap> (1,2,3) < (1,3,2);
    true    # $1^{(1,2,3)} = 2 \<\ 3 = 1^{(1,3,2)}$
    gap> (1,3,2,4) < (1,3,4,2);
    false    # $2^{(1,3,2,4)} = 4 > 1 = 2^{(1,3,4,2)}$ |

%%%%%%%%%%%%%%%%%%%%%%%%%%%%%%%%%%%%%%%%%%%%%%%%%%%%%%%%%%%%%%%%%%%%%%%%%
\Section{Operations for Permutations}%

'<p1> \*\ <p2>'

The operator '\*'  evaluates to the product of the two  permutations <p1>
and <p2>.

'<p1> / <p2>'

The operator  '/'  evaluates to the quotient $p1 \*  p2^{-1}$  of the two
permutations <p1> and <p2>.

'LeftQuotient( <p1>, <p2> )'

'LeftQuotient' returns  the left  quotient  $p1^{-1}  \* p2$ of  the  two
permutations <p1> and <p2>.  (This can also be written '<p1> mod <p2>'.)

'<p> \^\ <i>'

The operator '\^' evaluates to the <i>-th power of the permutation <p>.

'<p1> \^\ <p2>'

The operator '\^' evaluates to the conjugate $p2^{-1} \* p1 \* p2$ of the
permutation <p1> by the permutation <p2>.

'Comm( <p1>, <p2> )'

'Comm' returns the commutator $p1^{-1} \* p2^{-1} \* p1 \* p2$ of the two
permutations <p1> and <p2>.

'<i> \^\ <p>'

The operator '\^' evaluates to  the image $i^p$  of the  positive integer
<i> under the permutation <p>.

'<i> / <p>'

The operator  '/' evaluates to  the preimage  $i^{p^{-1}}$ of the integer
<i> under the permutation <p>.


'<list> \*\ <p>' \\
'<p> \*\ <list>'

The operator '\*' evaluates  to the  list of products of the permutations
in <list> with  the permutation <p>.  That  means that the value is a new
list <new>  such that  '<new>[<i>] = <list>[<i>]  \*\  <p>'  respectively
'<new>[<i>] = <p> \*\ <list>[<i>]'.

'<list> / <p>'

The operator '/' evaluates to the list  of  quotients of the permutations
in <list> with  the permutation  <p>.  That means that the value is a new
list <new> such that '<new>[<i>] = <list>[<i>] / <p>'.

For the precedence of the operators see "Operations".

%%%%%%%%%%%%%%%%%%%%%%%%%%%%%%%%%%%%%%%%%%%%%%%%%%%%%%%%%%%%%%%%%%%%%%%%%
\Section{IsPerm}

'IsPerm( <obj> )'

'IsPerm' returns 'true'  if  <obj>, which may be  an  object of arbitrary
type, is a permutation and 'false' otherwise.  It will signal an error if
<obj> is an unbound variable.

|    gap> IsPerm( (1,2) );
    true
    gap> IsPerm( 1 );
    false |

%%%%%%%%%%%%%%%%%%%%%%%%%%%%%%%%%%%%%%%%%%%%%%%%%%%%%%%%%%%%%%%%%%%%%%%%%
\Section{LargestMovedPointPerm}

'LargestMovedPointPerm( <perm> )'

'LargestMoverPointPerm'   returns  the   largest  point   moved   by  the
permutation  <perm>, i.e.,  the  largest  positive  integer <i> such that
'<i>\^<perm>  \<>  <i>'.   It will  signal an error  if <perm> is trivial
(see also "SmallestMovedPointPerm").

|    gap> LargestMovedPointPerm( (2,3,1) );
    3
    gap> LargestMovedPointPerm( (1,2)(1000,1001) );
    1001 |

%%%%%%%%%%%%%%%%%%%%%%%%%%%%%%%%%%%%%%%%%%%%%%%%%%%%%%%%%%%%%%%%%%%%%%%%%
\Section{SmallestMovedPointPerm}

'SmallestMovedPointPerm( <perm> )'

'SmallestMovedPointPerm'  returns  the  smallest   point  moved  by   the
permutation <perm>,  i.e., the  smallest  positive integer  <i> such that
'<i>\^<perm> \<>  <i>'.   It will signal  an error if  <perm>  is trivial
(see also "LargestMovedPointPerm").

|    gap> SmallestMovedPointPerm( (4,7,5) );
    4 |

%%%%%%%%%%%%%%%%%%%%%%%%%%%%%%%%%%%%%%%%%%%%%%%%%%%%%%%%%%%%%%%%%%%%%%%%%
\Section{SignPerm}

'SignPerm( <perm> )'

'SignPerm' returns the *sign* of the permutation <perm>.

The sign $s$ of a permutation $p$ is defined by
$s = \prod_{i \< j}{(i^p - j^p)} / \prod_{i \< j}{(i - j)}$,
where $n$ is the largest point moved by $p$ and $i,j$ range over $1...n$.

One can easily show that *sign* is equivalent to the *determinant* of the
*permutation  matrix* of <perm>.  Thus  it   is obvious that the function
*sign* is a homomorphism.

|    gap> SignPerm( (1,2,3)(5,6) );
    -1 |

%%%%%%%%%%%%%%%%%%%%%%%%%%%%%%%%%%%%%%%%%%%%%%%%%%%%%%%%%%%%%%%%%%%%%%%%%
\Section{SmallestGeneratorPerm}

'SmallestGeneratorPerm( <perm> )'

'SmallestGeneratorPerm' returns  the smallest permutation that  generates
the same cyclic group as the permutation <perm>.

|    gap> SmallestGeneratorPerm( (1,4,3,2) );
    (1,2,3,4) |

Note that 'SmallestGeneratorPerm' is very efficient, even when <perm> has
huge order.

%%%%%%%%%%%%%%%%%%%%%%%%%%%%%%%%%%%%%%%%%%%%%%%%%%%%%%%%%%%%%%%%%%%%%%%%%
\Section{ListPerm}

'ListPerm( <perm> )'

'ListPerm' returns a list <list> that contains the images of the positive
integers  under the permutation <perm>.   That means  that '<list>[<i>] =
<i>\^<perm>',  where  <i> lies between 1 and  the largest  point moved by
<perm> (see "LargestMovedPointPerm").

|    gap> ListPerm( (1,2,3,4) );
    [ 2, 3, 4, 1 ]
    gap> ListPerm( () );
    [  ] |

'PermList' (see "PermList") performs the inverse operation.

%%%%%%%%%%%%%%%%%%%%%%%%%%%%%%%%%%%%%%%%%%%%%%%%%%%%%%%%%%%%%%%%%%%%%%%%%
\Section{PermList}

'PermList( <list> )'

'PermList' returns the permutation <perm> that moves  points as describes
by the list  <list>.  That  means that '<i>\^<perm> = <list>[<i>]' if <i>
lies between 1 and the length of <list>, and  '<i>\^<perm> = <i>' if  <i>
is larger than the length of the list <list>.  It will signal an error if
<list> does not define a permutation,  i.e.,  if <list> is not a list  of
integers without holes,  or if <list>  contains  an integer  twice, or if
<list> contains an integer not in the range '[1..Length(<list>)]'.

|    gap> PermList( [6,2,4,1,5,3] );
    (1,6,3,4)
    gap> PermList( [] );
    () |

'ListPerm' (see "ListPerm") performs the inverse operation.

%%%%%%%%%%%%%%%%%%%%%%%%%%%%%%%%%%%%%%%%%%%%%%%%%%%%%%%%%%%%%%%%%%%%%%%%%
\Section{RestrictedPerm}

'RestrictedPerm( <perm>, <list> )'

'RestrictedPerm'  returns the new permutation <new> that  operates on the
points  in the list <list> in the same way as the permutation <perm>, and
that fixes those points that are not in <list>.  <list> must be a list of
positive  integers  such  that  for  each   <i>  in  <list>   the   image
'<i>\^<perm>' is also in <list>, i.e., it must be  the union of cycles of
<perm>.

|    gap> RestrictedPerm( (1,2,3)(4,5), [4,5] );
    (4,5) |

%%%%%%%%%%%%%%%%%%%%%%%%%%%%%%%%%%%%%%%%%%%%%%%%%%%%%%%%%%%%%%%%%%%%%%%%%
\Section{MappingPermListList}

'MappingPermListList( <list1>, <list2> )'

'MappingPermListList'   returns   a   permutation   <perm>    such   that
'<list1>[<i>] \^\ <perm> = <list2>[<i>]'.  <perm> fixes all points larger
then the  maximum  of the  entries in <list1> and <list2>.  If there  are
several    such    permutations,    it    is    not    specified    which
'MappingPermListList' returns.  <list1>  and  <list2>  must  be lists  of
positive integers  of the same length, and neither may contain an element
twice.

|    gap> MappingPermListList( [3,4], [6,9] );
    (3,6,4,9,8,7,5)
    gap> MappingPermListList( [], [] );
    () |

%%%%%%%%%%%%%%%%%%%%%%%%%%%%%%%%%%%%%%%%%%%%%%%%%%%%%%%%%%%%%%%%%%%%%%%%%
%%
%E  Emacs . . . . . . . . . . . . . . . . . . . . . local Emacs variables
%%
%%  Local Variables:
%%  mode:               outline
%%  outline-regexp:     "\\\\Chapter\\|\\\\Section"
%%  fill-column:        73
%%  eval:               (hide-body)
%%  End:
%%



