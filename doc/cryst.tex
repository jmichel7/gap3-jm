%%%%%%%%%%%%%%%%%%%%%%%%%%%%%%%%%%%%%%%%%%%%%%%%%%%%%%%%%%%%%%%%%%%%%%%%%%%%%
%%
%A  cryst.tex                   GAP documentation                Bettina Eick
%A                                                            & Franz Gaehler
%A                                                            & Werner Nickel
%%
%Y  Copyright 1990-1997,  Lehrstuhl D fuer Mathematik,  RWTH Aachen,  Germany
%%

%%%%%%%%%%%%%%%%%%%%%%%%%%%%%%%%%%%%%%%%%%%%%%%%%%%%%%%%%%%%%%%%%%%%%%%%%
\Chapter{CrystGap--The Crystallographic Groups Package}%

The  CrystGap package provides  functions for the computation with affine
crystallographic groups, in particular  space groups.  Also  provided are
some functions dealing with related  linear matrix groups, such as  point
groups.  For  the definition of the  standard crystallographic notions we
refer to the International Tables \cite{Hah95}, in particular the chapter
by  Wondratschek  \cite{Won95},  and   to the   introductory   chapter in
\cite{BBNWZ78}.  Some  material  can also be  found in  the chapters "The
Crystallographic Groups Library" and "Irreducible Maximal Finite Integral
Matrix   Groups".  The principal   algorithms  used in  this package  are
described in \cite{egn97b}, a preprint  of which in  included in the 'doc'
directory of this package.

CrystGap is  implemented in the {\GAP} language,  and  runs on any system
supporting {\GAP}~3.4.4. The function 'WyckoffLattice', however, requires
the share package  {\sf XGap}, which in  turn runs only  under Unix.  The
functions  described in this   chapter  can be used  only  after  loading
CrystGap with the command 
| 
     gap> RequirePackage( "cryst" ); 
|

CrystGap has been developed by

Bettina Eick\\
:Lehrstuhl D f{\accent127 u}r Mathematik, RWTH Aachen, 
D-52056 Aachen, Germany \\ 
:e-mail\:\ 'Bettina.Eick@math.RWTH-Aachen.de'

Franz G{\accent127 a}hler \\
:Centre de Physique Th{\accent19 e}orique, Ecole Polytechnique, 
F-91128 Palaiseau, France\\
:e-mail\:\ 'gaehler@pth.polytechnique.fr'

Werner Nickel \\
:School of Mathematical and Computational Sciences, 
 University of St~Andrews, \\
:St~Andrews, Fife KY16 9SS, Scotland \\
:e-mail\:\ 'werner@dcs.st-and.ac.uk'

Please send bug reports,  suggestions and other  comments to any of these
e-mail addresses.

The   first and  third  authors acknowledge  financial  support from  the
Graduiertenkolleg {\it Ana\-lyse und Konstruktion in der Mathematik}. The
second author   was supported  by the  Swiss  Bundesamt  f{\accent127 u}r
Bildung und Wissenschaft  in the framework  of the  HCM programme of  the
European   Community.  This collaboration was in   part  made possible by
financial support from the HCM project {\it Computational Group Theory}.

%%%%%%%%%%%%%%%%%%%%%%%%%%%%%%%%%%%%%%%%%%%%%%%%%%%%%%%%%%%%%%%%%%%%%%%%%
\Section{Crystallographic Groups}

An  affine crystallographic group  <G> is a  subgroup of the group of all
Euclidean motions of $d$-dimensional  space,  with the property that  its
subgroup  <T> of   all pure  translations  is  a  freely abelian,  normal
subgroup of  <G>,  which has rank at  most  equal to <d>,   and which has
finite index in <G>.
 
In this package, the term *CrystGroup* always  refers to such an *affine*
crystallographic group. Linear matrix groups, whether crystallographic or
not,  will carry  different  designations  (see  below).  CrystGroups are
represented as special matrix groups, whose  elements are affine matrices
of the form
|
                     [ M 0 ]
                     [ t 1 ]
|
acting on row vectors $(x,1)$ from the right. Note that this is different
from the crystallographic convention, where matrices usually act from the
left on column vectors (see also  "The Crystallographic Groups Library").
We have  adopted this convention  to maintain compatibility with the rest
of {\GAP}.

The ``linear\" parts <M> of the elements of a CrystGroup <G> generate the
*point  group*  <P>   of  <G>,  which   is isomorphic   to  the  quotient
<G>/<T>. There is a natural homomorphism from <G> to <P>, whose kernel is
<T>.   The  translation vectors of the   elements of <T>  generate a free
Z-module <L>, called the *translation  lattice* of <G>.  CrystGroups  can
be defined  with respect to any  basis of Euclidean space, but internally
most computations will be done in a  basis which contains  a basis of <L>
(see "More about Crystallographic Groups").

CrystGroups carry  a special  operations record 'CrystGroupOps',  and are
identified with  a tag  'isCrystGroup'.  CrystGroups must  be constructed
with  a call to   'CrystGroup'  (see "CrystGroup")  which  sets  the  tag
'isCrystGroup'  to    'true',   and sets     the  operations  record   to
'CrystGroupOps'.

*Warning\:* The groups in {\GAP}\' s crystallographic groups library (see
"The Crystallographic Groups Library"),  whether they are extracted  with
'SpaceGroup'  or  'TransposedSpaceGroup', are  *not*  CrystGroups in  the
sense of this package,  because CrystGroups have different record entries
and  a  different  operations record.  However,  a group  extracted  with
'TransposedSpaceGroup' from that library can be converted to a CrystGroup
by a call to 'CrystGroup' (see "CrystGroup").

%%%%%%%%%%%%%%%%%%%%%%%%%%%%%%%%%%%%%%%%%%%%%%%%%%%%%%%%%%%%%%%%%%%%%%%%%
\Section{Space Groups}

A CrystGroup which has  a translation subgroup of full  rank is called  a
*space group*. Certain functions are available only for space groups, and
not  for general CrystGroups, notably all  functions dealing with Wyckoff
positions (see "Wyckoff Positions").

Space  groups which are equivalent  under conjugation in the affine group
(shortly\:\ affine  equivalent  space groups)  are said  to belong to the
same *space group type*.  As is well known, in three dimensions there are
219 such space group  types (if only  conjugation by transformations with
positive determinant is allowed, there are 230).

Representatives of all space group types in dimensions  2, 3 and 4 can be
obtained from  the  crystallographic groups  library contained in  {\GAP}
(see "The Crystallographic Groups  Library"). They must be extracted with
the  function 'CrystGroup', and  not with the  usual extraction functions
'SpaceGroup' and 'TransposedSpaceGroup'  of that library, as these latter
functions return groups which do not have an operations record that would
allow  to  compute  with  them.  'CrystGroup'  accepts  exactly the  same
arguments as 'SpaceGroup' and 'TransposedSpaceGroup'. It returns the same
group as 'TransposedSpaceGroup', but  equipped with a  working operations
record.

Space   group types   (and  thus  space   groups)   are classified   into
$\Z$-classes and  $\Q$-classes.  Two  space  groups belong   to  the same
$\Z$-class  if  their  point  groups, expressed   in  a   basis of  their
respective   translation   lattices,   are   conjugate  as  subgroups  of
$GL(d,\Z)$. If the point groups are conjugate  as subgoups of $GL(d,\Q)$,
the  two space groups  are  said  to  be  in  the same $\Q$-class.   This
provides also  a classification of point  groups  (expressed in a lattice
basis, i.e., integral point groups) into $\Z$-classes and $\Q$-classes.

For a given finite integral matrix  group <P>, representing a point group
expressed in a  lattice basis, a  set of representative  space groups for
each space  group type  in the $\Z$-class  of <P>  can  be  obtained with
'SpaceGroupsPointGroup' (see "SpaceGroupsPointGroup").  If, moreover, the
normalizer of <P>  in  $GL(d,\Z)$ is known (see  "NormalizerGL"), exactly
one   representative  is     obtained  for  each    space    group  type.
Representatives of all  $\Z$-classes of maximal irreducible finite  point
groups are contained in a {\GAP} library (see "Irreducible Maximal Finite
Integral  Matrix  Groups") in  all dimensions   up to 11,   and  in prime
dimensions up to  23.   For some  other  dimensions, at  least $\Q$-class
representatives are available.
 
Important information about  a space group is   contained in its  *affine
normalizer* (see "AffineNormalizer"), which is the normlizer of the space
group   in the affine   group.  In  a way,  the  affine normalizer can be
regarded as the symmetry of the space group.

*Warning\:* Groups which are called  space groups  in this manual  should
not  be  confused  with  groups  extracted  with  'SpaceGroup'  from  the
crystallographic   groups library    (see  "The  Crystallographic  Groups
Library"). The latter are not CrystGroups in the sense of this package.

%%%%%%%%%%%%%%%%%%%%%%%%%%%%%%%%%%%%%%%%%%%%%%%%%%%%%%%%%%%%%%%%%%%%%%%%%
\Section{More about Crystallographic Groups}

In  this  section  we  describe  how  a  CrystGroup  <G>  is  represented
internally.  The casual  user can skip this section  in  a first reading.
Although the generators of a CrystGroup can be  specified with respect to
any basis,  most computations are  done internally in a special, standard
basis, which is stored  in 'G.internalBasis'. The results  are translated
into the user-specified basis only afterwards. 'G.internalBasis' consists
of a (standard) basis of the translation lattice of <G>, complemented, if
necessary, with  suitable standard basis vectors.  The standard  basis of
the translation lattice is stored in 'G.translations'.

As soon as 'G.internalBasis' has been determined, both the CrystGroup <G>
and its point group <P> obtain a component 'internalGenerators'.  For the
point group <P>, the  component 'P.internalGenerators' contains a set  of
generators of <P>, expressed with  respect to the 'internalBasis' of <G>,
whereas for  the  CrystGroup  <G> the   component  'G.internalGenerators'
contains a set of homomorphic preimages of 'P.internalGenerators' in <G>,
also expressed in the 'internalBasis' of <G>. Thus 'G.internalGenerators'
does  not  contain any translation  generators.  These are   easy to add,
however\:\ With respect   to the internal  basis,   the translations  are
generated by the first <k> standard basis vectors, where  <k> is the rank
of the translation lattice.

Note that  the  'internalGenerators'  of both   a point  group <P> and  a
CrystGroup  <G> may be changed by  some  functions, notably by 'FpGroup'.
Thus they need  not have  any obvious  connection  to  'P.generators' and
'G.generators',  respectively.  Internal record  entries of  a CrystGroup
should *never be changed* by the user.

%%%%%%%%%%%%%%%%%%%%%%%%%%%%%%%%%%%%%%%%%%%%%%%%%%%%%%%%%%%%%%%%%%%%%%%%%
\Section{CrystGroup}

'CrystGroup( <matgroup> )'

'CrystGroup( <generating matrices> )'

'CrystGroup( <list of generators>, <identity> )'

'CrystGroup( <integers> )'

'CrystGroup( <string> )'

'CrystGroup' accepts as arguments either a group of affine matrices, or a
list of generating affine  matrices, or an  argument identifying  a space
group from the crystallographic  groups library, i.e., a  list of  two or
five  integers,  or a string    containing a Hermann-Mauguin  symbol, and
converts it into a CrystGroup in the sense of this package.  'CrystGroup'
tests whether the generators are indeed affine matrices.

%%%%%%%%%%%%%%%%%%%%%%%%%%%%%%%%%%%%%%%%%%%%%%%%%%%%%%%%%%%%%%%%%%%%%%%%%
\Section{IsCrystGroup}

'IsCrystGroup( <G> )'

tests whether 'G.isCrystGroup' is present and 'true'. 'G.isCrystGroup' is
set by 'CrystGroup'.

%%%%%%%%%%%%%%%%%%%%%%%%%%%%%%%%%%%%%%%%%%%%%%%%%%%%%%%%%%%%%%%%%%%%%%%%%
\Section{PointGroup}
\index{PointGroup!of a CrystGroup}

'PointGroup( <G> )'

extracts  the  point  group  <P>  of a    space group  <G>,   binds it to
'G.pointGroup', and returns it.  It also determines the homomorphism from
<G> to <P>, and binds it to 'G.pointHomom'. A  point group <P> has always
a   component  'P.isPointGroup'    set  to   'true',    and  a  component
'P.crystGroup' containing the CrystGroup from which it was constructed.

%%%%%%%%%%%%%%%%%%%%%%%%%%%%%%%%%%%%%%%%%%%%%%%%%%%%%%%%%%%%%%%%%%%%%%%%%
\Section{TranslationsCrystGroup}

'TranslationsCrystGroup( <G> )'

determines a  basis  of  the translation lattice   of <G>,  binds   it to
'G.translations', and returns it.  Note  that this translation lattice is
always invariant under the  point group <P>  of <G>.  If 'G.translations'
is not yet present, a finite presentation  of <P> needs to be determined.
A basis  of the translation lattice  can also be  added by the user, with
'AddTranslationsCrystGroup' (see "AddTranslationsCrystGroup").

*Warning\:* The  component 'G.translations' must *never* be  set by hand.
The functions 'TranslationsCrystGroup'  and  'AddTranslationsCrystGroups'
have important (and wanted) side effects.

%%%%%%%%%%%%%%%%%%%%%%%%%%%%%%%%%%%%%%%%%%%%%%%%%%%%%%%%%%%%%%%%%%%%%%%%%
\Section{AddTranslationsCrystGroup}%
\index{TranslationsCrystGroup!add translations}

'AddTranslationsCrystGroup( <G>, <basis> )'

Since  'TranslationsCrystGroup' (see   "TranslationsCrystGroup") needs  a
presentation of the point group,  the computation of 'G.translations' can
be rather time consuming. If a basis of the translation lattice is known,
'AddTranslationsCrystGroup'  can  be used   to add  this  knowledge  to a
CrystGroup.   If 'G.translations'  is already  known,  its  value is kept
without further  notice.  It is  the responsibility of  the user that the
basis handed  over to 'AddTranslationsCrystGroup' is  a  correct basis of
the     translation   lattice.    In  case   of    doubt,   the  function
'CheckTranslations'   (see "CheckTranslations")  can  be   used  to check
whether the basis added was indeed correct.

*Warning\:* The  component 'G.translations' must *never* be  set by hand.
The functions 'TranslationsCrystGroup'  and  'AddTranslationsCrystGroups'
have important (and wanted) side effects.

%%%%%%%%%%%%%%%%%%%%%%%%%%%%%%%%%%%%%%%%%%%%%%%%%%%%%%%%%%%%%%%%%%%%%%%%%
\Section{CheckTranslations}%
\index{TranslationsCrystGroup!check translations}

'CheckTranslations( <G> )'

checks whether 'G.translations' is indeed correct. If 'G.translations' is
incorrect, a warning message is printed, otherwise {\GAP} remains silent.
In the case  of an incorrect  translation basis a  new CrystGroup must be
created, and the computations  must be started  afresh, because the wrong
translation  basis  may    have produced   wrong  information components.
'CheckTranslations'  is  useful  if  a   basis   has    been  added  with
'AddTranslationsCrystGroup',  and doubts  arise  later whether the  basis
added was correct.

%%%%%%%%%%%%%%%%%%%%%%%%%%%%%%%%%%%%%%%%%%%%%%%%%%%%%%%%%%%%%%%%%%%%%%%%%
\Section{ConjugatedCrystGroup}%
\index{CrystGroup!conjugated}

'ConjugatedCrystGroup( <G>, <c> )'

returns a new CrystGroup which is a conjugate of <G>.  The conjugator <c>
can either be a $d$-dimensional linear matrix (which then is complemented
with the zero translation), or  a ($d+1$)-dimensional affine matrix.  The
generators are conjugated as $g^c  = c*g*c^{-1}$.  Some components  which
are  bound in  <G> are   copied  and  translated to   the  new basis,  in
particular   'G.generators',   'G.translations',  'G.internalBasis',  and
'G.wyckoffPositons'.  If 'G.internalBasis' is bound,

'ConjugatedCrystGroup( G, G.internalBasis )'

returns a CrystGroup whose translation lattice (of rank $k$) is generated
by  the first  $k$  rows  of  the identity matrix. 'ConjugatedCrystGroup'
allows as input only a parent CrystGroup.

%%%%%%%%%%%%%%%%%%%%%%%%%%%%%%%%%%%%%%%%%%%%%%%%%%%%%%%%%%%%%%%%%%%%%%%%%
\Section{FpGroup for point groups}%
\index{FpGroup!for point groups}

'FpGroup( <P> )'

computes a finite  presentation of the point  group <P>, and binds  it to
'P.fpGroup'.  If <P>  (and thus its  CrystGroup 'G  \:= P.crystGroup') is
solvable, a power-commutator presentation  is returned. 

*Warning\:*   If <P>  is  solvable, the     abstract generators are   not
necessarily   isomorphic   images  of  'P.generators'   (see  "More about
Crystallographic Groups").

%%%%%%%%%%%%%%%%%%%%%%%%%%%%%%%%%%%%%%%%%%%%%%%%%%%%%%%%%%%%%%%%%%%%%%%%%
\Section{FpGroup for CrystGroups}%
\index{FpGroup!for CrystGroups}

'FpGroup( <G> )'

computes a finite  presentation of  the CrystGroup <G>,  and  binds it to
'G.fpGroup'.   If  the   point  group   (and thus  <G>)  is  solvable,  a
power-commutator presentation is  returned.  The  presentation  is always
an extension  of the presentation of the  point  group (which is computed
if necessary).   

*Warning\:*  The abstract  generators  of    the  presentation are    not
necessarily   isomorphic   images   of 'G.generators'  (see   "More about
Crystallographic Groups").

%%%%%%%%%%%%%%%%%%%%%%%%%%%%%%%%%%%%%%%%%%%%%%%%%%%%%%%%%%%%%%%%%%%%%%%%%
\Section{MaximalSubgroupsRepresentatives}%
\index{MaximalSubgroupsRepresentatives!for CrystGroups}

'MaximalSubgroupsRepresentatives( S, \"translationEqual\", [, ind ] )'

'MaximalSubgroupsRepresentatives( S, \"classEqual\", ind )'

'MaximalSubgroupsRepresentatives( S, ind )'

returns a list of conjugacy class representatives of maximal subgroups of
the CrystGroup <S>. If <ind> is present, which must  be a prime or a list
of primes, only those subgroups are returned whose index  is a power of a
prime contained  in or equal  to <ind>.  If the flag ``translationEqual\"
is present, only those subgroups are returned which are translation-equal
(translationengleich) with <S>.  If the <flag> ``classEqual\" is present,
only  those subgroups  are  return which are  class-equal (klassengleich)
with <S>. <ind> is optional only if the flag ``latticeEqual\" is present. 
In all other cases, <ind> is required.

%%%%%%%%%%%%%%%%%%%%%%%%%%%%%%%%%%%%%%%%%%%%%%%%%%%%%%%%%%%%%%%%%%%%%%%%%
\Section{IsSpaceGroup}

'IsSpaceGroup( <S> )'

determines   whether   the  CrystGroup   <S>  is   a  space  group   (see
"Crystallographic Groups").

%%%%%%%%%%%%%%%%%%%%%%%%%%%%%%%%%%%%%%%%%%%%%%%%%%%%%%%%%%%%%%%%%%%%%%%%%
\Section{IsSymmorphicSpaceGroup}

'IsSymmorphicSpaceGroup( <S> )'

determines whether the space  group <S> is  symmorphic.  A space group is
called *symmorphic* if it  is equivalent to a  semidirect product  of its
point group with its translation subgroup.


%%%%%%%%%%%%%%%%%%%%%%%%%%%%%%%%%%%%%%%%%%%%%%%%%%%%%%%%%%%%%%%%%%%%%%%%%
\Section{SpaceGroupsPointGroup}

'SpaceGroupsPointGroup( <P> )'

'SpaceGroupsPointGroup( <P>, <normalizer elements> )'

where $P$  is any finite   subgroup of $GL(d,Z),$  returns  a list of all
space groups  with   point  group  $P,$ up   to  conjugacy in  the   full
translation group of Euclidean space. All these space groups are returned
as  CrystGroups  in standard  representation.   If a  second  argument is
present,  which must be a  list of elements of the  normalizer  of $P$ in
$GL(d,Z),$ only space  groups  inequivalent under conjugation with  these
elements are returned.  If  these normalizer elements, together with $P,$
generate  the full  normalizer  of $P$  in  $GL(d,Z)$,  then  exactly one
representative of each space group type is obtained.

%%%%%%%%%%%%%%%%%%%%%%%%%%%%%%%%%%%%%%%%%%%%%%%%%%%%%%%%%%%%%%%%%%%%%%%%%
\Section{Wyckoff Positions}%
\index{WyckoffBasis}\index{WyckoffTranslation}\index{WyckoffStabilizer}%
\index{WyckoffPosClass}\index{WyckoffSpaceGroup}\index{IsWyckoffPosition}

A  Wyckoff position  of  a space group  <G>  is an equivalence   class of
points in  Euclidean  space,  having  stabilizers  which  are   conjugate
subgroups of <G>.  Apart from a subset of lower dimension, which contains
points with  even bigger  stabilizers, a Wyckoff   position consists of a
<G>-orbit of  some affine subspace  <A>. A Wyckoff position $W$ therefore
can  be  specified  by a  representative  affine  subspace  $A$  and  its
stabilizer subgroup. In CrystGap,  a Wyckoff position <W> is  represented 
as a record with the following components\:

     'W.basis': \\
          Basis of the linear space $L$  parallel to $A.$  This basis is 
          also a basis of the  intersection of $L$ with  the translation 
          lattice of $S$. \\
          Can be extracted with 'WyckoffBasis( W )'.

     'W.translation': \\  
          'W.translation' is such that 'A = L + W.translation'. \\
          Can be extracted with 'WyckoffTranslation( W )'.

     'W.stabilizer': \\   
          The stabilizer subgroup of any generic point in $A.$ \\
          Can be extracted with 'WyckoffStabilizer( W )'.

     'W.class': \\      
          Wyckoff  positions  carry  the same  class label  if and  only
          if their  stabilizers have  point groups  which  are conjugate 
          subgroups of the point group of $S.$ \\
          Can be extracted with 'WyckoffPosClass( W )'.

     'W.spaceGroup': \\      
          The space group of which it is a Wyckoff position. \\
          Can be extracted with 'WyckoffSpaceGroup( W )'.


     'W.isWyckoffPosition': \\
          A flag identifying the record as a Wyckoff position.  It is set
          to true.\\ 
          Can be tested with 'IsWyckoffPosition( W )'.

     'W.operations': \\
          The  operations  record  of  a Wyckoff  position.  It currently
          contains only a Print function. 

%%%%%%%%%%%%%%%%%%%%%%%%%%%%%%%%%%%%%%%%%%%%%%%%%%%%%%%%%%%%%%%%%%%%%%%%%
\Section{WyckoffPositions}

'WyckoffPositions( <G> )'

returns the  list of all  Wyckoff positions  of the  space group  <G>.

%%%%%%%%%%%%%%%%%%%%%%%%%%%%%%%%%%%%%%%%%%%%%%%%%%%%%%%%%%%%%%%%%%%%%%%%%
\Section{WyckoffPositionsByStabilizer}

'WyckoffPositionsByStabilizer( <G>, <U> )',

where <G>  is a space group  and <U> a subgroup  of the  point group or a
list  of such subgroups,    determines  only the Wyckoff positions   (see
"WyckoffPositions")  having  a  representative   affine   subspace  whose
stabilizer has a point  group equal to  the subgroup <U> or  contained in
the list <U>, respectively.

%%%%%%%%%%%%%%%%%%%%%%%%%%%%%%%%%%%%%%%%%%%%%%%%%%%%%%%%%%%%%%%%%%%%%%%%%
\Section{WyckoffPositionsQClass}

'WyckoffPositionsQClass( <G>, <S> )'

For space groups with   larger point groups,  most  of  the time in   the
computation  of  Wyckoff  positions  (see  "WyckoffPositions")  is  spent
computing the subgroup lattice of the point  group.  If Wyckoff positions
are needed for several space  groups which are in  the same Q class,  and
therefore have   isomorphic point groups, one  can  avoid recomputing the
same subgroup lattice  for each of them as  follows.  For the computation
of the Wyckoff positions of the first space group  <S> one uses a call to
'WyckoffPositions'.  For  the remaining space  groups, <S> is then passed
as a second argument to 'WyckoffPositionsQClass( <G>,  <S> )', which uses
some of the results already obtained for <S>.

%%%%%%%%%%%%%%%%%%%%%%%%%%%%%%%%%%%%%%%%%%%%%%%%%%%%%%%%%%%%%%%%%%%%%%%%%
\Section{WyckoffOrbit}

'WyckoffOrbit( W )'

takes a Wyckoff position <W> (see "Wyckoff Positions") and returns a list
of Wyckoff positions which  are  different representations of  <W>,  such
that the representative affine subspaces of these representations form an
orbit under the space group <G> of <W>, modulo lattice translations.

%%%%%%%%%%%%%%%%%%%%%%%%%%%%%%%%%%%%%%%%%%%%%%%%%%%%%%%%%%%%%%%%%%%%%%%%%
\Section{WyckoffLattice}

'WyckoffLattice( <G> )'

If  a point $x$ in a  Wyckoff position $W_1$ has  a stabilizer which is a
subgroup of the stabilizer of some point $y$ in a Wyckoff position $W_2,$
then the  closure of $W_1$ will contain  $W_2$. These incidence relations
are best represented in a  graph.  'WyckoffLattice( <G> )' determines and
displays this graph using XGAP (note that  XGAP runs only under Unix plus
the  X   Window  System). Each    Wyckoff position is   represented  by a
vertex. If $W_1$ contains $W_2$, its vertex is placed below that of $W_2$
(i.e., Wyckoff  positions with bigger stabilizers  are placed higher up),
and the two are connected, either directly  (if there is no other Wyckoff
position  in between) or indirectly. With  the left mouse button and with
the XGAP 'CleanUp' menu it is possible to change the layout  of the graph
(see the XGAP manual).  When clicking with  the right mouse  button  on a
vertex,  a pop up  menu appears,  which  allows to  obtain the  following
information  about the  representative  affine  subspace  of the  Wyckoff
position\:

     'StabDim'\:: \\
          Dimension of the affine subspace of stable points.

     'StabSize'\:: \\
          Size of the stabilizer subgroup.

     'ClassSize'\:: \\
          Number of Wyckoff positions  having a  stabilizer  whose  point 
          group is in the same subgroup conjugacy class.

     'IsAbelian', 'IsCyclic', 'IsNilpotent', 'IsPerfect', 'IsSimple',
     'IsSolvable'\:: \\
          Information about the stabilizer subgroup.

     'Isomorphism'\:: \\
          Isomorphism type of the  stabilizer subgroup.  Works  only  for
          small sizes.

     'ConjClassInfo'\:: \\
          Prints  (in the {\GAP} window)  information  about each  of the 
          conjugacy  classes  of the  stabilizer,  namely the order,  the 
          trace and the determinant of its elements,  and the size of the  
          conjugacy class. Note that trace refers here only to the  trace
          of the point group part,  without the trailing 1 of the  affine
          matrix.

     'Translation'\:: \\ 
          The representative point of the affine subspace.

     'Basis'\:: \\
          The basis of the linear space parallel to the affine subspace.


%%%%%%%%%%%%%%%%%%%%%%%%%%%%%%%%%%%%%%%%%%%%%%%%%%%%%%%%%%%%%%%%%%%%%%%%%
\Section{NormalizerGL}%
\index{normalizer!in $GL(d,\Z)$}

'NormalizerGL( G )',

where <G>  is a finite subgroup of  <GL(d,Z)>, returns  the normalizer of
<G> in <GL(d,Z)>. At present, this  function is available only for groups
which are the point   group of a   CrystGroup   extracted from  the space
group library.

%%%%%%%%%%%%%%%%%%%%%%%%%%%%%%%%%%%%%%%%%%%%%%%%%%%%%%%%%%%%%%%%%%%%%%%%%
\Section{CentralizerGL}%
\index{centralizer!in $GL(d,\Z)$}

'CentralizerGL( G )',

where <G> is  a finite subgroup of <GL(d,Z)>,  returns the centralizer of
<G> in <GL(d,Z)>. At present, this function is  available only for groups
which  are the  point group  of a  CrystGroup   extracted from the  space
group library.

%%%%%%%%%%%%%%%%%%%%%%%%%%%%%%%%%%%%%%%%%%%%%%%%%%%%%%%%%%%%%%%%%%%%%%%%%
\Section{PointGroupsBravaisClass}

'PointGroupsBravaisClass( B )'

'PointGroupsBravaisClass( B [, norm ] )'

where  <B> is   a  finite  integral matrix  group,   returns  a  list  of
representatives of those conjugacy classes  of subgroups of <B> which are
in the same Bravais class  as <B>. These  representatives are returned as
parent groups, not   subgroups.  If  <B>  is a  Bravais   group, the list
contains a  representative of  each point  group in the  Bravais class of
<B>. If a second argument is present, which must be a list of elements of
the normalizer  of <B> in  $GL(d,\Z)$, only  subgroups inequivalent under
conjugation with these elements are returned.

%%%%%%%%%%%%%%%%%%%%%%%%%%%%%%%%%%%%%%%%%%%%%%%%%%%%%%%%%%%%%%%%%%%%%%%%%
\Section{TranslationNormalizer}%
\index{normalizer!in translation group}

'TranslationNormalizer( S )'

returns the normalizer  of the space group  <S>  in the  full translation
group. At present,  this function is  implemented only  for space groups,
not for general  CrystGroups.  The translation normalizer <TN> of <S> may
contain   a  continuous subgroup  <C>.  A  basis  of  the  space  of such
continuous translations  is bound  in 'TN.continuousTranslations'.  Since
this subgroup is  not finitely generated,   it is *not* contained in  the
group  generated by  'TN.generators'. Properly  speaking, the translation
normalizer is the span of <TN> and <C> together.

%%%%%%%%%%%%%%%%%%%%%%%%%%%%%%%%%%%%%%%%%%%%%%%%%%%%%%%%%%%%%%%%%%%%%%%%%
\Section{AffineNormalizer}%
\index{normalizer!in affine group}

'AffineNormalizer( S )'

returns   the  affine normalizer of    the space group   <S>.  The affine
normalizer contains the translation  normalizer as a subgroup.  Similarly
as  with   'TranslationNormalizer',   the  subgroup  <C>   of  continuous
translations, which  is not finitely generated,  is not part of the group
that  is  returned.   However,  a   basis of  the    space of  continuous
translations is bound in the component 'continuousTranslations'.

At present, this function is available only for for space groups, not for
general 'CrystGroups'.  Moreover, the 'NormalizerGL' (see "NormalizerGL")
of the point group of <S> must be know, which currently  is the case only
for CrystGroups  extracted from the  space  group library.

%%%%%%%%%%%%%%%%%%%%%%%%%%%%%%%%%%%%%%%%%%%%%%%%%%%%%%%%%%%%%%%%%%%%%%%%%
\Section{AffineInequivalentSubgroups}

'AffineInequivalentSubgroups( sub )'

takes as input a  list of subgroups with common  parent space  group <S>,
and returns  a sublist of those  which are affine inequivalent. For this,
the affine  normalizer of <S> is required,  which currently  is available
only if <S> is a space group extracted from the space groups library.

%%%%%%%%%%%%%%%%%%%%%%%%%%%%%%%%%%%%%%%%%%%%%%%%%%%%%%%%%%%%%%%%%%%%%%%%%
\Section{Other functions for CrystGroups}%
\index{CrystGroups!other functions}

In the operations  record  of  a  'CrystGroup' many of  the  usual {\GAP}
functions are replaced with a 'CrystGroup'  specific implementation.  For
other  functions   the default   implementation   can  be  used.    Since
'CrystGroups' are  matrix groups, all functions  which  work for a finite
matrix group should work also for a finite  'CrystGroup' (i.e., one which
contains no pure translations).  Of  course,  functions which require   a
*finite*   group as  input  will  work   only for finite   'CrystGroups'.
Following is a (probably not exhaustive) list of functions that are known
to work for also for *infinite* 'CrystGroups'.

|
     in
     Parent, IsParent, Group, IsGroup
     Subgroup, IsSubgroup, AsSubgroup, Index 
     Centralizer, Centre, Normalizer 
     Closure, NormalClosure
     Intersection, NormalIntersection
     ConjugacyClassSubgroups, ConjugateSubgroups
     DerivedSubgroup, CommutatorSubgroup, Core 
     DerivedSeries, SubnormalSeries
     FactorGroup, CommutatorFactorGroup
     ConjugateSubgroup, TrivialSubgroup 
     IsAbelian, IsCentral, IsTrivial
     IsNormal, IsSubnormal, IsPerfect, IsSolvable 
|


The following functions work for   'CrystGroups' *provided* the  subgroup
<H> has *finite index* in <G>.  The elements  of the resulting domain are
given in ascending order (with respect to an ad hoc, but fixed ordering).

|
     Cosets( G, H )
     RightCosets( G, H )
     LeftCosets( G, H )
|

The   following  functions    dealing  with group   operations  work  for
'CrystGroups'  provided  the orbits of the   action  are *finite*.  Since
'CrystGroups'  are   not  finite   in general,  this  is    a non-trivial
requirement, and so some care is needed.

|
     Orbit( G, d, opr )
     Orbits( G, D, opr )
     OrbitLengths( G, D, opr )
     Stabilizer( G, d, opr )
     RepresentativeOperation( G, d, e, opr )
     RepresentativesOperation( G, d, opr )
|

The following functions have  a 'CrystGroup' specific implementation, but
work for *finite* 'CrystGroups' only\:

|
     Elements( G )
     ConjugacyClasses( G )
     PermGroup( G )
     SylowSubgroup( G, p )
|

%%%%%%%%%%%%%%%%%%%%%%%%%%%%%%%%%%%%%%%%%%%%%%%%%%%%%%%%%%%%%%%%%%%%%%%%%
\Section{Color Groups}

Elements of a  color  group <C> are colored   in the following  way.  The
elements having the same color as 'C.identity' form a subgroup <H>, which
has finite index <n> in  <C>.  <H> is  called the 'ColorSubgroup' of <C>.
Elements of <C> have the same color if and  only if they  are in the same
right coset of <H> in <C>.  A  fixed list of right  cosets of <H> in <C>,
called  'ColorCosets',  therefore determines  a  labelling of the colors,
which runs  from 1 to  <n>.  Elements of <H> by  definition have color 1,
i.e., the  coset with representative   'C.identity' is always  the  first
element of 'ColorCosets'.  Right multiplication by a fixed element <g> of
<C> induces a  permutation <p(g)>  of the colors  of  the parent of  <C>.
This defines a natural homomorphism of <C>  into the permutation group of
degree  <n>.   The  image    of    this  homomorphism  is    called   the
'ColorPermGroup'  of <C>,   and  the  homomorphism to  it is  called  the
'ColorHomomorphism' of <C>.


%%%%%%%%%%%%%%%%%%%%%%%%%%%%%%%%%%%%%%%%%%%%%%%%%%%%%%%%%%%%%%%%%%%%%%%%%
\Section{ColorGroup}

A color group is constructed with

'ColorGroup( G, H ),'

which returns a colored copy of <G>, with color subgroup <H>. <G> must be
a  parent group, and <H> must  be a finite  index subgroup  of <G>. Color
subgroups must be  constructed as subgroups  of color parent  groups, and
not  by coloring  uncolored  subgroups. Subgroups   of color  groups will
inherit the  coloring of  their parent, including   the labelling of  the
colors.

Color groups are identified with a tag  'isColorGroup'.  They always have
a component 'colorSubgroup'.  Color parent groups  moreover always have a
component 'colorCosets', which fixes a labelling of the colors.

Groups which may be colored  include,  in particular, 'CrystGroups',  but
coloring  of any  finite   group,  such as   a  finite  matrix  group  or
permutation group, should work as well.

%%%%%%%%%%%%%%%%%%%%%%%%%%%%%%%%%%%%%%%%%%%%%%%%%%%%%%%%%%%%%%%%%%%%%%%%%
\Section{IsColorGroup}

'IsColorGroup( G )'         

checks whether 'G.isColorGroup' is bound and true.

%%%%%%%%%%%%%%%%%%%%%%%%%%%%%%%%%%%%%%%%%%%%%%%%%%%%%%%%%%%%%%%%%%%%%%%%%
\Section{ColorSubgroup}

'ColorSubgroup( G )'

returns the color subgroup of <G>.

%%%%%%%%%%%%%%%%%%%%%%%%%%%%%%%%%%%%%%%%%%%%%%%%%%%%%%%%%%%%%%%%%%%%%%%%%
\Section{ColorCosets}

'ColorCosets( G )'

returns the color cosets of <G>.

%%%%%%%%%%%%%%%%%%%%%%%%%%%%%%%%%%%%%%%%%%%%%%%%%%%%%%%%%%%%%%%%%%%%%%%%%
\Section{ColorOfElement}

'ColorOfElement( G, elem )'

returns the color of an element.

%%%%%%%%%%%%%%%%%%%%%%%%%%%%%%%%%%%%%%%%%%%%%%%%%%%%%%%%%%%%%%%%%%%%%%%%%
\Section{ColorPermGroup}

'ColorPermGroup( G )'

returns the ColorPermGroup of <G>, which is the permutation group induced
by <G> acting on the colors of the parent of <G>.

%%%%%%%%%%%%%%%%%%%%%%%%%%%%%%%%%%%%%%%%%%%%%%%%%%%%%%%%%%%%%%%%%%%%%%%%%
\Section{ColorHomomorphism}

'ColorHomomorphism( G )'

returns the homomomorphism from <G> to its ColorPermGroup.

%%%%%%%%%%%%%%%%%%%%%%%%%%%%%%%%%%%%%%%%%%%%%%%%%%%%%%%%%%%%%%%%%%%%%%%%%
\Section{Subgroup for color groups}%
\index{Subgroup!for color groups}

If <C> is a color group, 

'Subgroup( C, [elems] )'

returns a colored subgroup of <C>, whereas

'C.operations.UncoloredSubgroup( C, [elems] )'

returns an ordinary, uncolored subgroup.

%%%%%%%%%%%%%%%%%%%%%%%%%%%%%%%%%%%%%%%%%%%%%%%%%%%%%%%%%%%%%%%%%%%%%%%%%
\Section{PointGroup for color CrystGroups}%
\index{PointGroup!for color CrystGroups}

If <C> is a color 'CrystGroup' whose color  subgroup is lattice-equal (or
translationengleich) with <C>, the point group of <C> can consistently be
colored. In that case,

'PointGroup( C )'                             

returns  a colored   point group.  Otherwise,  the  point   group will be
uncolored. An uncolored point group can always be obtained with

'C.operations.UncoloredPointGroup( C )'

%%%%%%%%%%%%%%%%%%%%%%%%%%%%%%%%%%%%%%%%%%%%%%%%%%%%%%%%%%%%%%%%%%%%%%%%%
\Section{Inequivalent colorings of space groups}%
\index{colorings!inequivalent!for space groups}

Two  colorings of    a space  group   <S>   are equivalent    if the  two
'ColorSubgroups' are conjugate in the affine normalizer of <S>.

'AffineInequivalentSubgroups( L )'

where <L> is a list of sub space groups with  a common parent space group
<S>, returns  a  list  of affine  inequivalent  subgroups  from  <L>.  At
present, this routine is  supported only for CrystGroups constructed from
the space group library.

A list of prime index <p> subgroups of <S> (actually, a list of conjugacy
class representatives of such subgroups) can be obtained with

'Filtered( MaximalSubgroupsRepresentatives( S, p ), U -> U.index = p )'

These two routines together therefore allow to determine all inequivalent
colorings of <S> with <p> colors.







