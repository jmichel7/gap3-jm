%%%%%%%%%%%%%%%%%%%%%%%%%%%%%%%%%%%%%%%%%%%%%%%%%%%%%%%%%%%%%%%%%%%%%%%%%%%%%
%%
%A  matint.msk                  GAP documentation            Alexander Hulpke
%A                                                           Thomas Breuer
%A                                                           Rob Wainwright
%%
%Y  (C) 1998 School Math and Comp. Sci., University of St.  Andrews, Scotland
%Y  Copyright (C) 2002 The GAP Group
%% Ported to GAP3 J. Michel 2005
%%
\Chapter{Integral matrices and lattices}

This is a subset of the functions available in GAP4, ported to GAP3 to be
used by CHEVIE.

%%%%%%%%%%%%%%%%%%%%%%%%%%%%%%%%%%%%%%%%%%%%%%%%%%%%%%%%%%%%%%%%%%%%%%%%%%%%%
\Section{NullspaceIntMat}
'NullspaceIntMat( <mat> )'

If <mat> is a matrix with integral entries, this function returns a list of
vectors  that forms  a basis  of the  integral nullspace  of <mat>, i.e. of
those vectors in the nullspace of <mat> that have integral entries.

|    gap> mat:=[[1,2,7],[4,5,6],[7,8,9],[10,11,19],[5,7,12]];;
    gap> NullspaceMat(mat);
    [ [ 1, 0, 3/4, -1/4, -3/4 ], [ 0, 1, -13/24, 1/8, -7/24 ] ]
    gap> NullspaceIntMat(mat);
    [ [ 1, 18, -9, 2, -6 ], [ 0, 24, -13, 3, -7 ] ] |

%%%%%%%%%%%%%%%%%%%%%%%%%%%%%%%%%%%%%%%%%%%%%%%%%%%%%%%%%%%%%%%%%%%%%%%%%%%%%
\Section{SolutionIntMat}
'SolutionIntMat( <mat>, <vec> )'

If <mat> is a matrix with integral entries and <vec> a vector with integral
entries,  this function returns a vector <x> with integer entries that is a
solution  of the equation '<x>\*<mat>=<vec>'. It returns 'false' if no such
vector exists.

|    gap> mat:=[[1,2,7],[4,5,6],[7,8,9],[10,11,19],[5,7,12]];;
    gap> SolutionMat(mat,[95,115,182]);
    [ 47/4, -17/2, 67/4, 0, 0 ]
    gap> SolutionIntMat(mat,[95,115,182]);
    [ 2285, -5854, 4888, -1299, 0 ] |

%%%%%%%%%%%%%%%%%%%%%%%%%%%%%%%%%%%%%%%%%%%%%%%%%%%%%%%%%%%%%%%%%%%%%%%%%%%%%
\Section{SolutionNullspaceIntMat}
'SolutionNullspaceIntMat( <mat>, <vec> )'

This  function returns  a list  of length  two, its  first entry  being the
result  of a call  to 'SolutionIntMat' with  same arguments, the second the
result of 'NullspaceIntMat' applied to the matrix <mat>. The calculation is
performed faster than if two separate calls would be used.

|    gap> mat:=[[1,2,7],[4,5,6],[7,8,9],[10,11,19],[5,7,12]];;
    gap> SolutionNullspaceIntMat(mat,[95,115,182]);
    [ [ 2285, -5854, 4888, -1299, 0 ],
      [ [ 1, 18, -9, 2, -6 ], [ 0, 24, -13, 3, -7 ] ] ]|

%%%%%%%%%%%%%%%%%%%%%%%%%%%%%%%%%%%%%%%%%%%%%%%%%%%%%%%%%%%%%%%%%%%%%%%%%%%%%
\Section{BaseIntMat}
'BaseIntMat( <mat> )'

If <mat> is a matrix with integral entries, this function returns a list of
vectors  that forms a basis of the integral row space of <mat>, i.e. of the
set of integral linear combinations of the rows of <mat>.

|    gap> mat:=[[1,2,7],[4,5,6],[10,11,19]];;
    gap> BaseIntMat(mat);
    [ [ 1, 2, 7 ], [ 0, 3, 7 ], [ 0, 0, 15 ] ]|

%%%%%%%%%%%%%%%%%%%%%%%%%%%%%%%%%%%%%%%%%%%%%%%%%%%%%%%%%%%%%%%%%%%%%%%%%%%%%
\Section{BaseIntersectionIntMats}
'BaseIntersectionIntMats( <m>, <n> )'

If  <m> and <n> are matrices with integral entries, this function returns a
list  of vectors that forms a basis of the intersection of the integral row
spaces of <m> and <n>.

|    gap> nat:=[[5,7,2],[4,2,5],[7,1,4]];;
    gap> BaseIntMat(nat);
    [ [ 1, 1, 15 ], [ 0, 2, 55 ], [ 0, 0, 64 ] ]
    gap> BaseIntersectionIntMats(mat,nat);
    [ [ 1, 5, 509 ], [ 0, 6, 869 ], [ 0, 0, 960 ] ]|

%%%%%%%%%%%%%%%%%%%%%%%%%%%%%%%%%%%%%%%%%%%%%%%%%%%%%%%%%%%%%%%%%%%%%%%%%%%%%
\Section{ComplementIntMat}
'ComplementIntMat( <full>, <sub> )'

Let  <full> be a list of integer  vectors generating an Integral module <M>
and  <sub>  a  list  of  vectors  defining  a  submodule <S>. This function
computes  a free basis for <M> that  extends <S>, that is, if the dimension
of  <S> is <n> it determines a  basis $\{b_1,\ldots,b_m\}$ for <M>, as well
as  <n> integers  $x_i$ such  that $x_i\mid  x_j$ for  $i\< j$  and the <n>
vectors $s_i\:=x_i\cdot b_i$ for $i=1,\ldots,n$ form a basis for <S>.

It returns a record with the following components\:

'complement':\\
   the vectors $b_{n+1}$ up to $b_m$ (they generate a complement to <S>).

'sub':\\
   the vectors $s_i$ (a basis for <S>).

'moduli':\\
   the factors $x_i$.

|    gap> m:=IdentityMat(3);;
    gap> n:=[[1,2,3],[4,5,6]];;
    gap> ComplementIntMat(m,n);
    rec( complement := [ [ 0, 0, 1 ] ], sub := [ [ 1, 2, 3 ], [ 0, 3, 6 ] ],
      moduli := [ 1, 3 ] ) |

%%%%%%%%%%%%%%%%%%%%%%%%%%%%%%%%%%%%%%%%%%%%%%%%%%%%%%%%%%%%%%%%%%%%%%%%%%%%%
\Section{TriangulizedIntegerMat}
'TriangulizedIntegerMat( <mat> )'

Computes  an  integral  upper  triangular  form  of  a  matrix with integer
entries.

|    gap> m:=[[1,15,28],[4,5,6],[7,8,9]];;
    gap> TriangulizedIntegerMat(m);
    [ [ 1, 15, 28 ], [ 0, 1, 1 ], [ 0, 0, 3 ] ]|

%%%%%%%%%%%%%%%%%%%%%%%%%%%%%%%%%%%%%%%%%%%%%%%%%%%%%%%%%%%%%%%%%%%%%%%%%%%%%
\Section{TriangulizedIntegerMatTransform}
'TriangulizedIntegerMatTransform( <mat> )'

Computes  an  integral  upper  triangular  form  of  a  matrix with integer
entries.  It returns a record with a  component 'normal' (a matrix in upper
triangular  form) and a component 'rowtrans' that gives the transformations
done to the original matrix to bring it into upper triangular form.

|    gap> m:=[[1,15,28],[4,5,6],[7,8,9]];;
    gap> n:=TriangulizedIntegerMatTransform(m);
    rec( normal := [ [ 1, 15, 28 ], [ 0, 1, 1 ], [ 0, 0, 3 ] ],
      rowC := [ [ 1, 0, 0 ], [ 0, 1, 0 ], [ 0, 0, 1 ] ],
      rowQ := [ [ 1, 0, 0 ], [ 1, -30, 17 ], [ -3, 97, -55 ] ], rank := 3,
      signdet := 1, rowtrans := [ [ 1, 0, 0 ], [ 1, -30, 17 ], [ -3, 97, -55 ] ] )
    gap> n.rowtrans*m=n.normal;
    true|

%%%%%%%%%%%%%%%%%%%%%%%%%%%%%%%%%%%%%%%%%%%%%%%%%%%%%%%%%%%%%%%%%%%%%%%%%%%%%
\Section{TriangulizeIntegerMat}
'TriangulizeIntegerMat( <mat> )'

Changes  <mat> to be in  upper triangular form. (The  result is the same as
that  of 'TriangulizedIntegerMat', but  <mat> will be  modified, thus using
less memory.)

|    gap> m:=[[1,15,28],[4,5,6],[7,8,9]];;
    gap> TriangulizeIntegerMat(m); m;
    [ [ 1, 15, 28 ], [ 0, 1, 1 ], [ 0, 0, 3 ] ]|

%%%%%%%%%%%%%%%%%%%%%%%%%%%%%%%%%%%%%%%%%%%%%%%%%%%%%%%%%%%%%%%%%%%%%%%%%%%%%
\Section{HermiteNormalFormIntegerMat}
'HermiteNormalFormIntegerMat( <mat> )'

This  operation computes  the Hermite  normal form  of a  matrix <mat> with
integer  entries. The Hermite Normal Form  (HNF), $H$ of an integer matrix,
$A$  is a row  equivalent upper triangular  form such that all off-diagonal
entries  are reduced modulo the  diagonal entry of the  column they are in.
There exists a unique unimodular matrix $Q$ such that $QA = H$.

|    gap> m:=[[1,15,28],[4,5,6],[7,8,9]];;
    gap> HermiteNormalFormIntegerMat(m);
    [ [ 1, 0, 1 ], [ 0, 1, 1 ], [ 0, 0, 3 ] ]|

%%%%%%%%%%%%%%%%%%%%%%%%%%%%%%%%%%%%%%%%%%%%%%%%%%%%%%%%%%%%%%%%%%%%%%%%%%%%%
\Section{HermiteNormalFormIntegerMatTransform}
'HermiteNormalFormIntegerMatTransform( <mat> )'

This  operation computes  the Hermite  normal form  of a  matrix <mat> with
integer entries. It returns a record with components 'normal' (a matrix $H$
of  the Hermite normal form) and  'rowtrans' (a unimodular matrix $Q$) such
that $Q$<mat>$=H$

|    gap> m:=[[1,15,28],[4,5,6],[7,8,9]];;
    gap> n:=HermiteNormalFormIntegerMatTransform(m);
    rec( normal := [ [ 1, 0, 1 ], [ 0, 1, 1 ], [ 0, 0, 3 ] ],
      rowC := [ [ 1, 0, 0 ], [ 0, 1, 0 ], [ 0, 0, 1 ] ],
      rowQ := [ [ -2, 62, -35 ], [ 1, -30, 17 ], [ -3, 97, -55 ] ], rank := 3,
      signdet := 1,
      rowtrans := [ [ -2, 62, -35 ], [ 1, -30, 17 ], [ -3, 97, -55 ] ] )
    gap> n.rowtrans*m=n.normal;
    true|

%%%%%%%%%%%%%%%%%%%%%%%%%%%%%%%%%%%%%%%%%%%%%%%%%%%%%%%%%%%%%%%%%%%%%%%%%%%%%
\Section{SmithNormalFormIntegerMat}
'SmithNormalFormIntegerMat( <mat> )'

This  operation  computes  the  Smith  normal  form  of a matrix <mat> with
integer entries. The Smith Normal Form,$S$, of an integer matrix $A$ is the
unique  equivalent diagonal  form with  $S_i$ dividing  $S_j$ for $i \< j$.
There exist unimodular integer matrices $P, Q$ such that $PAQ = S.$

|    gap> m:=[[1,15,28],[4,5,6],[7,8,9]];;
    gap> SmithNormalFormIntegerMat(m);
    [ [ 1, 0, 0 ], [ 0, 1, 0 ], [ 0, 0, 3 ] ]|

%%%%%%%%%%%%%%%%%%%%%%%%%%%%%%%%%%%%%%%%%%%%%%%%%%%%%%%%%%%%%%%%%%%%%%%%%%%%%
\Section{SmithNormalFormIntegerMatTransforms}
'SmithNormalFormIntegerMatTransforms( <mat> )'

This  operation  computes  the  Smith  normal  form  of a matrix <mat> with
integer  entries. It  returns a  record with  components 'normal' (a matrix
$S$),  'rowtrans' (a matrix  $P$), and 'coltrans'  (a matrix $Q$) such that
$P$<mat>$Q=S$.

|    gap> m:=[[1,15,28],[4,5,6],[7,8,9]];;
    gap> n:=SmithNormalFormIntegerMatTransforms(m);
    rec( normal := [ [ 1, 0, 0 ], [ 0, 1, 0 ], [ 0, 0, 3 ] ],
      rowC := [ [ 1, 0, 0 ], [ 0, 1, 0 ], [ 0, 0, 1 ] ],
      rowQ := [ [ -2, 62, -35 ], [ 1, -30, 17 ], [ -3, 97, -55 ] ],
      colC := [ [ 1, 0, 0 ], [ 0, 1, 0 ], [ 0, 0, 1 ] ],
      colQ := [ [ 1, 0, -1 ], [ 0, 1, -1 ], [ 0, 0, 1 ] ], rank := 3,
      signdet := 1,
      rowtrans := [ [ -2, 62, -35 ], [ 1, -30, 17 ], [ -3, 97, -55 ] ],
      coltrans := [ [ 1, 0, -1 ], [ 0, 1, -1 ], [ 0, 0, 1 ] ] )
    gap> n.rowtrans*m*n.coltrans=n.normal;
    true|

%%%%%%%%%%%%%%%%%%%%%%%%%%%%%%%%%%%%%%%%%%%%%%%%%%%%%%%%%%%%%%%%%%%%%%%%%%%%%
\Section{DiagonalizeIntMat}
'DiagonalizeIntMat( <mat> )'

This  function changes <mat> to its SNF. (The result is the same as that of
'SmithNormalFormIntegerMat',  but <mat>  will be  modified, thus using less
memory.)

|    gap> m:=[[1,15,28],[4,5,6],[7,8,9]];;
    gap> DiagonalizeIntMat(m);m;
    [ [ 1, 0, 0 ], [ 0, 1, 0 ], [ 0, 0, 3 ] ]|

%%%%%%%%%%%%%%%%%%%%%%%%%%%%%%%%%%%%%%%%%%%%%%%%%%%%%%%%%%%%%%%%%%%%%%%%%%%%%
\Section{NormalFormIntMat}

All  the previous  routines build  on the  following ``workhorse'' routine:

'NormalFormIntMat( <mat>, <options> )'

This  general operation for computation of various Normal Forms is probably
the most efficient.

Options bit values:
\begin{itemize}
\item{0/1} Triangular Form / Smith Normal Form.

\item{2}   Reduce off diagonal entries.

\item{4}   Row Transformations.

\item{8}   Col Transformations.

\item{16}   Destructive (the original matrix may be destroyed)
\end{itemize}

Compute  a Triangular, Hermite  or Smith form  of the $n  \times m$ integer
input  matrix  $A$.  Optionally,  compute  $n  \times  n$  and $m \times m$
unimodular  transforming matrices $Q, P$  which satisfy $QA =  H$ or $QAP =
S$.

Note  option is a value ranging from 0  - 15 but not all options make sense
(eg  reducing off diagonal entries with SNF option selected already). If an
option makes no sense it is ignored.

Returns  a record  with component  'normal' containing  the computed normal
form  and optional components  'rowtrans' and/or 'coltrans'  which hold the
respective transformation matrix. Also in the record are components holding
the sign of the determinant, signdet, and the Rank of the matrix, rank.


|    gap> m:=[[1,15,28],[4,5,6],[7,8,9]];;
    gap> NormalFormIntMat(m,0);  # Triangular, no transforms
    rec( normal := [ [ 1, 15, 28 ], [ 0, 1, 1 ], [ 0, 0, 3 ] ], rank := 3,
      signdet := 1 )
    gap> NormalFormIntMat(m,6);  # Hermite Normal Form with row transforms
    rec( normal := [ [ 1, 0, 1 ], [ 0, 1, 1 ], [ 0, 0, 3 ] ],
      rowC := [ [ 1, 0, 0 ], [ 0, 1, 0 ], [ 0, 0, 1 ] ],
      rowQ := [ [ -2, 62, -35 ], [ 1, -30, 17 ], [ -3, 97, -55 ] ], rank := 3,
      signdet := 1,
      rowtrans := [ [ -2, 62, -35 ], [ 1, -30, 17 ], [ -3, 97, -55 ] ] )
    gap> NormalFormIntMat(m,13); # Smith Normal Form with both transforms
    rec( normal := [ [ 1, 0, 0 ], [ 0, 1, 0 ], [ 0, 0, 3 ] ],
      rowC := [ [ 1, 0, 0 ], [ 0, 1, 0 ], [ 0, 0, 1 ] ],
      rowQ := [ [ -2, 62, -35 ], [ 1, -30, 17 ], [ -3, 97, -55 ] ],
      colC := [ [ 1, 0, 0 ], [ 0, 1, 0 ], [ 0, 0, 1 ] ],
      colQ := [ [ 1, 0, -1 ], [ 0, 1, -1 ], [ 0, 0, 1 ] ], rank := 3,
      signdet := 1,
      rowtrans := [ [ -2, 62, -35 ], [ 1, -30, 17 ], [ -3, 97, -55 ] ],
      coltrans := [ [ 1, 0, -1 ], [ 0, 1, -1 ], [ 0, 0, 1 ] ] )
    gap> last.rowtrans*m*last.coltrans;
    [ [ 1, 0, 0 ], [ 0, 1, 0 ], [ 0, 0, 3 ] ]|

%%%%%%%%%%%%%%%%%%%%%%%%%%%%%%%%%%%%%%%%%%%%%%%%%%%%%%%%%%%%%%%%%%%%%%%%%%%%%
\Section{AbelianInvariantsOfList}
'AbelianInvariantsOfList( <list> )'

Given  a list of  positive integers, this  routine returns a  list of prime
powers,  such that the prime  power factors of the  entries in the list are
returned in sorted form.

|    gap> AbelianInvariantsOfList([4,6,2,12]);
    [ 2, 2, 3, 3, 4, 4 ]|

%%%%%%%%%%%%%%%%%%%%%%%%%%%%%%%%%%%%%%%%%%%%%%%%%%%%%%%%%%%%%%%%%%%%%%%%%%%%%
\Section{Determinant of an integer matrix}
\index{determinant!integer matrix}

'DeterminantIntMat( <mat> )'

Computes  the determinant of  an integer matrix  using the same strategy as
'NormalFormIntMat'.  This method is faster  in general for matrices greater
than  $20  \times  20$  but  quite  a  lot  slower for smaller matrices. It
therefore   passes   the   work   to   the  more  general  'DeterminantMat'
(see~"DeterminantMat") for these smaller matrices.

%%%%%%%%%%%%%%%%%%%%%%%%%%%%%%%%%%%%%%%%%%%%%%%%%%%%%%%%%%%%%%%%%%%%%%%%%%%%%
\Section{Diaconis-Graham normal form}
\index{Diaconis-Graham normal form}

'DiaconisGraham( <mat>, <moduli>)'

Diaconis  and Graham (see \cite{dg99}) defined a normal form for generating
sets of abelian groups. Here <moduli> should be a list of positive integers
such  that 'moduli[i+1]' divides 'moduli[i]'  for all 'i', representing the
abelian  group $A=\Z/moduli[1]\times\ldots\times\Z/moduli[n]$. The integral
matrix  <m> should have <n> columns where 'n=Length(moduli)', and each line
(with  the <i>-th element  taken 'mod moduli[i]')  represents an element of
the group $A$.

The  function returns 'false' if the set  of elements of $A$ represented by
the  lines of $m$ does not generate  $A$. Otherwise it returns a record 'r'
with fields

'r.normal':         the Diaconis-Graham normal form, a matrix of same shape
    as 'm' where either the first 'n' lines are the identity matrix and the
    remaining  lines are '0',  or 'Length(m)=n' and  '.normal' differs from
    the  identity matrix only in the  entry '.normal[n][n]', which is prime
    to 'moduli[n]'.

'r.rowtrans':        a unimodular matrix such that  
    'r.normal=List(r.rowtrans\*m,v->Zip(v,moduli,
                                       function(x,y)return x mod y;end))'

Here is an example\:

|    gap> DiaconisGraham([[3,0],[4,1]],[10,5]);
    rec(
      rowtrans := [ [ -13, 10 ], [ 4, -3 ] ],
      normal := [ [ 1, 0 ], [ 0, 2 ] ] )|
