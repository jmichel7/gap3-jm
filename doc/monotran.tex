%%%%%%%%%%%%%%%%%%%%%%%%%%%%%%%%%%%%%%%%%%%%%%%%%%%%%%%%%%%%%%%%%%%%%%%%%%%%%
%%
%A  GAP documentation                                   Goetz.Pfeiffer@UCG.IE
%%
%A  $Id: monotran.tex,v 2.2 1997/11/24 12:53:51 goetz Exp $
%%
%Y  Copyright (C) 1997, Mathematics Dept, University College Galway, Ireland.
%%
%%  This file documents the functions for transformation monoids.
%%
\Chapter{Transformation Monoids}

A transformation monoid is a monoid of transformations of $n$ points (see
chapter "Transformations").   These  monoids occur   for example in   the
theory of finite state  automata and  as  the results of enumerations  of
finitely presented monoids.  In the theory of semigroups and monoids they
ply to some extend the role that is taken  by permutation groups in group
theory.  In fact, there are  close  relations between permutation  groups
and transformation monoids.  One of these relations  is manifested by the
Sch\accent127utzenberger group of an element  of a transformation monoid,
which  is represented  as  a permutation  group rather   than a group  of
transformations.   Another relation which  is used by most functions that
deal with transformation monoids is the fact that a transformation monoid
can be efficiently described in terms  of several permutation groups (for
details see~\cite{LPRR1} and~\cite{LPRR2}).

This   chapter describes the    functions  that deal with  transformation
monoids.

The chapter  starts with  the  description of   the function  that  tests
whether  or   not  a  given   object  is a   transformation  monoid  (see
"IsTransMonoid").  Then there is  the function that determines the degree
of a transformation monoid (see "Degree of a Transformation Monoid").

There are  a function to   construct  the full transformation  monoid  of
degree $n$ (see "FullTransMonoid") and a function to construct the monoid
of all partial transformations of degree $n$ (see "PartialTransMonoid").

Then there are a function that determines all  images of a transformation
monoid (see  "ImagesTransMonoid") and   a function that    determines all
kernels of a transformation monoid (see "KernelsTransMonoid").

Because  each transformation monoid   is   a domain  all  set   theoretic
functions can  be   applied to it  (see   chapter  \"Domains\"\ and  "Set
Functions  for  Transformation Monoids").  Also because  a transformation
monoid is after  all a monoid all  monoid functions can  be applied to it
(see chapter   "Monoids and Semigroups"      and "Monoid Functions    for
Transformation Monoids").

Next the functions that determine Green classes in transformation monoids
are described (see "H Classes for Transformation Monoids", "R Classes for
Transformation Monoids", "L Classes  for Transformation Monoids", and  "D
Classes for Transformation Monoids").

Finally,  there is  a  section   about how  a  transformation  monoid  is
displayed (see "Display a  Transformation Monoid").  The last section  in
this  chapter  describes how  transformation   monoids are represented as
records in {\GAP} (see "Transformation Monoid Records").

The functions described here are in the file '\"monotran.g\"'.

%%%%%%%%%%%%%%%%%%%%%%%%%%%%%%%%%%%%%%%%%%%%%%%%%%%%%%%%%%%%%%%%%%%%%%%%%
\Section{IsTransMonoid}%
\index{test!for transformation monoid}

'IsTransMonoid( <obj> )'

'IsTransMonoid'  returns  'true' if the object    <obj>, which may  be an
object  of an  arbitrary type,  is  a transformation monoid, and  'false'
otherwise.  It will signal an error if <obj> is an unbound variable.

|    gap> IsTransMonoid( Monoid( [ Transformation( [ 1, 2, 1 ] ) ] ) );
    true
    gap> IsTransMonoid( Group( (1,2), (1,2,3,4) ) );
    false
    gap> IsTransMonoid( [ 1, 2, 1 ] );
    false |

%%%%%%%%%%%%%%%%%%%%%%%%%%%%%%%%%%%%%%%%%%%%%%%%%%%%%%%%%%%%%%%%%%%%%%%%%
\Section{Degree of a Transformation Monoid}

'Degree( <M> )'

'Degree' returns the degree of a  transformation monoid <M>, which is the
common degree of all the generators of <M>.

|    gap> Degree( Monoid( Transformation( [ 1, 2, 1 ] ) ) );
    3 |

The *degree* of a transformation monoid is the  number  of points it acts
upon.

%%%%%%%%%%%%%%%%%%%%%%%%%%%%%%%%%%%%%%%%%%%%%%%%%%%%%%%%%%%%%%%%%%%%%%%%%
\Section{FullTransMonoid}%
\index{full transformation monoid}

'FullTransMonoid( <n> )'

'FullTransMonoid' returns the full transformation monoid of degree <n>.

|    gap> M:= FullTransMonoid( 8 );
    Monoid( [ Transformation( [ 2, 1, 3, 4, 5, 6, 7, 8 ] ), 
      Transformation( [ 8, 1, 2, 3, 4, 5, 6, 7 ] ), 
      Transformation( [ 2, 2, 3, 4, 5, 6, 7, 8 ] ) ] ) 
    gap> Size( M );
    16777216 |

The   *full transformation monoid* of  degree  $n$ is  the  monoid of all
transformations of degree $n$.

%%%%%%%%%%%%%%%%%%%%%%%%%%%%%%%%%%%%%%%%%%%%%%%%%%%%%%%%%%%%%%%%%%%%%%%%%
\Section{PartialTransMonoid}%
\index{partial transformation}

'PartialTransMonoid( <n> )'

'PartialTransMonoid'  returns  the monoid  of partial  transformations of
degree <n>.

|    gap> M:= PartialTransMonoid( 8 );
    Monoid( [ Transformation( [ 2, 1, 3, 4, 5, 6, 7, 8, 9 ] ), 
      Transformation( [ 8, 1, 2, 3, 4, 5, 6, 7, 9 ] ), 
      Transformation( [ 9, 2, 3, 4, 5, 6, 7, 8, 9 ] ), 
      Transformation( [ 2, 2, 3, 4, 5, 6, 7, 8, 9 ] ) ] )
    gap> Size( M );
    1000000000 |
    
A *partial transformation*  of degree $n$  is a  mapping from $\{1,  ...,
n\}$ to itself where  every point $i \in  \{1, ..., n\}$  has at most one
image.  Here the undefined point is represented by $n+1$.

%%%%%%%%%%%%%%%%%%%%%%%%%%%%%%%%%%%%%%%%%%%%%%%%%%%%%%%%%%%%%%%%%%%%%%%%%
\Section{ImagesTransMonoid}

'ImagesTransMonoid( <M> )'

'ImagesTransMonoid'  returns the set    of images of  all elements of the
transformation monoid <M> (see "Image of a Transformation").

|    gap> M:= Monoid( Transformation( [ 1, 4, 4, 2 ] ),    
    >       Transformation( [ 2, 4, 4, 4 ] ) );;   
    gap> ImagesTransMonoid(M);                                              
    [ [ 1, 2, 3, 4 ], [ 1, 2, 4 ], [ 2 ], [ 2, 4 ], [ 4 ] ] |

%%%%%%%%%%%%%%%%%%%%%%%%%%%%%%%%%%%%%%%%%%%%%%%%%%%%%%%%%%%%%%%%%%%%%%%%%
\Section{KernelsTransMonoid}

'KernelsTransMonoid( <M> )'

'KernelsTransMonoid' returns  the set of kernels of  all  elements of the
transformation monoid <M> (see "Kernel of a Transformation").

|    gap> M:= Monoid( [ Transformation( [ 1, 4, 4, 2 ] ),    
    >       Transformation( [ 2, 4, 4, 4 ] ) ] );;   
    gap> KernelsTransMonoid(M);
    [ [ [ 1 ], [ 2 ], [ 3 ], [ 4 ] ], [ [ 1 ], [ 2, 3 ], [ 4 ] ], 
      [ [ 1 ], [ 2, 3, 4 ] ], [ [ 1, 2, 3, 4 ] ] ] |

%%%%%%%%%%%%%%%%%%%%%%%%%%%%%%%%%%%%%%%%%%%%%%%%%%%%%%%%%%%%%%%%%%%%%%%%%
\Section{Set Functions for Transformation Monoids}

All set theoretic  functions described in   chapter \"Domains\" are  also
applicable  to transformation  monoids.    This section  describes  which
functions are implemented specially for transformation monoids. Functions
not mentioned here  are handled by the default  methods  described in the
respective sections of chapter \"Domains\".

'Size( <M> )'%
\index{Size!for Transformation Monoids}

'Size'  calls 'RClasses' (see "RClasses"),  if necessary, and returns the
sum of the sizes of all $R$ classes of <M>.

|    gap> Size( Monoid( Transformation( [ 1, 2, 1 ] ) ) );
    2 |

\vspace{5mm}
'Elements( <M> )'
\index{Elements!for Transformation Monoids}

'Elements' calls  'RClasses' (see "RClasses")  if necessary,  and returns
the union of the elements of all $R$ classes of <M>.

|    gap> Elements( Monoid( Transformation( [ 1, 2, 1 ] ) ) );
    [ Transformation( [ 1, 2, 1 ] ), Transformation( [ 1 .. 3 ] ) ] |

\vspace{5mm}
'<obj> in <M>'%
\index{in!for Transformation Monoids}%
\index{membership test!for Transformation Monoids}

The  membership test of elements of  transformation  monoids first checks
whether  <obj>  is    a    transformation  in the   first     place  (see
"IsTransformation") and if so whether the degree of <obj> (see "Degree of
a Transformation") coincides  with the  degree of <M>  (see "Degree  of a
Transformation Monoid").  Then the image and  the kernel of <obj> is used
to locate the possible $R$ class of <obj> in  <M> and the membership test
is delegated to that $R$ class (see "Set Functions for Green Classes").

|   gap> M:= Monoid( Transformation( [ 1, 2, 1 ] ) );;
    gap> (1,2,3) in M;
    false
    gap> Transformation( [1, 2, 1 ] ) in M;
    true
    gap> Transformation( [ 1, 2, 1, 4 ] ) in M;
    false | 

%%%%%%%%%%%%%%%%%%%%%%%%%%%%%%%%%%%%%%%%%%%%%%%%%%%%%%%%%%%%%%%%%%%%%%%%%
\Section{Monoid Functions for Transformation Monoids}

All  functions   described in chapter   "Monoids  and Semigroups"  can be
applied to transformation monoids.

Green   classes  are special  subsets  of  a   transformation monoid.  In
particular,  they are domains so  all set theoretic functions for domains
(see chapter \"Domains\")   can be applied to   Green classes.  This   is
described in "Set Functions for Green  Classes".  Single Green classes of
a transformation  monoid are constructed by  the  functions 'RClass' (see
"RClass"  and "R   Classes for  Transformation   Monoids"), 'LClass' (see
"LClass" and   "L  Classes for  Transformation   Monoids"), 'DClass' (see
"DClass" and "D Classes  for Transformation Monoids"), and  'HClass' (see
"HClass"  and "H Classes for Transformation  Monoids").   The list of all
Green classes  of a given type  in a transformation monoid is constructed
by    the   functions   'RClasses'  (see    "RClasses"),  'LClasses' (see
"LClasses"),   'DClasses'    (see    "DClasses"),  and   'HClasses'  (see
"HClasses").

%%%%%%%%%%%%%%%%%%%%%%%%%%%%%%%%%%%%%%%%%%%%%%%%%%%%%%%%%%%%%%%%%%%%%%%%%
\Section{SchutzenbergerGroup for Transformation Monoids}

'SchutzenbergerGroup( <M>, <s> )'\\
'SchutzenbergerGroup( <class> )'

'SchutzenbergerGroup' returns the  Sch\accent127utzenberger  group of the
element  <s> in the  transformation monoid <M> as  a permutation group on
the image of <s>.

In     the   second    form      'SchutzenbergerGroup'    returns     the
Sch\accent127utzenberger   group  of  the    Green class   <class>   of a
transformation monoid, where <class> is either an H  class, an R class or
a D class.  The  Sch\accent127utzenberger group of an  H class <class> is
the  same  as   the  Sch\accent127utzenberger  group  of   <class>.   The
Sch\accent127utzenberger group of  an R class  <class> is the generalised
right Sch\accent127utzenberger group of the representative of <class> and
the  Sch\accent127utzenberger  group  of an    L  class  <class> is   the
generalised left Sch\accent127utzenberger group  of the representative of
<class>.  Note  that the Sch\accent127utzenberger of  an R  class is only
unique up to conjugation.

%%%%%%%%%%%%%%%%%%%%%%%%%%%%%%%%%%%%%%%%%%%%%%%%%%%%%%%%%%%%%%%%%%%%%%%%%
\Section{H Classes for Transformation Monoids}

In addition to  the usual components of   an H class record,  the  record
representing the H class  <hClass> of <s>  in a transformation monoid can
have the  following  components.     They are created  by   the  function
'SchutzenbergerGroup'   (see  "SchutzenbergerGroup")    which is   called
whenever the size, the list of elements of <hClass>, or a membership test
in <hClass> is asked for.

'schutzenbergerGroup':\\
        set to   the Sch\accent127utzenberger group     of <hClass> as  a
        permutation      group  on       the  set     of    images     of
        '<hClass>.representative'    (see   "SchutzenbergerGroup      for
        Transformation Monoids").

'R':\\
        the R class of '<hClass>.representative.'

'L':\\
        the L class of '<hClass>.representative.'
        
The  following functions have a  special implementation in terms of these
components.

\vspace{5mm}
'Size( <hClass> )'%
\index{Size!of an H class of transformations}

returns    the size  of  the   H  class  <hClass>.   This  function calls
'SchutzenbergerGroup' and determines the size of <hClass>  as the size of
the resulting group.

\vspace{5mm}
'Elements( <hClass> )'%
\index{Elements!of an H class of transformations}

returns the set of elements of the H class <hClass>.  This function calls
'SchutzenbergerGroup' and  determines the set  of elements of <hClass> as
the  set   of  elements  of   the   resulting  group   multiplied by  the
representative of <hClass>.

\vspace{5mm}
'<x> in <hClass>'%
\index{in!for H class of transformations}%
\index{membership test!for H class of transformations}

returns 'true' if  <x> is an element of  the H class <hClass> and 'false'
otherwise.   This function calls  'SchutzenbergerGroup' and tests whether
the  quotient   of the   representative   of    <hClass> and  <x>    (see
"PermLeftQuoTrans") is in the resulting group.

%%%%%%%%%%%%%%%%%%%%%%%%%%%%%%%%%%%%%%%%%%%%%%%%%%%%%%%%%%%%%%%%%%%%%%%%%
\Section{R Classes for Transformation Monoids}

In addition to  the usual components of   an R class record,  the  record
representing the R class  <rClass> of <s>  in a transformation monoid can
have the  following  components.     They are created  by   the  function
'SchutzenbergerGroup'   (see  "SchutzenbergerGroup")    which is   called
whenever the size, the list of elements of <rClass>, or a membership test
in <rClass> is asked for.

'schutzenbergerGroup':\\
        set to   the Sch\accent127utzenberger group     of <rClass> as  a
        permutation      group  on       the  set     of    images     of
        '<rClass>.representative'    (see   "SchutzenbergerGroup      for
        Transformation Monoids").
        
'images':\\        
        is the  list  of different  image  sets occurring in the  R class
        <rClass>.  The first entry   in this list  is  the image set   of
        '<rClass>.representative'.
        
'rMults':\\        
        is a  list  of  permutations  such  that    the product  of   the
        representative of <rClass> and the  inverse of the $i$th entry in
        the list yields  an element  of <rClass> whose  image  set is the
        $i$th entry in the list '<rClass>.images'.

The  following functions have a  special implementation in terms of these
components.

\vspace{5mm}
'Size( <rClass> )'%
\index{Size!of an R class of transformations}

returns    the size  of  the   R  class  <rClass>.   This  function calls
'SchutzenbergerGroup' and determines the size of <rClass>  as the size of
the resulting group times the length of the list '<rClass>.images'.

\vspace{5mm}
'Elements( <rClass> )'%
\index{Elements!of an R class of transformations}

returns the set of elements of the R class <rClass>.  This function calls
'SchutzenbergerGroup' and determines the  set of elements of  <rClass> as
the  set   of    elements of  the  resulting    group   multiplied by the
representative of   <rClass> and each   single   permutation in the  list
'<rClass>.rMults'.

\vspace{5mm}
'<x> in <rClass>'%
\index{in!for R class of transformations}%
\index{membership test!for R class of transformations}

returns 'true' if <x>  is an element of the  R class <rClass> and 'false'
otherwise.  This function calls  'SchutzenbergerGroup' and tests  whether
the  quotient   of   the  representative    of  <rClass>  and   '<x>  \*\ 
<rClass>.rMults[<i>]' (see  "PermLeftQuoTrans") is in the resulting group
where <i> is the position of the image set of <x> in '<rClass>.images'.

\vspace{5mm}
'HClasses( <rClass> )'
\index{HClasses!of an R class of transformations}

returns the list of H classes contained in the R class <rClass>.

%%%%%%%%%%%%%%%%%%%%%%%%%%%%%%%%%%%%%%%%%%%%%%%%%%%%%%%%%%%%%%%%%%%%%%%%%
\Section{L Classes for Transformation Monoids}

In addition to  the usual components of   an L class record,  the  record
representing the L class  <lClass> of <s>  in a transformation monoid can
have the  following  components.     They are created  by   the  function
'SchutzenbergerGroup'   (see  "SchutzenbergerGroup")    which is   called
whenever the size, the list of elements of <lClass>, or a membership test
in <lClass> is asked for.

'schutzenbergerGroup':\\
        set to   the Sch\accent127utzenberger group     of <lClass> as  a
        permutation      group  on       the  set     of    images     of
        '<lClass>.representative'    (see   "SchutzenbergerGroup      for
        Transformation Monoids").
        
'kernels':\\
        is the list   of different   kernels   occurring in the  L  class
        <lClass>.  The   first  entry  in  this  list is   the kernel  of
        '<rClass>.representative'.
        
'lMults':\\        
        is a list  of  binary  relations such  that  the product  of  the
        inverse of the $i$th entry in the list  and the representative of
        <rClass> yields an element of <rClass>  whose kernel is the $i$th
        entry in the list '<rClass>.kernels'.

The  following functions have a  special implementation in terms of these
components.

\vspace{5mm}
'Size( <lClass> )'%
\index{Size!of an L class of transformations}

returns    the size  of  the   L  class  <lClass>.   This  function calls
'SchutzenbergerGroup' and determines the size of <lClass>  as the size of
the resulting group times the length of the list '<lClass>.kernels'.

\vspace{5mm}
'Elements( <lClass> )'%
\index{Elements!of an L class of transformations}

returns the set of elements of the L class <lClass>.  This function calls
'SchutzenbergerGroup'  and determines the set of  elements of <lClass> as
the   set   of elements  of  the   resulting   group premultiplied by the
representative of <lClass> and each  single binary  relation in the  list
'<lClass>.lMults'.

\vspace{5mm}
'<x> in <lClass>'%
\index{in!for L class of transformations}%
\index{membership test!for L class of transformations}

returns 'true' if  <x> is an element of  the L class <lClass> and 'false'
otherwise.  This function calls  'SchutzenbergerGroup' and tests  whether
the quotient of  the representative of <lClass> and '<lClass>.lMults[<i>]
\*\ <x>' (see "PermLeftQuoTrans") is in the  resulting group where <i> is
the position of the kernel of <x> in '<lClass>.kernels'.

\vspace{5mm}
'HClasses( <lClass> )'
\index{HClasses!of an L class of transformations}

returns the list of H classes contained in the L class <lClass>.

%%%%%%%%%%%%%%%%%%%%%%%%%%%%%%%%%%%%%%%%%%%%%%%%%%%%%%%%%%%%%%%%%%%%%%%%%
\Section{D Classes for Transformation Monoids}

In  addition to the   usual components of  a D  class record,  the record
representing  the D class <dClass> of  <s> in a transformation monoid can
have  the following  components.   They   are  created by   the  function
'SchutzenbergerGroup'   (see    "SchutzenbergerGroup") which   is  called
whenever the size, the list of elements of <dClass>, or a membership test
in <dClass> is asked for.

'schutzenbergerGroup':\\
        set to   the Sch\accent127utzenberger group     of <dClass> as  a
        permutation      group  on       the  set     of    images     of
        '<dClass>.representative'    (see   "SchutzenbergerGroup      for
        Transformation Monoids").
        
'H':\\
        set to the H class of '<dClass>.representative'.

'L':\\
        set to the L class of '<dClass>.representative'.

'R':\\
        set to the R class of '<dClass>.representative'.
        
'rCosets':\\
        contains  a   list  of  (right)  coset  representatives   of  the
        Sch\accent127utzenberger       group     of   <dClass>   in   the
        Sch\accent127utzenberger group of the R class '<dClass>.R'.


The  following functions have a  special implementation in terms of these
components.

\vspace{5mm}
'Size( <dClass> )'%
\index{Size!of a D class of transformations}

returns the   size of   the  D  class  <dClass>.   This   function  calls
'SchutzenbergerGroup' and determines the size of <dClass> in terms of the
sizes  of the resulting  group and the Sch\accent127utzenberger groups of
the R class '<dClass>.R' and the L class '<dClass>.L'.

\vspace{5mm}
'Elements( <dClass> )'%
\index{Elements!of a D class of transformations}

returns the set of elements of the D class <dClass>.  This function calls
'SchutzenbergerGroup'  and determines the set of  elements of <dClass> as
the union of cosets of the Sch\accent127utzenberger  group of the L class
'<dClass>.L'  determined  through  the multipliers '<dClass>.rCosets' and
'<dClass>.R.rMults'.

\vspace{5mm}
'<x> in <dClass>'%
\index{in!for D class of transformations}%
\index{membership test!for D class of transformations}

returns 'true' if  <x> is an element of  the D class <dClass> and 'false'
otherwise.  This function  calls 'SchutzenbergerGroup'  and tests whether
the quotient  of the representative  of <dClass> and a suitable translate
of <x> can be found in one  of the cosets of the Sch\accent127utzenberger
group   of  the   L   class   '<dClass>.L'   determined   by    the  list
'<dClass>.rCosets'.

\vspace{5mm}
'HClasses( <dClass> )'
\index{HClasses!of a D class of transformations}

returns the list of H classes contained in the D class <dClass>.

\vspace{5mm}
'LClasses( <dClass> )'
\index{LClasses!of a D class of transformations}

returns the list of L classes contained in the D class <dClass>.

\vspace{5mm}
'RClasses( <dClass> )'
\index{RClasses!of a D class of transformations}

returns the list of R classes contained in the D class <dClass>.

%%%%%%%%%%%%%%%%%%%%%%%%%%%%%%%%%%%%%%%%%%%%%%%%%%%%%%%%%%%%%%%%%%%%%%%%%
\Section{Display a Transformation Monoid}

'Display( <M> )'

'Display' displays the Green class structure of the transformation monoid
<M>.  Each D class is displayed  as a single item on  a line according to
its rank.  A D class displayed as

'[<a>.<b>.<d>]'

is a regular D  class with a Sch\accent127utzenberger  group of size <a>,
consisting of <b> L classes, or <d> R classes.  A D class displayed as 

'\{<a>.<b>x<c>.<d>x<e>\}'

is  a nonregular D  class with  a  Sch\accent127utzenberger group of size
<a>, consisting of $<b> \times <c>$ L classes (of which <c> have the same
kernel), or   $<d> \times <e>$ R  classes   (of which <e> have   the same
image).

|    gap> M:= Monoid( Transformation( [ 7, 7, 1, 1, 5, 6, 5, 5 ] ), 
    > Transformation( [ 3, 8, 3, 7, 4, 6, 4, 5 ] ) );;
    gap> Size( M );
    27
    gap> Display( M );
    Rank 8: [1.1.1] 
    Rank 6: {1.1x1.1x1} 
    Rank 5: {1.1x1.1x1} 
    Rank 4: {1.1x1.1x1} [2.1.1] 
    Rank 3: {1.1x1.4x1} [1.3.4] 
    Rank 2: [1.5.1] |

%%%%%%%%%%%%%%%%%%%%%%%%%%%%%%%%%%%%%%%%%%%%%%%%%%%%%%%%%%%%%%%%%%%%%%%%%
\Section{Transformation Monoid Records}

In addition  to  the usual  components  of a  monoid record  (see "Monoid
Records and Semigroup Records") the  record representing a transformation
monoid <M> has a component

'isTransMonoid':\\
        which is always set to 'true'.
        
Moreover,  such  a  record will  (after  a while)   acquire the following
components.

'orbitClasses':\\
        a list of R classes of <M> such that every orbit of image sets is
        represented exactly once.

'images':\\
        the list of  lists  where 'images[ <k>    ]' is the  list  of all
        different image sets of size <k> of the elements of <M>.
        
'imagePos':\\
        stores  the relation  between 'orbitClasses'  and  'images'.  The
        image set 'images[<k>][<l>]'  occurs in the orbit  of the R class
        with index 'imagePos[<k>][<l>]' in the list 'orbitClasses'.
        
'rClassReps':\\
        a list of  lists,  where 'rClassReps[<l>]' contains  the complete
        list  of representatives  of  the R  classes  with the same image
        orbit as the R class 'orbitClasses[<l>]'.
        
'lTrans':\\
        a  list of lists,   where 'lTrans[<l>][<k>]' is  a transformation
        $\alpha$ such that 'lTrans[<l>][<k>] \*\ rClassReps[<l>][<k>]' is
        an element of the R class 'orbitClasses[<l>]'.
        
'kernels':\\
        a  list of lists, where  'kernels[<l>][<k>]' is the common kernel
        of the elements in the R class of 'rClassReps[<l>][<k>]'.

%%%%%%%%%%%%%%%%%%%%%%%%%%%%%%%%%%%%%%%%%%%%%%%%%%%%%%%%%%%%%%%%%%%%%%%%%%%%%
%%
%E  Emacs . . . . . . . . . . . . . . . . . . . . . . . local emacs variables
%%
%%  Local Variables:
%%  mode:               LaTeX
%%  outline-regexp:     "\\\\Chapter\\|\\\\Section"
%%  fill-column:        73
%%  TeX-master:         "manual"
%%  eval:               (outline-minor-mode)
%%  End:
%%

