%%%%%%%%%%%%%%%%%%%%%%%%%%%%%%%%%%%%%%%%%%%%%%%%%%%%%%%%%%%%%%%%%%%%%%%%%%%%%
%%
%A  chvutil.tex       CHEVIE documentation       Meinolf Geck, Frank Luebeck,
%A                                                Jean Michel, G"otz Pfeiffer
%%
%Y  Copyright (C) 1992 - 2010  Lehrstuhl D f\"ur Mathematik, RWTH Aachen, IWR
%Y  der Universit\"at Heidelberg, University of St. Andrews, and   University
%Y  Paris VII.
%%
%%  This  file  contains  utility functions.
%%%%%%%%%%%%%%%%%%%%%%%%%%%%%%%%%%%%%%%%%%%%%%%%%%%%%%%%%%%%%%%%%%%%%%%%%

\Chapter{Partitions and symbols}

The  functions  described  below,  used  in  various  parts of the \CHEVIE\
package, sometimes duplicate or have similar functions to some functions in
other packages (like the SPECHT package). It is hoped that a review of this
area will be done in the future.

The combinatorial objects dealt with here are *partitions*, *beta-sets* and
*symbols*.  A  partition  in  \CHEVIE\  is  a  decreasing  list of strictly
positive  integers  $p_1\ge  p_2\ge  \ldots  p_n>0$, represented as a \GAP\
list.  A beta-set is  a \GAP\ Set  of positive integers,  up to the *shift*
equivalence  relation. This equivalence relation  is the transitive closure
of the elementary equivalence of $[s_1,\ldots,s_n]$ and
$[0,1+s_1,\ldots,1+s_n]$. An equivalence class has exactly one member which
does not contain $0$\: it is called the normalized beta-set. To a partition
$p_1\ge  p_2\ge\ldots\ge p_n>0$ is associated  a beta-set, whose normalized
representative   is  $p_n,p_{n-1}+1,\ldots,p_1+n-1$.  Conversely,  to  each
beta-set is associated a partition, the one giving by the above formula its
normalized representative.

A  symbol  is  a  list  $S=[S_1,..,S_n]$  of  beta-sets,  taken  modulo the
equivalence  relation  generated  by  two  elementary  equivalences\:\  the
simultaneous shift of all beta-sets, and the cyclic permutation of the list
(in  the  particular  case  where  $n=2$  it  is  thus an unordered pair of
beta-sets). This time there is a unique normalized symbol where 0 is not in
the intersection of the $S_i$.

A  basic invariant  attached to symbols is the *rank*, defined as

|Sum(S,Sum)-QuoInt((Sum(S,Length)-1)*(Sum(S,Length)-Length(S)+1),2*Length(S))|

Another  function attached to symbols is the *shape* 'List(S,Length)'; when
$n=2$  one can assume that $S_1$ has at  least the same length as $S_2$ and
the   difference  of  cardinals   'Length(S[1])-Length(S[2])',  called  the
*defect*, is then an invariant of the symbol.

Partitions  and pairs  of partitions  are parameters  for characters of the
Weyl groups of classical types, and tuples of partitions are parameters for
characters of imprimitive complex reflection groups. Symbols with two lines
are  parameters for the unipotent characters of classical Chevalley groups,
and more general symbols for the unipotent characters of Spetses associated
to  complex reflection  groups. The  rank of  the symbol is the semi-simple
rank of the corresponding Chevalley group or Spetses.

Symbols  of rank  $n$ and  defect $0$  parameterize characters  of the Weyl
group  of type  $D_n$, and  symbols of  rank $n$  and defect divisible by 4
parameterize  unipotent characters of split  orthogonal groups of dimension
$2n$.  Symbols of rank $n$ and defect congruent to $2 \pmod 4$ parameterize
unipotent  characters  of  non-split  orthogonal  groups of dimension $2n$.
Symbols  of rank  $n$ and  defect $1$  parameterize characters  of the Weyl
group  of  type  $B_n$,  and  finally  symbols  of  rank $n$ and odd defect
parameterize unipotent characters of symplectic groups of dimension $2n$ or
orthogonal groups of dimension $2n+1$.

%%%%%%%%%%%%%%%%%%%%%%%%%%%%%%%%%%%%%%%%%%%%%%%%%%%%%%%%%%%%%%%%%%%%%%%%%
\Section{Compositions}%
\index{Compositions}%

'Compositions( <n>[,<i>] )'

Returns  the list of compositions of the integer <n> (the compositions with
<i> parts if a second argument <i> is given).

|    gap> Compositions(4);
    [ [ 1, 1, 1, 1 ], [ 2, 1, 1 ], [ 1, 2, 1 ], [ 3, 1 ], [ 1, 1, 2 ],
      [ 2, 2 ], [ 1, 3 ], [ 4 ] ]
    gap> Compositions(4,2);
    [ [ 3, 1 ], [ 2, 2 ], [ 1, 3 ] ]|

%%%%%%%%%%%%%%%%%%%%%%%%%%%%%%%%%%%%%%%%%%%%%%%%%%%%%%%%%%%%%%%%%%%%%%%%%
\Section{PartBeta}%
\index{PartBeta}%

'PartBeta( <b> )'

Here  <b>  is  an  increasing  list  of  integers  representing a beta-set.
'PartBeta' returns corresponding the partition (see the introduction of the
section for definitions).

|    gap> PartBeta([0,4,5]);
    [ 3, 3 ]|

%%%%%%%%%%%%%%%%%%%%%%%%%%%%%%%%%%%%%%%%%%%%%%%%%%%%%%%%%%%%%%%%%%%%%%%%%
\Section{ShiftBeta}%
\index{ShiftBeta}%

'ShiftBeta( <b>, <n> )'

Here  <b>  is  an  increasing  list  of  integers  representing a beta-set.
'ShiftBeta'  returns the  set shifted  by <n>  (see the introduction of the
section for definitions).

|    gap> ShiftBeta([4,5],3);
    [ 0, 1, 2, 7, 8 ]|

%%%%%%%%%%%%%%%%%%%%%%%%%%%%%%%%%%%%%%%%%%%%%%%%%%%%%%%%%%%%%%%%%%%%%%%%%
\Section{PartitionTupleToString}%
\index{PartitionTupleToString}%

'PartitionTupleToString( <tuple> )'

converts the partition tuple <tuple>  to a string where the partitions
are separated by a dot.

|    gap> d:=PartitionTuples(3,2);
    [ [ [ 1, 1, 1 ], [  ] ], [ [ 1, 1 ], [ 1 ] ], [ [ 1 ], [ 1, 1 ] ],
      [ [  ], [ 1, 1, 1 ] ], [ [ 2, 1 ], [  ] ], [ [ 1 ], [ 2 ] ],
      [ [ 2 ], [ 1 ] ], [ [  ], [ 2, 1 ] ], [ [ 3 ], [  ] ],
      [ [  ], [ 3 ] ] ]
    gap>  for i in d do
    >      Print( PartitionTupleToString( i ),"   ");
    >  od; Print("\n");
    111.   11.1   1.11   .111   21.   1.2   2.1   .21   3.   .3|

%%%%%%%%%%%%%%%%%%%%%%%%%%%%%%%%%%%%%%%%%%%%%%%%%%%%%%%%%%%%%%%%%%%%%%%%%
\Section{SymbolPartitionTuple}%
\index{SymbolPartitionTuple}%

'SymbolPartitionTuple( <p>, <s>)'

returns the symbol of shape <s> associated to partition tuple <p>.

In the most general case, <s> is a list of positive integers of same length
as <p> and the BetaSets for <p> are shifted accordingly (a constant integer
may be added to <s> to make the shifts possible).

When  <s> is a  positive integer it  is interpreted as  |[s,0,0,...]| and a
negative integer is interpreted as |[0,-s,-s,....]| so when <p> is a double
partition  one gets the  symbol of defect  <s> associated to  <p>; as other
uses  the principal  series of  G(e,1,r) is |SymbolPartitionTuple(p,1)| and
that of G(e,e,r) is |SymbolPartitionTuple(p,0)|.

Note. The function works also for periodic <p> for G(e,e,r) provided $s=0$.

|    gap> SymbolPartitionTuple([[1,2],[1]],1);
    [ [ 2, 2 ], [ 1 ] ]
    gap> SymbolPartitionTuple([[1,2],[1]],0);
    [ [ 2, 2 ], [ 0, 2 ] ]
    gap> SymbolPartitionTuple([[1,2],[1]],-1);
    [ [ 2, 2 ], [ 0, 1, 3 ] ]|

%%%%%%%%%%%%%%%%%%%%%%%%%%%%%%%%%%%%%%%%%%%%%%%%%%%%%%%%%%%%%%%%%%%%%%%%%
\Section{Tableaux}%
\index{Tableaux}%

'Tableaux(<partition tuple> or <partition>)'

returns  the list  of standard  tableaux associated  to the partition tuple
<tuple>,  that  is  a  filling  of  the  associated young diagrams with the
numbers  '[1..Sum(<tuple>,Sum)]' such that the  numbers increase across the
rows and down the columns. If the imput is a single partition, the standard
tableaux for that partition are returned.

|    gap> Tableaux([[2,1],[1]]);
    [ [ [ [ 2, 4 ], [ 3 ] ], [ [ 1 ] ] ],
      [ [ [ 1, 4 ], [ 3 ] ], [ [ 2 ] ] ],
      [ [ [ 1, 4 ], [ 2 ] ], [ [ 3 ] ] ],
      [ [ [ 2, 3 ], [ 4 ] ], [ [ 1 ] ] ],
      [ [ [ 1, 3 ], [ 4 ] ], [ [ 2 ] ] ],
      [ [ [ 1, 2 ], [ 4 ] ], [ [ 3 ] ] ],
      [ [ [ 1, 3 ], [ 2 ] ], [ [ 4 ] ] ],
      [ [ [ 1, 2 ], [ 3 ] ], [ [ 4 ] ] ] ]
    gap> Tableaux([2,2]);
    [ [ [ 1, 3 ], [ 2, 4 ] ], [ [ 1, 2 ], [ 3, 4 ] ] ]|

%%%%%%%%%%%%%%%%%%%%%%%%%%%%%%%%%%%%%%%%%%%%%%%%%%%%%%%%%%%%%%%%%%%%%%%%%%%%%
\Section{DefectSymbol}
\index{DefectSymbol}

'DefectSymbol( <s> )'

Let  <s>'=[S,T]'  be  a  symbol  given  as a  pair  of  lists  (see  the
introduction to the  section). 'DefectSymbol' returns the  defect of <s>,
equal to 'Length(S)-Length(T)'.

|    gap> DefectSymbol([[1,2],[1,5,6]]);
    -1|

%%%%%%%%%%%%%%%%%%%%%%%%%%%%%%%%%%%%%%%%%%%%%%%%%%%%%%%%%%%%%%%%%%%%%%%%%%%%%
\Section{RankSymbol}
\index{RankSymbol}

'RankSymbol( <s> )'

Let <s>$=[S_1,..,S_n]$  be a symbol given  as a tuple of  lists (see the
introduction to the section). 'RankSymbol' returns the rank of <s>.

|    gap> RankSymbol([[1,2],[1,5,6]]);
    11|

%%%%%%%%%%%%%%%%%%%%%%%%%%%%%%%%%%%%%%%%%%%%%%%%%%%%%%%%%%%%%%%%%%%%%%%%%
\Section{Symbols}
\index{Symbols}

'Symbols( <n>, <d> )'

Returns  the list of all  two-line symbols of defect  <d> and rank <n> (see
the  introduction for definitions). If $d=0$ the symbols with equal entries
are  returned  twice,  represented  as  the  first  entry,  followed by the
repetition factor 2 and an ordinal number 0 or 1, so that 'Symbols(<n>, 0)'
returns  a set of parameters  for the characters of  the Weyl group of type
$D_n$.

|    gap> Symbols(2,1);
    [ [ [ 1, 2 ], [ 0 ] ], [ [ 0, 2 ], [ 1 ] ], [ [ 0, 1, 2 ], [ 1, 2 ] ],
      [ [ 2 ], [  ] ], [ [ 0, 1 ], [ 2 ] ] ]
    gap> Symbols(4,0);
    [ [ [ 1, 2 ], 2, 0 ], [ [ 1, 2 ], 2, 1 ],
      [ [ 0, 1, 3 ], [ 1, 2, 3 ] ], [ [ 0, 1, 2, 3 ], [ 1, 2, 3, 4 ] ],
      [ [ 1, 2 ], [ 0, 3 ] ], [ [ 0, 2 ], [ 1, 3 ] ],
      [ [ 0, 1, 2 ], [ 1, 2, 4 ] ], [ [ 2 ], 2, 0 ], [ [ 2 ], 2, 1 ],
      [ [ 0, 1 ], [ 2, 3 ] ], [ [ 1 ], [ 3 ] ], [ [ 0, 1 ], [ 1, 4 ] ],
      [ [ 0 ], [ 4 ] ] ]|

%%%%%%%%%%%%%%%%%%%%%%%%%%%%%%%%%%%%%%%%%%%%%%%%%%%%%%%%%%%%%%%%%%%%%%%%%
\Section{SymbolsDefect}
\index{SymbolsDefect}

'SymbolsDefect( <e>, <r>, <def> , <inh>)'

Returns  the list of  symbols defined by  Malle for Unipotent characters of
imprimitive  Spetses. Returns <e>-symbols of  rank <r>, defect <def> (equal
to  0 or  1) and  content equal  to <inh>  modulo <e>. Thus the symbols for
unipotent  characters of  'G(d,1,r)' are  given by 'SymbolsDefect(d,r,0,1)'
and those for unipotent characters of 'G(e,e,r)' by
'SymbolsDefect(e,r,0,0)'.

|    gap> SymbolsDefect(3,2,0,1);
    [ [ [ 1, 2 ], [ 0 ], [ 0 ] ], [ [ 0, 2 ], [ 1 ], [ 0 ] ],
      [ [ 0, 2 ], [ 0 ], [ 1 ] ], [ [ 0, 1, 2 ], [ 1, 2 ], [ 0, 1 ] ],
      [ [ 0, 1 ], [ 1 ], [ 1 ] ], [ [ 0, 1, 2 ], [ 0, 1 ], [ 1, 2 ] ],
      [ [ 2 ], [  ], [  ] ], [ [ 0, 1 ], [ 2 ], [ 0 ] ],
      [ [ 0, 1 ], [ 0 ], [ 2 ] ], [ [ 1 ], [ 0, 1, 2 ], [ 0, 1, 2 ] ],
      [ [  ], [ 0, 2 ], [ 0, 1 ] ], [ [  ], [ 0, 1 ], [ 0, 2 ] ],
      [ [ 0 ], [  ], [ 0, 1, 2 ] ], [ [ 0 ], [ 0, 1, 2 ], [  ] ] ]
    gap> List(last,StringSymbol);
    [ "(12,0,0)", "(02,1,0)", "(02,0,1)", "(012,12,01)", "(01,1,1)",
      "(012,01,12)", "(2,,)", "(01,2,0)", "(01,0,2)", "(1,012,012)",
      "(,02,01)", "(,01,02)", "(0,,012)", "(0,012,)" ]
    gap> SymbolsDefect(3,3,0,0);
    [ [ [ 1 ], 3, 0 ], [ [ 1 ], 3, 1 ], [ [ 1 ], 3, 2 ], 
      [ [ 0, 1 ], [ 1, 2 ], [ 0, 2 ] ], [ [ 0, 1 ], [ 0, 2 ], [ 1, 2 ] ], 
      [ [ 0, 1, 2 ], [ 0, 1, 2 ], [ 1, 2, 3 ] ], [ [ 0 ], [ 1 ], [ 2 ] ], 
      [ [ 0 ], [ 2 ], [ 1 ] ], [ [ 0, 1 ], [ 0, 1 ], [ 1, 3 ] ], 
      [ [ 0 ], [ 0 ], [ 3 ] ], [ [ 0, 1, 2 ], [  ], [  ] ], 
      [ [ 0, 1, 2 ], [ 0, 1, 2 ], [  ] ] ]
    gap> List(last,StringSymbol);
    [ "(1+)", "(1E3)", "(1E3^2)", "(01,12,02)", "(01,02,12)", 
      "(012,012,123)", "(0,1,2)", "(0,2,1)", "(01,01,13)", "(0,0,3)", 
      "(012,,)", "(012,012,)" ]|

%%%%%%%%%%%%%%%%%%%%%%%%%%%%%%%%%%%%%%%%%%%%%%%%%%%%%%%%%%%%%%%%%%%%%%%%%
\Section{CycPolGenericDegreeSymbol}%
\index{CycPolGenericDegreeSymbol}%

'CycPolGenericDegreeSymbol( <s> )'

Let <s>$=[S_1,..,S_n]$  be a symbol given  as a tuple of  lists (see the
introduction to  the section). 'CycPolGenericDegreeSymbol' returns  as a
'CycPol' the generic degree of  the unipotent character parameterized by
<s>.

|    gap> CycPolGenericDegreeSymbol([[1,2],[1,5,6]]);
    1/2q^13P5P6P7P8^2P9P10P11P14P16P18P20P22|

%%%%%%%%%%%%%%%%%%%%%%%%%%%%%%%%%%%%%%%%%%%%%%%%%%%%%%%%%%%%%%%%%%%%%%%%%
\Section{CycPolFakeDegreeSymbol}%
\index{CycPolFakeDegreeSymbol}%

'CycPolFakeDegreeSymbol( <s> )'

Let <s>$=[S_1,..,S_n]$  be a symbol given  as a tuple of  lists (see the
introduction  to the  section).  'CycPolFakeDegreeSymbol'  returns as  a
'CycPol' the  fake degree  of the  unipotent character  parameterized by
<s>.

|    gap> CycPolFakeDegreeSymbol([[1,5,6],[1,2]]);
    q^16P5P7P8P9P10P11P14P16P18P20P22|

%%%%%%%%%%%%%%%%%%%%%%%%%%%%%%%%%%%%%%%%%%%%%%%%%%%%%%%%%%%%%%%%%%%%%%%%%
\Section{LowestPowerGenericDegreeSymbol}%
\index{LowestPowerGenericDegreeSymbol}%

'LowestPowerGenericDegreeSymbol( <s> )'

Let  <s>'=[S1,..,Sn]' be  a symbol  given as  a pair  of lists  (see the
introduction to  the section).  'LowestPowerGenericDegreeSymbol' returns
the  valuation  of  the  generic   degree  of  the  unipotent  character
parameterized by <s>.

|    gap> LowestPowerGenericDegreeSymbol([[1,2],[1,5,6]]);
    13|

%%%%%%%%%%%%%%%%%%%%%%%%%%%%%%%%%%%%%%%%%%%%%%%%%%%%%%%%%%%%%%%%%%%%%%%%%
\Section{HighestPowerGenericDegreeSymbol}%
\index{HighestPowerGenericDegreeSymbol}%

'HighestPowerGenericDegreeSymbol( <s> )'

Let  <s>'=[S1,..,Sn]'  be  a  symbol  given as  a  pair  of  lists  (see
the  introduction  to  the  section).  'HighestPowerGenericDegreeSymbol'
returns  the degree  of the  generic degree  of the  unipotent character
parameterized by <s>.

|    gap> HighestPowerGenericDegreeSymbol([[1,5,6],[1,2]]);
    91|

%%%%%%%%%%%%%%%%%%%%%%%%%%%%%%%%%%%%%%%%%%%%%%%%%%%%%%%%%%%%%%%%%%%%%%%%%
