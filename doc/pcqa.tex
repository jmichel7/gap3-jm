%%%%%%%%%%%%%%%%%%%%%%%%%%%%%%%%%%%%%%%%%%%%%%%%%%%%%%%%%%%%%%%%%%%%%%%%%%%%%
%%
%A  pcqa.tex                   GAP documentation                     Eddie Lo
%A                                                           
%%
%Y  Copyright 1990-1995,  Lehrstuhl D fuer Mathematik,  RWTH Aachen,  Germany
%%
%%  This is a tex file  which describes the GAP  interface functions  for the
%%  polycyclic quotient algorithm package by Eddie Lo.
%%
\Chapter{The Polycyclic Quotient Algorithm Package}
\index{}

This package is written by  Eddie Lo.  The original program is  available
for anonymous ftp at  math.rutgers.edu. The program is an  implementation
of the  Baumslag-Cannonito-Miller  polycyclic quotient algorithm  and  is
written in C. For more details read \cite{BCMa},\cite{BCMb}, Section 11.6
of \cite{Sims94}and \cite{Lo}.

This package contains functions to compute the polycyclic quotients which
appear in the derived series of a finitely presented group.

Currently, there are five functions implemented in this package:\\
    'CallPCQA'            (see "CallPCQA"),\\
    'ExtendPCQA'          (see "ExtendPCQA"),\\
    'AbelianComponent'    (see "AbelianComponent"),\\
    'HirschLength'        (see "HirschLength"),\\
    'ModuleAction'        (see "ModuleAction").

Eddie Lo \\
email\: hlo@math.rutgers.edu 

%%%%%%%%%%%%%%%%%%%%%%%%%%%%%%%%%%%%%%%%%%%%%%%%%%%%%%%%%%%%%%%%%%%%%%%%%
\Section{Installing the PCQA Package}

The PCQA is  written in C  and the  package can  only  be installed under
UNIX.  It has  been tested on  SUNs running SunOS and  on IBM PCs running
FreeBSD 2.1.0.  It requires the  GNU multiple precision arithmetic.  Make
sure that this library is installed before trying to install the PCQA.

If you  got a complete binary  and source distribution  for your machine,
nothing   has to  be  done  if you  want  to use  the  PCQA  for a single
architecture.  If  you want to use  the PCQA  for machines with different
architectures skip the extraction  and compilation  part of this  section
and proceed with the creation of shell scripts described below.

If you  got a complete source distribution,  skip the  extraction part of
this section and proceed with the compilation part below.

In the example we will assume that you, as user 'gap', are installing the
PCQA  package for use by several  users on a network  of two SUNs, called
'bert' and  'tiffy', and a NeXTstation, called   'bjerun'. We assume that
{\GAP}  is also  installed on these   machines following the instructions
given in "Installation of GAP for UNIX".

Note that certain parts  of  the  output  in the examples should  only be
taken as rough outline, especially file sizes and file dates are *not* to
be taken literally.

First of all you  have to get the  file  'pcqa.zoo' (see "Getting  GAP").
Then you must locate the {\GAP} directories containing 'lib/' and 'doc/',
this is usually 'gap3r4p?' where  '?' is to be  be replaced by the  patch
level.

|    gap@tiffy:~ > ls -l
    drwxr-xr-x  11 gap    gap      1024 Nov  8 15:16 gap3r4p3
    -rw-r--r--   1 gap    gap    106307 Jan 24 15:16 pcqa.zoo |

Unpack the  package using 'unzoo' (see "Installation  of  GAP for UNIX").
Note that you  must be in the  directory containing 'gap3r4p?'  to unpack
the  files.   After  you have  unpacked  the source   you  may remove the
*archive-file*.

|    gap@tiffy:~ > unzoo -x pcqa.zoo
    gap@tiffy:~ > ls -l gap3r4p3/pkg/pcqa
    -rw-r--r--   1 gap    gap      3697 Dec 14 15:58 Makefile
    drwxr-xr-x   2 gap    gap      1024 Dec 14 15:57 bin/
    drwxr-xr-x   2 gap    gap      1024 Dec 14 16:12 doc/
    drwxr-xr-x   2 gap    gap      1024 Dec 15 18:28 examples/
    -rw-r--r--   1 gap    gap     11819 Dec 14 13:31 init.g
    drwxr-xr-x   2 gap    gap      3072 Dec 14 16:03 src/ |

Switch into the directory 'src/' and type 'make' to compile the PCQA.  If
the header files for  the GNU multiple  precision arithmetic are *not* in
'/usr/local/include' you must set 'GNUINC' to  the correct directory.  If
the  library  for the     GNU  multiple  precision arithmetic    is *not*
'/usr/local/lib/libgmp.a'  you must set  'GNULIB'.   In our case we first
compile the SUN version.

|    gap@tiffy:~ > cd gap3r4p3/pkg/pcqa/src
    gap@tiffy:../src > make GNUINC=/usr/gnu/include \
                            GNULIB=/usr/gnu/lib/libmp.a
    |\#| you will see a lot of messages |

If you want to use  the PCQA on  multiple architectures you have to  move
the executable to unique name.

|    gap@tiffy:../pcqa > mv bin/pcqa bin/pcqa-sun-sparc-sunos |

Now repeat the compilation  for the NeXTstation. *Do not* forget to clean
up.

|    gap@tiffy:../pcqa > rlogin bjerun
    gap@bjerun:~ > cd gap3r4p3/pkg/pcqa/src
    gap@bjerun:../src > make clean
    gap@bjerun:../src > make
    |\#| you will see a lot of messages
    gap@bjerun:../src > mv bin/pcqa ../bin/pcqa-next-m68k-mach
    gap@bjerun:../src > exit
    gap@tiffy:../src > |

Switch into the  subdirectory 'bin/' and  create a script which will call
the correct binary for each machine.  A skeleton shell script is provided
in 'bin/pcqa.sh'.

|    gap@tiffy:../src > cd ..
    gap@tiffy:../pcqa > cat bin/pcqa.sh
    |\#|!/bin/csh
    switch ( `hostname` )
      case 'bert':
      case 'tiffy':
        exec $0-dec-mips-ultrix $* ;
        breaksw ;
      case 'bjerun':
        exec $0-next-m68k-mach $* ;
        breaksw ;
      default:
        echo "pcqa: sorry, no executable exists for this machine" ;
        breaksw ;
    endsw
    |<ctr>-'D'|
    gap@tiffy:../pcqa > chmod 755 bin/pcqa|

Now it is time to test the package.

|    gap> RequirePackage("pcqa");
    gap> f := FreeGroup(2);
    Group( f.1, f.2 )
    gap> g := f/[f.1*f.2*f.1*f.2^-1*f.1^-1*f.2^-1];;
    gap> ds := CallPCQA( g, 2 );
    rec(
      isDerivedSeries := true,
      DerivedLength := 2,
      QuotientStatus := 0,
      PolycyclicPresentation := rec(
	  Generators := 3,
	  ExponentList := [ 0, 0, 0 ],
	  ppRelations := [ [ [ 0, 1, -1 ], [ 0, 1, 0 ] ],
	                   [ [ 0, 0, 1 ] ] ],
	  pnRelations := [ [ [ 0, -1, 1 ], [ 0, -1, 0 ] ],
	                   [ [ 0, 0, -1 ] ] ],
	  npRelations := [ [ [ 0, 0, 1 ], [ 0, -1, 1 ] ],
	                   [ [ 0, 0, 1 ] ] ],
	  nnRelations := [ [ [ 0, 0, -1 ], [ 0, 1, -1 ] ],
	                   [ [ 0, 0, -1 ] ] ],
	  PowerRelations := [  ] ),
      Homomorphisms := rec(
	  Epimorphism := [ [ 1, 1, 0 ], [ 1, 0, 0 ] ],
	  InverseMap := [ [ [ 2, 1 ] ], [ [ 3, -1 ], [ 1, 1 ] ],
	                  [ [ 1, 1 ], [ 3, -1 ] ] ] ),
      MembershipArray := [ 1, 3 ] )
    gap> ExtendPCQA( g, ds.PolycyclicPresentation, ds.Homomorphisms );
    rec(
        QuotientStatus := 5 ) |

%%%%%%%%%%%%%%%%%%%%%%%%%%%%%%%%%%%%%%%%%%%%%%%%%%%%%%%%%%%%%%%%%%%%%%%%%

\Section{Input format}

This package uses the finitely presented group data structure  defined in
GAP (see *Finitely Presented Groups*). It also defines and uses two types
of  data structures.  One data structure defines a  consistent polycyclic
presentation of a  polycyclic group and the other defines a  homomorphism
and an inverse map between the finitely presented group and its quotient.

%%%%%%%%%%%%%%%%%%%%%%%%%%%%%%%%%%%%%%%%%%%%%%%%%%%%%%%%%%%%%%%%%%%%%%%%%

\Section{CallPCQA}
\index{CallPCQA}

'CallPCQA( <G, n> )'

This  function attempts to compute the   quotient of a finitely presented
group $G$ by  the $n+1$-st term of its  derived series.  A record made up
of   four   fields   is  returned.     The    fields are *DerivedLength*,
*QuotientStatus* , *PolycyclicPresentation* and *Homomorphisms* .  If the
quotient is not polycyclic then the field  *QuotientStatus* will return a
positive number.    The group element represented   by the module element
with that positive  number generates normally  a subgroup which cannot be
finitely generated.  In this  case the field *DerivedLength* will  denote
the  biggest integer $k$ such  that the quotient of   $G$ by the $k+1$-st
term in the derived  series  is polycyclic.  The appropriate   polycyclic
presentation  and maps will be   returned. If the field  *QuotientStatus*
returns -1, then for some number $k \< n$, the $k$-th term of the derived
series is  the same as  the $k+1$-st term of  the derived  series. In the
remaining case *QuotientStatus* returns 0.

The  field *PolycyclicPresentation* is a  record made up of seven fields.
The various conjugacy relations are stored in  the fields *ppRelations* ,
*pnRelations*, *npRelations* and *nnRelations*. Each of these four fields
is  an array of  exponent  sequences which correspond  to the appropriate
left sides of the conjugacy relations .  If $a_1,a_2,...,a_n$ denotes the
polycyclic  generators  and $A_1,A_2,...,A_n$  their respective inverses,
then the field *ppRelations* stores the relations of the form $a_j^{a_i}$
with  $i  \<  j$, *pnRelations*     stores  the relations  of  the   form
$A_j^{a_i}$, *npRelations* stores the  relations of the form  $a_j^{A_i}$
and  *nnRelations* stores  the relations  of  the form  $A_j^{A_i}$.  The
positive  and negative power relations  are  stored together similarly in
the field *PowerRelations*.  The field *Generators* denotes the number of
polycyclic  generators in the presentation  and  the field *ExponentList*
contains the exponent   of the power  relations.   If there  is no  power
relation  involving a  generator,then   the  corresponding entry  in  the
*ExponentList* is equal to 0.

The field  *Homomorphisms* consists of a  homomorphism from the  finitely
presented group to the polycyclic group and an inverse map backward.  The
field  *Epimorphism*  stores the image of the generators of the  finitely
presented group as exponent sequences of the polycyclic group . The field
*InverseMap* stores a preimage of the polycyclic generators as a  word in
the finitely presented group.

|
    gap> F := FreeGroup(2);
    Group( f.1, f.2 )
    gap> G := F/[F.1*F.2*F.1*F.2^-1*F.1^-1*F.2^-1];
    Group( f.1, f.2 )
    gap> ans := CallPCQA(G,2);
    rec(
      DerivedLength := 2,
      QuotientStatus := 0,
      PolycyclicPresentation := rec(
          Generators := 3,
          ExponentList := [ 0, 0, 0 ],
          ppRelations := [ [ [ 0, 1, -1 ], [ 0, 1, 0 ] ],
	                   [ [ 0, 0, 1 ] ] ],
          pnRelations := [ [ [ 0, -1, 1 ], [ 0, -1, 0 ] ],
	                   [ [ 0, 0, -1 ] ] ],
          npRelations := [ [ [ 0, 0, 1 ], [ 0, -1, 1 ] ],
	                   [ [ 0, 0, 1 ] ] ],
          nnRelations := [ [ [ 0, 0, -1 ], [ 0, 1, -1 ] ],
	                   [ [ 0, 0, -1 ] ] ],
          PowerRelations := [  ] ),
      Homomorphisms := rec(
          Epimorphism := [ [ 1, 1, 0 ], [ 1, 0, 0 ] ],
          InverseMap := [ [ [ 2, 1 ] ], [ [ 3, -1 ], [ 1, 1 ] ],
	                  [ [ 1, 1 ], [ 3, -1 ] ] ] ),
      MembershipArray := [ 1, 3 ] )
|

%%%%%%%%%%%%%%%%%%%%%%%%%%%%%%%%%%%%%%%%%%%%%%%%%%%%%%%%%%%%%%%%%%%%%%%%%

\Section{ExtendPCQA}
\index{ExtendPCQA}

'ExtendPCQA( <G, CPP, HOM, m, n> )'

This function takes as input a finitely presented group $G$, a consistent
polycyclic presentation $CPP$ ("CallPCQA") of a polycyclic quotient $G/N$
of $G$, an epimorphism and an inverse map as in the field *Homomorphisms*
in "CallPCQA". It determines whether the quotient $G/[N,N]$ is polycyclic
and  returns the flag *QuotientStatus* . It also  returns the  polycyclic
presentation and the appropriate homomorphism and map if the quotient  is
polycyclic.

When the parameter $m$ is a positive number the quotient  $G/[N,N]N^m$ is
computed.  When it is a negative number,  and if $K/[N,N]$ is the torsion
part of $N/[N,N]$, then the quotient $G/[N,N]K$ is computed.  The default
case is when $m = 0$.  If there are only  three arguments in the function
call, $m$ will be taken to be zero.

When the parameter  $n$ is a nonzero number,  the quotient  $G/[N,G]$  is
computed instead.  Otherwise the quotient $G/[N,N]$ is computed.  If this
argument is not assigned by the user,  then $n$ is set to zero. Different
combinations of  $m$ and  $n$ give different quotients. For example, when
*ExtendPCQA* is called with $m = 6$ and $n = 1$,the quotient $G/[N,G]N^6$
is computed.

|
    gap> ExtendPCQA(G,ans.PolycyclicPresentation,ans.Homomorphisms);
    rec(
       QuotientStatus := 5 )
    gap> ExtendPCQA(G,ans.PolycyclicPresentation,ans.Homomorphisms,6,1);
    rec(
      QuotientStatus := 0,
      PolycyclicPresentation := rec(
        Generators := 4,
        ExponentList := [ 0, 0, 0, 6 ],
        ppRelations := [ [[ 0, 1, -1, 0 ],[ 0, 1, 0, 0 ],[ 0, 0, 0, 1 ]],
                         [[ 0, 0, 1, 1 ],[ 0, 0, 0, 1 ]],
                         [[ 0, 0, 0, 1 ]] ],
        pnRelations := [ [[ 0, -1, 1, 5 ],[ 0, -1, 0, 0 ],[ 0, 0, 0, 5]],
                         [[ 0, 0, -1, 5 ],[ 0, 0, 0, 5 ]],
                         [[ 0, 0, 0, 5 ]] ],
        npRelations := [ [[ 0, 0, 1, 0 ],[ 0, -1, 1, 0 ],[ 0, 0, 0, 1 ]],
                         [[ 0, 0, 1, 5 ],[ 0, 0, 0, 1 ]],
                         [[ 0, 0, 0, 1 ]] ],
        nnRelations := [ [[ 0, 0, -1, 0 ],[ 0, 1, -1, 5 ],[ 0, 0, 0, 5]],
                         [[ 0, 0, -1, 1 ],[ 0, 0, 0, 5 ]],
                         [[ 0, 0, 0, 5 ]] ],
        PowerRelations := [ ,,,,,, [ 0, 0, 0, 0 ], [ 0, 0, 0, 5 ] ] ),
      Homomorphisms := rec(
        Epimorphism := [ [ 1, 1, 0, 0 ], [ 1, 0, 0, 0 ] ],
        InverseMap :=
          [ [[ 2, 1 ]], [[ 3, -1 ],[ 1, 1 ]], [[ 1, 1 ],[ 3, -1 ]],
            [[ 5, -1 ],[ 4, -1 ],[ 5, 1 ],[ 4, 1 ]] ] ),
      Next := 4 )
|

%%%%%%%%%%%%%%%%%%%%%%%%%%%%%%%%%%%%%%%%%%%%%%%%%%%%%%%%%%%%%%%%%%%%%%%%%

\Section{AbelianComponent}
\index{AbelianComponent}

'AbelianComponent( <QUOT> )'

This function  takes  as input the output of a  *CallPCQA*  function call
(see "CallPCQA") or an  *ExtendPCQA* function call (see "ExtendPCQA") and
returns the structure of the abelian groups  which appear as quotients in
the derived series.  The structure of each of these quotients is given by
an array of nonnegative integers.Read the section on *ElementaryDivisors*
for details.

|
    gap> F := FreeGroup(3);
    Group( f.1, f.2, f.3 )
    gap> G := F/[F.1*F.2*F.1*F.2,F.2*F.3^2*F.2*F.3,F.3^6];
    Group( f.1, f.2, f.3 )
    gap> quot := CallPCQA(G,2);;
    gap> AbelianComponent(quot);
    [ [ 1, 2, 12 ], [ 0, 0, 0, 0, 0, 0, 0, 0, 0, 0, 0 ] ]
|

%%%%%%%%%%%%%%%%%%%%%%%%%%%%%%%%%%%%%%%%%%%%%%%%%%%%%%%%%%%%%%%%%%%%%%%%%

\Section{HirschLength}
\index{HirschLength}

'HirschLength( <CPP> )'

This function takes as  input a  consistent polycyclic presentation  (see
"CallPCQA") and returns the Hirsch length of the group presented.

|
    gap> HirschLength(quot.PolycyclicPresentation);
    11
|

%%%%%%%%%%%%%%%%%%%%%%%%%%%%%%%%%%%%%%%%%%%%%%%%%%%%%%%%%%%%%%%%%%%%%%%%%

\Section{ModuleAction}
\index{ModuleAction}

'ModuleAction( <QUOT> )'

This function  takes as input the   output of a *CallPCQA*  function call
(see "CallPCQA") or an *ExtendPCQA* function call (see "ExtendPCQA").  If
the quotient $G/[N,N]$  returned by the function  call is polycyclic then
*ModuleAction*   computes the  action    of  the  polycyclic   generators
corresponding to $G/N$ on  the polycyclic generators  of $N/[N,N]$.   The
result is returned as an array of matrices.  Notice that the Smith normal
form of  $G/[N,N]$ is returned by the  function *CallPCQA* as part of the
polycyclic presentation.

|
gap> ModuleAction(quot);
[ [ [ 1, 0, 0, 0, 0, 0, 0, 1, 1, 1, 0 ],
    [ 0, 0, 0, 0, 0, 0, -1, 0, 0, 0, 0 ],
    [ 0, 0, 0, 0, 0, 0, 0, -1, 0, 0, 0 ], 
    [ 0, 0, 0, 0, 0, 0, 0, 0, -1, 0, 0 ], 
    [ 0, 0, 0, 0, 0, 0, 0, 0, 0, -1, 0 ], 
    [ 0, 0, 0, 0, 0, 0, 0, 0, 0, 0, -1 ], 
    [ 0, 0, 0, 0, -1, 0, 0, 0, 0, 0, 0 ], 
    [ 0, 0, 0, 0, 0, -1, 0, 0, 0, 0, 0 ], 
    [ 0, 1, 1, 1, 1, 1, 0, 0, 0, 0, 0 ], 
    [ 0, -1, 0, 0, 0, 0, 0, 0, 0, 0, 0 ], 
    [ 0, 0, -1, 0, 0, 0, 0, 0, 0, 0, 0 ] ], 
  [ [ -1, 0, -1, -1, -1, 0, 0, 0, 0, 0, 0 ], 
    [ 0, 0, 0, 0, 0, 0, 0, 0, 0, 1, 0 ],
    [ 0, 0, 0, 0, 0, 0, 0, 0, 0, 0, 1 ],
    [ 0, 0, 0, 0, 0, 0, -1, -1, -1, -1, -1 ], 
    [ 0, 0, 0, 0, 0, 0, 1, 0, 0, 0, 0 ],
    [ 0, 0, 0, 0, 0, 0, 0, 1, 0, 0, 0 ],
    [ 0, 1, 0, 0, 0, 0, 0, 0, 0, 0, 0 ], 
    [ 0, 0, 1, 0, 0, 0, 0, 0, 0, 0, 0 ],
    [ 0, 0, 0, 1, 0, 0, 0, 0, 0, 0, 0 ],
    [ 0, 0, 0, 0, 1, 0, 0, 0, 0, 0, 0 ], 
    [ 0, 0, 0, 0, 0, 1, 0, 0, 0, 0, 0 ] ], 
  [ [ 1, -1, 0, 0, 0, 0, 0, 0, 0, 1, 0 ], 
    [ 0, -1, -1, -1, -1, -1, 0, 0, 0, 0, 0 ], 
    [ 0, 1, 0, 0, 0, 0, 0, 0, 0, 0, 0 ],
    [ 0, 0, 1, 0, 0, 0, 0, 0, 0, 0, 0 ],
    [ 0, 0, 0, 1, 0, 0, 0, 0, 0, 0, 0 ], 
    [ 0, 0, 0, 0, 1, 0, 0, 0, 0, 0, 0 ], 
    [ 0, 0, 0, 0, 0, 0, -1, -1, -1, -1, -1 ], 
    [ 0, 0, 0, 0, 0, 0, 1, 0, 0, 0, 0 ],
    [ 0, 0, 0, 0, 0, 0, 0, 1, 0, 0, 0 ],
    [ 0, 0, 0, 0, 0, 0, 0, 0, 1, 0, 0 ], 
    [ 0, 0, 0, 0, 0, 0, 0, 0, 0, 1, 0 ] ] ]
|


