%%%%%%%%%%%%%%%%%%%%%%%%%%%%%%%%%%%%%%%%%%%%%%%%%%%%%%%%%%%%%%%%%%%%%%%%%%%%%
%%
%A  GAP documentation                                   Goetz.Pfeiffer@UCG.IE
%%
%A  $Id: transfor.tex,v 2.0 1997/05/05 16:35:52 goetz Exp $
%%
%Y  Copyright (C) 1997, Mathematics Dept, University College Galway, Ireland.
%%
%%  This file documents the functions for transformations on $n$ points.
%%
\Chapter{Transformations}

A transformation of degree  $n$ is a  map from the  set $\{1,  ... , n\}$
into itself.   Thus a transformation $\alpha$  of degree $n$ associates a
positive integer $i^\alpha$ less than or equal  to $n$ to each number $i$
between $1$ and $n$.

The degree of a transformation may not be larger than $2^{28}-1$ which is
(currently) the highest index that can be accessed in a list.

Special   cases  of    transformations   are permutations    (see chapter
\"Permutations\").  However,  a   permutation must   be  converted to   a
transformation before  most   of  the  functions   in  this  chapter  are
applicable.

The product of transformations is defined via  composition of maps.  Here
transformations are multiplied in such a way that they act from the right
on   the set  $\{1,    ... ,   n\}$.   That   is,  the  product  of   the
transformations $\alpha$ and $\beta$ of degree $n$ is defined by
\[
  i\^(\alpha\beta) = (i\^\alpha)\^\beta\quad\mbox{for all }i = 1, ... ,n.
\]
With respect  to this  multiplication  the set of all  transformations of
degree $n$ forms a monoid\:\ the full transformation monoid of degree $n$
(see chapter "Transformation Monoids").

Each transformation of  degree $n$ is  considered an element of  the full
transformation  monoid  of degree  $n$ although   it is  not necessary to
construct  a   full  transformation     monoid  before   working     with
transformations.  But you can  only multiply two transformations  if they
have the same  degree.  You  can,  however, multiply a transformation  of
degree $n$ by a permutation of degree $n$.

Transformations are entered and displayed by giving their lists of images
as an argument to the function 'Transformation'.

|    gap> Transformation( [ 3, 3, 4, 2, 5 ] );
    Transformation( [ 3, 3, 4, 2, 5 ] )
    gap> Transformation( [ 3, 3, 2 ] ) * Transformation( [ 1, 2, 1 ] );
    Transformation( [ 1, 1, 2 ] )|

This  chapter  describes functions that  deal  with transformations.  The
first sections describe the representation  of a transformation in {\GAP}
(see  "More  about   Transformations")   and   how a  transformation   is
constructed as a {\GAP} object (see "Transformation").  The next sections
describe  the comparisons and  the   operations which are available   for
transformations (see "Comparisons of Transformations" and "Operations for
Transformations").   There are a  function  to test  whether an arbitrary
object  is  a transformation (see "IsTransformation")  and  a function to
construct  the   identity    transformation   of a   given   degree  (see
"IdentityTransformation").   Then   there  are   functions   that compute
attributes of transformations (see "Degree of a Transformation", "Rank of
a  Transformation",   "Image of a   Transformation",  and  "Kernel  of  a
Transformation").   Finally,   there   are a  function    that converts a
permutation to a transformation (see "TransPerm") and a function that, if
possible converts a transformation to a permutation (see "PermTrans").

The functions described here are in the file '\"transfor.g\"'.

%%%%%%%%%%%%%%%%%%%%%%%%%%%%%%%%%%%%%%%%%%%%%%%%%%%%%%%%%%%%%%%%%%%%%%%%%
\Section{More about Transformations}

A transformation $\alpha$ on $n$ points is completely defined by its list
of images.   It   is stored as  a  record  with   the  following category
components.

'isTransformation': \\
        is always set to 'true'.

'domain': \\
        is always set to 'Transformations'.

Moreover it has the identification component

'images':\\
        containing  the list of  images in  such a way that  $i\^\alpha =
        \alpha.'images[i]'$ for all $i \leq n$.

The multiplication    of  these   transformations   can  be   efficiently
implemented by using the sublist operator '\{\ \}'.  The product '<r> \*\ 
<l>'  of    two  transformations  <l>   and   <r>  can  be   computed  as
'Transformation( <r>.images\{ <l>.images \}  )'.  Note that the order has
been chosen to have transformations act from the right on their domain.

%%%%%%%%%%%%%%%%%%%%%%%%%%%%%%%%%%%%%%%%%%%%%%%%%%%%%%%%%%%%%%%%%%%%%%%%%
\Section{Transformation}

'Transformation( <lst> )'

'Transformation' returns the transformation  defined by the list <lst> of
images.  Each entry in <lst> must be a positive integer not exceeding the
length of <lst>.

|    gap> Transformation( [ 1, 4, 4, 2 ] );
    Transformation( [ 1, 4, 4, 2 ] )|

%%%%%%%%%%%%%%%%%%%%%%%%%%%%%%%%%%%%%%%%%%%%%%%%%%%%%%%%%%%%%%%%%%%%%%%%%
\Section{IdentityTransformation}

'IdentityTransformation( <n> )'

'IdentityTransformation'  returns,  for any  positive  <n>, the  identity
transformation of degree $n$.

|    gap> IdentityTransformation( 4 );
    Transformation( [ 1 .. 4 ] )|

The identity transformation  of degree $n$  acts  as the  identity in the
full transformation monoid of degree $n$ (see "FullTransMonoid").

%%%%%%%%%%%%%%%%%%%%%%%%%%%%%%%%%%%%%%%%%%%%%%%%%%%%%%%%%%%%%%%%%%%%%%%%%
\Section{Comparisons of Transformations}%
\index{equality!of transformations}%
\index{ordering!of transformations}

'<tr1> = <tr2>'\\
'<tr1> \<> <tr2>'

The equality operator '=' applied to two  transformations <tr1> and <tr2>
evaluates to 'true' if the  two transformations are  equal and to 'false'
otherwise.  The inequality  operator '\<>' applied to two transformations
<tr1> and <tr2> evaluates to  'true' if the  two transformations are  not
equal and to 'false' otherwise.  A transformation can also be compared to
any other object  that is not a  transformation, of course they are never
equal.
    
Two transformations are considered equal if and only if their image lists
are equal  as lists.  In particular,  equal transformations must have the
same degree.

|    gap> Transformation( [ 1, 2, 3, 4 ] ) = IdentityTransformation( 4 );
    true
    gap> Transformation( [ 1, 4, 4, 2 ] ) = 
    > Transformation( [ 1, 4, 4, 2, 5 ] );
    false|

\vspace{5mm}
'<tr1> \<\ <tr2>' \\
'<tr1> \<= <tr2>' \\
'<tr1>  >  <tr2>' \\
'<tr1>  >= <tr2>'

The  operators  '\<', '\<=',   '>', and '>='  evaluate   to 'true' if the
transformation <tr1> is less  than, less than  or equal to, greater than,
or  greater  than or equal  to  the transformation <tr2>, and  to 'false'
otherwise.

Let  <tr1>  and <tr2> be  two transformations  that  are not equal.  Then
<tr1> is considered smaller than <tr2> if and  only if the list of images
of <tr1> is (lexicographically) smaller than the list of images of <tr2>.
Note that this  way the  smallest  transformation of  degree  $n$ is  the
transformation that maps every point to $1$.

You can also compare transformations  with objects of  other types.  Here
any object that is  not a transformation  will be considered smaller than
any transformation.

%%%%%%%%%%%%%%%%%%%%%%%%%%%%%%%%%%%%%%%%%%%%%%%%%%%%%%%%%%%%%%%%%%%%%%%%%
\Section{Operations for Transformations}
 
'<tr1> \*\ <tr2>'%
\index{product!of transformations}

The operator '\*'  evaluates to  the product  of the two  transformations
<tr1> and <tr2>.

\vspace{5mm}
'<tr> \*\ <perm>'\\
'<perm> \*\ <tr>'%
\index{product!of permutation and transformation}

The operator '\*' evaluates to the product of the transformation <tr> and
the permutation <perm> in the given order if the degree of <perm> is less
than or equal to the degree of <tr>.

\vspace{5mm}
'<list> \*\ <tr>' \\
'<tr> \*\ <list>'%
\index{product!of list and transformation}

The operator '\*'  evaluates to the list of  products of the  elements in
<list> with the transformation <tr>.  That means that  the value is a new
list <new> such that '<new>[<i>] = <list>[<i>] \*\ <tr>' or '<new>[<i>] =
<tr> \*\ <list>[<i>]', respectively.

\vspace{5mm}
'<i> \^\ <tr>'%
\index{image!under transformation}

The operator '\^'  evaluates  to the image  $<i>\^<tr>$ of  the  positive
integer <i> under the transformation <tr> if <i> is  less than the degree
of <tr>.

\vspace{5mm}
'<tr> \^\ 0'

The operator '\^' evaluates to  the identity transformation on $n$ points
if <tr> is a transformation on $n$ points (see "IdentityTransformation").

\vspace{5mm}
'<tr> \^\ <i>'%
\index{power!of transformation}

For a  positive integer  <i>  the operator  '\^' evaluates to  the <i>-th
power of the transformation <tr>.

\vspace{5mm}
'<tr> \^\ -1'%
\index{inverse!of transformation}

The operator '\^' evaluates to the  inverse mapping of the transformation
<tr> which  is represented  as   a binary relation  (see chapter  "Binary
Relations").

%%%%%%%%%%%%%%%%%%%%%%%%%%%%%%%%%%%%%%%%%%%%%%%%%%%%%%%%%%%%%%%%%%%%%%%%%
\Section{IsTransformation}%
\index{test!for transformation}

'IsTransformation( <obj> )'

'IsTransformation' returns  'true' if  <obj>,  which may  be an object of
arbitrary type,  is  a  transformation  and 'false' otherwise.    It will
signal an error if <obj> is an unbound variable.

|    gap> IsTransformation( Transformation( [ 2, 1 ] ) );
    true
    gap> IsTransformation( 1 );
    false |

%%%%%%%%%%%%%%%%%%%%%%%%%%%%%%%%%%%%%%%%%%%%%%%%%%%%%%%%%%%%%%%%%%%%%%%%%
\Section{Degree of a Transformation}

'Degree( <trans> )'

'Degree' returns the degree of the transformation <trans>.

|    gap> Degree( Transformation( [ 3, 3, 4, 2, 5 ] ) );
    5|
    
The *degree* of a  transformation is the number of  points it  is defined
upon.  It can therefore be  read off as the length  of the list of images
of the transformation.

%%%%%%%%%%%%%%%%%%%%%%%%%%%%%%%%%%%%%%%%%%%%%%%%%%%%%%%%%%%%%%%%%%%%%%%%%
\Section{Rank of a Transformation}

'Rank( <trans> )'

'Rank' returns the rank of the transformation <trans>.

|    gap> Rank( Transformation( [ 3, 3, 4, 2, 5 ] ) );
    5|

The *rank* of a transformation is the number of  points in its image.  It
can  therefore be determined  as  the size of the   set of images  of the
transformation.

%%%%%%%%%%%%%%%%%%%%%%%%%%%%%%%%%%%%%%%%%%%%%%%%%%%%%%%%%%%%%%%%%%%%%%%%%
\Section{Image of a Transformation}

'Image( <trans> )'

'Image' returns the image of the transformation <trans>.

|    gap> Image( Transformation( [ 3, 3, 4, 2, 5 ] ) );
    [ 2, 3, 4, 5 ]|

The   *image* of  a  transformation is  the  set  of its  images.   For a
transformation of degree $n$  this is  always  a subset of the  set $\{1,
 ... , n\}$.

%%%%%%%%%%%%%%%%%%%%%%%%%%%%%%%%%%%%%%%%%%%%%%%%%%%%%%%%%%%%%%%%%%%%%%%%%
\Section{Kernel of a Transformation}

'Kernel( <trans> )'

'Kernel' returns the kernel of the transformation <trans>.

|    gap> Kernel( Transformation( [ 3, 3, 4, 2, 5 ] ) );
    [ [ 1, 2 ], [ 3 ], [ 4 ], [ 5 ] ] |

The *kernel*  of a transformation is the  set of its nonempty  preimages. 
For a transformation of degree $n$ this is  always a partition of the set
$\{1,  ... , n\}$.

%%%%%%%%%%%%%%%%%%%%%%%%%%%%%%%%%%%%%%%%%%%%%%%%%%%%%%%%%%%%%%%%%%%%%%%%%
\Section{PermLeftQuoTrans}
\index{quotient!of transformations}

'PermLeftQuoTrans( <tr1>, <tr2> )'

Given transformations <tr1> and  <tr2> with equal  kernel and  image, the
permutation induced by '<tr1>\^-1 \*\ <tr2>'  on the set 'Image( <tr1> )'
is computed.

|    gap> a:= Transformation( [ 8, 7, 5, 3, 1, 3, 8, 8 ] );;
    gap> Image(a);  Kernel(a);
    [ 1, 3, 5, 7, 8 ]
    [ [ 1, 7, 8 ], [ 2 ], [ 3 ], [ 4, 6 ], [ 5 ] 
    gap> b:= Transformation( [ 1, 3, 8, 7, 5, 7, 1, 1 ] );;
    gap> Image(b) = Image(a);  Kernel(b) = Kernel(a);
    true
    true
    gap> PermLeftQuoTrans(a, b);
    (1,5,8)(3,7) |

%%%%%%%%%%%%%%%%%%%%%%%%%%%%%%%%%%%%%%%%%%%%%%%%%%%%%%%%%%%%%%%%%%%%%%%%%
\Section{TransPerm}%
\index{convert!permutation to transformation}

'TransPerm( <n>, <perm> )'

'TransPerm' returns the bijective transformation of  degree <n> that acts
on the set $\{1,  ... ,  n\}$ in the same way  as the permutation  <perm>
does.

|    gap> TransPerm( 4, (1,2,3) );
    Transformation( [ 2, 3, 1, 4 ] )|

%%%%%%%%%%%%%%%%%%%%%%%%%%%%%%%%%%%%%%%%%%%%%%%%%%%%%%%%%%%%%%%%%%%%%%%%%
\Section{PermTrans}%
\index{convert!transformation to permutation}

'PermTrans( <trans> )'

'PermTrans' returns  the   permutation  defined by   the   transformation
<trans>.  If <trans> is not bijective, an error is signaled by 'PermList'
(see \"PermList\").

|    gap> PermTrans( Transformation( [ 2, 3, 1, 4 ] ) );
    (1,2,3)|

%%%%%%%%%%%%%%%%%%%%%%%%%%%%%%%%%%%%%%%%%%%%%%%%%%%%%%%%%%%%%%%%%%%%%%%%%%%%%
%%
%E  Emacs . . . . . . . . . . . . . . . . . . . . . . . local emacs variables
%%
%%  Local Variables:
%%  mode:               LaTeX
%%  outline-regexp:     "\\\\Chapter\\|\\\\Section"
%%  fill-column:        73
%%  TeX-master:         "manual"
%%  eval:               (outline-minor-mode)
%%  End:
%%

