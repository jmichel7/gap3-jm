%%%%%%%%%%%%%%%%%%%%%%%%%%%%%%%%%%%%%%%%%%%%%%%%%%%%%%%%%%%%%%%%%%%%%%%%%%%%%
%%
%A  matrix.tex                  GAP documentation            Martin Schoenert
%%
%A  @(#)$Id: matrix.tex,v 1.1.1.1 1996/12/11 12:36:48 werner Exp $
%%
%Y  Copyright 1990-1992,  Lehrstuhl D fuer Mathematik,  RWTH Aachen,  Germany
%%
%%  This file describes the matrix data type, its operations  and  functions.
%%
%H  $Log: matrix.tex,v $
%H  Revision 1.1.1.1  1996/12/11 12:36:48  werner
%H  Preparing 3.4.4 for release
%H
%H  Revision 3.7  1994/06/23  13:30:29  vfelsch
%H  modified an example
%H
%H  Revision 3.6  1994/06/23  08:50:40  ahulpke
%H  added PrintArray
%H
%H  Revision 3.5  1994/02/28  06:57:50  fceller
%H  fixed description of 'KronckerProduct'
%H
%H  Revision 3.4  1993/02/19  10:48:42  gap
%H  adjustments in line length and spelling
%H
%H  Revision 3.3  1993/02/11  17:14:10  martin
%H  vectors may now contain records
%H
%H  Revision 3.2  1993/02/05  09:05:59  felsch
%H  examples fixed
%H
%H  Revision 3.1  1992/03/20  16:34:52  martin
%H  initial revision under RCS
%%
\Chapter{Matrices}%
\index{matrix}\index{type!matrices}

Matrices are an  important tool in algebra.  A matrix nicely represents a
homomorphism  between two vector spaces with respect to a choice of bases
for  the  vector  spaces.   Also  matrices  represent systems  of  linear
equations.

In {\GAP}  matrices are represented by list of vectors  (see  "Vectors").
The vectors must all have the same length, and their elements must lie in
a  common  field.   The  field  may  be  the  field  of   rationals  (see
"Rationals"), a cyclotomic field (see "Cyclotomics"), a finite field (see
"Finite Fields"),  or a library and/or user  defined field (or ring) such
as a polynomial ring (see "Polynomials").

The first  section in this chapter describes the operations applicable to
matrices (see "Operations for Matrices").   The next  sections  describes
the function that tests whether an object is a matrix (see "IsMat").  The
next  sections describe the functions that create  certain  matrices (see
"IdentityMat", "NullMat", "TransposedMat",  and "KroneckerProduct").  The
next  sections  describe functions that  compute  certain  characteristic
values of matrices  (see  "DimensionsMat", "TraceMat",  "DeterminantMat",
"RankMat", and  "OrderMat").   The  next  sections describe the functions
that are related to the interpretation of a matrix  as a system of linear
equations   (see   "TriangulizeMat",   "BaseMat",   "NullspaceMat",   and
"SolutionMat").   The last  two  sections  describe  the  functions  that
diagonalize    an    integer    matrix    (see    "DiagonalizeMat"    and
"ElementaryDivisorsMat").

Because  matrices are  just a special  case of lists, all operations  and
functions  for  lists  are  applicable  to  matrices  also  (see  chapter
"Lists").   This especially includes  accessing elements of a matrix (see
"List Elements"), changing elements of  a matrix (see "List Assignment"),
and comparing matrices (see "Comparisons of Lists").

%%%%%%%%%%%%%%%%%%%%%%%%%%%%%%%%%%%%%%%%%%%%%%%%%%%%%%%%%%%%%%%%%%%%%%%%%
\Section{Operations for Matrices}

'<mat> + <scalar>' \\
'<scalar> + <mat>'

This  forms evaluates  to  the  sum of  the matrix  <mat> and the  scalar
<scalar>.  The elements of <mat> and <scalar> must lie in a common field.
The sum is a new matrix where each entry is the sum of the  corresponding
entry of <mat> and <scalar>.

'<mat1> + <mat2>'

This  form evaluates to the  sum  of the two matrices <mat1> and  <mat2>,
which  must have  the  same dimensions and whose  elements must lie in  a
common field.  The sum is a new matrix where each entry is the sum of the
corresponding entries of <mat1> and <mat2>.

'<mat> - <scalar>' \\
'<scalar> - <mat>' \\
'<mat1> - <mat2>'

The definition  for the '-' operator are similar to the above definitions
for the '+' operator, except that '-' subtracts of course.

'<mat> \*\ <scalar>' \\
'<scalar> \*\ <mat>'

This forms evaluate  to the product of the matrix  <mat> and  the  scalar
<scalar>.  The elements of <mat> and <scalar> must lie in a common field.
The product  is  a  new matrix  where each entry  is the  product of  the
corresponding entries of <mat> and <scalar>.

'<vec> \*\ <mat>'

This form evaluates to  the product  of  the vector <vec> and  the matrix
<mat>.   The length of <vec>  and  the  number  of rows of <mat> must  be
equal.  The  elements of <vec> and <mat> must lie  in a common field.  If
<vec> is a vector of length <n> and <mat> is  a matrix with <n>  rows and
<m> columns, the product is a new  vector of length <m>.   The element at
position  <i> is  the sum  of  '<vec>[<l>] \*\  <mat>[<l>][<i>]' with <l>
running from 1 to <n>.

'<mat> \*\ <vec>'

This form evaluates to the product of  the  matrix  <mat> and the  vector
<vec>.  The number of columns  of <mat> and  the length  of <vec> must be
equal.  The elements of <mat>  and <vec> must lie in a common  field.  If
<mat> is a  matrix with <m> rows and <n> columns and <vec> is a vector of
length <n>, the  product  is a new vector of length <m>.  The element  at
position  <i>  is the sum  of  '<mat>[<i>][<l>]  \*\ <vec>[<l>]' with <l>
running from 1 to <n>.

'<mat1> \*\ <mat2>'

This form evaluates to the product of the two matrices <mat1> and <mat2>.
The number of columns of <mat1> and the number of rows of  <mat2> must be
equal.  The elements of <mat1> and <mat2> must lie in a common field.  If
<mat1> is a matrix with <m>  rows and <n> columns and  <mat2> is a matrix
with  <n> rows and <o> columns, the result is a new  matrix with <m> rows
and  <o> columns.  The element in row <i> at position <k> of the  product
is the sum of '<mat1>[<i>][<l>] \*\  <mat2>[<l>][<k>]'  with  <l> running
from 1 to <n>.

'<mat1> / <mat2>' \\
'<scalar> / <mat>' \\
'<mat> / <scalar>' \\
'<vec> / <mat>'

In general  '<left>  / <right>'  is defined  as '<left> \*\ <right>\^-1'.
Thus in the above forms the right operand must always be invertable.

'<mat> \^\ <int>'

This form evaluates  to  the <int>-th  power of the matrix  <mat>.  <mat>
must be a square matrix, <int> must be an integer.  If <int> is negative,
<mat> must be invertible.   If  <int>  is 0,  the result is the  identity
matrix, even if <mat> is not invertible.

'<mat1> \^\ <mat2>'

This form evaluates to the conjugation of the matrix <mat1> by the matrix
<mat2>,  i.e., to '<mat2>\^-1  \*\  <mat1>  \*\  <mat2>'.  <mat2> must be
invertible and <mat1> must be such that these product can be computed.

'<vec> \^\ <mat>'

This  is  in  every  respect  equivalent  to  '<vec>  \*\  <mat>'.   This
operations reflects the fact that matrices operate on the vector space by
multiplication from the right.

'<scalar> + <matlist>' \\
'<matlist> + <scalar>' \\
'<scalar> - <matlist>' \\
'<matlist> - <scalar>' \\
'<scalar> \*\ <matlist>' \\
'<matlist> \*\ <scalar>' \\
'<matlist> / <scalar>'

A scalar <scalar>  may also  be added,  subtracted,  multiplied with,  or
divide into a whole list of matrices <matlist>.  The result is a new list
of matrices where each  matrix is the result  of performing the operation
with the corresponding matrix in <matlist>.

'<mat> \*\ <matlist>' \\
'<matlist> \*\ <mat>'

A  matrix  <mat>  may  also be multiplied  with a whole list  of matrices
<matlist>.  The  result  is a new list of matrices, where each matrix  is
the product of <mat> and the corresponding matrix in <matlist>.

'<matlist> / <mat>'

This form evaluates to  '<matlist> \*\ <mat>\^-1'.  <mat>  must of course
be invertable.

'<vec> \*\ <matlist>'

This form evaluates to  the product  of the vector <vec> and the  list of
matrices  <mat>.  The length  <l> of  <vec> and <matlist> must be  equal.
All matrices in <matlist> must have the same dimensions.  The elements of
<vec> and the elements of the matrices in <matlist> must lie in a  common
field.   The product  is the sum of '<vec>[<i>] \*\  <matlist>[<i>]' with
<i> running from 1 to <l>.

'Comm( <mat1>, <mat2> )'

'Comm' returns the  commutator of  the matrices <mat1> and  <mat2>, i.e.,
'<mat1>\^-1  \*\ <mat2>\^-1  \*\ <mat1>  \*\  <mat2>'.  <mat1> and <mat2>
must be invertable and such that these product can be computed.

There is  one exception to  the rule that  the operands or their elements
must  lie in common field.  It is allowed  that  one operand is  a finite
field element, a finite field vector, a finite field matrix, or a list of
finite  field  matrices, and the other operand  is an integer, an integer
vector, an integer matrix,  or  a list of integer matrices.  In this case
the integers are interpreted  as  '<int> \*\ <GF>.one', where <GF> is the
finite field (see "Operations for Finite Field Elements").

For  all  the above operations the result  is new, i.e., not identical to
any other list  (see  "Identical Lists").  This is the  case  even if the
result is equal  to  one of  the  operands, e.g., if  you add zero  to  a
matrix.

%%%%%%%%%%%%%%%%%%%%%%%%%%%%%%%%%%%%%%%%%%%%%%%%%%%%%%%%%%%%%%%%%%%%%%%%%
\Section{IsMat}

'IsMat( <obj> )'

'IsMat' return 'true' if <obj>, which can be an object of arbitrary type,
is  a matrix and 'false'  otherwise.  Will cause an error  if <obj> is an
unbound variable.

|    gap> IsMat( [ [ 1, 0 ], [ 0, 1 ] ] );
    true    # a matrix is a list of vectors
    gap> IsMat( [ [ 1, 2, 3, 4, 5 ] ] );
    true
    gap> IsMat( [ [ Z(2)^0, 0*Z(2) ], [ 0*Z(2), Z(2)^0 ] ] );
    true
    gap> IsMat( [ [ Z(2)^0, 0 ], [ 0, Z(2)^0 ] ] );
    false    # 'Z(2)\^0' and '0' do not lie in a common field
    gap> IsMat( [ 1, 0 ] );
    false    # a vector is not a matrix
    gap> IsMat( 1 );
    false    # neither is a scalar |

%%%%%%%%%%%%%%%%%%%%%%%%%%%%%%%%%%%%%%%%%%%%%%%%%%%%%%%%%%%%%%%%%%%%%%%%%
\Section{IdentityMat}

'IdentityMat( <n> )' \\
'IdentityMat( <n>, <F> )'

'IdentityMat' returns the identity matrix  with <n> rows  and <n> columns
over  the  field  <F>.  If  no field is given, 'IdentityMat'  returns the
identity matrix over the field of rationals.  Each call to 'IdentityMat'
returns a new matrix, so it is safe to modify the result.

|    gap> IdentityMat( 3 );
    [ [ 1, 0, 0 ], [ 0, 1, 0 ], [ 0, 0, 1 ] ]
    gap> PrintArray( last );
    [ [  1,  0,  0 ],
      [  0,  1,  0 ],
      [  0,  0,  1 ] ]
    gap> PrintArray( IdentityMat( 3, GF(2) ) );
    [ [  Z(2)^0,  0*Z(2),  0*Z(2) ],
      [  0*Z(2),  Z(2)^0,  0*Z(2) ],
      [  0*Z(2),  0*Z(2),  Z(2)^0 ] ] |

%%%%%%%%%%%%%%%%%%%%%%%%%%%%%%%%%%%%%%%%%%%%%%%%%%%%%%%%%%%%%%%%%%%%%%%%%
\Section{NullMat}

'NullMat( <m> )' \\
'NullMat( <m>, <n> )' \\
'NullMat( <m>, <n>, <F> )'

'NullMat'  returns the null matrix  with <m> rows and  <n> columns over the
field  <F>; if <n> is omitted,  it is assumed equal to  <m>. If no field is
given,  'NullMat' returns the null matrix over the field of rationals. Each
call to 'NullMat' returns a new matrix, so it is safe to modify the result.

|    gap> PrintArray( NullMat( 2, 3 ) );
    [ [  0,  0,  0 ],
      [  0,  0,  0 ] ]
    gap> PrintArray( NullMat( 2, 2, GF(2) ) );
    [ [  0*Z(2),  0*Z(2) ],
      [  0*Z(2),  0*Z(2) ] ] |

%%%%%%%%%%%%%%%%%%%%%%%%%%%%%%%%%%%%%%%%%%%%%%%%%%%%%%%%%%%%%%%%%%%%%%%%%
\Section{TransposedMat}

'TransposedMat( <mat> )'

'TransposedMat'  returns  the  transposed  of   the  matrix  <mat>.   The
transposed matrix is a new  matrix  <trn>, such that '<trn>[<i>][<k>]' is
'<mat>[<k>][<i>]'.

|    gap> TransposedMat( [ [ 1, 2 ], [ 3, 4 ] ] );
    [ [ 1, 3 ], [ 2, 4 ] ]
    gap> TransposedMat( [ [ 1..5 ] ] );
    [ [ 1 ], [ 2 ], [ 3 ], [ 4 ], [ 5 ] ] |

%%%%%%%%%%%%%%%%%%%%%%%%%%%%%%%%%%%%%%%%%%%%%%%%%%%%%%%%%%%%%%%%%%%%%%%%%
\Section{KroneckerProduct}

'KroneckerProduct( <mat1>, ..., <matn> )'

'KroneckerProduct'  returns the  Kronecker product  of the matrices <mat1>,
...,  <matn>. If <mat1> is a  <m> by <n> matrix and  <mat2> is a <o> by <p>
matrix,  the  Kronecker  product  of  <mat1>  by  <mat2> is a '<m>\*<o>' by
'<n>\*<p>'  matrix,  such  that  the  entry  in row '(<i1>-1)\*<o>+<i2>' at
position '(<k1>-1)\*<p>+<k2>' is '<mat1>[<i1>][<k1>] \*\
<mat2>[<i2>][<k2>]'.

|    gap> mat1 := [ [ 0, -1, 1 ], [ -2, 0, -2 ] ];;
    gap> mat2 := [ [ 1, 1 ], [ 0, 1 ] ];;
    gap> PrintArray( KroneckerProduct( mat1, mat2 ) );
    [ [   0,   0,  -1,  -1,   1,   1 ],
      [   0,   0,   0,  -1,   0,   1 ],
      [  -2,  -2,   0,   0,  -2,  -2 ],
      [   0,  -2,   0,   0,   0,  -2 ] ] |

%%%%%%%%%%%%%%%%%%%%%%%%%%%%%%%%%%%%%%%%%%%%%%%%%%%%%%%%%%%%%%%%%%%%%%%%%
\Section{DimensionsMat}

'DimensionsMat( <mat> )'

'DimensionsMat' returns the dimensions of the matrix <mat>  as  a list of
two integers.  The first entry is the number of rows of <mat>, the second
entry is the number of columns.

|    gap> DimensionsMat( [ [ 1, 2, 3 ], [ 4, 5, 6 ] ] );
    [ 2, 3 ]
    gap> DimensionsMat( [ [ 1 .. 5 ] ] );
    [ 1, 5 ] |

%%%%%%%%%%%%%%%%%%%%%%%%%%%%%%%%%%%%%%%%%%%%%%%%%%%%%%%%%%%%%%%%%%%%%%%%%
\Section{IsDiagonalMat}

'IsDiagonalMat( <mat> )'

<mat> must be a matrix. This function returns 'true' if <mat> is square and
all  entries  'mat[i][j]'  with  'i\<>j'  are  equal  to '0\*mat[i][j]' and
'false' otherwise.

|    gap> mat := [ [ 1, 2 ], [ 3, 1 ] ];;
    gap> IsDiagonalMat( mat );
    false |

%%%%%%%%%%%%%%%%%%%%%%%%%%%%%%%%%%%%%%%%%%%%%%%%%%%%%%%%%%%%%%%%%%%%%%%%%
\Section{IsLowerTriangularMat}

'IsLowerTriangularMat( <mat> )'

<mat>  must be  a matrix.  This  function  returns 'true' if all  entries
 'mat[i][j]' with 'j>i'   are   equal  to '0\*mat[i][j]'   and    'false'
 otherwise.

|    gap> a := [ [ 1, 2 ], [ 3, 1 ] ];;
    gap> IsLowerTriangularMat( a );
    false
    gap> a[1][2] := 0;;
    gap> IsLowerTriangularMat( a );
    true |

%%%%%%%%%%%%%%%%%%%%%%%%%%%%%%%%%%%%%%%%%%%%%%%%%%%%%%%%%%%%%%%%%%%%%%%%%
\Section{IsUpperTriangularMat}

'IsUpperTriangularMat( <mat> )'

<mat>  must  be  a  matrix.  This  function  returns  'true' if all entries
'mat[i][j]'  with  'j  \<  i'  are  equal  to  '0\*mat[i][j]'  and  'false'
otherwise.

|    gap> a := [ [ 1, 2 ], [ 3, 1 ] ];;
    gap> IsUpperTriangularMat( a );
    false
    gap> a[2][1] := 0;;
    gap> IsUpperTriangularMat( a );
    true |

%%%%%%%%%%%%%%%%%%%%%%%%%%%%%%%%%%%%%%%%%%%%%%%%%%%%%%%%%%%%%%%%%%%%%%%%%
\Section{DiagonalOfMat}

'DiagonalOfMat( <mat> )'

This function returns the list of diagonal entries of the matrix <mat>,
that is the list of '<mat>[i][i]'.

|    gap> mat := [ [ 1, 2 ], [ 3, 1 ] ];;
    gap> DiagonalOfMat( mat );
    [ 1, 1 ] |

%%%%%%%%%%%%%%%%%%%%%%%%%%%%%%%%%%%%%%%%%%%%%%%%%%%%%%%%%%%%%%%%%%%%%%%%%
\Section{DiagonalMat}%
\index{Matrices!direct sum of}

'DiagonalMat( <mat1>, ... , <matn> )'

returns the block diagonal direct sum of the matrices <mat1>, $\ldots$,
<matn>. Blocks of size $1\times 1$ may be given as scalars.

|    gap> C1 := [ [   2,  -1,   0,   0 ],
    >            [  -1,   2,  -1,   0 ],
    >            [   0,  -1,   2,  -1 ],
    >            [   0,   0,  -1,   2 ] ];;
    gap> C2 := [ [   2,   0,  -1 ],
    >            [   0,   2,  -1 ],
    >            [  -1,  -1,   2 ] ];;
    gap> PrintArray( DiagonalMat( C1, 3, C2 ) );
    [ [   2,  -1,   0,   0,   0,   0,   0,   0 ],
      [  -1,   2,  -1,   0,   0,   0,   0,   0 ],
      [   0,  -1,   2,  -1,   0,   0,   0,   0 ],
      [   0,   0,  -1,   2,   0,   0,   0,   0 ],
      [   0,   0,   0,   0,   3,   0,   0,   0 ],
      [   0,   0,   0,   0,   0,   2,   0,  -1 ],
      [   0,   0,   0,   0,   0,   0,   2,  -1 ],
      [   0,   0,   0,   0,   0,  -1,  -1,   2 ] ]|

To make a diagonal matrix with a specified diagonal <d> use
'ApplyFunc(DiagonalMat, d )'.

‣ PermutationMat( perm, dim[, F] )	( function )
returns a matrix in dimension dim over the field given by F (i.e. the smallest field containing the element F or F itself if it is a field) that represents the permutation perm acting by permuting the basis vectors as it permutes points.

%%%%%%%%%%%%%%%%%%%%%%%%%%%%%%%%%%%%%%%%%%%%%%%%%%%%%%%%%%%%%%%%%%%%%%%%%
\Section{PermutationMat}%
\index{Matrices!permutation}

'PermutationMat( <perm>, <dim> [,<F>])'

returns  a matrix  in dimension  <dim> over  the field  <F> (by default the
rationals)  that represents the permutation  <perm> acting by permuting the
basis vectors as it permutes points.

|    gap> PermutationMat((1,2,3),4);
    [ [ 0, 1, 0, 0 ], [ 0, 0, 1, 0 ], [ 1, 0, 0, 0 ], [ 0, 0, 0, 1 ] ]|

%%%%%%%%%%%%%%%%%%%%%%%%%%%%%%%%%%%%%%%%%%%%%%%%%%%%%%%%%%%%%%%%%%%%%%%%%
\Section{TraceMat}

'TraceMat( <mat> )'

'TraceMat' returns  the trace  of the square matrix <mat>.   The trace is
the sum of all entries on the diagonal of <mat>.

|    gap> TraceMat( [ [ 1, 2, 3 ], [ 4, 5, 6 ], [ 7, 8, 9 ] ] );
    15
    gap> TraceMat( IdentityMat( 4, GF(2) ) );
    0*Z(2) |

%%%%%%%%%%%%%%%%%%%%%%%%%%%%%%%%%%%%%%%%%%%%%%%%%%%%%%%%%%%%%%%%%%%%%%%%%
\Section{DeterminantMat}

'DeterminantMat( <mat> )'

'DeterminantMat' returns the determinant of the square matrix <mat>.  The
determinant is defined by\\
$\sum_{p \in Symm(n)}{sign(p)\prod_{i=1}^{n}{mat[i][i^p]}}$.

|    gap> DeterminantMat( [ [ 1, 2 ], [ 3, 4 ] ] );
    -2
    gap> DeterminantMat( [ [ 0*Z(3), Z(3)^0 ], [ Z(3)^0, Z(3) ] ] );
    Z(3) |

Note that 'DeterminantMat' does not use  the above  definition to compute
the  result.   Instead  it  performs a Gaussian  elimination.   For large
rational matrices this may take very long, because the entries may become
very large, even if the final result is a small integer.

%%%%%%%%%%%%%%%%%%%%%%%%%%%%%%%%%%%%%%%%%%%%%%%%%%%%%%%%%%%%%%%%%%%%%%%%%
\Section{RankMat}

'RankMat( <mat> )'

'RankMat'  returns the  rank of the matrix <mat>.  The rank is defined as
the  dimension  of the vector  space spanned by the rows  of  <mat>.   It
follows  that  a <n> by  <n>  matrix is invertible exactly if its rank is
<n>.

|    gap> RankMat( [ [ 4, 1, 2 ], [ 3, -1, 4 ], [ -1, -2, 2 ] ] );
    2 |

Note that 'RankMat' performs a  Gaussian elimination.  For large rational
matrices  this  may take  very long, because the  entries may become very
large.

%%%%%%%%%%%%%%%%%%%%%%%%%%%%%%%%%%%%%%%%%%%%%%%%%%%%%%%%%%%%%%%%%%%%%%%%%
\Section{OrderMat}

'OrderMat( <mat> )'

'OrderMat' returns  the order of the invertible square matrix <mat>.  The
order <ord> is the smallest positive integer such  that '<mat>\^<ord>' is
the identity.

|    gap> OrderMat( [ [ 0*Z(2), 0*Z(2), Z(2)^0 ],
    >                 [ Z(2)^0, Z(2)^0, 0*Z(2) ],
    >                 [ Z(2)^0, 0*Z(2), 0*Z(2) ] ] );
    4 |

'OrderMat'  first  computes <ord1>  such that  the first  standard  basis
vector  is  mapped  by  '<mat>\^<ord1>' onto  itself.   It  does this  by
applying <mat> repeatedly to the  first standard basis  vector.   Then it
computes <mat1> as '<mat1>\^<ord1>'.   Then it  computes <ord2> such that
the second  standard  basis vector is  mapped  by  '<mat1>\^<ord2>'  onto
itself.  This process is repeated until all basis vectors are mapped onto
themselves.  'OrderMat' warns you that the order may be infinite, when it
finds that the order must be larger than 1000.

%%%%%%%%%%%%%%%%%%%%%%%%%%%%%%%%%%%%%%%%%%%%%%%%%%%%%%%%%%%%%%%%%%%%%%%%%
\Section{TriangulizeMat}

'TriangulizeMat( <mat> )'

'TriangulizeMat'  brings the matrix <mat>  into upper triangular form. Note
that  <mat> is changed. A matrix is in upper triangular form when the first
nonzero  entry in each  row is one  and lies further  to the right than the
first  nonzero  entry  in  the  previous  row. Furthermore, above the first
nonzero  entry in each row all entries  are zero. Note that the matrix will
have  trailing zero rows if  the rank of <mat>  is not maximal. The rows of
the  resulting  matrix  span  the  same  vectorspace  than  the rows of the
original matrix <mat>. The function returns the indices of the lines of the
orginal  matrix  corresponding  to  the  non-zero lines of the triangulized
matrix.

|    gap> m := [ [ 0, -3, -1 ], [ -3, 0, -1 ], [ 2, -2, 0 ] ];;
    gap> TriangulizeMat( m ); m;
    [ 2, 1 ]
    [ [ 1, 0, 1/3 ], [ 0, 1, 1/3 ], [ 0, 0, 0 ] ] |

Note  that for large rational matrices 'TriangulizeMat' may take very long,
because  the entries may become very large during the Gaussian elimination,
even if the final result contains only small integers.

%%%%%%%%%%%%%%%%%%%%%%%%%%%%%%%%%%%%%%%%%%%%%%%%%%%%%%%%%%%%%%%%%%%%%%%%%
\Section{BaseMat}

'BaseMat( <mat> )'

'BaseMat'  returns a  standard  base for the vector space  spanned by the
rows of the matrix <mat>.  The standard base is in upper triangular form.
That means that  the  first nonzero vector in each  row  is  one and lies
further to the right than  the first nonzero  entry in the previous  row.
Furthermore, above  the  first  nonzero entry in each row all entries are
zero.

|    gap> BaseMat( [ [ 0, -3, -1 ], [ -3, 0, -1 ], [ 2, -2, 0 ] ] );
    [ [ 1, 0, 1/3 ], [ 0, 1, 1/3 ] ] |

Note that  for  large  rational  matrices 'BaseMat'  may take very  long,
because  the  entries   may  become   very  large   during  the  Gaussian
elimination, even if the final result contains only small integers.

%%%%%%%%%%%%%%%%%%%%%%%%%%%%%%%%%%%%%%%%%%%%%%%%%%%%%%%%%%%%%%%%%%%%%%%%%
\Section{NullspaceMat}

'NullspaceMat( <mat> )'

'NullspaceMat' returns a base for the nullspace of the matrix <mat>.  The
nullspace is the set of vectors <vec> such that '<vec> \*\ <mat>'  is the
zero vector.   The returned  base is the standard  base for the nullspace
(see "BaseMat").

|    gap> NullspaceMat( [ [ 2, -4, 1 ], [ 0, 0, -4 ], [ 1, -2, -1 ] ] );
    [ [ 1, 3/4, -2 ] ] |

Note that for large rational matrices 'NullspaceMat' may take very  long,
because   the   entries   may  become  very  large  during  the  Gaussian
elimination, even if the final result only contains small integers.

%%%%%%%%%%%%%%%%%%%%%%%%%%%%%%%%%%%%%%%%%%%%%%%%%%%%%%%%%%%%%%%%%%%%%%%%%
\Section{SolutionMat}

'SolutionMat( <mat>, <vec> )'

'SolutionMat'  returns  one solution  of the equation '<x>  \*\  <mat>  =
<vec>' or 'false' if no such solution exists.

|    gap> SolutionMat( [ [ 2, -4, 1 ], [ 0, 0, -4 ], [ 1, -2, -1 ] ],
    >                  [ 10, -20, -10 ] );
    [ 5, 15/4, 0 ]
    gap> SolutionMat( [ [ 2, -4, 1 ], [ 0, 0, -4 ], [ 1, -2, -1 ] ],
    >                  [ 10, 20, -10 ] );
    false |

Note  that for large rational  matrices 'SolutionMat' may take very long,
because   the  entries  may   become  very  large   during  the  Gaussian
elimination, even if the final result only contains small integers.

%%%%%%%%%%%%%%%%%%%%%%%%%%%%%%%%%%%%%%%%%%%%%%%%%%%%%%%%%%%%%%%%%%%%%%%%%
\Section{DiagonalizeMat}

'DiagonalizeMat( <mat> )'

'DiagonalizeMat'  transforms the  integer matrix <mat> by  multiplication
with unimodular  (i.e., determinant +1  or -1) integer matrices  from the
left  and from the right into diagonal form (i.e., only diagonal  entries
are  nonzero).  Note  that  'DiagonalizeMat'  changes <mat>  and  returns
nothing.  If  there  are  several  diagonal matrices  to  which  <mat> is
equivalent,  it is not  specified  which one is computed, except that all
zero entries  on  the diagonal are collected at the lower right  end (see
"ElementaryDivisorsMat").

|    gap> m := [ [ 0, -1, 1 ], [ -2, 0, -2 ], [ 2, -2, 4 ] ];;
    gap> DiagonalizeMat( m );  m;
    [ [ 1, 0, 0 ], [ 0, 2, 0 ], [ 0, 0, 0 ] ] |

Note that for large integer matrices 'DiagonalizeMat' may take very long,
because the entries may become very large during the computation, even if
the final result only contains small integers.

%%%%%%%%%%%%%%%%%%%%%%%%%%%%%%%%%%%%%%%%%%%%%%%%%%%%%%%%%%%%%%%%%%%%%%%%%
\Section{ElementaryDivisorsMat}

'ElementaryDivisorsMat( <mat> )'

'ElementaryDivisors' returns a list of the elementary divisors, i.e., the
unique <d> with '<d>[<i>]'  divides '<d>[<i>+1]'  and <mat> is equivalent
to a diagonal matrix  with the elements  '<d>[<i>]' on the  diagonal (see
"DiagonalizeMat").

|    gap> m := [ [ 0, -1, 1 ], [ -2, 0, -2 ], [ 2, -2, 4 ] ];;
    gap> ElementaryDivisorsMat( m );
    [ 1, 2, 0 ] |

%%%%%%%%%%%%%%%%%%%%%%%%%%%%%%%%%%%%%%%%%%%%%%%%%%%%%%%%%%%%%%%%%%%%%%%%%
\Section{PrintArray}

'PrintArray( <mat > )'

'PrintArray' displays the matrix <mat> in a pretty way.

|    gap> m := [[1,2,3,4],[5,6,7,8],[9,10,11,12]];
    [ [ 1, 2, 3, 4 ], [ 5, 6, 7, 8 ], [ 9, 10, 11, 12 ] ]
    gap> PrintArray( m );
    [ [   1,   2,   3,   4 ],
      [   5,   6,   7,   8 ],
      [   9,  10,  11,  12 ] ]|

