%%%%%%%%%%%%%%%%%%%%%%%%%%%%%%%%%%%%%%%%%%%%%%%%%%%%%%%%%%%%%%%%%%%%%%%%%%%%%
%%
%A  agwords.tex                 GAP documentation                Frank Celler
%%
%A  @(#)$Id: agwords.tex,v 1.2 1997/01/20 14:04:08 gap Exp $
%%
%Y  Copyright 1990-1992,  Lehrstuhl D fuer Mathematik,  RWTH Aachen,  Germany
%%
%%  This file contains descriptions of the  ag word  datatype,  the operations
%%  and functions for this type.
%%
%H  $Log: agwords.tex,v $
%H  Revision 1.2  1997/01/20 14:04:08  gap
%H  vfelsch removed the hyphen in terms like AG-system or PAG-system
%H
%H  Revision 1.1.1.1  1996/12/11 12:36:42  werner
%H  Preparing 3.4.4 for release
%H
%H  Revision 3.12  1993/03/11  17:53:24  fceller
%H  strings are now lists
%H
%H  Revision 3.11  1993/02/19  10:48:42  gap
%H  adjustments in line length and spelling
%H
%H  Revision 3.10  1993/02/15  15:00:11  fceller
%H  removed "%T" lines
%H
%H  Revision 3.9  1993/02/15  14:23:56  felsch
%H  DefineName eliminated
%H
%H  Revision 3.8  1993/02/02  15:11:45  felsch
%H  examples fixed
%H
%H  Revision 3.7  1993/01/04  10:59:55  fceller
%H  changed 'DepthAgWord' applied to the identity
%H
%H  Revision 3.6  1992/04/07  08:25:20  fceller
%H  fixed a few typos
%H
%H  Revision 3.5  1992/04/03  13:15:38  fceller
%H  chnaged 'Shifted...' into 'Sifted...'
%H
%H  Revision 3.4  1992/03/30  07:51:02  fceller
%H  changed 'Exponents' slightly
%H
%H  Revision 3.3  1992/02/07  18:27:21  fceller
%H  Initial GAP 3.1 release.
%H
%H  Revision 3.1  1991/04/11  11:34:01  martin
%H  Initial revision under RCS
%%
\Chapter{Words in Finite Polycyclic Groups}%
\index{Ag Words}

Ag words  are  the  {\GAP}  datatype  for  elements of finite  polycyclic
groups.  Unlike permutations, which are all considered to be  elements of
one  large symmetric group,  each  ag word belongs to  a specified group.
Only ag words of the same finite polycyclic group can be multiplied.

The following  sections describe  ag words  and their parent  groups (see
"More  about Ag  Words"),   how  ag  words  are  compared (see  "Ag  Word
Comparisons"), functions for ag words and some low level functions for ag
words (starting at "CentralWeight" and "CanonicalAgWord").

For operations  and functions defined  for group elements  in general see
"Comparisons of Group Elements", "Operations for Group Elements".

%%%%%%%%%%%%%%%%%%%%%%%%%%%%%%%%%%%%%%%%%%%%%%%%%%%%%%%%%%%%%%%%%%%%%%%%%%%%%

\Section{More about Ag Words}

Let $G$ be a group and $G  = G_0 >  G_1 > ...  >  G_n = 1$ be a subnormal
series of $G \neq 1$ with finite cyclic factors, i.e., $G_i \lhd G_{i-1}$
for all $i=1, ..., n$ and $G_{i-1} = \langle G_i, g_i \rangle$.  Then $G$
will  be  called  an  *ag group*  \index{ag group}  with  *AG  generating
sequence*  \index{AG  generating  sequence}  or,  for short,  *AG system*
\index{AG system} $(g_1,  ..., g_n)$. Let $o_i$ be the  order of $G_{i-1}
/ G_i$. If all $o_1, ..., o_n$ are primes the system $(g_1, ..., g_n)$ is
called a  *PAG system*  \index{PAG system}.  With respect to  a given  AG
system the group $G$ has a so called *power-commutator presentation*

\begin{center}
  \begin{tabular}{lcll}
    ${g_i}^{o_i}$ & $=$ & $w_{ii}(g_{i+1},..., g_n)$ &
      for $1\leq i\leq n$,\\
    $[g_i,g_j]$ & $=$ & $w_{ij}(g_{j+1},...,g_n)$ &
      for $1\leq j\< i\leq n$\\
  \end{tabular}
\end{center}

and a so called *power-conjugate presentation*

\begin{center}
  \begin{tabular}{lcll}
    ${g_i}^{o_i}$ & $=$ & $w_{ii}(g_{i+1},..., g_n)$ & 
      for $1\leq i\leq n$,\\
    $g_i^{g_j}$ & $=$ & $w^{\prime}_{ij}(g_{j+1},...,g_n)$ &
      for $1\leq j\< i\leq n$.\\
  \end{tabular}
\end{center}

For both kinds of presentations we shall use  the term *AG presentation*.
Each element $g$ of $G$ can be expressed uniquely in the form

\begin{center}
  \begin{tabular}{cc}
    $g = g_1^{\nu_1}\* ...\* g_n^{\nu_n}$ & for $0 \leq \nu_i \< o_i$.
  \end{tabular}
\end{center}

We call the composition series $G_0 > G_1 > ... > G_n$ the *AG series* of
$G$ and define $\nu_i( g ) \:= \nu_i$.  If  $\nu_i = 0$ for  $i = 1, ...,
k-1$ and $\nu_k  \neq 0$, we call $\nu_k$  the *leading exponent* and $k$
the *depth* of $g$ and denote them by $\nu_k =\: \lambda( g )$ and $k =\:
\delta( g )$.  We call $o_k$ the *relative order* of $g$.

Each element $g$ of $G$ is called *ag  word* and we  say  that $G$ is the
parent group of $g$.  A  parent group   is constructed in   {\GAP}  using
'AgGroup' (see "AgGroup") or 'AgGroupFpGroup' (see "AgGroupFpGroup").

Our standard example in the following sections is  the symmetric group of
degree 4, defined by  the following sequence of {\GAP}  statements.   You
should   enter  them  before running  any   example.    For  details   on
'AbstractGenerators' see "AbstractGenerator".

|    gap> a  := AbstractGenerator( "a" );;  # (1,2)
    gap> b  := AbstractGenerator( "b" );;  # (1,2,3)
    gap> c  := AbstractGenerator( "c" );;  # (1,3)(2,4)
    gap> d  := AbstractGenerator( "d" );;  # (1,2)(3,4)
    gap> s4 := AgGroupFpGroup( rec(
    >        generators := [ a, b, c, d ],
    >        relators   := [ a^2, b^3, c^2, d^2, Comm( b, a ) / b,
    >                        Comm( c, a ) / d, Comm( d, a ),
    >                        Comm( c, b ) / ( c*d ), Comm( d, b ) / c,
    >                        Comm( d, c ) ] ) );
    Group( a, b, c, d )
    gap> s4.name := "s4";;
    gap> a := s4.generators[1];; b := s4.generators[2];;
    gap> c := s4.generators[3];; d := s4.generators[4];; |

%%%%%%%%%%%%%%%%%%%%%%%%%%%%%%%%%%%%%%%%%%%%%%%%%%%%%%%%%%%%%%%%%%%%%%%%%%%%%
\Section{Ag Word Comparisons}
\index{equality!of ag words}%
\index{ordering!of ag words}

'<g> \<\ <h>' \\
'<g> \<= <h>' \\
'<g> >= <h>' \\
'<g> > <h>'

The operators '\<', '>', '\<=' and '>=' return  'true' if <g> is strictly
less,  strictly greater, not  greater, not less, respectively,  than <h>.
Otherwise they return 'false'.

If <g> and <h> have a common parent group they  are compared with respect
to the AG series of this group.  If  two ag words have  different depths,
the  one with the  higher depth is  less than the   other one.  If two ag
words have the same  depth but different leading  exponents, the one with
the smaller leading exponent is less  than  the other one.  Otherwise the
leading generator is removed in both ag words and  the remaining ag words
are compared.

If <g>  and <h> do  not have a common parent  group, then the composition
lengths of the parent groups are compared.

You  can  compare  ag words with objects of other types.  Field elements,
unkowns, permutations and abstract  words  are  smaller  than  ag  words.
Objects of other types, i.e., functions, lists and records are larger.

|    gap> 123/47 < a;
    true
    gap> (1,2,3,4) < a;
    true
    gap> [1,2,3,4] < a;
    false
    gap> true < a;
    false
    gap> rec() < a;
    false
    gap> c < a;
    true
    gap> a*b < a*b^2;
    true |

%%%%%%%%%%%%%%%%%%%%%%%%%%%%%%%%%%%%%%%%%%%%%%%%%%%%%%%%%%%%%%%%%%%%%%%%%%%%%
\Section{CentralWeight}

'CentralWeight( <g> )'

'CentralWeight'  returns  the central  weight  of an  ag  word  <g>, with
respect to the central  series used  in  the combinatorial  collector, as
integer.

This presumes  that  <g>  belongs  to   a parent  group   for   which the
combinatorial collector is used. See "ChangeCollector" for details.

If <g> is the identity, 0 is returned.

Note that   'CentralWeight'   allows  records  that mimic   ag  words  as
arguments.

|    gap> d8 := AgGroup( Subgroup( s4, [ a, c, d ] ) );
    Group( g1, g2, g3 )
    gap> ChangeCollector( d8, "combinatorial" );
    gap> List( d8.generators, CentralWeight );
    [ 1, 1, 2 ] |

%%%%%%%%%%%%%%%%%%%%%%%%%%%%%%%%%%%%%%%%%%%%%%%%%%%%%%%%%%%%%%%%%%%%%%%%%%%%%
\Section{CompositionLength}

'CompositionLength( <g> )'

Let $G$ be the parent group of the ag word <g>.  Then 'CompositionLength'
returns the length of the AG series of $G$ as integer.

Note  that 'CompositionLength' allows records  that mimic   ag  words  as
arguments.

|    gap> CompositionLength( c );
    5 |

%%%%%%%%%%%%%%%%%%%%%%%%%%%%%%%%%%%%%%%%%%%%%%%%%%%%%%%%%%%%%%%%%%%%%%%%%%%%%
\Section{Depth}

'Depth( <g> )'

'Depth' returns the depth of an ag word <g> with respect to the AG series
of its parent group as integer.

Let $G$ be the parent  group of <g> and  $G=G_0  > ...  > G_n=\{1\}$  the
AG series of $G$.  Let $\delta$ be the maximal positive integer such that
<g> is an element of $G_{\delta-1}$. Then $\delta$ is the *depth* of <g>.

Note that 'Depth' allows record that mimic  ag  words as arguments.

|    gap> Depth( a );
    1
    gap> Depth( d );
    4
    gap> Depth( a^0 );
    5 |

%%%%%%%%%%%%%%%%%%%%%%%%%%%%%%%%%%%%%%%%%%%%%%%%%%%%%%%%%%%%%%%%%%%%%%%%%%%%%
\Section{IsAgWord}

'IsAgWord( <obj> )'

'IsAgWord' returns 'true' if <obj>, which can be  an arbitrary object, is
an ag word and 'false' otherwise.

|    gap> IsAgWord( 5 );
    false
    gap> IsAgWord( a );
    true |

%%%%%%%%%%%%%%%%%%%%%%%%%%%%%%%%%%%%%%%%%%%%%%%%%%%%%%%%%%%%%%%%%%%%%%%%%%%%%
\Section{LeadingExponent}

'LeadingExponent( <g> )'

'LeadingExponent'  returns  the leading  exponent  of  an ag word  <g> as
integer.

Let $G$ be the parent group of <g> and $(g_1, ..., g_n)$ the AG system of
$G$ and let $o_i$ be the relative order  of $g_i$.  Then the  element <g>
can be  expressed uniquely in the form $g_1^{\nu_1}\*  ...\* g_n^{\nu_n}$
for  integers $\nu_i$ such  that $0  \leq \nu_i  \<  o_i$.   The *leading
exponent* of <g> is the first nonzero $\nu_i$.

If <g> is the identity 0 is returned.

Although  'ExponentAgWord(   <g>, Depth( <g>   ) )'  returns  the leading
exponent of <g>, too, this function is faster and is able  to  handle the
identity.

Note  that  'LeadingExponent' allows  records   that  mimic ag   words as
arguments.

|    gap> LeadingExponent( a * b^2 * c^2 * d );
    1
    gap> LeadingExponent( b^2 * c^2 * d );
    2 |

%%%%%%%%%%%%%%%%%%%%%%%%%%%%%%%%%%%%%%%%%%%%%%%%%%%%%%%%%%%%%%%%%%%%%%%%%%%%%
\Section{RelativeOrder}

'RelativeOrder( <g> )'

'RelativeOrder' returns the relative order of an ag word <g> as integer.

Let $G$ be  the parent group of <g>  and $G=G_0  > ... >   G_n=\{1\}$ the
AG series of $G$.  Let $\delta$ be the maximal positive integer such that
<g> is an element of $G_{\delta-1}$.  The *relative  order* of <g> is the
index of  $G_{\delta+1}$ in  $G_\delta$,   that   is  the  order  of  the
factor group $G_\delta/G_{\delta+1}$.

If <g> is the identity 1 is returned.

Note that 'RelativeOrder' allows records that mimic agwords as arguments.

|    gap> RelativeOrder( a );
    2
    gap> RelativeOrder( b );
    3
    gap> RelativeOrder( b^2 * c * d );
    3 |

%%%%%%%%%%%%%%%%%%%%%%%%%%%%%%%%%%%%%%%%%%%%%%%%%%%%%%%%%%%%%%%%%%%%%%%%%%%%%
\Section{CanonicalAgWord}

'CanonicalAgWord( <U>, <g> )'

Let <U> be an ag  group with  parent group $G$, let <g> be an element  of
$G$. Let  $(u_1, ..., u_m)$ be  an  induced generating  system of <U> and
$(g_1,  ...,  g_n)$  be  a  canonical  generating  system  of  $G$.  Then
'CanonicalAgWord' returns a word  $x = <g> \* u = g_{i_1}^{e_1} \* ... \*
g_{i_k}^{e_k}$ such that $u\in <U>$ and no $i_j$ is equal to the depth of
any generator $u_l$.

|    gap> v4 := MergedCgs( s4, [ a*b^2, c*d ] );
    Subgroup( s4, [ a*b^2, c*d ] )
    gap> CanonicalAgWord( v4, a*c );
    b^2*d
    gap> CanonicalAgWord( v4, a*b*c*d );
    b
    gap> (a*b*c*d) * (a*b^2);
    b*c*d
    gap> last * (c*d);
    b |

%%%%%%%%%%%%%%%%%%%%%%%%%%%%%%%%%%%%%%%%%%%%%%%%%%%%%%%%%%%%%%%%%%%%%%%%%%%%%
\Section{DifferenceAgWord}

'DifferenceAgWord( <u>, <v> )'

'DifferenceAgWord' returns an ag word  $s$ representing the difference of
the exponent vectors of <u> and <v>.

Let $G$ be the parent group of <u> and <v>.  Let $(g_1, ..., g_n)$ be the
AG system of $G$  and $o_i$ be the relative order or $g_i$.  Then <u> can
be expressed uniquely as $g_1^{u_1}\* ...\* g_n^{u_n}$ for integers $u_i$
between $0$ and $o_i-1$ and <v> can be expressed uniquely as $g_1^{v_1}\*
...\*  g_n^{v_n}$  for integers  $v_i$  between  $0$  and  $o_i-1$.   The
function  'DifferenceAgWord' returns an  ag word $s  = g_1^{s_1}\*  ...\*
g_n^{s_n}$ with integer  $s_i$  such that $0 \leq  s_i \<  o_i$ and  $s_i
\equiv u_i - v_i$ mod $o_i$.

|    gap> DifferenceAgWord( a * b, a );
    b
    gap> DifferenceAgWord( a, b );
    a*b^2 
    gap> z27 := CyclicGroup( AgWords, 27 );
    Group( c27_1, c27_2, c27_3 )
    gap> x := z27.1 * z27.2;
    c27_1*c27_2
    gap> x * x;
    c27_1^2*c27_2^2
    gap> DifferenceAgWord( x, x );
    IdAgWord |

%%%%%%%%%%%%%%%%%%%%%%%%%%%%%%%%%%%%%%%%%%%%%%%%%%%%%%%%%%%%%%%%%%%%%%%%%%%%%
\Section{ReducedAgWord}

'ReducedAgWord( <b>, <x> )'

Let  <b> and <x>  be ag  words of the   same depth, then  'ReducedAgWord'
returns an ag word <a> such that <a> is an element of  the coset $U <b>$,
where $U$ is  the  cyclic group generated  by  <x>, and <a> has  a higher
depth than <b> and <x>.

Note that the relative order of <b> and <x> must be a prime.

Let $p$ be the relative order  of <b> and  <x>.  Let $\beta$ and $\xi$ be
the leading exponent of $b$ and  $x$ respectively.   Then  there exits an
integer $i$ such that $\xi \* i = \beta$ modulo  $p$.  We  can set $<a> =
<x>^{-i} <b>$.

Typically this function is used when  <b>  and  <x> occur in a generating
set of a subgroup $W$.  Then b can be replaced by  <a>  in the generating
set of <W>, but <a> and <x> have different depth.

|    gap> ReducedAgWord( a*b^2*c, a );
    b^2*c
    gap> ReducedAgWord( last, b );
    c |

%%%%%%%%%%%%%%%%%%%%%%%%%%%%%%%%%%%%%%%%%%%%%%%%%%%%%%%%%%%%%%%%%%%%%%%%%%%%%
\Section{SiftedAgWord}

'SiftedAgWord( <U>, <g> )'

'SiftedAgWord' tries to sift  an ag word <g>, which must be an element of
the parent group of an ag group <U>, through an induced generating system
of <U>. 'SiftedAgWord' returns the remainder of this shifting process.

The identity is returned if and only if <g> is an element of <U>.

Let  $u_1, ..., u_m$  be an induced  generating system of  <U>.  If there
exists an $u_i$ such that $u_i$ and <g> have the  same depth, then <g> is
reduced  with $u_i$ using   'ReducedAgWord' (see  "ReducedAgWord").   The
process is repeated until no $u_i$ can be found or the  <g> is reduced to
the identity.

'SiftedAgWord' allows factor group arguments.  See "Factor  Groups of Ag
Groups" for details.

Note that 'SiftedAgGroup' adds a record component '<U>.shiftInfo' to the
ag group record of <U>.  This entry is used by  subsequent calls with the
same ag group in order to speed up  computation.  If you  ever change the
component '<U>.igs' by  hand,  not  using  'Normalize', you must   unbind
'<U>.shiftInfo', otherwise all following  results of 'SiftedAgWord' will
be corrupted.

|    gap> s3 := Subgroup( s4, [ a, b ] );
    Subgroup( s4, [ a, b ] )
    gap> SiftedAgWord( s3, a * b^2 * c );
    c |

%%%%%%%%%%%%%%%%%%%%%%%%%%%%%%%%%%%%%%%%%%%%%%%%%%%%%%%%%%%%%%%%%%%%%%%%%%%%%
\Section{SumAgWord}

'SumAgWord( <u>, <v> )'

'SumAgWord' returns an ag  word $s$ representing the  sum of the exponent
vectors of <u> and <v>.

Let $G$ be the parent group of <u> and <v>.  Let $(g_1, ..., g_n)$ be the
AG system of $G$ and $o_i$ be the relative order or $g_i$.  Then  <u> can
be expressed uniquely as $g_1^{u_1}\* ...\* g_n^{u_n}$ for integers $u_i$
between $0$ and $o_i-1$ and <v> can be expressed uniquely as $g_1^{v_1}\*
...\*  g_n^{v_n}$ for integers  $v_i$  between  $0$  and  $o_i-1$.   Then
'SumAgWord'  returns an ag word  $s =  g_1^{s_1}\*  ...\* g_n^{s_n}$ with
integer $s_i$  such that $0 \leq s_i \< o_i$ and  $s_i \equiv u_i  + v_i$
mod $o_i$.

|    gap> SumAgWord( b, a );
    a*b
    gap> SumAgWord( a*b, a );
    b
    gap> RelativeOrderAgWord( a );
    2 
    gap> z27 := CyclicGroup( AgWords, 27 );
    Group( c27_1, c27_2, c27_3 )
    gap> x := z27.1 * z27.2;
    c27_1*c27_2
    gap> y := x ^ 2;
    c27_1^2*c27_2^2
    gap> x * y;
    c27_2*c27_3
    gap> SumAgWord( x, y );
    IdAgWord |

%%%%%%%%%%%%%%%%%%%%%%%%%%%%%%%%%%%%%%%%%%%%%%%%%%%%%%%%%%%%%%%%%%%%%%%%%%%%%
\Section{ExponentAgWord}

'ExponentAgWord( <g>, <k> )'

'ExponentAgWord' returns  the exponent of the <k>.th generator in  an  ag
word <g> as integer,  where <k>  refers to the numbering of generators of
the parent group of <g>.

Let $G$ be the parent group of <g> and $(g_1, ..., g_n)$ the AG system of
$G$ and let $o_i$ be the  relative order  of $g_i$.  Then the element <g>
can be  expressed uniquely in the form $g_1^{\nu_1}\* ...\*  g_n^{\nu_n}$
for integers  $\nu_i$  between  $0$  and $o_i-1$.  The *exponent*  of the
<k>.th generator is $\nu_{<k>}$.

See also "ExponentsAgWord" and "Exponents".

|    gap> ExponentAgWord( a * b^2 * c^2 * d, 2 );
    2
    gap> ExponentAgWord( a * b^2 * c^2 * d, 4 );
    1
    gap> ExponentAgWord( a * b^2 * c^2 * d, 3 );
    0
    gap> a * b^2 * c^2 * d;
    a*b^2*d |

%%%%%%%%%%%%%%%%%%%%%%%%%%%%%%%%%%%%%%%%%%%%%%%%%%%%%%%%%%%%%%%%%%%%%%%%%%%%%
\Section{ExponentsAgWord}

'ExponentsAgWord( <g> )'\\
'ExponentsAgWord( <g>, <s>, <e> )'\\
'ExponentsAgWord( <g>, <s>, <e>, <root> )'

In its first form 'ExponentsAgWord' returns  the exponent vector of an ag
word <g>, with respect to the AG system of the supergroup of <g>, as list
of integers.  In the second form 'ExponentsAgWord' returns the sublist of
the  exponent  vector  of <g>  starting   at position  <s> and  ending at
position <e>   as list of integers.  In   the third  form the  vector  is
returned as list of finite field elements  over the same finite  field as
<root>.

Let $G$ be the parent group of <g> and $(g_1, ..., g_n)$ the AG system of
$G$ and let $o_i$ be the relative order  of $g_i$.  Then  the element <g>
can  be expressed uniquely in the  form $g_1^{\nu_1}\* ...\* g_n^{\nu_n}$
for integers $\nu_i$ between $0$ and $o_i-1$.  The exponent vector of <g>
is the list '[$\nu_1$, ..., $\nu_n$]'.

Note that you must use 'Exponents' if  you want to  get the exponent list
of <g>  with  respect not  to  the parent  group  of <g>  but  to a given
subgroup, which contains <g>.  See "Exponents" for details.

|    gap> ExponentsAgWord( a * b^2 * c^2 * d );
    [ 1, 2, 0, 1 ]
    gap> a * b^2 * c^2 * d;
    a*b^2*d |

%%%%%%%%%%%%%%%%%%%%%%%%%%%%%%%%%%%%%%%%%%%%%%%%%%%%%%%%%%%%%%%%%%%%%%%%%%%%%
%%
%Emacs setup
%E Local Variables:
%E mode:           outline
%E outline-regexp: "\\\\Chapter\\|\\\\Section\\|\\%Emacs"
%E fill-column:    73
%E eval:           (hide-body)
%E End:
%%
