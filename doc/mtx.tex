%%%%%%%%%%%%%%%%%%%%%%%%%%%%%%%%%%%%%%%%%%%%%%%%%%%%%%%%%%%%%%%%%%%%%%%%%%%%%
%%
%A  mtx.tex                  GAP documentation                  Thomas Breuer
%%
%A  @(%)$Id: mtx.tex,v 1.3 1997/01/24 17:14:43 gap Exp $
%%
%Y  Copyright 1991-1993,  Lehrstuhl D fuer Mathematik,  RWTH Aachen,  Germany
%%
%%  This file contains the interface from {\GAP} to the {\MeatAxe}, i.e., the
%%  definition of {\MeatAxe} matrices and permutations, functions for format
%%  conversions, and operations for {\MeatAxe} matrices that allow to access
%%  {\MeatAxe} programs via {\GAP}.
%%
%H  $Log: mtx.tex,v $
%H  Revision 1.3  1997/01/24 17:14:43  gap
%H  Commitments before releasing everything
%H
%H  Revision 1.2  1997/01/23 09:59:03  gap
%H  adjusted changes in examples due to new word generator of the MeatAxe
%H
%H  Revision 1.1.1.1  1996/12/11 12:40:24  werner
%H  Preparing 3.4.4 for release
%H
%H  Revision 3.4  1994/09/19  13:53:29  sam
%H  changed the format of 'MeatAxe' permutations
%H
%H  Revision 3.3  1994/06/17  19:02:47  vfelsch
%H  examples adjusted to version 3.4
%H
%H  Revision 3.2  1994/06/10  04:41:23  vfelsch
%H  updated examples
%H
%H  Revision 3.1  1994/06/03  08:57:20  mschoene
%H  changed a few things to avoid LaTeX warnings
%H
%H  Revision 3.0  1994/05/19  13:54:35  sam
%H  Initial Revision under RCS
%H
%%
\def\MeatAxe{\sf MeatAxe}
\Chapter{The MeatAxe}

This chapter describes the main functions  of the {\MeatAxe}  (Version~2.0)
share library  for  computing  with  finite field matrices, permutations,
matrix groups, matrix algebras, and their modules.

For the installation of the package, see "Installing the MeatAxe Package".

The chapter consists of seven parts.

First the idea of using the {\MeatAxe} via {\GAP} is introduced
(see "More about the MeatAxe in GAP", "GapObject"), and an example shows
how the programs can be used (see "Using the MeatAxe in GAP. An Example").

The second part describes functions and operations for single {\MeatAxe}
*matrices* (see "MeatAxe Matrices", "MeatAxeMat", "Operations for MeatAxe
Matrices", "Functions for MeatAxe Matrices", "BrauerCharacterValue").

The third part describes functions and operations for single {\MeatAxe}
*permutations* (see "MeatAxe Permutations", "MeatAxePerm", "Operations for
MeatAxe Permutations", "Functions for MeatAxe Permutations").

The fourth part describes functions and operations for *groups* of
{\MeatAxe} matrices (see "MeatAxe Matrix Groups", "Functions for MeatAxe
Matrix Groups").

(Groups of {\MeatAxe} permutations are not yet supported.)

The fifth part describes functions and operations for *algebras* of
{\MeatAxe} matrices (see "MeatAxe Matrix Algebras", "Functions for MeatAxe
Matrix Algebras").

The sixth part describes functions and operations for *modules* for
{\MeatAxe} matrix algebras (see "MeatAxe Modules", "Set Theoretic Functions
for MeatAxe Modules", "Vector Space Functions for MeatAxe Modules", "Module
Functions for MeatAxe Modules").

The last part describes the data structures (see "MeatAxe Object Records").

\vspace{5mm}

If you want to use the functions in this package you must load it using

|    gap> RequirePackage( "meataxe" );
    &I  The MeatAxe share library functions are available now.
    &I  All files will be placed in the directory
    &I     '/var/tmp/tmp.017545'
    &I  Use 'MeatAxe.SetDirectory( <path> )' if you want to change. |

%%%%%%%%%%%%%%%%%%%%%%%%%%%%%%%%%%%%%%%%%%%%%%%%%%%%%%%%%%%%%%%%%%%%%%%%%%%%%
\Section{More about the MeatAxe in GAP}

The {\MeatAxe} can be used to speed up computations that are possible also
using ordinary {\GAP} functions.  But more interesting are functions that
are not (or not yet) available in the {\GAP} library itself, such as that
for the computation of submodule lattices (see "Module Functions for MeatAxe
Modules").

The syntax of the functions is the usual {\GAP} syntax, so it might be
useful to read the chapters about algebras and modules in {\GAP}
(see chapters "Algebras", "Modules") if you want to work with {\MeatAxe}
modules.

The main idea is to let the {\MeatAxe} functions do the main work,
and use {\GAP} as a shell.  Since in {\MeatAxe} philosophy, each object is
contained in its own file, {\GAP}\'s task is mainly to keep track of these
files.  For example, for {\GAP} a {\MeatAxe} matrix is a record containing
at least information about the file name, the underlying finite field, and
the dimensions of the matrix (see "MeatAxe Matrices").
Multiplying two such matrices means to invoke the multiplication program
of {\MeatAxe}, to store the result in a new file, and notify this to {\GAP}.

This idea is used not only for basic calculations but also to access
elaborate and powerful algorithms, for example the program to compute the
composition factors of a module, or the submodule lattice (see "Module
Functions for MeatAxe Modules").

In order to avoid conversion overhead the {\MeatAxe} matrices are read into
{\GAP} only if the user explicitly applies 'GapObject' (see "GapObject"),
or applies an operator (like multiplication) to a {\MeatAxe} matrix and an
ordinary {\GAP} object.

Some of the functions, mainly 'CompositionFactors', print useful information
if the variable 'InfoMeatAxe' is set to the value 'Print'.
The default of 'InfoMeatAxe' is 'Print', if you want to suppress the
information you should set 'InfoMeatAxe' to 'Ignore'.

For details about the implementation of the standalone functions,
see~\cite{Rin93}.

%%%%%%%%%%%%%%%%%%%%%%%%%%%%%%%%%%%%%%%%%%%%%%%%%%%%%%%%%%%%%%%%%%%%%%%%%%%%%
\Section{GapObject}

'GapObject( <mtxobj> )'

returns the {\GAP} object corresponding to the {\MeatAxe} object <mtxobj>
which may be a {\MeatAxe} matrix, a {\MeatAxe} permutation, a {\MeatAxe}
matrix algebra, or a {\MeatAxe} module.

Applied to an ordinary {\GAP} object, 'GapObject' simply returns this object.

|    gap> m:= [ [ 0, 1, 0 ], [ 0, 0, 1 ], [ 1, 0, 0 ] ] * GF(2).one;;
    gap> mam:= MeatAxeMat( m, "file2" );;
    &I  calling 'maketab' for field of size 2
    gap> GapObject( mam );
    [ [ 0*Z(2), Z(2)^0, 0*Z(2) ], [ 0*Z(2), 0*Z(2), Z(2)^0 ], 
      [ Z(2)^0, 0*Z(2), 0*Z(2) ] ]
    gap> map:= MeatAxePerm( (1,2,3), 3 );;
    gap> perm:= GapObject( map );
    (1,2,3)
    gap> GapObject( perm );
    (1,2,3) |

%%%%%%%%%%%%%%%%%%%%%%%%%%%%%%%%%%%%%%%%%%%%%%%%%%%%%%%%%%%%%%%%%%%%%%%%%%%%%
\Section{Using the MeatAxe in GAP. An Example}

In this example we compute the 2-modular irreducible representations and
Brauer characters of the alternating group $A_5$.
Perhaps it will raise the question whether one uses the {\MeatAxe} in {\GAP}
or {\GAP} for the {\MeatAxe}.

First we take a permutation representation of $A_5$ and convert the
generators into {\MeatAxe} matrices over the field $GF(2)$.

|    gap> a5:= Group( (1,2,3,4,5), (1,2,3) );;
    gap> Size( a5 );
    60
    gap> f:= GF(2);;
    gap> m1:= MeatAxeMat( a5.1, f, [5,5] );
    MeatAxeMat( "/var/tmp/tmp.017545/a", GF(2), [ 5, 5 ] )
    gap> m2:= MeatAxeMat( a5.2, f, [5,5] );; |

'm1' and 'm2' are records that know about the files where the matrices
are stored.  Let\'s look at such a matrix (without reading the file
into {\GAP}).

|    gap> Display( m1 );
    MeatAxe.Matrix := [
    [0,1,0,0,0],
    [0,0,1,0,0],
    [0,0,0,1,0],
    [0,0,0,0,1],
    [1,0,0,0,0]
    ]*Z(2); |

Next we inspect the 5 dimensional permutation module over $GF(2)$.
It contains a trivial submodule 'fix', its quotient is called 'quot'.

|    gap> a:= UnitalAlgebra( f, [ m1, m2 ] );;
    gap> nat:= NaturalModule( a );;
    gap> fix:= FixedSubmodule( nat );;
    gap> Dimension( fix );
    1
    gap> quot:= nat / fix;; |

The action on 'quot' is described by an algebra of $4\times 4$ matrices,
the corresponding module turns out to be absolutely irreducible.
Of course the action on 'fix' would yield $1\times 1$ matrices,
the generators being the identity.
So we found already two of the four absolutely irreducible representations.

|    gap> op:= Operation( a, quot );
    UnitalAlgebra( GF(2),
    [ MeatAxeMat( "/var/tmp/tmp.017545/t/g.1", GF(2), [ 4, 4 ], a.1 ),
      MeatAxeMat( "/var/tmp/tmp.017545/t/g.2", GF(2), [ 4, 4 ], a.2 ) ] )
    gap> nm:= NaturalModule( op );;
    gap> IsIrreducible( nm );
    true
    gap> IsAbsolutelyIrreducible( nm );
    true
    gap> deg4:= nm.ring;; |

Now we form the tensor product of the 4 dimensional module with itself,
and compute the composition factors.

|    gap> tens:= KroneckerProduct( nm, nm );;
    gap> comp:= CompositionFactors( tens );;
    &I   Name Mult  SF
    &I     1a    4   1
    &I     4a    1   1
    &I     4b    2   2
    &I
    &I  Ascending composition series:
    &I  4a  1a  4b  1a  1a  4b  1a  
    gap> IsIrreducible( comp[3] );
    true
    gap> IsAbsolutelyIrreducible( comp[3] );
    false |

The information printed by 'CompositionFactors' told that there is an
irreducible but not absolutely irreducible factor '4b' of dimension 4,
and we will enlarge the field in order to split this module.

|    gap> sf:= SplittingField( comp[3] );
    GF(2^2)
    gap> new:= UnitalAlgebra( sf, [ comp[3].ring.1, comp[3].ring.2 ] );;
    &I  calling 'maketab' for field of size 4
    gap> nat:= NaturalModule( new );;
    gap> comp:= CompositionFactors( nat );;
    &I   Name Mult  SF
    &I     2a    1   1
    &I     2b    1   1
    &I
    &I  Ascending composition series:
    &I  2a  2b  
    gap> deg2:= List( comp, x -> x.ring );; |

Now the representations are known.  Let\'s calculate the Brauer characters.
For that, we need representatives of the 2-regular conjugacy classes of
$A_5$.

|    gap> repres:= [ a.1^0, a.1 * a.2 * a.1^3, a.1, a.1^2 ];;
    gap> List( repres, OrderMeatAxeMat );
    [ 1, 3, 5, 5 ] |

The expression of the representatives of each irreducible representation
in terms of the generators can be got using 'MappedExpression'.

|    gap> abstracts:= List( repres, x -> x.abstract );
    [ a.one, a.1*a.2*a.1^3, a.1, a.1^2 ]
    gap> mapped:= List( [ 1 .. 4 ],
    >   x-> MappedExpression( abstracts[x],
    >       a.freeAlgebra.generators, deg4.generators ) );;
    gap> List( mapped, OrderMeatAxeMat );
    [ 1, 3, 5, 5 ]
    gap> List( mapped, BrauerCharacterValue );
    [ 4, 1, -1, -1 ]
    gap> mapped:= List( [ 1 .. 4 ],
    >   x-> MappedExpression( abstracts[x],
    >       a.freeAlgebra.generators, deg2[1].generators ) );;
    gap> List( mapped, BrauerCharacterValue );
    [ 2, -1, E(5)^2+E(5)^3, E(5)+E(5)^4 ] |

The Brauer character of the trivial module is well-known, and
that of the other 2-dimensional module is a Galois conjugate of the
computed one, so we computed the 2-modular Brauer character table
of $A_5$.

It is advisable to remove all the {\MeatAxe} files before leaving {\GAP}.
Call 'MeatAxe.Unbind();' (see "MeatAxe.Unbind").

%%%%%%%%%%%%%%%%%%%%%%%%%%%%%%%%%%%%%%%%%%%%%%%%%%%%%%%%%%%%%%%%%%%%%%%%%%%%%
\Section{MeatAxe Matrices}\index{IsMeatAxeMat}

{\MeatAxe} matrices behave similar to lists of lists that are regarded as
matrices by {\GAP}, e.g., there are functions like 'Rank' or 'Transposed'
that work for both types, and one can multiply or add two {\MeatAxe}
matrices, the result being again a {\MeatAxe} matrix.
But one cannot access rows or single entries of a {\MeatAxe} matrix <mat>,
for example '<mat>[1]' will cause an error message.

{\MeatAxe} matrices are constructed or notified by "MeatAxeMat" 'MeatAxeMat'.

\vspace{5mm}

'IsMeatAxeMat( <obj> )'

returns 'true' if <obj> is a {\MeatAxe} matrix, and 'false' otherwise.

%%%%%%%%%%%%%%%%%%%%%%%%%%%%%%%%%%%%%%%%%%%%%%%%%%%%%%%%%%%%%%%%%%%%%%%%%%%%
\Section{MeatAxeMat}

'MeatAxeMat( <mat> [, <F> ] [, <abstract> ] [, <filename> ] )'

returns a {\MeatAxe} matrix corresponding to the matrix <mat>, viewed over
the finite field <F>, or over the field of all entries of <mat>.

If <mat> is already a {\MeatAxe} matrix then the call means that it
shall now be viewed over the field <F> which may be smaller or larger
than the field <mat> was viewed over.

The optional argument <abstract> is an element of a free algebra
(see chapter "Finitely Presented Algebras") that represents the matrix
in terms of generators.

If the optional argument <filename> is given, the {\MeatAxe} matrix is
written to the file with this name; a matrix constructed this way will
*not* be removed by a call to 'MeatAxe.Unbind'.  Otherwise {\GAP} creates
a temporary file under the directory 'MeatAxe.direc'.

\vspace{5mm}

'MeatAxeMat( <perm>, <F>, <dim> [,<abstract>][,<filename>] )'

does the same for a permutation <perm> that shall be converted into
a permutation matrix over the field <F>, with '<dim>[1]' rows and
'<dim>[2]' columns.

\vspace{5mm}

'MeatAxeMat( <file>, <F>, <dim> [, <abstract> ] )'

is the {\MeatAxe} matrix stored on file <file>, viewed over the field
<F>, with dimensions <dim>, and representation <abstract>.
This may be used to make a shallow copy of a {\MeatAxe} matrix, or to
notify {\MeatAxe} matrices that were not produced by {\GAP}.
Such matrices are *not* removed by calls to 'MeatAxe.Unbind'.

*Note*\:\ No field change is allowed here.

|    gap> f:= GF(2);;
    gap> m:= [ [ 0, 1, 0 ], [ 0, 0, 1 ], [ 1, 0, 0 ] ] * f.one;;
    gap> m1:= MeatAxeMat( m, "file2" );
    MeatAxeMat( "/var/tmp/tmp.005046/file2", GF(2), [ 3, 3 ] )
    gap> p:= (1,2,3);;
    gap> m2:= MeatAxeMat( p, f, [ 3, 3 ], "file" );
    MeatAxeMat( "/var/tmp/tmp.005046/file", GF(2), [ 3, 3 ] )
    gap> Display( m2 );
    MeatAxe.Matrix := [
    [0,1,0],
    [0,0,1],
    [1,0,0]
    ]*Z(2);
    gap> n:= MeatAxeMat( "file", f, [ 3, 3 ] );; # just notify a matrix |

%%%%%%%%%%%%%%%%%%%%%%%%%%%%%%%%%%%%%%%%%%%%%%%%%%%%%%%%%%%%%%%%%%%%%%%%%%%%%
\Section{Operations for MeatAxe Matrices}

*Comparisons of MeatAxe Matrices*

'<m1> = <m2>':\\ evaluates to 'true' if the two {\MeatAxe} matrices have the
                 same entries and are viewed over the same field, and to
                 'false' otherwise.
                 The test for equality uses a shell script that is produced
                 when it is needed for the first time.

'<m1> \<\ <m2>': \\ evaluates to 'true' if and only if this relation holds
                    for the file names of the two {\MeatAxe} matrices.

\vspace{5mm}

*Arithmetic Operations of MeatAxe Matrices*

The following arithmetic operations are admissible for {\MeatAxe} matrices.

'<m1> + <m2>':\\   sum of the two {\MeatAxe} matrices <m1>, <m2>

'<m1> - <m2>':\\   difference of the two {\MeatAxe} matrices <m1>, <m2>

'<m1> \*\ <m2>':\\ product of the two {\MeatAxe} matrices <m1>, <m2>

'<m1> \^\ <m2>':\\ conjugation of the {\MeatAxe} matrix <m1> by <m2>

'<m1> \^\ <n>':\\  <n>-th power of the {\MeatAxe} matrix <m1>,
                   for an integer <n>

%%%%%%%%%%%%%%%%%%%%%%%%%%%%%%%%%%%%%%%%%%%%%%%%%%%%%%%%%%%%%%%%%%%%%%%%%%%%%
\Section{Functions for MeatAxe Matrices}
\index{Algebra for MeatAxe Matrices}
\index{CharacteristicPolynomial for MeatAxe matrices}
\index{Display for MeatAxe matrices}
\index{Dimensions for MeatAxe matrices}
\index{Group for MeatAxe Matrices}
\index{KroneckerProduct for MeatAxe matrices}
\index{Order for MeatAxe matrices}
\index{Rank for MeatAxe matrices}
\index{SumIntersectionSpaces for MeatAxe matrices}
\index{Trace for MeatAxe matrices}
\index{Transposed for MeatAxe matrices}

The following functions that work for ordinary matrices in {\GAP} also
work for {\MeatAxe} matrices.

'UnitalAlgebra( <F>, <gens> )' : \\
     returns the unital <F>-algebra generated by the {\MeatAxe} matrices
     in the list <gens>.

'Base( <mtxmat> )': \\
     returns a {\MeatAxe} matrix whose rows form a vector space basis of the
     row space; the basis is in semi-echelon form.

'BaseNullspace( <mtxmat> )': \\
     returns a {\MeatAxe} matrix in semi-echelon form whose rows are a basis
     of the nullspace of the {\MeatAxe} matrix <mtxmat>.

'CharacteristicPolynomial( <mtxmat> )' : \\
     returns the characteristic polynomial of the {\MeatAxe} matrix <mtxmat>.
     The factorization of this polynomial is stored.

'Dimensions( <mtxmat> )' : \\
     returns the list '[ <nrows>, <ncols> ]' where <nrows> is
     the number of rows, <ncols> is the number of columns of
     the {\MeatAxe} matrix <mtxmat>.

'Display( <mtxmat> )' : \\
     displays the {\MeatAxe} matrix <mtxmat> (without reading into {\GAP}).

'Group( <m1>, <m2>, \ldots <mn> )' \\
'Group( <gens>, <id> )' : \\
     returns the group generated by the {\MeatAxe} matrices <m1>, <m2>,
     \ldots <mn>, resp. the group generated by the {\MeatAxe} matrices in
     the list <gens>, where <id> is the appropriate identity {\MeatAxe}
     matrix.

'InvariantForm( <mtxmats> )' : \\
     returns a {\MeatAxe} matrix $M$ such that $X^{tr} M X = M$ for all
     {\MeatAxe} matrices in the list <mtxmats> if such a matrix exists,
     and 'false' otherwise.  Note that the algebra generated by <mtxmats>
     must act irreducibly, otherwise an error is signalled.

'KroneckerProduct( <m1>, <m2> )' : \\
     returns a {\MeatAxe} matrix that is the Kronecker product of the
     {\MeatAxe} matrices <m1>, <m2>.

'Order( MeatAxeMatrices, <mtxmat> )' : \\
     returns the multiplicative order of the {\MeatAxe} matrix <mtxmat>,
     if this exists.
     This can be computed also by 'OrderMeatAxeMat( <mtxmat> )'.

'Rank( <mtxmat> )': \\
     returns the rank of the {\MeatAxe} matrix <mtxmat>.

'SumIntersectionSpaces( <mtxmat1>, <mtxmat2> )' : \\
     returns a list of two {\MeatAxe} matrices, both in semi-echelon form,
     whose rows are a basis of the sum resp. the intersection of row spaces
     generated by the {\MeatAxe} matrices <m1> and <m2>, respectively.

'Trace( <mtxmat> )' : \\
     returns the trace of the {\MeatAxe} matrix <mtxmat>.

'Transposed( <mtxmat> )' : \\
     returns the transposed matrix of the {\MeatAxe} matrix <mtxmat>.

%%%%%%%%%%%%%%%%%%%%%%%%%%%%%%%%%%%%%%%%%%%%%%%%%%%%%%%%%%%%%%%%%%%%%%%%%%%%%
\Section{BrauerCharacterValue}

'BrauerCharacterValue( <mtxmat> )'

returns the Brauer character value of the {\MeatAxe} matrix <mtxmat>, which
must of course be an invertible matrix of order relatively prime to the
characteristic of its entries.

|    gap> g:= MeatAxeMat( (1,2,3,4,5), GF(2), [ 5, 5 ] );;
    gap> BrauerCharacterValue( g );
    0 |

(This program was originally written by
J{\accent127 u}rgen M{\accent127 u}ller.)

%%%%%%%%%%%%%%%%%%%%%%%%%%%%%%%%%%%%%%%%%%%%%%%%%%%%%%%%%%%%%%%%%%%%%%%%%%%%%
\Section{MeatAxe Permutations}\index{IsMeatAxePerm}

{\MeatAxe} permutations behave similar to permutations in {\GAP}, e.g., one
can multiply two {\MeatAxe} permutations, the result being again a {\MeatAxe}
permutation.  But one cannot map single points by a {\MeatAxe} permutation
using the exponentiation operator '\^'.

{\MeatAxe} permutations are constructed or notified by "MeatAxePerm"
'MeatAxePerm'.

\vspace{5mm}

'IsMeatAxePerm( <obj> )'

returns 'true' if <obj> is a {\MeatAxe} permutation, and 'false' otherwise.

%%%%%%%%%%%%%%%%%%%%%%%%%%%%%%%%%%%%%%%%%%%%%%%%%%%%%%%%%%%%%%%%%%%%%%%%%%%%
\Section{MeatAxePerm}

'MeatAxePerm( <perm>, <maxpoint> )' \\
'MeatAxePerm( <perm>, <maxpoint>, <filename> )'

return a {\MeatAxe} permutation corresponding to the permutation <perm>,
acting on the points '[ 1 .. <maxpoint> ]'.
If the optional argument <filename> is given, the {\MeatAxe} permutation
is written to the file with this name; a permutation constructed this way
will *not* be removed by a call to 'MeatAxe.Unbind'.  Otherwise {\GAP}
creates a temporary file under the directory 'MeatAxe.direc'.

\vspace{5mm}

'MeatAxePerm( <file>, <maxpoint> )'

is the {\MeatAxe} permutation stored on file <file>.
This may be used to notify {\MeatAxe} permutations that were not produced
by {\GAP}.  Such permutations are *not* removed by calls to 'MeatAxe.Unbind'.

|    gap> p1:= MeatAxePerm( (1,2,3), 3 );
    MeatAxePerm( "/var/tmp/tmp.005046/a", 3 )
    gap> p2:= MeatAxePerm( (1,2), 3, "perm2" );
    MeatAxePerm( "/var/tmp/tmp.005046/perm2", 3 );
    gap> p:= p1 * p2;
    MeatAxePerm( "/var/tmp/tmp.005046/b", 3 )
    gap> Display( p );
    MeatAxe.Perms := [
        (2,3)
    ]; |

%%%%%%%%%%%%%%%%%%%%%%%%%%%%%%%%%%%%%%%%%%%%%%%%%%%%%%%%%%%%%%%%%%%%%%%%%%%%%
\Section{Operations for MeatAxe Permutations}

*Comparisons of MeatAxe Permutations*

'<m1> = <m2>':\\ evaluates to 'true' if the two {\MeatAxe} permutations are
                 equal as permutations, and to 'false' otherwise.
                 The test for equality uses a shell script that is produced
                 when it is needed for the first time.

'<m1> \<\ <m2>': \\ evaluates to 'true' if and only if this relation holds
                    for the file names of the two {\MeatAxe} permutations.

\vspace{5mm}

*Arithmetic Operations of MeatAxe Permutations*

The following arithmetic operations are admissible for {\MeatAxe}
permutations.

'<m1> \*\ <m2>':\\ product of the two {\MeatAxe} permutations <m1>, <m2>

'<m1> \^\ <m2>':\\ conjugation of the {\MeatAxe} permutation <m1> by <m2>

'<m1> \^\ <n>':\\  <n>-th power of the {\MeatAxe} permutation <m1>,
                   for an integer <n>

%%%%%%%%%%%%%%%%%%%%%%%%%%%%%%%%%%%%%%%%%%%%%%%%%%%%%%%%%%%%%%%%%%%%%%%%%%%%%
\Section{Functions for MeatAxe Permutations}
\index{Display for MeatAxe Permutations}
\index{Order for MeatAxe Permutations}

The following functions that work for ordinary permutations in {\GAP} also
work for {\MeatAxe} permutations.

'Display( <mtxperm> )' : \\
     displays the {\MeatAxe} permutation <mtxperm> (without reading the file
     into {\GAP}).

'Order( MeatAxePermutations, <mtxperm> )' : \\
     returns the multiplicative order of the {\MeatAxe} permutation
     <mtxperm>.
     This can be computed also by 'OrderMeatAxePerm( <mtxperm> )'.

%%%%%%%%%%%%%%%%%%%%%%%%%%%%%%%%%%%%%%%%%%%%%%%%%%%%%%%%%%%%%%%%%%%%%%%%%%%%%
\Section{MeatAxe Matrix Groups}

Groups of {\MeatAxe} matrices are constructed using the usual 'Group'
command.

Only very few functions are available for {\MeatAxe} matrix groups.
For most of the applications one is interested in matrix algebras, e.g., 
matrix representations as computed by 'Operation' when applied to an
algebra and a module.  For a permutation representation of a group of
{\MeatAxe} matrices, however, it is necessary to call 'Operation'
with a group as first argument (see "Functions for MeatAxe Matrix Groups").

%%%%%%%%%%%%%%%%%%%%%%%%%%%%%%%%%%%%%%%%%%%%%%%%%%%%%%%%%%%%%%%%%%%%%%%%%%%%%
\Section{Functions for MeatAxe Matrix Groups}
\index{Operation for MeatAxe Matrix Groups}
\index{RandomOrders for MeatAxe Matrix Groups}

The following functions are overlaid in the operations record of
{\MeatAxe} matrix groups.

'Operation( <G>, <M> )' : \\
     Let <M> a {\MeatAxe} module acted on by the group <G> of {\MeatAxe}
     matrices.
     'Operation( <G>, <M> )' returns a permutation group with action on
     the points equivalent to that of <G> on the vectors of the module <M>.

'RandomOrders( <G> )' : \\
     returns a list with the orders of 120 random elements of the {\MeatAxe}
     matrix group <G>.

It should be noted that no set theoretic functions (such as 'Size') are
provided for {\MeatAxe} matrix groups, and also group theoretic functions
(such as 'SylowSubgroup') will not work.

%%%%%%%%%%%%%%%%%%%%%%%%%%%%%%%%%%%%%%%%%%%%%%%%%%%%%%%%%%%%%%%%%%%%%%%%%%%%%
\Section{MeatAxe Matrix Algebras}

Algebras of {\MeatAxe} matrices are constructed using the usual 'Algebra'
or 'UnitalAlgebra' commands.

*Note* that *all* these algebras are regarded to be unital, that is, also
if you construct an algebra by calling 'Algebra' you will get a unital
algebra.

{\MeatAxe} matrix algebras are used to construct and describe {\MeatAxe}
modules and their structure (see "MeatAxe Modules").

For functions for {\MeatAxe} matrix algebras see "Functions for MeatAxe
Matrix Algebras".

%%%%%%%%%%%%%%%%%%%%%%%%%%%%%%%%%%%%%%%%%%%%%%%%%%%%%%%%%%%%%%%%%%%%%%%%%%%%%
\Section{Functions for MeatAxe Matrix Algebras}
\index{Fingerprint for MeatAxe Matrix Algebras}
\index{Module for MeatAxe Matrix Algebras}
\index{NaturalModule for MeatAxe Matrix Algebras}
\index{RandomOrders for MeatAxe Matrix Algebras}

The following functions are overlaid in the operations record of
{\MeatAxe} matrix algebras.

'Fingerprint( <A> )'\\
'Fingerprint( <A>, <list> )' : \\
    returns the fingerprint of <A>, i.e., a list of nullities of
    six ``standard\'\'\ words in <A> (for 2-generator algebras only) or of
    the words with numbers in <list>.

|    gap> f:= GF(2);;
    gap> a:= UnitalAlgebra( f, [ MeatAxeMat( (1,2,3,4,5), f, [5,5] ),
    >                            MeatAxeMat( (1,2)      , f, [5,5] ) ] );;
    gap> Fingerprint( a );
    [ 1, 1, 1, 3, 0, 1 ] |

'Module( <matalg>, <gens> )' : \\
    returns the module generated by the rows of the {\MeatAxe} matrix <gens>,
    and acted on by the {\MeatAxe} matrix algebra <matalg>.
    Such a module will usually contain the vectors of a basis in the 'base'
    component.

'NaturalModule( <matalg> )' : \\
    returns the <n>-dimensional space acted on by the {\MeatAxe} matrix
    algebra <matalg> which consists of $n \times n$ {\MeatAxe} matrices.

'Operation( <A>, <M> )' : \\
     Let <M> be a {\MeatAxe} module acted on by the {\MeatAxe} matrix
     algebra <A>.
     'Operation( <A>, <M> )' returns a {\MeatAxe} matrix algebra of
     $n \times n$ matrices (where $n$ is the dimension of <M>),
     with action on its natural module equivalent to that of <A> on <M>. \\
     *Note\:* \ 
     If <M> is a quotient module, it must be a quotient of the entire space.

'RandomOrders( <A> )' : \\
     returns a list with the orders of 120 random elements of the {\MeatAxe}
     matrix algebra <A>, provided that the generators of <A> are invertible.

It should be noted that no set theoretic functions (such as 'Size') and
vector space functions (such as 'Base') are provided for {\MeatAxe} matrix
algebras, and also algebra functions (such as 'Centre') will not work.

%%%%%%%%%%%%%%%%%%%%%%%%%%%%%%%%%%%%%%%%%%%%%%%%%%%%%%%%%%%%%%%%%%%%%%%%%%%%%
\Section{MeatAxe Modules}

{\MeatAxe} modules are vector spaces acted on by {\MeatAxe} matrix algebras.
In the {\MeatAxe} standalone these modules are described implicitly because
the matrices contain all the necessary information there.
In {\GAP} the modules are the concrete objects whose properties are inspected
(see "Module Functions for MeatAxe Modules").



Note that most of the usual set theoreic and vector space functions are not
provided for {\MeatAxe} modules (see "Set Theoretic Functions for MeatAxe
Modules", "Vector Space Functions for MeatAxe Modules").

%%%%%%%%%%%%%%%%%%%%%%%%%%%%%%%%%%%%%%%%%%%%%%%%%%%%%%%%%%%%%%%%%%%%%%%%%%%%%
\Section{Set Theoretic Functions for MeatAxe Modules}
\index{Size for MeatAxe Modules}
\index{Intersection for MeatAxe Modules}

'Size( <M> )' : \\
    returns the size of the {\MeatAxe} module <M>.

'Intersection( <M1>, <M2> )' : \\
    returns the intersection of the two {\MeatAxe} modules <M1>, <M2> as a
    {\MeatAxe} module.

%%%%%%%%%%%%%%%%%%%%%%%%%%%%%%%%%%%%%%%%%%%%%%%%%%%%%%%%%%%%%%%%%%%%%%%%%%%%%
\Section{Vector Space Functions for MeatAxe Modules}
\index{Base for MeatAxe Modules}
\index{Dimension for MeatAxe Modules}

'Base( <M> )' : \\
    returns a {\MeatAxe} matrix in semi-echelon form whose rows are a
    vector space basis of the {\MeatAxe} module <M>.

'Basis( <M>, <mtxmat> )' : \\
    returns a basis record for the {\MeatAxe} module <M> with basis vectors
    equal to the rows of <mtxmat>.

'Dimension( <M> )' : \\
    returns the dimension of the {\MeatAxe} module <M>.

'SemiEchelonBasis( <M> )' : \\
    returns a basis record of the {\MeatAxe} module <M> that is
    semi-echelonized (see "SemiEchelonBasis").

%%%%%%%%%%%%%%%%%%%%%%%%%%%%%%%%%%%%%%%%%%%%%%%%%%%%%%%%%%%%%%%%%%%%%%%%%%%%%
\Section{Module Functions for MeatAxe Modules}
\index{CompositionFactors for MeatAxe Modules}
\index{KroneckerProduct for MeatAxe Modules}
\index{IsIrreducible for MeatAxe Modules}
\index{IsAbsolutelyIrreducible for MeatAxe Modules}
\index{IsEquivalent for MeatAxe Modules}
\index{SplittingField for MeatAxe Modules}
\index{Lattice for MeatAxe Modules}

'CompositionFactors( <M> )' : \\
    For a {\MeatAxe} module <M> that is acted on by the algebra <A>,
    this returns a list of {\MeatAxe} modules which are the actions of <A>
    on the factors of a composition series of <M>.
    The factors occur with same succession (and multiplicity) as in the
    composition series.  The printed information means the following
    (for this example, see "Using the MeatAxe in GAP. An Example").

|    gap> tens:= KroneckerProduct( nm, nm );;
    gap> comp:= CompositionFactors( tens );;
    &I   Name Mult  SF
    &I     1a    4   1
    &I     4a    1   1
    &I     4b    2   2
    &I
    &I  Ascending composition series:
    &I  4a  1a  4b  1a  1a  4b  1a  |

    : The column with header 'Name' lists the different composition factors
      by a name consisting of the dimension and a letter to distinguish
      different modules of same dimension, the 'Mult' columns lists the
      multiplicities of the composition factor in the module, and the 'SF'
      columns lists the exponential indices of the fields of definition in
      the splitting fields.
      In this case there is one 1-dimensional module '1a' with multiplicity
      4 that is absolutely irreducible, also one 4-dimensional absolutely
      irreducible module '4a' of dimension 4, and with multiplicity 2 we
      have a 4-dimensional module '4b' that is not absolutely irreducible,
      with splitting field of order $p^{2n}$ when the field of definition
      had order $p^n$.

'FixedSubmodule( <M> )' : \\
    returns the submodule of fixed points in the {\MeatAxe} module <M> under
    the action of the generators of '<M>.ring'.

'GeneratorsSubmodule( <L>, <nr> )' : \\
    returns a {\MeatAxe} matrix whose rows are a vector space basis of the
    <nr>-th basis of the module with submodule lattice <L>.  The lattice
    can be computed using the 'Lattice' command (see below).

'GeneratorsSubmodules( <M> )' : \\
    returns a list of {\MeatAxe} matrices, one for each submodule of the
    {\MeatAxe} module <M>, whose rows are a vector space basis of the
    submodule.  This works only if <M> is a natural module.

'IsAbsolutelyIrreducible( <M> )' : \\
    returns 'true' if the {\MeatAxe} module <M> is absolutely irreducible,
    'false' otherwise.

'IsEquivalent( <M1>, <M2> )' : \\
    returns 'true' if the irreducible {\MeatAxe} modules <M1> and <M2>
    are equivalent, and 'false' otherwise.  If both <M1> and <M2> are
    reducible, an error is signalled.

'IsIrreducible( <M> )' : \\
    returns 'true' if the {\MeatAxe} module <M> is irreducible, 'false'
    otherwise.

'KroneckerProduct( <M1>, <M2> )' : \\
    returns the Kronecker product of the {\MeatAxe} modules <M1>, <M2>.
    It is *not* checked that the acting rings are compatible.

'Lattice( <M> )' : \\
    returns a list of records, each describing a component of the submodule
    lattice of <M>; it has the components 'dimensions' (a list, at position
    <i> the dimension of the <i>-th submodule), 'maxes' (a list, at
    position <i> the list of indices of the maximal submodules of submodule
    no. <i>), 'weights' (a list of edge weights), and 'XGAP' (a list used
    to display the submodule lattice in {\sf XGAP}).
    *Note* that <M> must be a natural module.

'SplittingField( <M> )' : \\
    returns the splitting field of the {\MeatAxe} module <M>.

'StandardBasis( <M>, <seed> )' : \\
    returns a standard basis record for the {\MeatAxe} module <M>.

%%%%%%%%%%%%%%%%%%%%%%%%%%%%%%%%%%%%%%%%%%%%%%%%%%%%%%%%%%%%%%%%%%%%%%%%%%%%%
\Section{MeatAxe.Unbind}

'MeatAxe.Unbind( <obj1>, <obj2>, ..., <objn> )'\\
'MeatAxe.Unbind( <listofobjects> )'

Called without arguments, this removes all files and directories
constructed by calls of 'MeatAxeMat' and 'Group', provided they are still
notified in 'MeatAxe.files', 'MeatAxe.dirs' and 'MeatAxe.fields'.

Otherwise all those files in 'MeatAxe.files', 'MeatAxe.dirs' and
'MeatAxe.fields' are removed that are specified in the argument list.

Before leaving {\GAP} after using the {\MeatAxe} functions
you should always call

|    gap> MeatAxe.Unbind(); |

%%%%%%%%%%%%%%%%%%%%%%%%%%%%%%%%%%%%%%%%%%%%%%%%%%%%%%%%%%%%%%%%%%%%%%%%%%%%%
\Section{MeatAxe Object Records}

*MeatAxe matrix records*

A {\MeatAxe} matrix in {\GAP} is a record that has necessarily the components

'isMeatAxeMat':\\ always true,

'isMatrix':\\   always true,

'domain':\\     the record 'MeatAxeMatrices',

'file' :\\      the name of the file that contains the matrix in {\MeatAxe}
                format,

'field':\\      the (finite) field the matrix is viewed over,

'dimensions':\\ list containing the numbers of rows and columns,

'operations':\\ the record 'MeatAxeMatOps'.

Optional components are

'structure':\\ algebra or group that contains the matrix,

'abstract':\\  an element of a free algebra (see "FreeAlgebra")
               representing the construction of the matrix in terms of
               generators.

Furthermore the record is used to store information whenever it is 
computed, e.g., the rank, the multiplicative order, and the inverse
of a {\MeatAxe} matrix.

\vspace{5mm}

*MeatAxe permutation records*

A {\MeatAxe} permutation in {\GAP} is a record that has necessarily
the components

'isMeatAxePerm':\\ always true,

'isPermutation':\\ always true,

'domain':\\ the record 'MeatAxePermutations',

'file':\\  the name of the file that contains the permutation in
           {\MeatAxe} format,

'maxpoint': \\ an integer <n> that means that the permutation acts on
               the point set '[ 1 .. <n> ]'

'operations':\\ the record 'MeatAxePermOps'.

Optional components are

'structure':\\ group that contains the permutation, and

'abstract':\\  an element of a free algebra (see "FreeAlgebra")
               representing the construction of the permutation in terms of
               generators.

Furthermore the record is used to store information whenever it is 
computed, e.g., the multiplicative order, and the inverse
of a {\MeatAxe} permutation.

\vspace{5mm}

*MeatAxe*

'MeatAxe' is a record that contains information about the usage of the
{\MeatAxe} with {\GAP}.  Currently it has the following components.

'PATH':\\     the path name of the directory that contains the {\MeatAxe}
              executables ,

'fields':\\   a list where position <i> is bound if and only if
            the field of order <i> has already been constructed by
            the 'maketab' command; in this case it contains the
            name of the 'pxxx.zzz' file,

'files':\\    a list of all file names that were constructed by calls
            to {\MeatAxe} (for allowing to make clean),

'dirs':\\     a list of all directory names that were constructed by calls
            to {\MeatAxe} (for allowing to make clean),

'gennames':\\ list of strings that are used as generator names
            in 'abstract' components of {\MeatAxe} matrices,

'alpha':\\    alphabet over which 'gennames' entries are formed,

'direc':\\ directory that contains all the files that are constructed
           using {\MeatAxe} functions,

'EXEC' : \\
    function of arbitrary many string arguments that calls 'Exec'
    for the concatenation of these arguments in the directory
    'MeatAxe.direc'.

'Maketab':\\ function that produces field information files,

'SetDirecory':\\ function that sets the 'direc' component,

'TmpName':\\ function of zero arguments that produces file names in the
             directory 'MeatAxe.direc',

'Unbind':\\   function to delete files (see "MeatAxe.Unbind").

Furthermore some components are bound intermediately when {\MeatAxe} output
files are read.  So you should better not use the 'MeatAxe' record to store
your own objects.

\vspace{5mm}

*Field information*

The correspondence between the {\MeatAxe} numbering and the {\GAP}
numbering of the elements of a finite field <F> is given by the function
'FFList' (see "FFList").
The element of <F> corresponding to {\MeatAxe} number <n> is
'FFList( <F> )[ <n>+1 ]',
and the {\MeatAxe} number of the field element <z> is
'Position( FFList( <F> ), <z> ) -1'.

%%%%%%%%%%%%%%%%%%%%%%%%%%%%%%%%%%%%%%%%%%%%%%%%%%%%%%%%%%%%%%%%%%%%%%%%%%%%%
%%
%E  Emacs . . . . . . . . . . . . . . . . . . . . . . . local emacs variables
%%
%%  Local Variables:
%%  mode:               outline
%%  outline-regexp:     "%F\\|%V\\|%E"
%%  fill-column:        73
%%  fill-prefix:        "%%  "
%%  eval:               (hide-body)
%%  End:
