%%%%%%%%%%%%%%%%%%%%%%%%%%%%%%%%%%%%%%%%%%%%%%%%%%%%%%%%%%%%%%%%%%%%%%%%%%%%%
%%
%A  chvirr.tex       CHEVIE documentation       Meinolf Geck, Frank Luebeck,
%A                                                Jean Michel, G"otz Pfeiffer
%%
%%
%Y  Copyright (C) 2005   Lehrstuhl D f\"ur Mathematik, RWTH Aachen, IWR
%Y  der Universit\"at Heidelberg, University of St. Andrews, and   University
%Y  Paris VII.
%%
%%  This  file  contains  the  description  of  the  GAP functions of CHEVIE
%%  to get characters and classes of irreducible complex reflection groups
%%
%%%%%%%%%%%%%%%%%%%%%%%%%%%%%%%%%%%%%%%%%%%%%%%%%%%%%%%%%%%%%%%%%%%%%%%%%
\def\Sym{{\mathfrak S}}

\Chapter{Classes and representations for reflection groups}

The  'CharTable' of  a finite  complex reflection  group <W> is computed in
{\CHEVIE}  using  the  decomposition  of  <W>  in  irreducible  groups (see
"ReflectionType"). For each irreducible group the character table is either
computed using recursive formulas for the infinite series, or read into the
system  from  a  library  file  for  the exceptional types. Thus, character
tables  can be obtained  quickly even for  very large groups (e.g., $E_8$).
Similar remarks apply for conjugacy classes.

The  conjugacy  classes  and  irreducible  characters of irreducible finite
complex reflection groups have canonical labelings by certain combinatorial
objects;  these  labelings  are  used  in  the  tables  of \CHEVIE. For the
classes,  these are partitions or partition tuples for the infinite series,
or,   for  exceptional   Coxeter  groups,   Carter\'s  admissible  diagrams
\cite{Car72}  (for other  primitive complex  reflection groups  we just use
words  in the generators to specify the classes). For the characters, these
are  again partitions or partition tuples  for the infinite series, and for
the  others they are pairs of two  integers $(d,e)$ where $d$ is the degree
of  the character and $e$ is the smallest symmetric power of the reflection
representation  containing  the  given  character  as  a  constituent  (the
$b$-invariant  of the character). This information is obtained by using the
functions  'ChevieClassInfo' and 'ChevieCharInfo'  (and some of  it is also
available   more  directly  via  the  functions  'CharParams',  'CharName',
'HighestPowerFakeDegrees').  When you display the  character table in \GAP,
the canonical labelings for classes and characters are those displayed.

A   typical  example  is  'CoxeterGroup(\"A\",<n>)',  the  symmetric  group
$\Sym_{n+1}$  where classes and characters  are parameterized by partitions
of $n+1$.

|    gap> W := CoxeterGroup( "A", 3 );;
    gap> Display( CharTable( W ));
    A3

          2    3    2    3    .  2
          3    1    .    .    1  .

            1111  211   22   31  4
         2P 1111 1111 1111   31 22
         3P 1111  211   22 1111  4

    1111       1   -1    1    1 -1
    211        3   -1   -1    .  1
    22         2    .    2   -1  .
    31         3    1   -1    . -1
    4          1    1    1    1  1
    |

The 'charTable' record (computed the first time the function 'CharTable' is
called) is a usual character table record as defined in \GAP, but with some
additional  components.  The  components  'classtext', 'classnames' contain
information  as  described  for  'ChevieClassInfo' (see "ChevieClassInfo").
There  is also  a field  'irredinfo', which  is a  list of records for each
irreducible  character which have components  'charname' and 'charparam' as
described for 'ChevieCharInfo' (see "ChevieCharInfo").

|    gap> W := CoxeterGroup( "G", 2);;
    gap> ct := CharTable( W );
    CharTable( "G2" )
    gap> ct.classtext;
    [ [  ], [ 2 ], [ 1 ], [ 1, 2 ], [ 1, 2, 1, 2 ], [ 1, 2, 1, 2, 1, 2 ] ]
    gap> ct.classnames;
    [ "A0", "~A1", "A1", "G2", "A2", "A1+~A1" ]
    gap> ct.irredinfo;
    [ rec(
          charparam := [ [ 1, 0 ] ],
          charname := "\\phi_{1,0}" ), rec(
          charparam := [ [ 1, 6 ] ],
          charname := "\\phi_{1,6}" ), rec(
          charparam := [ [ 1, 3, 1 ] ],
          charname := "\\phi_{1,3}'" ), rec(
          charparam := [ [ 1, 3, 2 ] ],
          charname := "\\phi_{1,3}''" ), rec(
          charparam := [ [ 2, 1 ] ],
          charname := "\\phi_{2,1}" ), rec(
          charparam := [ [ 2, 2 ] ],
          charname := "\\phi_{2,2}" ) ]|

Recall  that our groups acts a reflection group on the vector space $V$, so
have  fake degrees  (see "FakeDegree").  The valuation  and degree of these
give  two  integers  $b,B$  for  each  irreducible  character  of  $W$ (see
"LowestPowerFakeDegrees" and "HighestPowerFakeDegrees"). For finite Coxeter
groups,   the  valuation  and   degree  of  the   generic  degrees  of  the
one-parameter  generic Hecke algebra give two  more integers $a,A$ (see the
functions  "LowestPowerGenericDegrees",  "HighestPowerGenericDegrees",  and
\cite[Ch.11]  {Car85} for  more details).  These will  also be  used in the
operations  of truncated  inductions explained  in the  chapter "Reflection
subgroups".

Iwahori-Hecke  algebras and  cyclotomic Hecke  algebras also have character
tables, see the corresponding chapters.

We  now describe for each type our conventions for labeling the classes and
characters.

\smallskip
\noindent  {\em Type  $A_n$ ($n  \ge 0$)}.  In this  case we  have $W \cong
\Sym_{n+1}$. The classes and characters are labeled by partitions of $n+1$.
The  partition corresponding  to a  class describes  the cycle type for the
elements   in  that  class;  the  representative  in  '.classtext'  is  the
concatenation of the words corresponding to each part, and to a part $i$ is
associated the product of $i-1$ consecutive generators (starting one higher
that  the  last  generator  used  for  the  previous  parts). The partition
corresponding  to a character describes the type of the Young subgroup such
that  the  trivial  character  induced  from  this  subgroup  contains that
character  with  multiplicity~$1$  and  such  that  every  other  character
occurring  in this induced character has a higher $a$-value. Thus, the sign
character   corresponds  to  the  partition  $(1^{n+1})$  and  the  trivial
character  to  the  partition  $(n+1)$.  The  character  of  the reflection
representation of $W$ is labeled by $(n,1)$.

\medskip
\noindent  {\em  Type  $B_n$  ($n  \ge  2$)}.  In  this  case $W=W(B_n)$ is
isomorphic  to the wreath product of the cyclic group of order~$2$ with the
symmetric   group   $\Sym_n$.   Hence   the   classes  and  characters  are
parameterized by pairs of partitions such that the total sum of their parts
equals~$n$.  The pair corresponding  to a class  describes the signed cycle
type  for  the  elements  in  that  class,  as  in \cite{Car72}. We use the
convention   that  if  $(\lambda,\mu)$  is   such  a  pair  then  $\lambda$
corresponds  to  the  positive  and  $\mu$  to  the  negative cycles. Thus,
$(1^n,-)$  and $(-,1^n)$ label  the trivial class  and the class containing
the longest element, respectively. The pair corresponding to an irreducible
character is determined via Clifford theory, as follows.

We  have a  semidirect product  decomposition $W(B_n)=N\rtimes\Sym_n$ where
$N$  is the standard $n$-dimensional $\F_2^n$-vector space. For $a,b \ge 0$
such  that $n=a+b$  let $\eta_{a,b}$  be the  irreducible character  of $N$
which   takes  value~$1$  on  the  first~$a$  standard  basis  vectors  and
value~$-1$  on the next $b$ standard basis vectors of~$N$. Then the inertia
subgroup  of $\eta_{a,b}$ has the form $T_{a,b}\:=N.(\Sym_a \times \Sym_b)$
and  we  can  extend  $\eta_{a,b}$  trivially  to  an irreducible character
$\tilde\eta_{a,b}$  of $T_{a,b}$. Let $\alpha$ and $\beta$ be partitions of
$a$  and $b$, respectively. We take the tensor product of the corresponding
irreducible  characters  of  $\Sym_a$  and  $\Sym_b$  and regard this as an
irreducible   character  of  $T_{a,b}$.  Multiplying  this  character  with
$\tilde\eta_{a,b}$ and inducing to $W(B_n)$ yields an irreducible character
$\chi=  \chi_{(\alpha,\beta)}$ of $W(B_n)$. This defines the correspondence
between irreducible characters and pairs of partitions as above.

For   example,  the  pair  $((n),-)$   labels  the  trivial  character  and
$(-,(1^n))$  labels  the  sign  character.  The  character  of  the natural
reflection representation is labeled by $((n-1),(1))$.

\smallskip
\noindent  {\em Type  $D_n$ ($n  \ge 4$)}.  In this  case $W=W(D_n)$ can be
embedded  as a subgroup  of index~$2$ into  the Coxeter group $W(B_n)$. The
intersection  of a  class of  $W(B_n)$ with  $W(D_n)$ is  either empty or a
single  class in $W(D_n)$ or  splits up into two  classes in $W(D_n)$. This
also  leads to a  parameterization of the  classes of $W(D_n)$  by pairs of
partitions $(\lambda,\mu)$ as before but where the number of parts of $\mu$
is  even and where there are two classes of this type if $\mu$ is empty and
all  parts of  $\lambda$ are  even. In  the latter  case we  denote the two
classes  in $W(D_n)$ by  $(\lambda,+)$ and $(\lambda,-)$,  where we use the
convention   that   the   class   labeled   by   $(\lambda,+)$  contains  a
representative  which can be written  as a word in $\{s_1,s_3,\ldots,s_n\}$
and  $(\lambda,-)$ contains a representative which can be written as a word
in $\{s_2,s_3, \ldots,s_n\}$.

By  Clifford theory the restriction of an irreducible character of $W(B_n)$
to  $W(D_n)$  is  either  irreducible  or  splits  up  into two irreducible
components.  Let $(\alpha,\beta)$ be a pair of partitions with total sum of
parts  equal to $n$.  If $\alpha \neq  \beta$ then the  restrictions of the
irreducible characters of $W(B_n)$ labeled by $(\alpha,\beta)$ and $(\beta,
\alpha)$  are irreducible and equal. If $\alpha=\beta$ then the restriction
of  the character labeled by  $(\alpha,\alpha)$ splits into two irreducible
components which we denote by $(\alpha,+)$ and $(\alpha,-)$. Note that this
can  only happen  if $n$  is even.  In order  to fix  the notation we use a
result of \cite{Ste89} which describes the value of the difference of these
two  characters  on  a  class  of  the  form  $(\lambda,+)$ in terms of the
character  values of  the symmetric  group $\Sym_{n/2}$.  Recall that it is
implicit  in the  notation $(\lambda,+)$  that all  parts of  $\lambda$ are
even.  Let $\lambda^\prime$ be the partition  of $n/2$ obtained by dividing
each  part by~$2$. Then  the value of $\chi_{(\alpha,-)}-\chi_{(\alpha,+)}$
on an element in the class $(\lambda,+)$ is given by $2^{k(\lambda)}$ times
the  value of the irreducible character of $\Sym_{n/2}$ labeled by $\alpha$
on  the class of  cycle type $\lambda^\prime$.  (Here, $k(\lambda)$ denotes
the number of non-zero parts of $\lambda$.)

The  labels for the trivial, the  sign and the natural reflection character
are  the same as  for $W(B_n)$, since  these characters are restrictions of
the corresponding characters of $W(B_n)$.

\smallskip{\em The groups $G(d,1,n)$}.
They  are isomorphic to the wreath product of the cyclic group of order~$d$
with  the symmetric  group $\Sym_n$.  Hence the  classes and characters are
parameterized  by $d$-tuples of partitions such that the total sum of their
parts  equals~$n$. The words chosen as  representatives of the classes are,
when  $d>2$, computed in a slightly different  way than for $B_n$, in order
to  agree with the words on which  Ram and Halverson compute the characters
of  the Hecke algebra. First the parts  of the $d$ partitions are merged in
one  big partition  and sorted  in increasing  order. Then,  to a  part $i$
coming  from the $j$-th partition is associated the word $(l+1\ldots1\ldots
l+1)^{j-1}l+2\ldots l+i$ where $l$ is the highest generator used to express
the previous part.

The  $d$-tuple corresponding to an  irreducible character is determined via
Clifford  theory in  a similar  way than  for the  $B_n$ case. The identity
character  has the first  partition with one  part equal $n$  and the other
ones  empty. The character of the  reflection representations has the first
two  partitions with one part  equal respectively to $n-1$  and to $1$, and
the other partitions empty.

\smallskip{\em The groups $G(de,e,n)$}.
They  are normal  subgroups of  index $e$  in $G(de,1,n)$.  The quotient is
cyclic,  generated by the image $g$  of the first generator of $G(de,1,n)$.
The  classes are parameterized as the  classes of $G(de,e,n)$ with an extra
information for a component of a class which splits.

According  to \cite{Hu85},  a class  $C$ of  $G(de,1,n)$ parameterized by a
$de$-partition  $(S_0,\ldots,S_{de-1})$  is  in  $G(de,e,n)$ if $e$ divides
$\sum_i  i \sum_{p\in S_i}p$. It splits in  $d$ classes for the largest $d$
dividing $e$ and all parts of all $S_i$ and such that $S_i$ is empty if $d$
does  not  divide  $i$.  If  $w$  is  in  $C$ then 'g\^i w g\^-i' for 'i in
[0..d-1]' are representatives of the classes of $G(de,e,n)$ which meet $C$.
They are described by appending the integer $i$ to the label for $C$.

The  characters are described by Clifford theory. We make $g$ act on labels
for  characters of $G(de,1,n)$  . The action  of $g$ permutes circularly by
$d$  the partitions in the $de$-tuple.  A character has same restriction to
$G(de,e,n)$  as its transform by $g$.  The number of irreducible components
of its restriction is equal to the order $k$ of its stabilizer under powers
of  $g$.  We  encode  a  character  of  $G(de,e,n)$  by first, choosing the
smallest  for lexicographical order label  of a character whose restriction
contains  it; then this label is periodic with a motive repeated $k$ times;
we  represent the  character by  one of  these motives,  to which we append
$E(k)^i$ for 'i in [0..k-1]' to describe which component of the restriction
we choose.

\smallskip
{\em  Types  $G_2$  and  $F_4$}.  The  matrices of character values and the
orderings  and labelings of the irreducible characters are exactly the same
as in \cite[p.412/413]{Car85}\: in type $G_2$ the character
$\phi_{1,3}^\prime$  takes the value -1 on the reflection associated to the
long  simple  root;  in  type  $F_4$,  the characters $\phi_{1,12}^\prime$,
$\phi_{2,4}^\prime$,     $\phi_{4,7}^\prime$,    $\phi_{8,9}^\prime$    and
$\phi_{9,6}^\prime$  occur in  the induced  of the  identity from the $A_2$
corresponding  to the short simple  roots; the pairs ($\phi_{2,16}^\prime$,
$\phi_{2,4}^{\prime\prime}$) and ($\phi_{8,3}^\prime$,
$\phi_{8,9}^{\prime\prime}$)  are related by tensoring by sign; and finally
$\phi_{6,6}^{\prime\prime}$  is  the  exterior  square  of  the  reflection
representation.  Note, however, that  in {\CHEVIE} we  put the long root at
the left of the Dynkin diagrams to be in accordance with the conventions in
\cite[(4.8) and (4.10)]{Lus85}.

The  classes are labeled  by Carter\'s admissible  diagrams \cite{Car72}. A
character is labeled by a pair $(d,b)$ where $d$ denotes the degree and $b$
the  corresponding $b$-invariant. If there  are several characters with the
same pair $(d,b)$ we attach a prime to them, as in \cite{Car85}.

\smallskip  {\em Types $E_6,E_7,E_8$}. The character tables are obtained by
specialization  of those of  the Hecke algebra.  The classes are labeled by
Carter\'s  admissible diagrams \cite{Car72}. A  character is labeled by the
pair  $(d,b)$ where  $d$ denotes  the degree  and $b$  is the corresponding
$b$-invariant.  For  these  types,  this  gives  a  unique  labeling of the
characters.

\smallskip
{\em  Non-crystallographic types $I_2(m)$, $H_3$, $H_4$}. In these cases we
do  not have canonical labelings for the  classes. We label them by reduced
expressions.

Each  character for type  $H_3$ is uniquely  determined by the pair $(d,b)$
where  $d$ is the degree and  $b$ the corresponding $b$-invariant. For type
$H_4$  there are just  two characters (those  of degree~$30$) for which the
corresponding  pairs are  the same.  These two  characters are nevertheless
distinguished by their fake degrees\:\ the character
$\phi_{30,10}^\prime$  has  fake  degree  $q^{10}+q^{12}+$  higher terms,
while $\phi_{30,10}^{\prime\prime}$ has fake degree $q^{10}+q^{14}+$ higher
terms.  The characters in  the \CHEVIE-table for  type $H_4$ are ordered in
the same way as in \cite{AL82}.

Finally,  the characters  of degree~$2$  for type  $I_2(m)$ are  ordered as
follows.  The matrix representations affording the characters of degree~$2$
are given by\:
$$ \rho_j \colon s_1s_2 \mapsto
\left(\begin{array}{cc}E(m)^j&0\\0&E(m)^{-j}\end{array}\right),
\quad s_1\mapsto\left(\begin{array}{cc}0&1\\1&0\end{array}\right),$$
where  $1 \le j \le  \lfloor (m-1)/2\rfloor$. The reflection representation
is  $\rho_1$. The  characters in  the \CHEVIE-table  are ordered by listing
first the characters of degree 1 and then $\rho_1,\rho_2,\ldots$.

\smallskip
{\em Primitive complex reflection groups $G_4$ to $G_{34}$}.
The groups $G_{23}=H_3$, $G_{28}=F_4$, $G_{30}=H_4$ are exceptional Coxeter
groups and have been explained above. Similarly for the other groups labels
for  characters consist primarily of the pair $(d,b)$ where $d$ denotes the
degree  and $b$ is the corresponding  $b$-invariant. This is sufficient for
$G_4$, $G_{12}$, $G_{22}$ and $G_{24}$. For other groups there are pairs or
triples  of characters which  have the same  $(d,b)$ value. We disambiguate
these  according to  the conventions  of \cite{Mal00}  for $G_{27}, G_{29},
G_{31}, G_{33}$ and $G_{34}$\:

\begin{itemize}
\item  For $G_{27}$\:\
The   fake  degree  of   $\phi_{3,5}^\prime$  (resp.  $\phi_{3,20}^\prime$,
$\phi_{8,9}^{\prime\prime}$) has smaller degree that of
$\phi_{3,5}^{\prime\prime}$       (resp.      $\phi_{3,20}^{\prime\prime}$,
$\phi_{8,9}^\prime$).     The    characters     $\phi_{5,15}^\prime$    and
$\phi_{5,6}^\prime$  occur  with  multiplicity  1  in  the induced from the
trivial  character of the parabolic subgroup of type $A_2$ generated by the
first  and third generator (it is  asserted mistakenly in \cite{Mal00} that
$\phi_{5,6}^{\prime\prime}$  does not occur in this induced; it occurs with
multiplicity 2).
\item  For $G_{29}$\:\
The  character $\phi_{6,10}^{\prime\prime\prime}$ is the exterior square of
$\phi_{4,1}$; its complex conjugate is
$\phi_{6,10}^{\prime\prime\prime\prime}$. The character
$\phi_{15,4}^{\prime\prime}$  occurs in  $\phi_{4,1}\otimes\phi_{4,3}$; the
character  $\phi_{15,12}^{\prime\prime}$ is tensored  by the sign character
from  $\phi_{15,4}^{\prime\prime}$. Finally  $\phi_{6,10}^\prime$ occurs in
the  induced from the trivial character  of the standard parabolic subgroup
of type $A_3$ generated by the first, second and fourth generators.
\item  For $G_{31}$\:\
The     characters    $\phi_{15,8}^\prime$,    $\phi_{15,20}^\prime$    and
$\phi_{45,8}^{\prime\prime}$  occur in  $\phi_{4,1}\otimes\phi_{20,7}$; the
character  $\phi_{20,13}^\prime$ is complex conjugate of $\phi_{20,7}$; the
character $\phi_{45,12}^\prime$ is tensored by sign of
$\phi_{45,8}^\prime$.  The two terms of maximal degree of the fakedegree of
$\phi_{30,10}^\prime$ are $q^{50}+q^{46}$ while for
$\phi_{30,10}^{\prime\prime}$ they are $q^{50}+2q^{46}$.
\item  For $G_{33}$\:\
The  terms of maximal degree of  the fakedegree of $\phi_{10,8}^\prime$ are
$q^{28}+q^{26}$  while for  $\phi_{10,8}^\prime$ they  are $q^{28}+q^{24}$.
The  terms of maximal degree of  the fakedegree of $\phi_{40,5}^\prime$ are
$q^{31}+q^{29}$    while   for    $\phi_{40,5}^{\prime\prime}$   they   are
$q^{31}+2q^{29}$.  The character $\phi_{10,17}^\prime$  is tensored by sign
of  $\phi_{10,8}^\prime$ and  $\phi_{40,14}^\prime$ is  tensored by sign of
$\phi_{40,5}^\prime$.
\item  For $G_{34}$\:\
The character $\phi_{20,33}^\prime$ occurs in
$\phi_{6,1}\otimes\phi_{15,14}$.   The  character  $\phi_{70,9}^\prime$  is
rational.    The    character    $\phi_{70,9}^{\prime\prime}$   occurs   in
$\phi_{6,1}\otimes\phi_{15,14}$.  The  character  $\phi_{70,45}^\prime$  is
rational.The  character  $\phi_{70,45}^{\prime\prime}$  is  tensored by the
determinant   character  of   $\phi_{70,9}^{\prime\prime}$.  The  character
$\phi_{560,18}^\prime$ is rational. The character
$\phi_{560,18}^{\prime\prime\prime}$ occurs in
$\phi_{6,1}\otimes\phi_{336,17}$.   The   character  $\phi_{280,12}^\prime$
occurs in $\phi_{6,1}\otimes\phi_{336,17}$. The character
$\phi_{280,30}^{\prime\prime}$  occurs in $\phi_{6,1}\otimes\phi_{336,17}$.
The character $\phi_{540,21}^\prime$ occurs in
$\phi_{6,1}\otimes\phi_{105,20}$.  The  character  $\phi_{105,8}^\prime$ is
complex  conjugate of $\phi_{105,4}$, and $\phi_{840,13}^\prime$ is complex
conjugate  of  $\phi_{840,11}$.  The  character  $\phi_{840,23}^\prime$  is
complex conjugate of $\phi_{840,19}$. Finally $\phi_{120,21}^\prime$ occurs
in induced from the trivial character of the standard parabolic subgroup of
type  $A_5$  generated  by  the  generators  of $G_{34}$ with the third one
omitted.
\end{itemize}

For the groups $G_5$ and $G_7$ we adopt the following conventions.
For $G_5$ they are compatible with those of \cite{MR03} and \cite{BMM14}.
\begin{itemize}
\item  For $G_5$\:\
We  let |W:=ComplexReflectionGroup(5)|,  so the  generators in \CHEVIE\ are
|W.1| and |W.2|.

The     character    $\phi_{1,4}^\prime$    (resp.    $\phi_{1,12}^\prime$,
$\phi_{2,3}^\prime$)  takes the value $1$ (resp. |E(3)|, |-E(3)|) on |W.1|.
The   character   $\phi_{1,8}^{\prime\prime}$   is   complex  conjugate  to
$\phi_{1,16}$,  and the character  $\phi_{1,8}^\prime$ is complex conjugate
to  $\phi_{1,4}^\prime$  .  The  character  $\phi_{2,5}^{\prime\prime}$  is
complex  conjugate to $\phi_{2,1}$; $\phi_{2,5}^\prime$ take the value $-1$
on  |W.1|.  The  character  $\phi_{2,7}^\prime$  is  complex  conjugate  to
$\phi_{2,5}^\prime$.

\item  For $G_7$\:\
We  let |W:=ComplexReflectionGroup(7)|,  so the  generators in \CHEVIE\ are
|W.1|, |W.2| and |W.3|.

The  characters $\phi_{1,4}^\prime$ and $\phi_{1,10}^\prime$ take the value
$1$   on  |W.2|.  The   character  $\phi_{1,8}^{\prime\prime}$  is  complex
conjugate  to $\phi_{1,16}$ and $\phi_{1,8}^\prime$ is complex conjugate to
$\phi_{1,4}^\prime$.     The     characters     $\phi_{1,12}^\prime$    and
$\phi_{1,18}^\prime$   take  the  value  |E(3)|  on  |W.2|.  The  character
$\phi_{1,14}^{\prime\prime}$  is  complex  conjugate  to  $\phi_{1,22}$ and
$\phi_{1,14}^\prime$  is  complex  conjugate  to  $\phi_{1,10}^\prime$. The
character   $\phi_{2,3}^\prime$  takes  the  value  |-E(3)|  on  |W.2|  and
$\phi_{2,13}^\prime$   takes  the  value  $-1$  on  |W.2|.  The  characters
$\phi_{2,11}^{\prime\prime}$, $\phi_{2,5}^{\prime\prime}$,
$\phi_{2,7}^{\prime\prime\prime}$ and $\phi_{2,1}$ are Galois conjugate, as
well   as   the   characters   $\phi_{2,7}^\prime$,   $\phi_{2,13}^\prime$,
$\phi_{2,11}^\prime$ and $\phi_{2,5}^\prime$. The character
$\phi_{2,9}^\prime$    is   complex   conjugate    to   $\phi_{2,15}$   and
$\phi_{2,9}^{\prime\prime\prime}$ is complex conjugate to
$\phi_{2,3}^\prime$.
\end{itemize}

Finally,  for  the  remaining  groups  $G_6,  G_8$  to  $G_{11}, G_{13}$ to
$G_{21}$, $G_{25}$, $G_{26}$, $G_{32}$ and $G_{33}$ there are only pairs of
characters  with  same  value  $(d,b)$.  We  give labels uniformly to these
characters by applying in order the following rules \:\

\begin{itemize}
\item   If  the   two  characters   have  different   fake  degrees,  label
$\phi_{d,b}^\prime$   the  one  whose  fake   degree  is  minimal  for  the
lexicographic order of polynomials (starting with the highest term).

\item  For the not yet labeled pairs, if only one of the two characters has
the   property  that  in  its  Galois  orbit  at  least  one  character  is
distinguished by its $(d,b)$-invariant, label it $\phi_{d,b}^\prime$.

\item  For the not yet  labeled pairs, if the  minimum of the $(d,b)$-value
(for  the  lexicographic  order  $(d,b)$)  in  the Galois orbits of the two
character  is different,  label $\phi_{d,b}^\prime$  the character with the
minimal minimum.

\item
We  define now a new invariant for  characters\:\ consider all the pairs of
irreducible  characters  $\chi$  and  $\psi$  uniquely  determined by their
$(d,b)$-invariant such that $\phi$ occurs with non-zero multiplicity $m$ in
$\chi\otimes\psi$. We define $t(\phi)$ to be the minimal (for lexicographic
order) possible list $(d(\chi),b(\chi),d(\psi),b(\psi),m)$.

For the not yet labeled pairs, if  the t-invariants are different,
label $\phi_{d,b}^\prime$ the character with the minimal $t$-invariant.
\end{itemize}

After  applying  the  last  rule  all  the  pairs  will be labelled for the
considered  groups.  The  labelling  obtained  is  compatible for $G_{25}$,
$G_{26}$,  $G_{32}$ and  $G_{33}$ with  that of  \cite{Mal00} and for $G_8$
with that described in \cite{MR03}.

We  should  emphasize  that  for  all  groups  with  a  few exceptions, the
parameters  for characters do  not depend on  any non-canonical choice. The
exceptions   are  $G(de,e,n)$  with  $e>1$,  and  $G_5$,  $G_7$,  $G_{27}$,
$G_{28}$,  $G_{29}$ and  $G_{34}$, groups  which admit  outer automorphisms
preserving  the set of reflections,  so choices of a  particular value on a
particular generator must be made for characters which are not invariant by
these automorphisms.

\noindent{\em   Labels  for  the  classes.}  For  the  exceptional  complex
reflection  groups, the labels for  the classes represent the decomposition
of  a representative  of the  class as  a product  of generators,  with the
additional  conventions that 'z' represents the generator of the center and
for  well-generated groups 'c'  represents a Coxeter  element (a product of
the  generators  which  is  a  regular  element  for the highest reflection
degree).

%%%%%%%%%%%%%%%%%%%%%%%%%%%%%%%%%%%%%%%%%%%%%%%%%%%%%%%%%%%%%%%%%%%%%%%%%
\Section{ChevieClassInfo}
\index{ChevieClassInfo}

'ChevieClassInfo( <W> )'

returns  information about the  conjugacy classes of  the finite reflection
group <W>. The result is a record with three components\:\

'classtext':\\ contains words in the generators  describing representatives
     of  each conjugacy class.  Each word is  a list of  integers where the
     generator  $s_i$ is represented by the integer $i$. For finite Coxeter
     groups, it is the same as
     'List(ConjugacyClasses(W),x->CoxeterWord(W,Representative(x)))',   and
     each  such representative is of minimal  length in its conjugacy class
     and is a \"very good\" element in the sense of \cite{GM97}.

'classparams':\\   The elements of this list are tuples which have one
     component  for each irreducible component of <W>. These components for
     the infinite series, contain partitions or partition tuples describing
     the  class (see the introduction).  For the exceptional Coxeter groups
     they  contain  Carter\'s  admissible  diagrams,  see \cite{Car72}. For
     exceptional complex reflection groups they contain in general the same
     information as in classtext.

'classnames':\\ Contains strings describing  the conjugacy classes, made
     out of the information in 'classparams'.

|    gap> ChevieClassInfo(CoxeterGroup( "D", 4 ));
    rec(
      classtext :=
       [ [  ], [ 1, 2 ], [ 1, 2, 3, 1, 2, 3, 4, 3, 1, 2, 3, 4 ], [ 1 ],
          [ 1, 2, 3 ], [ 1, 2, 4 ], [ 1, 4 ], [ 2, 4 ],
          [ 1, 3, 1, 2, 3, 4 ], [ 1, 3 ], [ 1, 2, 3, 4 ], [ 1, 4, 3 ],
          [ 2, 4, 3 ] ],
      classparams :=
       [ [ [ [ 1, 1, 1, 1 ], [  ] ] ], [ [ [ 1, 1 ], [ 1, 1 ] ] ],
          [ [ [  ], [ 1, 1, 1, 1 ] ] ], [ [ [ 2, 1, 1 ], [  ] ] ],
          [ [ [ 1 ], [ 2, 1 ] ] ], [ [ [ 2 ], [ 1, 1 ] ] ],
          [ [ [ 2, 2 ], '+' ] ], [ [ [ 2, 2 ], '-' ] ],
          [ [ [  ], [ 2, 2 ] ] ], [ [ [ 3, 1 ], [  ] ] ],
          [ [ [  ], [ 3, 1 ] ] ], [ [ [ 4 ], '+' ] ], [ [ [ 4 ], '-' ] ] ],
      classnames := [ "1111.", "11.11", ".1111", "211.", "1.21", "2.11",
          "22.+", "22.-", ".22", "31.", ".31", "4.+", "4.-" ])
    gap> ChevieClassInfo(ComplexReflectionGroup(3,1,2));
    rec(
      classparams :=
       [ [ [ [ 1, 1 ], [  ], [  ] ] ], [ [ [ 1 ], [ 1 ], [  ] ] ],
          [ [ [ 1 ], [  ], [ 1 ] ] ], [ [ [  ], [ 1, 1 ], [  ] ] ],
          [ [ [  ], [ 1 ], [ 1 ] ] ], [ [ [  ], [  ], [ 1, 1 ] ] ],
          [ [ [ 2 ], [  ], [  ] ] ], [ [ [  ], [ 2 ], [  ] ] ],
          [ [ [  ], [  ], [ 2 ] ] ] ],
      classtext :=
       [ [  ], [ 1 ], [ 1, 1 ], [ 1, 2, 1, 2 ], [ 1, 1, 2, 1, 2 ],
          [ 1, 1, 2, 1, 2, 2, 1, 2 ], [ 2 ], [ 1, 2 ], [ 1, 1, 2 ] ],
      classnames := [ "11..", "1.1.", "1..1", ".11.", ".1.1", "..11",
          "2..", ".2.", "..2" ])|

See also the introduction of this section.

%%%%%%%%%%%%%%%%%%%%%%%%%%%%%%%%%%%%%%%%%%%%%%%%%%%%%%%%%%%%%%%%%%%%%%%%%
\Section{CharNames for reflection groups}
\index{CharNames for reflection groups}

'CharNames( <W> [,<options>] )'

returns  the  list  of  character  names  for the reflection group <W>. The
optional  <options> is a record which can give alternative names in certain
cases, or a different formatting of names in general.

|    gap> W:=CoxeterGroup("G",2);
    CoxeterGroup("G",2)
    gap> CharNames(W);
    [ "phi{1,0}", "phi{1,6}", "phi{1,3}'", "phi{1,3}''", "phi{2,1}",
      "phi{2,2}" ]
    gap> CharNames(W,rec(TeX:=true));
    [ "\\phi_{1,0}", "\\phi_{1,6}", "\\phi_{1,3}'", "\\phi_{1,3}''",
      "\\phi_{2,1}", "\\phi_{2,2}" ]
    gap> CharNames(W,rec(spaltenstein:=true));
    [ "1", "eps", "epsl", "epsc", "theta'", "theta''" ]
    gap> CharNames(W,rec(spaltenstein:=true,TeX:=true));
    [ "1", "\\varepsilon", "\\varepsilon_l", "\\varepsilon_c",
      "\\theta'", "\\theta''" ]|

The  last two  commands show  the character  names used by Spaltenstein and
Lusztig when describing the Springer correspondence.

%%%%%%%%%%%%%%%%%%%%%%%%%%%%%%%%%%%%%%%%%%%%%%%%%%%%%%%%%%%%%%%%%%%%%%%%%
\Section{CharParams for reflection groups}
\index{CharParams for reflection groups}

'CharParams( <W> )'

this  function returns the list of parameters for irreducible characters of
<W>\:  partitions for type 'A', double partitions for type 'B', etc$\ldots$
as  described in the introduction. For exceptional groups they are pairs or
triples,  beginning with the  dimension, the valuation  of the fake degree,
and  an ordinal  number if  more than  one character  shares the  first two
invariants.

|    gap> CharParams(CoxeterGroup("G",2));
    [ [ [ 1, 0 ] ], [ [ 1, 6 ] ], [ [ 1, 3, 1 ] ], [ [ 1, 3, 2 ] ],
      [ [ 2, 1 ] ], [ [ 2, 2 ] ] ]|

%%%%%%%%%%%%%%%%%%%%%%%%%%%%%%%%%%%%%%%%%%%%%%%%%%%%%%%%%%%%%%%%%%%%%%%%%
\Section{ChevieCharInfo}
\index{ChevieCharInfo}

'ChevieCharInfo( <W> )'

returns  information  about the  irreducible  characters  of the  finite
reflection group <W>.  The result is a record with the following
components\:\

'charparams':\\ contains parameters for the irreducible characters as
     described  in the  introduction or  returned by 'CharParams(<W>)'. The
     parameters   are  tuples  with  one  component  for  each  irreducible
     irreducible  component of <W>  (as given by  'ReflectionType'). For an
     irreducible  component which  is an  imprimitive reflection  group the
     component  of  the  'charparam'  is  a  tuple of partitions, and for a
     primitive  irreducible group  it is  a pair  $(d,e)$ where  $d$ is the
     degree of the character and $e$ is the smallest symmetric power of the
     character  of the  reflection representation  which contains the given
     character as a component.

'charnames':\\ strings describing the irreducible characters, computed from
     the 'charparams'. This is the same as 'CharNames(<W>)'.

'positionId':\\ the  position of the  trivial character  in the  character
     table  of <W>  (which is  also returned by the function 'PositionId').

'positionDet':\\  Contains the position of the determinant character in the
   character  table  of  <W>  (which  is  also  returned  by  the  function
   'PositionDet'). For Coxeter groups this is the sign character.

'extRefl':\\  Only  present  if  <W>  is  irreducible,  in  which  case the
     reflection  representation  of  <W>  and  all  its exterior powers are
     irreducible.  It then contains the position  of the exterior powers of
     the reflection representation in the character table.

'b':\\ contains  the    result   of    'LowestPowerFakeDegrees(<W>)'.

'B':\\ contains  the    result   of    'HighestPowerFakeDegrees(<W>)'.

'a':\\ Only filled for Spetsial groups. Contains the result of
   'LowestPowerGenericDegrees(<W>)'.

'A':\\ Only filled for Spetsial groups. Contains the result of
   'HighestPowerGenericDegrees(<W>)'.

'opdam':\\ Contains the permutation of the characters obtained by composing
   the  Opdam involution with complex  conjugation. This permutation has an
   interpretation  as a Galois action on the characters of 'H\:=Hecke(W,x)'
   where  'x\:=Indeterminate(Cyclotomics)'\: if 'H' splits by taking $v$ an
   $e$-th  root of  $x$, '.opdam'  records the  permutation effected by the
   Galois action 'v->E(e)\*v'.

|    gap> ChevieCharInfo(ComplexReflectionGroup(22));
    rec(
      extRefl := [ 1, 5, 2 ],
      charparams :=
       [ [ [ 1, 0 ] ], [ [ 1, 30 ] ], [ [ 2, 11 ] ], [ [ 2, 13 ] ],
          [ [ 2, 1 ] ], [ [ 2, 7 ] ], [ [ 3, 2 ] ], [ [ 3, 6 ] ],
          [ [ 3, 12 ] ], [ [ 3, 16 ] ], [ [ 4, 3 ] ], [ [ 4, 6 ] ],
          [ [ 4, 9 ] ], [ [ 4, 8 ] ], [ [ 5, 4 ] ], [ [ 5, 10 ] ],
          [ [ 6, 7 ] ], [ [ 6, 5 ] ] ],
      opdam := ( 3, 5)( 4, 6)(11,13)(12,14)(17,18),
      b := [ 0, 30, 11, 13, 1, 7, 2, 6, 12, 16, 3, 6, 9, 8, 4, 10, 7, 5 ],
      charnames := [ "phi{1,0}", "phi{1,30}", "phi{2,11}", "phi{2,13}",
          "phi{2,1}", "phi{2,7}", "phi{3,2}", "phi{3,6}", "phi{3,12}",
          "phi{3,16}", "phi{4,3}", "phi{4,6}", "phi{4,9}", "phi{4,8}",
          "phi{5,4}", "phi{5,10}", "phi{6,7}", "phi{6,5}" ],
      positionId := 1,
      positionDet := 2,
      B := [ 0, 30, 19, 17, 29, 23, 18, 14, 28, 24, 27, 22, 21, 24, 20,
          26, 23, 25 ] )
    gap>  ChevieCharInfo( CoxeterGroup( "G", 2 ) );
    rec(
      charparams :=
       [ [ [ 1, 0 ] ], [ [ 1, 6 ] ], [ [ 1, 3, 1 ] ], [ [ 1, 3, 2 ] ],
          [ [ 2, 1 ] ], [ [ 2, 2 ] ] ],
      extRefl := [ 1, 5, 2 ],
      a := [ 0, 6, 1, 1, 1, 1 ],
      A := [ 0, 6, 5, 5, 5, 5 ],
      b := [ 0, 6, 3, 3, 1, 2 ],
      spaltenstein :=
       [ "1", "\\varepsilon", "\\varepsilon_l", "\\varepsilon_c",
          "\\theta'", "\\theta''" ],
      positionId := 1,
      positionDet := 2,
      B := [ 0, 6, 3, 3, 5, 4 ],
      charnames := [ "phi{1,0}", "phi{1,6}", "phi{1,3}'", "phi{1,3}''",
          "phi{2,1}", "phi{2,2}" ] )|

For  irreducible groups, the returned  record contains sometimes additional
information\:

for  $F_4$:\\ the field 'kondo' gives  the labeling of the characters given
  by Kondo, also used in \cite[(4.10)]{Lus85}.

for  $E_6,  E_7,  E_8$:\\  the  field  'frame'  gives  the  labeling of the
  characters  given  by  Frame,  also  used  in  \cite[(4.11),  (4.12), and
  (4.13)]{Lus85}.

for  $G_2$:\\ the field 'spaltenstein' gives the labeling of the characters
  given by Spaltenstein.

for  $G(de,e,2)$  even  $e$  and  $d>1$:\\  the  field  'malle'  gives  the
  parameters for the characters used by Malle in \cite{Mal96}.

%%%%%%%%%%%%%%%%%%%%%%%%%%%%%%%%%%%%%%%%%%%%%%%%%%%%%%%%%%%%%%%%%%%%%%%%%
\Section{FakeDegrees}
\index{FakeDegrees}

'FakeDegrees( <W>, <q> )'

returns  a list holding the fake degrees of the reflection group $W$ on the
vector  space $V$, evaluated at <q>. These are the graded multiplicities of
the  irreducible characters of $W$ in the quotient $SV/I$ where $SV$ is the
symmetric  algebra of $V$ and $I$ is the ideal generated by the homogeneous
invariants  of  positive  degree  in  $SV$.  The  ordering  of  the  result
corresponds to the ordering of the characters in 'CharTable(W)'.

|    gap> q := X( Rationals );; q.name := "q";;
    gap> FakeDegrees( CoxeterGroup( "A", 2 ), q );
    [ q^3, q^2 + q, q^0 ]|

%%%%%%%%%%%%%%%%%%%%%%%%%%%%%%%%%%%%%%%%%%%%%%%%%%%%%%%%%%%%%%%%%%%%%%%%%
\Section{FakeDegree}
\index{FakeDegree}

'FakeDegree( <W>, <phi>, <q> )'

returns  the  fake degree  of  the  character  of parameter  <phi>  (see
"CharParams")  of  the  reflection  group $W$,  evaluated  at  <q>  (see
"FakeDegrees" for a definition of the fake degrees).

|    gap> q := X( Rationals );; q.name := "q";;
    gap> FakeDegree( CoxeterGroup( "A", 2 ), [ [ 2, 1 ] ], q );
    q^2 + q|

%%%%%%%%%%%%%%%%%%%%%%%%%%%%%%%%%%%%%%%%%%%%%%%%%%%%%%%%%%%%%%%%%%%%%%%%%
\Section{LowestPowerFakeDegrees}
\index{LowestPowerFakeDegrees}

'LowestPowerFakeDegrees( <W> )'

return a list holding the $b$-function for all irreducible characters of
$W$,  that is,  for each  character $\chi$,  the valuation  of the  fake
degree of $\chi$. The ordering of the result corresponds to the ordering
of  the characters  in 'CharTable(W)'.  The advantage  of this  function
compared to calling  'FakeDegrees' is that one does not  have to provide
an indeterminate,  and that it  may be much  faster to compute  than the
fake degrees.

|    gap> LowestPowerFakeDegrees( CoxeterGroup( "D", 4 ) );
    [ 6, 6, 7, 12, 4, 3, 6, 2, 2, 4, 1, 2, 0 ]|

%%%%%%%%%%%%%%%%%%%%%%%%%%%%%%%%%%%%%%%%%%%%%%%%%%%%%%%%%%%%%%%%%%%%%%%%%
\Section{HighestPowerFakeDegrees}
\index{HighestPowerFakeDegrees}

'HighestPowerFakeDegrees( <W> )'

returns a list  holding the $B$-function for  all irreducible characters
of  $W$, that  is, for  each character  $\chi$, the  degree of  the fake
degree of $\chi$. The ordering of the result corresponds to the ordering
of  the characters  in 'CharTable(W)'.  The advantage  of this  function
compared to calling  'FakeDegrees' is that one does not  have to provide
an indeterminate,  and that it  may be much  faster to compute  than the
fake degrees.

|    gap> HighestPowerFakeDegrees( CoxeterGroup( "D", 4 ) );
    [ 10, 10, 11, 12, 8, 9, 10, 6, 6, 8, 5, 6, 0 ]|

%%%%%%%%%%%%%%%%%%%%%%%%%%%%%%%%%%%%%%%%%%%%%%%%%%%%%%%%%%%%%%%%%%%%%%%%%
\Section{Representations}
\index{Representations}

'Representations( <W>[, <l>])'

returns  a  list  holding,  for  each  irreducible character of the complex
reflection  group  <W>,  a  list  of  matrices  images  of  the  generating
reflections  of <W>  in a  model of  the corresponding representation. This
function  is based on the classification,  and is not yet fully implemented
for  $G_{34}$;  88  representations  are  missing  out  of  169,  that is 4
representations of dim. 105, 3 of dim. 315, 6 of dim. 420, 4 of dim.840 and
those  of dim. 120, 140, 189, 280, 384,  504, 540, 560, 630, 720, 729, 756,
896, 945, 1260 and 1280.

If  there is a  second argument, it  can be a  list of indices (or a single
integer) and only the representations with these indices (or that index) in
the list of all representations are returned.

|    gap> Representations(CoxeterGroup("B",2));
    [ [ [ [ 1 ] ], [ [ -1 ] ] ],
      [ [ [ 1, 0 ], [ -1, -1 ] ], [ [ 1, 2 ], [ 0, -1 ] ] ],
      [ [ [ -1 ] ], [ [ -1 ] ] ], [ [ [ 1 ] ], [ [ 1 ] ] ],
      [ [ [ -1 ] ], [ [ 1 ] ] ] ]
    gap> Representations(ComplexReflectionGroup(4),7);
    [ [ [ E(3)^2, 0, 0 ], [ 2*E(3)^2, E(3), 0 ], [ E(3), 1, 1 ] ],
      [ [ 1, -1, E(3) ], [ 0, E(3), -2*E(3)^2 ], [ 0, 0, E(3)^2 ] ] ]|

%%%%%%%%%%%%%%%%%%%%%%%%%%%%%%%%%%%%%%%%%%%%%%%%%%%%%%%%%%%%%%%%%%%%%%%%%
\Section{LowestPowerGenericDegrees}
\index{LowestPowerGenericDegrees}

'LowestPowerGenericDegrees( <W> )'

returns  a list holding the $a$-function  for all irreducible characters of
the  Coxeter  group  or  Spetsial  reflection  group <W>, that is, for each
character  $\chi$, the  valuation of  the generic  degree of $\chi$ (in the
one-parameter  Hecke  algebra  'Hecke(W,X(Cyclotomics))'  corresponding  to
<W>).  The  ordering  of  the  result  corresponds  to  the ordering of the
characters in 'CharTable(W)'.

|    gap> LowestPowerGenericDegrees( CoxeterGroup( "D", 4 ) );
    [ 6, 6, 7, 12, 3, 3, 6, 2, 2, 3, 1, 2, 0 ]|

%%%%%%%%%%%%%%%%%%%%%%%%%%%%%%%%%%%%%%%%%%%%%%%%%%%%%%%%%%%%%%%%%%%%%%%%%
\Section{HighestPowerGenericDegrees}
\index{HighestPowerGenericDegrees}

'HighestPowerGenericDegrees( <W> )'

returns  a list holding the $A$-function  for all irreducible characters of
the  Coxeter  group  or  Spetsial  reflection  group <W>, that is, for each
character  $\chi$,  the  degree  of  the  generic  degree of $\chi$ (in the
one-parameter  Hecke  algebra  'Hecke(W,X(Cyclotomics))'  corresponding  to
<W>).  The  ordering  of  the  result  corresponds  to  the ordering of the
characters in 'CharTable(W)'.

|    gap> HighestPowerGenericDegrees( CoxeterGroup( "D", 4 ) );
    [ 10, 10, 11, 12, 9, 9, 10, 6, 6, 9, 5, 6, 0 ]|

%%%%%%%%%%%%%%%%%%%%%%%%%%%%%%%%%%%%%%%%%%%%%%%%%%%%%%%%%%%%%%%%%%%%%%%%%
\Section{PositionDet}
\index{PositionDet}

'PositionDet( <W> )'

return  the position of the determinant character in the character table of
the group <W> (for Coxeter groups this is the sign character).

|    gap> W := CoxeterGroup( "D", 4 );;
    gap> PositionDet( W );
    4|

See also 'ChevieCharInfo' ("ChevieCharInfo").

%%%%%%%%%%%%%%%%%%%%%%%%%%%%%%%%%%%%%%%%%%%%%%%%%%%%%%%%%%%%%%%%%%%%%%%%%
\Section{DetPerm}
\index{DetPerm}

'DetPerm( <W> )'

return  the permutation of the characters of the reflection group <W> which
is effected when tensoring by the determinant character (for Coxeter groups
this is the sign character).

|    gap> W := CoxeterGroup( "D", 4 );;
    gap> DetPerm( W );
    [ 8, 9, 11, 13, 5, 6, 12, 1, 2, 10, 3, 7, 4 ]|

%%%%%%%%%%%%%%%%%%%%%%%%%%%%%%%%%%%%%%%%%%%%%%%%%%%%%%%%%%%%%%%%%%%%%%%%%
