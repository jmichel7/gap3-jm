%%%%%%%%%%%%%%%%%%%%%%%%%%%%%%%%%%%%%%%%%%%%%%%%%%%%%%%%%%%%%%%%%%%%%%%%%%%%%
%A  chvunir.tex       CHEVIE documentation      Olivier Dudas and Jean Michel
%%
%Y  Copyright (C) 2010 - 2015  University Paris VII
%%
%%  This  file  contains  the  description  of  the  GAP functions of CHEVIE
%%  dealing with unipotent radicals of Borel subgroups.
%%%%%%%%%%%%%%%%%%%%%%%%%%%%%%%%%%%%%%%%%%%%%%%%%%%%%%%%%%%%%%%%%%%%%%%%%%%%%
\Chapter{Unipotent elements of reductive groups}

This  chapter  describes  functions  allowing  to  make computations in the
unipotent  radical of a  Borel subgroup of  a connected algebraic reductive
group;  the  implementation  of  these  functions  was initially written by
Olivier Dudas.

The  unipotent radical of a  Borel subgroup is the  product in any order of
root  subgroups associated  to the  positive roots.  We fix an order, which
gives a canonical form to display elements and to compare them.

The  computations use the Steinberg relations between root subgroups, which
come from the choice of a Chevalley basis of the Lie algebra. The reference
we  follow is chapters 4 to 6  of the book \cite{Car72b} ``Simple groups of
Lie type\'\' by R.W. Carter (Wiley 1972).

We start with a root datum specified by a \CHEVIE\ Coxeter group record 'W'
and  build a record which contains  information about the maximal unipotent
subgroup  of  the  corresponding  reductive  group,  that  is the unipotent
radical of the Borel subgroup determined by the positive roots.

|    gap> W:=CoxeterGroup("E",6);; U:=UnipotentGroup(W);
    UnipotentGroup(CoxeterGroup("E",6))|

Now, if $\alpha=$'W.roots[2]', we make the element $u_\alpha(4)$
of the root subgroup $u_\alpha$\:

|    gap> U.Element(2,4);
    u2(4)|

If we do not specify the coefficient we make by default $u_\alpha(1)$, so we
have also\:

|    gap> U.Element(2)^4;
    u2(4)|

We can make more complicated elements\:

|    gap> U.Element(2,4)*U.Element(4,5);
    u2(4) * u4(5)
    gap> U.Element(2,4,4,5);
    u2(4) * u4(5)|

If the roots are not in order the element is normalized\:

|    gap> u:=U.Element(4,5,2,4);
    u2(4) * u4(5) * u8(-20)|

It is possible to display the decomposition of the roots in simple roots
instead of their index\:

|    gap> Display(u,rec(root:=true));
    u010000(4) * u000100(5) * u010100(-20)|

The coefficients in the root subgroups can be elements of arbitrary rings.
Here is an example using |Mvp|s (see "Mvp")\:

|    gap>  W:=CoxeterGroup("E",8);; U:=UnipotentGroup(W);
    UnipotentGroup(CoxeterGroup("E",8))
    gap> u:=U.Element(List([1..8],i->[i,Z(2)*Mvp(SPrint("x",i))]));
    u1(Z(2)^0x1) * u2(Z(2)^0x2) * u3(Z(2)^0x3) * u4(Z(2)^0x4) * u5(Z(2)^0x\
    5) * u6(Z(2)^0x6) * u7(Z(2)^0x7) * u8(Z(2)^0x8)
    gap> Display(u^16,rec(root:=true));
    u22343210(Z(2)^0x1^2x2^2x3^3x4^4x5^3x6^2x7) * 
    u12343211(Z(2)^0x1x2^2x3^3x4^4x5^3x6^2x7x8) * 
    u12243221(Z(2)^0x1x2^2x3^2x4^4x5^3x6^2x7^2x8) * 
    u12233321(Z(2)^0x1x2^2x3^2x4^3x5^3x6^3x7^2x8) * 
    u22343211(Z(2)^0x1^2x2^2x3^3x4^4x5^3x6^2x7x8) * 
    u12243321(Z(2)^0x1x2^2x3^2x4^4x5^3x6^3x7^2x8) * 
    u12244321(Z(2)^0x1x2^2x3^2x4^4x5^4x6^3x7^2x8) * 
    u22343321(Z(2)^0x1^2x2^2x3^3x4^4x5^3x6^3x7^2x8) * 
    u12344321(Z(2)^0x1x2^2x3^3x4^4x5^4x6^3x7^2x8) * 
    u22344321(Z(2)^0x1^2x2^2x3^3x4^4x5^4x6^3x7^2x8) * 
    u23354321(Z(2)^0x1^2x2^3x3^3x4^5x5^4x6^3x7^2x8) * 
    u22454321(Z(2)^0x1^2x2^2x3^4x4^5x5^4x6^3x7^2x8) * 
    u23465432(Z(2)^0x1^2x2^3x3^4x4^6x5^5x6^4x7^3x8^2)
    gap> u^32;
    ()|

The  above computation shows  that in characteristic  2 the exponent of the
unipotent group of $E_8$ is 32. More precisely, squaring doubles the height
of  the involved  roots, so  in the  above $u^{16}$  involves only roots of
height 16 or more.

Various  actions are  defined on  unipotent elements.  Elements of the Weyl
group  act (through certain representatives) as long as no root subgroup is
in their inversion set\:

|    gap> W:=CoxeterGroup("G",2);
    CoxeterGroup("G",2)
    gap> U:=UnipotentGroup(W);
    UnipotentGroup(CoxeterGroup("G",2))
    gap> u:=U.Element(1,Mvp("x"),3,Mvp("y"));
    u1(x) * u3(y)
    gap> u^(W.2*W.1);
    u4(y) * u5(x)
    gap> u^W.1;      
    Error, u1(x) * u3(y) should have no coefficient on root 1
     in
    <rec1> ^ <rec2> called from
    main loop
    brk>|

Semisimple elements act by conjugation\:

|    gap> s:=SemisimpleElement(W,[E(3),2]);
    <E(3),2>
    gap> u^s;
    u1(E3x) * u3(2E3y)|

As well as unipotent elements:

|    gap> u^U.Element(2);
    u1(x) * u3(x+y) * u4(-x-2y) * u5(x+3y) * u6(3xy+x^2+3y^2)|
 
%%%%%%%%%%%%%%%%%%%%%%%%%%%%%%%%%%%%%%%%%%%%%%%%%%%%%%%%%%%%%%%%%%%%%%%%%
\Section{UnipotentGroup}
\index{UnipotentGroup}

'UnipotentGroup(<W>)'

<W>  should  be  a  Coxeter  group  record  representing a Weyl group. This
function  returns a  record representing  the unipotent  radical $\bU$ of a
Borel subgroup of the reductive group of Weyl group <W>.

The result is a record with the following fields\:

'weylGroup':\\ contains <W>.

'specialPairs':\\  Let $\<$ be the order on  the roots of <W> resulting from
     some total order on the ambient vector space (\CHEVIE\ chooses such an
     order  once  and  for  all  and  it  has  nothing to do with the field
     '.order'  of the unipotent  group record). A  pair $(r,s)$ of roots is
     *special*  if $r\<s$  and $r+s$  is a  root. The field '.specialPairs'
     contains twice the list of triples $(r,s,r+s)$ for special pairs\:\ it
     contains first this list, sorted by $(r+s,r)$, then it contains a copy
     of  the  list  in  the  order  $(s,r,r+s)$. Roots in these triples are
     represented  as  their  index  in  'Parent(W).roots'.  Thanks to the
     repetition,  each ordered pair  of positive roots  whose sum is a root
     appears exactly once in '.specialPairs'.

'chevalleyConstants':\\  The Lie algebra of $\bU$ has a *Chevalley basis*
     $e_r$  indexed  by  roots,  with  the property that $[e_r,e_s]=N_{r,s}
     e_{r+s}$  for some integer constants $N_{r,s}$  for each pair of roots
     whose  sum is a root. The list 'chevalleyConstants', of same length as
     '.specialPairs', contains the corresponding integers $N_{r,s}$.

'commutatorConstants':\\  These are the constants $C_{r,s,i,j}$ which occur
     in the commutator formula for two root subgroups\:
     $$u_s(u)u_r(t)=u_r(t)u_s(u)\prod_{i,j>0}
     u_{ir+js}(C_{r,s,i,j}(-t)^iu^j),$$  where the product  is over all the
     roots  of the given  shape. The list  '.commutatorConstants' is of the
     same  length as  '.specialPairs' and  contains for  each pair of roots
     $(r,s)$   a  list  of  quadruples  $[i,j,ir+js,C_{r,s,i,j}]$  for  all
     possible values of $i,j$ for this pair.

'order':\\  An order on the roots, used to give a canonical form to unipotent
     elements  by listing the root subgroups in that order. '.order' is the
     list of indices of roots in 'Parent(W)', listed in the desired order.

|    gap> W:=CoxeterGroup("G",2);     
    CoxeterGroup("G",2)
    gap> U:=UnipotentGroup(W);
    UnipotentGroup(CoxeterGroup("G",2))
    gap> U.specialPairs;
    [ [ 1, 2, 3 ], [ 2, 3, 4 ], [ 2, 4, 5 ], [ 1, 5, 6 ], [ 3, 4, 6 ], 
      [ 2, 1, 3 ], [ 3, 2, 4 ], [ 4, 2, 5 ], [ 5, 1, 6 ], [ 4, 3, 6 ] ]
    gap> U.chevalleyConstants;
    [ 1, 2, 3, 1, 3, -1, -2, -3, -1, -3 ]
    gap> U.commutatorConstants;
    [ [ [ 1, 1, 3, 1 ], [ 1, 2, 4, -1 ], [ 1, 3, 5, 1 ], [ 2, 3, 6, 2 ] ], 
      [ [ 1, 1, 4, 2 ], [ 2, 1, 5, 3 ], [ 1, 2, 6, -3 ] ], 
      [ [ 1, 1, 5, 3 ] ], [ [ 1, 1, 6, 1 ] ], [ [ 1, 1, 6, 3 ] ], 
      [ [ 1, 1, 3, -1 ], [ 2, 1, 4, -1 ], [ 3, 1, 5, -1 ], 
          [ 3, 2, 6, -1 ] ], 
      [ [ 1, 1, 4, -2 ], [ 2, 1, 6, -3 ], [ 1, 2, 5, 3 ] ], 
      [ [ 1, 1, 5, -3 ] ], [ [ 1, 1, 6, -1 ] ], [ [ 1, 1, 6, -3 ] ] ]|
   
A unipotent group record also contains functions for creating and normalizing
unipotent elements.

'U.Element(<r>)'

'U.Element($r_1$,$c_1$,..,$r_n$,$c_n$)'

In the first form the function creates the element $u_r(1)$, and in the second
form the element $u_{r_1}(c_1)\ldots u_{r_n}(c_n)$

|    gap> U.Element(2);
    u2(1)
    gap> U.Element(1,2,2,4);
    u1(2) * u2(4)
    gap> U.Element(2,4,1,2);
    u1(2) * u2(4) * u3(-8) * u4(32) * u5(-128) * u6(512)|

'U.CanonicalForm(<l>[,order])'

The  function  takes  a  list  of  pairs  '[r,c]'  representing a unipotent
element,  where 'r'  is a  root and  'c' the corresponding coefficient, and
puts  it in  canonical form,  reordering the  terms to  agree with 'U.order'
using  the commutation  relations. If  a second  argument is given, this is
used instead of 'U.order'.

|    gap> U.CanonicalForm([[2,4],[1,2]]);
    [ [ 1, 2 ], [ 2, 4 ], [ 3, -8 ], [ 4, 32 ], [ 5, -128 ], [ 6, 512 ] ]
    gap> U.CanonicalForm(last,[6,5..1]);
    [ [ 2, 4 ], [ 1, 2 ] ]|

%%%%%%%%%%%%%%%%%%%%%%%%%%%%%%%%%%%%%%%%%%%%%%%%%%%%%%%%%%%%%%%%%%%%%%%%%
\Section{Operations for Unipotent elements}

The  arithmetic operations '\*', '/' and  '\^' work for unipotent elements.
They also have 'Print' and 'String' methods.

|    gap> u:=U.Element(1,4,3,-6);
    u1(4) * u3(-6)
    gap> u^-1;
    u1(-4) * u3(6)
    gap> u:=U.Element(1,4,2,-6);
    u1(4) * u2(-6)
    gap> u^-1;                  
    u1(-4) * u2(6) * u3(24) * u4(-144) * u5(864) * u6(6912)
    gap> u^0;
    ()
    gap> u*u;
    u1(8) * u2(-12) * u3(24) * u4(432) * u5(6048) * u6(-17280)
    gap> String(u);
    "u1(4) * u2(-6)"
    gap> Format(u^2,rec(root:=true));
    "u10(8) * u01(-12) * u11(24) * u12(432) * u13(6048) * u23(-17280)"|

'u\^n'  gives  the  'n'-th  power  of  'u'  when  'n'  is an integer and 'u'
conjugate  by 'n' when 'n' is a  unipotent element, a semisimple element or
an element of the Weyl group.

%%%%%%%%%%%%%%%%%%%%%%%%%%%%%%%%%%%%%%%%%%%%%%%%%%%%%%%%%%%%%%%%%%%%%%%%%
\Section{IsUnipotentElement}
\index{IsUnipotentElement}

'IsUnipotentElement(<u>)'

This  function returns  'true' if  'u' is  a unipotent  element and 'false'
otherwise.

|    gap> IsUnipotentElement(U.Element(2));
    true
    gap> IsUnipotentElement(2);           
    false|

%%%%%%%%%%%%%%%%%%%%%%%%%%%%%%%%%%%%%%%%%%%%%%%%%%%%%%%%%%%%%%%%%%%%%%%%%
\Section{UnipotentDecompose}
\index{UnipotentDecompose}

'UnipotentDecompose(<w>,<u>)'

'u'  should be a unipotent element and  'w' an element of the corresponding
Weyl  group.  If  $\bU$  is  the  unipotent  radical  of the Borel subgroup
determined  by the positive roots, and $\bU^-$ the unipotent radical of the
opposite  Borel,  this  function  decomposes  $u$  into  its  component  in
$\bU\cap{}^w\bU^-$ and its component in $\bU\cap{}^w\bU$.

|    gap> u:=U.Element(2,Mvp("y"),1,Mvp("x"));               
    u1(x) * u2(y) * u3(-xy) * u4(xy^2) * u5(-xy^3) * u6(2x^2y^3)
    gap> UnipotentDecompose(W.1,u);
    [ u1(x), u2(y) * u3(-xy) * u4(xy^2) * u5(-xy^3) * u6(2x^2y^3) ]
    gap> UnipotentDecompose(W.2,u);
    [ u2(y), u1(x) ]|

%%%%%%%%%%%%%%%%%%%%%%%%%%%%%%%%%%%%%%%%%%%%%%%%%%%%%%%%%%%%%%%%%%%%%%%%%
\Section{UnipotentAbelianPart}
\index{UnipotentAbelianPart}

'UnipotentAbelianPart(<u>)'

If  $\bU$ is the unipotent subgroup and $D(\bU)$ its derived subgroup, this
function   returns  the  projection   of  the  unipotent   element  'u'  on
$\bU/D(\bU)$, that is its coefficients on the simple roots.

|    gap> u:=U.Element(2,Mvp("y"),1,Mvp("x"));
    u1(x) * u2(y) * u3(-xy) * u4(xy^2) * u5(-xy^3) * u6(2x^2y^3)
    gap> UnipotentAbelianPart(u);
    u1(x) * u2(y)|

%%%%%%%%%%%%%%%%%%%%%%%%%%%%%%%%%%%%%%%%%%%%%%%%%%%%%%%%%%%%%%%%%%%%%%%%%
