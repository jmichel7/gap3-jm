%%%%%%%%%%%%%%%%%%%%%%%%%%%%%%%%%%%%%%%%%%%%%%%%%%%%%%%%%%%%%%%%%%%%%%%%%%%%%
%%
%A  boolean.tex                 GAP documentation            Martin Schoenert
%%
%A  @(#)$Id: boolean.tex,v 1.1.1.1 1996/12/11 12:36:43 werner Exp $
%%
%Y  Copyright 1990-1992,  Lehrstuhl D fuer Mathematik,  RWTH Aachen,  Germany
%%
%%  This file describes the boolean datatype and its operations.
%%
%H  $Log: boolean.tex,v $
%H  Revision 1.1.1.1  1996/12/11 12:36:43  werner
%H  Preparing 3.4.4 for release
%H
%H  Revision 3.5  1993/03/11  17:53:56  fceller
%H  strings are now lists
%H
%H  Revision 3.4  1993/02/19  10:48:42  gap
%H  adjustments in line length and spelling
%H
%H  Revision 3.3  1991/12/27  16:07:04  martin
%H  revised everything for GAP 3.1 manual
%H
%H  Revision 3.2  1991/07/26  12:34:01  martin
%H  improved the index
%H
%H  Revision 3.1  1991/07/25  16:16:59  martin
%H  fixed some minor typos
%H
%H  Revision 3.0  1991/04/11  11:10:48  martin
%H  Initial revision under RCS.
%H
%%
\Chapter{Booleans}%
\index{type!boolean}\index{logical}

The two *boolean*   values are 'true' and  'false'.   They stand for  the
*logical* values  of the same name.  They  appear mainly as values of the
conditions in 'if'-statements and 'while'-loops.

This  chapter contains  sections   describing  the operations  that   are
available   for  the  boolean  values   (see   "Comparisons of Booleans",
"Operations for Booleans").

Further this chapter contains a section  about the function 'IsBool' (see
"IsBool").  Note that it is a convention that the name of a function that
tests a   property, and  returns  'true' and   'false'  according to  the
outcome, starts with 'Is', as in 'IsBool'.

%%%%%%%%%%%%%%%%%%%%%%%%%%%%%%%%%%%%%%%%%%%%%%%%%%%%%%%%%%%%%%%%%%%%%%%%%
\Section{Comparisons of Booleans}%
\index{comparisons!of booleans}

'<bool1> = <bool2>', '<bool1> \<> <bool2>'

The equality  operator '=' evaluates to  'true' if the two boolean values
<bool1> and <bool2> are equal, i.e., both are 'true' or both are 'false',
and 'false' otherwise.  The inequality operator '\<>' evaluates to 'true'
if the two boolean  values <bool1> and  <bool2> are different and 'false'
otherwise.  This operation is also called the *exclusive or*, because its
value is 'true' if exactly one of <bool1> or <bool2> is 'true'.

You  can compare boolean  values with objects  of other types.  Of course
they are never equal.

|    gap> true = false;
    false
    gap> false = (true = false);
    true
    gap> true <> 17;
    true |

'<bool1> \<\ <bool2>', '<bool1> \<= <bool2>',\\
'<bool1>  > <bool2>', '<bool1>  >= <bool2>'

The operators  '\<', '\<=',  '>', and   '=>'  evaluate to 'true'  if  the
boolean value <bool1> is less than, less than  or equal to, greater than,
and greater than or equal to the  boolean value <bool2>.  The ordering of
boolean values is defined by 'true \<\ false'.

You can compare  boolean values  with objects of other  types.  Integers,
rationals, cyclotomics, permutations, and  words are smaller than boolean
values.  Objects of the other types, i.e., functions, lists,  and records
are larger.

|    gap> true < false;
    true
    gap> false >= true;
    true
    gap> 17 < true;
    true
    gap> true < [17];
    true |

%%%%%%%%%%%%%%%%%%%%%%%%%%%%%%%%%%%%%%%%%%%%%%%%%%%%%%%%%%%%%%%%%%%%%%%%%%%%%
\Section{Operations for Booleans}%
\index{operations!for booleans}\index{logical operations}%
\index{or}\index{and}\index{not}

'<bool1> or <bool2>'

The logical operator 'or' evaluates to 'true' if at  least one of the two
boolean operands <bool1> and <bool2> is  'true' and to 'false' otherwise.

'or' first evaluates <bool1>.  If the value is neither 'true' nor 'false'
an error is signalled.  If the value is 'true', then  'or' returns 'true'
*without*  evaluating  <bool2>.   If  the value  is  'false',  then  'or'
evaluates <bool2>.  Again, if the value is neither 'true'  nor 'false' an
error is signalled.  Otherwise 'or' returns  the value  of <bool2>.  This
*short-circuited* evaluation  is important if   the value of  <bool1>  is
'true' and evaluation of <bool2> would take much time or cause an error.

'or' is associative, i.e., it is allowed to write '<b1> or <b2> or <b3>',
which is interpreted as  '(<b1> or <b2>) or <b3>'.   'or' has the  lowest
precedence  of the logical  operators.   All logical operators have lower
precedence than the comparison operators '=', '\<', 'in', etc.

|    gap> true or false;
    true
    gap> false or false;
    false
    gap> i := -1;;  l := [1,2,3];;
    gap> if i <= 0 or l[i] = false  then Print("aha\n");  fi;
    aha    # no error, because 'l[i]' is not evaluated |

'<bool1> and <bool2>'

The  logical operator 'and' evaluates  to 'true' if both boolean operands
<bool1> and <bool2> are 'true' and to 'false' otherwise.

'and'  first evaluates  <bool1>.    If the value   is neither 'true'  nor
'false' an  error is signalled.   If the  value  is 'false',  then  'and'
returns 'false'  *without* evaluating <bool2>.  If  the value is  'true',
then 'and' evaluates <bool2>.  Again, if the value  is neither 'true' nor
'false' an  error is signalled.   Otherwise   'and' returns the  value of
<bool2>.  This *short-circuited* evaluation is important if the  value of
<bool1> is 'false'  and evaluation  of   <bool2> would take much time  or
cause an error.

'and' is  associative, i.e., it is  allowed  to write '<b1>  and <b2> and
<b3>',  which is interpreted  as '(<b1> and <b2>)  and  <b3>'.  'and' has
higher precedence than  the  logical 'or' operator,  but lower  than  the
unary  logical 'not'  operator.    All   logical operators  have    lower
precedence than the comparison operators '=', '\<', 'in', etc.

|    gap> true and false;
    false
    gap> true and true;
    true
    gap> false and 17;
    false    # is no error, because '17' is never looked at |

'not <bool>'

The logical operator 'not' returns 'true'  if the boolean value <bool> is
'false' and 'true' otherwise.  An error is  signalled if <bool>  does not
evaluate to 'true' or 'false'.

'not' has higher  precedence than the  other logical operators,  'or' and
'and'.  All logical operators  have lower precedence than  the comparison
operators '=', '\<', 'in', etc.

|    gap> not true;
    false
    gap> not false;
    true |

%%%%%%%%%%%%%%%%%%%%%%%%%%%%%%%%%%%%%%%%%%%%%%%%%%%%%%%%%%%%%%%%%%%%%%%%%
\Section{IsBool}%
\index{test!for a boolean}

'IsBool( <obj> )'

'IsBool' returns 'true' if <obj>, which may  be an object of an arbitrary
type, is a boolean value and 'false' otherwise.  'IsBool' will  signal an
error if <obj> is an unbound variable.

|    gap> IsBool( true );
    true
    gap> IsBool( false );
    true
    gap> IsBool( 17 );
    false |

%%%%%%%%%%%%%%%%%%%%%%%%%%%%%%%%%%%%%%%%%%%%%%%%%%%%%%%%%%%%%%%%%%%%%%%%%
%%
%E  Emacs . . . . . . . . . . . . . . . . . . . . . local Emacs variables
%%
%%  Local Variables:
%%  mode:               outline
%%  outline-regexp:     "\\\\Chapter\\|\\\\Section"
%%  fill-column:        73
%%  eval:               (hide-body)
%%  End:
%%



