%%%%%%%%%%%%%%%%%%%%%%%%%%%%%%%%%%%%%%%%%%%%%%%%%%%%%%%%%%%%%%%%%%%%%%%%%%%%%
%%
%A  set.tex                     GAP documentation            Martin Schoenert
%%
%A  @(#)$Id: set.tex,v 1.1.1.1 1996/12/11 12:36:50 werner Exp $
%%
%Y  Copyright 1990-1992,  Lehrstuhl D fuer Mathematik,  RWTH Aachen,  Germany
%%
%%  This file describes those functions that mainly deal  with  proper  sets.
%%
%H  $Log: set.tex,v $
%H  Revision 1.1.1.1  1996/12/11 12:36:50  werner
%H  Preparing 3.4.4 for release
%H
%H  Revision 3.7  1994/05/30  14:52:41  vfelsch
%H  index entries rearranged
%H
%H  Revision 3.6  1993/02/19  10:48:42  gap
%H  adjustments in line length and spelling
%H
%H  Revision 3.5  1993/02/04  16:03:43  felsch
%H  examples fixed
%H
%H  Revision 3.4  1992/04/02  21:06:23  martin
%H  changed *domain functions* to *set theoretic functions*
%H
%H  Revision 3.3  1991/12/27  16:07:04  martin
%H  revised everything for GAP 3.1 manual
%H
%H  Revision 3.2  1991/07/26  09:01:07  martin
%H  changed |\GAP\ | to |{\GAP}|
%H
%H  Revision 3.1  1991/07/25  16:16:59  martin
%H  fixed some minor typos
%H
%H  Revision 3.0  1991/04/11  11:33:34  martin
%H  Initial revision under RCS
%H
%%
\Chapter{Sets}%
\index{multisets}

A  very important mathematical concept,  maybe the most important of all,
are sets.  Mathematically  a *set* is  an abstract object such that  each
object  is either an element  of the  set or it  is not.   So  a set is a
collection like  a list, and in fact {\GAP} uses lists to represent sets.
Note that this of course implies that {\GAP} only deals with finite sets.

Unlike a list a set must not contain an element several times.  It simply
makes no sense to say   that an object is   twice  an element of a   set,
because an object is either an element of a set, or it is not.  Therefore
the list that is used  to represent a set has  no duplicates, that is, no
two elements of such a list are equal.

Also unlike a  list a set does not  impose any ordering  on the elements.
Again it  simply makes no  sense to say  that an object  is  the first or
second etc.  element of  a set, because,  again, an  object is  either an
element of a set, or it is not.  Since ordering is  not defined for a set
we can put the elements in any order into the  list used to represent the
set.  We put the elements sorted into the  list, because this ordering is
very practical.  For example if we convert a  list into a  set we have to
remove  duplicates, which is  very  easy to do  after  we have sorted the
list, since then equal elements will be next to each other.

In   short   sets are represented  by   sorted  lists without  holes  and
duplicates in {\GAP}.  Such  a list is  in this document called a  proper
set.  Note that we guarantee this representation, so  you may make use of
the fact that a set is represented by a sorted list in your functions.

In some contexts (for example see "Combinatorics"),  we also want to talk
about multisets.  A *multiset* is like a set,  except that an element may
appear several  times  in a multiset.   Such multisets are represented by
sorted lists with holes that may have duplicates.

The first section in this chapter  describes the  functions to test if an
object is a set and to convert objects to sets (see "IsSet" and "Set").

The next section describes the function that tests if two sets  are equal
(see "IsEqualSet").

The next sections  describe  the destructive  functions that  compute the
standard   set   operations   for   sets    (see   "AddSet", "RemoveSet",
"UniteSet", "IntersectSet", and "SubtractSet").

The last   section  tells  you more   about   sets  and   their  internal
representation (see "More about Sets").

All set theoretic functions, especially  'Intersection' and 'Union', also
accept  sets  as  arguments.  Thus  all  functions  described in  chapter
"Domains" are applicable to sets (see "Set Functions for Sets").

Since sets are just  a  special case of lists,   all the  operations  and
functions for lists, especially  the membership test  (see "In"),  can be
used for sets just as well (see "Lists").

%%%%%%%%%%%%%%%%%%%%%%%%%%%%%%%%%%%%%%%%%%%%%%%%%%%%%%%%%%%%%%%%%%%%%%%%%
\Section{IsSet}%
\index{test!for a set}

'IsSet( <obj> )'

'IsSet'  returns   'true' if the   object   <obj> is  a set  and  'false'
otherwise.  An object is a  set if it is a  sorted lists without holes or
duplicates.  Will cause an  error if evaluation of   <obj> is an  unbound
variable.

|    gap> IsSet( [] );
    true
    gap> IsSet( [ 2, 3, 5, 7, 11 ] );
    true
    gap> IsSet( [, 2, 3,, 5,, 7,,,, 11 ] );
    false        # this list contains holes
    gap> IsSet( [ 11, 7, 5, 3, 2 ] );
    false        # this list is not sorted
    gap> IsSet( [ 2, 2, 3, 5, 5, 7, 11, 11 ] );
    false        # this list contains duplicates
    gap> IsSet( 235711 );
    false        # this argument is not even a list |

%%%%%%%%%%%%%%%%%%%%%%%%%%%%%%%%%%%%%%%%%%%%%%%%%%%%%%%%%%%%%%%%%%%%%%%%%
\Section{Set}%
\index{convert!to a set}

'Set( <list> )'

'Set' returns a  new proper  set, which  is represented as  a sorted list
without holes or duplicates, containing the elements of the list <list>.

'Set' returns a new list even if the list <list> is already a proper set,
in this  case  it is   equivalent to  'ShallowCopy' (see  "ShallowCopy").
Thus the result is  a new list  that is not  identical to any other list.
The elements of  the result are however identical  to elements of <list>.
If <list> contains equal elements, it is  not specified to which of those
the element of the result is identical (see "Identical Lists").

|    gap> Set( [3,2,11,7,2,,5] );
    [ 2, 3, 5, 7, 11 ]
    gap> Set( [] );
    [  ] |

%%%%%%%%%%%%%%%%%%%%%%%%%%%%%%%%%%%%%%%%%%%%%%%%%%%%%%%%%%%%%%%%%%%%%%%%%
\Section{IsEqualSet}%
\index{test!for set equality}

'IsEqualSet( <list1>, <list2> )'

'IsEqualSet' returns  'true' if the   two lists <list1>  and <list2>  are
equal *when viewed as sets*, and  'false' otherwise.  <list1> and <list2>
are equal if  every element of <list1> is  also an element of <list2> and
if every element of <list2> is also an element of <list1>.

If both lists are proper sets then they are of course  equal if  and only
if they are also equal as  lists.  Thus  'IsEqualSet( <list1>, <list2> )'
is equivalent to 'Set( <list1>  ) = Set( <list2> )'  (see "Set"), but the
former is more efficient.

|    gap> IsEqualSet( [2,3,5,7,11], [11,7,5,3,2] );
    true
    gap> IsEqualSet( [2,3,5,7,11], [2,3,5,7,11,13] );
    false |

%%%%%%%%%%%%%%%%%%%%%%%%%%%%%%%%%%%%%%%%%%%%%%%%%%%%%%%%%%%%%%%%%%%%%%%%%
\Section{AddSet}%
\index{add!an element to a set}

'AddSet( <set>, <elm> )'

'AddSet' adds <elm>, which may be an elment of an  arbitrary type, to the
set   <set>, which must  be  a proper  set,  otherwise an  error  will be
signalled.  If <elm> is already an element of the  set <set>, the  set is
not  changed.  Otherwise <elm> is inserted  at the  correct position such
that <set> is again a set afterwards.

|    gap> s := [2,3,7,11];;
    gap> AddSet( s, 5 );  s;
    [ 2, 3, 5, 7, 11 ]
    gap> AddSet( s, 13 );  s;
    [ 2, 3, 5, 7, 11, 13 ]
    gap> AddSet( s, 3 );  s;
    [ 2, 3, 5, 7, 11, 13 ] |

'RemoveSet' (see "RemoveSet") is the counterpart of 'AddSet'.

%%%%%%%%%%%%%%%%%%%%%%%%%%%%%%%%%%%%%%%%%%%%%%%%%%%%%%%%%%%%%%%%%%%%%%%%%
\Section{RemoveSet}%
\index{remove!an element from a set}

'RemoveSet( <set>, <elm> )'

'RemoveSet' removes   the  element  <elm>,  which  may be   an  object of
arbitrary  type, from the set <set>,  which must be  a  set, otherwise an
error will be signalled.  If  <elm>  is  not an  element of <set> nothing
happens.  If <elm>  is  an element it is removed   and  all the following
elements in the list are moved one position forward.

|    gap> s := [ 2, 3, 4, 5, 6, 7 ];;
    gap> RemoveSet( s, 6 );
    gap> s;
    [ 2, 3, 4, 5, 7 ]
    gap> RemoveSet( s, 10 );
    gap> s;
    [ 2, 3, 4, 5, 7 ] |

'AddSet' (see "AddSet") is the counterpart of 'RemoveSet'.

%%%%%%%%%%%%%%%%%%%%%%%%%%%%%%%%%%%%%%%%%%%%%%%%%%%%%%%%%%%%%%%%%%%%%%%%%
\Section{UniteSet}%
\index{union!of sets}

'UniteSet( <set1>, <set2> )'

'UniteSet' unites the set <set1> with the set <set2>.  This is equivalent
to adding all the elements  in <set2>  to <set1> (see "AddSet").   <set1>
must be a proper set, otherwise an  error is  signalled.  <set2> may also
be  list that  is  not a  proper set,  in  which case 'UniteSet' silently
applies 'Set' to it first (see "Set").  'UniteSet' returns nothing, it is
only called to change <set1>.

|    gap> set := [ 2, 3, 5, 7, 11 ];;
    gap> UniteSet( set, [ 4, 8, 9 ] );  set;
    [ 2, 3, 4, 5, 7, 8, 9, 11 ]
    gap> UniteSet( set, [ 16, 9, 25, 13, 16 ] );  set;
    [ 2, 3, 4, 5, 7, 8, 9, 11, 13, 16, 25 ] |

The  function  'UnionSet'   (see  "Set  Functions   for  Sets")  is   the
nondestructive counterpart to the destructive procedure 'UniteSet'.

%%%%%%%%%%%%%%%%%%%%%%%%%%%%%%%%%%%%%%%%%%%%%%%%%%%%%%%%%%%%%%%%%%%%%%%%%
\Section{IntersectSet}%
\index{intersection!of sets}

'IntersectSet( <set1>, <set2> )'

'IntersectSet' intersects  the set <set1> with the  set <set2>.  This  is
equivalent  to removing all  the elements  that are not  in  <set2>  from
<set1> (see  "RemoveSet").  <set1> must be a  set, otherwise  an error is
signalled.  <set2> may be a list that is not a proper  set, in which case
'IntersectSet'   silently  applies  'Set' to      it first  (see  "Set").
'IntersectSet' returns nothing, it is only called to change <set1>.

|    gap> set := [ 2, 3, 4, 5, 7, 8, 9, 11, 13, 16 ];;
    gap> IntersectSet( set, [ 3, 5, 7, 9, 11, 13, 15, 17 ] );  set;
    [ 3, 5, 7, 9, 11, 13 ]
    gap> IntersectSet( set, [ 9, 4, 6, 8 ] );  set;
    [ 9 ] |

The function 'IntersectionSet'  (see  "Set Functions  for Sets")  is  the
nondestructive counterpart to the destructive procedure 'IntersectSet'.

%%%%%%%%%%%%%%%%%%%%%%%%%%%%%%%%%%%%%%%%%%%%%%%%%%%%%%%%%%%%%%%%%%%%%%%%%
\Section{SubtractSet}%
\index{subtract!a set from another}

'SubtractSet( <set1>, <set2> )'

'SubtractSet'  subtracts  the set  <set2>  from the set  <set1>.  This is
equivalent to  removing  all the elements in   <set2>  from  <set1>  (see
"RemoveSet").   <set1> must  be  a  proper  set, otherwise   an  error is
signalled.  <set2> may be a list that is not a proper  set, in which case
'SubtractSet' applies  'Set'   to it   first  (see "Set").  'SubtractSet'
returns nothing, it is only called to change <set1>.

|    gap> set := [ 2, 3, 4, 5, 6, 7, 8, 9, 10, 11 ];;
    gap> SubtractSet( set, [ 6, 10 ] );  set;
    [ 2, 3, 4, 5, 7, 8, 9, 11 ]
    gap> SubtractSet( set, [ 9, 4, 6, 8 ] );  set;
    [ 2, 3, 5, 7, 11 ] |

The function 'Difference'   (see  "Difference")  is   the  nondestructive
counterpart to destructive the procedure 'SubtractSet'.

%%%%%%%%%%%%%%%%%%%%%%%%%%%%%%%%%%%%%%%%%%%%%%%%%%%%%%%%%%%%%%%%%%%%%%%%%
\Section{Set Functions for Sets}%
\index{test!for subsets}

As was already mentioned  in  the introduction to this chapter all domain
functions also accept sets as arguments.  Thus all functions described in
the chapter "Domains"  are applicable to sets.   This  section  describes
those functions where  it might be helpful to know  the implementation of
those functions for sets.

'IsSubset( <set1>, <set2> )'%
\index{IsSubset!for sets}

This is implemented by 'IsSubsetSet', which you can call directly to save
a little  bit of  time.  Either argument to  'IsSubsetSet' may also  be a
list that is not a proper set, in which  case 'IsSubset' silently applies
'Set' (see "Set") to it first.

'Union( <set1>, <set2> )'%
\index{union!of sets}

This is implemented by 'UnionSet', which you can call  directly to save a
little bit of time.  Note that  'UnionSet' only accepts two  sets, unlike
'Union',  which accepts several sets or  a list of  sets.  The  result of
'UnionSet' is a new set,  represented as  a sorted list  without holes or
duplicates.  Each argument to 'UnionSet' may also be a list that is not a
proper set, in which case 'UnionSet'  silently  applies 'Set' (see "Set")
to this argument.  'UnionSet' is implemented in terms of its  destructive
counterpart 'UniteSet' (see "UniteSet").

'Intersection( <set1>, <set2> )'%
\index{intersection!of sets}

This is implemented by 'IntersectionSet', which you can call  directly to
save a little bit of time.  Note that 'IntersectionSet'  only accepts two
sets, unlike 'Intersection',  which  accepts several sets  or  a list  of
sets.  The  result of 'IntersectionSet' is  a new  set,  represented as a
sorted  list     without  holes   or  duplicates.   Each   argument    to
'IntersectionSet' may also be a list  that is not  a proper set, in which
case  'IntersectionSet'  silently  applies  'Set' (see  "Set")    to this
argument.  'IntersectionSet' is implemented in  terms of its  destructive
counterpart 'IntersectSet' (see "IntersectSet").

The result of 'IntersectionSet' and 'UnionSet' is always a new list, that
is not  identical to any other list.   The elements of that  list however
are identical to the corresponding elements of <set1>.   If <set1> is not
a proper list it is not specified to which of a number  of equal elements
in <set1> the element in the result is identical (see "Identical Lists").

%%%%%%%%%%%%%%%%%%%%%%%%%%%%%%%%%%%%%%%%%%%%%%%%%%%%%%%%%%%%%%%%%%%%%%%%%
\Section{More about Sets}

In the previous section we defined a proper  set as a sorted list without
holes or duplicates.  This representation is not  only nice to use, it is
also a good internal representation supporting efficient algorithms.  For
example the 'in' operator can use binary instead of a linear search since
a set is sorted.  For another example 'Union' only has to merge the sets.

However, all  those set functions  also allow lists that are  not  proper
sets,  silently making  a copy  of it and  converting this copy to a set.
Suppose all the functions would have to test  their arguments every time,
comparing  each element  with its  successor, to see if  they  are proper
sets.  This would chew up most  of  the performance advantage again.  For
example suppose 'in' would have to run  over the whole list, to see if it
is  a  proper set, so  it could  use the  binary search.   That  would be
ridiculous.

To avoid this a  list that is  a proper set  may, but need  not, have  an
internal flag set that tells  those functions that  this list is indeed a
proper set.  Those functions do not have to check this argument then, and
can use the more  efficient algorithms.  This  section tells  you  when a
proper set obtains this flag,  so you can write your  functions in such a
way that you make best use of the algorithms.

The results of 'Set', 'Difference', 'Intersection'  and 'Union' are known
to be sets by construction, and thus have the flag set upon creation.

If an argument to 'IsSet', 'IsEqualSet', 'IsSubset', 'Set', 'Difference',
'Intersection' or  'Union' is a proper  set, that does  not  yet have the
flag set, those functions will notice that and set the flag for this set.
Note that 'in' will use linear search if the  right operand does not have
the flag set, will therefore not detect  if it is  a proper set and will,
unlike the functions above, never set the flag.

If you change a proper set, that does have this  flag set, by assignment,
'Add'   or 'Append' the  set  will generally lose  it  flag,  even if the
change is such that the resulting list is still a proper set.  However if
the set has more than 100 elements and the value assigned or added is not
a list and not a record and the resulting list is still a proper set than
it will keep  the flag.  Note that  changing a list  that is not a proper
set will never set the flag, even if the resulting list  is a proper set.
Such a set will obtain the flag only if it is passed to a set function.

Suppose you have built a proper set  in such a way that  it does not have
the flag set, and that you now want  to perform lots of membership tests.
Then you  should call 'IsSet'  with that set   as an argument.   If it is
indeed  a proper set  'IsSet' will set the flag,  and the subsequent 'in'
operations will use  the more efficient binary  search.  You can think of
the call to 'IsSet' as a hint to {\GAP} that this list is a proper set.

There is no way you can set the flag for an ordinary  list  without going
through the checking in 'IsSet'.  The  internal  functions depend so much
on the fact that a list with  this flag set  is indeed sorted and without
holes and duplicates that the risk would be too high to allow setting the
flag without such a check.

%%%%%%%%%%%%%%%%%%%%%%%%%%%%%%%%%%%%%%%%%%%%%%%%%%%%%%%%%%%%%%%%%%%%%%%%%
%%
%E  Emacs . . . . . . . . . . . . . . . . . . . . . local Emacs variables
%%
%%  Local Variables:
%%  mode:               outline
%%  outline-regexp:     "\\\\Chapter\\|\\\\Section"
%%  fill-column:        73
%%  eval:               (hide-body)
%%  End:
%%



