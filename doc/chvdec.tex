%%%%%%%%%%%%%%%%%%%%%%%%%%%%%%%%%%%%%%%%%%%%%%%%%%%%%%%%%%%%%%%%%%%%%%%%%%%%%
%%
%A  chvdec.tex       CHEVIE documentation                      Jean Michel
%%
%Y  Copyright (C) 1996-2001  University  Paris VII.
%%
%%  This  file  documents functions for decimal and complex numbers
%%
%%%%%%%%%%%%%%%%%%%%%%%%%%%%%%%%%%%%%%%%%%%%%%%%%%%%%%%%%%%%%%%%%%%%%%%%%

\Chapter{CHEVIE utility functions -- Decimal and complex numbers}

The original incentive for the functions  described in this file was to
get the  ability to decide if  a cyclotomic number which  happens to be
real is positive  or negative (this is  needed to tell if a  root of an
arbitrary Coxeter group is negative  or positive). Of course, there are
other  uses for  fixed-precision real  and complex  numbers, which  the
functions  described here  provide. A  special feature  of the  present
implementation  is that  to make  evaluation of  cyclotomics relatively
fast, a  cache of primitive  roots of  unity is maintained  (the cached
values are kept for the largest precision ever used to compute them).

We first describe a general facility  to build complex numbers as pairs
of real numbers. The real numbers in  the pairs can be of any type that
\GAP\ knows  about\:\ integers, rationals, cyclotomics,  or elements of
any ring actually.

%%%%%%%%%%%%%%%%%%%%%%%%%%%%%%%%%%%%%%%%%%%%%%%%%%%%%%%%%%%%%%%%%%%%%%%%%
\Section{Complex}%
\index{Complex}%

'Complex( <r>[, <i>] )'

In the first form, defines a complex  number whose real part is <r> and
imaginary part is <i>. If omitted, <i>  is taken to be 0. There are two
special cases  when there is only  one argument\:\ if <r>  is already a
complex,  it is  returned; and  if <r>  is a  cyclotomic number,  it is
converted to the pair of its real  and imaginary part and returned as a
complex.

|    gap> Complex(0,1);
    I
    gap> Complex(E(3));
    -1/2+ER(3)/2I
    gap> Complex(E(3)^2);
    -1/2-ER(3)/2I
    gap> x:=X(Rationals);;x.name:="x";;Complex(0,x);
    xI|

The last line shows that the arguments to 'Complex' can be of any ring.
Complex numbers are  represented as a record with two  fields, '.r' and
'.i' holding the real and imaginary part respectively.

%%%%%%%%%%%%%%%%%%%%%%%%%%%%%%%%%%%%%%%%%%%%%%%%%%%%%%%%%%%%%%%%%%%%%%%%%%%%
\Section{Operations for complex numbers}

The arithmetic operations '+', '-', '\*', '/' and '\^' work for complex
numbers. They also have 'Print' and 'String' methods.

|    gap> Complex(0,1);
    I
    gap> last+1;
    1+I
    gap> last^2;
    2I
    gap> last^2;
    -4
    gap> last+last2;
    -4+2I
    gap> x:=X(Rationals);;x.name:="x";;Complex(0,x);
    xI
    gap> last^2;
    -x^2|

Finally  we  should  mention  the  'FormatGAP'  method,  which allows to print
complex numbers in a way such that they can be read back in \GAP\:\

|    gap> a:=Complex(1/2,1/3);
    1/2+1/3I
    gap> FormatGAP(a);
    "Complex(1/2,1/3)"|

%%%%%%%%%%%%%%%%%%%%%%%%%%%%%%%%%%%%%%%%%%%%%%%%%%%%%%%%%%%%%%%%%%%%%%%%%%%%
\Section{ComplexConjugate}
\index{ComplexConjugate}

'ComplexConjugate( <c>)'

This  function   applies  complex  conjugation  to   its  argument.  It
knows  ho  to do  this  for  cyclotomic  numbers  (it then  just  calls
'GaloisCyc(<c>,-1)'),  complex  numbers,   lists  (it  conjugates  each
element of the list), polynomials (it conjugates each coefficient), and
can  be taught  to conjugate  elements 'x'  of an  arbitrary domain  by
defining 'x.operations.ComplexConjugate'.

|    gap> ComplexConjugate(Complex(0,1));
    -I
    gap> ComplexConjugate(E(3));
    E(3)^2
    gap> x:=X(Cyclotomics);;x.name:="x";;ComplexConjugate(x+E(3));
    x + (E(3)^2)|

%%%%%%%%%%%%%%%%%%%%%%%%%%%%%%%%%%%%%%%%%%%%%%%%%%%%%%%%%%%%%%%%%%%%%%%%%%%%
\Section{IsComplex}
\index{IsComplex}

'IsComplex( <c>)'

This function returns 'true' iff its argument is a complex number.

|    gap> IsComplex(Complex(1,0));
    true
    gap> IsComplex(E(4));
    false|

%%%%%%%%%%%%%%%%%%%%%%%%%%%%%%%%%%%%%%%%%%%%%%%%%%%%%%%%%%%%%%%%%%%%%%%%%
\Section{evalf}%
\index{evalf}%

'evalf( <c> [, <prec>] )'

The  name  of  this  function  intentionally mimics  that  of  a  Maple
function.  It computes  a  fixed-precision decimal  number with  <prec>
digits  after the  decimal  point approximating  its  argument; if  not
given, <prec> is taken  to be 10 (this can be  changed via the function
'SetDecimalPrecision',  see below).  Trailing zeroes  are not  shown on
output, so the  actual precision may be more than  the number of digits
shown.

|    gap> evalf(1/3);
    0.3333333333|

As one  can see,  the resulting 'decimal  numbers' have  an appropriate
Print method (which uses the String method).

|    gap> evalf(1/3,20);
    0.33333333333333333333|

'evalf' can also be applied  to cyclotomic or complex numbers, yielding
a complex which is a pair of decimal numbers.

|    gap> evalf(E(3));
    -0.5+0.8660254038I|

'evalf' works also for strings (the result is truncated if too precise)

|    gap> evalf(".3450000000000000000001"); # precision is 10
    0.345|

and for lists (it is applied recursively to each element).

|    gap> evalf([E(5),1/7]);
    [ 0.3090169944+0.9510565163I, 0.1428571429 ]|

Finally, an 'evalf'  method can be defined for elements  of any domain.
One has been defined in \CHEVIE\ for complex numbers\:\

|    gap> a:=Complex(1/2,1/3);
    1/2+1/3I
    gap> evalf(a);
    0.5+0.3333333333I|

%%%%%%%%%%%%%%%%%%%%%%%%%%%%%%%%%%%%%%%%%%%%%%%%%%%%%%%%%%%%%%%%%%%%%%%%%
\Section{Rational}
\index{Rational}

'Rational(<d>)'

<d>  is a  'decimal' number.  The function  returns the  rational number
which is actually represented by <d>

|    gap> evalf(1/3);
    0.3333333333
    gap> Rational(last);
    33333333333/100000000000|

%%%%%%%%%%%%%%%%%%%%%%%%%%%%%%%%%%%%%%%%%%%%%%%%%%%%%%%%%%%%%%%%%%%%%%%%%%%%
\Section{SetDecimalPrecision}%
\index{SetDecimalPrecision}%

'SetDecimalPrecision( <prec> )'

This function  sets the  default precision to  be used  when converting
numbers to decimal numbers without giving a second argument to 'evalf'.

|    gap> SetDecimalPrecision(20);
    gap> evalf(1/3);
    0.33333333333333333333
    gap> SetDecimalPrecision(10);|

%%%%%%%%%%%%%%%%%%%%%%%%%%%%%%%%%%%%%%%%%%%%%%%%%%%%%%%%%%%%%%%%%%%%%%%%%%%%
\Section{Operations for decimal numbers}

\index{GetRoot}
The arithmetic operations '+', '-', '\*', '/' and '\^' work for decimal
numbers, as well as the function 'GetRoot' and the comparison functions
'\<', |>|, etc...  The precision of the result of  an operation is that
of the  least precise number used.  They can be raised  to a fractional
power\:\ 'GetRoot(d,n)'  is equivalent  to 'd\^(1/n)'.  Decimal numbers
also have 'Print' and 'String' methods.

|    gap> evalf(1/3)+1;
    1.3333333333
    gap> last^3;
    2.3703703704
    gap> evalf(E(3));
    -0.5+0.8660254038I
    gap> last^3;
    1
    gap> evalf(ER(2));
    1.4142135624
    gap> GetRoot(evalf(2),2);
    1.4142135624
    gap> evalf(2)^(1/2);
    1.4142135624
    gap> evalf(1/3,20);
    0.33333333333333333333
    gap> last+evalf(1);
    1.3333333333
    gap> last2+1;
    1.33333333333333333333|

Finally  we should mention the 'FormatGAP'  method, which, given option 'GAP',
allows  to print decimal numbers in  a way such that they  can be read back in
\GAP\:\

|    gap> FormatGAP(evalf(1/3));
    "evalf(33333333333/100000000000,10)"|

%%%%%%%%%%%%%%%%%%%%%%%%%%%%%%%%%%%%%%%%%%%%%%%%%%%%%%%%%%%%%%%%%%%%%%%%%%%%
\Section{Pi}%
\index{Pi}%

'Pi( [<prec>])'

This function  returns a  decimal approximation  to $\pi$,  with <prec>
digits  (or if  <prec>  is omitted  with the  default  umber of  digits
defined by 'SetDecimalPrecision', initially 10).

|    gap> Pi();
    3.1415926536
    gap> Pi(34);
    3.1415926535897932384626433832795029|

%%%%%%%%%%%%%%%%%%%%%%%%%%%%%%%%%%%%%%%%%%%%%%%%%%%%%%%%%%%%%%%%%%%%%%%%%%%%
\Section{Exp}%
\index{Exp}%

'Exp( <x>)'

This  function returns  the complex  exponential of  <x>. The  argument
should be a decimal or a decimal complex. The result has as many digits
of precision as the argument.

|    gap> Exp(evalf(1));
    2.7182818285
    gap> Exp(evalf(1,20));
    2.71828182845904523536
    gap> Exp(Pi()*evalf(E(4)));
    -1|

The code of 'Exp' shows how easy it is to use decimal numbers.

|    gap> Print(Exp,"\n");
    function ( x )
      local  res, i, p, z;
      z := 0 * evalf( x );
      res := z;
      p := 1;
      i := 1;
      while p <> z  do
        res := p + res;
        p := 1 / i * p * x;
        i := i + 1;
      od;
      return res;
    end|

%%%%%%%%%%%%%%%%%%%%%%%%%%%%%%%%%%%%%%%%%%%%%%%%%%%%%%%%%%%%%%%%%%%%%%%%
\Section{IsDecimal}%
\index{IsDecimal}%

'IsDecimal( <x>)'

returns 'true' iff <x> is a decimal number.

|    gap> IsDecimal(evalf(1));
    true
    gap> IsDecimal(evalf(E(3)));
    false|

%%%%%%%%%%%%%%%%%%%%%%%%%%%%%%%%%%%%%%%%%%%%%%%%%%%%%%%%%%%%%%%%%%%%%%%%
