%%%%%%%%%%%%%%%%%%%%%%%%%%%%%%%%%%%%%%%%%%%%%%%%%%%%%%%%%%%%%%%%%%%%%%%%%%%%%
%%
%A  chvalgg.tex       CHEVIE documentation       Meinolf Geck, Frank Luebeck,
%A                                                Jean Michel, G"otz Pfeiffer
%%
%Y  Copyright (C) 1992 - 2010  Lehrstuhl D f\"ur Mathematik, RWTH Aachen, IWR
%Y  der Universit\"at Heidelberg, University of St. Andrews, and   University
%Y  Paris VII.
%%
%%  This  file  contains  the  description  of  the  GAP functions of CHEVIE
%%  dealing with algebraic groups and semisimple elements.
%%
%%%%%%%%%%%%%%%%%%%%%%%%%%%%%%%%%%%%%%%%%%%%%%%%%%%%%%%%%%%%%%%%%%%%%%%%%
\newcommand\bG{{\mathbf G}}
\newcommand\bH{{\mathbf H}}
\newcommand\bS{{\mathbf S}}
\newcommand\bT{{\mathbf T}}
\Chapter{Algebraic groups and semi-simple elements}

Let  us fix an algebraically closed field  $K$ and let $\bG$ be a connected
reductive  algebraic group over $K$. Let $\bT$ be a maximal torus of $\bG$,
let  $X(\bT)$  be  the  character  group  of $\bT$ (resp. $Y(\bT)$ the dual
lattice  of one-parameter subgroups of $\bT$) and $\Phi$ (resp $\Phi^\vee$)
the roots (resp. coroots) of $\bG$ with respect to $\bT$.

Then   $\bG$  is  determined   up  to  isomorphism   by  the  *root  datum*
$(X(\bT),\Phi,  Y(\bT),\Phi^\vee)$.  In  algebraic  terms, this consists in
giving  a free  $\Z$-lattice $X=X(\bT)$  of dimension  the *rank*  of $\bT$
(which is also called the *rank* of $\bG$), and a root system $\Phi \subset
X$, and giving similarly the dual roots $\Phi^\vee\subset Y=Y(\bT)$.

This  is obtained  by a  slight generalization  of our  setup for a Coxeter
group  $W$. This time we assume the canonical basis of the vector space $V$
on  which $W$  acts is  a $\Z$-basis  of $X$,  and $\Phi$ is specified by a
matrix  'W.simpleRoots' whose lines are the  simple roots expressed in this
basis   of   $X$.   Similarly   $\Phi^\vee$   is   described  by  a  matrix
'W.simpleCoroots'  whose lines are  the simple coroots  in the basis of $Y$
dual to the chosen basis of $X$. The duality pairing between $X$ and $Y$ is
the  canonical one, that is the pairing  between vectors $x\in X$ and $y\in
Y$  is  given  in  \GAP\  by  'x\*y'.  Thus,  we  must  have  the  relation
'W.simpleCoroots\*TransposedMat(W.simpleRoots)=CartanMat(W)'.

We get that in \CHEVIE\ by a new form of the function 'CoxeterGroup', where
the  arguments are  the two  matrices 'W.simpleRoots' and 'W.simpleCoroots'
described  above. The roots need not generate $V$, so the matrices need not
be  square. For instance, the root datum of  the linear group of rank 3 can
be specified as\:

|    gap> W := CoxeterGroup( [ [ -1, 1, 0], [ 0, -1, 1 ] ],
    >                       [ [ -1, 1, 0], [ 0, -1, 1 ] ] );
    CoxeterGroup([[-1,1,0],[0,-1,1]],[[-1,1,0],[0,-1,1]])
    gap> MatXPerm( W, W.1);
    [ [ 0, 1, 0 ], [ 1, 0, 0 ], [ 0, 0, 1 ] ]|

here  the symmetric group on 3 letters  acts by permutation of the basis of
$X$.  The dimension of $X$ (the length of the vectors in '.simpleRoots') is
the  *rank* and the dimension  of the subspace generated  by the roots (the
length  of '.simpleroots') is called the *semi-simple rank*. In the example
the  rank is 3 and  the semisimple rank is  2.

The  default form 'W\:=CoxeterGroup(\"A\",2)' defines the adjoint algebraic
group  (the group with a trivial center). In that case $\Phi$ is a basis of
$X$,  so 'W.simpleRoots' is the  identity matrix and 'W.simpleCoroots'
is   the  Cartan  matrix  'CartanMat(W)'  of  the  root  system.  The  form
'CoxeterGroup(\"A\",2,\"sc\")'  constructs the  semisimple simply connected
algebraic  group, where 'W.simpleRoots' is the transposed of 'CartanMat(W)'
and 'W.simpleCoroots' is the identity matrix.

There  is an extreme form  of root data which  requires another function to
specify\:  when <W> is  the trivial |CoxeterGroup()|  and there are thus no
roots  (in this case $\bG$ is a torus), the root datum cannot be determined
by  the roots,  but is  entirely determined  by the  rank $r$. The function
'Torus(<r>)' constructs such a root datum.

Finally,  there  is  also  a  function  'RootDatum'  which understands some
familiar names for the algebraic groups and gives the results that could be
obtained   by   giving   the   appropriate   matrices  'W.simpleRoots'  and
'W.simpleCoroots'\:

|    gap> RootDatum("gl",3);   # same as the previous example
    RootDatum("gl",3)|

{\bf Semisimple elements}

It  is also possible  to compute with  semi-simple elements. The first type
are  finite order  elements of  $\bT$, which  over an  algebraically closed
field  $K$  are  in  bijection  with  elements  of  $Y\otimes  \Q/\Z$ whose
denominator is prime to the characteristic of $K$. These are represented as
elements of a vector space of rank $r$ over $\Q$, taken 'Mod1' whenever the
need arises, where 'Mod1' is the function which replaces the numerator of a
fraction   with  the   numerator  'mod'   the  denominator;   the  fraction
$\frac{p}{q}$  represents a  primitive $q$-th  root of  unity raised to the
$p$-th  power.  In  this  representation,  multiplication of roots of unity
becomes  addition 'Mod1' of rationals and  raising to the power $n$ becomes
multiplication  by $n$. We  call this the  ``additive\'\' representation of
semisimple elements.

Here  is an example of computations  with semisimple-elements given as list
of $r$ elements of $\Q/\Z$.

|    gap> G:=RootDatum("sl",4);
    RootDatum("sl",4)
    gap> L:=ReflectionSubgroup(G,[1,3]);
    ReflectionSubgroup(RootDatum("sl",4), [ 1, 3 ])
    gap> AlgebraicCentre(L);
    rec(
      Z0 :=
       SubTorus(ReflectionSubgroup(RootDatum("sl",4), [ 1, 3 ]),[ [ 1, 2, \
    1 ] ]),
      AZ := Group( <0,0,1/2> ),
      descAZ := [ [ 1, 2 ] ] )
    gap> SemisimpleSubgroup(last.Z0,3);
    Group( <1/3,2/3,1/3> )
    gap> e:=Elements(last);
    [ <0,0,0>, <1/3,2/3,1/3>, <2/3,1/3,2/3> ]|

First,  the group  $\bG=SL_4$ is  constructed, then  the Levi  subgroup $L$
consisting  of block-diagonal matrices  of shape $2\times  2$. The function
'AlgebraicCentre'  returns a record with \:\ the neutral component $Z^0$ of
the  centre $Z$ of  $L$, represented by  a basis of  $Y(Z^0)$, a complement
subtorus  $S$ of $\bT$ to $Z^0$ represented similarly by a basis of $Y(S)$,
and  semi-simple elements  representing the  classes of  $Z$ modulo $Z^0$ ,
chosen  in $S$. The classes $Z/Z^0$ also biject to the fundamental group as
given  by the  field '.descAZ',  see "AlgebraicCentre"  for an explanation.
Finally the semi-simple elements of order 3 in $Z^0$ are computed.

|    gap> e[2]^G.2;
    <1/3,0,1/3>
    gap> Orbit(G,e[2]);
    [ <1/3,2/3,1/3>, <1/3,0,1/3>, <2/3,0,1/3>, <1/3,0,2/3>, <2/3,0,2/3>,
      <2/3,1/3,2/3> ]|

Since  over an algebraically  closed field $K$  the points of  $\bT$ are in
bijection  with $Y\otimes  K^\times$ it  is also  possible to represent any
point  of $\bT$ over $K$ as a list of $r$ non-zero elements of $K$. This is
the ``multiplicative\'\' representation of semisimple elements. here is the
same   computation  as  above  performed  with  semisimple  elements  whose
coefficients are in the finite field 'GF(4)'\:

|    gap> s:=SemisimpleElement(G,List([1,2,1],i->Z(4)^i));
    <Z(2^2),Z(2^2)^2,Z(2^2)>
    gap> s^G.2;
    <Z(2^2),Z(2)^0,Z(2^2)>
    gap> Orbit(G,s);
    [ <Z(2^2),Z(2^2)^2,Z(2^2)>, <Z(2^2),Z(2)^0,Z(2^2)>,
      <Z(2^2)^2,Z(2)^0,Z(2^2)>, <Z(2^2),Z(2)^0,Z(2^2)^2>,
      <Z(2^2)^2,Z(2)^0,Z(2^2)^2>, <Z(2^2)^2,Z(2^2),Z(2^2)^2> ]|

We  can  compute  the  centralizer  $C_\bG(s)$  of  a semisimple element in
$\bG$\:

|    gap> G:=CoxeterGroup("A",3);
    CoxeterGroup("A",3)
    gap> s:=SemisimpleElement(G,[0,1/2,0]);
    <0,1/2,0>
    gap> Centralizer(G,s);
    (A1xA1)<1,3>.(q+1)|

The  result is an  extended reflection group;  the reflection group part is
the Weyl group of $C_\bG^0(s)$ and the extended part are representatives of
$C_\bG(s)$  modulo  $C_\bG^0(s)$  taken  as  diagram  automorphisms  of the
reflection  part.  Here  is  is  printed  as a coset $C_\bG^0(s)\phi$ which
generates $C_\bG(s)$.

%%%%%%%%%%%%%%%%%%%%%%%%%%%%%%%%%%%%%%%%%%%%%%%%%%%%%%%%%%%%%%%%%%%%%%%%%
\Section{CoxeterGroup (extended form)}
\index{CoxeterGroup}

'CoxeterGroup( <simpleRoots>, <simpleCoroots> )'

'CoxeterGroup( <C>[, \"sc\"] )'

'CoxeterGroup( <type1>, <n1>, ... , <typek>, <nk>[, \"sc\"] )'

The  above are  extended forms  of the  function 'CoxeterGroup' allowing to
specify  more general root data. In the first  form a set of roots is given
explicitly  as the lines of  the matrix <simpleRoots>, representing vectors
in  a vector space  <V>, as well  as a set  of coroots as  the lines of the
matrix <simpleCoroots> expressed in the dual basis of $V^\vee$. The product
'<C>=<simpleCoroots>\*TransposedMat(<simpleRoots>)'  must be a valid Cartan
matrix.  The dimension of <V> can be greater than 'Length(<C>)'. The length
of  <C> is  called the  *semisimple rank*  of the  Coxeter datum, while the
dimension of <V> is called its *rank*.

In the second form <C> is a Cartan matrix, and the call 'CoxeterGroup(<C>)'
is  equivalent  to  'CoxeterGroup(IdentityMat(Length(<C>)),<C>)'.  When the
optional '\"sc\"' argument is given the situation is reversed\:\ the simple
coroots  are given  by the  identity matrix,  and the  simple roots  by the
transposed  of <C> (this corresponds to the embedding of the root system in
the  lattice  of  characters  of  a  maximal  torus in a *simply connected*
algebraic group).

The argument '\"sc\"' can also be given in the third form with the same
effect.

The  following fields in a Coxeter group record complete the description of
the corresponding *root datum*\:

'simpleRoots':\\ the matrix of simple roots

'simpleCoroots':\\ the matrix of simple coroots

'matgens':\\  the matrices  (in row  convention ---  that is,  the matrices
operate *from the right*) of the simple reflections of the Coxeter group.

%%%%%%%%%%%%%%%%%%%%%%%%%%%%%%%%%%%%%%%%%%%%%%%%%%%%%%%%%%%%%%%%%%%%%%%%%
\Section{RootDatum}
\index{RootDatum}

'RootDatum(<type>,<n>)'

This  function returns the root datum  for the algebraic group described by
<type>  and <n>. The  types understood as  of now are\: '\"gl\"', '\"sl\"',
'\"pgl\"',    '\"sp\"',   '\"so\"',    '\"psp\"',   '\"csp\"',   '\"pso\"',
'\"halfspin\"', '\"spin\"', '\"gpin\"', '\"F4\"' and '\"G2\"'.

|    gap> RootDatum("spin",8);# same as CoxeterGroup(\"D\",4,\"sc\")
    RootDatum("spin",8)|

%%%%%%%%%%%%%%%%%%%%%%%%%%%%%%%%%%%%%%%%%%%%%%%%%%%%%%%%%%%%%%%%%%%%%%%%%
\Section{Dual for root Data}
\index{Dual}

'Dual(<W>)'

This function returns the dual root datum of the root datum <W>, describing
the   Langlands  dual  algebraic  group.   The  fields  '.simpleRoots'  and
'.simpleCoroots' are swapped in the dual compared to <W>.

|    gap> W:=CoxeterGroup("B",3);
    CoxeterGroup("B",3)
    gap> Dual(W);
    CoxeterGroup("C",3,sc)|

%%%%%%%%%%%%%%%%%%%%%%%%%%%%%%%%%%%%%%%%%%%%%%%%%%%%%%%%%%%%%%%%%%%%%%%%%
\Section{Torus}
\index{Torus}

'Torus(<rank>)'

This  function returns the \CHEVIE\ object corresponding to the notion of a
torus  of dimension <rank>, a Coxeter group  of semisimple rank 0 and given
<rank>.  This corresponds to a split torus; the extension to Coxeter cosets
is more useful (see "Torus for Coxeter cosets").

|    gap> Torus(3);
    Torus(3)
    gap> ReflectionName(last);
    "(q-1)^3"|

%%%%%%%%%%%%%%%%%%%%%%%%%%%%%%%%%%%%%%%%%%%%%%%%%%%%%%%%%%%%%%%%%%%%%%%%%
\Section{FundamentalGroup for algebraic groups}
\index{FundamentalGroup}

'FundamentalGroup(<W>)'

This  function returns the fundamental group of the algebraic group defined
by  the Coxeter  group record  <W>. This  group is  returned as  a group of
diagram  automorphisms of the corresponding affine Weyl group, that is as a
group  of permutations of  the set of  simple roots enriched  by the lowest
root  of  each  irreducible  component.  The  definition  we  take  of  the
fundamental  group  of  a  (not  necessarily semisimple) reductive group is
$(P\cap  Y(\bT))/Q$ where $P$ is the  coweight lattice (the dual lattice in
$Y(\bT)\otimes\Q$  of the root  lattice) and $Q$  is the coroot latice. The
bijection  between elements of $P/Q$ and diagram automorphisms is explained
in   the  context  of  non-irreducible   groups  for  example  in  \cite[\S
3.B]{Bon05}.

|    gap> W:=CoxeterGroup("A",3);
    CoxeterGroup("A",3)
    gap> FundamentalGroup(W);
    Group( ( 1, 2, 3,12) )
    gap> W:=CoxeterGroup("A",3,"sc");
    CoxeterGroup("A",3,"sc")
    gap> FundamentalGroup(W);
    Group( () )|

%%%%%%%%%%%%%%%%%%%%%%%%%%%%%%%%%%%%%%%%%%%%%%%%%%%%%%%%%%%%%%%%%%%%%%%%%
\Section{IntermediateGroup}
\index{IntermediateGroup}

'IntermediateGroup(<W>, <indices>)'

This computes a Weyl group record representing a semisimple algebraic group
intermediate  between  the  adjoint  group  ---  obtained  by  a  call like
'CoxeterGroup(\"A\",3)'---  and the simply  connected semi-simple group ---
obtained  by  a  call  like  'CoxeterGroup(\"A\",3,\"sc\")'.  The  group is
specified  by specifying  a subset  of the  *minuscule weights*,  which are
weights  whose scalar product with every coroot is in $-1,0,1$ (the weights
are  the elements of the *weight lattice*, the lattice in $X(\bT)\otimes\Q$
dual  to the coroot lattice). The non-trivial characters of the (algebraic)
center  of a semi-simple simply connected  algebraic group are in bijection
with  the minuscule weights; this set is also in bijection with $P/Q$ where
$P$  is  the  weight  lattice  and  $Q$  is  the  root  lattice.  If <W> is
irreducible,  the minuscule  weights are  part of  the basis  of the weight
lattice  given by the *fundamental weights*, which is the dual basis of the
simple  coroots. They  can thus  be specified  by an  <index> in the Dynkin
diagram  (see "PrintDiagram").  The constructed  group has lattice $X(\bT)$
generated  by the sum  of the root  lattice and the  weights with the given
<indices>.  If  <W>  is  not  irreducible,  a  minuscule weight is a sum of
minuscule  weights in different components. An element of <indices> is thus
itself  a list,  interpreted as  representing the  sum of the corresponding
weights.

|    gap> W:=CoxeterGroup("A",3);;
    gap> IntermediateGroup(W,[]); # adjoint
    CoxeterGroup("A",3)
    gap> FundamentalGroup(last);
    Group( ( 1, 2, 3,12) )
    gap> IntermediateGroup(W,[2]);# intermediate
    CoxeterGroup([[2,0,-1],[0,1,0],[0,0,1]],[[1,-1,0],[-1,2,-1],[1,-1,2]])
    gap> FundamentalGroup(last);
    Group( ( 1, 3)( 2,12) )|

%%%%%%%%%%%%%%%%%%%%%%%%%%%%%%%%%%%%%%%%%%%%%%%%%%%%%%%%%%%%%%%%%%%%%%%%%
\Section{WeightInfo}
\index{WeightInfo}

'WeightInfo(<W>)'

$W$  is a  Coxeter group  record describing  an algebraic  group $\bG$. The
function  is independent  of the  isogeny type  of $\bG$ and describes some
information  about the root system  of $\bG$. It returns  a record with the
following fields\:

'.minusculeWeights': the minuscule weights, described as their position in 
   the list of fundamental weights. For non-irreducible groups, a weight is
   the  sum of  at most  one weight  in each  irreducible component.  It is
   represented  as the list of its components. For consistency, in the case
   of an irreducible system, a weight is represented as a one-element list.

'.minusculeCoweights': the minuscule coweights, represented in the same 
   manner as the minuscule weights

'.decompositions': for each coweight, its decomposition in terms of the
   generators  of the adjoint  fundamental group (given  by the list of the
   exponents of the generators). Together with the next field it enables to
   work out the group structure of the adjoint fundamental group.

'.moduli': the list of orders of the generators of the fundamental group.

'.AdjointFundamentalGroup': the list of generators of the adjoint fundamental
   group, given as permutations.

'.CenterSimplyConnected': A list of semisimple elements generating the center
   of the universal covering of $\bG$.

|    gap> WeightInfo(CoxeterGroup("A",2,"B",2));
    rec(
      minusculeWeights := [ [ 1, 3 ], [ 1 ], [ 2, 3 ], [ 2 ], [ 3 ] ],
      minusculeCoweights := [ [ 1, 4 ], [ 1 ], [ 2, 4 ], [ 2 ], [ 4 ] ],
      decompositions := [ [ 1, 1 ], [ 1, 0 ], [ 2, 1 ], [ 2, 0 ], [ 0, 1 ] ],
      moduli := [ 3, 2 ],
      CenterSimplyConnected := [ [ 2/3, 1/3, 0, 0 ], [ 0, 0, 1/2, 0 ] ],
      AdjointFundamentalGroup := [ ( 1, 2,12), ( 4,14) ] )|

%%%%%%%%%%%%%%%%%%%%%%%%%%%%%%%%%%%%%%%%%%%%%%%%%%%%%%%%%%%%%%%%%%%%%%%%%
\Section{SemisimpleElement}
\index{SemisimpleElement}

'SemisimpleElement(<W>,<v>[,<additive>])'

<W>  should be  a root  datum, given  as a  Coxeter group record for a Weyl
group,  and  <v>  a  list  of  length  'W.rank'. The result is a semisimple
element record, which has the fields\:

'.v': the given list, taken 'Mod1' if its elements are rationals.

'.group': the parent of the group <W>.

|    gap> G:=CoxeterGroup("A",3);;
    gap> s:=SemisimpleElement(G,[0,1/2,0]);
    <0,1/2,0>|

If  all elements of $v$ are rational  numbers, they are converted by 'Mod1'
to fractions between  $0$ and  $1$ representing  roots of  unity, and these
roots  of unity are multiplied by adding  'Mod1' the fractions. In this way
any semisimple element of finite order can be represented.

If  the entries  are not  rational numbers,  they are  assumed to represent
elements  of a field which are  multiplied or added normally. To explicitly
control  if  the  entries  are  to  be  treated  additively or not, a third
argument  can be given\:  if 'true' the  entries are treated additively, or
not if 'false'. For entries to be treated additively, they must belong to a
domain for which the method 'Mod1' had been defined.

%%%%%%%%%%%%%%%%%%%%%%%%%%%%%%%%%%%%%%%%%%%%%%%%%%%%%%%%%%%%%%%%%%%%%%%%%
\Section{Operations for semisimple elements}

The  arithmetic operations '\*', '/' and '\^' work for semisimple elements.
They  also have 'Print' and 'String' methods. We first give an element with
elements of $\Q/\Z$ representing roots of unity.

|    gap> G:=CoxeterGroup("A",3);
    CoxeterGroup("A",3)
    gap> s:=SemisimpleElement(G,[0,1/2,0]);
    <0,1/2,0>
    gap> t:=SemisimpleElement(G,[1/2,1/3,1/7]);
    <1/2,1/3,1/7>
    gap> s*t;
    <1/2,5/6,1/7>
    gap> t^3;
    <1/2,0,3/7>
    gap> t^-1;
    <1/2,2/3,6/7>
    gap> t^0;
    <0,0,0>
    gap> String(t);
    "<1/2,1/3,1/7>"|

then a similar example with elements of 'GF(5)':

|    gap> s:=SemisimpleElement(G,Z(5)*[1,2,1]);
    <Z(5),Z(5)^2,Z(5)>
    gap> t:=SemisimpleElement(G,Z(5)*[2,3,4]);
    <Z(5)^2,Z(5)^0,Z(5)^3>
    gap> s*t;
    <Z(5)^3,Z(5)^2,Z(5)^0>
    gap> t^3;
    <Z(5)^2,Z(5)^0,Z(5)>
    gap> t^-1;
    <Z(5)^2,Z(5)^0,Z(5)>
    gap> t^0;
    <Z(5)^0,Z(5)^0,Z(5)^0>
    gap> String(t);
    "<Z(5)^2,Z(5)^0,Z(5)^3>"|

The operation '\^' also works for applying an element of its defining Weyl
group to a semisimple element, which allows orbit computations\:

|    gap> s:=SemisimpleElement(G,[0,1/2,0]);
    <0,1/2,0>
    gap> s^G.2;
    <1/2,1/2,1/2>
    gap> Orbit(G,s);
    [ <0,1/2,0>, <1/2,1/2,1/2>, <1/2,0,1/2> ]|

The operation '\^' also works for applying a root to a semisimple element\:

|    gap> s:=SemisimpleElement(G,[0,1/2,0]);
    <0,1/2,0>
    gap> s^G.roots[4];
    1/2
    gap> s:=SemisimpleElement(G,Z(5)*[1,1,1]);
    <Z(5),Z(5),Z(5)>
    gap> s^G.roots[4];
    Z(5)^2|

\index{Frobenius}
'Frobenius(  <WF> )':\\ If  'WF' is a  Coxeter coset associated to the
Coxeter   group  <W>,  the  function  'Frobenius'  returns  the  associated
automorphism which can be applied to semisimple elements, see "Frobenius".

|    gap> W:=CoxeterGroup("D",4);;WF:=CoxeterCoset(W,(1,2,4));;
    gap> s:=SemisimpleElement(W,[1/2,0,0,0]);
    <1/2,0,0,0>
    gap> F:=Frobenius(WF);
    function ( arg ) ... end
    gap> F(s);
    <0,0,0,1/2>
    gap> F(s,-1);
    <0,1/2,0,0>|

%%%%%%%%%%%%%%%%%%%%%%%%%%%%%%%%%%%%%%%%%%%%%%%%%%%%%%%%%%%%%%%%%%%%%%%%%
\Section{Centralizer for semisimple elements}
\index{Centralizer for semisimple elements}

'Centralizer( <W>, <s>)'

<W>  should be a Weyl group record or and extended reflection group and <s>
a  semisimple element for <W>. This  function returns the stabilizer of the
semisimple element <s> in <W>, which describes also $C_\bG(s)$, if $\bG$ is
the  algebraic  group  described  by  <W>.  The  stabilizer  is an extended
reflection group, with the reflection group part equal to the Weyl group of
$C_\bG^0(s)$,  and  the  diagram  automorphism  part being those induced by
$C_\bG(s)/C_\bG^0(s)$ on $C_\bG^0(s)$.

|    gap> G:=CoxeterGroup("A",3);
    CoxeterGroup("A",3)
    gap> s:=SemisimpleElement(G,[0,1/2,0]);
    <0,1/2,0>
    gap> Centralizer(G,s);
    (A1xA1)<1,3>.(q+1)|

%%%%%%%%%%%%%%%%%%%%%%%%%%%%%%%%%%%%%%%%%%%%%%%%%%%%%%%%%%%%%%%%%%%%%%%%%
\Section{SubTorus}
\index{SubTorus}

'SubTorus(<W>,<Y>)'

The  function returns the subtorus $\bS$ of  the maximal torus $\bT$ of the
reductive group represented by the Weyl group record <W> such that $Y(\bS)$
is  the (pure) sublattice  of $Y(\bT)$ generated  by the (integral) vectors
<Y>.  A basis of $Y(\bS)$ adapted to $Y(\bT)$ is computed and stored in the
field  'S.generators' of the returned  subtorus object. Here, adapted means
that  there is  a set  of integral  vectors, stored in 'S.complement', such
that  'M\:=Concatenation(S.generators,S.complement)' is a basis of $Y(\bT)$
(equivalently  $M\in\text{GL}(\Z^{\text{rank}(W)})$. An error  is raised if
<Y> does not define a pure sublattice.

|    gap> W:=CoxeterGroup("A",4);;
    gap> SubTorus(W,[[1,2,3,4],[2,3,4,1],[3,4,1,2]]);
    Error, not a pure sublattice in
    SubTorus( W, [ [ 1, 2, 3, 4 ], [ 2, 3, 4, 1 ], [ 3, 4, 1, 2 ] ]
     ) called from
    main loop
    brk>
    gap> SubTorus(W,[[1,2,3,4],[2,3,4,1],[3,4,1,1]]);
    SubTorus(CoxeterGroup("A",4),[ [ 1, 0, 3, -13 ], [ 0, 1, 2, 7 ], [ 0,
    0, 4, -3 ] ])|

%%%%%%%%%%%%%%%%%%%%%%%%%%%%%%%%%%%%%%%%%%%%%%%%%%%%%%%%%%%%%%%%%%%%%%%%%
\Section{Operations for Subtori}

The operation 'in' can test if a semisimple element belongs to a subtorus\:

|    gap> W:=RootDatum("gl",4);;
    gap> r:=AlgebraicCentre(W);
    rec(
      Z0 := SubTorus(RootDatum("gl",4),[ [ 1, 1, 1, 1 ] ]),
      AZ := Group( <0,0,0,0> ),
      descAZ := [ [ 1 ] ] )
    gap> SemisimpleElement(W,[1/4,1/4,1/4,1/4]) in r.Z0;
    true|

The operation 'Rank' gives the rank of the subtorus\:

|    gap> Rank(r.Z0);
    1|

%%%%%%%%%%%%%%%%%%%%%%%%%%%%%%%%%%%%%%%%%%%%%%%%%%%%%%%%%%%%%%%%%%%%%%%%%
\Section{AlgebraicCentre}
\index{AlgebraicCentre}

'AlgebraicCentre( <W> )'

<W>  should be a Weyl group record,  or an extended Weyl group record. This
function  returns a  description of  the centre  $Z$ of the algebraic group
defined by <W> as a record with the following fields\:

'Z0':  the neutral component $Z^0$  of $Z$ as a subtorus of $\bT$.

'AZ':  representatives  of  $A(Z)\:=Z/Z^0$  given  as a group of semisimple
elements.

'descAZ':  if <W> is not an extended Weyl group, describes the inclusion of
$A(Z)$  in the center 'pi' of  the corresponding simply connected group. It
contains  a list elements  given as words  in the generators  of 'pi' which
generate $A(Z)$.

|    gap> G:=CoxeterGroup("A",3,"sc");;
    gap> L:=ReflectionSubgroup(G,[1,3]);
    ReflectionSubgroup(CoxeterGroup("A",3,"sc"), [ 1, 3 ])
    gap> AlgebraicCentre(L);
    rec(
      Z0 :=
       SubTorus(ReflectionSubgroup(CoxeterGroup("A",3,"sc"), [ 1, 3 ]),[ [\
     1, 2, 1 ] ]),
      AZ := Group( <0,0,1/2> ),
      descAZ := [ [ 1, 2 ] ] )
    gap> G:=CoxeterGroup("A",3);;
    gap> s:=SemisimpleElement(G,[0,1/2,0]);;
    gap> Centralizer(G,s);
    (A1xA1)<1,3>.(q+1)
    gap> AlgebraicCentre(last);
    rec(
      Z0 := SubTorus(ReflectionSubgroup(CoxeterGroup("A",3), [ 1, 3 ]),),
      AZ := Group( <0,1/2,0> ) )|

Note  that in versions of \CHEVIE\ prior  to april 2017, the field 'Z0' was
not a  subtorus  but  what  is  now  'Z0.generators', and there was a field
'complement' which is now 'Z0.complement'.

%%%%%%%%%%%%%%%%%%%%%%%%%%%%%%%%%%%%%%%%%%%%%%%%%%%%%%%%%%%%%%%%%%%%%%%%%
\Section{SemisimpleSubgroup}
\index{SemisimpleSubgroup}

'SemisimpleSubgroup( <S>, <n> )'

This  function  returns  the  subgroup  of  semi-simple  elements  of order
dividing  <n>  in  the  subtorus  $S$.

|    gap> G:=CoxeterGroup("A",3,"sc");;
    gap> L:=ReflectionSubgroup(G,[1,3]);;
    gap> z:=AlgebraicCentre(L);;
    gap> z.Z0;
    SubTorus(ReflectionSubgroup(CoxeterGroup("A",3,"sc"), [ 1, 3 ]),[ [ 1,\
     2, 1 ] ])
    gap> SemisimpleSubgroup(z.Z0,3);
    Group( <1/3,2/3,1/3> )
    gap> Elements(last);
    [ <0,0,0>, <1/3,2/3,1/3>, <2/3,1/3,2/3> ]|

%%%%%%%%%%%%%%%%%%%%%%%%%%%%%%%%%%%%%%%%%%%%%%%%%%%%%%%%%%%%%%%%%%%%%%%%%
\Section{IsIsolated}
\index{IsIsolated}

'IsIsolated(<W>,<s>)'

<s> should be a semi-simple element for the algebraic group $\bG$ specified
by  the Weyl  group record  <W>. A  semisimple element  <s> of an algebraic
group  $\bG$ is isolated  if the connected  component $C_\bG^0(s)$ does not
lie  in  a  proper  parabolic  subgroup  of $\bG$. This function tests this
condition.

|    gap> W:=CoxeterGroup("E",6);;
    gap> QuasiIsolatedRepresentatives(W);
    [ <0,0,0,0,0,0>, <0,0,0,1/3,0,0>, <0,1/6,1/6,0,1/6,0>,
      <0,1/2,0,0,0,0>, <1/3,0,0,0,0,1/3> ]
    gap> Filtered(last,x->IsIsolated(W,x));
    [ <0,0,0,0,0,0>, <0,0,0,1/3,0,0>, <0,1/2,0,0,0,0> ]|

%%%%%%%%%%%%%%%%%%%%%%%%%%%%%%%%%%%%%%%%%%%%%%%%%%%%%%%%%%%%%%%%%%%%%%%%%
\Section{IsQuasiIsolated}
\index{IsQuasiIsolated}

'IsQuasiIsolated(<W>,<s>)'

<s> should be a semi-simple element for the algebraic group $\bG$ specified
by  the Weyl  group record  <W>. A  semisimple element  <s> of an algebraic
group  $\bG$  is  quasi-isolated  if  $C_\bG(s)$  does  not lie in a proper
parabolic subgroup of $\bG$. This function tests this condition.

|    gap> W:=CoxeterGroup("E",6);;
    gap> QuasiIsolatedRepresentatives(W);
    [ <0,0,0,0,0,0>, <0,0,0,1/3,0,0>, <0,1/6,1/6,0,1/6,0>,
      <0,1/2,0,0,0,0>, <1/3,0,0,0,0,1/3> ]
    gap> Filtered(last,x->IsQuasiIsolated(ReflectionSubgroup(W,[1,3,5,6]),x));
    [ <0,0,0,0,0,0>, <0,0,0,1/3,0,0>, <0,1/2,0,0,0,0> ]|

%%%%%%%%%%%%%%%%%%%%%%%%%%%%%%%%%%%%%%%%%%%%%%%%%%%%%%%%%%%%%%%%%%%%%%%%%
\Section{QuasiIsolatedRepresentatives}
\index{QuasiIsolatedRepresentatives}

'QuasiIsolatedRepresentatives(<W>[,<p>])'

<W>  should  be  a  Weyl  group  record corresponding to an algebraic group
$\bG$. This function returns a list of semisimple elements for $\bG$, which
are  representatives  of  the  $\bG$-orbits  of  quasi-isolated  semisimple
elements.  It  follows  the  algorithm  given  by C. Bonnaf{\accent19 e} in
\cite{Bon05}.  If a second argument <p>  is given, it gives representatives
of those quasi-isolated elements which exist in characteristic <p>.

|    gap> W:=CoxeterGroup("E",6);;QuasiIsolatedRepresentatives(W);
    [ <0,0,0,0,0,0>, <0,0,0,1/3,0,0>, <0,1/6,1/6,0,1/6,0>,
      <0,1/2,0,0,0,0>, <1/3,0,0,0,0,1/3> ]
    gap> List(last,x->IsIsolated(W,x));
    [ true, true, false, true, false ]
    gap> W:=CoxeterGroup("E",6,"sc");;QuasiIsolatedRepresentatives(W);
    [ <0,0,0,0,0,0>, <1/3,0,2/3,0,1/3,2/3>, <1/2,0,0,1/2,0,1/2>,
      <2/3,0,1/3,0,1/3,2/3>, <2/3,0,1/3,0,2/3,1/3>, <2/3,0,1/3,0,2/3,5/6>,
      <5/6,0,2/3,0,1/3,2/3> ]
    gap> List(last,x->IsIsolated(W,x));
    [ true, true, true, true, true, true, true ]
    gap> QuasiIsolatedRepresentatives(W,3);
    [ <0,0,0,0,0,0>, <1/2,0,0,1/2,0,1/2> ]|

%%%%%%%%%%%%%%%%%%%%%%%%%%%%%%%%%%%%%%%%%%%%%%%%%%%%%%%%%%%%%%%%%%%%%%%%%
\Section{SemisimpleCentralizerRepresentatives}
\index{SemisimpleCentralizerRepresentatives}

'SemisimpleCentralizerRepresentatives(<W> [,<p>])'

<W>  should  be  a  Weyl  group  record corresponding to an algebraic group
$\bG$.  This  function  returns  a  list  giving  representatives  $\bH$ of
$\bG$-orbits  of reductive  subgroups of  $\bG$ which  can be  the identity
component  of the centralizer of a  semisimple element. Each group $\bH$ is
specified  by  a  list  <h>  of  reflection  indices in <W> such that $\bH$
corresponds  to  'ReflectionSubgroup(W,h)'.  If  a  second  argument <p> is
given,  only the list of the centralizers which occur in characteristic <p>
is returned.

|    gap> W:=CoxeterGroup("G",2);
    CoxeterGroup("G",2)
    gap> l:=SemisimpleCentralizerRepresentatives(W);
    [ [  ], [ 1 ], [ 1, 2 ], [ 1, 5 ], [ 2 ], [ 2, 6 ] ]
    gap> List(last,h->ReflectionName(ReflectionSubgroup(W,h)));
    [ "(q-1)^2", "A1.(q-1)", "G2", "A2<1,5>", "~A1<2>.(q-1)",
      "~A1<2>xA1<6>" ]
    gap> SemisimpleCentralizerRepresentatives(W,2);
    [ [  ], [ 1 ], [ 1, 2 ], [ 1, 5 ], [ 2 ] ]|

%%%%%%%%%%%%%%%%%%%%%%%%%%%%%%%%%%%%%%%%%%%%%%%%%%%%%%%%%%%%%%%%%%%%%%%%%
