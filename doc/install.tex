%%%%%%%%%%%%%%%%%%%%%%%%%%%%%%%%%%%%%%%%%%%%%%%%%%%%%%%%%%%%%%%%%%%%%%%%%%%%%
%%
%A  install.tex                 GAP documentation            Martin Schoenert
%%
%H  @(#)$Id: install.tex,v 1.2 1997/04/03 16:06:59 gap Exp $
%%
%Y  Copyright (C)  1993,  Lehrstuhl D fuer Mathematik,  RWTH Aachen,  Germany
%%
%%  This file describes the installation and features of \GAP\ on various OS.
%%
%%  Currently UNIX, IBM PC compatibles under   Windows, or  OS/2, and
%%  the Atari ST under TOS on the Atari ST are available.  We hope to provide
%%  ports to DEC  VAX or  AXP systems under VMS and the Apple Macintosh under
%%  MPW soon.
%%
%%  Modified Jean Michel 2016 to reflect gap3-jm distribution
%%
%%  To read about the  system you are   interested in, search for  the string
%%  'Section{GAP for <system>}', e.g., 'Section{GAP for UNIX}'.
%%
%H  Revision 1.2  1997/04/03 16:06:59  Steve
%H
%H  Revision 1.1.1.1  1996/12/11 12:36:46  werner
%H  Preparing 3.4.4 for release
%H
%H  Revision 3.19  1994/07/08  09:52:27  vfelsch
%H  updated release number
%H
%H  Revision 3.18  1994/03/23  09:43:05  fceller
%H  changed internet address of pell
%H
%H  Revision 3.17  1993/11/09  00:08:10  martin
%H  updated installation descriptions
%H
%H  Revision 3.16  1993/05/06  11:35:47  fceller
%H  fixed a misspelling
%H
%H  Revision 3.15  1993/04/15  18:03:42  fceller
%H  added 'dehn.mth.pdx.edu'
%H
%H  Revision 3.14  1993/02/19  11:30:42  gap
%H  removed overfull hboxes
%H
%H  Revision 3.13  1993/02/19  10:48:42  gap
%H  adjustments in line length and spelling
%H
%H  Revision 3.12  1993/02/18  07:55:21  felsch
%H  some misprints removed
%H
%H  Revision 3.11  1993/02/17  15:07:32  felsch
%H  example fixed
%H
%H  Revision 3.10  1993/02/16  15:06:04  fceller
%H  fixed example, changed description of 'gapexe.next'
%H
%H  Revision 3.9  1993/02/11  15:51:53  martin
%H  updated for 3.2
%H
%H  Revision 3.8  1992/05/25  18:09:39  martin
%H  added "Upgrading GAP"
%H
%H  Revision 3.7  1992/05/22  08:58:45  martin
%H  added the sections for IBM PC compatibles
%H
%H  Revision 3.6  1992/04/09  11:36:01  martin
%H  made a few changes so that two LaTeX passes suffice
%H
%H  Revision 3.5  1992/04/07  22:48:42  martin
%H  fixed the examples
%H
%H  Revision 3.4  1992/04/02  18:03:40  martin
%H  added the banner
%H
%H  Revision 3.3  1992/03/25  15:49:27  martin
%H  changed the output of 'ls <gap-dir>/doc'
%H
%H  Revision 3.2  1992/03/23  12:37:54  martin
%H  added "Getting GAP"
%H
%H  Revision 3.1  1992/03/23  11:58:51  martin
%H  improved the installation sections
%H
%H  Revision 3.0  1991/12/27  16:10:27  martin
%H  initial revision under RCS
%H
%%
\Chapter{Getting and Installing GAP}%
\index{installation}\index{options}

{\GAP}  runs on several  different operating systems.  It behaves slightly
different  on  each  of  those.  This  chapter  describes  the behaviour of
{\GAP},  the  installation,  and  the  options  on some of those operating
systems.

Currently   it  contains  sections   for  *UNIX*  (see   "GAP  for  UNIX"),
*WINDOWS*  (see  "GAP  for  Windows"),  and  *Mac/OSX*  (see "GAP for
Mac/OSX").

For other systems the section "Porting GAP" gives hints how to
approach such a port.

%%%%%%%%%%%%%%%%%%%%%%%%%%%%%%%%%%%%%%%%%%%%%%%%%%%%%%%%%%%%%%%%%%%%%%%%%
\Section{Getting GAP}%
\index{ftp}

{\GAP}  is distributed  *free of  charge*. You  can give  it away  to your
colleagues.  {\GAP} is *not* in the  public domain, however. In particular
you  are  not  allowed  to  incorporate  {\GAP}  or  parts  thereof into a
commercial product.

This distribution of {\GAP} is maintained by Jean Michel,
'jean.michel@imj-prg.fr'.  I  would  appreciate  if  let  me know, e.g., by
sending  a  short  e-mail  message,  if  you  are using it. I also take bug
reports.

If you publish some result that was partly obtained using {\GAP}, we would
appreciate  it if you  would cite {\GAP},  just as you  would cite another
paper that you used. Specifically, please refer to |
[S+ 97] Martin Sch"onert et.al.  GAP -- Groups, Algorithms, and Programming.
        Lehrstuhl D f"ur Mathematik, Rheinisch Westf"alische Technische
        Hochschule, Aachen, Germany, sixth edition, 1997.
| Again we would appreciate if you could inform us about such a paper.

This  distribultion contains *full  source* for everything,  the C code for
the  kernel, the {\GAP} code for the library, and the {\LaTeX} code for the
manual,  which has at present about 1900  pages. So it should be no problem
to  get {\GAP}, even if you have a rather uncommon system. Of course, ports
to  non  UNIX  systems  may  require  some  work. We already have ports for
MS-DOS/Windows,  and Apple Mac. Note that about 50 MByte of main memory and
about  100MB  of  disk  space  are  required  to  run {\GAP}. A full {\GAP}
installation,  including all share  packages and data  libraries uses up to
100MB of disk space.

The  easiest way  to get  this {\GAP}  distribution is  to download it from
|http://webusers.imj-prg.fr/~jmichel/gap3|.

The original site for the distribution is
|http://www-gap.dcs.st-and.ac.uk/~gap|, but it now distributes {\GAP4}.

At  |http://webusers.imj-prg.fr/~jmichel/gap3| you  can browse  this manual
and download the system.

%%%%%%%%%%%%%%%%%%%%%%%%%%%%%%%%%%%%%%%%%%%%%%%%%%%%%%%%%%%%%%%%%%%%%%%%%
\Section{GAP for UNIX}%

{\GAP}  runs best under UNIX.  In fact it has  being developed under UNIX.
{\GAP}  running on any UNIX machine  should behave exactly as described in
the manual.

The  section  "Installation  of  GAP  for  UNIX"  describes how you install
{\GAP}  on a  UNIX machine,  and the  section "Features  of GAP  for UNIX"
describe the options that {\GAP} accepts under UNIX.

%%%%%%%%%%%%%%%%%%%%%%%%%%%%%%%%%%%%%%%%%%%%%%%%%%%%%%%%%%%%%%%%%%%%%%%%%
\Section{Installation of GAP for UNIX}%
\index{installation!under UNIX}\index{UNIX!installation}

Installation of {\GAP} for UNIX is fairly easy.

First go to the directory where you want to install {\GAP}. If you will be
the  only  user  using  {\GAP},  you  probably  should  install  it in you
homedirectory.  If  other  users  will  be  using  {\GAP} also, you should
install  it in a  public place, such  as '/usr/local/lib/'. {\GAP} will be
installed in a subdirectory 'gap3-jm' of this directory. You can later move
{\GAP}  to a different location.  For example you can  first install it in
your  homedirectory and when it works  move it to '/usr/local/lib/'. In the
following  example we will assume that you want to install {\GAP} for your
own use in your homedirectory. Note that certain parts of the output in the
examples  should only be taken as  rough outline, especially file sizes and
file dates are *not* to be taken literally.

Get  the distribution  'gap3-jmxxx.tar.gz' where  'xxx' is  the date of the
version  you are downloading (like  'gap3-jm19feb18.tar.gz' for the version
released  on 30 november 2016). Unpack this archive in the chosen directory
with the command

|   tar -xvzf gap3-jm19feb18.tar.gz|

This  will create a |gap3-jm| directory containing the {\GAP} distribution.
Then  edit  the  shell  script  'gap.sh'  in  the  'gap3-jm/bin'  directory
according  to the instructions in this file (the main thing to do is to set
the  variable |GAP_DIR| to the directory  where you installed {\GAP}). Then
copy  this script to a directory in  your search path, i.e., '~/bin/'. This
script will start {\GAP}.

If  there is no executable in the  |bin| directory matching your system, it
means  you  are  attempting  a  new  port. Change into the source directory
'gap3-jm/src/'  and  execute  'make'  to  see which compilation targets are
predefined.  Choose the best  matching target. There  is a good chance that
'linux' or 'linux32' will do the job, otherwise create a new target.

Now  start {\GAP}  and try  a few  things. The  '-b' option suppresses the
banner.  Note that {\GAP} has  to read most of  the library for the fourth
statement  below.

|    you@ernie:~ > gap -b
    gap> 2 * 3 + 4;
    10
    gap> Factorial( 30 );
    265252859812191058636308480000000
    gap> Factors( 10^42 + 1 );
    [ 29, 101, 281, 9901, 226549, 121499449, 4458192223320340849 ]
    gap> m11 := Group((1,2,3,4,5,6,7,8,9,10,11),(3,7,11,8)(4,10,5,6));;
    gap> Size( m11 );
    7920
    gap> Factors( 7920 );
    [ 2, 2, 2, 2, 3, 3, 5, 11 ]
    gap> Number( ConjugacyClasses( m11 ) );
    10 |

Especially try the command line editing and history facilities, because the
are  probably the  most machine  dependent feature  of {\GAP}. Enter a few
commands  and then  make sure  that <ctr>-'P'  redisplays the last command,
that  <ctr>-'E' moves  the cursor  to the  end of  the line, that <ctr>-'B'
moves the cursor back one character, and that <ctr>-'D' deletes single
characters.  So, after entering the above commands, typing\\
<ctr>-'P' <ctr>-'P' <ctr>-'E' <ctr>-'B' 
<ctr>-'B' <ctr>-'B' <ctr>-'B' <ctr>-'D' '1'\\
should give the following line.

|    gap> Factors( 7921 );
    [ 89, 89 ] |

If  command line editing does not work,  remove the file 'system.o' and try
to  compile with a different target, i.e., 'bsd' instead of 'linux' or vice
versa.  If neither works, we suggest  that you disable command line editing
by  calling {\GAP} always with the '-n'  option. In any case we would like
to hear about such problems.

If your operating  system has job control, make  sure that  you can still
stop {\GAP}, which is usually done by pressing <ctr>-'Z'.

If  you  want  to  redo  the  manual  after  some changes, you need to have
{\LaTeX}  installed, and |ruby| to make the |html| version. Go to the 'doc'
subdirectory  of  |gap3-jm|  and  do  |make  doc|.  This  will  make  files
|gap3-jm/doc/manual.dvi|, |gap3-jm/manual.pdf| and |gap3-jm/htm/chap*.htm|.
To  remove the unneeded  |.aux| files, you  can excute |make  clean| in the
|doc|  directory. The  full manual  is, to  put it  mildly, now rather long
(1900  pages). 

Thats all, finally you are done. We hope that you will enjoy using {\GAP}.
If you have problems, do not hesitate to contact us.

%%%%%%%%%%%%%%%%%%%%%%%%%%%%%%%%%%%%%%%%%%%%%%%%%%%%%%%%%%%%%%%%%%%%%%%%%
\Section{Features of GAP for UNIX}%
\index{features!under UNIX}\index{UNIX!features}%
\index{options!under UNIX}\index{UNIX!options}%
\index{.gaprc}

When you start {\GAP}  for UNIX, you may  specify a number of  options on
the command-line to change  the default behaviour  of {\GAP}.  All  these
options  start with a  hyphen '-', followed by a  single letter.  Options
must not be   grouped, e.g.,  'gap  -gq' is   illegal, use   'gap -g  -q'
instead.  Some options  require an argument, this  must follow the option
and  must be separated   by a <space>, e.g.,  'gap   -m 256k', it is  not
correct to say 'gap -m256k' instead.

{\GAP} for UNIX will only accept lower case options.

As  is  described  in  the  previous  section (see "Installation of GAP for
UNIX") usually you will not execute {\GAP} directly. Instead you will call
a  shell script, with the name 'gap',  which in turn executes {\GAP}. This
shell  script sets some options  as necessary to make  {\GAP} work on your
system.  This means  that the  default settings  mentioned below may not be
what you experience when you execute {\GAP} on your system.

'-g'

The  option '-g' tells {\GAP} to print an information message every time a
garbage collection is performed.

|#G  collect garbage, 1567022 used,  412991 dead,  84.80MB free, 512MB total|

For  example,  this  tells  you  that  there  are 1567022 live objects that
survived a garbage collection, that 412991 unused objects were reclaimed by
it,  and  that  84  MBytes  of  totally  allocated 512 MBytes are available
afterwards.

'-l <libname>'
The  option '-l' tells {\GAP} that the  library of {\GAP} functions is in
the directory <libname>. Per default <libname> is 'lib/', i.e., the library
is  normally expected in the subdirectory  'lib/' of the current directory.
{\GAP}  searches for the  library files, whose  filenames end in '.g', and
which  contain the functions initially known to {\GAP}, in this directory.
<libname> should end with a pathname separator, i.e., '/', but {\GAP} will
silently   add  one  if   it  is  missing.   {\GAP}  will  read  the  file
'<libname>/init.g' during startup. If {\GAP} cannot find this file it will
print the following warning

|    gap: hmm, I cannot find 'lib/init.g', maybe use option '-l <lib>'?|

If  you want  a bare  bones {\GAP},  i.e., if  you do not need any library
functions,  you may ignore this warning, otherwise you should leave {\GAP}
and  start it  again, specifying  the correct  library path  using the '-l'
option.

It  is  also  possible  to  specify  several  alternative  library paths by
separating  them with semicolons ';'. Note that in this case all path names
must  end with the pathname separator '/'. {\GAP} will then search for its
library files in all those directories in turn, reading the first it finds.
E.g.,  if you specify  |-l "lib/;/usr/local/lib/gap3-jm/lib/"| {\GAP} will
search  for a library file first in  the subdirectory 'lib/' of the current
directory,   and  if   it  does   not  find   it  there  in  the  directory
|/usr/local/lib/gap3-jm/lib/|. This way you can built your own directory of
{\GAP} library files that override the standard ones.

{\GAP}  searches for the  group files, whose  filenames end in '.grp', and
which  contain the groups initially known  to {\GAP}, in the directory one
gets  by replacing the string 'lib' in  <libname> with the string 'grp'. If
you  do not want to put the group directory 'grp/' in the same directory as
the  'lib/'  directory,  for  example  if  you  want to put the groups onto
another   hard  disk  partition,  you  have   to  edit  the  assignment  in
'<libname>/init.g' that reads

|    GRPNAME := ReplacedLib( "grp" );|

This  path  can  also  consist  of  several  alternative paths, just as the
library path. If the library path consists of several alternative paths the
default  value for this path will consist  of the same paths, where in each
component the last occurrence of 'lib/' is replaced by 'grp/'.

Similar  considerations apply to the character table files. Those filenames
end  in '.tbl'.  {\GAP} looks  for those  files in  the directory given by
'TBLNAME'.  The default value for 'TBLNAME'  is obtained by replacing 'lib'
in <libname> with 'tbl'.

'-h <docname>'

The option '-h' tells {\GAP} that the on-line documentation for {\GAP} is
in  the directory <docname>. Per default <docname> is obtained by replacing
'lib'  in  <libname>  with  'doc'.  <docname>  should  end  with a pathname
separator,  i.e., '/', but {\GAP} will silently  add one if it is missing.
{\GAP}  will read the  file '<docname>/manual.toc' when  you first use the
help  system. If {\GAP} cannot find this  file it will print the following
warning

|    help: hmm, I cannot open the table of contents file 'doc/manual.toc'
    maybe you should use the option '-h <docname>'?|

'-m <memory>'

The  option '-m' tells {\GAP} to  allocate <memory> bytes at startup time.
If  the last character of <memory> is 'k'  or 'K' it is taken in KBytes and
if the last character is 'm' or 'M' <memory> is taken in MBytes.

Under  UNIX the default amount  of memory allocated by  {\GAP} is 4 MByte.
The  amount of memory  should be large  enough so that  computations do not
require  too  many  garbage  collections.  On  the  other  hand  if {\GAP}
allocates  more virtual memory  than is physically  available it will spend
most of the time paging.

'-n'

The option  '-n'  tells {\GAP}  to disable  the line editing  and history
(see "Line Editing").

You  may want to do  this if the command  line editing is incompatible with
another  program that is used to run {\GAP}. For example if {\GAP} is run
from  inside a GNU Emacs shell window,  '-n' should be used since otherwise
every input line will be echoed twice, once by Emacs and once by {\GAP}.

'-b'

The  option  '-b'  tells  {\GAP}  to  suppress the banner. That means that
{\GAP} immediately prints the prompt. This is useful when you get tired of
the banner after a while.

'-q'

The option '-q' tells {\GAP} to be quiet. This means that {\GAP} does not
display  the banner and the  prompts 'gap>'. This is  useful if you want to
run {\GAP} as a filter with input and output redirection and want to avoid
the the banner and the prompts clobbering the output file.

'-e'

The  option '-e' tells {\GAP} not to  act on 'ctrl-D'. This means that you
have  to type  explicitely 'quit;'  to exit  error loops  or {\GAP} at the
prompt.  This  may  be  useful  if  you  find  'ctrl-D' too easy to type by
accident.

'-x <length>'

With this option you can tell {\GAP} how long lines are. {\GAP} uses this
value to decide when to split long lines. The  default value is 80.

'-y <length>'

With  this option  you can  tell {\GAP}  how many  lines your  screen has.
{\GAP}  uses this value to decide after  how many lines of on-line help it
should display |-- <space> for more --|. The default value is 24.

{\GAP}   does  not   read  the   variables  specifying   the  screen  size
automatically. On most shells you can tell {\GAP} by giving the options:

|-x $COLUMNS -y $LINES|

Further  arguments are taken as filenames of files that are read by {\GAP}
during  startup,  after  '<libname>/init.g'  is  read, but before the first
prompt  is printed. The files are read in  the order in that they appear on
the command line. {\GAP} only accepts 14 filenames on the command line. If
a file cannot be opened {\GAP} will print an error message and will abort.

When  you start {\GAP},  it looks for  the file with  the name '.gaprc' in
your   homedirectory.  If   such  a   file  is   found  it  is  read  after
'<libname>/init.g',  but before any  of the files  mentioned on the command
line  are read. You can use this  file for your private customizations. For
example,  if you have a  file containing functions or  data that you always
need,  you could read this from '.gaprc'. Or  if you find some of the names
in  the library too long, you could define abbreviations for those names in
'.gaprc'. The following sample '.gaprc' file does both.

|    Read("/usr/you/dat/mygroups.grp");
    Op := Operation;
    OpHom := OperationHomomorphism;
    RepOp := RepresentativeOperation;
    RepsOp := RepresentativesOperation; |

'-r'

The option '-r' tells {\GAP} to ignore a '.gaprc' file. This may be useful
to see if a problem may be caused by the contents of your '.gaprc' file.

%%%%%%%%%%%%%%%%%%%%%%%%%%%%%%%%%%%%%%%%%%%%%%%%%%%%%%%%%%%%%%%%%%%%%%%%%
\Section{GAP for Windows}%
\index{GAP for Windows}\index{Windows}

This  sections contain information about  {\GAP} that is  specific to the
port  of {\GAP} for IBM PC   compatibles under Windows (simply
called {\GAP} for Windows below).

To run  {\GAP} for Windows  you  need an IBM PC   compatible with an Intel
80386, Intel 80486, or Intel Pentium processor, it will not run on IBM PC
compatibles with  an Intel 80186  or  Intel 80286 processor.   The system
must have at least 4 MByte of main  memory and a harddisk.  The operating
system  must be  Windows version  5.0  or later  or Windows  3.1  or later
(earlier versions may work, but this has not been tested).

The section "Copyright of GAP  for Windows" describes  the copyright as it
applies  to the  executable version  that  we  distribute.  The   section
"Installation  of GAP  for Windows"  describes how you  install {\GAP} for
Windows, and  the section "Features    of GAP for  Windows" describes   the
special features of {\GAP} for Windows.

%%%%%%%%%%%%%%%%%%%%%%%%%%%%%%%%%%%%%%%%%%%%%%%%%%%%%%%%%%%%%%%%%%%%%%%%%
\Section{Copyright of GAP for Windows}

In  addition  to the  general copyright   for  {\GAP} set  forth  in  the
Copyright the following terms apply to {\GAP} for Windows.

The system  dependent  part for {\GAP}  for Windows  was  written by Steve
Linton.  He assigns   the copyright to the  Lehrstuhl  D fuer Mathematik.
Many thanks to Steve Linton for his work.

The executable of {\GAP} for Windows that  we distribute was compiled with
DJ Delorie\'s port   of the Free Software  Foundation\'s  GNU C  compiler
version 2.7.2.   The compiler can be obtained  by  anonymous 'ftp' from a
variety of general public FTP archives.  Many thanks to the Free Software
Foundation and DJ Delorie for this amazing piece of work.

The GNU C compiler is

*Copyright (C) 1989, 1991 Free Software Foundation, Inc.
675 Mass Ave, Cambridge, MA 02139, USA*

under the  terms  of the GNU General Public License (GPL).  Note that the
GNU  GPL  states  that the  mere act  of compiling  does not  affect  the
copyright status of {\GAP}.

The modifications to the compiler to make it operating under  Windows, the
functions from  the standard library 'libpc.a', the  modifications of the
functions from the standard library  'libc.a' to make  them operate under
Windows, and the DOS extender  'go32' (which is prepended to 'gapexe.386')
are

*Copyright (C) 1991 DJ Delorie,
24 Kirsten Ave, Rochester NH 03867-2954, USA*

also under the terms of  the GNU GPL.  The  terms of the GPL require that
we make the source code for 'libpc.a' available.  They can be obtained by
writing to Steve Linton (however, it may be easier for  you to 'ftp' them
from 'grape.ecs.clarkson.edu' yourself).  They  also require  that {\GAP}
falls under the GPL too, i.e.,  is distributed freely, which it basically
does anyhow.

The functions in 'libc.a' that {\GAP} for the 386 uses are

*Copyright (c) 1988 Regents of the University of California*

under the following terms

*All rights reserved.*

*Redistribution and use in source and binary forms are permitted provided
that the above copyright notice and this paragraph are  duplicated in all
such forms and that  any documentation, advertising  materials, and other
materials  related to  such  distribution and  use acknowledge  that  the
software  was developed by  the University of California, Berkeley.   The
name  of the University  may not be  used to endorse or promote  products
derived from this software without specific prior written permission.*

*THIS  SOFTWARE  IS  PROVIDED AS IS  AND WITHOUT  ANY EXPRESS OR  IMPLIED
WARRANTIES,  INCLUDING,  WITHOUT  LIMITATION, THE  IMPLIED WARRANTIES  OF
MERCHANTIBILITY AND FITNESS FOR A PARTICULAR PURPOSE.*

%%%%%%%%%%%%%%%%%%%%%%%%%%%%%%%%%%%%%%%%%%%%%%%%%%%%%%%%%%%%%%%%%%%%%%%%%
\Section{Installation of GAP for Windows}%
\index{installation!under Windows}\index{Windows!installation}
\index{installation!under Windows}\index{Windows!installation}

Installation of {\GAP} on WIndows is similar to Unix. 

First go to a directory where  you  want to install  {\GAP}, e.g., |c:\|.
{\GAP} will be installed in a subdirectory |gap3-jm\| of this directory.
You can later move {\GAP} to another location, for  example you can first
install it in  |d:\tmp\| and once  it works move it  to  |c:\|.   In  the
following  example we assume that  you want  to install {\GAP} in  |c:\|.
Note that  certain parts of  the output in the examples  should  only  be
taken as rough outline, especially file sizes and file dates are *not* to
be taken literally.

Get  the {\GAP} distribution onto your IBM PC  compatible --- see the Unix
instructions how to get it from the web.

Change into the directory |gap-jm\bin\| and  edit the script 'gap.cmd',
which starts {\GAP}, according  to the instructions in  this  file.  Then
copy  this  script to a  directory in your  search path, e.g., |c:\bin\|,
with the commands

|    C: > chdir gap-jm\bin
    C:\GAPR4P4\BIN > edit gap.cmd
    # edit the script 'gap.cmd'
    C:\GAPR4P4\BIN > copy gap.cmd c:\bin\gap.cmd
    C:\GAPR4P4\BIN > chdir ..\..
    C: > |

When you later move {\GAP} to another  location  you must only  edit this
script.

An alternative possibility  is  to compile a   version of {\GAP} for  use
under Windows, on a UNIX system, using a *cross-compiler*. Cross-compiling
versions  of   'gcc' can be  found  on  some FTP    archives, or compiled
according to    the  instructions   supplied   with  the  'gcc'    source
distribution.

Start {\GAP} and try a few things.  Note that {\GAP} has  to read most of
the library for the fourth statement below,  so this takes quite a while.
Subsequent definitions of groups will be much faster.

|    C: > gap -b
    gap> 2 * 3 + 4;
    10
    gap> Factorial( 30 );
    265252859812191058636308480000000
    gap> Factors( 10^42 + 1 );
    [ 29, 101, 281, 9901, 226549, 121499449, 4458192223320340849 ]
    gap> m11 := Group((1,2,3,4,5,6,7,8,9,10,11),(3,7,11,8)(4,10,5,6));;
    gap> Size( m11 );
    7920
    gap> Factors( 7920 );
    [ 2, 2, 2, 2, 3, 3, 5, 11 ]
    gap> Number( ConjugacyClasses( m11 ) );
    10 |

Especially try the command  line editing  and history facilities, because
the are probably the most  machine dependent feature  of {\GAP}.  Enter a
few commands   and then make  sure  that  <ctr>-'P' redisplays   the last
command, that  <ctr>-'E'  moves the cursor  to the end  of the line, that
<ctr>-'B' moves the cursor back one character, and that <ctr>-'D' deletes
single characters.  So after entering the above three commands typing\\
<ctr>-'P' <ctr>-'P' <ctr>-'E' <ctr>-'B'
<ctr>-'B' <ctr>-'B' <ctr>-'B' <ctr>-'D' '1'\\
should give the following line.

|    gap> Factors( 7921 );
    [ 89, 89 ] |

If you have a big version of {\LaTeX} available  you may now want to make
a printed copy of the  manual.  Change into the directory |gap3-jm\doc\|
and  run {\LaTeX} *twice* on the   source.  The first  pass with {\LaTeX}
produces the '.aux'  files, which resolve  all the cross references.  The
second pass produces  the final formatted  *dvi* file 'manual.dvi'.  This
will take quite a while, since the manual is large.  Then print the *dvi*
file.  How you actually print the 'dvi' file produced by {\LaTeX} depends
on the printer you  have, the version of {\LaTeX}  you have,  and whether
you use a  {\TeX}-shell  or not, so we   will not attempt to describe  it
here.

|    C: > chdir gap3-jm\doc
    C:\GAPR4P4\DOC > latex manual
    # about 2000 messages about undefined references
    C:\GAPR4P4\DOC > latex manual
    # there should be no warnings this time
    C:\GAPR4P4\DOC > dir manual.dvi
    -a---   4591132 Nov 13 23:29 manual.dvi
    C:\GAPR4P4\DOC > chdir ..\..
    C: > |

Note that because of the large  number of  cross references in the manual
you need a *big*  {\LaTeX} to format the {\GAP}  manual.  If you  see the
error message  'TeX capacity  exceeded', you do  not have a big {\LaTeX}.
In this case  you  may also  obtain the  already    formatted *dvi*  file
'manual.dvi' from  the  same place where   you obtained the   rest of the
{\GAP} distribution.

Note that, apart   from   the  '\*.tex' files  and the file  'manual.bib'
(bibliography database),  which you absolutely need, we  supply  also the
files 'manual.toc' (table of   contents), 'manual.ind' (unsorted  index),
'manual.idx' (sorted  index), and  'manual.bbl' (bibliography).  If those
files are missing, or  if you prefer  to do everything  yourself, here is
what you will have to do.  After  the first  pass with {\LaTeX}, you will
have  preliminary 'manual.toc' and  'manual.ind'  files.   All  the  page
numbers are  still incorrect, because the  do not account  for  the pages
used by  the table of contents itself.   Now 'bibtex  manual' will create
'manual.bbl' from 'manual.bib'.  After  the second pass with {\LaTeX} you
will  have a  correct   'manual.toc'  and 'manual.ind'.  'makeindex'  now
produces the sorted index 'manual.idx' from 'manual.ind'.  The third pass
with {\LaTeX} incorporates this index into the manual.

|    C: > chdir gap3-jm\doc
    # about 2000 messages about undefined references
    C:\GAPR4P4\DOC > bibtex manual
    # 'bibtex' prints the name of each file it is scanning
    C:\GAPR4P4\DOC > latex manual
    # still some messages about undefined citations
    C:\GAPR4P4\DOC > makeindex manual
    # 'makeindex' prints some diagnostic output
    C:\GAPR4P4\DOC > latex manual
    # there should be no warnings this time
    C:\GAPR4P4\DOC > chdir ..\..
    C: > |

The full   manual is, to put  it  mildly,  now  rather long  (almost 1600
pages).  For  this  reason, it  may  be  more convenient   just to *print
selected  chapters*. This can  be  done  using  the  |\includeonly| LaTeX
command,  which  is  present in    'manual.tex'  (around line 240),   but
commented out. To use  this, you must first  *LaTeX the whole  manual* as
normal, to  obtain  the complete set of   '.aux' files and determine  the
pages and numbers of  all the chapters  and sections. After that, you can
edit 'manual.tex' to uncomment  the |\includeonly| command and select the
chapters you want. A good start can be to include only the first chapter,
from  the    file  'aboutgap.tex',   by     editing the  line   to   read
|\includeonly{aboutgap}|.  The  next step is  to LaTeX  the manual again.
This time only the   selected chapter(s) and  the  table of contents  and
indices will be  processed, producing a  shorter 'dvi' file  that you can
print by whatever means applies locally.

|    C:\GAPR4P4\DOC > latex manual
    # many messages about undefined references, 1600 pages output	
    C:\GAPR4P4\DOC > edit manual.tex
    # edit line 241 to include only 'aboutgap'
    C:\GAPR4P4\DOC > latex manual
    # pages 0-196 and 1503-1553 only output no warnings
    C:\GAPR4P4\DOC > dir manual.dvi
    -a---   1291132 Nov 13 23:29 manual.dvi
    C:\GAPR4P4\DOC > 
    # now print the DVI file in whatever way is appropriate
|


Thats  all, finally you  are  done.  We  hope  that you  will enjoy using
{\GAP}.  If you have problems, do not hesitate to contact us.

%%%%%%%%%%%%%%%%%%%%%%%%%%%%%%%%%%%%%%%%%%%%%%%%%%%%%%%%%%%%%%%%%%%%%%%%%
\Section{Features of GAP for Windows}
\index{features!under Windows}\index{Windows!features}%
\index{options!under Windows}\index{Windows!options}%
\index{gap.rc}\index{.gaprc}

Note that {\GAP} for Windows  will use up to  128 MByte of extended memory
(using XMS, VDISK memory  allocation strategies) or  up  to 128  MByte of
expanded memory (using VCPI programs, such as QEMM  and 386MAX) and up to
128 MByte of disk space for swapping.

If you hit  <ctr>-'C'  the DOS extender  ('go32')  catches it and  aborts
{\GAP} immediately.  The keys <ctr>-'Z' and <alt>-'C' can be used instead
to interrupt {\GAP}.

The arrow keys <left>, <right>, <up>, <down>, <home>, <end>, and <delete>
can be used for command line editing with their intuitive meaning.

Pathnames may  be  given   inside {\GAP}  using  either  slash  ('/')  or
backslash (|\|) as a separator (though |\|  must be escaped in strings of
course).

When  you  start  {\GAP}  you  may specify a  number   of options  on the
command-line to  change the  default   behaviour of {\GAP}.   All   these
options start with a  hyphen '-', followed  by a single  letter.  Options
must not  be   grouped, e.g., 'gap -gq'    is illegal, use  'gap  -g  -q'
instead.  Some  options require an argument,  this must follow the option
and must   be separated by   a <space>, e.g., 'gap  -m  256k', it  is not
correct to say 'gap -m256k' instead.

{\GAP} for Windows accepts the following (lowercase) options.

'-g'

The options '-g' tells {\GAP} to print a information message every time a
garbage collection is performed.

|    #G collect garbage, 1931 used, 5012 dead, 912 KB free, 3072 KB total|

For example, this tells you that there are 1931 live objects that survived
a garbage collection, that 5012 unused objects  were reclaimed by it, and
that 912 KByte of totally allocated 3072 KBytes are available afterwards.

'-l <libname>'

The option '-l' tells {\GAP}  that the library  of {\GAP} functions is in
the  directory <libname>.  Per  default  <libname> is  'lib/', i.e.,  the
library  is normally expected in the subdirectory  'lib/'  of the current
directory.  {\GAP} searches for the library files, whose filenames end in
'.g', and which contain the functions initially known to  {\GAP}, in this
directory.   <libname> should end  with a  pathname separator, i.e., |\|,
but {\GAP} will silently add one  if it is missing.  {\GAP} will read the
file <libname>|\init.g| during startup.  If {\GAP} cannot find this  file
it will print the following warning

|    gap: hmm, I cannot find 'lib\init.g', maybe use option '-l <lib>'?|

If you  want a bare bones {\GAP},  i.e., if you do  not  need any library
functions, you may ignore this warning, otherwise you should leave {\GAP}
and start  it again, specifying the correct  library path  using the '-l'
option.

It is  also possible to  specify several  alternative library  paths  by
separating them  with semicolons ';'.   Note that in  this  case all path
names must end with the pathname separator  |\|.  {\GAP} will then search
for its library files in all those directories in turn, reading the first
it finds.  E.g., if  you specify |-l "lib\;\usr\local\lib\gap3-jm\lib\"|
{\GAP} will search for a library file first in the subdirectory |lib\| of
the current directory, and if it does not find it  there in the directory
|\usr\local\lib\gap3-jm\lib\|. This way you can built your own directory
of {\GAP} library files that override the standard ones.

{\GAP}  searches for the group files,  whose filenames end in '.grp', and
which contain the groups initially known to  {\GAP}, in the directory one
gets by replacing the string 'lib' in <libname> by the  string 'grp'.  If
you do not want  to put the group  directory |grp\| in the same directory
as the |lib\| directory,  for example if you want  to put the groups onto
another   hard  disk  partition,  you have  to   edit the   assignment in
<libname>|\init.g| that reads

|    GRPNAME := ReplacedString( LIBNAME, "lib", "grp" );|

This path  can also consist of  several  alternative  paths, just as the
library path.  If the library path consists of several alternative paths
the default value for this path will consist of the same paths, where in
each component the last occurrence of |lib\| is replaced by |grp\|.

Similar considerations    apply to   the  character  table  files.  Those
filenames end in '.tbl'.   {\GAP} looks for  those files in the directory
given  by 'TBLNAME'.  The default  value   for 'TBLNAME'  is  obtained by
replacing 'lib' in <libname> with 'tbl'.

'-h <docname>'

The  option '-h' tells {\GAP} that the on-line documentation for {\GAP} is
in  the directory  <docname>.   Per  default  <docname>  is  obtained  by
replacing 'lib' in <libname>  with 'doc'.   <docname> should  end  with a
pathname separator, i.e., |\|, but {\GAP} will silently  add one if it is
missing.  {\GAP} will read the file <docname>|\manual.toc| when you first
use the help system.  If {\GAP} cannot find this file  it will print  the
following warning

|    help: hmm, I cannot open the table of contents file 'doc\manual.toc'
    maybe you should use the option '-h <docname>'?|

'-m <memory>'

The option '-m' tells {\GAP} to  allocate <memory> bytes at startup time.
If the last character of <memory> is 'k' or 'K' it is taken in KBytes and
if the last character is 'm' or 'M' <memory> is taken in MBytes.

{\GAP} for Windows  will by default allocate  4 MBytes of  memory.  If you
specify  '-m <memory>' {\GAP} will  only allocate that  much memory.  The
amount of  memory  should be large  enough so  that computations  do  not
require too  many garbage   collections.  On the   other  hand if  {\GAP}
allocates more virtual memory than is  physically available it will spend
most of the time paging.

'-n'

The  options  '-n' tells {\GAP} to  disable  the line editing and history
(see "Line Editing").

There does not seem to be a good reason to do this on IBM PC compatibles.

'-b'

The option '-b'  tells {\GAP} to  suppress the banner.  That  means  that
{\GAP} immediately prints the prompt.  This is useful  when you get tired
of the banner after a while.

'-q'

The  option '-q' tells  {\GAP} to be quiet.  This  means that {\GAP} does
not  display the banner and the  prompts 'gap>'.  This  is useful  if you
want to run {\GAP} as a filter with input and output redirection and want
to avoid the the banner and the prompts clobber the output file.

'-x <length>'

With this  option you can tell  {\GAP} how long   lines are.  {\GAP} uses
this value to decide when to split long lines.

The default value is 80,  which  is correct if  you start {\GAP} from the
desktop or one of the usual shells.  However, if you  start {\GAP} from a
window shell such as 'gemini', you may want to decrease  this  value.  If
you have a larger monitor, or use a smaller font, or  redirect the output
to a printer, you may want to increase this value.

'-y <length>'

With this  option you can  tell {\GAP} how  many lines  your  screen has.
{\GAP} uses this value to decide after how  many lines of on-line help it
should display |-- <space> for more --|.

The default value is  24, which is  the right value  if  you start {\GAP}
from the desktop or one  of  the  usual shells.   However, if  you  start
{\GAP} from a window  shell such as 'gemini',   you may want  to decrease
this value.  If  you  have a larger monitor,  or use  a smaller  font, or
redirect the output to a printer, you may want to increase this value.

'-z <freq>'

{\GAP} for  Windows  checks  in  regular  intervals  whether the user  has
entered    <ctr>-'Z' or <alt>-'C'  to   interrupt an ongoing computation.
Under Windows this requires reading the keyboard status (UNIX on the other
hand will deliver a  signal to {\GAP}  when the user  entered <ctr>-'C'),
which   is rather expensive.   Therefor   {\GAP} only reads the  keyboard
status  every <freq>-th time.  The  default is 20.   With the option '-z'
this value  can be changed.  Lower  values make {\GAP} more responsive to
interrupts, higher values make {\GAP} a little bit faster.

Further arguments are taken as filenames of files that are read by {\GAP}
during startup, after <libname>|\init.g| is read,  but  before  the first
prompt is  printed.  The files are read in the order in  that they appear
on  the command line.  {\GAP}  only  accepts 14 filenames on the  command
line.  If a file  cannot be opened {\GAP} will print an error message and
will abort.

When you start {\GAP}, it looks  for  the file  with the name 'gap.rc' in
your  homedirectory (i.e., the   directory   defined by  the  environment
variable     'HOME').  If  such   a file  is    found it   is read  after
<libname>|\init.g|, but before any of the  files mentioned on the command
line are read.   You can use  this file for your private  customizations.
For example, if you  have a file  containing  functions  or data that you
always need, you could read this from 'gap.rc'.   Or if you find  some of
the names in  the  library too long,  you could define abbreviations  for
those names in 'gap.rc'.  The following sample 'gap.rc' file does both.

|    Read("c:\\gap\\dat\\mygroups.grp");
    Op := Operation;
    OpHom := OperationHomomorphism;
    RepOp := RepresentativeOperation;
    RepsOp := RepresentativesOperation; |

%%%%%%%%%%%%%%%%%%%%%%%%%%%%%%%%%%%%%%%%%%%%%%%%%%%%%%%%%%%%%%%%%%%%%%%%%
\Section{GAP for Mac/OSX}%
\index{Mac/OSX}\index{Apple}\index{Macintosh}

This sections contain  information about {\GAP} that  is specific to  the
port of  {\GAP} for Apple  Macintosh  systems under Mac/OSX (simply  called
{\GAP} for Mac/OSX below).

To run {\GAP} for Mac/OSX you need <to be written>

The  section "Copyright of GAP for  Mac/OSX" describes the  copyright as it
applies  to  the executable  version  that   we distribute.  The  section
"Installation   of GAP for  Mac/OSX" describes  how you  install {\GAP} for
Mac/OSX, and the section "Features of GAP  for Mac/OSX" describes the special
features of {\GAP} for Mac/OSX.

%%%%%%%%%%%%%%%%%%%%%%%%%%%%%%%%%%%%%%%%%%%%%%%%%%%%%%%%%%%%%%%%%%%%%%%%%
\Section{Copyright of GAP for Mac/OSX}%

<to be written>

%%%%%%%%%%%%%%%%%%%%%%%%%%%%%%%%%%%%%%%%%%%%%%%%%%%%%%%%%%%%%%%%%%%%%%%%%
\Section{Installation of GAP for Mac/OSX}%

<to be written>

%%%%%%%%%%%%%%%%%%%%%%%%%%%%%%%%%%%%%%%%%%%%%%%%%%%%%%%%%%%%%%%%%%%%%%%%%
\Section{Features of GAP for Mac/OSX}%

<to be written>

%%%%%%%%%%%%%%%%%%%%%%%%%%%%%%%%%%%%%%%%%%%%%%%%%%%%%%%%%%%%%%%%%%%%%%%%%
\Section{Porting GAP}

Porting {\GAP} to a new  operating system should  not be very  difficult.
However,  {\GAP} expects some features  from the operating system and the
compiler  and porting {\GAP}  to a system or with  a compiler that do not
have those features may prove very difficult.

The design of {\GAP} makes it quite portable.  {\GAP} consists of a small
kernel written in the programming language  C and a large library written
in the programming language provided by  the {\GAP} kernel, which is also
called {\GAP}.

Once the kernel has been ported, the library poses no additional problem,
because  all those functions only  need the kernel to  work, they need no
additional support from the environment.

The kernel  itself  is  separated  into a  large   part that  is  largely
operating system and compiler independent, and one file that contains all
the operating system and compiler dependent functions.  Usually only this
file must be modified to port {\GAP} to a new operating system.

Now lets take a  look at the  minimal support that  {\GAP} needs from the
operating system and the machine.

First of all  you need  enough filespace.    The kernel sources  and  the
object files need between 3.5 MByte and 4 MByte, depending on the size of
object files   produced   by your compiler.   The  library   takes  up an
additional  4.8 MBytes, and the online  documentation also needs 4 MByte.
So you  need about 13   MByte of available   filespace, for example on  a
harddisk.

Next  you need enough main  memory  in your computer.   The  size of  the
{\GAP}  kernel varies between  different machine,  with  as little as 300
KByte (compiled with  GNU C on an   Atari ST) and  as  much as 600  KByte
(compiled   with UNIX cc on  a  HP 9000/800).  Add   to that the fact the
library of functions that {\GAP} loads takes up another 1.5 MByte.  So it
is  clear that at least  4 MByte  of main memory  are required  to do any
serious work with {\GAP}.

Note that this implies that there is no point in trying to port {\GAP} to
plain Windows  running on IBM PCs  and compatibles.  The version of {\GAP}
for IBM  PC compatibles that we provide  runs on machines  with the Intel
80386, Intel 80486,  Pentium or Pentium Pro  processor under extended DOS
in protected 32 bit mode.  (This is also necessary, because, as mentioned
below, {\GAP} wants to view its memory as a large flat address space.)

Next lets turn to the requirements for the C compiler and its library.

As was already mentioned, the {\GAP} kernel is written in the C language.
We  have tried  to use  as few features  of  the C language as  possible.
{\GAP} has  been compiled without  problems with compilers that adhere to
the old  definition from Kernighan and Ritchie,  and with  compilers that
adhere to the new definition from the ANSI-C standard.

{\GAP} was wriiten for 32-bit compilers ('sizeof(int)==4'), but it has been
ported  by Jean  Michel to  64 bits,  allowing use  of terabytes of memory.
Since  Jean Michel works  on Linux machines,  this 64-bit version works for
now only on such machines. The versions distributed for MAC/OSX and Windows
are still 32-bit.

Dependencies  on  the  operating  system  or  compiler are separated in one
special  file which is called the 'system'  file. When you port {\GAP} to a
new  operating system, you probably have to create a new 'system' file. You
should  however look through the 'system.c' file that we supply and take as
much  code from them as possible.  Currently 'system.c' supports Linux with
gcc, Windows with the DJGPP compiler, and OS/X with gcc.

The 'system' file contains the following functions.

First  file  input and output.  The   functions used by  the three system
files mentioned above are 'fopen', 'fclose',  'fgets', and 'fputs'.  They
are pretty standard, and in fact are in the ANSI C standard library.  The
only thing that may be necessary is to make sure that files are opened in
text  mode.  However, the most  important transformation  applied in text
mode seems to be to replace  the end of line sequence <newline>-<return>,
used in some  operating  systems, with a  single  <newline>, used  in  C.
However, since  {\GAP} treats <newline> and  <return> both as whitespaces
even this is not absolutely necessary.

Second you need  character oriented input  from the keyboard   and to the
screen.  This is not absolutely necessary, you can  use the line oriented
input and output described above.  However, if  you want the  history and
the command  line editing, {\GAP}  must  be  able   to read every  single
character as the user types it without echo, and also must be able to put
single characters  to  the screen.    Reading  characters  unblocked  and
without echo is very  system dependent.

Third you need a  way to detect if the  user typed <ctr>-'C' to interrupt
an   ongoing  computation   in  {\GAP}.  Again    this is not  absolutely
necessary,  you can   do without.   However  if you want  to   be able to
interrupt computations,  {\GAP} must be able to  receive  the  interrupt.
This can be done in two ways.  Under  UNIX you can  catch the signal that
is   generated  if   the  user types <ctr>-'C'  ('SIGINT').   Under other
operating systems that do not support such signals you can poll the input
stream at regular intervals and simply look for <ctr>-'C'.

Fourth you  need  a way  to  find out how long  {\GAP} has  been running.
Again this is not  absolutely necessary.  You can simply always return 0,
fooling  {\GAP} into  thinking that it is extremely fast.  However if you
want timing statistics, {\GAP} must be able to find out how much CPU time
it has spent.

The  last and most important function  is the function to allocate memory
for {\GAP}.  {\GAP}  assumes that  it can allocate the initial  workspace
with the function  'SyGetmem' and expand this workspace  on  demand  with
further calls to 'SyGetmem'.  The  memory allocated by consecutive calls
to 'SyGetmem' must be  adjacent, because {\GAP}  wants to manage a single
large  block  of memory.  Usually this   can be done  with  the C library
function 'sbrk'.  If this does  not work, you  can define a  large static
array in  the  'system'  file and return  this  on    the  first call  to
'SyGetmem' and return 0  on subsequent calls  to indicate that this array
cannot dynamically be expanded.
