%%%%%%%%%%%%%%%%%%%%%%%%%%%%%%%%%%%%%%%%%%%%%%%%%%%%%%%%%%%%%%%%%%%%%%%%%%%%%
%%
%A  chvintr.tex       CHEVIE documentation       Meinolf Geck, Frank Luebeck,
%A                                                Jean Michel, G"otz Pfeiffer
%%
%Y  Copyright (C) 1992 - 2002  Lehrstuhl D f\"ur Mathematik, RWTH Aachen, IWR
%Y  der Universit\"at Heidelberg, University of St. Andrews, and   University
%Y  Paris VII.
%%
%%  This  file  contains  an introduction to the GAP-part of CHEVIE.
%%%%%%%%%%%%%%%%%%%%%%%%%%%%%%%%%%%%%%%%%%%%%%%%%%%%%%%%%%%%%%%%%%%%%%%%%%%%%
\def\Sym{{\mathfrak S}}

\Chapter{The CHEVIE Package Version 4 -- a short introduction}

\CHEVIE\   is  a  joint  project  of  Meinolf  Geck,  Gerhard  Hiss,  Frank
L{\accent127  u}beck,  Gunter  Malle,  Jean  Michel,  and G{\accent127 o}tz
Pfeiffer.  We document here  the development version  4 of the \GAP-part of
\CHEVIE. This is a package in the \GAP3 language, which implements

\begin{itemize}
\item
algorithms  for\:\ finite  complex reflection  groups and  their cyclotomic
Hecke  algebras, arbitrary Coxeter groups,  the corresponding braid groups,
Kazhdan-Lusztig   bases,  left  cells,  root  data,  unipotent  characters,
unipotent  and semi-simple  elements of  algebraic groups, Green functions,
etc...
\item
contains  library files  holding information  for finite complex reflection
groups giving conjugacy classes, fake degrees, generic degrees, irreducible
characters,  representations of  the associated  Hecke algebras, associated
unipotent  characters  and  unipotent  classes  (for  Weyl  groups, or more
generally, \"Spetsial\" groups).
\end{itemize}

The  package  is  automatically  loaded  if  you use the \GAP\ distribution
|gap3-jm|; otherwise, you need to load it using

|    gap> RequirePackage("chevie");
    --- Loading package chevie ------ version of 2018 Feb 19 ------
    If you use CHEVIE in your work please cite the authors as follows:
    [Jean Michel] The development version of the CHEVIE package of GAP3
     Journal of algebra 435 (2015) 308--336
    [Meinolf Geck, Gerhard Hiss, Frank Luebeck, Gunter Malle, Goetz Pfeiffer]
     CHEVIE -- a system for computing and processing generic character tables
     Applicable Algebra in Engineering Comm. and Computing 7 (1996) 175--210|

Compared  to version 3, it  is more general. For  example, one can now work
systematically  with arbitrary Coxeter  groups, not necessarily represented
as permutation groups. Quite a few functions also work for arbitrary finite
groups  generated by complex reflections.  Some functions have changed name
to  reflect the more general functionality.  We have kept most former names
working  for compatibility, but we do  not guarantee that they will survive
in future releases.

Many  objects associated  with finite  Coxeter groups  admit some canonical
labeling  which  carries  additional  information.  These  labels are often
important  for applications  to Lie  theory and  related areas.  The groups
constructed  in the  package are  permutation or  matrix groups, so all the
functions  defined for such groups work;  but often there are improvements,
exploiting  the particular nature of these groups. For example, the generic
{\GAP}  function  'ConjugacyClasses'  applied  to  a Coxeter group does not
invoke the general algorithm for computing conjugacy classes of permutation
groups  in  \GAP,  but  first  decomposes  the  given  Coxeter  group  into
irreducible components, and then reads canonical representatives of minimal
length  in the various classes of these irreducible components from library
files.  These  canonical  representatives  also  come  with some additional
information,  for  example  the  class  names in exceptional groups reflect
Carter\'s admissible diagrams and in classical groups are given in terms of
partitions.  In a similar way, the function 'CharTable' does not invoke the
Dixon--Schneider  algorithm  but  proceeds  in  a  similar way as described
above.  Moreover, in  the resulting  character table  the classes  have the
labels  described above and the characters also have canonical labels, e.g.
partitions  of $n$ in  the case of  the symmetric group  $\Sym_n$, which is
also  the  Coxeter  group  of  type  $A_{n-1}$  (see  "ChevieClassInfo" and
"ChevieCharInfo").  The normal forms we use, and the associated labeling of
classes and characters for the individual types, are explained in detail in
the  various to chapters. The same  considerations extend to some extent to
all finite complex reflection groups.

Thus, most of the disk space required by {\CHEVIE} is occupied by the files
containing  the basic  information about  the finite irreducible reflection
groups.  These files  are called  'weyla.g', 'cmplxg24.g'  etc.\ up  to the
biggest  file  'cmplxg34.g'  whose  size  is about $660$~KBytes. These data
files  are structured in a uniform manner  so that any piece of information
can be extracted separately from them. (For example, it is not necessary to
first  compute  the  character  table  in  order  to  have  labels  for the
characters and classes.)

Several   computations  in   the  literature   concerning  the  irreducible
characters  of finite Coxeter groups and Iwahori--Hecke algebras can now be
checked  or  re-computed  by  anyone  who  is  willing  to  use  {\GAP} and
\CHEVIE.  Re-doing such  computations and  comparing with existing tables
has  helped discover bugs in the  programs and misprints in the literature;
we  believe that having the possibility  of repeating such computations and
experimenting  with the results  has increased the  reliability of the data
and the programs. For example, it is now a trivial matter to re-compute the
tables  of induce/restrict matrices  (with the appropriate  labeling of the
characters) for exceptional finite Weyl groups (see Section
"ReflectionSubgroup").  These  matrices  have  various  applications in the
representation  theory  of  finite  reductive  groups,  see  chapter  4  of
Lusztig\'s book \cite{Lus85}.

We  ourselves have used these programs to prove results about the existence
of  elements with  special properties  in the  conjugacy classes  of finite
Coxeter  groups (see  \cite{GP93}, \cite{GM97}),  and to  compute character
tables  of Iwahori--Hecke  algebras of  exceptional type (see \cite{Gec94},
\cite{GM97}). For a survey, see also \cite{Chv96}. Quite a few computations
with finite complex reflection groups have also been made in \CHEVIE.

$\bullet$  The user should observe limitations  on storage for working with
these  programs, e.g.,  the command  'Elements' applied  to a Weyl group of
type $E_8$ needs a computer with 360GB of main memory!

$\bullet$ There is a function 'InfoChevie' which is set equal to the {\GAP}
function   'Ignore'  when  you   load  \CHEVIE.  If   you  redefine  it  by
'InfoChevie\:=Print;'   then  the  {\CHEVIE}   functions  will  print  some
additional information in the course of their computations.

Of  course, our hope is that more applications will be added in the future!
For  contributions to {\CHEVIE}  have created a  directory 'contr' in which
the  corresponding  files  are  distributed  with \CHEVIE. However, they do
remain  under the authorship and the responsibility of their authors. Files
from   that  directory   can  be   read  into   {\GAP}  using  the  command
'ReadChv(\"contr/filename\")'.  At present, the  directory 'contr' contains
the following files\:

'affa'   by  F.~Digne:  it   contains  functions  to   work  with  periodic
permutations of the integers, with the affine Coxeter group of type $\tilde
A$  seen as  a group  of periodic  permutations, and with the corresponding
dual Garside monoid.

'arikidec'  by N.~Jacon: it contains  functions for computing the canonical
basis  of an arbitrary irreducible integrable highest weight representation
of   the  quantum   group  of   the  affine   special  linear   group  $U_v
(\widehat{sl}_e)$. It also computes the decomposition matrix of Ariki-Koike
algebras, where the parameters are power of a e-th root of unity in a field
of characteristic zero.

'braidsup'  by  J.~Michel:  it  contains  some  supplementary  programs for
working with braids (or more generally Garside monoids).

'brbase'  by  M.~Geck  and  S.~Kim:  it  contains  programs  for  computing
bi-grassmannians  and the  base for  the Bruhat--Chevalley  order on finite
Coxeter groups (see \cite{GK96}).

'chargood'  by  M.~Geck  and  J.~Michel:  it  contains  functions  (used in
\cite{GM97})   implementing  algorithms  to  compute  character  tables  of
Iwahori--Hecke algebras, especially that of type $E_8$.

'cp'  by J.~Michel and  G.~Neaime: it contains  a function to construct the
Corran-Picantin monoid as an interval monoid, following Neaime\'s work.

'hecbloc'  by  M.~Geck:  it  contains  functions  for  computing blocks and
defects  of characters of  Iwahori--Hecke algebras specialized  at roots of
unity over the rational numbers.

'minrep'  by  M.~Geck  and  G.~Pfeiffer:  it  contains  programs  (used  in
\cite{GP93})  for  computing  representatives  of  minimal  length  in  the
conjugacy classes of finite Coxeter groups.

'murphy'  by A.~Mathas: it contains  programs which allow calculations with
the Murphy basis of the Hecke algebra of type $A$.

'rouquierblockdata'  by M.~Chlouveraki and J.~Michel: it contains functions
to  compute the Rouquier blocks of  1-cylotomic Hecke algebras of arbitrary
complex reflection groups.

'specpie'  by M.~Geck and G.~Malle: it contains functions for computing the
Green-like  polynomials  (or  rather  rational  functions)  associated with
special  pieces of the unipotent variety  (or ``special'' characters in the
case of complex reflection groups'').

'xy'  by  J.~Michel  and  R.~Rouquier:  it  contains  a function to display
graphically elements of Hecke modules for affine Weyl groups of rank 2.

Finally,  it should  be mentioned  that the  tables of  Green functions for
finite  groups of Lie type which are  in the {\MAPLE}-part of {\CHEVIE} are
now  obtainable by using  the {\CHEVIE} routines  for unipotent classes and
the associated intersection cohomology complexes.

\noindent  {\bf Acknowledgments.} We  wish to thank  the Aachen {\GAP} team
for general support.

We  also  gratefully  acknowledge  financial  support  by  the  DFG  in the
framework  of the Forschungsschwerpunkt  \"Algorithmische Zahlentheorie und
Algebra\"\ from 1992 to 1998.

We  are indebted to  Andrew Mathas for  contributing the initial version of
functions for the various Kazhdan-Lusztig bases in kl.g.
%%%%%%%%%%%%%%%%%%%%%%%%%%%%%%%%%%%%%%%%%%%%%%%%%%%%%%%%%%%%%%%%%%%%%%%%%
