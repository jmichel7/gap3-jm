% parent manual.tex

%%%%%%%%%%%%%%%%%%%%%%%%%%%%%%%%%%%%%%%%%%%%%%%%%%%%%%%%%%%%%%%%%%%%%%%%%%%%%
%%
%A  aut.tex                     GAP documentation               Michael Smith
%%
%%
%%  This file contains the documentation of the code for computing
%%  automorphism groups of groups given by sag-group presentations.
%%
%%
%%%%%%%%%%%%%%%%%%%%%%%%%%%%%%%%%%%%%%%%%%%%%%%%%%%%%%%%%%%%%%%%%%%%%%%%%
\Chapter{Automorphism Groups of Special Ag Groups}


This chapter describes functions which compute and display information
about automorphism groups of finite soluble groups.


The  algorithm used  for computing  the automorphism  group requires  that the
soluble   group  be  given  in  terms  of  a  special  ag  presentation.  Such
presentations  are described in  the chapter of  the {\GAP} manual which deals
with  'Special  Ag  Groups'.  Given  a  group  presented  by  an  arbitrary ag
presentation,  a special  ag presentation  can be  computed using the function
'SpecialAgGroup'.


The  automorphism  group  is  returned  as  a  standard  {\GAP}  group record.
Automorphisms  are represented by their action on the sag group generating set
of the input group. The order of the automorphism group is also computed.


The performance of the automorphism group algorithm is highly dependent on the
structure  of the  input group.  Given two  groups with  the same  sequence of
LG-series  factor groups it  will usually take  much less time  to compute the
automorphism group of the one with the larger automorphism group. For example,
it takes less than 1 second (Sparc 10/52) to compute the automorphism group of
the  exponent  7  extraspecial  group  of  order  $7^3$. It takes more than 40
seconds  to compute  the automorphism  group of  the exponent  49 extraspecial
group  of order $7^3$. The  orders of the automorphism  groups are $98784$ and
$2058$  respectively.   It takes only 20  minutes (Sparc 10/52) to compute the
automorphism group of the 2-generator Burnside group of exponent 6, a group of
order  $2^{28}\cdot 3^{25}$  whose automorphism  group has  order $2^{40}\cdot
3^{53}\cdot  5\cdot 7$; note,  however, that it  can take substantially longer
than this to compute the automorphism groups of some of the groups of order 64
(for  nilpotent groups one should  use the function 'AutomorphismsPGroup' from
the ANU PQ package instead).


The  following section describes  the function that  computes the automorphism
group  of a  special ag  group (see  "AutGroupSagGroup"). It  is followed by a
description   of  automorphism  group  elements   and  their  operations  (see
"Automorphism   Group  Elements"   and  "Operations   for  Automorphism  Group
Elements").  Functions  for  obtaining  some  structural information about the
automorphism    group   are    described   next    (see   "AutGroupStructure",
"AutGroupFactors" and "AutGroupSeries"). Finally, a function that converts the
automorphism   group  into  a  form  which  may  be  more  suitable  for  some
applications is described (see "AutGroupConverted").

%%%%%%%%%%%%%%%%%%%%%%%%%%%%%%%%%%%%%%%%%%%%%%%%%%%%%%%%%%%%%%%%%%%%%%%%%
\Section{AutGroupSagGroup}

'AutGroupSagGroup(<G>)' \\ 
'AutGroupSagGroup(<G>, <l>)'


Given  a special  ag group  <G>, the  function 'AutGroupSagGroup' computes the
automorphism  group of <G>. It returns a group generated by automorphism group
elements  (see  "Automorphism  Group  Elements").  The  order of the resulting
automorphism group can be obtained by applying the function 'Size' to it.

If the optional argument <l> is supplied, the automorphism group of $G/G_l$ is
computed, where $G_l$ is the $l$-th term of the LG-series of $G$ (see 'More
about Special Ag Groups').

%--------------------------------------------------------------------------
|    gap> C6 := CyclicGroup(AgWords, 6);;
    gap> S3 := SymmetricGroup(AgWords, 3);;
    gap> H := WreathProduct(C6,S3);;
    gap> G := SpecialAgGroup(H / Centre(H));;
    gap> G := RenamedGensSagGroup(G, "g"); # rename gens of G to [g1,g2,..,g12]
    Group( g1, g2, g3, g4, g5, g6, g7, g8, g9, g10, g11, g12 )
    gap> G.name := "G";;
    gap> A := AutGroupSagGroup(G);
    Group( Aut(G, [ g1*g2, g2, g3, g4, g5, g6, g7, g8, g9, g10, g11, g12 
     ]), Aut(G, [ g1, g2, g3^2, g4^2*g6^2*g7, g5^2*g6*g7^2, g6*g8^2, 
      g7*g8^2, g8^2, g10*g11, g10, g9*g10, g9*g11*g12 ]), Aut(G, 
    [ g1, g2, g3, g4, g5^2*g6*g7^2, g6*g7, g7^2, g8^2, g9, g10, g11, g12 
     ]), Aut(G, [ g1, g2, g3, g4*g6*g7^2, g5*g6^2*g7, g6, g7, g8, g9, g10, 
      g11, g12 ]), Aut(G, [ g1, g2, g3, g4, g5*g6*g7^2, g6, g7, g8, g9, 
      g10, g11, g12 ]), Aut(G, [ g1, g2, g3, g4^2, g5*g6^2*g7, g6^2*g8, 
      g7^2*g8, g8, g10*g11, g10, g9*g10, g9*g11*g12 ]), Aut(G, 
    [ g1, g2, g3, g4*g6^2*g7, g5*g6*g7^2, g6, g7, g8, g9, g10, g11, g12 
     ]), InnerAut(G, g1), InnerAut(G, g3), InnerAut(G, g4), InnerAut(G, 
    g5), InnerAut(G, g6), Aut(G, [ g1, g2, g3*g7*g8, g4, g5, g6*g8, g7, 
      g8, g9, g10, g11, g12 ]), InnerAut(G, g7*g8), Aut(G, 
    [ g1, g2, g3, g4, g5*g8, g6, g7, g8, g9, g10, g11, g12 ]), InnerAut(G, 
    g8^2), Aut(G, [ g1, g2, g3, g4, g5, g6, g7, g8, g9, g9*g11, g9*g10, 
      g10*g11*g12 ]), Aut(G, [ g1, g2, g3, g4, g5, g6, g7, g8, g10*g12, 
      g10, g9*g11*g12, g9*g10 ]), InnerAut(G, g10), InnerAut(G, 
    g11), InnerAut(G, g12), InnerAut(G, g9) )
    gap> Size(A);
    30233088
    gap> PrimePowersInt(last);
    [ 2, 9, 3, 10 ]|
%--------------------------------------------------------------------------

\bigskip

The size of the outer automorphism group is easily computed as follows.

%%----------------------------------------------------------------------
|    gap> innersize := Size(G) / Size(Centre(G));
    23328
    gap> outersize := Size(A) / innersize;
    1296|
%--------------------------------------------------------------------------

%%%%%%%%%%%%%%%%%%%%%%%%%%%%%%%%%%%%%%%%%%%%%%%%%%%%%%%%%%%%%%%%%%%%%%%%%
\Section{Automorphism Group Elements}

An element <a> of an automorphism group is a group element record with the
following additional components\:

'isAut': \\ Is bound to 'true' if <a> is an automorphism record.

'group': \\ Is the special ag group <G> on which the automorphism <a> acts.

'images': \\ Is the list of images of the generating set of <G> under <a>.
That is, 'a.images[i]' is the image of 'G.generators[i]' under the automorphism.

\bigskip

The following components may also be defined for an automorphism group element\:


'inner': \\ If this component is bound, then it is either an element <g> of
<G> indicating that <a> is the inner automorphism of <G> induced by <g>, or it is
'false' indicating that <a> is not an inner automorphism.

'weight': \\ This component is set for the elements of the generating set
of the full automorphism group of a sag group. It stores the weight of the
generator (see "AutGroupStructure").


\bigskip

Along with most of the functions that can be applied to any group elements
(e.g. 'Order' and 'IsTrivial'), the following functions are specific to
automorphism group elements\:


\bigskip

'IsAut(<a>)'

The function 'IsAut' returns 'true' if <a> is an automorphism record, and
'false' otherwise.

\bigskip

'IsInnerAut(<a>)'

Returns 'true' if <a> is an inner automorphism, and 'false' otherwise. If
'a.inner' is already bound, then the information stored there is used.  If
'a.inner' is not bound, 'IsInnerAut' determines whether <a> is an inner
automorphism, and sets 'a.inner' appropriately before returning the answer.



%%%%%%%%%%%%%%%%%%%%%%%%%%%%%%%%%%%%%%%%%%%%%%%%%%%%%%%%%%%%%%%%%%%%%%%%%
\Section{Operations for Automorphism Group Elements}


'<a> = <b>'

For  automorphism group  elements <a>  and <b>,  the operator '=' evaluates to
'true'  if the automorphism  records correspond to  the same automorphism, and
'false'  otherwise. Note that this may return 'true' even when the two records
themselves are different (one of them may have more information stored in it).


'<a> {\*} <b>'

For automorphism group elements <a> and <b>, the operator '\*' evaluates to the
product $a b$ of the automorphisms.

\bigskip

'<a> / <b>'

For automorphism group elements <a> and <b>, the operator '/' evaluates to the
quotient $a b^{-1}$ of the automorphisms.

\bigskip

'<a> {\^} <i>'

For an automorphism group element <a> and an integer <i>, the operator '\^'
evaluates to the <i>-th power $a^i$ of <a>.

\bigskip

'<a> {\^} <b>'

For automorphism group elements <a> and <b>, the operator '\^' evaluates to the
conjugate $b^{-1} a b$ of <a> by <b>.

\bigskip

'Comm(<a>, <b>)'

The function 'Comm' returns the commutator $a^{-1} b^{-1} a b$
of the two automorphism group elements <a> and <b>.

\bigskip

'<g> {\^} <a>'

For a sag group element <g> and an automorphism group element <a>, the operator
'\^' evaluates to the image $g^a$ of the ag word <g> under the automorphism <a>.
The sag group element <g> must be an element of 'a.group'.

\bigskip

'<S> {\^} <a>'

For a subgroup <S> of a sag group and an automorphism group element <a>, the
operator '\^' evaluates to the image $S^a$ of the subgroup <S> under the
automorphism <a>.  The subgroup <S> must be a subgroup of 'a.group'.

\bigskip

'<list> {\*} <a>' \\
'<a> {\*} <list>'

For a list <list> and an automorphism group element <a>, the operator '\*'
evaluates to the list whose <i>-th entry is '<list>[<i>] {\*} <a>' or '<a>
{\*} <list>[<i>]' respectively.

\bigskip

'<list> {\^} <a>'

For a list <list> and an automorphism group element <a>, the operator '\^'
evaluates to the list whose <i>-th entry is '<list>[<i>] {\^} <a>'.

\bigskip

Note that the action of automorphism group elements on the elements of the sag
group via the operator '\^' corresponds to the default action 'OnPoints'
(see 'Other Operations') so that the functions 'Orbit' and 'Stabilizer' can
be used in the natural way. For example\:

%--------------------------------------------------------------------------
|    gap> Orbit(A, G.7);
    [ g7, g7*g8^2, g7^2, g7^2*g8, g7*g8, g7^2*g8^2 ]
    gap> Length(last);
    6
    gap> S := Subgroup(G, [G.11, G.12]);
    Subgroup( G, [ g11, g12 ] )
    gap> Size(S);
    4
    gap> Orbit(A, S);
    [ Subgroup( G, [ g11, g12 ] ), Subgroup( G, [ g9*g10, g9*g11*g12 ] ) ]
    gap> Intersection(last);
    Subgroup( G, [  ] )|
%--------------------------------------------------------------------------

%%%%%%%%%%%%%%%%%%%%%%%%%%%%%%%%%%%%%%%%%%%%%%%%%%%%%%%%%%%%%%%%%%%%%%%%%
\Section{AutGroupStructure}


'AutGroupStructure(<A>)'

The generating set of the automorphism group returned by 'AutGroupSagGroup' is
closely  related to a  particular subnormal series  of the automorphism group.
This function displays a description of the factors of this series.


Let  $A$ be the automorphism group  of $G$. Let $G=G_1 >  G_2 > \ldots > G_m >
G_{m+1}=1$  be the LG-series of $G$ (see  'More about Special Ag Groups'). For
$0  \leq i \leq m$ let $A_{2i+1}$ be  the subgroup of $A$ containing all those
automorphisms  which induce the identity on $G/G_{i+1}$. Clearly $A_1 = A$ and
$A_{2m+1}  = 1$.  Furthermore, let  $A_{2i+2}$ be  the subgroup  of $A_{2i+1}$
containing those automorphisms which also act trivially on the quotient $G_i /
G_{i+1}$. Note that $A_2/A_3$ is always trivial. Thus the subnormal series
%
$$A = A_1 \geq A_2 \geq \ldots \geq A_{2m+1} = 1$$
%
of $A$ is obtained. The subgroup $A_i$ is the *weight* $i$ subgroup of $A$.
The *weight* of a generator $\alpha$ of $A$ is defined to be the least $i$
such that $\alpha \in A_{i}$.


The  function  'AutGroupStructure'  takes  as  input an automorphism group <A>
computed  using  'AutGroupSagGroup'  and  prints  out  a  description  of  the
non-trivial factors of the subnormal series of the automorphism group <A>.


The  factor of *weight*  $i$ is $A_i/A_{i+1}$.  A factor of  even weight is an
elementary abelian group, and it is described by giving its order. A factor of
odd   weight  is  described  by  giving   a  generating  set  for  a  faithful
representation  of it as a matrix group acting  on a layer of the LG-series of
$G$ (the weight $2i-1$ factor acts on the LG-series layer $G_i/G_{i+1}$).


%--------------------------------------------------------------------------
|    gap> AutGroupStructure(A);;
    
     Order of full automorphism group is 30233088 = 2^9 * 3^10
    
     Factor of size 2 (matrix group, weight 1)
      Field: GF(2)
        [1 1]
        [0 1]
    
     Factor of size 2 (matrix group, weight 3)
      Field: GF(3)
        [2]
    
     Factor of size 36 = 2^2 * 3^2 (matrix group, weight 5)
      Field: GF(3)
        [1 0 0]    [1 0 1]    [1 0 0]    [2 0 0]    [1 0 2]
        [0 2 1]    [0 1 2]    [0 1 1]    [0 1 2]    [0 1 1]
        [0 0 1]    [0 0 1]    [0 0 1]    [0 0 2]    [0 0 1]
    
        [2 0 0]    [1 0 1]
        [0 2 0]    [0 1 0]
        [0 0 2]    [0 0 1]
    
     Factor of size 27 = 3^3 (elementary abelian, weight 6)
    
     Factor of size 3 (elementary abelian, weight 8)
    
     Factor of size 27 = 3^3 (elementary abelian, weight 10)
    
     Factor of size 6 = 2 * 3 (matrix group, weight 11)
      Field: GF(2)
        [1 0 0 0]    [0 1 0 1]
        [1 0 1 0]    [0 1 0 0]
        [1 1 0 0]    [1 0 1 1]
        [0 1 1 1]    [1 1 0 0]
    
     Factor of size 16 = 2^4 (elementary abelian, weight 12)
    |
%--------------------------------------------------------------------------
    
\bigskip

As mentioned earlier, each generator of the automorphism group has its weight
stored in the record component 'weight'.

%--------------------------------------------------------------------------
|    gap> List(Generators(A), a -> a.weight);
    [ 1, 3, 5, 5, 5, 5, 5, 5, 5, 6, 6, 6, 8, 10, 10, 10, 11, 11, 12, 12, 
      12, 12 ]|
%--------------------------------------------------------------------------

\bigskip

Note that the subgroup $A_i$ of $A$ is generated by the elements of the
generating set of $A$ whose weights are at least $i$.  Hence, in analogy to
strong generating sets of permutation groups, the generating set of $A$ is
a *strong generating set* relative to the chain of subgroups $A_i$.

\bigskip

The generating set of a matrix group displayed by 'AutGroupStructure'
corresponds directly to the list of elements of the corresponding weight in
'A.generators'. In the example above, the first matrix listed at weight 5
corresponds to 'A.generators[3]', and the last matrix listed at weight 5
corresponds to 'A.generators[9]'.


It  is also  worth noting  that the  generating set  for an automorphism group
returned  by 'AutGroupSagGroup' can be heavily redundant. In the example given
above,  the weight 5 matrix group can be  generated by just three of the seven
elements listed (for example elements 1, 5 and 6). The other four elements can
be   discarded  from  the  generating  set  for  the  matrix  group,  and  the
corresponding elements of the generating set for $A$ can also be discarded.

%%%%%%%%%%%%%%%%%%%%%%%%%%%%%%%%%%%%%%%%%%%%%%%%%%%%%%%%%%%%%%%%%%%%%%%%%
\Section{AutGroupFactors}

'AutGroupFactors(<A>)' 


The  function  'AutGroupFactors'  takes  as  input  an  automorphism group <A>
computed  by 'AutGroupSagGroup' and returns  a list containing descriptions of
the  non-trivial factors $A_i/A_{i+1}$ (see "AutGroupStructure"). Each element
of  this list  is either  a list  $[p, e]$  which indicates that the factor is
elementary  abelian of order $p^e$,  or a matrix group  which is isomorphic to
the corresponding factor.

%--------------------------------------------------------------------------
|    gap> fact:=AutGroupFactors(A);;
    gap> F := fact[3];;
    gap> D := DerivedSubgroup(F);;
    gap> Nice(Generators(D));
      Field: GF(3)
        [1 0 0]
        [0 1 2]
        [0 0 1]
    gap> S := SylowSubgroup(F,2);;
    gap> Nice(Generators(S));
      Field: GF(3)
        [2 0 0]    [1 0 0]
        [0 1 1]    [0 2 2]
        [0 0 2]    [0 0 1]|
%--------------------------------------------------------------------------

Of course, the factors of the returned series can be examine further. For
example:

%--------------------------------------------------------------------------
|    gap> F := fact[3];;
    gap> D := DerivedSubgroup(F);;
    gap> Nice(Generators(D));
      Field: GF(3)
        [1 0 0]
        [0 1 2]
        [0 0 1]
    gap> S := SylowSubgroup(F,2);;
    gap> Nice(Generators(S));
      Field: GF(3)
        [2 0 0]    [1 0 0]
        [0 1 1]    [0 2 2]
        [0 0 2]    [0 0 1]|
%--------------------------------------------------------------------------


%%%%%%%%%%%%%%%%%%%%%%%%%%%%%%%%%%%%%%%%%%%%%%%%%%%%%%%%%%%%%%%%%%%%%%%%%
\Section{AutGroupSeries}


'AutGroupSeries(<A>)' 

The  function  'AutGroupSeries'  takes  as  input  an  automorphism  group <A>
computed  by 'AutGroupSagGroup' and returns  a list containing those subgroups
$A_i$   of  $A$  which  give  non-trivial   quotients  $A_i  /  A_{i+1}$  (see
"AutGroupStructure").

%--------------------------------------------------------------------------
|    gap> series:=AutGroupSeries(A);;
    gap> series[7].weight;
    11
    gap> series[8].weight;
    12|
%--------------------------------------------------------------------------


\bigskip

Each of the subgroups in the list has its weight stored in record component
'weight'.

%--------------------------------------------------------------------------
|    gap> series[7].weight;
    11
    gap> series[8].weight;
    12|
%--------------------------------------------------------------------------


%%%%%%%%%%%%%%%%%%%%%%%%%%%%%%%%%%%%%%%%%%%%%%%%%%%%%%%%%%%%%%%%%%%%%%%%%
\Section{AutGroupConverted}

'AutGroupConverted (<A>)'

Convert the automorphism group returned by 'AutGroupSagGroup' into a group
generated by 'GroupHomomorphismByImages' records, and return the resulting
group.  Note that this function should not be used unless absolutely
necessary, since operations for elements of the resulting group are
substantially slower than operations with automorphism records.


%--------------------------------------------------------------------------
|    gap> H := AutGroupConverted(A);
    Group( GroupHomomorphismByImages( G, G, 
    [ g1, g2, g3, g4, g5, g6, g7, g8, g9, g10, g11, g12 ], 
    [ g1*g2, g2, g3, g4, g5, g6, g7, g8, g9, g10, g11, g12 
     ] ), GroupHomomorphismByImages( G, G, 
    [ g1, g2, g3, g4, g5, g6, g7, g8, g9, g10, g11, g12 ], 
    [ g1, g2, g3^2, g4^2*g6^2*g7, g5^2*g6*g7^2, g6*g8^2, g7*g8^2, g8^2, 
      g10*g11, g10, g9*g10, g9*g11*g12 
     ] ), GroupHomomorphismByImages( G, G, 
    [ g1, g2, g3, g4, g5, g6, g7, g8, g9, g10, g11, g12 ], 
    [ g1, g2, g3, g4, g5^2*g6*g7^2, g6*g7, g7^2, g8^2, g9, g10, g11, g12 
     ] ), GroupHomomorphismByImages( G, G, 
    [ g1, g2, g3, g4, g5, g6, g7, g8, g9, g10, g11, g12 ], 
    [ g1, g2, g3, g4*g6*g7^2, g5*g6^2*g7, g6, g7, g8, g9, g10, g11, g12 
     ] ), GroupHomomorphismByImages( G, G, 
    [ g1, g2, g3, g4, g5, g6, g7, g8, g9, g10, g11, g12 ], 
    [ g1, g2, g3, g4, g5*g6*g7^2, g6, g7, g8, g9, g10, g11, g12 
     ] ), GroupHomomorphismByImages( G, G, 
    [ g1, g2, g3, g4, g5, g6, g7, g8, g9, g10, g11, g12 ], 
    [ g1, g2, g3, g4^2, g5*g6^2*g7, g6^2*g8, g7^2*g8, g8, g10*g11, g10, 
      g9*g10, g9*g11*g12 ] ), GroupHomomorphismByImages( G, G, 
    [ g1, g2, g3, g4, g5, g6, g7, g8, g9, g10, g11, g12 ], 
    [ g1, g2, g3, g4*g6^2*g7, g5*g6*g7^2, g6, g7, g8, g9, g10, g11, g12 
     ] ), GroupHomomorphismByImages( G, G, 
    [ g1, g2, g3, g4, g5, g6, g7, g8, g9, g10, g11, g12 ], 
    [ g1, g2, g3, g4^2, g5^2, g6^2*g7^2*g8^2, g7*g8^2, g8^2, g10*g11, g10, 
      g9*g10, g9*g11*g12 ] ), GroupHomomorphismByImages( G, G, 
    [ g1, g2, g3, g4, g5, g6, g7, g8, g9, g10, g11, g12 ], 
    [ g1, g2, g3, g4*g6*g7^2, g5, g6, g7, g8, g9, g10, g11, g12 
     ] ), GroupHomomorphismByImages( G, G, 
    [ g1, g2, g3, g4, g5, g6, g7, g8, g9, g10, g11, g12 ], 
    [ g1*g4^2, g2, g3*g6^2*g7, g4, g5*g7^2*g8, g6*g8^2, g7*g8^2, g8, 
      g10*g11, g9*g10*g12, g11*g12, g11 
     ] ), GroupHomomorphismByImages( G, G, 
    [ g1, g2, g3, g4, g5, g6, g7, g8, g9, g10, g11, g12 ], 
    [ g1*g5^2, g2, g3, g4*g7*g8^2, g5, g6, g7, g8, g9, g10, g11, g12 
     ] ), GroupHomomorphismByImages( G, G, 
    [ g1, g2, g3, g4, g5, g6, g7, g8, g9, g10, g11, g12 ], 
    [ g1*g6^2*g7*g8, g2, g3, g4*g8, g5, g6, g7, g8, g9, g10, g11, g12 
     ] ), GroupHomomorphismByImages( G, G, 
    [ g1, g2, g3, g4, g5, g6, g7, g8, g9, g10, g11, g12 ], 
    [ g1, g2, g3*g7*g8, g4, g5, g6*g8, g7, g8, g9, g10, g11, g12 
     ] ), GroupHomomorphismByImages( G, G, 
    [ g1, g2, g3, g4, g5, g6, g7, g8, g9, g10, g11, g12 ], 
    [ g1, g2, g3, g4*g8, g5, g6, g7, g8, g9, g10, g11, g12 
     ] ), GroupHomomorphismByImages( G, G, 
    [ g1, g2, g3, g4, g5, g6, g7, g8, g9, g10, g11, g12 ], 
    [ g1, g2, g3, g4, g5*g8, g6, g7, g8, g9, g10, g11, g12 
     ] ), GroupHomomorphismByImages( G, G, 
    [ g1, g2, g3, g4, g5, g6, g7, g8, g9, g10, g11, g12 ], 
    [ g1*g8, g2, g3, g4, g5, g6, g7, g8, g9, g10, g11, g12 
     ] ), GroupHomomorphismByImages( G, G, 
    [ g1, g2, g3, g4, g5, g6, g7, g8, g9, g10, g11, g12 ], 
    [ g1, g2, g3, g4, g5, g6, g7, g8, g9, g9*g11, g9*g10, g10*g11*g12 
     ] ), GroupHomomorphismByImages( G, G, 
    [ g1, g2, g3, g4, g5, g6, g7, g8, g9, g10, g11, g12 ], 
    [ g1, g2, g3, g4, g5, g6, g7, g8, g10*g12, g10, g9*g11*g12, g9*g10 
     ] ), GroupHomomorphismByImages( G, G, 
    [ g1, g2, g3, g4, g5, g6, g7, g8, g9, g10, g11, g12 ], 
    [ g1, g2, g3, g4*g9*g12, g5, g6, g7, g8, g9, g10, g11, g12 
     ] ), GroupHomomorphismByImages( G, G, 
    [ g1, g2, g3, g4, g5, g6, g7, g8, g9, g10, g11, g12 ], 
    [ g1*g9*g10*g11, g2, g3, g4*g12, g5, g6, g7, g8, g9, g10, g11, g12 
     ] ), GroupHomomorphismByImages( G, G, 
    [ g1, g2, g3, g4, g5, g6, g7, g8, g9, g10, g11, g12 ], 
    [ g1*g9*g11, g2, g3, g4*g11*g12, g5, g6, g7, g8, g9, g10, g11, g12 
     ] ), GroupHomomorphismByImages( G, G, 
    [ g1, g2, g3, g4, g5, g6, g7, g8, g9, g10, g11, g12 ], 
    [ g1*g9*g10*g11, g2, g3, g4*g9*g10*g11, g5, g6, g7, g8, g9, g10, g11, 
      g12 ] ) )|
%--------------------------------------------------------------------------

