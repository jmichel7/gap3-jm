%%%%%%%%%%%%%%%%%%%%%%%%%%%%%%%%%%%%%%%%%%%%%%%%%%%%%%%%%%%%%%%%%%%%%%%%%%%%%
%%
%A  string.tex                  GAP documentation            Martin Schoenert
%%
%A  @(#)$Id: string.tex,v 1.1.1.1 1996/12/11 12:36:51 werner Exp $
%%
%Y  Copyright 1990-1992,  Lehrstuhl D fuer Mathematik,  RWTH Aachen,  Germany
%%
%%  This file describes the string datatype and its functions.
%%
%H  $Log: string.tex,v $
%H  Revision 1.1.1.1  1996/12/11 12:36:51  werner
%H  Preparing 3.4.4 for release
%H
%H  Revision 3.10  1993/03/11  17:47:16  fceller
%H  changed description of comparison
%H
%H  Revision 3.9  1993/02/19  10:48:42  gap
%H  adjustments in line length and spelling
%H
%H  Revision 3.8  1993/02/12  11:41:19  felsch
%H  new example fixed
%H
%H  Revision 3.7  1993/02/11  12:20:47  martin
%H  changed reference "Strings" to "Strings and Characters"
%H
%H  Revision 3.6  1993/02/11  12:10:05  martin
%H  strings are now lists too
%H
%H  Revision 3.5  1993/02/09  10:40:18  felsch
%H  examples fixed
%H
%H  Revision 3.4  1991/12/27  16:07:04  martin
%H  revised everything for GAP 3.1 manual
%H
%H  Revision 3.3  1991/07/26  12:34:01  martin
%H  improved the index
%H
%H  Revision 3.2  1991/07/26  09:01:07  martin
%H  changed |\GAP\ | to |{\GAP}|
%H
%H  Revision 3.1  1991/07/25  16:16:59  martin
%H  fixed some minor typos
%H
%H  Revision 3.0  1991/04/11  11:35:22  martin
%H  Initial revision under RCS
%H
%%
\Chapter{Strings and Characters}%
\index{type!strings}%
\index{doublequotes}\index{backslash}\index{special character sequences}

A *character*  is simply an object in {\GAP} that represents an arbitrary
character from the character set  of  the  operating  system.   Character
literals  can  be  entered  in  {\GAP}  by  enclosing  the  character  in
*singlequotes* '\''.

|    gap> 'a';
    'a'
    gap> '*';
    '*' |

A *string* is simply a dense list of characters.  Strings are used mainly
in filenames and error messages.  A string literal can either be  entered
simply  as  the  list of characters or by  writing the characters between
*doublequotes* '\"'.   {\GAP}  will always output strings  in  the latter
format.

|    gap> s1 := ['H','a','l','l','o',' ','w','o','r','l','d','.'];
    "Hallo world."
    gap> s2 := "Hallo world.";
    "Hallo world."
    gap> s1 = s2;
    true
    gap> s3 := "";
    ""    # the empty string
    gap> s3 = [];
    true |

Note that a string is just a special case  of a list.  So everything that
is possible for lists (see  "Lists")  is also possible for strings.  Thus
you can  access  the characters in  such  a string (see "List Elements"),
test  for  membership (see  "In"), etc.  You can  even assign  to such  a
string (see "List Assignment").  Of course unless you  assign a character
in such a way  that the  list  stays dense,  the  resulting  list will no
longer be a string.

|    gap> Length( s2 );
    12
    gap> s2[2];
    'a'
    gap> 'e' in s2;
    false
    gap> s2[2] := 'e';;  s2;
    "Hello world." |

If a string is displayed as result of an evaluation (see "Main Loop"), it
is  displayed  with enclosing   doublequotes.  However,  if  a  string is
displayed by 'Print',  'PrintTo', or 'AppendTo' (see "Print",  "PrintTo",
"AppendTo") the enclosing doublequotes are dropped.

|    gap> s2;
    "Hello world."
    gap> Print( s2 );
    Hello world.gap> |

There  are a number  of *special character  sequences* that  can be  used
between   the  single  quote  of  a  character  literal  or  between  the
doublequotes  of  a  string  literal to  specify  characters,  which  may
otherwise be inaccessible.  They consist of two characters.  The first is
a  backslash $\backslash$.  The second may be any character.  The meaning
is given in the following list

'n':    *newline character*.  This  is  the character that, at   least on
        UNIX systems, separates  lines in a  text file.  Printing of this
        character in a string has the effect of moving  the  cursor  down
        one line and back to the beginning  of the line.

'\"':    *doublequote character*.  Inside  a string a doublequote  must be
        escaped by the backslash,  because it is otherwise interpreted as
        end of the string.

'\'':    *singlequote character*.  Inside a character a  singlequote  must
        escaped by the backslash, because it is  otherwise interpreted as
        end of the character.

$\backslash$:  *backslash character*.  Inside  a string a backslash  must
        be   escaped  by  another  backslash,    because it is  otherwise
        interpreted as first character of an escape sequence.

'b':    *backspace character*.  Printing this character should  have  the
        effect of moving the cursor back one character.  Whether it works
        or not is system dependent and should not be relied upon.

'r':    *carriage return character*.  Printing this character should have
        the effect of moving the cursor back to the beginning of the same
        line.  Whether this works or not is again system dependent.

'c':    *flush character*.  This character is not  printed.  Its  purpose
        is to flush the output queue.  Usually {\GAP} waits until it sees
        a <newline> before it prints a string.  If you want to display  a
        string that does not include this character use $\backslash c$.

other:  For any other character the backslash is simply ignored.

Again, if the line is displayed as result of  an evaluation, those escape
sequences are displayed in the  same way that they  are input.  They  are
displayed in their special way only by 'Print', 'PrintTo', or 'AppendTo'.

|    gap> "This is one line.\nThis is another line.\n";
    "This is one line.\nThis is another line.\n"
    gap> Print( last );
    This is one line.
    This is another line. |

It is not allowed to enclose a <newline> inside the  string.  You can use
the special  character  sequence  $\backslash n$  to  write  strings that
include <newline>  characters.  If, however, a string  is too long to fit
on a single line it is possible to *continue* it  over several lines.  In
this case  the last character of  each  line, except the  last  must be a
backslash.  Both backslash and <newline>  are thrown away.  Note that the
same continuation mechanism is available for identifiers and integers.

|    gap> "This is a very long string that does not fit on a line \
    gap> and is therefore continued on the next line.";
    "This is a very long string that does not fit on a line and is therefo\
    re continued on the next line."
    # note that the output is also continued, but at a different place |

This chapter contains sections  describing  the function that creates the
printable representation  of a string  (see "String"), the functions that
create new   strings  (see    "ConcatenationString",  "SubString"),   the
functions that tests  if an object  is   a  string  (see "IsString"), the
string comparisons (see "Comparisons of Strings"), and the  function that
returns the length of a string (see "Length").

%%%%%%%%%%%%%%%%%%%%%%%%%%%%%%%%%%%%%%%%%%%%%%%%%%%%%%%%%%%%%%%%%%%%%%%%%
\Section{String}%
\index{convert!to a string}

'String( <obj> )' \\
'String( <obj>, <length> )'

'String' returns a representation of the <obj>, which may be an object of
arbitrary type, as a string.   This string should approximate  as closely
as possible the character sequence you see if you print <obj>.

If <length> is given it must be an integer.  The absolute value gives the
minimal  length of the  result.  If  the string  representation of  <obj>
takes  less than that  many  characters  it is   filled with blanks.   If
<length> is positive it is filled on the left, if <length> is negative it
is filled on the right.

|    gap> String( 123 );
    "123"
    gap> String( [1,2,3] );
    "[ 1, 2, 3 ]"
    gap> String( 123, 10 );
    "       123"
    gap> String( 123, -10 );
    "123       "
    gap> String( 123, 2 );
    "123" |

%%%%%%%%%%%%%%%%%%%%%%%%%%%%%%%%%%%%%%%%%%%%%%%%%%%%%%%%%%%%%%%%%%%%%%%%%
\Section{ConcatenationString}%
\index{concatenation!of strings}\index{append!one string to another}

'ConcatenationString( <string1>, <string2> )'

'ConcatenationString'   returns   the  concatenation of the   two strings
<string1> and  <string2>.   This is a  new string  that  starts  with the
characters of <string1> and ends with the characters of <string2>.

|    gap> ConcatenationString( "Hello ", "world.\n" );
    "Hello world.\n" |

Because strings are now lists, 'Concatenation' (see "Concatenation") does
exactly the  right  thing,  and  the  function  'ConcatenationString'  is
obsolete.

%%%%%%%%%%%%%%%%%%%%%%%%%%%%%%%%%%%%%%%%%%%%%%%%%%%%%%%%%%%%%%%%%%%%%%%%%
\Section{SubString}%
\index{substring!of a string}\index{part!of a string}

'SubString( <string>, <from>, <to> )'

'SubString'  returns the  substring of  the string  <string> that begins at
position <from> and continues to position <to>. The characters at these two
positions  are included.  Indexing is  done with  origin 1, i.e., the first
character  is at position 1. <from> and  <to> must be integers and are both
silently  forced into  the range  '1..Length(<string>)' (see  "Length"). If
<to> is less than <from> the substring is empty.

|    gap> SubString( "Hello world.\n", 1, 5 );
    "Hello"
    gap> SubString( "Hello world.\n", 5, 1 );
    "" |

Because strings are  now  lists, substrings  can  also be extracted  with
'<string>\{[<from>..<to>]\}' (see "List  Elements").  'SubString'  forces
<from> and <to> into  the range  '1..Length(<string>)', which  the  above
does not, but apart from that 'SubString' is obsolete.

%%%%%%%%%%%%%%%%%%%%%%%%%%%%%%%%%%%%%%%%%%%%%%%%%%%%%%%%%%%%%%%%%%%%%%%%%
\Section{Comparisons of Strings}%
\index{comparisons!of strings}

'<string1> = <string2>', '<string1> \<> <string2>'

The  equality  operator  '='   evaluates to  'true'   if  the two strings
<string1> and <string2> are equal and  'false' otherwise.  The inequality
operator '\<>' returns 'true' if  the two strings <string1> and <string2>
are not equal and 'false' otherwise.

|    gap> "Hello world.\n" = "Hello world.\n";
    true
    gap> "Hello World.\n" = "Hello world.\n";
    false    # string comparison is case sensitive
    gap> "Hello world." = "Hello world.\n";
    false    # the first string has no <newline>
    gap> "Goodbye world.\n" = "Hello world.\n";
    false
    gap> [ 'a', 'b' ] = "ab";
    true |

'<string1> \<\ <string2>', '<string1> \<= <string2>',
'<string1>  > <string2>', '<string1>  => <string2>'

The operators '\<', '\<=', '>', and '=>' evaluate to 'true' if the string
<string1> is less than, less than or equal to, greater than, greater than
or  equal to the string  <string2> respectively.  The ordering of strings
is  lexicographically according  to the order  implied by the underlying,
system dependent, character set.

You  can  also compare  objects  of other types, for example integers  or
permutations with strings.  As strings  are  dense  character  lists they
compare  with other objects  as lists  do, i.e., they are  never equal to
those  objects, records  (see "Records") are  greater  than  strings, and
objects of every other type are smaller than strings.

|    gap> "Hello world.\n" < "Hello world.\n";
    false    # the strings are equal
    gap> "Hello World.\n" < "Hello world.\n";
    true    # in ASCII uppercase letters come before lowercase letters
    gap> "Hello world." < "Hello world.\n";
    true    # prefixes are always smaller
    gap> "Goodbye world.\n" < "Hello world.\n";
    true    # 'G' comes before 'H', in ASCII at least |

%%%%%%%%%%%%%%%%%%%%%%%%%%%%%%%%%%%%%%%%%%%%%%%%%%%%%%%%%%%%%%%%%%%%%%%%%
\Section{IsString}%
\index{test!for a string}

'IsString( <obj> )'

'IsString' returns 'true' if the object <obj>, which  may be an object of
arbitrary type, is a  string and 'false' otherwise.   Will cause an error
if <obj> is an unbound variable.

|    gap> IsString( "Hello world.\n" );
    true
    gap> IsString( "123" );
    true
    gap> IsString( 123 );
    false
    gap> IsString( [ '1', '2', '3' ] );
    true
    gap> IsString( [ '1', '2', , '4' ] );
    false    # strings must be dense
    gap> IsString( [ '1', '2', 3 ] );
    false    # strings must only contain characters |

%%%%%%%%%%%%%%%%%%%%%%%%%%%%%%%%%%%%%%%%%%%%%%%%%%%%%%%%%%%%%%%%%%%%%%%%%
\Section{Join}%

'Join( <list> [, <delimiter>] )'

The  function 'Join' is similar  to the Perl function  of the same name. It
first  applies the  function 'String'  to all  elements of the <list>, then
joins  the  resulting  strings,  separated  by  the  given  <delimiter> (if
omitted, '\",\"' is used as a delimiter)

|   gap> Join([1..4]);
   "1,2,3,4"
   gap> Join([1..4],"foo");
   "1foo2foo3foo4"|

%%%%%%%%%%%%%%%%%%%%%%%%%%%%%%%%%%%%%%%%%%%%%%%%%%%%%%%%%%%%%%%%%%%%%%%%%
\Section{SPrint}%

'SPrint(<s1>,...,<sn>)'

'SPrint(<s1>,...,<sn>)' is a synonym for 'Join([<s1>,...,<sn>],\"\")'. That
is, it first applies the function 'String' to all arguments, then joins the
resulting  strings. If '<s1>,...,<sn>'  have string methods,  the effect of
'Print(SPrint(<s1>,...,<sn>))' is the same as directly
'Print(<s1>,...,<sn>)'.

|   gap> SPrint(1,"a",[3,4]);
   "1a[ 3, 4 ]"|

%%%%%%%%%%%%%%%%%%%%%%%%%%%%%%%%%%%%%%%%%%%%%%%%%%%%%%%%%%%%%%%%%%%%%%%%%
\Section{PrintToString}%

'PrintToString(<s>,<s1>,...,<sn>)'

'PrintToString(<s>,<s1>,...,<sn>)'   appends  to  string   <s>  the  string
'SPrint(<s1>,...,<sn>)'.

|   gap> s:="a";
   "a"
   gap> PrintToString(s,[1,2]);
   gap> s;
   "a[ 1, 2 ]"|

%%%%%%%%%%%%%%%%%%%%%%%%%%%%%%%%%%%%%%%%%%%%%%%%%%%%%%%%%%%%%%%%%%%%%%%%%
\Section{Split}%

'Split( <s> [, <delimiter>] )'

This function is similar to the Perl function of the same name. It splits
the string <s> at each occurrence of the <delimiter> (a character).
If <delimiter> is omitted, '\',\'' is used as a delimiter.

|   gap> Split("14,2,2,1,");
    [ "14", "2", "2", "1", "" ]|

%%%%%%%%%%%%%%%%%%%%%%%%%%%%%%%%%%%%%%%%%%%%%%%%%%%%%%%%%%%%%%%%%%%%%%%%%
\Section{StringDate}%

'StringDate(<days>)'

'StringDate([<day>, <month>, <year>])'

This  function converts to a readable string  a date, which can be a number
of days since 1-Jan-1970 or a list '[<day>, <month>, <year>]'.

|   gap> StringDate([11,3,1998]);
   "11-Mar-1998"
   gap> StringDate(2^14);
   "10-Nov-2014"|

%%%%%%%%%%%%%%%%%%%%%%%%%%%%%%%%%%%%%%%%%%%%%%%%%%%%%%%%%%%%%%%%%%%%%%%%%
\Section{StringTime}%

'StringTime(<msec>)'

'StringTime([<hour>, <min>, <sec>, <msec>])'

This function converts to a readable string atime,  which can be a number of
milliseconds or a list '[<hour>, <min>, <sec>, <msec>]'.

|   gap> StringTime([1,10,5,13]);
   " 1:10:05.013"
   gap> StringTime(2^22);
   " 1:09:54.304"|

%%%%%%%%%%%%%%%%%%%%%%%%%%%%%%%%%%%%%%%%%%%%%%%%%%%%%%%%%%%%%%%%%%%%%%%%%
\Section{StringPP}%

'StringPP(<int>)'

returns a string representing the prime factor decomposition of the integer
int.

|   gap> StringPP(40320);
"2^7*3^2*5*7"|

%%%%%%%%%%%%%%%%%%%%%%%%%%%%%%%%%%%%%%%%%%%%%%%%%%%%%%%%%%%%%%%%%%%%%%%%%
%%
%E  Emacs . . . . . . . . . . . . . . . . . . . . . local Emacs variables
%%
%%  Local Variables:
%%  mode:               outline
%%  outline-regexp:     "\\\\Chapter\\|\\\\Section"
%%  fill-column:        73
%%  eval:               (hide-body)
%%  End:
%%
