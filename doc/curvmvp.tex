%%%%%%%%%%%%%%%%%%%%%%%%%%%%%%%%%%%%%%%%%%%%%%%%%%%%%%%%%%%%%%%%%%%%%%%%%%%%%
%%
%A  curvmvp.tex       VKCURVE documentation                     Jean Michel
%A  $Id: curvmvp.tex,v 1.2 1997/03/31 15:58:21 gap Exp $
%%
%Y  Copyright (C) 1997-2001  University  Paris VII.
%%
%%  This  file  documents MultiVariate Polynomials
%%
%%%%%%%%%%%%%%%%%%%%%%%%%%%%%%%%%%%%%%%%%%%%%%%%%%%%%%%%%%%%%%%%%%%%%%%%%
\def\CHEVIE{{\sf CHEVIE}}
\Chapter{Multivariate polynomials and rational fractions}

The  functions  described  in  this  file  were  written  to  alleviate the
deficiency of \GAP\ in manipulating multi-variate polynomials. In \GAP\ one
can  only define  one-variable polynomials  over a  given ring; this allows
multi-variate polynomials by taking this ring to be a polynomial ring; but,
in  addition to providing little flexibility in the choice of coefficients,
this  \"full\"\ representation makes  for somewhat inefficient computation.
The  use of the  'Mvp' (MultiVariate Polynomials)  described here is faster
than  \GAP\ polynomials as soon as there are two variables or more. What is
implemented   here  is   actually  \"Puiseux   polynomials\",  i.e.  linear
combinations  of monomials  of the  type $x_1^{a_1}\ldots  x_n^{a_n}$ where
$x_i$ are variables and $a_i$ are exponents which can be arbitrary rational
numbers. Some functions described below need their argument to involve only
variables  to integral powers;  we will refer  to such objects as \"Laurent
polynomials\";  some functions  require further  that variables  are raised
only  to positive powers\:\ we refer then to \"true polynomials\". Rational
fractions  ('RatFrac')  have  been  added,  thanks to work of Gwena{\accent
127e}lle Genet (the main difficulty there was to write an algorithm for the
Gcd  of multivariate polynomials, a  non-trivial task). The coefficients of
our  polynomials  can  in  principle  be  elements  of  any  ring, but some
algorithms like division or Gcd require the coefficients of their arguments
to be invertible.

%%%%%%%%%%%%%%%%%%%%%%%%%%%%%%%%%%%%%%%%%%%%%%%%%%%%%%%%%%%%%%%%%%%%%%%%%
\Section{Mvp}%
\index{Mvp}%

'Mvp( <string s> [, <coeffs v>] )'

Defines an indeterminate with name <s> suitable to build multivariate
polynomials.

|    gap> x:=Mvp("x");y:=Mvp("y");(x+y)^3;
    x
    y
    3xy^2+3x^2y+x^3+y^3|

If a  second argument (a vector  of coefficients <v>) is  given, returns
'Sum([1..Length(v)],i->Mvp(s)\^(i-1)\*v[i])'.

|    gap> Mvp("a",[1,2,0,4]);
    1+2a+4a^3|

'Mvp( <polynomial x>)'

Converts  the \GAP\  polynomial  <x> to  an  'Mvp'. It  is  an error  if
'x.baseRing.indeterminate.name' is not bound; otherwise this is taken as
the name of the 'Mvp' variable.

|    gap> q:=Indeterminate(Rationals);
    X(Rationals)
    gap> Mvp(q^2+q);
    Error, X(Rationals) should have .name bound in
    Mvp( q ^ 2 + q ) called from
    main loop
    brk> 
    gap> q.name:="q";;
    gap> Mvp(q^2+q); 
    q+q^2|

'Mvp( <FracRat x>)'

Returns 'false' if the argument rational fraction is not in fact a Laurent
polynomial. Otherwise returns that polynomial.

|    gap> Mvp(x/y);
    xy^-1
    gap> Mvp(x/(y+1));
    false|

'Mvp( <elm>, <coeff>)'

Build efficiently an 'Mvp' from the given list of coefficients and the list
<elm>   describing  the  corresponding  monomials.  A  monomial  is  itself
described  by a record with a field  '.elm' containing the list of involved
variable  names and a  field '.coeff' containing  the list of corresponding
exponents.

|    gap> Mvp([rec(elm:=["y","x"],coeff:=[1,-1])],[1]);       
    x^-1y|

'Mvp( <scalar x>)'

A  scalar is  anything which  is  not one  of the  previous types  (like
a  cyclotomic,  or a  finite-field-element,  etc).  Returns the  constant
multivariate polynomial whose constant term is <x>.

|    gap> Degree(Mvp(1));
    0|

%%%%%%%%%%%%%%%%%%%%%%%%%%%%%%%%%%%%%%%%%%%%%%%%%%%%%%%%%%%%%%%%%%%%%%%%%%%%
\Section{Operations for Mvp}

The arithmetic operations '+', '-', '\*',  '/' and '\^' work for 'Mvp's.
They also  have 'Print' and  'String' methods. The operations  '+', '-',
'\*' work  for any inputs. '/'  works only for Laurent  polynomials, and
may return  a rational  fraction (see  below); if one  is sure  that the
division is exact, one should call 'MvpOps.ExactDiv' (see below).

|    gap> x:=Mvp("x");y:=Mvp("y");
    x
    y
    gap> a:=x^(-1/2);
    x^(-1/2)
    gap> (a+y)^4;
    x^-2+4x^(-3/2)y+6x^-1y^2+4x^(-1/2)y^3+y^4
    gap> (x^2-y^2)/(x-y);
    x+y
    gap> (x-y^2)/(x-y);
    (x-y^2)/(x-y)
    gap> (x-y^2)/(x^(1/2)-y);
    Error, x^(1/2)-y is not a polynomial with respect to x
     in
    V.operations.Coefficients( V, v ) called from
    Coefficients( q, var ) called from
    MvpOps.ExactDiv( x, q ) called from
    fun( arg[1][i] ) called from
    List( p, function ( x ) ... end ) called from
    ...
    brk>|

Only monomials can be raised to a non-integral power; they can be raised
to  a fractional  power of  denominator  'b' only  if 'GetRoot(x,b)'  is
defined  where 'x'  is  their  leading coefficient.  For  an 'Mvp'  <m>,
the  function  'GetRoot(m,n)' is  equivalent  to  'm\^(1/n)'. Raising  a
non-monomial Laurent polynomial  to a negative power  returns a rational
fraction.

|    gap> (2*x)^(1/2);
    ER(2)x^(1/2)
    gap> (evalf(2)*x)^(1/2);
    1.4142135624x^(1/2)
    gap> GetRoot(evalf(2)*x,2);
    1.4142135624x^(1/2)|

\index{Degree}
The 'Degree' of a monomial is the sum of  the exponent of the variables.
The 'Degree' of an 'Mvp' is the largest degree of a monomial.

|    gap> a;
    x^(-1/2)
    gap> Degree(a);
    -1/2
    gap> Degree(a+x);
    1
    gap> Degree(Mvp(0));
    -1|

\index{Valuation}
The 'Valuation' of an 'Mvp' is the minimal degree of a monomial.

|    gap> a;
    x^(-1/2)
    gap> Valuation(a);
    -1/2
    gap> Valuation(a+x);
    -1/2
    gap> Valuation(Mvp(0));
    -1|

The  'Format' routine formats  'Mvp's in such  a way that  they can be read
back  in by \GAP\ or by some other systems, by giving an appropriate option
as  a second argument, or using the functions 'FormatGAP', 'FormatMaple' or
'FormatTeX'.  The 'String'  method is  equivalent to  'Format', and gives a
compact display.

|    gap> p:=7*x^5*y^-1-2;  
    -2+7x^5y^-1
    gap> Format(p);        
    "-2+7x^5y^-1"
    gap> FormatGAP(p);  
    "-2+7*x^5*y^-1"
    gap> FormatMaple(p);
    "-2+7*x^5*y^(-1)"
    gap> FormatTeX(p);  
    "-2+7x^5y^{-1}"|

\index{Value}
The 'Value' method evaluates an 'Mvp' by fixing simultaneously the value
of several  variables. The  syntax is  'Value(x, [  <string1>, <value1>,
<string2>, <value2>, $\ldots$ ])'.

|    gap> p;
    -2+7x^5y^-1
    gap> Value(p,["x",2]);
    -2+224y^-1
    gap> Value(p,["y",3]);
    -2+7/3x^5
    gap> Value(p,["x",2,"y",3]);
    218/3|

One  should pay  attention to  the fact  that the  last value  is not  a
rational  number,  but  a  constant 'Mvp'  (for  consistency).  See  the
function 'ScalMvp' below for how to convert such constants to their base
ring.

|    gap> Value(p,["x",y]);
    -2+7y^4
    gap> Value(p,["x",y,"y",x]);
    -2+7x^-1y^5|

Evaluating an  'Mvp' which is  a Puiseux  polynomial may cause  calls to
'GetRoot'

|    gap> p:=x^(1/2)*y^(1/3);
    x^(1/2)y^(1/3)
    gap> Value(p,["x",y]);
    y^(5/6)
    gap>  Value(p,["x",2]);
    ER(2)y^(1/3)
    gap>  Value(p,["y",2]);
    Error, : unable to compute 3-th root of 2
     in
    GetRoot( values[i], d[i] ) called from
    f.operations.Value( f, x ) called from
    Value( p, [ "y", 2 ] ) called from
    main loop
    brk>|

\index{Derivative}
The  function 'Derivative(p,v)' returns the  derivative of 'p' with respect
to  the variable given by the string 'v'; if 'v' is not given, with respect
to the first variable in alphabetical order.

|    gap>  p:=7*x^5*y^-1-2;
    -2+7x^5y^-1
    gap> Derivative(p,"x");
    35x^4y^-1
    gap> Derivative(p,"y");
    -7x^5y^-2
    gap> Derivative(p);
    35x^4y^-1
    gap>  p:=x^(1/2)*y^(1/3);
    x^(1/2)y^(1/3)
    gap>  Derivative(p,"x");
    1/2x^(-1/2)y^(1/3)
    gap>  Derivative(p,"y");
    1/3x^(1/2)y^(-2/3)
    gap>  Derivative(p,"z");
    0|

\index{Coefficients}
The function 'Coefficients(<p>, <var>)' is defined only for 'Mvp's which
are polynomials in the variable <var> . It returns as a list the list of
coefficients of <p> with respect to <var>.

|    gap> p:=x+y^-1;
    y^-1+x
    gap> Coefficients(p,"x");
    [ y^-1, 1 ]
    gap> Coefficients(p,"y");
    Error, y^-1+x is not a polynomial with respect to y
     in
    V.operations.Coefficients( V, v ) called from
    Coefficients( p, "y" ) called from
    main loop
    brk>|

The same  caveat is  applicable to 'Coefficients'  as to  'Value'\:\ the
result  is always  a  list of  'Mvp's.  To  get a  list  of scalars  for
univariate polynomials represented as 'Mvp's, one should use 'ScalMvp'.

\index{ComplexConjugate}
Finally we  mention the  functions 'ComplexConjugate' and  'evalf' which
are defined using  for coefficients the 'Complex'  and 'Decimal' numbers
of the \CHEVIE\ package.

|    gap> p:=E(3)*x+E(5);
    E5+E3x
    gap> evalf(p);
    0.3090169944+0.9510565163I+(-0.5+0.8660254038I)x
    gap> p:=E(3)*x+E(5);          
    E5+E3x
    gap> ComplexConjugate(p);
    E5^4+E3^2x
    gap> evalf(p);
    0.3090169944+0.9510565163I+(-0.5+0.8660254038I)x
    gap> ComplexConjugate(last);
    0.3090169944-0.9510565163I+(-0.5-0.8660254038I)x|

%%%%%%%%%%%%%%%%%%%%%%%%%%%%%%%%%%%%%%%%%%%%%%%%%%%%%%%%%%%%%%%%%%%%%%%%%
\Section{IsMvp}%
\index{IsMvp}%

'IsMvp( <p> )'

Returns 'true' if <p> is an 'Mvp' and false otherwise.

|    gap> IsMvp(1+Mvp("x"));
    true
    gap> IsMvp(1);         
    false|

%%%%%%%%%%%%%%%%%%%%%%%%%%%%%%%%%%%%%%%%%%%%%%%%%%%%%%%%%%%%%%%%%%%%%%%%%
\Section{ScalMvp}%
\index{ScalMvp}%

'ScalMvp( <p> )'

If  <p> is  an  'Mvp' then  if  <p>  is a  scalar,  return that  scalar,
otherwise return  'false'. Or  if <p>  is a  list, then  apply 'ScalMvp'
recursively to  it (but return false  if it contains any  'Mvp' which is
not a scalar). Else assume <p> is already a scalar and thus return <p>.

|    gap> v:=[Mvp("x"),Mvp("y")];        
    [ x, y ]
    gap> ScalMvp(v);
    false
    gap> w:=List(v,p->Value(p,["x",2,"y",3]));
    [ 2, 3 ]
    gap> Gcd(w);
    Error, sorry, the elements of <arg> lie in no common ring domain in
    Domain( arg[1] ) called from
    DefaultRing( ns ) called from
    Gcd( w ) called from
    main loop
    brk> 
    gap> Gcd(ScalMvp(w));
    1|

%%%%%%%%%%%%%%%%%%%%%%%%%%%%%%%%%%%%%%%%%%%%%%%%%%%%%%%%%%%%%%%%%%%%%%%%%%%%
\Section{Variables for Mvp}%
\index{Variables for Mvp}%

'Variables for Mvp( <p> )'

Returns the list of variables of the 'Mvp' <p> as a sorted list of strings.

|    gap> Variables(x+x^4+y);
    [ "x", "y" ]|

%%%%%%%%%%%%%%%%%%%%%%%%%%%%%%%%%%%%%%%%%%%%%%%%%%%%%%%%%%%%%%%%%%%%%%%%%%%%
\Section{LaurentDenominator}%
\index{LaurentDenominator}%

'LaurentDenominator( <p1>, <p2>, ... )'

Returns the unique monomial 'm' of minimal degree such that for all the
Laurent polynomial arguments <p1>, <p2>, etc... the product $m\* p_i$ is
a true polynomial.

|    gap> LaurentDenominator(x^-1,y^-2+x^4);
    xy^2|

%%%%%%%%%%%%%%%%%%%%%%%%%%%%%%%%%%%%%%%%%%%%%%%%%%%%%%%%%%%%%%%%%%%%%%%%%%%%
\Section{OnPolynomials}%
\index{OnPolynomials}%

'OnPolynomials( <m>, <p> [,<vars>] )'

Implements  the action of  a matrix on  'Mvp's. <vars> should  be a list of
strings representing variables. If $v$'=List(vars,Mvp)', the polynomial $p$
is  changed  by  replacing  in  it  $v_i$  by $(v\times m)_i$. If <vars> is
omitted, it is taken to be 'Variables(p)'.

|    gap> OnPolynomials([[1,2],[3,1]],x+y);    
    3x+4y|

%%%%%%%%%%%%%%%%%%%%%%%%%%%%%%%%%%%%%%%%%%%%%%%%%%%%%%%%%%%%%%%%%%%%%%%%%%%%
\Section{FactorizeQuadraticForm}%
\index{FactorizeQuadraticForm}%

'FactorizeQuadraticForm( <p>)'

<p>  should be an 'Mvp' of degree  2 which represents a quadratic form. The
function  returns a list of two linear forms of which <p> is the product if
such  forms exist, otherwise it returns 'false' (it returns [Mvp(1),<p>] if
<p> is of degree 1).

|    gap> FactorizeQuadraticForm(x^2-y^2+x+3*y-2);
    [ -1+x+y, 2+x-y ]
    gap> FactorizeQuadraticForm(x^2+x+1);        
    [ -E3+x, -E3^2+x ]
    gap> FactorizeQuadraticForm(x*y+1);  
    false|

%%%%%%%%%%%%%%%%%%%%%%%%%%%%%%%%%%%%%%%%%%%%%%%%%%%%%%%%%%%%%%%%%%%%%%%%%%%%

The next functions have been provided by Gwena{\accent 127e}lle Genet

%%%%%%%%%%%%%%%%%%%%%%%%%%%%%%%%%%%%%%%%%%%%%%%%%%%%%%%%%%%%%%%%%%%%%%%%%%%%
\Section{MvpGcd}
\index{MvpGcd}

'MvpGcd( <p1>, <p2>, ...)'

Returns  the Gcd  of the  'Mvp' arguments.  The arguments  must be  true
polynomials.

|    gap> MvpGcd(x^2-y^2,(x+y)^2);
    x+y|

%%%%%%%%%%%%%%%%%%%%%%%%%%%%%%%%%%%%%%%%%%%%%%%%%%%%%%%%%%%%%%%%%%%%%%%%%%%%
\Section{MvpLcm}
\index{MvpLcm}

'MvpLcm( <p1>, <p2>, ...)'

Returns  the Lcm  of the  'Mvp' arguments.  The arguments  must be  true
polynomials.

|    gap> MvpLcm(x^2-y^2,(x+y)^2);
    xy^2-x^2y-x^3+y^3|

%%%%%%%%%%%%%%%%%%%%%%%%%%%%%%%%%%%%%%%%%%%%%%%%%%%%%%%%%%%%%%%%%%%%%%%%%%%%
\Section{RatFrac}%
\index{RatFrac}%

'RatFrac( <num> [,<den>] )'

Build the rational fraction ('RatFrac') with numerator <num> and denominator 
<den> (when <den> is omitted it is taken to be 1).

|    gap> RatFrac(x,y);
    x/y
    gap> RatFrac(x*y^-1);
    x/y|

%%%%%%%%%%%%%%%%%%%%%%%%%%%%%%%%%%%%%%%%%%%%%%%%%%%%%%%%%%%%%%%%%%%%%%%%%%%%
\Section{Operations for RatFrac}

The arithmetic operations '+', '-', '\*',  '/' and '\^' work for 'RatFrac's.
They also  have 'Print' and  'String' methods.

|    gap> 1/(x+1)+y^-1;
    (1+x+y)/(y+xy)
    gap> 1/(x+1)*y^-1;
    1/(y+xy)
    gap> 1/(x+1)/y;   
    1/(y+xy)
    gap> 1/(x+1)^-2;
    1+2x+x^2|

Similarly to 'Mvp's, 'RatFrac's hav 'Format' and 'Value' methods:

|    gap> Format(1/(x*y+1));
    "1/(1+xy)"
    gap> FormatGAP(1/(x*y+1));
    "1/(1+x*y)"
    gap> Value(1/(x*y+1),["x",2]);
    1/(1+2y)|

%%%%%%%%%%%%%%%%%%%%%%%%%%%%%%%%%%%%%%%%%%%%%%%%%%%%%%%%%%%%%%%%%%%%%%%%%
\Section{IsRatFrac}%
\index{IsRatFrac}%

'IsRatFrac( <p> )'

Returns 'true' if <p> is an 'Mvp' and false otherwise.

|    gap> IsRatFrac(1+RatFrac(x));
    true
    gap> IsRatFrac(x);         
    false|

%%%%%%%%%%%%%%%%%%%%%%%%%%%%%%%%%%%%%%%%%%%%%%%%%%%%%%%%%%%%%%%%%%%%%%%%%
