%%%%%%%%%%%%%%%%%%%%%%%%%%%%%%%%%%%%%%%%%%%%%%%%%%%%%%%%%%%%%%%%%%%%%%%%%%%%%
%%
%A  homomorp.tex                GAP documentation            Martin Schoenert
%%
%A  @(#)$Id: homomorp.tex,v 1.1.1.1 1996/12/11 12:36:46 werner Exp $
%%
%Y  Copyright 1990-1992,  Lehrstuhl D fuer Mathematik,  RWTH Aachen,  Germany
%%
%%  This  file  describes  homomorphisms,  their  functions  and  operations.
%%
%H  $Log: homomorp.tex,v $
%H  Revision 1.1.1.1  1996/12/11 12:36:46  werner
%H  Preparing 3.4.4 for release
%H
%H  Revision 3.4  1993/02/19  10:48:42  gap
%H  adjustments in line length and spelling
%H
%H  Revision 3.3  1993/02/09  16:26:02  felsch
%H  examples fixed
%H
%H  Revision 3.2  1992/03/27  16:51:48  martin
%H  added examples
%H
%H  Revision 3.1  1992/03/25  15:40:06  martin
%H  initial revision under RCS
%H
%%
\Chapter{Homomorphisms}

An important special class of mappings are homomorphisms.

A mapping $map$  is  a  *homomorphism*  if  the  source and the range are
domains of the same  category, and $map$  respects their structure.   For
example, if both source and range  are groups and  for each $x,y$  in the
source  $(xy)^{map}   =  x^{map}   y^{map}$,  then   $map$  is  a   group
homomorphism.

{\GAP}  currently  supports  field and  group homomorphisms  (see  "Field
Homomorphisms", "Group Homomorphisms").

Homomorphism  are  created  by  *homomorphism  constructors*,  which  are
ordinary  {\GAP}   functions   that   return   homomorphisms,   such   as
'FrobeniusAutomorphism'      (see       "FrobeniusAutomorphism")       or
'NaturalHomomorphism' (see "NaturalHomomorphism").

The first section in  this  chapter  describes  the  function  that tests
whether  a  mapping is a  homomorphism (see "IsHomomorphism").   The next
sections  describe  the  functions that  test  whether a homomorphism has
certain     properties     (see     "IsMonomorphism",    "IsEpimorphism",
"IsIsomorphism",  "IsEndomorphism",  and  "IsAutomorphism").   The   last
section describes the function that computes the kernel of a homomorphism
(see "Kernel").

Because homomorphisms  are just a special case of mappings all operations
and   functions  described  in  chapter  "Mappings"  are   applicable  to
homomorphisms.   For  example, the image  of  an  element <elm>  under  a
homomorphism  <hom> can be computed by '<elm> \^\ <hom>' (see "Operations
for Mappings").

%%%%%%%%%%%%%%%%%%%%%%%%%%%%%%%%%%%%%%%%%%%%%%%%%%%%%%%%%%%%%%%%%%%%%%%%%
\Section{IsHomomorphism}

'IsHomomorphism( <map> )'

'IsHomomorphism'  returns 'true'  if the mapping <map> is  a homomorphism
and 'false' otherwise.  Signals  an  error  if <map>  is  a  multi valued
mapping.

A mapping  $map$  is a *homomorphism*  if the  source and  the  range are
sources of  the  same category, and  $map$ respects the  structure.   For
example,  if both  source and range are groups  and for each $x,y$ in the
source $(xy)^{map} = x^{map} y^{map}$, then $map$ is a homomorphism.

|    gap> g := Group( (1,2,3,4), (2,4), (5,6,7) );;  g.name := "g";;
    gap> p4 := MappingByFunction( g, g, x -> x^4 );
    MappingByFunction( g, g, function ( x )
        return x ^ 4;
    end )
    gap> IsHomomorphism( p4 );
    true
    gap> p5 := MappingByFunction( g, g, x -> x^5 );
    MappingByFunction( g, g, function ( x )
        return x ^ 5;
    end )
    gap> IsHomomorphism( p5 );
    true
    gap> p6 := MappingByFunction( g, g, x -> x^6 );
    MappingByFunction( g, g, function ( x )
        return x ^ 6;
    end )
    gap> IsHomomorphism( p6 );
    false |

'IsHomomorphism' first tests if the flag '<map>.isHomomorphism' is bound.
If  the flag  is  bound,  it  returns  its  value.   Otherwise  it  calls
'<map>.source.operations.IsHomomorphism( <map> )', remembers the returned
value in '<map>.isHomomorphism', and returns it.

The functions  usually called this  way  are 'IsGroupHomomorphism' if the
source  of <map> is a group  and  'IsFieldHomomorphism'  if the source of
<map> is a field (see "IsGroupHomomorphism", "IsFieldHomomorphism").

%%%%%%%%%%%%%%%%%%%%%%%%%%%%%%%%%%%%%%%%%%%%%%%%%%%%%%%%%%%%%%%%%%%%%%%%%
\Section{IsMonomorphism}

'IsMonomorphism( <map> )'

'IsMonomorphism' returns 'true' if the  mapping  <map> is a  monomorphism
and 'false'  otherwise.  Signals  an  error  if <map> is  a  multi valued
mapping.

A mapping  is  a *monomorphism*  if it is an  injective homomorphism (see
"IsInjective", "IsHomomorphism").

|    gap> g := Group( (1,2,3,4), (2,4), (5,6,7) );;  g.name := "g";;
    gap> p4 := MappingByFunction( g, g, x -> x^4 );
    MappingByFunction( g, g, function ( x )
        return x ^ 4;
    end )
    gap> IsMonomorphism( p4 );
    false
    gap> p5 := MappingByFunction( g, g, x -> x^5 );
    MappingByFunction( g, g, function ( x )
        return x ^ 5;
    end )
    gap> IsMonomorphism( p5 );
    true |

'IsMonomorphism' first test if the flag  '<map>.isMonomorphism' is bound.
If the  flag is  bound,  it  returns  this  value.   Otherwise  it  calls
'<map>.operations.IsMonomorphism(  <map> )', remembers the returned value
in '<map>.isMonomorphism', and returns it.

The  default function  called  this  way  is 'MappingOps.IsMonomorphism',
which calls the functions 'IsInjective' and 'IsHomomorphism', and returns
the logical  *and*  of the  results.   This  function is seldom overlaid,
because  all  the  interesting   work  is   done  in   'IsInjective'  and
'IsHomomorphism'.

%%%%%%%%%%%%%%%%%%%%%%%%%%%%%%%%%%%%%%%%%%%%%%%%%%%%%%%%%%%%%%%%%%%%%%%%%
\Section{IsEpimorphism}

'IsEpimorphism( <map> )'

'IsEpimorphism' returns 'true' if the mapping <map> is an epimorphism and
'false' otherwise.  Signals an error if <map> is a multi valued mapping.

A mapping is an *epimorphism*  if  it is an surjective  homomorphism (see
"IsSurjective", "IsHomomorphism").

|    gap> g := Group( (1,2,3,4), (2,4), (5,6,7) );;  g.name := "g";;
    gap> p4 := MappingByFunction( g, g, x -> x^4 );
    MappingByFunction( g, g, function ( x )
        return x ^ 4;
    end )
    gap> IsEpimorphism( p4 );
    false
    gap> p5 := MappingByFunction( g, g, x -> x^5 );
    MappingByFunction( g, g, function ( x )
        return x ^ 5;
    end )
    gap> IsEpimorphism( p5 );
    true |

'IsEpimorphism' first test if the  flag '<map>.isEpimorphism'  is  bound.
If  the  flag  is  bound,  it  returns this  value.  Otherwise  it  calls
'<map>.operations.IsEpimorphism( <map> )',  remembers the  returned value
in '<map>.isEpimorphism', and returns it.

The default function called this way is 'MappingOps.IsEpimorphism', which
calls the functions  'IsSurjective' and 'IsHomomorphism', and returns the
logical *and* of the results.  This function is seldom  overlaid, because
all the interesting work is done in 'IsSurjective' and 'IsHomomorphism'.

%%%%%%%%%%%%%%%%%%%%%%%%%%%%%%%%%%%%%%%%%%%%%%%%%%%%%%%%%%%%%%%%%%%%%%%%%
\Section{IsIsomorphism}

'IsIsomorphism( <map> )'

'IsIsomorphism' returns 'true' if the mapping <map> is an isomorphism and
'false' otherwise.  Signals an error if <map> is a multi valued mapping.

A mapping is  an *isomorphism*  if  it  is a bijective homomorphism  (see
"IsBijection", "IsHomomorphism").

|    gap> g := Group( (1,2,3,4), (2,4), (5,6,7) );;  g.name := "g";;
    gap> p4 := MappingByFunction( g, g, x -> x^4 );
    MappingByFunction( g, g, function ( x )
        return x ^ 4;
    end )
    gap> IsIsomorphism( p4 );
    false
    gap> p5 := MappingByFunction( g, g, x -> x^5 );
    MappingByFunction( g, g, function ( x )
        return x ^ 5;
    end )
    gap> IsIsomorphism( p5 );
    true |

'IsIsomorphism' first test  if the flag  '<map>.isIsomorphism'  is bound.
If  the  flag  is bound,  it  returns this  value.   Otherwise  it  calls
'<map>.operations.IsIsomorphism(  <map> )', remembers  the returned value
in '<map>.isIsomorphism', and returns it.

The default function called this way is 'MappingOps.IsIsomorphism', which
calls the functions  'IsInjective', 'IsSurjective', and 'IsHomomorphism',
and  returns the logical *and*  of the results.  This  function is seldom
overlaid,  because  all the interesting work  is  done  in 'IsInjective',
'IsSurjective', and 'IsHomomorphism'.

%%%%%%%%%%%%%%%%%%%%%%%%%%%%%%%%%%%%%%%%%%%%%%%%%%%%%%%%%%%%%%%%%%%%%%%%%
\Section{IsEndomorphism}

'IsEndomorphism( <map> )'

'IsEndomorphism'  returns  'true' if the mapping <map> is a  endomorphism
and 'false' otherwise.   Signals an  error if  <map>  is a  multi  valued
mapping.

A   mapping   is  an  *endomorphism*  if  it   is   a  homomorphism  (see
"IsHomomorphism") and the range is a subset of the source.

|    gap> g := Group( (1,2,3,4), (2,4), (5,6,7) );;  g.name := "g";;
    gap> p4 := MappingByFunction( g, g, x -> x^4 );
    MappingByFunction( g, g, function ( x )
        return x ^ 4;
    end )
    gap> IsEndomorphism( p4 );
    true
    gap> p5 := MappingByFunction( g, g, x -> x^5 );
    MappingByFunction( g, g, function ( x )
        return x ^ 5;
    end )
    gap> IsEndomorphism( p5 );
    true |

'IsEndomorphism' first test if the flag  '<map>.isEndomorphism' is bound.
If  the  flag  is  bound,  it  returns this  value.   Otherwise  it calls
'<map>.operations.IsEndomorphism( <map> )', remembers  the returned value
in '<map>.isEndomorphism', and returns it.

The  default  function called  this  way is  'MappingOps.IsEndomorphism',
which  tests   if   the   range  is  a   subset  of  the  source,   calls
'IsHomomorphism', and  returns the  logical *and* of  the results.   This
function is seldom overlaid,  because all the interesting work is done in
'IsSubset' and 'IsHomomorphism'.

%%%%%%%%%%%%%%%%%%%%%%%%%%%%%%%%%%%%%%%%%%%%%%%%%%%%%%%%%%%%%%%%%%%%%%%%%
\Section{IsAutomorphism}

'IsAutomorphism( <map> )'

'IsAutomorphism'  returns 'true' if the mapping  <map> is an automorphism
and 'false'  otherwise.  Signals  an  error  if <map> is  a multi  valued
mapping.

A  mapping is an *automorphism* if it is an isomorphism where  the source
and the range are equal (see "IsIsomorphism", "IsEndomorphism").

|    gap> g := Group( (1,2,3,4), (2,4), (5,6,7) );;  g.name := "g";;
    gap> p4 := MappingByFunction( g, g, x -> x^4 );
    MappingByFunction( g, g, function ( x )
        return x ^ 4;
    end )
    gap> IsAutomorphism( p4 );
    false
    gap> p5 := MappingByFunction( g, g, x -> x^5 );
    MappingByFunction( g, g, function ( x )
        return x ^ 5;
    end )
    gap> IsAutomorphism( p5 );
    true |

'IsAutomorphism' first test if  the flag '<map>.isAutomorphism' is bound.
If the  flag  is  bound,  it  returns  this  value.   Otherwise it  calls
'<map>.operations.IsAutomorphism( <map> )', remembers  the returned value
in '<map>.isAutomorphism', and returns it.

The  default function called  this  way  is  'MappingOps.IsAutomorphism',
which calls the functions 'IsEndomorphism' and 'IsBijection', and returns
the  logical *and*  of the  results.   This function  is seldom overlaid,
because  all  the  interesting  work  is  done  in  'IsEndomorphism'  and
'IsBijection'.

%%%%%%%%%%%%%%%%%%%%%%%%%%%%%%%%%%%%%%%%%%%%%%%%%%%%%%%%%%%%%%%%%%%%%%%%%
\Section{Kernel}

'Kernel( <hom> )'

'Kernel'  returns the kernel  of the  homomorphism <hom>.   The kernel is
usually returned as a source,  though in some cases it  might be returned
as a proper set.

The kernel is the set of elements that  are mapped  <hom> to the identity
element of '<hom>.range',  i.e., to '<hom>.range.identity' if  <hom> is a
group homomorphism, and to '<hom>.range.zero' if <hom> is a ring or field
homomorphism.  The kernel is a substructure of '<hom>.source'.

|    gap> g := Group( (1,2,3,4), (2,4), (5,6,7) );;  g.name := "g";;
    gap> p4 := MappingByFunction( g, g, x -> x^4 );
    MappingByFunction( g, g, function ( x )
        return x ^ 4;
    end )
    gap> Kernel( p4 );
    Subgroup( g, [ (1,2,3,4), (1,4)(2,3) ] )
    gap> p5 := MappingByFunction( g, g, x -> x^5 );
    MappingByFunction( g, g, function ( x )
        return x ^ 5;
    end )
    gap> Kernel( p5 );
    Subgroup( g, [  ] ) |

'Kernel' first tests if the field  '<hom>.kernel' is bound.  If the field
is    bound     it    returns   its  value.        Otherwise   it   calls
'<hom>.source.operations.Kernel(  <hom> )',  remembers the returned value
in '<hom>.kernel', and returns it.

The  functions  usually  called   this  way   from  the  dispatcher   are
'KernelGroupHomomorphism'      and     'KernelFieldHomomorphism'     (see
"KernelGroupHomomorphism", "KernelFieldHomomorphism").

%%%%%%%%%%%%%%%%%%%%%%%%%%%%%%%%%%%%%%%%%%%%%%%%%%%%%%%%%%%%%%%%%%%%%%%%%
%%
%E  Emacs . . . . . . . . . . . . . . . . . . . . . local Emacs variables
%%
%%  Local Variables:
%%  mode:               outline
%%  outline-regexp:     "\\\\Chapter\\|\\\\Section"
%%  fill-column:        73
%%  eval:               (hide-body)
%%  End:
%%



